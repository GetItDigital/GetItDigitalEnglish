% Ladevorgang einer Induktivität L in Serie mit Widerstand R an idealer DC-Spannungsquelle Uq ab t=0 mit i(0)=0
% AxisEnv 1: Spannung der Spannungsquelle
% AxisEnv 2: Spannung an Induktivität
% AxisEnv 3: Strom durch Induktivität
\begin{tikzpicture}[x=1cm,y=1cm]
	\begin{axis}[
		name=AxisQuelle,
		%%%%%%% START SAME OPTIONS %%%%%%%
		xmin=-0.5, xmax=19.5,
		ymin=-0.125, ymax=1.375,
		xlabel={$t$},
		ylabel={$u_1(t)$},
		axis x line=center,
		axis y line=center,
		%ylabel style={anchor=east, at={(ticklabel cs:0.95)}},
		ylabel style={yshift=3pt},
		xtick distance=1, ytick distance=1,
		xticklabels={},yticklabels={},
		mark=none,
		grid=none,
		width=10cm,
		height=3.5cm,
		extra x ticks={1,5,10},
		extra x tick labels={$\tau$,$5\tau$,$t_1$},
		%%%%%%% END OF SAME OPTIONS %%%%%%%
		extra y ticks={1},
		extra y tick labels={$U_0$},
	]
		\addplot[color=voltage,very thick,	domain=-1:0] {0};% t<0           		
		\addplot[color=voltage, very thick] coordinates {(0,0)(0,1)};% t=0 
		\addplot[color=voltage,very thick,	domain=0:10] {1};% 0<t<t_1
		\addplot[color=voltage, very thick] coordinates {(10,1)(10,0)};% 5=t_1
		\addplot[color=voltage,very thick,	domain=10:20] {0};% t>t_1
	\end{axis}
	\begin{axis}[
		at={($(AxisQuelle.south west)+(0,-0.25cm)$)}, % Position 2. Plot
		anchor=north west,
		name=AxisSpannung,
		%%%%%%% START SAME OPTIONS %%%%%%%
		xmin=-0.5, xmax=19.5,
		ymin=-1.375, ymax=1.375,
		xlabel={$t$},
		ylabel={$u_L(t)$},
		axis x line=center,
		axis y line=center,
		%ylabel style={anchor=east, at={(ticklabel cs:0.95)}},
		ylabel style={yshift=3pt},
		xtick distance=1, ytick distance=1,
		xticklabels={},yticklabels={},
		mark=none,
		grid=none,
		width=10cm,
		height=5cm,
		extra x ticks={1,5,10},
		extra x tick labels={$\tau$,$5\tau$,$t_1$},
		extra y ticks={-1,1},
		extra y tick labels={$-U_0$,$\hphantom{+}U_0$},
		%%%%%%% END OF SAME OPTIONS %%%%%%%
	]
		%\addplot[color=black,dashed, domain=0:20] {+1}; % Grenzwert
		\addplot[color=black,dashed, domain=0:10] {-1}; % Grenzwert
		\addplot[color=voltage,very thick,	domain=-1:0] {0};  						% t<0
		\addplot[color=voltage, very thick] coordinates {(0,0)(0,1)}; 				% t=0 Stromsprung
		\addplot[color=voltage,very thick,	domain=0:10,samples=61] {e^(-x)}; 		% 0<t<t_1 Stromverlauf
		\addplot[color=voltage, very thick] coordinates {(10,0)(10,-1)}; 				% t=t_1 Stromsprung
		\addplot[color=voltage,very thick,	domain=10:20,samples=61] {-e^(-x+10)};	% t>t_1 Stromverlauf
		%\addplot[color=black,dashed, domain=0:1]{x}; % Gradient in t=0, schneidet Asymptote bei t=tau
	\end{axis}
	\begin{axis}[
		at={($(AxisSpannung.south west)+(0,-0.25cm)$)}, % Position 3. Plot
		anchor=north west,
		name=AxisStrom,
		%%%%%%% START SAME OPTIONS %%%%%%%
		xmin=-0.5, xmax=19.5,
		ymin=-0.125, ymax=1.375,
		xlabel={$t$},
		ylabel={$i_L(t)$},
		axis x line=center,
		axis y line=center,
		%ylabel style={anchor=east, at={(ticklabel cs:0.95)}},
		ylabel style={yshift=3pt},
		xtick distance=1, ytick distance=1,
		xticklabels={},yticklabels={},
		mark=none,
		grid=none,
		width=10cm,
		height=3.5cm,
		extra x ticks={1,5,10},
		extra x tick labels={$\tau$,$5\tau$,$t_1$},
		extra y ticks={1},
		extra y tick labels={$U_0/R$},
		%%%%%%% END OF SAME OPTIONS %%%%%%%
		clip=false,
	]
		\addplot[color=black,dashed, domain=0:5] {1}; % Asymptote
		\addplot[color=current,very thick,	domain=-1:0] {0};           	
		\addplot[color=current,very thick,	domain=0:10,samples=61] {1-e^(-x)};% t>0 
		\addplot[color=current,very thick,	domain=10:20,samples=61] {e^(-x+10)};% t>0 
	\end{axis}
\end{tikzpicture}