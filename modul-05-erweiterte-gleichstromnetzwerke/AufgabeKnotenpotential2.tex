\subsection{Knotenpotentialverfahren 2}
\Aufgabe{
    Im folgenden Netzwerk werden zwei parallel verschaltete Ersatzschaltbilder von Batteriezellen dargestellt. 
    Sie bestehen aus den Spannungsquellen $U_1$ und $U_2$ sowie den Innenwiderständen $R_{i1}$ und $R_{i2}$. 
    Führen Sie die Analyse mit dem Knotenpotentialverfahren durch. 

    \begin{center}
        \begin{circuitikz} 
            \draw (1,5.5) to [short, o-*] (1,5)
            (0,5) to [short] (2,5)
            (1,5.5) node [left] {A};
            \draw (0,1) [V<, name=U1] to (0,3);
            \draw (0,3) to [R=$R_\mathrm{i1}$] (0,5)
            (2,1) [V<, name=U2] to (2,3);
            \draw (2,3) to [R=$R_\mathrm{i2}$] (2,5)
            (1,0.5) to [short, o-*] (1,1)
            (0,1) to [short] (2,1)
            (1,0.5) node [left] {B};
            
            \varrmore{U1}{$U_\mathrm{1}$};
            \varrmore{U2}{$U_\mathrm{2}$};
        \end{circuitikz}
    \end{center}

    \begin{enumerate}
        \item[\bf a)] Formen Sie das Netzwerk um, sodass das Knotenpotentialverfahren anwendbar wird.  
        \item[\bf b)] Stellen Sie die Kontenadmittanzmatrix (KAM) auf.
        \item[\bf c)] Stellen Sie den Vektor der Knoteneinströmungen (I) auf.
        \item[\bf d)] Lösen Sie das Gleichungssystem und berechnen Sie die Spannung $U_0$ zwischen den Klemmen $A$ und $B$.
    \end{enumerate}
    }

\Loesung{
	\begin{itemize}
        \item[\bf a)]
            Umwandlung:

            \begin{figure}[H]
                \begin{circuitikz} 
                    \draw (0,0.5) to [I, name=I1] (0,3.5);
                    \draw (1.5,0.5) to [R=$G_\mathrm{i1}$] (1.5,3.5);
                    \draw (4.5,0.5) to [I, name=I2] (4.5,3.5);
                    \draw (3,0.5) to [R=$G_\mathrm{i2}$] (3,3.5);
                    \draw (0,3.5)to [short, -*] (1.5,3.5) to [short, -*] (2.25,3.5) to [short, -*] (3,3.5) to [short] (4.5,3.5)
                    (2.25,3.5) to [short,-o] (2.25,4);
                    \draw (2.25,4) node [left] {A};
                    \draw (2.25,3.5) node [above right] {$k_1$};
                    \draw (0,0.5)to [short, -*] (1.5,0.5) to [short, -*] (2.25,0.5) to [short, -*] (3,0.5) to [short] (4.5,0.5)
                    (2.25,0.5) to [short,-o] (2.25,0)
                    (2.25,0) node [left] {B};
                    \draw (2.25,0.5) node [below right] {$k_0$};
                    
                    \iarrmore{I1}{$I_\mathrm{1}$};
                    \iarrmore{I2}{$I_\mathrm{2}$};
                \end{circuitikz}
                \centering
            \end{figure}

            \begin{eq}
                I_1 = \frac{U_1}{R_\mathrm{i1}} = \frac{0,8\ V}{4,8\ \Omega} = 166\ mA      \qquad
                I_2 = \frac{U_2}{R_\mathrm{i2}} = \frac{1,5\ V}{0,8\ \Omega} = 1,875\ A      \nonumber
            \end{eq}

            \begin{eq}
                G_\mathrm{i1} = \frac{1}{R_\mathrm{i1}} = \frac{1}{4,8\ \Omega} = 208\ m\Omega      \qquad
                G_\mathrm{i2} = \frac{1}{R_\mathrm{i2}} = \frac{1}{0,8\ \Omega} = 1,25\ \Omega      \nonumber
            \end{eq}

		\item[\bf b)]
            Kontenadmittanzmatrix:

            \begin{eq}
                KAM = (G_\mathrm{i1} + G_\mathrm{i2}) = 208\ m\Omega + 1,25\ \Omega     \nonumber
            \end{eq}

        \item[\bf c)]
            Vektor der Knoteneinströmungen: \\

            \begin{eq}
                I = (I_1 + I_2) = 166\ mA + 1,875\ A    \nonumber
            \end{eq}

        \item[\bf d)]
            Gleichungssystem lösen:\\

            \begin{eqa}
                KAM \cdot U &= I     \nonumber\\ 
                (G_\mathrm{i1} + G_\mathrm{i2}) \cdot U_0 &= (I_1 + I_2)     \nonumber\\ 
                (208\ m\Omega + 1,25\ \Omega) \cdot U_0 &= (166\ mA + 1,875\ A)  \nonumber\\ 
                U_0 = \frac{I_1 + I_2}{G_\mathrm{i1} + G_\mathrm{i2}} 
                &= \frac{166\ mA + 1,875\ A}{208\ m\Omega + 1,25\ \Omega} 
                = 1,4\ V    \nonumber
            \end{eqa}

	\end{itemize}
}