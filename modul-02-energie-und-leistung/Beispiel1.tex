\subsection{Gleichstrommaschine 1\label{AufgGM1}}
\Aufgabe{
    Eine fremderregte Gleichstrommaschine hat die folgenden Angaben auf ihrem Typenschild:
    \begin{itemize}
        \item Nennleistung: $41,8\,\text{kW}$
        \item Nenndrehzahl: $1900\,\frac{1}{\text{min}}$
        \item Ankernennspannung: $440\,\text{V}$
        \item Ankernennstrom: $100\,\text{A}$
        \item Erregung: $240\,\text{V}$
        \item Erregerstrom: $10\,\text{A}$
        \item Leerlaufdrehzahl: $2000\,\frac{1}{\text{min}}$
    \end{itemize}

    Hinweis: Die Nennleistung auf einem Motortypenschild bezeichnet immer die abgegebene mechanische Leistung im Nennpunkt.

    \begin{itemize}
        \item[\bf a)]
        Wie groß ist das Drehmoment der Maschine im Nennpunkt?
        \item[\bf b)]
        Berechnen Sie den Wirkungsgrad der Maschine im Nennpunkt.	
        \item[\bf c)]
        Berechnen Sie das Produkt aus Erregerfluss und der Ankerkonstanten $K\cdot \Phi$
        \item[\bf d)]
        Wie groß ist der Ankerwiderstand $R_A$?
    \end{itemize}
}

\Loesung{
	\begin{itemize}
		
		\item[\bf a)]
		Aus der Nennleistung und der Nenndrehzahl ergibt sich direkt das Nenndrehmoment:
		\begin{align*}
		    P_\text{mech} &= M\cdot \omega\\
		    M &= \frac{P_\text{mech}}{\omega} = \frac{41,8\,\text{kW}}{2\pi\cdot 1900\,\frac{1}{\text{min}} \cdot \frac{1\,\text{min}}{60\,\text{s}}} = 210,08\,\text{Nm}
		\end{align*}
		\item[\bf b)]
		Im Motorbetrieb ist der Wirkungsgrad der Quotient aus mechanischer bezogen auf die elektrische Leistung (siehe Gleichung \ref{GlLeistung}). Hierbei muss sowohl die Ankerleistung, als auch die Erregerleistung berücksichtigt werden.
		\begin{eqa}
			\eta &= \frac{P_\text{mech}}{P_\text{elektr}} = \frac{41,8\,\text{kW}}{440\,\text{V}\cdot 100\,\text{A} + 240\,\text{V}\cdot 10\,\text{A}} = 90,08\,\%\nonumber
		\end{eqa}
		
		\item[\bf c)]
		Nach Gleichung \ref{GlInduzierteSpannungGM} wird die induzierte Spannung berechnet. Im Leerlauf muss die induzierte Spannung der Ankerspannung entsprechen. Gleiches geht aus Gleichung \ref{GlfremderregteGM1} hervor, da im Leerlauf auch das Drehmoment Null sein muss.
		\begin{eqa}
			U_q &= K \cdot \Phi \cdot \omega\tag{\ref{GlInduzierteSpannungGM}}\\
			K\cdot \Phi &= \frac{U_q}{\omega} = \frac{440\,\text{V}}{2\pi\cdot 2000\,\frac{1}{\text{min}} \cdot \frac{1\,\text{min}}{60\,\text{s}}} = 2,1\,\text{Vs}\nonumber
		\end{eqa}
		\item[\bf d)]
		Hier sind zwei Ansätze möglich. Es kann Gleichung \ref{GlfremderregteGM1} auf den Ankerwiderstand umgeformt werden:
		\begin{eqa}
			n &= \frac{U_A}{2\pi K\cdot \Phi} - \frac{R_A\cdot M_i}{2\pi(K\cdot \Phi)^2}\tag{\ref{GlfremderregteGM1}}\\
			R_A &= \left(\frac{U_A}{2\pi K\cdot \Phi} - n\right) \cdot \frac{2\pi(K\cdot \Phi)^2}{M_i}\nonumber\\
			&= \left(\frac{440\,\text{V}}{2\pi\cdot 2,1\,\text{Vs}} - \frac{1900}{60\,\text{s}}\right)\cdot \frac{2\pi\cdot (2,1\,\text{Vs})^2}{210,08\,\text{Nm}} = 220\,\text{m}\Omega\nonumber
		\end{eqa}
		Die andere Berechnungsvariante führt über das Ersatzschaltbild in Abbildung \ref{AbbGMFremderregt}. Im Nennbetrieb können wir die Quellenspannung $U_q$ berechnen. Diese muss kleiner als die Quellenspannung im Leerlauf sein.
		\begin{eqa}
			U_q &= K \cdot \Phi \cdot \omega \tag{\ref{GlInduzierteSpannungGM}}\\
			U_{q,n}&=  2,1\,\text{Vs} \cdot 2\pi\cdot 1900\,\frac{1}{\text{min}} \cdot \frac{1\,\text{min}}{60\,\text{s}} = 418\,\text{V}\nonumber
		\end{eqa}
		Der Spannungsabfall am Widerstand $R_A$ muss durch dem Maschenumlauf die Differenzspannung zwischen Ankerspannung und induzierter Spannung betragen. Da der Ankerstrom im Nennbetrieb bekannt ist, kann der Widerstand über das Ohmsche Gesetz berechnet werden.
		\begin{equation*}
		    R_A = \frac{U_A - U_q}{I_A} = \frac{440\,\text{V} - 418\,\text{V}}{100\,\text{A}} = 220\,\text{m}\Omega
		\end{equation*}
	\end{itemize}
}