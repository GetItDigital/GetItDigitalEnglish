\begin{circuitikz}[scale=0.7, transform shape]
    \ctikzset{
        resistors/scale=0.8,              
        tripoles/en amp/height=1.4, % Höhe des OPV             
        tripoles/en amp/width=1.4,   % Breite des OPV
        tripoles/en amp/input height=0.45
    }
    \draw
    (0,0) node[en amp] (opamp) {}
    % R2 mit korrigiertem Strompfeil
    (opamp.-) -- ++(-0.4,0) node[circ] {} to[R, l_=$R_2$] ++(-1.5,0) node[ocirc] (R2left) {}
    % R1 mit korrigiertem Strompfeil
    (opamp.-) -- ++(-0.4,0) --++(0,1) to[R, l_=$R_1$] ++(-1.5,0) node[ocirc] (R1left) {}
    % Erdung
    (opamp.+) -- ++(0,0) node[ground] {}
    % Ausgang
    (opamp.out) to[short, *-o] ++(0,0) node[right] {$U_{A}$}
    % R3 mit Strompfeil
    (opamp.out) -- ++(0,1.5) to[R, l_=$R_3$] ++(-2,0) -- ++(0,-1.05) node[circ] {}
    % Spannungen U_E1 und U_E2 an den Knoten links von R1 und R2
    (R1left) node[left] {$U_{E1}$}
    (R2left) node[left] {$U_{E2}$};
\end{circuitikz}