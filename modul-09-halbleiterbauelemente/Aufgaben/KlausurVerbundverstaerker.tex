\subsection{Transistor-Verbundverstärker\label{Aufg27}}
% Alt: Klausuraufgabe 2
\Aufgabe{
  Zwei npn-Bipolartransistoren bilden zusammen einen rauscharmen Verstärker für Signale im Bereich unter $1\, \mathrm{\mu V}$.
  Die Versorgungsspannung beträgt $U_1=10\,\mathrm{V}$. Am Kollektor des Transistors $\mathrm{T_2}$ liegt eine Spannung von $4\,\mathrm{V}$ gegenüber Masse an.
  Im Arbeitspunkt von $T_1$ und $T_2$ kann für die Basis-Emitter-Spannungen jeweils $U_\mathrm{BE}=600\,\mathrm{mV}$ angenommen werden.
  Die Kollektorströme $\mathrm{I_{C}}$ sind sehr groß gegenüber den Basisströmen $I_\mathrm{B}$ anzunehmen, daher gilt näherungsweise $I_\mathrm{C} = I_\mathrm{E}$.
  Zusätzlich sind die folgenden Parameter gegeben:

  $R_1 = 10\, \mathrm{k\Omega}$,
  $R_2 = 10\, \mathrm{k\Omega}$,
  $R_3 = 2\, \mathrm{k\Omega}$,
  $R_4 = 300\, \mathrm{k\Omega}$,
  $R_5 = 10\, \mathrm{M\Omega}$ \\
  $C_1 = 1\, \mathrm{\mu F}$,
  $C_2 = 1\, \mathrm{\mu F}$ \\
  $U_1 = 10\, \mathrm{V}$,
  $\hat{u}_\mathrm{E} = 0,1\, \mathrm{mV}$ \\

  \begin {figure} [H]
  \centering
  \begin{tikzpicture}
    \draw (0,0)
    node[npn] (Q1) {$T_1$};

    \draw (Q1.collector)
    to[short, -*] (0,1)
    to [R, l_=$R_\mathrm{1}$,] (0,4);

    \draw (Q1.base)
    to[short, -*] (-1,0)
    to[C, l_=$C_\mathrm{1}$](-4,0)
    to[sV, name=UE, v_] (-4,-4)
    node[ground]{};

    \draw (Q1.emitter)
    -- (0,-1)
    node[ground]{};

    \draw (-1,0)
    -- ++ (0,-2)
    to[R,l=$R_\mathrm{4}$, -*] (3,-2);

    \draw (3,0)
    node[npn] (Q2) {$T_2$};

    \draw (Q2.emitter)
    to[short](3,-2)
    to[R,l=$R_\mathrm{3}$] (3,-4)
    node[ground]{};

    \draw (0,1)
    -- (1,1)
    -- (2,0)
    to[short] (Q2.base);

    \draw (Q2.collector)
    to[short,-*](3,1)
    to[R,l_=$R_\mathrm{2}$](3,4)
    -- (6,4)
    to[V,name=U1, v] (6,2)
    node[ground]{};

    \draw (0,4)
    to[short, -*] (3,4);

    \draw (3,1)
    to[C,l=$C_\mathrm{2}$](6,1)
    to[R,l=$R_\mathrm{5}$](6,-1)
    node[ground]{};

    \varrmore{U1}{$U_\mathrm{1}$};
    \varrmore{UE}{$u_\mathrm{E}(t)$};
\end{tikzpicture}
  \label{fig:FigKlausurVerbundverstaerker}
  \end {figure}

  \begin{itemize}
    \item[a)]
          Wie groß ist der maximale Strom durch $R_2$ im Arbeitspunkt ($I_\mathrm{R2}=I_\mathrm{C,\,T2}$)?
    \item[b)]
          Wie groß ist der Strom durch $R_3$ näherungsweise, wenn $R_4 >> R_3$?
    \item[c)]
          Wie groß ist die Spannung, die über $R_3$ abfällt?
    \item[d)]
          Wie groß ist die Gleichspannung an der Basis von $T_2$ ($U_\mathrm{B,\,T2}=U_\mathrm{CE,\,T1}$)?
    \item[e)]
          Wie groß ist der Strom durch $R_1$ ($I_\mathrm{R1}=I_\mathrm{C,\,T1}$)?
    \item[f)]
          Erläutern Sie in Stichworten die Arbeitspunktstabiliserung der gesamten Schaltung auf Grund der Gegenkopplung über $R_4$.
    \item[g)]
          Wie groß ist die Steilheit $S$ von $T_\mathrm{1}$, wenn zusätzlich $U_\mathrm{T} = 26\, \mathrm{mV}$ gegeben ist?
    \item[h)]
          Wie groß ist die Spannungsverstärkung $A_\mathrm{V} = \frac{\Delta U_\mathrm{CE}}{\Delta U_\mathrm{BE}}$, der ersten Transistorstufe?
    \item[i)]
          Wie groß ist die Spannungsverstärkung $A_\mathrm{V}$ der zweiten Transistorstufe, unter Verwendung der Definition $A_\mathrm{V,T2}=\frac{\Delta U_\mathrm{C}}{\Delta U_\mathrm{B}}$?
          Zur Lösung müssen zusätzlich die Widerstände $R_2$ und $R_3$ berücksichtigt werden.
    \item[j)]
          Wie groß ist die Spannungsverstärkung der beiden Stufen $T_\mathrm{1}$ und $T_\mathrm{2}$ zusammen?
    \item[k)]
          Zeichnen Sie das Kleinsignal-Ersatzschaltbild der gesamten Schaltung. Ersetzen Sie darin die beiden Transistoren im Arbeitspunkt jeweils durch eine steuerbare Stromquelle $S=\beta\cdot i_\mathrm{B}$ und einen differentiellen Eingangswiderstand $r_\mathrm{BE}$.
  \end{itemize}
}



\Loesung{
  \begin{itemize}
    \item[a)]
          \begin{align*}
            I_\mathrm{C} & = \frac{6\,\mathrm{V}}{10\,\mathrm{k\Omega}} = 0,6\,\mathrm{mA}
          \end{align*}

    \item[b)]
          \begin{align*}
            I_\mathrm{C,T2} & = 0,6\,\mathrm{mA}
          \end{align*}

    \item[c)]
          \begin{align*}
            U_{\mathrm{R3}} & = 0,6\,\mathrm{mA} \cdot 2\,\mathrm{k\Omega} = 1,2\, \mathrm{V}
          \end{align*}

    \item[d)]
          \begin{align*}
            1,2\, \mathrm{V} + 0,6\,\mathrm{V} = 1,8\,\mathrm{V}
          \end{align*}

    \item[e)]
          \begin{align*}
            I_\mathrm{C,T1} & = \frac{8,2\,\mathrm{V}}{10\,\mathrm{k\Omega}} = 0,82\,\mathrm{mA}
          \end{align*}

    \item[f)]
    \item Erhöhung von $I_\mathrm{C}$ führt zu einem Spannungsanstieg über $R_4$
    \item dadurch sinkt die Basisspannung von $T_\mathrm{1}$, was $I_\mathrm{C}$ wieder reduziert
    \item negative Rückkopplung stabilisiert Arbeitspunkt

    \item[g)]
          \begin{align*}
            S & = \frac{I_\mathrm{C,T1}}{U_\mathrm{T}} = \frac{0,82\,\mathrm{mA}}{26\,\mathrm{mA}} = 31,54\, \frac{\mathrm{mA}}{\mathrm{V}}
          \end{align*}

    \item[h)]
          \begin{align*}
            A_\mathrm{V} & = S \cdot (-R_\mathrm{C}) = -315
          \end{align*}

    \item[i)]
          \begin{align*}
            A_\mathrm{V} & = -\frac{R_\mathrm{C}}{R_\mathrm{E}} = \frac{10\,\mathrm{k\Omega}}{2\,\mathrm{k\Omega}} = -5
          \end{align*}

    \item[j)]
          \begin{align*}
            -5 \cdot -315 = 1575 \approx 1600
          \end{align*}

    \item[k)] Ersatzschaltbild
          \begin{figure}[H]
            \centering
            \begin{tikzpicture}
    \draw (0,0) to[short, o-*] (1,0) to[short, -o] (3,0)node[above]{B};
    \draw (5,0)node[above]{C} to[short, o-*] (7,0) to[short,-o] (8,0)node[above]{B};
    \draw (10,0)node[above]{C} to[short, o-*] (12,0) to[short, -*] (13,0) to[short, -o] (15,0);
    \draw (0,-2) to[short, o-*] (1,-2) -- (1.5,-2)node[ground]{};
    \draw (1.5,-2) to[short, -] (3,-2) to[short, -*] (4,-2)node[below]{E} to[short, -*] (7,-2) to[short, -*] (9,-2)node[below]{E} to[short, -] (10.5,-2) to[R=$R_3$, -*] (12,-2) to[short, -*] (13,-2) to[short, -o] (15,-2);


    \draw (15,0) to[open,v>, name=ua] (15,-2);
    \draw (0,0) to[open,v>, name=ue] (0,-2);

    \draw (1,0) to[R=$R_4$] (1,-2);
    \draw (3,0) to[R=$r_\mathrm{BE 1}$] (3,-2);
    \draw (5,0) to[I, l=$S_1$] (5,-2);
    \draw (7,0) to[R=$R_1$] (7,-2);
    \draw (8,0) to[R=$r_\mathrm{BE2}$] (8,-2);
    \draw (10,0) to[I, l=$S_2$] (10,-2);
    \draw (12,0) to[R=$R_2$] (12,-2);
    \draw (13,0) to[R=$R_5$] (13,-2);


    \varrmore{ue}{$u_\mathrm{E}$};
    \varrmore{ua}{$u_\mathrm{A}$};

\end{tikzpicture}
          \end{figure}
          Siehe: Abschnitt \ref{sec:DarlingtonTransistor}
  \end{itemize}
}