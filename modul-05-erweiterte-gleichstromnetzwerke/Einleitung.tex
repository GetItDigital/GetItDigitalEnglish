\begin{frame}
	\fta{Einleitung}

	\s{
		Während die Berechnungen von Serienschaltungen ohne parallele geschaltete 
		Bestandteile recht aufwendungsarm zu bewerkstelligen sind, wird die Analyse von 
		Gleichstromnetzwerken mit Knoten, welche n Abzweigungen aufweisen, wobei die
		Anzahl der Zweige dabei $n \geq 2$ ist, bei steigender Knotenanzahl schnell 
		unübersichtlich. Eine vielschichtige Betrachtung des Gleichstromnetzwerkes ist 
		hier von Nöten. Für die Berechnung von Gleichstromnetzwerken stehen die 
		folgenden Methoden zur Verfügung: 
		}

	\b{
		In dem Modul 'Erweiterten Gleichstromnetzwerke' werden die folgenden Inhalte
		erläutert:
		}

	\begin{itemize}
		\item Knoten- und Maschenregel
		\item Superposition
		\item Knotenpotentialanalyse
		\item Maschenstromanalyse
	\end{itemize}

	\s{
		Auch die Analyse von elektrischen Netzwerken mit mehr als einer Strom- 
		oder Spannungsquelle ist nicht trivial. Die Bestimmung aller Ströme und 
		Spannungen des elektrischen Netzwerkes aus Abbildung \ref{BildEinleitungsnetzwerk} 
		ist allein durch die Berechnungen von Serienschaltungen und Parallelschaltungen nicht 
		möglich. 
		}
 
	\fu{
		\resizebox{.5\textwidth}{!}{%
		\begin{circuitikz}
    \draw (1,2) to [R=$ $,-*] (2.5,3.5)
    to [R=$ $,-*] (4.5,3.5)
    to [R=$ $,-*] (6,2)
    to [R=$ $,-*] (6,0)
    to [R=$ $,-*] (4.5,-1.5)
    to [R=$ $,-*] (2.5,-1.5)
    to [R=$ $,-*] (1,0)
    to [R=$ $,-*] (1,2)
    (0,0) to [I=\red{$ $}] (0,2)
    (0,2) to [short] (1,2)
    (0,0) to [short] (1,0)
    (4.5,0) to [I=\red{$ $}] (4.5,2)
    (4.5,0) to [short] (6,0)
    (4.5,2) to [short] (4.5,3.5)
    (2.5,3.5) to [R=$ $] (2.5,1)
    (2.5,1) to [V={$ $}] (2.5,-1.5)
    (6,2) to [short] (7,2)
    to [R=$ $] (7,0)
    to [V={$ $}] (7,-1.5)
    to [short] (4.5,-1.5);
\end{circuitikz}%
		}
	}{{\bf Beispielnetzwerk.} Elektrisches Gleichstromnetzwerk mit einer Vielzahl an Widerständen,
	Stromquellen und Spannungsquellen. \label{BildEinleitungsnetzwerk}}
\end{frame}