\subsection{Ermittlung von Basis-Emitter-Spannung und Basisstrom\label{Aufg12}}
% Alt: Aufgabe 12
\Aufgabe{
  Gegeben ist die abgebildete Transistorschaltung mit den folgenden Parametern: $U_0=\mathrm{5\,V}$ ,
  $U_L=\mathrm{15\,V}$, $R_\mathrm{V}=\mathrm{100\,k\Omega}$ und $R_\mathrm{V}=\mathrm{500\,\Omega}$.
  
  \begin{figure}[H]
    \centering
    \begin{tikzpicture}
     \draw (4,2) node[npn, tr circle] (Q1) {};
 
     \draw (Q1.C) -- (4,4) 
     to [R,l=$R_\mathrm{L}$] (7,4)
     to [V, name=Ul, v] (7,0);

     \draw (Q1.E) to [short, -*] (4,0) 
     -- (7,0);

     \draw (Q1.B) to [R,l_=$R_\mathrm{V}$, i<, name=Ib] (0,2)
     to [V, name=U0, v] (0,0)
     to [short] (4,0);

     \draw (Q1.B) ++ (-0.25,0) coordinate(Tb);
     \draw (Q1.E) ++ (0,-0.25) coordinate(Te);
     \draw (Tb) to [open, name=ube, v, voltage=european] (Te);

     \varrmore{ube}{$U_\mathrm{BE}$};
     \varrmore{U0}{$U_\mathrm{0}$};
     \varrmore{Ul}{$U_\mathrm{L}$};
     \iarrmore{Ib}{$I_\mathrm{B}$};
 \end{tikzpicture}
    \label{fig:FigTransistorBasisstrom}
  \end{figure}

  \begin{itemize}
    \item[a)] Wie groß ist die Spannung $U_\mathrm{BE}$ näherungsweise, wenn es sich um einen Siliziumtransistor handelt?
    \item[b)] Wie groß ist der Strom $I_\mathrm{B}$?
  \end{itemize}
}

\Loesung{
  \begin{itemize}
    \item[a)]
          Diese Spannung ist eine materialbedingte Eigenschaft des Silizium-Transistors und bleibt nahezu konstant, unabhängig von der Basisspannung \( U_\text{B} \).
          Ein npn-Transistor hat im leitenden Zustand eine typische Basis-Emitter-Spannung von:
          \begin{equation*}
            U_\text{BE} \approx 0,6\,\text{V} \text{ bis } 0,7\,\text{V}
          \end{equation*}
          
          Obwohl \( U_\text{0} = 5\,\text{V} \) ist, begrenzt der Transistor die Spannung \( U_{BE} \) auf ca. \( 0,6\,\text{V} \), sobald der Transistor leitend ist.  
          Der überschüssige Spannungsabfall wird durch andere Ströme und Bauteile der Schaltung ausgeglichen.
          
    \item[b)]
          \begin{align*}
            I_\text{B} = \frac{5\,\text{V} - 0,6\,\text{V}}{100\,\text{k}\Omega} = 44\,\mu\text{A}
          \end{align*}
  \end{itemize}
  Siehe: Abschnitt \ref{sec:Bipolartransistor} und \ref{fig:StroemeUndSpannungenBeimBipolartransistor}
}