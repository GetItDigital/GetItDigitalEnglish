\subsection{Dimensionierung einer Komparatorschaltung mit Diode\label{Aufg7}}
% Alt: Aufgabe 7
\Aufgabe{
    Dimensionieren Sie die Beschaltung des Komparators so, dass das unten skizzierte Übertragungsverhalten entsteht.
    Die Durchlassspannung der Diode wird mit $U_\mathrm{D} = \mathrm{0,7\,V}$ angenommen. 
    Die maximale Verlustleistung des Widerstandes beträgt 1/4\,W.
\begin{figure}[H]
    \centering
    % Linkes Bild
    \begin{subfigure}[b]{0.45\textwidth}
        \centering
        \begin{tikzpicture}
\ctikzset{tripoles/en amp/input height=0.45}
\draw (0,0)node[en amp](E){N1}
    (E.out)
    (E.-)
    (E.+);
    \draw (E.-) to[short, -o] (-2,0.5);
    \draw (E.+) -- (-1.5,-0.5) -- (-1.5,-2) to[short, -*] (2,-2);
    \draw (E.out) -- (2,0) to[R,l=$R$, *-] (2,-2);
    \draw (2,0) to[short, -o] (3.5,0);

    \draw (2,-3) to[zDo, l=$V_\mathrm{1}$] (2,-2);
    \draw (2,-3) to[zDo, l=$V_\mathrm{2}$] (2,-4)node[ground]{};
    \draw (-2,-3.9) to[short, o-] (-2,-4)node[ground]{};
    \draw (3.5,-3.9) to[short, o-] (3.5,-4)node[ground]{};

    \draw (-2,0.5) to[open, v, name=ue] (-2,-4);
    \draw (3.5,0) to[open, v, name=ua] (3.5,-4);

    \varrmore{ue}{$U_\mathrm{e}$};
    \varrmore{ua}{$U_\mathrm{a}$};
\end{tikzpicture}
    \end{subfigure}
    % Rechtes Bild
    \begin{subfigure}[b]{0.45\textwidth}
        \centering
        \includesvg[width=0.9\textwidth]{Bilder/Aufgaben/FigKomparatorDiode}
    \end{subfigure}
    \label{fig:FigKomparatorDiode}
\end{figure}
%\speech{
%    Aufgabe 7: Dimensionierung eines Komparators  
%    
%    Gegeben ist eine Schaltung mit einem Komparator, der so dimensioniert werden soll,  
%    dass das dargestellte Übertragungsverhalten erreicht wird.  
%    
%    Die Schaltung besteht aus einem Operationsverstärker, der als Komparator arbeitet.  
%    Der nichtinvertierende Eingang ist mit Masse verbunden.  
%    Der invertierende Eingang erhält die Eingangsspannung U E.  
%    Am Ausgang befindet sich ein Widerstand R, der in Reihe mit zwei antiparallel geschalteten Dioden liegt.  
%    Diese Dioden begrenzen die Ausgangsspannung U A.  
%    
%    Zusätzlich sind folgende Randbedingungen gegeben:  
%    - Die Durchlassspannung der Dioden beträgt U D = 0,7 Volt.  
%    - Die maximale Verlustleistung des Widerstands beträgt 1/4 Watt.  
%    
%    Rechts neben der Schaltung ist ein Diagramm dargestellt, das das gewünschte Ausgangsverhalten zeigt.  
%    Die grüne Linie stellt die Eingangsspannung U E dar,  
%    die sich als periodisches Signal mit ansteigenden und abfallenden Spannungswerten verhält.  
%    Die blaue Linie stellt die Ausgangsspannung U A dar.  
%    Diese nimmt nur feste Spannungswerte an, was auf eine Begrenzung durch die Dioden hinweist.  
%    
%    Die Aufgabe besteht darin, die Werte der Bauteile so zu wählen, dass das dargestellte Verhalten erreicht wird.  
%    }
    
}


\Loesung{
    \textbf{Formel:}
    \begin{equation}
        P_\mathrm{R} = \frac{U_\mathrm{R}^2}{R}
    \end{equation}

    \textbf{Berechnung:}

    Spannung am Widerstand $R$:
    \begin{align*}
        U_\mathrm{R} &= U_\mathrm{a} - U_\mathrm{D} \\
        &= 14\,\mathrm{V} - 0,7\,\mathrm{V} \\
        &= 13,3\,\mathrm{V}
    \end{align*}

    Berechnung des Widerstands $R$:
    \begin{align*}
        P_\mathrm{R} &= \frac{U_\mathrm{R}^2}{R} \leq \frac{1}{4}\,\mathrm{W} \\
        R &= \frac{U_R^2}{P_{\text{max}}} \\
        &= \frac{(13,3\,\mathrm{V})^2}{0,25\,\mathrm{W}} \\
        &= \frac{176,89\,\mathrm{V}^2}{0,25\,\mathrm{W}} \\
        &= 707,56\,\Omega
    \end{align*}

    Normierter Widerstandswert:
    \begin{align*}
        R &= 710\,\Omega
    \end{align*}

    Überprüfung der Verlustleistung:
    \begin{align*}
        P_\mathrm{R} &= \frac{(13,3\,\mathrm{V})^2}{710\,\Omega} \approx 0,25\,\mathrm{W}
    \end{align*}

    Der Widerstand $R$ sollte den Wert:
    \begin{align*}
        R &= 710\,\Omega
    \end{align*}
    haben, um die maximale Verlustleistung nicht zu überschreiten und das gewünschte Übertragungsverhalten zu gewährleisten.
%    \speech{
%        Berechnung des Widerstands zur Einhaltung der maximalen Verlustleistung  
%        
%        Gegeben ist eine Schaltung mit einem Widerstand R, der so dimensioniert werden soll,  
%        dass die maximale Verlustleistung von einem Viertel Watt nicht überschritten wird.  
%        
%        Die Verlustleistung an einem Widerstand wird mit folgender Formel berechnet:  
%        P R = U R Quadrat durch R  
%        
%        Berechnung der Spannung am Widerstand:  
%        U R = U A - U D  
%        U R = 14 Volt - 0,7 Volt = 13,3 Volt  
%        
%        Berechnung des Widerstandswerts unter Berücksichtigung der maximalen Verlustleistung:  
%        P R = U R Quadrat durch R ist kleiner gleich 0,25 Watt  
%        
%        Umstellen nach R ergibt:  
%        R = U R Quadrat durch P max  
%        
%        Einsetzen der Werte:  
%        R = ( 13,3 Volt ) Quadrat durch 0,25 Watt  
%        R = 176,89 Volt Quadrat durch 0,25 Watt  
%        R = 707,56 Ohm  
%        
%        Der nächstgelegene normierte Widerstandswert ist:  
%        R = 710 Ohm  
%        
%        Überprüfung der Verlustleistung mit diesem Widerstandswert:  
%        P R = ( 13,3 Volt ) Quadrat durch 710 Ohm  
%        P R ist ungefähr 0,25 Watt  
%        
%        Das Ergebnis zeigt, dass ein Widerstand von 710 Ohm verwendet werden sollte,  
%        um die maximale Verlustleistung nicht zu überschreiten und das gewünschte Übertragungsverhalten zu gewährleisten.  
%        }
Siehe: Abschnitt \ref{fig: Verschaltung Verstaerker} und Tabelle \ref{tab:Grundschaltungen}

    }