\subsection{Analyse einer Spannungsverdoppler-Schaltung\label{Aufg33}}
% Alt: Praktikumsaufgabe 5
\Aufgabe{

    Analysieren Sie die Funktion der dargestellten Delon-Schaltung zur Spannungsverdopplung und berechnen Sie die Ausgangsspannung für verschiedene Eingangsspannungen.
    Zusätzlich sind die folgenden Parameter gegeben:
    
    $U_\mathrm{E1,eff}=110\,\mathrm{V}$,  $U_\mathrm{E2,eff}=230\,\mathrm{V}$ und der Diodendurchlassspannung von $U_\mathrm{f} = 0,7\,\mathrm{V}$.
    
    \begin{figure}[H]
    \centering
    \begin{tikzpicture}
    \draw (0,4)
    to[D,l=$D_\mathrm{1}$, *-*] (2,4)
    to[C,l=$C_\mathrm{1}$, -*] (2,2) 
    to[C,l=$C_\mathrm{2}$, -*] (2,0) 
    to[D,l=$D_\mathrm{2}$] (0,0) -- (0,4); 

    \draw (0,4) -- (-2,4)
    to[sV,name=UE, v_] (-2,2) -- (2,2);

    \draw (2,4) -- (4,4)
     to[R,l=$R_\mathrm{L}$] (4,0) -- (2,0); 
     
    \draw (4,4)
    to[short, *-o] (5.5,4)
    to[open,v^,name=UA] (5.5,0)
    to[short, o-*] (4,0);

    \varrmore{UE}{$u_{\mathrm{E}_i}(t)$};
    \varrmore{UA}{$U_\mathrm{A}$};
\end{tikzpicture}
    \label{fig:FigPraktikumSpannungsverdoppler}
    \end{figure}
    
    \begin{itemize}
        \item[a)] Erklären Sie das Funktionsprinzip der Delon-Schaltung und wie diese die Eingangsspannung verdoppelt.
        \item[b)] Berechnen Sie die theoretische Leerlauf-Ausgangsspannung $U_\mathrm{A}$ für die gegebenen Eingangsspannungen.
        \item[c)] Diskutieren Sie, wie sich die Ausgangsspannung $U_\mathrm{A}$ unter Lastbedingungen verhält und welche Faktoren die Spannungsstabilität beeinflussen.
    \end{itemize}
}


\Loesung{
    \begin{itemize}
        \item[a)] Funktionsprinzip der Delon-Schaltung  \\
              Die Schaltung arbeitet als Kaskaden-Gleichrichter. Während der positiven Halbwelle der Wechselspannung lädt sich $C_1$ über $D_1$ auf die Eingangsspitzen-Spannung $\hat{U}_\mathrm{E}$ auf. Während der negativen Halbwelle lädt sich $C_2$ über $D_2$ auf. Dadurch addieren sich die Spannungen von $C_1$ und $C_2$, wodurch am Ausgang eine theoretische Gleichspannung von ca. $2 \cdot \hat{U}_\mathrm{E}$ entsteht.
              
        \item[b)] Berechnung der Leerlauf-Ausgangsspannung $U_\mathrm{A}$
              Die Eingangsspannung ist als Effektivwert $U_\mathrm{E}$ gegeben. Die Spitzenwertspannung beträgt:
              $$ \hat{U}_\mathrm{E} = \sqrt{2} \cdot U_\mathrm{E} $$
              Da die Schaltung eine Spannungsverdopplung erzeugt, ergibt sich die Leerlauf-Ausgangsspannung:
              $$ U_\mathrm{A} = 2 \cdot (\hat{U}_\mathrm{E} - U_\mathrm{D}) = 2 \cdot (\sqrt{2} \cdot U_E - 0{,}7\,\mathrm{V}) $$
              Für $ U_\mathrm{E} = 110\, \mathrm{V} $:
              $$ U_\mathrm{A} = 2 \cdot (\sqrt{2} \cdot 110\, \mathrm{V} - 0{,}7\, \mathrm{V}) \approx 308\, \mathrm{V} $$
              Für $ U_\mathrm{E} = 230\,V $:
              $$ U_\mathrm{A} = 2 \cdot (\sqrt{2} \cdot 230\,\mathrm{V} - 0{,}7\, \mathrm{V}) \approx 644\, \mathrm{V} $$
              
        \item[c)] Verhalten unter Lastbedingungen
              Im Leerlauf bleibt die Ausgangsspannung hoch. Unter Last fällt die Spannung ab, abhängig vom Innenwiderstand der Kondensatoren, der Belastung $R_\text{Last}$ und den Diodenverlusten. Der Spannungsabfall hängt von der Restwelligkeit ab, die durch die Lade- und Entladezeiten der Kondensatoren beeinflusst wird. Ein großer Laststrom reduziert die Spannung, da die Kondensatoren sich schneller entladen.
              
              Faktoren, die die Spannung beeinflussen:
              \begin{itemize}
                  \item Innenwiderstand der Dioden
                  \item Kapazität der Kondensatoren $C$
                  \item Belastung durch $R_\text{Last}$
                  \item Netzfrequenz ($50\, \text{Hz}$ oder $60 \, \text{Hz}$)
              \end{itemize}
    \end{itemize}
}