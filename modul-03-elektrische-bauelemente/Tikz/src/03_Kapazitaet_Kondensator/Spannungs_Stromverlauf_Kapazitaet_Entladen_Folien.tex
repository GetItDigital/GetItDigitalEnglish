\begin{tikzpicture}
    \begin{axis} [
            xlabel={Zeit},
            ylabel={\textcolor{cyan}{$u(t)$}/ \textcolor<3->{voltage}{$u_\mathrm{C}(t)$}/ \textcolor<5->{current}{$i(t)$}},
            xmin=0, xmax=12.5,
            ymin=-1.2, ymax=1.2,
            ytick={-1,-0.63,-0.37,0, 0.37, 0.63, 1},
            yticklabels={
                    -100\%, \only<6->{-63\%}, \only<6->{-37\%},
                    0, \only<4->{37\%}, \only<3->{63\%}, 100\%
                },
            xtick={0, 1, 2, 3, 4, 5, 6, 7, 8, 9, 10, 11, 12},
            xticklabels={0, \(\tau\), \(2\tau\), \(3\tau\), \(4\tau\), \(5\tau\), \(6\tau\), \(7\tau\), \(8\tau\),
                    \(9\tau\), \(10\tau\), \(11\tau\), \(12\tau\)},
            width=10cm,
            height=5cm,
            axis lines=middle,
            every axis x label/.style={at={(current axis.right of origin)},anchor=west},
            every axis y label/.style={at={(current axis.above origin)},anchor=south},
            font=\small,
            legend style={font=\scriptsize}
        ]



        % Rechtecksignal für die Spannung
        \addplot [cyan, thick] coordinates {(0,1) (5,1) (5,0) (12,0)};

        % Spannungskurve eines Kondensators (Ladekurve)
        \only<3->{\addplot [voltage, domain=0:5, samples=100] {ifthenelse(x<0, 0, 1 - exp(-(x)))};}
        \only<5->{\addplot [current, domain=0:5, samples=100] {ifthenelse(x<0, 0,     exp(-(x)))};}

        % Spannungskurve eines Kondensators (Entladekurve)
        \only<4->{\addplot [voltage, domain=5:12, samples=100] {ifthenelse(x<5, 0, exp(-(x-5)))};}
        \only<6->{\addplot [current, domain=5:12, samples=100] {ifthenelse(x<5, 0,-exp(-(x-5)))};}

        % Legende mit nur selektiv sichtbaren Einträgen
        % Legenden separat für Spannung und Strom
        \only<2->{\legend{ $u(t)$}}
        \only<3->{\legend{ $u(t)$, $u_\mathrm{C}(t)$}}
        \only<4->{\legend{ $u(t)$, $u_\mathrm{C}(t)$}}
        \only<5->{\legend{ $u(t)$, $u_\mathrm{C}(t)$, $i(t)$}}
        \only<6->{\legend{ $u(t)$, $u_\mathrm{C}(t)$, $i(t)$}}

        % Markierungen der Zeitkonstanten
        \only<3->{\draw[dashed] (axis cs:1,0) -- (axis cs:1,0.63) node[right] {};}
        \only<4->{\draw[dashed] (axis cs:6,0) -- (axis cs:6,0.37) node[right] {};}
        \only<3->{\draw[dashed] (axis cs:1,0) -- (axis cs:1,0.37) node[right] {};}
        \only<6->{\draw[dashed] (axis cs:6,-1) -- (axis cs:6,-0.37) node[right] {};}
        \only<5->{\draw[dashed] (axis cs:0,0.37) -- (axis cs:1,0.37) node[right] {};}
        \only<3->{\draw[dashed] (axis cs:0,0.63) -- (axis cs:1,0.63) node[right] {};}
        \only<6->{\draw[dashed] (axis cs:0,-0.37) -- (axis cs:6,-0.37) node[right] {};}
        \only<4->{\draw[dashed] (axis cs:0,0.37) -- (axis cs:6,0.37) node[right] {};}
        \only<6->{\draw [dashed] (axis cs:0,-1) -- (axis cs:12,-1);}
        \only<3->{\draw [dashed] (axis cs:5,1) -- (axis cs:12,1);}

        % Punkte zur Hervorhebung
        \only<3->{\addplot[
                only marks,
                mark=*,
                mark options={scale=0.7, fill=black}
            ] coordinates {
                    (1, 0.63)
                };}
        \only<5->{\addplot[
                only marks,
                mark=*,
                mark options={scale=0.7, fill=black}
            ] coordinates {
                    (1, 0.37)
                };}
        \only<4->{\addplot[
                only marks,
                mark=*,
                mark options={scale=0.7, fill=black}
            ] coordinates {
                    (6, 0.37)
                };}
        \only<6->{\addplot[
                only marks,
                mark=*,
                mark options={scale=0.7, fill=black}
            ] coordinates {
                    (6, -0.37)
                };}

    \end{axis}
\end{tikzpicture}