\subsection{Komplexe Wechselstromrechnung 3}
\Aufgabe{

    Das nachfolgende elektrische Netzwerk soll als Beispiel eines Ersatzschaltbildes einer Lithium-Ionen Zelle auf seine
    komplexen Größen analysiert werden. Das Ersatzschaltbild besteht aus einer Serienschaltung aus einer Spule $L$, dem 
    Serienwiderstand $R_\mathrm{S}$ und der Paralellschaltung der Paralellwiderstandes $R_\mathrm{P}$ und dem paralell
    verschalteten Kondensator $C_\mathrm{P}$. Die Berechnung der komplex Ströme und Spannungen aller Komponenten soll in 
    der folgenden Reihenfolge stattfinden:
    
    \begin{itemize}
        \item[\bf a)] Größe der komplexen Impedanzen aller Komponenten und der gesamten komplexen Impedanz
        \item[\bf b)] Komplexe Ströme durch die Komponenten
        \item[\bf c)] Komplexe Spannungen an den Komponenten
    \end{itemize}
    
    \begin{figure}[H]
        \centering
        \resizebox{0.7\textwidth}{!}{
            \begin{circuitikz}

    \draw(0,0) to [L=${L=900\ nH}$, v, i, name=L, o-] (3,0);
    \varrmore{L}{$U_\mathrm{L}$};
    \iarrmore{L}{$I_\mathrm{L}$};

    \draw(3,0) to [R=${R_\mathrm{S}=20\ m\Omega}$, v, i, name=Rs, -*] (6,0);
    \varrmore{Rs}{$U_\mathrm{R_\mathrm{S}}$};
    \iarrmore{Rs}{$I_\mathrm{R_\mathrm{S}}$};

    \draw(6,-1) to [short] (6,1);
    \draw(9,-1) to [short] (9,1);
    \draw(9,0) to [short, *-o] (10,0);

    \draw(6,1) to [R=${R_\mathrm{P}=15\ m\Omega}$, v, i, name=Rp] (9,1);
    \varrmore{Rp}{$U_\mathrm{R_\mathrm{P}}$};
    \iarrmore{Rp}{$I_\mathrm{R_\mathrm{P}}$};

    \draw(6,-1) to [C=${C_\mathrm{P}=200\ mF}$, v, i, name=C] (9,-1);
    \varrmore{C}{$U_\mathrm{C_\mathrm{P}}$};
    \iarrmore{C}{$I_\mathrm{C_\mathrm{P}}$};
\end{circuitikz}
        }
    \end{figure}
}


\Loesung{
    \begin{itemize}

        \item[\bf a)]
              \begin{eq}
                  X_\mathrm{L} = j \omega \cdot L = j \cdot 2\pi \cdot 800\ Hz \cdot 900\ nH  \nonumber
              \end{eq}
              \begin{eq}
                  X_\mathrm{Rs} = R_\mathrm{S} = 20\ m\Omega
              \end{eq}
              
        \item[\bf b)]
        
        \item[\bf c)]  
        
    \end{itemize}
    
}