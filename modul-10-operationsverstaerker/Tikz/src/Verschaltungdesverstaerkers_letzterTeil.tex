    \begin{circuitikz}[scale=0.8, transform shape]
        \ctikzset{tripoles/en amp/input height=-0.45,
        }
        \draw (0,0) node[en amp](E){};
        \node at ($(E) + (-1.2, 1.5)$) {\textbf{\LARGE D}}; % Label für den OPV im dritten Bild
        \draw (E.out) node[circ]{} -- ++(0.2,0) node[ocirc, right]{};
        \draw (E.-) -- ++(0, -1.3) node[circ] {} to[R, l=$R_2$] ++(2.38,0) -- ++(0,1.8);
        \draw (E.-) -- ++(0, -1.6) to[R, l_=$R_1$] ++(0, -1.2) -- ++(0,-0.2) -- ++(-0.2, 0) -- ++(0.4,0);
        \draw (E.+) node[circ]{} node[left]{};     
       
        % Spannungspfeil
        \draw[-{Triangle[width=3pt,length=4pt]}, color=spannung] ($(E.+) + (0, -0.1)$) -- ($(E.-) + (0, 0.1)$) node[midway, xshift=-32] {$U_{Diff}=0\,\text{V}$};
        \draw[-{Triangle[width=3pt,length=4pt]}, color=spannung] ($(E.out) + (0.26, -0.1)$) -- ($(E.out) + (0.26, -3.45)$) node[midway, xshift=10] {$U_A$};
        \draw[black] ($(E.out) + (0.26, -3.5)$) -- ++(-0.2, 0) -- ++(0.4,0);
        \draw[-{Triangle[width=3pt,length=4pt]}, color=spannung] (0.7,-2.1)-- ++(-1.5,0) node[midway, yshift=-8] {$U_{R2}$};
        \draw[-{Triangle[width=3pt,length=4pt]}, color=spannung] (-0.9,-2)-- ++(0,-1.5) node[midway, xshift=10] {$U_{R1}$};


        
        \node[red] at (0.9,-1.8) {\tikz \draw[red, -{Triangle[width=3pt,length=4pt]}] (0,0) -- (-0.01,0);};
        \node[above, color=red] at (0.9,-1.8)  {$I_{R2}$};

        \node[red] at ($(E.-)+(0,-1.5)$) {\tikz \draw[red, -{Triangle[width=3pt,length=4pt]}] (0,0) -- (0,-0.01);};
        \node[left, color=red] at ($(E.-)+(0,-1.4)$)  {$I_{R1}$};
      
        % Strompfeil oberhalb von R1
        %\draw[->, thick, red] (-1.19,-1) -- ++(0, -0.01);
        %\node[red] at (-1.5, -1) {$I_{R_1}$};
    \end{circuitikz}