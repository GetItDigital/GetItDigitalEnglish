\subsection{Berechnung des Vorwiderstands einer LED\label{Aufg10}}
% Alt: Aufgabe 10
\Aufgabe{
  Wie bei allen Dioden muss auch bei Leuchtdioden (LEDs) der Strom begrenzt werden. Dazu wird ein Vorwiderstand in Reihe geschaltet, wie in der unten gezeigten Schaltung. \newline
  Die eingesetzte LED besitzt eine Durchlassspannung von $U_\mathrm{D} = 2,2\,\mathrm{V}$. Die ideale Helligkeit wird bei einem Strom von $I_\mathrm{D} = 30\,\mathrm{mA}$ erreicht. Die Eingangsspannung beträgt $U_\mathrm{0} = 5\,\mathrm{V}$.
  
  \begin{figure}[H]
     \centering
     \begin{tikzpicture}
  \draw (0,2) to[V, name=U0] (0,0);
  \draw (0,2) to[R=$R_\mathrm{V}$, i, name=Id] (4,2);
  \draw (4,2) to[leDo, l=LED, v, name=Ud] (4,0) -- (0,0);

  \varrmore{U0}{$U_\mathrm{0}$};
  \varrmore{Ud}{$U_\mathrm{D}$};
  \iarrmore{Id}{$I_\mathrm{D}$};
\end{tikzpicture}
     \label{fig:FigLEDVorwiderstand}
  \end{figure}

    Welchen der folgenden Widerstandswerte würden Sie für $R$ verwenden, um die LED möglichst nahe an ihrem optimalen Betriebspunkt (d.h. mit optimaler Helligkeit) zu betreiben?\\
    $15\,\Omega$, $33\,\Omega$, $68\,\Omega$, $150\,\Omega$, $180\,\Omega$

  

%   \speech{ **Aufgabe 10**  
% Wie bei allen **Dioden** ist es wichtig, den **Strom durch das Bauelement zu begrenzen**.  
% Dazu werden **Vorwiderstände** eingesetzt, wie in der unten gezeigten Schaltung.  

% - Die verwendete **LED** hat eine **Durchlassspannung von \( U_D = 2.2V \)**.  
% - Die **optimale Helligkeit** wird bei einem **Strom von \( I_D = 30mA \)** erreicht.  
% - Die **Eingangsspannung** beträgt **\( U_0 = 5V \)**.  

% 

% **Aufgabenstellung:**  
% - Welchen der folgenden **Widerstandswerte** würden Sie für **\( R_V \)** wählen, um die LED möglichst **nahe an ihrem optimalen Betriebspunkt** zu betreiben?  
%   - **15Ω, 33Ω, 68Ω, 150Ω, 180Ω**

% 

%  **Beschreibung der Abbildung**  
% Die gezeigte **elektrische Schaltung** stellt eine **einfache LED-Vorwiderstandsschaltung** dar. Sie besteht aus:

% 1. **Spannungsquelle \( U_0 \)**  
%    - Symbolisiert durch eine **Gleichspannungsquelle mit 5V**.  
%    - Stellt die benötigte **Betriebsspannung für die LED** bereit.  

% 2. **Vorwiderstand \( R_V \)**  
%    - In **Reihe mit der LED geschaltet**.  
%    - Begrenzt den **Stromfluss durch die LED**, um Schäden zu vermeiden.  

% 3. **LED mit Durchlassspannung \( U_D \)**  
%    - Platziert **nach dem Vorwiderstand**.  
%    - Leuchtet, sobald die **Betriebsspannung \( U_0 \)** groß genug ist.  
%    - Hat eine **Durchlassspannung von 2.2V**.  
%    - Der **gewählte Widerstandswert beeinflusst die Stromstärke und damit die Helligkeit**.  

% 

% ### **Funktionsweise der Schaltung**  
% - Die **Gesamtspannung von 5V** teilt sich auf **Vorwiderstand \( R_V \)** und **LED \( U_D \)** auf.  
% - Der **Widerstand \( R_V \)** muss so gewählt werden, dass der **gewünschte Strom von 30mA** fließt.  
% - Durch **Ohm’sches Gesetz** lässt sich die **notwendige Widerstandswert berechnen**.  
% - Zu hohe Werte von **\( R_V \)** würden **die LED dunkler machen**, zu niedrige Werte **könnten die LED beschädigen**.  

% }

}


\Loesung{
Berechnung der Spannung am Vorwiderstand:
\begin{equation*}
    U_\mathrm{V} = U_\mathrm{0} - U_\mathrm{D} = 5\,\mathrm{V} - 2,2\,\mathrm{V} = 2,8\,\mathrm{V}
\end{equation*}

Berechnung des optimalen Vorwiderstands für \( I_\mathrm{D} = \mathrm{30\,mA} \):
\begin{equation*}
    R_\mathrm{V} = \frac{U_\mathrm{V}}{I_\mathrm{D}} = \frac{2,8\,\mathrm{V}}{30\,\mathrm{mA}} = 93,3\,\Omega
\end{equation*}

Berechnung des Stroms bei Auswahl eines konkreten Widerstandswertes:
\begin{equation*}
    I_\mathrm{D} = \frac{U_\mathrm{V}}{R} = \frac{2,8\,\mathrm{V}}{150\,\Omega} = 18,7\,\mathrm{mA}
\end{equation*}

Berechnung bei zwei Widerständen in Parallelschaltung mit zusätzlichem Serienwiderstand:
\begin{gather*}
    R_\mathrm{ges} = \frac{150\,\Omega \cdot 180\,\Omega}{150\,\Omega + 180\,\Omega} + 15\,\Omega = 96,8\,\Omega \\
    I_\mathrm{D} = \frac{2,8\,\mathrm{V}}{96,8\,\Omega} =28,9\,\mathrm{mA}
\end{gather*}

Alternative Reihenschaltung:
\begin{equation*}
    R_\mathrm{ges} = 4 \cdot 15\,\Omega + 33\,\Omega = 93\,\Omega
\end{equation*}

\textbf{Fazit:}  
Ein kombinierter Widerstand aus $4 \times 15\,\Omega + 33\,\Omega = 93\,\Omega$ oder eine Parallelschaltung mit $150\,\Omega \,\|\, 180\,\Omega + 15\,\Omega = 96,8\,\Omega$ führt zu einem Strom nahe dem optimalen Wert von  $30\,\mathrm{mA}$. \\
\textbf{Empfohlene Wahl:} Kombination mit Gesamtwiderstand nahe $93\,\Omega$.

  Siehe: Abschnitt \ref{sec:SpezielleDioden}
%   \speech{
% Lösung zu Aufgabe 10:

% In dieser Aufgabe geht es um eine Schaltung mit einer Leuchtdiode (LED), die durch einen Vorwiderstand geschützt wird. Ziel ist es, den optimalen Widerstandswert für die LED zu bestimmen, sodass sie mit der idealen Helligkeit betrieben wird.

% **Gegebene Werte:**
% - Die LED besitzt eine Durchlassspannung von 2,2 Volt.
% - Der Strom durch die LED sollte 30 Milliampere betragen.
% - Die Eingangsspannung beträgt 5 Volt.

% **Formel zur Berechnung der Spannung über dem Vorwiderstand \(R_V\):**
% Die Spannung über dem Vorwiderstand berechnet sich durch die Differenz zwischen der Eingangsspannung und der Durchlassspannung der LED:
% \[
% U_V = U_0 - U_D = 5V - 2,2V = 2,8V
% \]

% **Formel zur Berechnung des Vorwiderstandes \(R_V\):**
% Der Widerstand kann mit dem Ohmschen Gesetz berechnet werden:
% \[
% R_V = \frac{U_V}{I_D} = \frac{2,8V}{30mA} = 93,3 \Omega
% \]
% Das bedeutet, dass ein Widerstand von etwa 93 Ohm nötig ist, um den gewünschten Strom von 30 Milliampere durch die LED fließen zu lassen.

% **Berechnung für verschiedene Widerstände:**
% Nun wird überprüft, welche der vorgegebenen Widerstandswerte der optimalen Helligkeit am nächsten kommen.

% 1. **Für einen Widerstand von 150 Ohm:**
%    \[
%    I_D = \frac{2,8V}{150 \Omega} = 18,7mA
%    \]
%    Dies ist ein geringerer Strom als ideal, wodurch die LED weniger hell leuchten würde.

% 2. **Für einen Widerstand von 180 Ohm:**
%    \[
%    I_D = \frac{2,8V}{180 \Omega} = 15,6mA
%    \]
%    Auch hier würde die LED dunkler als optimal leuchten.

% 3. **Für eine Parallelschaltung von 150 Ohm und 180 Ohm:**
%    Der Gesamtwiderstand berechnet sich als:
%    \[
%    R_{gesamt} = \frac{150 \cdot 180}{150 + 180} = 96,8 \Omega
%    \]
%    Damit ergibt sich ein Strom durch die LED von:
%    \[
%    I_D = \frac{2,8V}{96,8 \Omega} = 28,9mA
%    \]
%    Dies ist sehr nahe am optimalen Wert von 30 Milliampere.

% **Fazit:**
% Der Widerstand von **93,3 Ohm** wäre optimal, allerdings ist dieser Wert nicht direkt in der Auswahl enthalten. Die beste praktische Wahl ist die Parallelschaltung von **150 Ohm und 180 Ohm**, da sie einen Widerstand von 96,8 Ohm ergibt und somit einen Stromfluss von 28,9 Milliampere ermöglicht, was der gewünschten LED-Helligkeit am nächsten kommt.
% }

}



