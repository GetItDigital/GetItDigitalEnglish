% Schaltsymbol eines Eintor (Zweipol)
\begin{tikzpicture}%
%   invisible boundbox
    \draw[draw=none](-2,-1.25) rectangle (+2,+1.25);% gleiche wie bei circ_zweitor.tex
%   Eintor (Zweipol)
    \node[draw=none,align=center] {Eintor\\Zweipol};
    \draw(0,0) node[dipchip, hide numbers, num pins=4, no topmark, external pins width=0](A){};%            Mittelpunkt des Zweipols
    \draw(0,0) node[dipchip, hide numbers, num pins=2, no topmark, external pins width=0, opacity=0](B){};% Ref. Pins mittig
    \draw(B.bpin 1) to[short, -o, name=i, i_<=$I$, !i] ++(-0.5,0) coordinate(In);%
    \draw(B.bpin 2) to[short,-o, name=i2, i>=$I$, !i] ++(0.5,0) coordinate(Out);%
    \draw(-1.75,+0.75) to [open, name=u, v^=$U$, !v] (+1.75,+0.75);%

    %draw arrows
    \iarronly{i};
    \iarronly{i2};
    \varronly{u};
    
\end{tikzpicture}%