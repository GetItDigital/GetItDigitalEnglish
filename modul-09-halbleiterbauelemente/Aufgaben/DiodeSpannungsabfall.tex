\subsection{Berechnung des Spannungsabfalls einer Diode\label{Aufg3}}
% Alt: Aufgabe 3
\Aufgabe{
  Gegeben ist eine Reihenschaltung aus einer Gleichspannungsquelle, einem Widerstand und einer Siliziumdiode mit einer Schleusenspannung von $U_\mathrm{S}=0,7\,\mathrm{V}$. 
  Welche Spannung fällt näherungsweise über dem Widerstand $R$ ab, wenn die Versorgungsspannung $U_0=5\,\mathrm{V}$ beträgt?
  \begin {figure} [H]
   \centering
   \begin{tikzpicture}
  % Batterie links
  \draw (0,2) to[V, name=U1] (0,0);
  \varrmore{U1}{$U_{0}$}; 

  % Widerstand
  \draw (0,2) to[R=$R$] (4,2);
  \draw (0,0) -- (4,0);

  % Diode rechts
  \draw (4,2) to[D=$D$, v, name=D1] (4,0);
  \varrmore{D1}{$U_\mathrm{D}$};

\end{tikzpicture}
   \label{fig:FigDiodeSpannungsabfall}
\end {figure}

% \speech{ **Aufgabe 3**  
% Gegeben ist eine **Reihenschaltung** aus einer **Gleichspannungsquelle**, einem **Widerstand** und einer **Siliziumdiode**.  
% Beantworten Sie die folgende Frage:  
% - Welche Spannung fällt **näherungsweise** über dem Widerstand \( R \) ab, wenn die **Versorgungsspannung \( U_0 = 5V \)** beträgt?  

%  **Beschreibung der Abbildung**  
% Die Schaltung besteht aus drei Hauptkomponenten:  

% 1. **Gleichspannungsquelle (\( U_0 \))**  
%    - Symbolisiert durch einen Kreis mit einem Richtungspfeil.  
%    - Die Spannung \( U_0 \) beträgt **5V** und versorgt die Schaltung.  

% 2. **Serienwiderstand (\( R \))**  
%    - Der Widerstand ist im oberen Schaltzweig platziert.  
%    - Seine Funktion ist die **Strombegrenzung** in der Schaltung.  

% 3. **Siliziumdiode (Si)**  
%    - In **Durchlassrichtung** geschaltet, mit der **Kathode nach Masse**.  
%    - Sie hat eine typische **Schwellenspannung von ca. 0,7V**, die bei Leitung überschritten werden muss.  

%  **Erwartetes Verhalten der Schaltung**  
% - Die Diode **leitet**, sobald \( U_0 > 0,7V \) beträgt.  
% - Die Spannung über dem Widerstand kann **näherungsweise berechnet werden** als:
%   \[
%   U_R = U_0 - U_D
%   \]
%   wobei \( U_D \approx 0,7V \) die **Durchlassspannung der Siliziumdiode** ist.  
% - Für \( U_0 = 5V \) ergibt sich:
%   \[
%   U_R \approx 5V - 0,7V = 4,3V
%   \]
%   - Diese Spannung fällt über den Widerstand \( R \) ab.  
%   - Der Strom durch die Schaltung kann mit dem **Ohm’schen Gesetz** bestimmt werden:
%     \[
%     I = \frac{U_R}{R}
%     \]
%   - Der Widerstand begrenzt den Strom durch die Diode, um eine Überlastung zu verhindern.  

%  **Zusammenfassung**  
% Die Abbildung zeigt eine **einfache Reihenschaltung** mit einer **Spannungsquelle, einem Widerstand und einer Diode**. Die Aufgabe erfordert die **Berechnung der Spannung über dem Widerstand**, wenn die **Diode in Durchlassrichtung betrieben wird**.}

 }


\Loesung{
  Da es sich um eine Siliziumdiode handelt, beträgt die Schleusenspannung und der daraus resultierende Spannungsabfall
  am Bauteil ungefähr $0,6 \text{ bis } 0,7\,\mathrm{V}$.

  In der Reihenschaltung ergibt sich der folgende Zusammenhang:
    \begin{equation*}
        U_0 = U_\mathrm{R} + U_\mathrm{D}
    \end{equation*}
  \begin{align*}
    U_\mathrm{R} &= U_0 - U_\mathrm{D}\\
    &= 5\,\mathrm{V} - 0,7\,\mathrm{V} \\
    &= 4,3\,\mathrm{V}
  \end{align*}
  Siehe: Abschnitt \ref{sec:ElektrischesVerhalten}
%   \speech{Es handelt sich um eine Siliziumdiode. Die Schleusenspannung der Diode beträgt ungefähr 0,6 bis 0,7 Volt. Dadurch ergibt sich ein entsprechender Spannungsabfall am Bauteil.  

% Da die Diode in einer Reihenschaltung mit einem Widerstand verbunden ist, gilt die Spannungsregel:  

% Die Gesamtspannung U null ist gleich der Summe aus der Spannung am Widerstand U R und der Spannung an der Diode U D.  

% Das bedeutet:  

% U R ist gleich U null minus U D.  

% Setzt man die Werte ein, ergibt sich:  

% 5 Volt minus 0,7 Volt ergibt 4,3 Volt.  

% Die Spannung am Widerstand beträgt also 4,3 Volt.}

}