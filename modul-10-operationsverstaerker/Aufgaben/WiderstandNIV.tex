\subsection{Widerstandsberechnung für nicht-invertierenden Verstärker\label{Aufg2}}
% Alt: Aufgabe 2
\Aufgabe{
	Gegeben ist die abgebildete Schaltung mit einem nichtinvertierenden 
    Verstärker. Dabei ist der Widerstand $R_\mathrm{2} = 1\,\mathrm{k \Omega}$. Was für Widerstandswerte 
    muss $R_\mathrm{1}$ aufweisen, damit die Verstärkung 10, 50 und 100 beträgt?

    \begin {figure} [H]
   	\centering
   	\begin{tikzpicture}
\ctikzset{tripoles/en amp/input height=-0.45}
            \draw (0,0)node[en amp, en amp text A](E){};
            \draw (E.out) to[short,-*] (2,0) to[short, -o] (4,0);
            \draw (E.-) -- (-1.5,-0.5)  --(-1.5,-1.5) to[short, -*] (2,-1.5);
            \draw (E.+) to[short, -o] (-4,0.5);  
            \draw (2,0) to[R, R=$R_\mathrm{2}$] (2,-1.5);
            \draw (2,-1.5) to[R, R=$R_\mathrm{1}$] (2,-3);
            \draw (4,-3) to[short, o-*] (2,-3) to[short, -*] (-1.5,-3)node[ground]{} ;
            \draw (-1.5,-3) to[short, -o] (-4,-3);
            \draw (4,0) to[open, v, name=ua] (4,-3);
            \draw (-4,0.5) to[open,v,name=ue](-4,-3);
            
            \varrmore{ue}{$U_\mathrm{E}$};
            \varrmore{ua}{$U_\mathrm{A}$};
\end{tikzpicture}

   	\label{fig:FigWiderstandNIV}
    \end {figure}
%	\speech{
%		Aufgabe 2: Nichtinvertierender Verstärker  
%		
%		Gegeben ist eine Schaltung mit einem nichtinvertierenden Operationsverstärker.  
%		Der Widerstand R2 hat einen Wert von 1 Kiloohm.  
%		Gesucht sind die passenden Widerstandswerte für R1, sodass die Verstärkung jeweils 10, 50 und 100 beträgt.  
%		
%		Die Schaltung besteht aus einem Operationsverstärker mit einer Rückkopplung über die Widerstände R1 und R2.  
%		Die Eingangsspannung U E wird an den nichtinvertierenden Eingang des Operationsverstärkers angelegt.  
%		Der invertierende Eingang ist über R1 mit Masse verbunden und über R2 mit dem Ausgang verbunden.  
%		
%		Die Verstärkung V eines nichtinvertierenden Verstärkers wird mit folgender Formel berechnet:  
%		
%		V = 1 + (R2 / R1)  
%		
%		Um die gesuchten Widerstandswerte zu bestimmen, muss diese Gleichung nach R1 umgestellt werden.  
%		}
		

}

\Loesung{
	Gemäß Formel nichtinvertierender Verstärker: 
	\begin{eqa}
		V = 1 + \frac{R_\mathrm{1}}{R_\mathrm{2}}
	\end{eqa}
	Formel nach $R_1$ umstellen:
	\begin{eqa}
		R_1 = R_2 \cdot (V - 1)
	\end{eqa}
	\begin{itemize}
		\item $V = 10$  
		\begin{align*}
			R_1 = \mathrm{1\,k\Omega \cdot (10 - 1) = 9\,k\Omega}
		\end{align*}
		\item $V = 50$
		\begin{align*}
			R_1 = \mathrm{1\,k\Omega \cdot (50 - 1) = 49\,k\Omega}
		\end{align*}
		\item $V = 100$
		\begin{align*}
			R_1 = \mathrm{1\,k\Omega \cdot (100 - 1) = 99\,k\Omega}
		\end{align*} 
	\end{itemize}

%	\speech{
%		Berechnung der Widerstände für einen nichtinvertierenden Verstärker  
%		
%		Die Verstärkung eines nichtinvertierenden Operationsverstärkers wird durch die Formel bestimmt:  
%		
%		V = 1 + R1 durch R2  
%		
%		Umstellen der Formel nach R1 ergibt:  
%		
%		R1 = R2 mal (V - 1)  
%		
%		Die Aufgabe besteht darin, den Widerstand R1 für verschiedene Verstärkungswerte zu berechnen,  
%		wenn der Widerstand R2 gleich 1 Kiloohm ist.  
%		
%		Für eine Verstärkung von 10:  
%		R1 = 1 Kiloohm mal (10 - 1) = 9 Kiloohm  
%		
%		Für eine Verstärkung von 50:  
%		R1 = 1 Kiloohm mal (50 - 1) = 49 Kiloohm  
%		
%		Für eine Verstärkung von 100:  
%		R1 = 1 Kiloohm mal (100 - 1) = 99 Kiloohm  
%		
%		Die berechneten Widerstandswerte zeigen, dass mit zunehmender Verstärkung  
%		der Wert von R1 deutlich größer wird.  
%		}
Siehe: Abschnitt \ref{fig: Verschaltung Verstaerker} und Tabelle \ref{tab:Grundschaltungen}
}