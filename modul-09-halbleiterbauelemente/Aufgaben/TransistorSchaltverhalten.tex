\subsection{Schaltverhalten eines Transistors\label{Aufg14}}
% Alt: Aufgabe 14
\Aufgabe{
    Gegeben ist die abgebildete Transistorschaltung mit den folgenden Parametern: 
    $U_0=\mathrm{5\,V}$ und $U_L=\mathrm{15\,V}$.
  Damit der gezeigte Siliziumtransistor schaltet, benötigt er einen Basisstrom von \( I_\mathrm{B}=\mathrm{50\,\mu A} \), um in den leitenden Zustand zu gelangen.
  
  \begin{itemize}
    \item[a)] Wie groß muss der Vorwiderstand \( R_\mathrm{V} \) gewählt werden, damit der Transistor durchschaltet?
    \item[b)] Der Transistor hat einen Stromverstärkungsfaktor \( \beta = 180 \) und einen Kollektorwiderstand \( R_\mathrm{C} = \mathrm{200\,\Omega} \). 
    Wie groß sind der Kollektorstrom \( I_\mathrm{C} \) und der Spannungsabfall zwischen Kollektor und Emitter \( U_\mathrm{CE} \)?
  \end{itemize}  
  \begin {figure} [H]
     \centering
     \begin{tikzpicture}
    \draw (4,2) node[npn, tr circle] (Q1) {};

    \draw (Q1.C) -- (4,4) 
    to [R,l=$R_\mathrm{C}$] (7,4)
    to [V, name=Ul, v] (7,0);

    \draw (Q1.E) to [short, -*] (4,0) 
    -- (7,0);

   \draw (Q1.B) to [R,l_=$R_\mathrm{V}$, i<, name=Ib] (0,2)
   to [V, name=U0, v] (0,0)
   to [short] (4,0);

    \varrmore{U0}{$U_\mathrm{0}$};
    \varrmore{Ul}{$U_\mathrm{L}$};
    \iarrmore{Ib}{$I_\mathrm{B}$};
\end{tikzpicture}
     \label{fig:FigTransistorSchaltverhalten}
  \end {figure}
  % \speech{Aufgabe 14  

  % Damit der gezeichnete Siliziumtransistor schaltet, braucht er einen Basisstrom von \( 50 \mu A \) um durchzuschalten.  
  
  % a) Wie groß muss der Widerstand \( R_V \) gewählt werden, damit der Transistor durchschaltet?  
  
  % b) Der Transistor hat eine Stromverstärkungsfaktor \( \beta = 180 \) und einen Kollektorwiderstand \( R_C \) von \( 200 \Omega \).  
  %    Wie groß ist der Kollektorstrom \( I_C \) und der Spannungsabfall zwischen Kollektor-Emitter \( U_{CE} \)?  
  
  % Beschreibung der Abbildung  
  
  % Die Abbildung zeigt eine Transistorschaltung mit einer **Spannungsquelle von \( 5V \)** auf der linken Seite, die über einen **Widerstand \( R_V \)** mit der Basis eines **npn-Transistors** verbunden ist.  
  
  % Die Basis-Emitter-Spannung \( U_{BE} \) ist mit einem blauen Pfeil gekennzeichnet, und der **Basisstrom \( I_B \)** ist mit einem roten Pfeil dargestellt.  
  
  % Auf der rechten Seite der Schaltung befindet sich eine **weitere Spannungsquelle mit \( 15V \)**, die über einen **Kollektorwiderstand \( R_C \)** mit dem **Kollektor des Transistors** verbunden ist. Der **Kollektorstrom \( I_C \)** fließt durch den Widerstand \( R_C \) und ist mit einem roten Pfeil markiert.  
  
  % Die Aufgabe besteht darin, den notwendigen Basiswiderstand \( R_V \) zu berechnen und aus der gegebenen Stromverstärkung \( \beta \) den **Kollektorstrom \( I_C \)** sowie den Spannungsabfall \( U_{CE} \) zu bestimmen.  
  % }  
  

  }


\Loesung{
\begin{itemize}
    \item[a)]
    Berechnung des Vorwiderstands für den Basisstrom:
    \begin{equation}
        R_\mathrm{V} = \frac{U}{I}
    \end{equation}
    \begin{align*}
        R_\mathrm{V} &= \frac{\mathrm{5\,V} - \mathrm{0{,}6\,V}}{\mathrm{50\,\mu A}} = \frac{\mathrm{4{,}4\,V}}{\mathrm{50\,\mu A}} = \mathrm{88\,k\Omega}
    \end{align*}

    \item[b)]
    Berechnung des Kollektorstroms mit Stromverstärkungsfaktor:
    \begin{equation}
        \beta = \frac{I_\mathrm{C}}{I_\mathrm{B}}
    \end{equation}
    \begin{align*}
        I_\mathrm{C} &= \beta \cdot I_\mathrm{B} = 180 \cdot \mathrm{50\,\mu A} = \mathrm{9\,mA}
    \end{align*}

    Berechnung des Spannungsabfalls am Kollektorwiderstand:
    \begin{align*}
        U_\mathrm{RC} &= I_\mathrm{C} \cdot R_\mathrm{C} = \mathrm{9\,mA} \cdot \mathrm{200\,\Omega} = \mathrm{1{,}8\,V}
    \end{align*}

    Berechnung der Spannung zwischen Kollektor und Emitter:
    \begin{align*}
        U_\mathrm{CE} &= \mathrm{15\,V} - \mathrm{1{,}8\,V} = \mathrm{13{,}2\,V}
    \end{align*}
\end{itemize}

    Siehe: Abschnitt \ref{sec:Bipolartransistor} und \ref{fig:StroemeUndSpannungenBeimBipolartransistor}
%     \speech{
% **Aufgabe 14 – Berechnung des Basiswiderstands und der Verstärkung eines npn-Transistors**

% Gegeben ist eine Schaltung mit einem **npn-Transistor**, der durch einen **Basiswiderstand** gesteuert wird. Die Aufgabe besteht darin, den **Basiswiderstand \( R_V \)** sowie die **Stromverstärkung \( \beta \)** und die zugehörigen Spannungen zu berechnen.

% ---

% ### Teil a) Berechnung des Widerstands \( R_V \)
% Die Formel für den Widerstand lautet:

% \[
% R_V = \frac{U}{I}
% \]

% Gegeben sind:
% - Eingangsspannung \( U = 5V \)
% - Basis-Emitter-Spannung \( U_{BE} = 0,6V \)
% - Basisstrom \( I_B = 50 \mu A \)

% Einsetzen der Werte:

% \[
% R_V = \frac{5V - 0,6V}{0,05 mA}
% \]

% \[
% R_V = \frac{4,4V}{0,05 mA} = 88 kΩ
% \]

% **Ergebnis**: Der Basiswiderstand \( R_V \) beträgt **88 kΩ**.

% ---

% ### Teil b) Berechnung der Stromverstärkung \( \beta \) und der zugehörigen Spannungen
% Die Formel für den Verstärkungsfaktor \( \beta \) lautet:

% \[
% \beta = \frac{I_C}{I_B}
% \]

% Gegeben:
% - Basisstrom \( I_B = 50 \mu A \)
% - Stromverstärkungsfaktor \( \beta = 180 \)

% Berechnung des Kollektorstroms \( I_C \):

% \[
% I_C = \beta \cdot I_B
% \]

% \[
% I_C = 180 \cdot 50 \mu A = 9 mA
% \]

% Nun wird die Spannung am **Kollektorwiderstand \( R_C \)** berechnet:

% \[
% U_{RC} = I_C \cdot R_C
% \]

% Gegeben:
% - \( R_C = 200Ω \)
% - \( I_C = 9 mA \)

% \[
% U_{RC} = 9 mA \cdot 200Ω = 1,8V
% \]

% Die **Kollektor-Emitter-Spannung \( U_{CE} \)** ergibt sich aus:

% \[
% U_{CE} = 15V - U_{RC}
% \]

% \[
% U_{CE} = 15V - 1,8V = 13,2V
% \]

% ---

% **Ergebnisse**:
% - Der Kollektorstrom \( I_C \) beträgt **9 mA**.
% - Die Spannung über dem Kollektorwiderstand \( U_{RC} \) beträgt **1,8V**.
% - Die Kollektor-Emitter-Spannung \( U_{CE} \) beträgt **13,2V**.
% }

}



