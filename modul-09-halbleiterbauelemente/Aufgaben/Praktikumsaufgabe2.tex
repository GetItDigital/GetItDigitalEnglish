\subsection{Praktikumsaufgabe 2\label{Aufg30}}
\Aufgabe{
    \begin{figure}[H]
    \centering
    \subsection{Praktikumsaufgabe 2\label{Aufg30}}
\Aufgabe{
    \begin{figure}[H]
    \centering
    \subsection{Praktikumsaufgabe 2\label{Aufg30}}
\Aufgabe{
    \begin{figure}[H]
    \centering
    \subsection{Praktikumsaufgabe 2\label{Aufg30}}
\Aufgabe{
    \begin{figure}[H]
    \centering
    \input{Bilder/Aufgaben/Praktikumsaufgabe2.tex}
    \label{fig:Praktikumsaufgabe2}
    \end{figure}
    Berechnen Sie den Gleich- und Wechselspannungsanteil an einem Widerstand $\mathrm{R_1}$.
    
    Gegeben:
    \begin{itemize}
        \item Eine gemischte Spannung mit Gleichspannungsanteil $\mathrm{U_{DC}} = 7{,}5\,\text{V}$
        \item Wechselspannungsanteil $\mathrm{U_{AC}} = 2{,}3\,\text{V}$
        \item Ein Kondensator mit $\mathrm{C} = 10\,\mu\text{F}$ wird vorgeschaltet.
    \end{itemize}
    
    \begin{itemize}
        \item[a)] Berechnen Sie die Gesamtspannung ohne den Kondensator.
        \item[b)] Beschreiben Sie den Effekt des Kondensators auf den Wechsel- und Gleichspannungsanteil.
    \end{itemize}
}



\Loesung{

}
    \label{fig:Praktikumsaufgabe2}
    \end{figure}
    Berechnen Sie den Gleich- und Wechselspannungsanteil an einem Widerstand $\mathrm{R_1}$.
    
    Gegeben:
    \begin{itemize}
        \item Eine gemischte Spannung mit Gleichspannungsanteil $\mathrm{U_{DC}} = 7{,}5\,\text{V}$
        \item Wechselspannungsanteil $\mathrm{U_{AC}} = 2{,}3\,\text{V}$
        \item Ein Kondensator mit $\mathrm{C} = 10\,\mu\text{F}$ wird vorgeschaltet.
    \end{itemize}
    
    \begin{itemize}
        \item[a)] Berechnen Sie die Gesamtspannung ohne den Kondensator.
        \item[b)] Beschreiben Sie den Effekt des Kondensators auf den Wechsel- und Gleichspannungsanteil.
    \end{itemize}
}



\Loesung{

}
    \label{fig:Praktikumsaufgabe2}
    \end{figure}
    Berechnen Sie den Gleich- und Wechselspannungsanteil an einem Widerstand $\mathrm{R_1}$.
    
    Gegeben:
    \begin{itemize}
        \item Eine gemischte Spannung mit Gleichspannungsanteil $\mathrm{U_{DC}} = 7{,}5\,\text{V}$
        \item Wechselspannungsanteil $\mathrm{U_{AC}} = 2{,}3\,\text{V}$
        \item Ein Kondensator mit $\mathrm{C} = 10\,\mu\text{F}$ wird vorgeschaltet.
    \end{itemize}
    
    \begin{itemize}
        \item[a)] Berechnen Sie die Gesamtspannung ohne den Kondensator.
        \item[b)] Beschreiben Sie den Effekt des Kondensators auf den Wechsel- und Gleichspannungsanteil.
    \end{itemize}
}



\Loesung{

}
    \label{fig:Praktikumsaufgabe2}
    \end{figure}
    Berechnen Sie den Gleich- und Wechselspannungsanteil an einem Widerstand $\mathrm{R_1}$.
    
    Gegeben:
    \begin{itemize}
        \item Eine gemischte Spannung mit Gleichspannungsanteil $\mathrm{U_{DC}} = 7{,}5\,\text{V}$
        \item Wechselspannungsanteil $\mathrm{U_{AC}} = 2{,}3\,\text{V}$
        \item Ein Kondensator mit $\mathrm{C} = 10\,\mu\text{F}$ wird vorgeschaltet.
    \end{itemize}
    
    \begin{itemize}
        \item[a)] Berechnen Sie die Gesamtspannung ohne den Kondensator.
        \item[b)] Beschreiben Sie den Effekt des Kondensators auf den Wechsel- und Gleichspannungsanteil.
    \end{itemize}
}



\Loesung{

}