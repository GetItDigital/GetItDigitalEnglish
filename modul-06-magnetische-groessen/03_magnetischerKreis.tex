\newvideofile{MagnetischerKreis}{Magnetischer Kreis}
\s{\section{Magnetischer Fluss und Flussdichte\label{Flussdichte}}}
\v{
	%Videoframe 
	\begin{frame}{Lernziele}
		\begin{Lernziele}{Magnetischer Kreis}
			Du kannst ...
			\begin{itemize}
				\item ... Grundgrößen wie magnetische Feldstärke, magnetischer Flussdichte, magnetischen Fluss und die Permeabilität definieren.
				\item ... den Zusammenhang zwischen diesen Größen erklären.
				\item ... das Verhalten ferromagnetischer Werkstoffe anhand der Hysterese-Kurve beschreiben.
				\item ... Rechenaufgaben im magnetischen Kreis lösen.
			\end{itemize}
		\end{Lernziele}
	\speech{LernzieleMagnetischerKreis}{1}{In diesem Video erhaltet ihr einen Einblick in die Grundgrößen des magnetischen Kreises. Dazu zählen die magnetische Flussdichte, der magnetische Fluss und die Permeabilität, die auch als magnetischer Widerstand bezeichnet wird. Darüber hinaus könnt ihr den Zusammenhang zwischen den Größen beschreiben und Rechnenbeispiele zum magnetischen Kreis lösen. Des Weiteren machen wir einen Exkurs in die Materialkunde. Hier lernt ihr die ferromagnetischen Eigenschaften wie Koerzitivfeldstärke, Remanenzflussdichte und Sättigung anhand der Hysterese-Kurve kennen.}
	\speech{LernzieleMagnetischerKreis}{2}{Test, test, test, test,  und ein weiterer Test}
\end{frame}
	}
\begin{frame} \ftx{Magnetischer Fluss und Flussdichte}
	\index{Magnetische Flussdichte}\index{Flussdichte}\index{Magnetischer Fluss}
	\s{
		Der magnetische Fluss ist analog zum elektrischen Kreis mit dem Strom vergleichbar. Entgegen der Terminologie findet jedoch kein Fluss von magnetischen Teilchen statt, sondern er ist sinnbildlich als \glqq die Menge an Magnetfeld\grqq\,zu verstehen und wirkt in Folge der magnetischen Spannung. Das Formelzeichen des magnetischen Flusses ist $\varPhi$, die Einheit ist Weber (Wb). Ein Weber (Wb) ist gleichbedeutend mit einer Volt-Sekunde (Vs).
		
		Die Kraftwirkung eines Magneten ist abgesehen vom magnetischen Fluss auch von der durchfluteten Fläche abhängig. Je dichter die Feldlinien konzentriert sind, desto größer ist die magnetische Wirkung. Das wird durch die magnetische Flussdichte $B$ beschrieben, die im einfachsten Fall (nicht gekrümmte Fläche, homogene Flussdichte) durch den Quotienten aus dem magnetischen Fluss $\varPhi$ und der Fläche $A$ definiert ist. Die Einheit der Flussdichte ist Tesla (T). Die Richung der Flussdichte $\vec{B}$ ist senkrecht zur Fläche, was durch den Normalenvektor zur Fläche $\vec{A}$ ausgedrückt wird.
		
		\begin{eq}
			\vec{B} = \frac{\varPhi}{\vec{A}}\qquad [\mathrm{T}]
		\end{eq}
		\nomenclature[ET]{T}{Tesla - magnetische Flussdichte \nomunit{$\mathrm{T}=\frac{\mathrm{V}\cdot\mathrm{s}}{\mathrm{m}^2}=\frac{\mathrm{kg}}{\mathrm{s}^2\cdot\mathrm{A}}$}}
		
		Im allgemeinen Fall (ohne die oben genannten Einschränkungen) gilt:
		
		\begin{eq}
			\varPhi = \iint\limits_A \vec{B}\cdot\d \vec{A} \label{magnFlussFormel}
		\end{eq}
		\begin{Merksatz}{Magnetische Flussdichte}
			Die magnetische Flussdichte $\vec{B}$ beschreibt die Konzentration des magnetischen Flusses $\varPhi$ senkrecht auf einer Fläche $A$.
		\end{Merksatz}
		
		Sowohl die magnetische Feldstärke als auch die magnetische Flussdichte sind vektorielle Größen. Sie lassen sich daher grafisch durch Feldlinien zeichnen. Der magnetische Fluss $\varPhi$ ist dagegen eine skalare Größe.
		
		Die magnetische Flussdichte $\vec{B}$ und die magnetische Feldstärke $\vec{H}$ sind über die Permeabilität\index{Permeabilität} $\mu$ verbunden. Die Permeabilität besteht aus dem Produkt einer materialunabhängigen Kenngröße, der magnetischen Feldkonstanten $\mu_0=1,256\,637\,062\cdot10^{-6}\,\frac{\mathrm{Vs}}{\mathrm{Am}}$, und einer materialspezifischen Permeabilität $\mu_{\mathrm{r}}$. Die magnetische Feldkonstante beschreibt die Permeabilität im Vakuum und war bis zur Neuordnung der SI-Einheiten im Jahr 2019 mit dem Wert $\mu_0 = 4\pi\cdot10^{-7}\,\frac{\mathrm{Vs}}{\mathrm{Am}}$ genau definiert. Jetzt ist sie mit einer Messunsicherheit behaftet.

	}
	\s{
		\begin{table}[H]
			\centering
			\caption{{\bf Materialabhängige Permeabilität.} Exemplarische Übersicht zur Permeabilitätszahl $\mu_{\mathrm{r}}$ unterschiedlicher Materialien:}
			\begin{tabular}{lc}
				\bf{Material}       & \bf{Permeabilitätszahl $\mu_{\mathrm{r}}$} \\
				\midrule
				Wasser    & $1 - 9,1 \cdot 10^{-6}$ \\
				Kupfer    & $1 - 6,4 \cdot 10^{-6}$ \\
				Luft      & $1 + 4 \cdot 10^{-7}$ \\
				Aluminium & $1 - 2,2 \cdot 10^{-5}$ \\
				Eisen     & $300$ bis $140000$ \\
			\end{tabular}%
		\end{table}
		\begin{eq}
			\vec{B} = \mu_0\cdot\mu_{\mathrm{r}}\cdot \vec{H}\label{GlFlussdichte}
		\end{eq}
	}
	
	\b{
		\begin{columns}
			\begin{column}{0.7\textwidth}
				\begin{minipage}{0.8\textwidth}
					% Nicht linearer Zusammenhang zwischen B und H -> Neukurve und Sättigung
					\onslide<1->{
						\fu{
							\begin{tikzpicture}[domain=-6:6, scale=0.73]
								\draw[->] (-6.1, 0) -- (6.2, 0) node[right] {$H$};
								\draw[->] (0, -5.1) -- (0, 5.2) node[above] {$B$};
								\draw (-0.2, 0) node[below] {0};
								
								\onslide<3->{
									\draw[color=blue, smooth, domain=0:6] plot[id=hysterese1] function{(((x-6.1)**3)/50)+4.485};
									\draw[->, color=blue, thick] (1, 1.82) -- (1.05, 1.91);
									\draw (6, 4.85) node[left] {Sättigung};
									\draw[color=blue] (2.2, -1.5) node[right] {Neukurve};
								}
								% Magnetisierung bleibt bei Reduktion erhalten (Punkt Br)
								\onslide<4->{
									\draw[color=red, smooth] plot[id=hysterese2] function{9/(1+0.4*exp(-1.1*x))-4.5};
									\draw[->, color=red, thick] (-0.55, 0.7) -- (-0.61, 0.55);
									\draw (-6, -4.85) node[right] {Sättigung};
									\draw (-0.2, 1.9) node[left] {$B_{\mathrm{r}}$} -- (0.2, 1.9);
								}
								% Gilt ebenso in andere Richtung -> Hystereseschleife
								\onslide<5->{
									\draw[color=red, smooth] plot[id=hysterese3] function{9/(1+2.5*exp(-1.1*x))-4.5};
									\draw[->, color=red, thick] (0.55, -0.7) -- (0.61, -0.55);
									\draw[color=red] (2.2, -2.3) node[right] {Hystereseschleife};
									\draw (-0.2, -1.9) -- (0.2, -1.9);
									
								}
								%Zum Entmagnetisieren umgekehrte Koeffizienzfeldstärke (Hc)
								\onslide<6->{
									\draw (-0.85, 0.2) -- (-0.85, -0.2);
									\draw (-1.1, 0.2);
									\draw (0.85, 0.2) -- (0.85, -0.2);
									\draw (1.1, -0.2) node[below] {$H_{\mathrm{c}}$};
								}
							\end{tikzpicture}
						}{}
					}
				\end{minipage}
			\end{column}
			
			\begin{column}{0.3\textwidth}
				\only<1-2>{Magnetische Flussdichte: \par  $\vec{B} \quad \left[\mathrm{T}\right]$\\
					Permeabilität: \par 
					$\mu\quad \left[1\right]$
					\begin{eq}
						\vec{B} = \mu_0\cdot\mu_{\mathrm{r}}\cdot \vec{H}
					\end{eq}
					
					\onslide<2->{
						Magnetischer Fluss: \par $\varPhi\quad \left[\mathrm{Wb}\right]$
						\begin{align*}
							\varPhi & = \iint\limits_A \vec{B}\cdot\d \vec{A} \\
							\vec{B} & = \frac{\varPhi}{\vec{A}}
						\end{align*}}
				}
				\only<3->{%
					
					Hartmagnetisch: \\
					$H_{\mathrm{c}} > 10\cdot 10^{3}\,\frac{\mathrm{A}}{\mathrm{m}}$\\
					z. B. Permanentmagnete\\
					\vspace{0.5cm}
					Weichmagnetisch:\\
					$H_{\mathrm{c}} < 500\,\frac{\mathrm{A}}{\mathrm{m}}$\\
					z. B. Aktoren\\
					\vspace{0.5cm}
				}
				\only<4->{
					Remanenzflußdichte $B_{\mathrm{r}}$\\
					\vspace{0.5cm}
				}
				\only<6->{
					Koerzitivfeldstärke $H_{\mathrm{c}}$
				}
				
			\end{column}
		\end{columns}
	}
	\s{
		In einem ferromagnetischen\index{Ferromagnetismus} Material verläuft der Zusammenhang zwischen der magnetischen Feld\-stärke $\vec{H}$ und der magnetischen Flussdichte $\vec{B}$ nicht linear. 
		Die Permeabilität geht mit steigender Magnetisierung in Sättigung, sodass die relative Permeabilität $\mu_{\mathrm{r}}$ von einem materialabhängigen Anfangswert gegen 1 läuft. 
		Wird das magnetische Feld wieder reduziert (oder auf Null gesetzt), bleibt die Magnetisierung in einem gewissen Maße erhalten (Punkt $B_{\mathrm{r}}$ in Abbildung \ref{Hysterese}). 
		Dieser Vorgang wird Remanenz\index{Remanenz} genannt. Die verbleibende magnetische Flussdichte bei einer magnetischen Feldstärke von Null ist die Remanenzflussdichte $B_{\mathrm{r}}$. 
		Um den Stoff wieder komplett zu untmagnetisieren\index{Entmagnetisieren}, wird eine umgekehrte magnetische Feldstärke, die Koerzitivfeldstärke\index{Koerzitivfeldstärke} $H_{\mathrm{c}}$, benötigt.\index{Magnetisierungskennlinie}
		\fu{
			\begin{tikzpicture}[domain=-6:6, scale=0.85]
	\draw[->] (-6.1, 0) -- (6.2, 0) node[right] {$H$};
	\draw[->] (0, -5.1) -- (0, 5.2) node[above] {$B$};
	\draw (-0.2, 0) node[below] {0};
	\draw[color=blue, smooth, domain=0:6] plot[id=hysterese1] function{(((x-6.1)**3)/50)+4.485};
	\draw[->, color=blue, thick] (1, 1.82) -- (1.05, 1.91);
	\draw[color=red, smooth] plot[id=hysterese2] function{9/(1+0.4*exp(-1.1*x))-4.5};
	\draw[->, color=red, thick] (-0.55, 0.7) -- (-0.61, 0.55);
	\draw[color=red, smooth] plot[id=hysterese3] function{9/(1+2.5*exp(-1.1*x))-4.5};
	\draw[->, color=red, thick] (0.55, -0.7) -- (0.61, -0.55);
	\draw[color=blue] (2.5, -1.5) node[right] {Neukurve};
	\draw[color=red] (2.5, -2.1) node[right] {Hystereseschleife};
	\draw (6, 4.85) node[left] {Sättigung};
	\draw (-6, -4.85) node[right] {Sättigung};
	\draw (-0.2, 1.9) node[left] {$B_{\mathrm{r}}$} -- (0.2, 1.9);
	\draw (-0.2, -1.9) -- (0.2, -1.9);
	\draw (-0.85, 0.2) -- (-0.85, -0.2);
	\draw (-1.1, 0.2);
	\draw (0.85, 0.2) -- (0.85, -0.2);
	\draw (1.1, -0.2) node[below] {$H_{\mathrm{c}}$};
\end{tikzpicture}
		}{{\bf Magnetisierungskurve eines ferromagnetischen Materials.} Die magnetische Feldstärke und die magnetische Flussdichte verhalten sich nicht linear zueinander. \label{Hysterese}}
		
		Anhand der Koerzitivfeldstärke werden ferromagnetische Materialien in hart- und weichmagnetische Materialien unterschieden. Hartmagnetische Werkstoffe (z. B. starke Dauermagnete aus Neodym-Eisen-Bor) verfügen dabei über einen Wert für $H_{\mathrm{c}}$ größer als $10\cdot 10^{3}\,\frac{\mathrm{A}}{\mathrm{m}}$, bei weichmagentischen Werkstoffen (z. B. Magnetkerne aus Mangan-Zink-Ferrit) liegt $H_{\mathrm{c}}$ bei kleiner als $500\,\frac{\mathrm{A}}{\mathrm{m}}$. Hartmagnetische Werkstoffe werden hauptsächlich für Permanentmagnete eingesetzt.\\
		
	}
		\speech{TestMagnetischeGroeßen}{1}{In diesem Video erhaltet ihr einen Einblick in die Grundgrößen des magnetischen Kreises. Dazu zählen die magnetische Flussdichte, der magnetische Fluss und die Permeabilität, die auch als magnetischer Widerstand bezeichnet wird. Darüber hinaus könnt ihr den Zusammenhang zwischen den Größen beschreiben und Rechnenbeispiele zum magnetischen Kreis lösen. Des Weiteren machen wir einen Exkurs in die Materialkunde. Hier lernt ihr die ferromagnetischen Eigenschaften wie Koerzitivfeldstärke, Remanenzflussdichte und Sättigung anhand der Hysterese-Kurve kennen.}

\end{frame}

\begin{frame} \ftx{Beispiel: Magnetische Flussdichte}
	\begin{bsp}{Magnetische Flussdichte}{}
		Im Inneren einer dicht gewickelten Ringspule soll die magnetische Feldstärke $H=100\,\frac{\mathrm{A}}{\mathrm{m}}$ erzeugt werden. Die Spule hat einen mittleren Radius von $5\,\mathrm{cm}$.
		\begin{enumerate}
			\item Berechnen Sie die erforderliche Stromstärke $I$ wenn die Spule mit $N=200$ Wicklungen versehen ist.\pause
			      
			      \s{Aus Gleichung \ref{Durchflutungsgesetz} und \ref{GlmagnFeldstaerke}:}
			      \begin{align*}
				      \varTheta & = H \cdot \ell_{\mathrm{m}} = N\cdot I                                                                                                                  \\
				      I         & =\frac{H\cdot\ell_{\mathrm{m}}}{N}=\frac{100\,\frac{\mathrm{A}}{\mathrm{m}}\cdot 2\cdot \pi\cdot 5\cdot 10^{-2}\,\mathrm{m}}{200} = 157,08\,\mathrm{mA}
			      \end{align*}\pause
			\item Wie groß wird die Flussdichte $B$ im Falle einer Luftspule ($\mu_{\mathrm{r}}=1$) oder einer eisengefüllten Spule ($\mu_{\mathrm{r}}=2000$ im Arbeitspunkt)?\pause
			      
			      \s{Aus Gleichung \ref{GlFlussdichte}:}
			      \begin{align*}
				      B_\mathrm{Luft}  & = \mu_0\cdot \mu_{\mathrm{r}}\cdot H = 1,256\cdot10^{-6}\,\tfrac{\mathrm{Vs}}{\mathrm{Am}} \cdot 100\,\tfrac{\mathrm{A}}{\mathrm{m}} = 125,6\,\mu\mathrm{T} \\
				      B_\mathrm{Eisen} & = 2000\cdot B_\mathrm{Luft} = 251,2\,\mathrm{mT}
			      \end{align*}
		\end{enumerate}
	\end{bsp}
\end{frame}

\section{Der magnetische Widerstand}

\begin{frame} \ftx{Magnetischer Widerstand}\index{Magnetischer Widerstand}

	\s{
		Im Kapitel \ref{durchflutung} wurde bereits die magnetische Spannung thematisiert, die auch Durchflutung genannt wird, sowie im Kapitel \ref{Flussdichte} der magnetische Fluss, der das Äquivalent zum elektrischen Strom darstellt. Es liegt daher die Vermutung nahe, dass es in einem magnetischen Kreis äquivalent zum ohmschen Widerstand auch einen magnetischen Widerstand\index{Magnetischer Widerstand} gibt. Er hat das Formelzeichen $R_{\mathrm{m}}$ und die Einheit $\frac{\mathrm{A}}{\mathrm{V}\mathrm{s}}$ und wird auch Reluktanz \index{Reluktanz}genannt.

		Ein Beispiel für einen magnetischen Kreis ist in Abbildung \ref{AbbMagnKreis} durch einen einfachen Eisenring mit einer einseitigen Spule dargestellt. Die Spule erzeugt eine Durchflutung $\varTheta$ analog zur elektrischen Spannung. Der Eisenkreis besteht aus vier Teilwiderständen, da der magnetische Widerstand sowohl von der Länge als auch von dem durchflossenen Querschnitt abhängig ist. Die vier Widerstände werden vom magnetischen Fluss $\varPhi$ durchströmt.
	}

	\fu{
		\def\bi{3.0}%Breite innen
\def\hi{2.0}%Höhe innen
\def\ba{5.0}%breite außen	
\def\ha{4.0}%Höhe außen		
\def\dx{0.8}%x-Verschiebung
\def\dy{0.5}%y-Verschiebung		
\def\dr{0.02}% round corner correction		
\def\countPrim{4}%Anzahl Primärwindungen

%Berechnungen
\def\lx{(\ba/2 - \bi/2)}%Breite Schenkel
\def\ly{(\ha/2 - \hi/2)}%Höhe Schenkel
\def\dyPrim{((\hi-\dy) / (\countPrim+1)/4)} %y-Verschiebung Wicklung Primärseite

\begin{circuitikz}[thick, every node/.style={transform shape, scale=1}, decoration={markings, mark=at position 0.5 with {\arrow{latex}}}]%global scale/.style={scale=1.0}, rotate=-5, xslant=-0.1
	\begin{scope}[even odd rule]
		\filldraw[rounded corners=2pt, fill=gray, rotate=-0, opacity=1.0] (\dx,\dy) rectangle ++(\ba,\ha) ({\lx+\dx},{\ly+\dy}) rectangle ++(\bi, \hi);%hinten
		\fill [rounded corners=2pt, fill=gray] (\ba, 0) --++ (0, \dy+\dr+\dr) --++(\dx, 0) --cycle;%rechte Seite
		\fill [rounded corners=2pt, fill=gray] (0, \ha) --++ (\dx+\dr+\dr, 0) --++(0, \dy)--cycle;%obere Seite
		\filldraw[rounded corners=2pt, fill=gray!50, rotate=-0] (0,0) rectangle ++(\ba, \ha) ({\lx},{\ly}) rectangle ++(\bi, \hi);%vorne
		\draw (\ba-\dr,\dr) --++(\dx, \dy);%Eckverbindungen
		\draw (\ba-\dr,\ha-\dr) --++(\dx, \dy);%Eckverbindungen
		\draw (\dr,\ha-\dr) --++(\dx, \dy);%Eckverbindungen
	\end{scope}



	%Richtungspfeile Fluss und Flächen
	\draw[magnetfeld, very thick, rounded corners=10pt, <-] ({\ba*0.4},{\ly/2}) -- ({\ba*0.55},{\ly/2});
	%Fläche unten
	\fill[fill=gray, fill opacity=1, blend mode=screen] ({\ba/2},0) -- ++(0,{\ly}) -- ++({\dx},{\dy}) -- ++(0,{-\ly});
	\draw ({\ba/2},0) node[below] {$A_4$};
	\draw[magnetfeld, very thick, rounded corners=10pt] ({\ba*0.55},{\ly/2}) -- ({\ba*0.7},{\ly/2}) node[right] {$\ell_4$};
	\draw[magnetfeld, very thick, rounded corners=10pt] ({\ba*0.8},{\ly/2}) -- ({\ba-\lx/2},{\ly/2}) -- ({\ba-\lx/2},{\ha*0.55});
	%Fläche rechts
	\fill[fill=gray, fill opacity=1, blend mode=screen] ({\ba-\lx+\dx},{\ha/(1.1+\dy)}) -- ++({\lx},0) -- ++({-\dx},{-\dy}) -- ++({-\lx},0);
	\draw ({\ba+\dx},{\ha/(1.1+\dy)}) node[right] {$A_3$};
	\draw[magnetfeld, very thick, rounded corners=10pt] ({\ba-\lx/2},{\ha*0.55}) -- ({\ba-\lx/2},{\ha*0.6}) node[above] {$\ell_3$};
	\draw[magnetfeld, very thick, rounded corners=10pt] ({\ba*0.55},{\ha-\ly/2}) -- ({\ba*0.45},{\ha-\ly/2}) node[left] {$\ell_2$};
	%Fläche oben
	\fill[fill=gray, fill opacity=1, blend mode=screen] ({\ba/2},{\ha-\ly}) -- ++(0,{\ly}) -- ++({\dx},{\dy}) -- ++(0,{-\ly});
	\draw ({\ba/2},{\ha-\ly}) node[below] {$A_2$};
	\draw[magnetfeld, very thick, rounded corners=10pt] ({\ba-\lx/2},{\ha*0.75}) -- ({\ba-\lx/2},{\ha-\ly/2}) -- ({\ba*0.55},{\ha-\ly/2});
	\draw[magnetfeld, very thick, rounded corners=10pt] ({\ba*0.35},{\ha-\ly/2}) -- ({\lx/2},{\ha-\ly/2}) -- ({\lx/2},{\ha*0.75}) node[below] {$\ell_1$};
	\draw[magnetfeld, very thick, rounded corners=10pt] ({\lx/2},{\ha*0.32}) -- ({\lx/2},{\ly/2}) -- ({\ba*0.3},{\ly/2}) node[right] {$\varPhi$};
	%Fläche links
	\fill[fill=gray, fill opacity=1, blend mode=screen] (\dx,{\ha/2.5}) -- ++({\lx},0) -- ++({-\dx},{-\dy}) -- ++({-\lx},0);
	\draw ({\dx+\lx},{\ha/2.5+0.2}) node[right] {$A_1$};
	\draw[magnetfeld, very thick, rounded corners=10pt] ({\lx/2},{\ha*0.6}) -- ({\lx/2},{\ha*0.32});

	%Primärwicklung
	%Zuleitung oben
	\draw[rounded corners=2pt, current, thick, postaction={decorate}] (-1,{\ly+\dr+\hi-\dy}) coordinate (U1oben)%Startpunkt oben
	-- ++(1,0) node[current, above, pos=0.4] {$\underline{I}$}; %Zuleitung
	%oberste Wicklung Primärwicklung
	\draw[rounded corners=2pt, current, thick]
	(0,{\ly+\dr+\hi-\dy}) %Startpunkt oben
	-- ++({\lx}, {-\dyPrim})% vordere Linie
	-- ++({\dx + 0.07},{-\dyPrim + \dy})%Linie nach hinten
	-- ++(-0.06,0.06); %Kreis rechts nach hinten
	%restliche Primärwicklungen
	\foreach \n in {2,...,\countPrim}
		{
			\draw[rounded corners=2pt, current, thick]
			(0,{\ly+\n*(\dyPrim * 4) + 0.08}) %Startpunkt oben
			-- ++(-0.07, -0.08)%Kreis links
			-- ++({\lx + 0.07}, {-\dyPrim})% vordere Linie
			-- ++({\dx + 0.07},{-\dyPrim + \dy})%Linie nach hinten
			-- ++(-0.06,0.06); %Kreis rechts nach hinten
		}
	%untere Zuleitung Primärwicklung
	\draw [current, thick, postaction={decorate}] (0, {\ly+((\hi-\dy) / (\countPrim+1))}) -- +(-1,0) coordinate (U1unten);
	%Primärspannung
	\draw (U1oben) to[voltage, open, v^=$\underline{U}$, o-o] (U1unten);
	\draw [-{Triangle[scale=1.2]}, color=voltage] (-1,2.4) -- (-1,1.44);
\end{circuitikz}
\begin{circuitikz}
	\draw [opacity=0](-1,0) -- ++(1,0); % Abstandhalter nach links
	\draw (0,0) coordinate (Uo) to[open, v=$\varTheta$] ++(0,-1) coordinate (Uu); % Quelle
	%\draw [-{Triangle[scale=1.2]}, color=voltage] (Uo) -- (Uu);		
	\draw (Uo) to[short, o-] ++(1,0) to[short] ++(0,0.5) coordinate (R2l);
	\draw (R2l) to[short] ++(0.5,0) to[R=$R_{m2}$] ++(2,0) coordinate (R2r); % R2
	\draw (R2r) to[color=magnetfeld, short, i=$\varPhi$] ++(0.5,0) coordinate (R3o);
	\draw (R3o) to[R=$R_{m3}$] ++(0,-3) coordinate (R3u); % R3
	\draw (R3u) to[R=$R_{m4}$] ++(-3,0) coordinate (R4l); % R4
	\draw (R4l) to[R=$R_{m1}$] ++(0,1.5) coordinate (R1o); % R1
	\draw (R1o) to[short, -o] (Uu);
\end{circuitikz}
	}{{\bf Magnetischer Kreis mit einem Eisenring.} Die vier magnetischen Teilwiderstände sind abhängig von der Länge, vom Material und dem Querschnitt des durchflossenen Eisenrings. \label{AbbMagnKreis}}

	\s{In einfachen Anordnungen (wie beispielsweise in Abbildung \ref{AbbMagnKreis} zu sehen) kann unter Vernachlässigung der Ecken der Widerstand durch die Gleichung \ref{GlmagnWiderstand} ausgedrückt werden. $\ell_{\mathrm{m}}$ ist wie bei der magnetischen Feldstärke die mittlere Feldlinienlänge des Widerstandes innerhalb des magnetischen Kreises. Zur Ermittlung der mittleren Feldlininenlänge ${\ell_{\mathrm{m}}}$ werden die Längen aller Seiten addiert.
	$A$ ist die Querschnittsfläche, die vom magnetischen Fluss durchflossen wird. $\mu$ ist die Permeabilität\index{Permeabilität} des Materials.
	Der magnetische Widerstand $R_{\mathrm{m}}$ berechnet sich nun aus der aufsummierten Länge der mittleren Feldlinienlänge ${\ell_{\mathrm{m}}}$ geteilt durch das Produkt der durchflossenen Querschnittsfläche $A$ und der Permeabilität $\mu_{\mathrm{r}} \cdot \mu_0$.
	
	\begin{eq}
		R_{\mathrm{m}} =\frac{\ell_{\mathrm{m}}}{\mu_{\mathrm{r}}\cdot\mu_0\cdot A} \qquad\left[\frac{\mathrm{A}}{\mathrm{V}\cdot\mathrm{s}}\right]\label{GlmagnWiderstand} 
	\end{eq}
	
	\begin{Merksatz}{Magnetischer Widerstand}
		Für die Berechnung des magnetischen Widerstands $R_{\mathrm{m}}$ wird die mittlere Feldlinienlängen ${\ell_{\mathrm{m}}}$ durch das Produkt aus der Permeabilität des Materials ${\mu_{\mathrm{r}}\cdot\mu_0}$ und der Querschnittsfläche $A$ geteilt. Bei unterschiedlicher Materialbeschaffenheit oder unterschiedlicher Querschnittsflächen innerhalb des magnetisierten Körpers werden zunächst die Teilwiderstände errechnet und anschließenden zum Gesamtwiderstand aufsummiert.
	\end{Merksatz}
	}
	\pause	
	\b{\begin{minipage}{0.5\textwidth}}
	\b{
		\begin{eq}
			R_{\mathrm{m}} =\frac{\ell_{\mathrm{m}}}{\mu_{\mathrm{r}}\cdot\mu_0\cdot A} \qquad\left[\frac{\mathrm{A}}{\mathrm{V} \cdot \mathrm{s}}\right]
		\end{eq}}
	\b{\end{minipage}}%
\end{frame}

\subsection{Der magnetische Kreis}
\begin{frame} \ftx{Magnetischer Kreis}\index{Magnetischer Kreis}		
	\b{
		Analog zum elektrischen Ohmschen Gesetz gilt für den magnetischen Kreis:
		\begin{eqa}
			\varTheta &= R_{\mathrm{m}}\cdot \varPhi
		\end{eqa}
	}
	\s{Werden nun alle bekannten magnetischen Größen in ihrer Zusammenwirkung betrachtet, ergibt sich die magnetische Analogie zum Ohmschen Gesetz. Sie beschreibt, dass die magnetische Spannung $\varTheta$  dem Produkt aus dem magnetischen Widerstand $R_{\mathrm{m}}$ und dem magnetischen Fluss $\varPhi$ entspricht.	
	\begin{eqa}
		\varTheta &= R_{\mathrm{m}}\cdot \varPhi
	\end{eqa}
	\begin{Merksatz}{Zusammenhänge der magnetischen Feldgrößen}


		Die magnetische Feldstärke $H$ im Spezialfall einer Ringkernspule errechnet sich aus dem Produkt der Windungsanzahl $N$ und dem Stromfluss $I$, geteilt durch die mittlere Feldlinienlänge $\ell_{\mathrm{m}}$. \par
		\begin{equation*}
			H=\frac{N\cdot I}{\ell_{\mathrm{m}}}
		\end{equation*}		
		Die magnetische Flussdichte $B$ wird durch  Multiplikation der magnetischen Feldstärke $H$ mit der Permeabilität $\mu_{\mathrm{r}} \cdot \mu_0$ bestimmt. \par
		\begin{equation*}	
			B = \mu_{\mathrm{r}} \cdot \mu_0 \cdot H 
		\end{equation*}	

		Der magnetische Fluss $\varPhi$ ergibt sich aus dem Integral der magnetischen Flussdichte $B$ über einer Fläche $A$.		\begin{equation*} 
			\varPhi = \iint\limits_A \vec{B} \cdot \mathrm{d} \vec{A}  
		\end{equation*}	


	\end{Merksatz}
	}
\end{frame}

