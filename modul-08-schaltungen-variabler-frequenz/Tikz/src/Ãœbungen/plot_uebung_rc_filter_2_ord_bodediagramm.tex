% Bode-Diagramm für RC-Tiefpass Vgl. 1. Ordnung, 2. Ordnung
% 1. Ordnung: RC-Tiefpass
% 2. Ordnung: 2x RC-Tiefpass in Reihe
% R*C=1e2 Hz

\begin{tikzpicture}[x=1cm,y=1cm]
% Amplitudengang
\begin{axis}[
    name=AxisAmplitudengang,
    xmode=log,
    ymode=log,
    xmin=1e0, xmax=1e4,
    ymin=0.5e-4, ymax=2e0,
    ylabel={$A(\omega)$},
    xlabel={$\omega$},
    width=12cm,
    height=6cm,
    grid=major,
    domain=1e0:1e4,
    samples=161,
    legend pos=south west,
]
    % 1. Ordnung: A(w) = 1 / ( 1 + (w*R*C)^2 )^0.5
    \addplot[color=red, very thick] {1 / sqrt( 1 + (x*1e-2)^2 )};
    \addlegendentry{einfaches RC-Glied}
    % 2. Ordnung: A(W) = 1 / ( (1-(w*R*C)^2)^2 + (3*w*R*C)^2 )^0.5
    \addplot[color=blue, very thick] {1 / sqrt( (1-(x*1e-2)^2)^2 + (3*x*1e-2)^2 )};
    \addlegendentry{doppeltes RC-Glied}
    
    s% 2. Ordnung entkoppelt: A(w) = A_1(w) * A_1(w)
    s\addplot[color=blue, dotted, thick] {1 / ( 1 + (x*1e-2)^2 )};
    s\addlegendentry{doppelt, entkoppelt}
\end{axis}

% Phasengang
\begin{axis}[
    at={($(AxisAmplitudengang.south west)+(0,-5.5cm)$)}, % Position relativ zu anderem Plot
    xmode=log,
    ymode=normal,
    xmin=1e0, xmax=1e4,
    ymin=-190, ymax=+10,
    ytick distance=45,
    xlabel={$\omega$},
    ylabel={$\varphi(\omega)\,[\degree]$},
    width=12cm,
    height=6cm,
    grid=major,
    domain=1e0:1e4,
    samples=161,
    legend pos=south west,
]
    % 1. Ordnung: phi(w) = -atan( w*R*C )
    \addplot[color=red, very thick] {-atan( x*1e-2 )};
    %\addlegendentry{einfaches RC-Glied}
    % 2. Ordnung: phi(w) = -2*atan( w*R*C )
    \addplot[color=blue, very thick, domain=1e0:0.99999e2] {-atan( (3*x*1e-2) / (1-(x*1e-2)^2) )}; 
    \addplot[color=blue, very thick, domain=1.00001e2:1e4, forget plot] {-atan( (3*x*1e-2) / (1-(x*1e-2)^2) ) -180}; % Sprung wegen atan Def.-Bereich 
    %\addlegendentry{doppeltes RC-Glied}
    
    % 2. Ordnung entkoppelt: phi(w) = phi_1(w) + phi_1(w)
    \addplot[color=blue, dotted, thick] {-2*atan( x*1e-2 )};
    %\addlegendentry{doppelt, entkoppelt}
\end{axis}
\end{tikzpicture}