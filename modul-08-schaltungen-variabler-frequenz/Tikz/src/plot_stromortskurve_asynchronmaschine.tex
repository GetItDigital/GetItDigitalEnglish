% Asynchronmaschine Ortskurve

% Da die Ortskurve der Asynchronmaschine nicht wie normal Re=x und Im=y ist, sondern Re=-Im und Im = Re, wurden die x und y Argumente verschoben und das Vorzeichen vom Imaginärteil geändert.

% Diese Parameter sind Beispielhaft für eine Asynchronmaschine und es wurden der Statorwiderstand und die Statorinduktivität vernachlässigt, da sie im stationären  Zustand vernachlässigbar sind.

% U_s = 400V
% R_s = 0 Ohm im stat. Zustand vernachlässigbar
% X_s = 20 Ohm 
% R_r´ = 0.5 Ohm
% X_r´ = 1.5 Ohm

% I_mu = U_s / j X_s = 400V / j 2 Ohm = 200 A / j
% I_s = I_mu - U_s / (R_r´/s + j X_r´)
% s=1 (520, 160) Koordinaten für die Node

\def\U_s{400}
\def\R_s{0}
\def\X_s{20}
\def\R_r{0.5}
\def\X_r{1}


\begin{tikzpicture}
    \begin{axis}[
        xmin=-50, xmax=630,
        ymin=-260, ymax=420,
        xlabel={Im},
        ylabel={Re},
	    axis x line=center,
        axis line style={-},
        axis y line=center,
        xtick={100, 200, 300, 400, 500, 600},
        xticklabels={-100, -200, -300, -400, -500, -600},
        width=10cm,
        height=10cm,
    ]

    \addplot[domain=0.01:1, samples=300, blue] % Motor
    ({(\U_s * \X_r)/(\R_r^2/x^2 + \X_r) +\U_s / 20},
    {(\U_s * \R_r /x)/(\R_r^2/x^2 + \X_r)});

    \addplot[domain=-0.01:-1, samples=300, color=green1] % Generator Teil 1
    ({(\U_s * \X_r)/(\R_r^2/x^2 + \X_r) +\U_s / 20},
    {(\U_s * \R_r /x)/(\R_r^2/x^2 + \X_r)});

    \addplot[domain=1:100, samples=300, red] % Bremse
    ({(\U_s * \X_r)/(\R_r^2/x^2 + \X_r) +\U_s / 20},
    {(\U_s * \R_r /x)/(\R_r^2/x^2 + \X_r)});

    \addplot[domain=-1:-100, samples=300, color=green1] % Generator TEil 2
    ({(\U_s * \X_r)/(\R_r^2/x^2 + \X_r)+ \U_s / 20},
    {(\U_s * \R_r /x)/(\R_r^2/x^2 + \X_r)});

    \draw[->, color=blue, thick](0,0) -- (0,400) ; % U_s
    \node[blue, anchor=east] at (0, 250) {$\underline{U}_s$};

    \draw[->, color=blue](0,0) -- (220,200); % I_s
    \node[color=blue] at (100, 130) {$\underline{I}_s$};

    \draw[->, color=blue](0,0) -- (20,0); % I_mu
    \draw[->, color=blue](20,-40) -- (10,-5); % I_mu->
    \node[color=blue] at (40, -60) {$\underline{I}_{\mu}$};

    \draw[color=blue, ->](220,200) -- (20,0); % I_r´
    \draw[color=blue](150, 50)node{$\underline{I}_r`$};

    \node[color=blue] at (300, 140) {$s=1$}; % s=1
    \fill[blue] (340,160) circle (0.05cm);

    \node[color=blue, anchor=south] at (220, 200) {$s_k$}; % s_k
    \fill[blue] (220,200) circle (0.05cm);
    \node[color=green1, anchor=north] at (220, -200) {$-s_k$}; % -s_k
    \fill[green] (220,-200) circle (0.05cm);

    \node[color=blue] at (100,-60){$s=0$}; % s=0
    \draw[color=blue, ->](70,-40) -- (20, 0); % Pfeil für s=0 weil kein Platz
    \fill[blue] (20,0) circle (0.05cm);

    \fill[blue] (420,0) circle (0.05cm);
    \node[color=blue,anchor=west] at (320,20) {$s \rightarrow \infty$}; % s -> infty

    \draw[->, black](0,0) -- (-50,0); % X - Achse geflipped

    \draw[blue, <->] (220,250) arc [start angle=90, end angle=53.1, radius=250]; % Motor
    \node[blue] at (300, 270) {Motor};

    \draw[red, <->] (470,0) arc [start angle=0, end angle=53.1, radius=250]; % Bremse
    \node[red] at (520, 100) {Bremse};

    \draw[green1, <->] (470,0) arc [start angle=0, end angle=-90, radius=250]; % Generator
    \node[green1] at (520, -100) {Generator};

    % Untransformierte Ortskurve alte Version
    %\addplot[domain=0.01:1, samples=300, blue] % Motor transformiert
    %({-(\U_s * \X_r)/(\R_r^2/x^2 + \X_r)},
    %{-(\U_s * \R_r /x)/(\R_r^2/x^2 + \X_r)});

    %\addplot[domain=-0.01:-1, samples=300, color=green1] % Generator transformiert Teil 1
    %({-(\U_s * \X_r)/(\R_r^2/x^2 + \X_r)},
    %{-(\U_s * \R_r /x)/(\R_r^2/x^2 + \X_r)});

    %\addplot[domain=-1:-100, samples=300, color=green1] % Generator transformiert Teil 2 
    %({-(\U_s * \X_r)/(\R_r^2/x^2 + \X_r)},
    %{-(\U_s * \R_r /x)/(\R_r^2/x^2 + \X_r)});

    %\addplot[domain=1:100, samples=300, red] % Bremse transformiert???
    %({-(\U_s * \X_r)/(\R_r^2/x^2 + \X_r)},
    %{-(\U_s * \R_r /x)/(\R_r^2/x^2 + \X_r)});

    %\draw[color=blue, ->](0,0) -- (-320,-160); % I_r
    %\draw[color=blue](-200, -130)node{$I_r$};

    %\draw[color=blue](-400, -180) node{$s=1$}; % s=1 transformiert

    %\draw[color=green1](-200, 200) node[anchor=south]{$s_k$}; % s_k
    %\fill[green] (-200,200) circle (0.05cm);
    %\draw[color=blue](-200, -200) node[anchor=north]{$-s_k$}; % -s_k
    %\fill[blue] (-200,-200) circle (0.05cm);

    \end{axis}
\end{tikzpicture}
% Grün ändern +
% Legende weg und mit den Pfeilen tauschen+
% Richtige Maschinenwerte einsetzen +
% Achsenbeschriftung anpassen+
% Achsen flippen +
% wieso ir? +
% Ir ist falsch +
% komplexe ströme(unterstrich) +
% axis cs: statt \draw und die Koordinate (hab node gemacht)+
