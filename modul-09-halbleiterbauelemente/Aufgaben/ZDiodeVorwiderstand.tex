\subsection{Dimensionierung des Vorwiderstands einer Zenerdiode\label{Aufg5}}
% Alt: Aufgabe 5
\Aufgabe{
  Die Zenerspannung der Diode beträgt $U_\mathrm{D}=6\,\mathrm{V}$ und die Diode darf eine maximale Verlustleistung von \(P_\mathrm{tot}=\mathrm{400\,mW} \) nicht überschreiten. 
  Die maximale Eingangsspannung beträgt \( U_0 = \mathrm{12\,V} \). \\
  Wie groß muss der Widerstand \( R \) mindestens sein, damit die Diode nicht zerstört wird?

\begin {figure} [H]
   \centering
   \begin{tikzpicture}

    \draw (0,2) to[V, name=U0] (0,0);
    \draw (0,2) to[R=$R$] (4,2);
    \draw (0,0) -- (4,0);
    \draw (4,0) to[zDo, l_=$D_\mathrm{z}$, v^<, name=Uz] (4,2);

    \varrmore{U0}{$U_0$}; 
    \varrmore{Uz}{$U_\mathrm{D}$}; 
  
  \end{tikzpicture}
   \label{fig:ZDiodeVorwiderstand}
\end {figure}

% \speech{ **Aufgabe 5**  
% Eine **Zenerdiode** mit einer **Zenerspannung von 6V** wird in einer Schaltung verwendet, um **Überspannung zu begrenzen**.  
% Die **maximale Verlustleistung** der Diode beträgt **400 mW**, und die **maximale Eingangsspannung \( U_0 \)** ist **12V**.  

% **Fragestellung:**  
% - Wie groß muss der **Widerstand \( R \)** mindestens sein, damit die **Diode niemals zerstört** wird?  

%  **Beschreibung der Abbildung**  
% Die Schaltung besteht aus den folgenden Komponenten:

% 1. **Gleichspannungsquelle \( U_0 \)**  
%    - Symbolisiert durch einen Kreis mit einem Pfeil.  
%    - Die **maximale Spannung beträgt 12V**.  

% 2. **Serienwiderstand \( R \)**  
%    - Platziert im oberen Zweig der Schaltung.  
%    - Begrenzung des **Stroms durch die Zenerdiode**.  

% 3. **Zenerdiode**  
%    - In **Sperrrichtung** geschaltet.  
%    - Sobald die **Spannung \( U_0 > 6V \)** erreicht, beginnt die Diode zu **leiten** und hält die Spannung auf **6V konstant**.  

%  **Funktionsweise der Schaltung**  
% - Wenn **\( U_0 < 6V \)**, ist die **Diode nicht leitend**, und es fließt kein Strom durch sie.  
% - Wenn **\( U_0 > 6V \)**, hält die **Zenerdiode die Spannung bei 6V**, indem sie den überschüssigen Strom ableitet.  
% - Der Widerstand \( R \) muss so **gewählt werden**, dass der **Strom durch die Diode die maximale Verlustleistung von 400 mW nicht überschreitet**.  


%  **Zusammenfassung der Aufgabe**  
% - Die Aufgabe erfordert die **Berechnung des minimalen Widerstandswerts \( R \)**, um die **maximale Verlustleistung der Zenerdiode nicht zu überschreiten**.  
% - Die Schaltung dient als **Spannungsregler**, der eine **konstante Spannung von 6V gewährleistet**, wenn die Eingangsspannung zu hoch ist.  
% - Die Berechnung erfolgt mit **Leistungsgesetzen und dem Ohm’schen Gesetz**, um den maximal zulässigen Strom durch die Diode zu bestimmen.  
% }

}


\Loesung{
    Die Spannung am Widerstand ergibt sich zu:
    \begin{align*}
        U_\mathrm{R} &= U_0 - U_\mathrm{D} = \mathrm{12\,V} - \mathrm{6\,V} = \mathrm{6\,V}
    \end{align*}
    Die maximale Verlustleistung der Diode ist:
    \begin{align*}
        P_\mathrm{tot} &= U_\mathrm{D} \cdot I = \mathrm{400\,mW}
    \end{align*}
    Daraus folgt der maximale Strom durch die Diode:
    \begin{align*}
        I &= \frac{\mathrm{400\,mW}}{\mathrm{6\,V}} = \mathrm{66{,}7\,mA}
    \end{align*}
    Um diesen Strom nicht zu überschreiten, ergibt sich der Mindestwert für den Widerstand:
    \begin{align*}
        R &= \frac{U_\mathrm{R}}{I} = \frac{\mathrm{6\,V}}{\mathrm{66{,}7\,mA}} \approx \mathrm{90\,\Omega}
    \end{align*}
  Siehe: Abschnitt \ref{sec:SpezielleDioden}

%   \speech{
% **Lösung zu Aufgabe 5**  

% Die gegebene Schaltung besteht aus einer Spannungsquelle mit **12 Volt**, einem Widerstand **R**, und einer **Zenerdiode**.  
% Die **Zenerdiode** begrenzt die Spannung an ihren Anschlüssen auf **6 Volt**.  
% Das bedeutet, dass die restlichen **6 Volt** über den Widerstand **R** abfallen müssen.  

% ### Berechnungen  

% **Berechnung des Spannungsabfalls über den Widerstand**  
% Da die **Gesamtspannung** der Quelle **12 Volt** beträgt und die Zenerdiode **6 Volt** benötigt, ergibt sich die Spannung über **R** als:  

% \[
% U_R = 12V - 6V = 6V
% \]  

% **Berechnung der Leistung**  
% Die gegebene **maximale Verlustleistung** der Zenerdiode beträgt **0,4 Milliwatt**.  
% Die elektrische Leistung wird nach der Formel berechnet:  

% \[
% P = U \cdot I
% \]  

% Einsetzen der bekannten Werte:  

% \[
% 0,4 mW = 6V \cdot I
% \]  

% Nach **Umstellen nach I** ergibt sich:  

% \[
% I = \frac{0,4 mW}{6V} = 66,7mA
% \]  

% **Berechnung des Widerstands \( R \)**  
% Nach dem **Ohmschen Gesetz** gilt:  

% \[
% U = R \cdot I
% \]  

% Umstellen nach **R**:  

% \[
% R = \frac{U}{I} = \frac{6V}{66,7mA} = 90 \Omega
% \]  

% ### Interpretation  

% Damit die **Zenerdiode** nicht beschädigt wird, muss der Widerstand **R** mindestens **90 Ohm** betragen.  
% Dieser Widerstand begrenzt den Strom auf **66,7 Milliamper**, sodass die **maximale Verlustleistung der Diode** nicht überschritten wird.  

% }

}



