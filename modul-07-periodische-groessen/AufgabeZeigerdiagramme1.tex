\subsection{Zeigerdiagramme 1}
\Aufgabe{
    
    Gegeben sei die Spannung $\hat{U} = 325\ V \cdot \mathrm{e}^{\mathrm{j}30^\circ}$.

    \begin{itemize}
        \item [\bf a)] Zeichnen Sie das Zeigerdiagramm für die Spannung $\hat{U}$ im komplexen Zahlenraum.
        \item [\bf b)] Berechnen Sie die Real- und Imaginärteile der komplexen Spannung $\hat{U}$.
        \item [\bf c)] Erklären Sie die Bedeutung des Phasenwinkels in Bezug auf die Wechselspannung.
    \end{itemize}
}

\Loesung{
	\begin{itemize}

	\item[\bf a)]
	Zeigerdiagramm für die Spannung $\hat{U}$ \\

    \begin{figure}[H]
        \centering
        \resizebox{0.3\textwidth}{!}{
        \begin{tikzpicture}
            % Achsen zeichnen
            \draw[->] (-0.5,0) -- (3,0) node[right] {$\Re$};
            \draw[->] (0,-0.5) -- (0,1.5) node[above] {$\Im$};
        
            % Zeiger zeichnen
            \draw[->, thick, color=voltage] (0,0) -- ({2 * cos(30)},{2 * sin(30)}) node[above right] {$\hat{U}$};
        
            % Winkel markieren
            \draw[->] (0,0) -- (0.5,0) node[below] {$0^\circ$};
        
            % Beschriftung
            \node at (2,0.8) {$30^\circ$};
        \end{tikzpicture}
        } 
    \end{figure} 

	\item[\bf b)] 
        Die Real- und Imaginärteile der Spannung $\hat{U}$ können mit den folgenden Formeln berechnet werden:

        \begin{eqa}
            U_{\text{Re}} &= \hat{U} \cdot \cos(\varphi) = 325 \cdot \cos(30^\circ) \approx 281,46\ V      \nonumber   \\
            U_{\text{Im}} &= \hat{U} \cdot \sin(\varphi) = 325 \cdot \sin(30^\circ) = 162,5\ V   \nonumber
        \end{eqa}

	\item[\bf c)] 
    Der Phasenwinkel $\varphi$ beschreibt den zeitlichen Versatz zwischen der Spannung und dem Strom in einem Wechselstromkreis. 
    Ein positiver Phasenwinkel deutet darauf hin, dass die Spannung der Strom nachfolgt (induktive Last), während ein negativer 
    Phasenwinkel anzeigt, dass der Strom der Spannung nachfolgt (kapazitive Last). 

\end{itemize}

}