\immediate\write18{makeindex -g -s Templates/Arbeit.ist TeXAux/Skript.idx}
\immediate\write18{cat TeXAux/Skript.nlo | sort | uniq -w 55 | makeindex -i -s nomencl.ist -o TeXAux/Skript.nls}
\immediate\write18{cd TeXAux;gnuplot *.gnuplot;cd ..;}

\documentclass[twoside, a4paper]{article}
\usepackage{beamerarticle}
\pdfminorversion=6



\usepackage[utf8]{inputenc}         % Zeichenkodierung auf UTF-8
\usepackage[ngerman]{babel}         % Deutsche Texte, Sonderzeichen
\usepackage{color, colortbl}		% Farben, Tabellenfarben
\usepackage{multirow}			   	% Tabellen
\usepackage{booktabs}				% better tables (\toprule, \midrule, \bottomrule)
\usepackage{ifthen}                 % Konditionen
\usepackage{makeidx}                % automatische Index-Generierung
\usepackage[german]{nomencl}        % Nomenklatur (Symbolverzeichnis)
\usepackage{etoolbox}               % Gliederung von der Nomenklatur
\usepackage{icomma}				    % kein Leerzeichen hinter einem Komma, wenn es in einer Zahl steht
\usepackage{amsmath}                % Mathematische Symbole
\usepackage{amsfonts}               % Mathematische Symbole
\usepackage{amssymb}                % Mathematische Symbole
\usepackage{mathtools}				% Addon for amsmath (z.B. für \mathclap)
\usepackage{cancel}				    % Durchstreichen von Formelteilen
\usepackage{textcomp, gensymb}	    % Grad-Zeichen
\usepackage{xargs}					% Eigene Kommandos mit optionalen Argumenten definieren \newcommandx* für eigenes \highlight command basierend auf hf-tikz
\usepackage{booktabs}	   			% Für schönere horizontale Linien (FH Aachen)
\usepackage[intlimits]{esint}		% Schönere Integrale
\usepackage{siunitx}	   			% Für SI-Einheiten (FH Aachen)
\usepackage{colortbl}	   			% Farben (FH Aachen)
\usepackage{subcaption}             % Für Unterabbildungen (FH Aachen)
%\usepackage{caption}                % Caption (FH Aachen)
\usepackage[nice]{nicefrac}			% Math fracs in nice!
\usepackage{xfrac}					% newer than nicefrac \sfrac{}{} > \nicefrac{}{}
\usepackage{bm}						% Hervorhebung von Teilen einer Formel

% float-Positionierungen:
\usepackage{float}  			    % Gleitobjekte mit H genau an die angegebene Stelle positionieren.
\usepackage{placeins} 			    % neuer Befehl: \FloatBarrier  zu dem Zeitpunkt werden alle Floats gesetzt.

% Schaltpläne und LaTeX-Grafiken
\usepackage[nosiunitx,european,straightvoltages]{circuitikz}
\usepackage{tikz}
\usepackage{siunitx}
\usepackage{pgfplots}			    % Bode Diagramme
\usepackage{mathrsfs}				% Laplace Zeichen
%tabellen damit schaltungen ausgerrichtet sind
\usepackage{array}
\setlength{\marginparwidth}{2cm}	%Eingefügt von Jonas, um die To-Dos richtig darzustellen

%Lernziele, Merksätze und Beispiele
\usepackage{varwidth}
\usepackage[most,skins,breakable]{tcolorbox}
% Grid Overlay zur Positionierungshilfe - DEBUGGING ONLY
%\usepackage[texcoord,grid,gridcolor=red!60,subgridcolor=green!60,gridunit=mm]{eso-pic}
\usepackage{graphicx}				% for subfigures and resizeboxtikz
\usepackage{adjustbox}[export]		% crop images relative to image width/height
\usepackage{tcolorbox}				% Farbige Boxen für Definitionen, Beispiele etc.
\tcbuselibrary{skins}
\tcbuselibrary{breakable}			% für Seitenumbrüche in tcolorbox
%\usepackage{varwidth}

\usepackage{svg}					%Einbindung von svg-Grafiken

\usepackage{overpic}	                            % Text auf Bild platzieren

\usepackage{silence}                                        % Suppress warnings with \WarningFilter
\WarningFilter{latex}{You have requested package}           % Suppress warning about package location

% Optionen des Grafikpakets TikZ
\usetikzlibrary{arrows}				% schönere Pfeilspitzen
\usetikzlibrary{datavisualization}
\usetikzlibrary{angles}				% Winkelberechnung und Darstellung
\usetikzlibrary{decorations.markings}
\usetikzlibrary{patterns}           % gefüllte Flächen
\usetikzlibrary{shapes.geometric}	% Geometrische Formen (z.B. Dreieck für Verstärker)
\usepgfplotslibrary{fillbetween}	% Fläche zwischen zwei Kurven
\tikzset{>=stealth'}				% schönere Pfeilspitzen

% Bilder in getrenntem Pfad:
\graphicspath{{Bilder/}{../Templates/Bilder/}}


% eigene Befehle
\newcommand{\red}[1]{\textcolor[rgb]{1.00,0.00,0.00}{#1}}
\newcommand{\blue}[1]{\textcolor[rgb]{0.00,0.00,1.00}{#1}}
\newcommand{\Hz}{\operatorname{Hz}} % für gerades Hz
\newcommand{\dB}{\operatorname{dB}} % für gerades dB (Dezibel)
\newcommand{\dBm}{\operatorname{dBm}} % für gerades dBm {Dezibel auf 1 mW bezogen}
\renewcommand{\Re}{\operatorname{Re}} % für gerades Re für Realteil wie in physics package (statt Fraktur-R wie in plain TeX)
\renewcommand{\Im}{\operatorname{Im}} % für gerades Im für Imaginärteil wie in phycis package (statt Fraktur-I wie in plain TeX)
\newcommand{\dt}[1][{}]{\frac{\mathrm{d}^{#1}}{\mathrm{d}t^{#1}}} % \dt = d/dt ; \dt[2] = d^2/dt^2
% Note: \Re, \Im in Superscript/Subscript immer mit {}, Bsp. ^{\Im} oder _{\Re} statt ^\Im oder _\Re damit beide styles funktionieren
\newcommand{\alttexxt}[1]{} % Alternativtext für Bilder. Wird zur Zeit noch ignoeriert.

% Durchstreichen von Formelteilen
\definecolor{Red}{rgb}{1,0,0}
\newcommand{\colorcancel}[2]{\renewcommand{\CancelColor}{\color{#2}}\cancel{#1}}	% \colorcancel{text to cancel}{color}
\newcommand{\redcancel}[2][red]{\renewcommand{\CancelColor}{\color{#1}}\cancel{#2}} % \redcancel[optional color]{text to cancel} <=compatible=> \cancel{text to cancel}

% Multi Footnotes (gleiche Nummer, gleicher Text)
\newcommand{\footm}{\footnotemark} %https://tex.stackexchange.com/questions/86650/how-to-display-the-footnote-in-the-bottom-of-the-slide-while-using-columns
\newcommand*{\footmc}[1][0]{\addtocounter{footnote}{#1}\footnotemark[\value{footnote}]{}}% footmark with either same number (default) or by optional argument incremented number
\newcommand{\foott}[1]{\footnotetext{#1}}
% 	blabla\footm         ... undso\footmc\\ % 1...1
% 	blabla\footmc[-1]    ... undso\footmc\\ % 2...2
%	blabla\footmc[-1]    ... undso\footmc 	% 3...3
%	\foott{1st info} \foott{2nd info} \foott{3rd info}	 % footnotetexts
\newcommand\CMT[1]\null% Intra-Line Comments à la \CMT{this is a comment} https://www.reddit.com/r/LaTeX/comments/1cmq3v/intraline_comments_comment_without_commenting_out/

% Symbole für Stern- und Dreieckschaltung
\newcommand{\Stern}{{\ifmmode%
		\text{\tikz \draw[] (0,0) -- (90:.9ex) -- (0,0) -- (-45:1.1ex) -- (0,0) -- (225:1.1ex);}%
		\else%
		\tikz \draw[] (0,0) -- (90:.9ex) -- (0,0) -- (-45:1.1ex) -- (0,0) -- (225:1.1ex);%
		\fi}}
\newcommand{\Dreieck}{{\ifmmode%
		\text{\tikz \draw[] (0,0) -- (65:1.5ex) --++ (-65:1.5ex) --cycle;}%
		\else%
		\tikz \draw[] (0,0) -- (65:1.5ex) --++ (-65:1.5ex) --cycle;%
		\fi}}

% Farbige Spannungs und Strompfeile
% Ref: Circuitikz Manual, Kapitel Advanced voltages, currents and flows https://texdoc.org/serve/circuitikz/0#subsection.5.8 (v. 1.6.7)
\ctikzset{bipole voltage style/.style={color=voltage}} % voltage label color
\ctikzset{bipole current style/.style={color=red}} 	% current label color
\ctikzset{!vi/.style={no v symbols, no i symbols}}	% add option for argument "!vi" to remove voltage and current arrows
\ctikzset{!v/.style={no v symbols}}	% add option for argument "!v" to remove voltage arrows
\ctikzset{!i/.style={no i symbols}}	% add option for argument "!i" to remove current arrows

\newcommand{\varronly}[1]{% {node}
	\draw [color=voltage] (#1-Vfrom) .. controls (#1-Vcont1) and (#1-Vcont2) .. (#1-Vto) node [currarrow, sloped, anchor=tip, allow upside down, pos=1]{} % arrow
}
\newcommand{\iarronly}[1]{% {node}
	%\draw [red, -] (#1-Ifrom) -- (#1-Ipos); % arrow shaft (optional)
	\node [color=red, currarrow, anchor=center,	rotate=\ctikzgetdirection{#1-Iarrow}] at (#1-Ipos) {} % arrow tip
}

% modified versions of above commands to draw arrows and labels together in one command with optional color argument
\newcommand{\varrmore}[3][voltage]{% [color]{node}{label}
    \draw[color=#1] (#2-Vfrom) .. controls (#2-Vcont1) and (#2-Vcont2) .. (#2-Vto) node [currarrow, sloped, anchor=tip, allow upside down, pos=1]{}; % arrow
    \node[color=#1, anchor=\ctikzgetanchor{#2}{Vlab}, inner sep=2pt] at (#2-Vlab) {#3} % label
	% note: label offset [...,inner sep=2pt] copied from internal macro definition, s. /usr/share/texlive/texmf-dist/tex/latex/circuitikz/circuitikz-1.2.7-body.tex line 23009)
}
\newcommand{\iarrmore}[3][red]{% [color]{node}{label}
	%\draw[#1,-] (#2-Ifrom) -- (#2-Ipos); % arrow shaft (optional)
	\node[color=#1, currarrow, anchor=center, rotate=\ctikzgetdirection{#2-Iarrow}] at (#2-Ipos) {}; % arrow tip
	\node[color=#1, anchor=\ctikzgetanchor{#2}{Ilab}] at (#2-Ipos) {#3} % label
}



% logarithmische Achsen für den Bode-Plot
\pgfplotsset{
	compat=1.18,
	log x ticks with fixed point/.style={
		xticklabel={
			\pgfkeys{/pgf/fpu=true}
			\pgfmathparse{exp(\tick)}%
			\pgfmathprintnumber[fixed relative, precision=3]{\pgfmathresult}
			\pgfkeys{/pgf/fpu=false}
		}
	},
	log y ticks with fixed point/.style={
		yticklabel={
			\pgfkeys{/pgf/fpu=true}{\tiny }
			\pgfmathparse{exp(\tick)}%
			\pgfmathprintnumber[fixed relative, precision=3]{\pgfmathresult}
			\pgfkeys{/pgf/fpu=false}
		}
	}
}


% Für minipage, weil \columns nur in Beamer funktioniert
\newlength\Colsep
\setlength\Colsep{10pt}

% Für hf-tikz, Offsets to adjust vertical bounds highlight boxes for typical equation sizes
%\usepackage{xparse} % needed for hf-tikz
\def\hfoffupper{0.3} 	% e.g. for 1
\def\hfoffuppera{0.45} 	% e.g. for \frac{1}{1}
\def\hfoffupperb{0.7}	% e.g. for \frac{\frac{\int^1_0 11}{\int^1_0 12}}{\frac{\int^1_0 21}{\int^1_0 22}}
\def\hfoffupperc{0.9}
\def\hfofflower{-0.12}	% analog zu upper offset
\def\hfofflowera{-0.21}
\def\hfofflowerb{-0.55}
\def\hfofflowerc{-0.7}

%Farbdefinitionen
\definecolor{GETgreen}{RGB}{25,185,145}				%Titelstreifen der Folien
\definecolor{leiter}{RGB}{207,163,118}				%elektrischer Leiter
\definecolor{magnetfeld}{RGB}{255,128,0}			%magnetisches Feld
\definecolor{flaeche}{RGB}{121,179,205}				%Querschnittsfläche
\definecolor{voltage}{RGB}{26, 91, 144}				%Spannungspfeile
\definecolor{spannung}{RGB}{26, 91, 144}			%farbe für spannung definieren
\definecolor{current}{RGB}{200, 0, 0}				%Strompfeile
\definecolor{strom}{RGB}{200, 0, 0}					%farbe für strom definieren
\definecolor{durchflutung}{RGB}{25, 210, 220}		%magnetiche Durchflutung Theta
\definecolor{Red}{rgb}{255, 0, 0}					%Reines rot z.B. zum durchstreichen von Elementen die sich wegkürzen
\definecolor{plotgreen}{RGB}{0,200,0}				% green
\definecolor{green1}{RGB}{0,200,0}					% green
\definecolor{magnetfeld}{RGB}{255,128,0}
\definecolor{statorw1}{RGB}{230,103,35}
\definecolor{statorw2}{RGB}{169,82,0}
\definecolor{statorw3}{RGB}{215,25,0}

%Farbkommandos für Gleichungen (equation color)
% usage: \ec[optional color]{mathsymbols} for any color or e.g. \ecg{mathsymbols} for green mathsymbols
% ref: https://tex.stackexchange.com/questions/21598/how-to-color-math-symbols
\newcommand{\ec}[2][black]{{\color{#1}#2}}	% set text/equation coloring, default black
\newcommand{\eci}[1]{{\color{red}#1}}		% current color
\newcommand{\ecv}[1]{{\color{voltage}#1}} 	% voltage color
\newcommand{\ecr}[1]{{\color{red}#1}} 		% red color
\newcommand{\ecg}[1]{{\color{green}#1}} 	% green color
\newcommand{\ecb}[1]{{\color{blue}#1}} 		% blue color
\newcommand{\ecp}[1]{{\color{purple}#1}} 	% purple color
\newcommand{\ecy}[1]{{\color{yellow}#1}} 	% yellow color
\newcommand{\ecc}[1]{{\color{cyan}#1}} 		% cyan color
\newcommand{\eco}[1]{{\color{orange}#1}} 	% orange color

%Lernziel
\newtcolorbox{Lernziele}[1]{enhanced,
	before skip=1cm,
	after skip=1cm,
	colframe=GETgreen,
	colbacktitle=GETgreen,
	colback=white!95!black,
	boxrule=0.5mm,
	title={Lernziele: #1},
	attach boxed title to top left={xshift=1.14cm,yshift*=-\tcboxedtitleheight/2}, varwidth boxed title*=-3cm,
	boxed title style={
		frame code={
			\path[left color=tcbcolback,right color=tcbcolback,
				middle color=tcbcolback]
				([xshift=1.6mm]frame.north west)[rounded corners=0.5mm]
				-- ([xshift=1.6mm]frame.north east)
				-- ([xshift=-1mm]frame.south east)
				-- ([xshift=-1mm]frame.south west)
				-- cycle;
			\path[left color=tcbcolback,right color=tcbcolback,
				middle color=tcbcolback]
				([xshift=-3mm]frame.north west)[rounded corners=0.5mm]
				-- ([xshift=0.3mm]frame.north west)
				-- ([xshift=-2.2mm]frame.south west)
				-- ([xshift=-5.5mm]frame.south west)
				-- cycle;
			\path[left color=tcbcolback,right color=tcbcolback,
				middle color=tcbcolback]
				([xshift=-7.6mm]frame.north west)[rounded corners=0.5mm]
				-- ([xshift=-4.3mm]frame.north west)
				-- ([xshift=-6.8mm]frame.south west)
				-- ([xshift=-10.1mm]frame.south west)
				-- cycle;
		},interior engine=empty,
	},fonttitle=\bfseries
}

%Merksatz
\newtcolorbox{Merksatz}[1]{enhanced,
	before skip=1cm,
	after skip=1cm,
	colframe=GETgreen,
	colbacktitle=GETgreen,
	colback=GETgreen!5!white,
	boxrule=0.5mm,
	title={Merke: #1},
	attach boxed title to top left={xshift=1.14cm,yshift*=-\tcboxedtitleheight/2}, varwidth boxed title*=-3cm,
	boxed title style={
		frame code={
			\path[left color=tcbcolback,right color=tcbcolback,
				middle color=tcbcolback]
				([xshift=1.6mm]frame.north west)[rounded corners=0.5mm]
				-- ([xshift=1.6mm]frame.north east)
				-- ([xshift=-1mm]frame.south east)
				-- ([xshift=-1mm]frame.south west)
				-- cycle;
			\path[left color=tcbcolback,right color=tcbcolback,
				middle color=tcbcolback]
				([xshift=-3mm]frame.north west)[rounded corners=0.5mm]
				-- ([xshift=0.3mm]frame.north west)
				-- ([xshift=-2.2mm]frame.south west)
				-- ([xshift=-5.5mm]frame.south west)
				-- cycle;
			\path[left color=tcbcolback,right color=tcbcolback,
				middle color=tcbcolback]
				([xshift=-7.6mm]frame.north west)[rounded corners=0.5mm]
				-- ([xshift=-4.3mm]frame.north west)
				-- ([xshift=-6.8mm]frame.south west)
				-- ([xshift=-10.1mm]frame.south west)
				-- cycle;
		},interior engine=empty,
	},fonttitle=\bfseries
}


% logarithmische Achsen für den Bode-Plot
\pgfplotsset{
	compat=1.18,
	log x ticks with fixed point/.style={
		xticklabel={
			\pgfkeys{/pgf/fpu=true}
			\pgfmathparse{exp(\tick)}%
			\pgfmathprintnumber[fixed relative, precision=3]{\pgfmathresult}
			\pgfkeys{/pgf/fpu=false}
		}
	},
	log y ticks with fixed point/.style={
		yticklabel={
			\pgfkeys{/pgf/fpu=true}{\tiny }
			\pgfmathparse{exp(\tick)}%
			\pgfmathprintnumber[fixed relative, precision=3]{\pgfmathresult}
			\pgfkeys{/pgf/fpu=false}
		}
	}
}


% Farbige Spannungs und Strompfeile
% Ref: Circuitikz Manual, Kapitel Advanced voltages, currents and flows https://texdoc.org/serve/circuitikz/0#subsection.5.8 (v. 1.6.7)
\ctikzset{!vi/.style={no v symbols, no i symbols}}  		% add option for argument "!vi" to remove voltage and current arrows

%Tabellenbefehle
\usepackage{ragged2e}
\newcolumntype{L}[1]{>{\raggedright\arraybackslash}p{#1}}
\newcolumntype{C}[1]{>{\centering\arraybackslash}p{#1}}
\newcolumntype{R}[1]{>{\raggedleft\arraybackslash}p{#1}}
\newcolumntype{J}[1]{>{\justifying\arraybackslash}p{#1}}

% Bibliographie
\usepackage[
	backend=biber,
	sorting=none,% Sortierung nach Reihenfolge der Zitate
]{biblatex} %biblatex mit biber laden
\addbibresource{Templates/Literatur.bib}
\usepackage[nottoc,notlot,notlof]{tocbibind}	%% Literaturverzeichnis im TOC
	 		% Allgemeine Settings einbinden
% Pakete nur für das Skript
\usepackage{fancyhdr}      		% Unterstrichene Kopfzeilen: selbstdefinierbar
\usepackage{sanitize-umlaut}	   % Umlaute im Index
\usepackage[pdftex,bookmarks,bookmarksopen=false,colorlinks,bookmarksnumbered]{hyperref} % Links im PDF anklickbar
\hypersetup{% \ref Links im Skript umfärben
    ,urlcolor=blue
    ,citecolor=blue
    ,linkcolor=blue
    }
\usepackage{hf-tikz}		                        % Highlighting Text/Math, Ref: https://tex.stackexchange.com/questions/52598/beamer-highlighting-aligned-math-with-overlay
% example1: \tikzmarkin<2->[above offset={\hfoffuppera},below offset={\hfofflowera},blend mode=multiply]{a} R = \frac{U}{I} \tikzmarkend{a}
% example2: \tikzmarkin<2-3>[above left offset={0,\hfoffuppera}, below right offset={0,\hfofflowera}]{markerid} \int^{\pi}_{0}f(x)\d x \tikzmarkend{markerid}
% option: blend mode=multiply (to not overshadow stuff behind the highlight box) and (in beamer mode) <n>, <n->, <n-m> for overlay (ignored in article)

\usepackage{enumitem} % Aufzählungen nummerisch oder alphabetisch
\usepackage{todonotes}	%Anmerkungen und to-dos erstellen

% Symbolverzeichnis
\makenomenclature
\makeindex                      % Indizierung aktiv

% Seitengrößen
\topmargin-.7cm                 % oberen Rand verkürzen
\textheight23.3cm               % Kerntextbereich vergrößern
\textwidth15.35cm               % Bereichsverbreiterung
\oddsidemargin0.5cm             % Den linken Rand verschmälern
\evensidemargin0.5cm            % Den linken Rand verschmälern (gerade S.)
\parindent0em                   % erste Zeile eines Absatzes nicht einrücken
\parskip1.5ex                   % Absatzabstand

% Seitentitel
\fancyhead{}
\fancyhead[EL,OR]{\thepage}% E - Even, O - Odd, R - Right, L - Left, C - Center
\fancyfoot{}
\newcommand{\sectionversion}{\tiny{Stand: \today}}
\fancyfoot[EL,OR]{}

\newcommand{\setheadempty}{\fancyhead[ER,OL]{}} % Seitennummerierung im Kopf
% Deklaration des Dokumentenkopfes:
\newcommand{\sectiontitle}{}
\newcommand{\newsection}[1]{\section{#1}\renewcommand{\sectiontitle}{#1}}

\newcommand{\setheadleftmark}{\fancyhead[ER,OL]{\leftmark}}
% Deklaration des Anhangs:
\newcommand{\setheadanhang}{
	\fancyhead[OL]{Anhang \thesection : \sectiontitle}
}
\newcommand{\setheadkapitel}{
	\fancyhead[OL]{\textsc{Kapitel \thesection : \sectiontitle}}
   \fancyhead[ER]{GET it digital Modul \GetItDigitalModulnumber : \GetItDigitalModulname}
	\fancyfoot[EL,OR]{}
}



% eigene Befehle
\renewcommand{\d}{\mathrm{d}}	%für das Differential d/dt, überschreibt \d für "Punkt unter"
\newcommand{\s}[1]{#1} % skript only
\renewcommand{\b}[1]{} % beamer only
\renewcommand{\v}[1]{} % Video only
\newcommand{\ftx}[1]{}
\newcommand{\fta}[1]{\FloatBarrier\section{#1}}
\newcommand{\ftb}[1]{\subsection{#1}}
\newcommand{\ftc}[1]{\subsubsection{#1}}

\newcommand{\f}[4]{\begin{figure}[H]\centering\includegraphics[#1]{#3}\caption{#4}\end{figure}}
\newcommand{\fu}[2]{\begin{figure}[H]\centering{#1}\caption{#2}\end{figure}}
\newcommand{\fo}[5]{\begin{figure}[H]\centering\begin{overpic}[#1]{#3}{#4}\end{overpic}\caption{#5}\end{figure}}
\newcommand{\foo}[5]{\begin{figure}[H]\begin{overpic}[#1]{#3}{#4}\end{overpic}\end{figure}}
\newcommand{\resizeboxb}[3]{#3} % resizebox but only apply in beamer (width, height, content)
\newcommand{\resizeboxsb}[3]{\resizebox{#1\textwidth}{!}{#3}} % #1: width skript, #2: width beamer, #3: content e.g. \resizeboxx{0.8}{1}{stuff}
\newenvironment{eq}{\begin{equation}}{\end{equation}}
\newenvironment{eqa}{\align}{\endalign}

%Übungsaufgaben
\newcommand{\Aufgabe}[1]{}
\renewcommand{\Loesung}[1]{}

%Beispiel
\newtcbtheorem[number within=section]{bsp}{Beispiel}{enhanced,
	before skip=1cm,
	after skip=1cm,
	colframe=GETgreen,
	colbacktitle=GETgreen,
	colback=white,
	boxrule=0.5mm,
	attach boxed title to top left={xshift=1.14cm,yshift*=-\tcboxedtitleheight/2}, varwidth boxed title*=-3cm,
	boxed title style={frame code={
					\path[left color=tcbcolback,right color=tcbcolback,
					middle color=tcbcolback]
					([xshift=1.6mm]frame.north west)[rounded corners=0.5mm]
					-- ([xshift=1.6mm]frame.north east)
					-- ([xshift=-1mm]frame.south east)
					-- ([xshift=-1mm]frame.south west)
					-- cycle;
					\path[left color=tcbcolback,right color=tcbcolback,
					middle color=tcbcolback]
					([xshift=-3mm]frame.north west)[rounded corners=0.5mm]
					-- ([xshift=0.3mm]frame.north west)
					-- ([xshift=-2.2mm]frame.south west)
					-- ([xshift=-5.5mm]frame.south west)
					-- cycle;
					\path[left color=tcbcolback,right color=tcbcolback,
					middle color=tcbcolback]
					([xshift=-7.6mm]frame.north west)[rounded corners=0.5mm]
					-- ([xshift=-4.3mm]frame.north west)
					-- ([xshift=-6.8mm]frame.south west)
					-- ([xshift=-10.1mm]frame.south west)
					-- cycle;
				},interior engine=empty,
		},
	fonttitle=\bfseries,
	title={#2},#1,breakable}{bsp}

%Highlighting
\newcommandx*\highlight[5][3=\hfofflower, 4=\hfoffupper,5=2]{\tikzmarkin[blend mode=multiply]{#1}(0,#3)(0,#4)#2\tikzmarkend{#1}}
% usage: \highlight{markerid}{markercontent}[offset below][offset upper][beamer overlay]
% Beispiele: (mit prädefinierten upper und lower offset values aus Settings.tex)
% $\highlight{id}{f(t)=\sin(\omega t)}$\quad
% $\highlight{ida}{\frac{1}{2}=\int f(t) \d t}[\hfofflowera][\hfoffuppera]$\quad
% $\highlight{idb}{\frac{\frac{\int^1_0 11}{\int^1_0 12}}{\frac{\int^1_0 21}{\int^1_0 22}}}[\hfofflowerb][\hfoffupperb][2-4]$



\pagenumbering{roman}
\setcounter{secnumdepth}{3}

% Fett-Text bei Aufzählungen
\setlist[enumerate]{font=\bfseries}
\setlist[itemize]{font=\bfseries}


% Das Ende der Seiten muss nicht immer auf gleicher Höhe sein
\raggedbottom

% Nummerierung von Formeln und Bilder
\numberwithin{figure}{section}
\numberwithin{equation}{section}
\numberwithin{table}{section}

% Symbolverzeichnis
\renewcommand\nomgroup[1]{%
	\item[\bfseries
	\ifstrequal{#1}{F}{Formelzeichen}{%
		\ifstrequal{#1}{C}{Konstanten}{%
			\ifstrequal{#1}{E}{Einheiten}{}}}%
	]}
\newcommand{\nomunit}[1]{%
	\renewcommand{\nomentryend}{\hspace*{\fill}#1}}
\renewcommand{\nomname}{Verwendete Größen und Formelzeichen}

% Bibliographie
\usepackage[
	backend=biber,
	sorting=none,% Sortierung nach Reihenfolge der Zitate
]{biblatex} %biblatex mit biber laden
\addbibresource{Templates/Literatur.bib}
\usepackage[nottoc,notlot,notlof]{tocbibind}	%% Literaturverzeichnis im TOC
	 	% Allgemeine Foliensettings einbinden
\usepackage[disabled]{../Video/beamervideo}  % Videoerstellung deaktiviert


\newcommand{\GetItDigitalModulnumber}{0}
\usepackage{import}				% relatives input

%PDF Eigenschaften
\hypersetup{
	pdftitle={GET it digital},
	pdfauthor={verschiedene Autoren}
}

\graphicspath{
	{Templates/Bilder/}
	{modul-01-elektrische-grundgroessen/Bilder/}
	{modul-02-energie-und-leistung/Bilder/}
	{modul-03-elektrische-bauelemente/Bilder/}
	{modul-04-grundlegende-gleichstromnetzwerke/Bilder/}
	{modul-05-erweiterte-gleichstromnetzwerke/Bilder/}
	{modul-06-magnetische-groessen/Bilder/}
	{modul-07-periodische-groessen/Bilder/}
	{modul-08-schaltungen-variabler-frequenz/Bilder/}
	{modul-09-halbleiterbauelemente/Bilder/}
	{modul-10-operationsverstaerker/Bilder/}
	{modul-11-elektrische-maschinen/Bilder/}
	{modul-12-schaltvorgaenge/Bilder/}
}

\institute{GET it digital}
\date{\today}


\begin{document}
\renewcommand{\partname}{Modul}
\thispagestyle{empty}
\begin{center}

	\begin{minipage}[c][2cm][c]{0.3\textwidth}
		\includegraphics[angle=270,width=\linewidth]{FHAachen-logo2010}
	\end{minipage}
	\hfill
	\begin{minipage}[c][2cm][c]{0.3\textwidth}
		\includegraphics[width=\linewidth]{Logo_Uni_Paderborn}
	\end{minipage}
	\hfill
	\begin{minipage}[c][2cm][c]{0.3\textwidth}
		\includegraphics[width=\linewidth]{BUW_Logo-weiss-auf-gruen-rgb}
	\end{minipage}
	\\
	\begin{minipage}[c][2cm][c]{0.3\textwidth}
		\includegraphics[width=\linewidth]{Fachhochschule_Südwestfalen_logo}
	\end{minipage}
	\hfill
	\begin{minipage}[c][2cm][c]{0.3\textwidth}
		\includegraphics[width=\linewidth]{logo-hrw}
	\end{minipage}
	\hfill
	\begin{minipage}[c][2cm][c]{0.3\textwidth}
		\begin{center}
			\includegraphics[width=0.7\linewidth]{FH_Dortmund-logo}
		\end{center}
	\end{minipage}

	\vfill
	%\vspace{23ex}
	{\large GET it digital} \\[10ex]
	{\large Grundlagen der Elektrotechnik}\\[10ex]
	\ \\
\end{center}
\vfill

\begin{minipage}[t]{0.36\textwidth}
	Ein Kooperationsvorhaben \\
	empfohlen durch die\\[0.3cm]
	\includegraphics[width=\linewidth]{dh_nrw.jpg}
\end{minipage}
\hfill
\begin{minipage}[t]{0.5\textwidth}
	gefördert durch\\[1cm]
	\includegraphics[width=\linewidth]{mkw.jpg}
\end{minipage}



\newpage
\pagestyle{fancyplain}

Die einzelnen Module wurden erstellt von:
\begin{itemize}
	\item Modul 1 und 4: Universität Paderborn, Henrik Bode
	\item Modul 2 und 3: Fachhochschule Aachen, Michael Hillgärtner, Sven Micun, Filimon Stergianos
	\item Modul 5: Fachhochschule Dortmund, Martin Kiel, Torben Meibeck
	\item Modul 6 und 11: Bergische Universität Wuppertal, Ralf Wegener, Grzegorz Lisicki, Josef Kirschner
	\item Modul 7: Fachhochschule Dortmund, Stefan Kempen, Torben Meibeck
	\item Modul 8 und 12: Fachhochschule Südwestfalen, Matthias Werle
	\item Modul 9 und 10: Hochschule Ruhr West, Marvin Kaminski, Kerstin Siebert, Jonas Brodmann, Dominik Thiem
\end{itemize}

\vfill
{\hfill Stand: \today}\\
{\includegraphics[width=0.2\linewidth]{by.png}}\\
Weiternutzung als OER ausdrücklich erlaubt: Dieses Werk und dessen Inhalte sind lizenziert unter CC BY 4.0. Ausgenommen von der Lizenz
sind die verwendeten Logos sowie alle anders gekennzeichneten Elemente. Nennung gemäß \href{https://open-educational-resources.de/oer-tullu-regel/}{TULLU-Regel} bitte wie folgt: \glqq GET it digital \grqq\ von verschiedenen Autoren \\ Lizenz: CC BY 4.0.

Der Lizenzvertrag ist hier abrufbar:\\
\url{https://creativecommons.org/licenses/by/4.0/deed.de}\\
Das Werk ist online verfügbar unter:\\
\url{https://getitdigital.uni-wuppertal.de/module/komplett/Skript.pdf}
\newpage

{\setlength{\parskip}{0.05ex}    %Verringerter Zeilenabstand für Verzeichnisse
	\tableofcontents
	\newpage
}

\printnomenclature
\cleardoublepage

\pagestyle{fancyplain}
\pagenumbering{arabic}
\fancyhead[OL]{\textsc{Kapitel \thesection : \sectiontitle}}
\fancyhead[ER]{GET it digital}
\fancyfoot[EL,OR]{}

\cleardoublepage
\part{Elektrische Grundgrößen}
\import{modul-01-elektrische-grundgroessen/}{Inhalt}

\cleardoublepage
\part{Energie und Leistung}
\import{modul-02-energie-und-leistung/}{Inhalt}

\cleardoublepage
\part{Elektrische Bauelemente}
\import{modul-03-elektrische-bauelemente/}{Inhalt}

\cleardoublepage
\part{Grundlegende Gleichstromnetzwerke}
\import{modul-04-grundlegende-gleichstromnetzwerke/}{Inhalt}

\cleardoublepage
\part{Erweiterte Gleichstromnetzwerke}
%\import{modul-05-erweiterte-gleichstromnetzwerke/}{Inhalt}

\cleardoublepage
\part{Magnetische Größen}
\import{modul-06-magnetische-groessen/}{Inhalt}

\cleardoublepage
\part{Periodische Größen}
%\import{modul-07-periodische-groessen/}{Inhalt}

\cleardoublepage
\part{Schaltungen variabler Frequenz}
%\import{modul-08-schaltungen-variabler-frequenz/}{Inhalt}

\cleardoublepage
\part{Halbleiterbauelemente}
%\import{modul-09-halbleiterbauelemente/}{Inhalt}

\cleardoublepage
\part{Operationsverstärker}
%\import{modul-10-operationsverstaerker/}{Inhalt}

\cleardoublepage
\part{Elektrische Maschinen}
\import{modul-11-elektrische-maschinen/}{Inhalt}

\cleardoublepage
\part{Schaltvorgänge}
%\import{modul-12-schaltvorgaenge/}{Inhalt}


\cleardoublepage
\appendix
\setheadanhang

%\section{Übungsaufgaben}
%\renewcommand{\Aufgabe}[1]{#1}
%\renewcommand{\Loesung}[1]{}
%\subsection{Beschreibung des Bändermodells\label{Aufg1} }
% Alt: Aufgabe 1
\Aufgabe{
  Beschreiben Sie ausführlich das Bändermodell von Festkörpern und seine Bedeutung für die Unterscheidung zwischen 
  Leiter, Halbleiter und Isolator. Gehen Sie dabei insbesondere auf die folgenden Aspekte ein:
  \begin{itemize}
    \item relevante Bänder und deren Besetzung mit Ladungsträgern.
    \item Bedeutung der Bandlücke und wie diese elektrische Eingeschaften von Materialien beeinflusst.
    \item Einfluss von Temperaturänderungen auf die Leitfähigkeit.
\end{itemize}
% \speech{Die Abbildung zeigt **Aufgabe 1**, die eine ausführliche Beschreibung des **Bändermodells von Festkörpern** erfordert. Dieses Modell ist entscheidend für das Verständnis der elektrischen Eigenschaften von Materialien und ihrer Unterscheidung in **Leiter, Halbleiter und Isolatoren**.  

% ### **Aufgabenstellung**
% Die Aufgabe fordert eine **detaillierte Erläuterung** des Bändermodells unter Berücksichtigung der folgenden Aspekte:

% 1. **Relevante Bänder und deren Besetzung mit Ladungsträgern**  
%    - In Festkörpern sind die **Elektronenenergiezustände** in **Bänder** aufgeteilt:  
%      - **Valenzband**: Enthält die Elektronen, die an chemischen Bindungen beteiligt sind.  
%      - **Leitungsband**: Hier können sich freie Elektronen bewegen und elektrische Leitung ermöglichen.  
%    - Die **Verfügbarkeit von Elektronen im Leitungsband** bestimmt die **Leitfähigkeit des Materials**.  
%    - Bei einem **Leiter (z. B. Metalle)** überlappen Valenz- und Leitungsband, wodurch **freie Ladungsträger ständig vorhanden sind**.  
%    - Bei einem **Halbleiter (z. B. Silizium)** gibt es eine **kleine Bandlücke**, die mit Temperatur oder Dotierung überbrückt werden kann.  
%    - Bei einem **Isolator (z. B. Glas)** ist die Bandlücke sehr groß, wodurch **fast keine Leitung möglich ist**.  

% 2. **Bedeutung der Bandlücke und ihr Einfluss auf elektrische Eigenschaften**  
%    - Die **Bandlücke (\( E_G \))** ist der Energiebereich zwischen Valenz- und Leitungsband.  
%    - Sie bestimmt, ob ein Material **ein Leiter, Halbleiter oder Isolator** ist:  
%      - **Leiter**: \( E_G = 0 \), kontinuierliche elektronische Zustände ermöglichen Leitung.  
%      - **Halbleiter**: \( 0 < E_G < 3eV \), Ladungsträger können durch thermische Anregung ins Leitungsband übergehen.  
%      - **Isolator**: \( E_G > 3eV \), sehr wenige Elektronen im Leitungsband, daher fast keine Leitfähigkeit.  
%    - Die Größe der Bandlücke beeinflusst auch **optische Eigenschaften**:  
%      - **Halbleiter mit kleiner Bandlücke** absorbieren und emittieren Licht (z. B. LEDs, Laser).  
%      - **Isolatoren mit großer Bandlücke** sind meist transparente Materialien (z. B. Quarzglas).  

% 3. **Einfluss von Temperaturänderungen auf die Leitfähigkeit**  
%    - **Mit steigender Temperatur steigt die Elektronenenergie**, wodurch mehr Elektronen die Bandlücke überwinden.  
%    - In **Halbleitern nimmt die Leitfähigkeit mit steigender Temperatur zu**, da mehr Elektronen ins Leitungsband übergehen.  
%    - In **Leitern nimmt die Leitfähigkeit mit steigender Temperatur ab**, da die Bewegung der Ladungsträger durch **Gitterstreuung behindert wird**.  
%    - **Isolatoren zeigen kaum Veränderung**, da ihre Bandlücke zu groß ist.  

% ### **Zusammenfassung der Aufgabe**
% - Das **Bändermodell** erklärt, warum manche Materialien **Strom leiten** und andere nicht.  
% - Die **Bandlücke** ist ein entscheidender Faktor für die **elektrischen und optischen Eigenschaften**.  
% - **Temperaturabhängigkeit** ist für Halbleitertechnologien besonders wichtig.  

% Diese Aufgabe fordert eine umfassende Erklärung der **physikalischen Grundlagen des Bändermodells** sowie dessen Auswirkungen auf **die Klassifikation und das Verhalten von Materialien**.}

}


\Loesung{
    Das Bändermodell erklärt die Energiezustände, die Elektronen in Festkörpern einnehmen können. 
    In einem einzelnen Atom sind die Elektronenzustände diskrete Energielevel.
    Siehe: Abschnitt \ref{sec:Bändermodell}

%     \speech{ 
% Das Bändermodell beschreibt die möglichen Energiezustände, die Elektronen in einem Festkörper einnehmen können. In einem einzelnen Atom sind die Energiezustände der Elektronen diskrete Energieniveaus, also genau bestimmte Werte.  

% In einem Festkörper, der aus vielen Atomen besteht, überlappen diese einzelnen Energieniveaus und bilden sogenannte Energie**bänder**. Die beiden wichtigsten Bänder sind:  

% 1. **Valenzband**: Das energetisch höchste besetzte Band bei niedrigen Temperaturen. Elektronen in diesem Band sind an ihre Atome gebunden und können sich nicht frei bewegen.  
% 2. **Leitungsband**: Ein energetisch höheres Band, in dem sich Elektronen frei bewegen können. Elektronen in diesem Band tragen zum elektrischen Stromfluss bei.  

% Zwischen diesen beiden Bändern befindet sich die **Bandlücke**, auch **Energielücke** genannt. Sie gibt an, wie viel Energie benötigt wird, um ein Elektron vom Valenzband in das Leitungsband zu heben. Die Größe der Bandlücke bestimmt, ob ein Material ein Leiter, Halbleiter oder Isolator ist:  

% - **Leiter** (z. B. Metalle): Keine oder sehr kleine Bandlücke, sodass Elektronen leicht ins Leitungsband übergehen können.  
% - **Halbleiter** (z. B. Silizium, Germanium): Mittlere Bandlücke, Elektronen können durch äußere Energie (z. B. Wärme, Licht) ins Leitungsband angehoben werden.  
% - **Isolatoren** (z. B. Glas, Keramik): Große Bandlücke, Elektronen können nur mit extrem hoher Energie ins Leitungsband gelangen.  

% Das Bändermodell ist essenziell für das Verständnis der elektrischen Eigenschaften von Materialien und die Funktionsweise von Halbleiterbauelementen wie Dioden und Transistoren. 
% }

  }




%{
    %\fta{Lösungen}
    %Lösungen für Beispielaufgaben 4.10
    %\textbf{Aufgabe 1}
    %\newline
    %Spannung über $R_\mathrm{V}$ ($U_\mathrm{V}$): \[ U_\mathrm{V} = U_\mathrm{0} - U_\mathrm{D} = 5\mathrm{V} - 2,2 \mathrm{V} = 2,8\mathrm{V}\]
    %\newline
    %Vorwiderstand:
    %\[
    %  R_\mathrm{V} = \frac{2,8\mathrm{V}}{30\mathrm{mA}} = 93,3\Omega 
    %\]

    %\textbf{Aufgabe 2}
    %\newline
   %a) 
   %\newline
   %\[ 
   %  \frac{5\mathrm{V}}{2\mathrm{k \Omega} + 0,3\mathrm{k \Omega}} = \frac{U_1}{300 \Omega}
   %\]
    % \newline
   %\[
   %  U_1 = 300 \Omega \cdot \frac{5 \mathrm{V}}{2\mathrm{k \Omega} + 0,3\mathrm{k \Omega}} = 0,65\mathrm{V}
   %\]
    % \newline
   %\[
   %  5 \mathrm{V} = 0,65 \mathrm{V} + U_\mathrm{RV} + 0,6 \mathrm{V}
   %\]
   % \newline
   %\[
   %  U_\mathrm{RV}= 5 \mathrm{V} - 0,65 \mathrm{V} - 0,6 \mathrm{V} = 3,75 \mathrm{V}
   %\]
   %\newline 
   %\[
   % R_\mathrm{V} = \frac{3,75 \mathrm{V}}{50 \mathrm{\mu A}} = 75 \mathrm{k \Omega}
   %\]  
   %\newline
   %b)
   %\newline
   %\[ 
   % I_\mathrm{C} = \frac{15 \mathrm{V} - 7,5\mathrm{V}}{500 \Omega} = 15 \mathrm{mA}
   %\]
   %\newline
   %\[
   % I_\mathrm{E} = I_\mathrm{C} + I_\mathrm{B} = 15 \mathrm{mA} + 50 \mathrm{\mu A} = 15,05 \mathrm{mA}
   %\]

%}
\subsection{Die Raumladungszone im pn-Übergang\label{Aufg2}}
% Alt: Aufgabe 2
\Aufgabe{
  Untersuchen Sie die Eigenschaften und die Bildung der Raumladungszone in pn-Übergängen von Halbleitern. 
  Beantworten Sie die folgenden Aspekte in Ihrer Ausarbeitung:
  \begin{itemize}
    \item Was ist die Raumladungszone und wie entsteht diese im pn-Übergang?
    \item Wie ist der Verlauf des elektrischen Feldes im Halbleiter?
    \item Welche inneren Vorgänge wirken auf die freien Ladungsträger?
    \item Wie verändert sich die Raumladungszone bei einer angelegten Spannung in Durchlass- und Sperrrichtung?

    \end{itemize}
%     \speech{Untersuchen Sie die **Eigenschaften und die Bildung der Raumladungszone** in **pn-Übergängen** von Halbleitern.  

% Beantworten Sie die folgenden Aspekte in Ihrer Ausarbeitung:  
% - **Was ist die Raumladungszone** und wie entsteht diese im **pn-Übergang**?  
% - **Wie ist der Verlauf des elektrischen Feldes** im Halbleiter?  
% - **Welche inneren Vorgänge** wirken auf die freien Ladungsträger?  
% - **Wie verändert sich die Raumladungszone** bei einer **angelegten Spannung** in **Durchlass- und Sperrrichtung**?}


}


\Loesung{
  Die RLZ ist eine schmale 
  Region um den pn-Übergang herum, in der positive Ionen aus der n-Schicht und negative Ionen aus der p-Schicht verbleiben, 
  nachdem die Diffusion abgeschlossen ist. In dieser Zone gibt es keine freien Ladungsträger, da die positiven und negativen Ladungen sich 
  gegenseitig neutralisieren. Dadurch entsteht ein elektrisches Feld ($\vec{E}$), das sowohl einen Driftstrom erzeugt als auch die Diffusion von weiteren 
  Ladungsträgern unterdrückt. Die Raumladungszone wirkt wie eine Sperrschicht und verhindert den Stromfluss in Sperrrichtung.
  Siehe: Abschnitt \ref{sec:pn-Übergang}

%  \speech{
% Die Raumladungszone (RLZ) ist eine schmale Region im Bereich des **pn-Übergangs** eines Halbleiters. Sie entsteht, wenn sich die freien Ladungsträger (Elektronen und Löcher) durch **Diffusion** über die Grenze zwischen der p- und n-Schicht hinweg bewegen.  

%  Aufbau der Raumladungszone:
% - In der **n-Schicht** verbleiben **positive Ionen**, weil die beweglichen Elektronen in die p-Schicht diffundiert sind.  
% - In der **p-Schicht** verbleiben **negative Ionen**, da die beweglichen Löcher in die n-Schicht gewandert sind.  

% Diese fixierten Ionen neutralisieren sich **gegenseitig**, wodurch die Raumladungszone frei von beweglichen Ladungsträgern ist.

%  Wirkung der Raumladungszone:
% - Es bildet sich ein **elektrisches Feld**, das die Bewegung weiterer Ladungsträger hemmt.  
% - Ein dadurch entstehender **Driftstrom** wirkt der Diffusion entgegen und stabilisiert die Raumladungszone.  
% - Diese Zone verhält sich wie eine **Sperrschicht** und verhindert, dass Strom in **Sperrrichtung** durch die Diode fließt.  

% Dadurch erhält die Diode ihre richtungsabhängigen Eigenschaften:  
% - In **Durchlassrichtung** (p-Schicht positiver als n-Schicht) kann Strom fließen, wenn eine ausreichend hohe Spannung anliegt.  
% - In **Sperrrichtung** (p-Schicht negativer als n-Schicht) blockiert die Raumladungszone den Stromfluss fast vollständig.  

% Die Raumladungszone ist ein fundamentales Konzept in der Halbleitertechnik und essenziell für das Verständnis von Dioden, Transistoren und anderen elektronischen Bauelementen.  
% }

}





\subsection{Berechnung des Spannungsabfalls einer Diode\label{Aufg3}}
% Alt: Aufgabe 3
\Aufgabe{
  Gegeben ist eine Reihenschaltung aus einer Gleichspannungsquelle, einem Widerstand und einer Siliziumdiode mit einer Schleusenspannung von $U_\mathrm{S}=0,7\,\mathrm{V}$. 
  Welche Spannung fällt näherungsweise über dem Widerstand $R$ ab, wenn die Versorgungsspannung $U_0=5\,\mathrm{V}$ beträgt?
  \begin {figure} [H]
   \centering
   \begin{tikzpicture}
  % Batterie links
  \draw (0,2) to[V, name=U1] (0,0);
  \varrmore{U1}{$U_{0}$}; 

  % Widerstand
  \draw (0,2) to[R=$R$] (4,2);
  \draw (0,0) -- (4,0);

  % Diode rechts
  \draw (4,2) to[D=$D$, v, name=D1] (4,0);
  \varrmore{D1}{$U_\mathrm{D}$};

\end{tikzpicture}
   \label{fig:FigDiodeSpannungsabfall}
\end {figure}

% \speech{ **Aufgabe 3**  
% Gegeben ist eine **Reihenschaltung** aus einer **Gleichspannungsquelle**, einem **Widerstand** und einer **Siliziumdiode**.  
% Beantworten Sie die folgende Frage:  
% - Welche Spannung fällt **näherungsweise** über dem Widerstand \( R \) ab, wenn die **Versorgungsspannung \( U_0 = 5V \)** beträgt?  

%  **Beschreibung der Abbildung**  
% Die Schaltung besteht aus drei Hauptkomponenten:  

% 1. **Gleichspannungsquelle (\( U_0 \))**  
%    - Symbolisiert durch einen Kreis mit einem Richtungspfeil.  
%    - Die Spannung \( U_0 \) beträgt **5V** und versorgt die Schaltung.  

% 2. **Serienwiderstand (\( R \))**  
%    - Der Widerstand ist im oberen Schaltzweig platziert.  
%    - Seine Funktion ist die **Strombegrenzung** in der Schaltung.  

% 3. **Siliziumdiode (Si)**  
%    - In **Durchlassrichtung** geschaltet, mit der **Kathode nach Masse**.  
%    - Sie hat eine typische **Schwellenspannung von ca. 0,7V**, die bei Leitung überschritten werden muss.  

%  **Erwartetes Verhalten der Schaltung**  
% - Die Diode **leitet**, sobald \( U_0 > 0,7V \) beträgt.  
% - Die Spannung über dem Widerstand kann **näherungsweise berechnet werden** als:
%   \[
%   U_R = U_0 - U_D
%   \]
%   wobei \( U_D \approx 0,7V \) die **Durchlassspannung der Siliziumdiode** ist.  
% - Für \( U_0 = 5V \) ergibt sich:
%   \[
%   U_R \approx 5V - 0,7V = 4,3V
%   \]
%   - Diese Spannung fällt über den Widerstand \( R \) ab.  
%   - Der Strom durch die Schaltung kann mit dem **Ohm’schen Gesetz** bestimmt werden:
%     \[
%     I = \frac{U_R}{R}
%     \]
%   - Der Widerstand begrenzt den Strom durch die Diode, um eine Überlastung zu verhindern.  

%  **Zusammenfassung**  
% Die Abbildung zeigt eine **einfache Reihenschaltung** mit einer **Spannungsquelle, einem Widerstand und einer Diode**. Die Aufgabe erfordert die **Berechnung der Spannung über dem Widerstand**, wenn die **Diode in Durchlassrichtung betrieben wird**.}

 }


\Loesung{
  Da es sich um eine Siliziumdiode handelt, beträgt die Schleusenspannung und der daraus resultierende Spannungsabfall
  am Bauteil ungefähr $0,6 \text{ bis } 0,7\,\mathrm{V}$.

  In der Reihenschaltung ergibt sich der folgende Zusammenhang:
    \begin{equation*}
        U_0 = U_\mathrm{R} + U_\mathrm{D}
    \end{equation*}
  \begin{align*}
    U_\mathrm{R} &= U_0 - U_\mathrm{D}\\
    &= 5\,\mathrm{V} - 0,7\,\mathrm{V} \\
    &= 4,3\,\mathrm{V}
  \end{align*}
  Siehe: Abschnitt \ref{sec:ElektrischesVerhalten}
%   \speech{Es handelt sich um eine Siliziumdiode. Die Schleusenspannung der Diode beträgt ungefähr 0,6 bis 0,7 Volt. Dadurch ergibt sich ein entsprechender Spannungsabfall am Bauteil.  

% Da die Diode in einer Reihenschaltung mit einem Widerstand verbunden ist, gilt die Spannungsregel:  

% Die Gesamtspannung U null ist gleich der Summe aus der Spannung am Widerstand U R und der Spannung an der Diode U D.  

% Das bedeutet:  

% U R ist gleich U null minus U D.  

% Setzt man die Werte ein, ergibt sich:  

% 5 Volt minus 0,7 Volt ergibt 4,3 Volt.  

% Die Spannung am Widerstand beträgt also 4,3 Volt.}

}
\subsection{Bestimmung des Arbeitspunkts und der Leistung einer Diode\label{Aufg4}}
% Alt: Aufgabe 4
\Aufgabe{
  Eine Diode hat die folgende Kennlinie im 
  Durchlassbereich. Sie wird in Reihe mit einer 
  idealen Spannungsquelle $U_0=1\,\mathrm{V}$ und einem 
  Widerstand mit dem Wert von $R=200\,\Omega$ geschaltet.

    \begin{figure}[H]
        \centering
        \begin{subfigure}[h]{0.45\textwidth}
            \centering
            \begin{circuitikz}
    
\draw (0,0) to [R,i, name=r0, l=$R$] (4,0) to [D=$D$, v, name=D1] (4,-3);
\draw (4,-3) to (0,-3);
\draw (0,0) to [V, name=U1] (0,-3);


\varrmore{U1}{$U_0$}; 
\varrmore{D1}{$U_\mathrm{D}$};
\iarrmore{r0}{$I$};

\end{circuitikz}
        \end{subfigure}
        \begin{subfigure}[h]{0.45\textwidth}
            \centering
            \includesvg[width=0.9\textwidth]{Bilder/Aufgaben/FigDiodeAP}
        \end{subfigure}
        \label{fig:FigDiodeAP}
    \end{figure}

    \begin{itemize}
        \item[a)]
        Welcher Strom $I$ und welche Spannung $U_\mathrm{D}$ 
        stellen sich ein? Lösen Sie das Problem graphisch.
        \item[b)]
        Welche elektrische Leistung wird an der Diode umgesetzt?
        \item[c)]
        Wie muss $R$ gewählt werden, damit sich ein Strom von $3\,\mathrm{mA}$ einstellt? 
    \end{itemize}
%     \speech{ **Aufgabe 4**  
% Eine **Diode mit gegebener Kennlinie im Durchlassbereich** wird in **Reihe mit einer idealen Spannungsquelle** \( U_0 = 1V \) und einem **Widerstand \( R = 200\Omega \)** geschaltet.

% Beantworten Sie die folgenden Fragen:  
% a) **Welcher Strom \( I \) und welche Spannung \( U_D \) stellen sich ein?**  
%    - Lösen Sie das Problem **graphisch** mit Hilfe der gegebenen **Kennlinie**.  
% b) **Welche elektrische Leistung wird an der Diode umgesetzt?**  
% c) **Wie muss \( R \) gewählt werden, damit sich ein Strom von 3 mA einstellt?**  



%  **Beschreibung der Abbildung**
% Die Abbildung besteht aus zwei Teilen:

% 1. **Schaltungsskizze (links):**  
%    - Eine **ideale Gleichspannungsquelle** mit einer Spannung von \( U_0 = 1V \) ist dargestellt.  
%    - Ein **Serienwiderstand \( R \)** von **200Ω** begrenzt den Stromfluss.  
%    - Eine **Siliziumdiode** ist in **Durchlassrichtung** geschaltet.  
%    - Der Strom \( I \) fließt durch den Widerstand und die Diode.  
%    - Die Spannung **\( U_D \)** fällt über der **Diode** ab.  

% 2. **Diodenkennlinie (rechts):**  
%    - **Auf der x-Achse** ist die **Diodenspannung \( U_D \)** in **Volt (V)** dargestellt.  
%    - **Auf der y-Achse** ist der **Diodenstrom \( I \)** in **Milliampere (mA)** aufgetragen.  
%    - Die **exponentielle Kennlinie** zeigt den **typischen Durchlassbereich einer Diode**.  
%    - Um die Lösung zu bestimmen, muss der **Arbeitspunkt** durch die **Schnittstelle zwischen der Diodenkennlinie und der Lastgeraden** ermittelt werden.  

%  **Zusammenfassung der Aufgabe**
% - Die Aufgabe erfordert die **Bestimmung des Stroms und der Spannung an der Diode**, wobei die **Kennlinie** als Grundlage für die Berechnung dient.  
% - Es soll die **umgesetzte elektrische Leistung** berechnet werden.  
% - Schließlich muss der **Widerstandswert \( R \) angepasst werden**, um einen bestimmten **gewünschten Strom von 3 mA** zu erreichen.  
% }

}
    
\Loesung{
    \begin{itemize}
        \item[a)]
        Im ersten Schritt wird die Wirkung der Spannungsquelle auf den Widerstand $R$ betrachtet. 
        Bei $U_0=1\,\mathrm{V}$ fließen durch den Widerstand $R$ ein Strom von $I_\mathrm{R}=\frac{U_0}{R}=\frac{1\,\mathrm{V}}{200\,\Omega}=5\,\mathrm{mA}$. 
        Es kann die Widerstandgerade vom Ursprung aus zu dem Punkt ($1\,\mathrm{V}$, $5\,\mathrm{mA}$) eingezeichnet werden (linke Abbildung).\\

        Gemäß des Aufbaus der Schaltung muss die Widerstandsgerade im nächsten Schritt gespiegelt werden, es entsteht die rechte Darstellung.
        Der Schnittpunkt beider Linien entspricht dem Arbeitspunkt ($U_\text{AP}$, $I_\text{AP}$), dieser lautet: ($0,75\,\mathrm{V}$, $1,2\mathrm{mA}$)  

        \begin{figure}[H]
            \begin{subfigure}[h]{0.45\linewidth}
                \centering
                \includesvg[width=\textwidth]{Bilder/Aufgaben/LsgDiodeAP1}
            \end{subfigure}
            \begin{subfigure}[h]{0.45\linewidth}
                \centering
                \includesvg[width=\textwidth]{Bilder/Aufgaben/LsgDiodeAP2}
            \end{subfigure}
        \end{figure}

        \item[b)]
        Die Leistung ergibt sich aus dem zuvor ermittelten Arbeitspunkt.
        \begin{equation*}
            P = U_\text{AP} \cdot I_\text{AP} = 0,75 \, \text{V} \cdot 1,2 \, \text{mA} = 0,9 \, \text{mW}
        \end{equation*}
        
        \item[c)]
        Ausgehend vom Punkt ($1\,\mathrm{V}, 0\,\mathrm{mA}$) zum Schnittpunkt mit der Diodenkennlinie bei $I=3\,\mathrm{mA}$ ergibt sich die gesuchte Widerstandsgerade.
        \begin{figure}[H]
            \centering
            \includesvg[width=0.45\linewidth]{Bilder/Aufgaben/LsgDiodeAP3}
        \end{figure}
        Es kann der Schnittpunkt ($0,86\,\mathrm{V}, 3\,\mathrm{mA}$) abegelesen werden.
    

        \begin{equation*}
            U_\mathrm{ges} = U - U_\mathrm{D} = 1 \, \mathrm{V} - 0,86 \, \mathrm{V} = 0,14 \, \mathrm{V}
        \end{equation*}


    \end{itemize}



\begin{itemize}
    \item[\bf c)] 
    \[
    U_{\text{ges}} = U - U_\text{D} = 1 \, \text{V} - 0,86 \, \text{V} = 0,14 \, \text{V}
    \]
    \[
    I = 3 \, \text{mA}
    \]
    \[
    R = \frac{U_{\text{ges}}}{I} = \frac{0,14 \, \text{V}}{3 \, \text{mA}} = 46,67 \, \Omega
    \]
\end{itemize}
Siehe: Abschnitt \ref{fig:Diodenkennlinie}


% \speech{
% **Lösung zu Aufgabe 4**  

% **Teilaufgabe a:**  
% Im ersten Schritt wird die Wirkung der Spannungsquelle auf den Widerstand \( R \) betrachtet.  
% Die gegebene Spannung beträgt \( U_0 = 1V \).  
% Da der Widerstand \( R = 200 \Omega \) beträgt, ergibt sich der Strom durch den Widerstand als:  
% \( I_R = \frac{U_0}{R} = \frac{1V}{200 \Omega} = 5mA \).  

% Die dazugehörige Widerstandsgerade kann im Diagramm als eine Linie von der Ursprungskoordinate \((0V, 0mA)\) bis zum Punkt \((1V, 5mA)\) eingezeichnet werden.  
% Diese Gerade gibt an, wie sich der Strom \( I_R \) in Abhängigkeit der Spannung \( U \) verhält.  

% In der linken Abbildung ist die Diodenkennlinie als eine exponentiell ansteigende blaue Kurve dargestellt.  
% Zusätzlich ist die Widerstandsgerade als schwarze Linie eingezeichnet.  
% Diese verläuft von \((0V, 0mA)\) bis zu \((1V, 5mA)\).  

% Da das Bauteil eine Diode enthält, muss die Widerstandsgerade gespiegelt werden.  
% Dies ist in der zweiten Abbildung auf der rechten Seite ersichtlich.  
% Die Widerstandsgerade wird so verschoben, dass sie nun von einem neuen Ursprungspunkt ausgeht.  
% Der Schnittpunkt dieser Linie mit der Diodenkennlinie entspricht dem Arbeitspunkt der Schaltung.  
% Dort gelten die Werte:  
% \( U_{AP} = 0,75V \) und \( I_{AP} = 1,2mA \).  

% **Teilaufgabe b:**  
% Die elektrische Leistung an der Diode wird aus dem zuvor bestimmten Arbeitspunkt berechnet.  
% Die allgemeine Formel für die Leistung lautet:  
% \( P = U \cdot I \).  
% Hier setzen wir die Werte des Arbeitspunktes ein:  
% \( P = 0,75V \cdot 1,2mA = 0,9mW \).  

% **Teilaufgabe c:**  
% Nun soll der Widerstand \( R \) so bestimmt werden, dass sich ein Strom von \( I = 3mA \) einstellt.  
% In der dritten Abbildung ist erneut eine Diodenkennlinie zu sehen.  
% Zusätzlich wurde eine neue Widerstandsgerade eingezeichnet, die nun durch den Punkt \((1V, 0mA)\) zum Schnittpunkt mit der Diodenkennlinie bei \( I = 3mA \) verläuft.  
% Dieser Schnittpunkt liegt bei der Spannung \( 0,86V \).  

% Der Spannungsabfall am Widerstand ergibt sich aus:  
% \( U_{ges} = U - U_D = 1V - 0,86V = 0,14V \).  
% Da der gewünschte Strom \( I = 3mA \) beträgt, berechnet sich der benötigte Widerstand durch das Ohmsche Gesetz:  
% \( R = \frac{U_{ges}}{I} = \frac{0,14V}{3mA} = 46,67 \Omega \).  

% Das bedeutet, dass für einen Strom von 3mA der Widerstand \( R \) auf 46,67 Ohm angepasst werden muss.  
% }

}

\subsection{Dimensionierung des Vorwiderstands einer Zenerdiode\label{Aufg5}}
% Alt: Aufgabe 5
\Aufgabe{
  Die Zenerspannung der Diode beträgt $U_\mathrm{D}=6\,\mathrm{V}$ und die Diode darf eine maximale Verlustleistung von \(P_\mathrm{tot}=\mathrm{400\,mW} \) nicht überschreiten. 
  Die maximale Eingangsspannung beträgt \( U_0 = \mathrm{12\,V} \). \\
  Wie groß muss der Widerstand \( R \) mindestens sein, damit die Diode nicht zerstört wird?

\begin {figure} [H]
   \centering
   \begin{tikzpicture}

    \draw (0,2) to[V, name=U0] (0,0);
    \draw (0,2) to[R=$R$] (4,2);
    \draw (0,0) -- (4,0);
    \draw (4,0) to[zDo, l_=$D_\mathrm{z}$, v^<, name=Uz] (4,2);

    \varrmore{U0}{$U_0$}; 
    \varrmore{Uz}{$U_\mathrm{D}$}; 
  
  \end{tikzpicture}
   \label{fig:ZDiodeVorwiderstand}
\end {figure}

% \speech{ **Aufgabe 5**  
% Eine **Zenerdiode** mit einer **Zenerspannung von 6V** wird in einer Schaltung verwendet, um **Überspannung zu begrenzen**.  
% Die **maximale Verlustleistung** der Diode beträgt **400 mW**, und die **maximale Eingangsspannung \( U_0 \)** ist **12V**.  

% **Fragestellung:**  
% - Wie groß muss der **Widerstand \( R \)** mindestens sein, damit die **Diode niemals zerstört** wird?  

%  **Beschreibung der Abbildung**  
% Die Schaltung besteht aus den folgenden Komponenten:

% 1. **Gleichspannungsquelle \( U_0 \)**  
%    - Symbolisiert durch einen Kreis mit einem Pfeil.  
%    - Die **maximale Spannung beträgt 12V**.  

% 2. **Serienwiderstand \( R \)**  
%    - Platziert im oberen Zweig der Schaltung.  
%    - Begrenzung des **Stroms durch die Zenerdiode**.  

% 3. **Zenerdiode**  
%    - In **Sperrrichtung** geschaltet.  
%    - Sobald die **Spannung \( U_0 > 6V \)** erreicht, beginnt die Diode zu **leiten** und hält die Spannung auf **6V konstant**.  

%  **Funktionsweise der Schaltung**  
% - Wenn **\( U_0 < 6V \)**, ist die **Diode nicht leitend**, und es fließt kein Strom durch sie.  
% - Wenn **\( U_0 > 6V \)**, hält die **Zenerdiode die Spannung bei 6V**, indem sie den überschüssigen Strom ableitet.  
% - Der Widerstand \( R \) muss so **gewählt werden**, dass der **Strom durch die Diode die maximale Verlustleistung von 400 mW nicht überschreitet**.  


%  **Zusammenfassung der Aufgabe**  
% - Die Aufgabe erfordert die **Berechnung des minimalen Widerstandswerts \( R \)**, um die **maximale Verlustleistung der Zenerdiode nicht zu überschreiten**.  
% - Die Schaltung dient als **Spannungsregler**, der eine **konstante Spannung von 6V gewährleistet**, wenn die Eingangsspannung zu hoch ist.  
% - Die Berechnung erfolgt mit **Leistungsgesetzen und dem Ohm’schen Gesetz**, um den maximal zulässigen Strom durch die Diode zu bestimmen.  
% }

}


\Loesung{
    Die Spannung am Widerstand ergibt sich zu:
    \begin{align*}
        U_\mathrm{R} &= U_0 - U_\mathrm{D} = \mathrm{12\,V} - \mathrm{6\,V} = \mathrm{6\,V}
    \end{align*}
    Die maximale Verlustleistung der Diode ist:
    \begin{align*}
        P_\mathrm{tot} &= U_\mathrm{D} \cdot I = \mathrm{400\,mW}
    \end{align*}
    Daraus folgt der maximale Strom durch die Diode:
    \begin{align*}
        I &= \frac{\mathrm{400\,mW}}{\mathrm{6\,V}} = \mathrm{66{,}7\,mA}
    \end{align*}
    Um diesen Strom nicht zu überschreiten, ergibt sich der Mindestwert für den Widerstand:
    \begin{align*}
        R &= \frac{U_\mathrm{R}}{I} = \frac{\mathrm{6\,V}}{\mathrm{66{,}7\,mA}} \approx \mathrm{90\,\Omega}
    \end{align*}
  Siehe: Abschnitt \ref{sec:SpezielleDioden}

%   \speech{
% **Lösung zu Aufgabe 5**  

% Die gegebene Schaltung besteht aus einer Spannungsquelle mit **12 Volt**, einem Widerstand **R**, und einer **Zenerdiode**.  
% Die **Zenerdiode** begrenzt die Spannung an ihren Anschlüssen auf **6 Volt**.  
% Das bedeutet, dass die restlichen **6 Volt** über den Widerstand **R** abfallen müssen.  

% ### Berechnungen  

% **Berechnung des Spannungsabfalls über den Widerstand**  
% Da die **Gesamtspannung** der Quelle **12 Volt** beträgt und die Zenerdiode **6 Volt** benötigt, ergibt sich die Spannung über **R** als:  

% \[
% U_R = 12V - 6V = 6V
% \]  

% **Berechnung der Leistung**  
% Die gegebene **maximale Verlustleistung** der Zenerdiode beträgt **0,4 Milliwatt**.  
% Die elektrische Leistung wird nach der Formel berechnet:  

% \[
% P = U \cdot I
% \]  

% Einsetzen der bekannten Werte:  

% \[
% 0,4 mW = 6V \cdot I
% \]  

% Nach **Umstellen nach I** ergibt sich:  

% \[
% I = \frac{0,4 mW}{6V} = 66,7mA
% \]  

% **Berechnung des Widerstands \( R \)**  
% Nach dem **Ohmschen Gesetz** gilt:  

% \[
% U = R \cdot I
% \]  

% Umstellen nach **R**:  

% \[
% R = \frac{U}{I} = \frac{6V}{66,7mA} = 90 \Omega
% \]  

% ### Interpretation  

% Damit die **Zenerdiode** nicht beschädigt wird, muss der Widerstand **R** mindestens **90 Ohm** betragen.  
% Dieser Widerstand begrenzt den Strom auf **66,7 Milliamper**, sodass die **maximale Verlustleistung der Diode** nicht überschritten wird.  

% }

}




\subsection{Einsatz einer Zenerdiode zur Spannungsstabilisierung\label{Aufg6}}
% Alt: Aufgabe 6
\Aufgabe{
  Mit Hilfe einer Zenerdiode soll ein Lastwiderstand \( R_\mathrm{L} = \mathrm{160\,\Omega} \) mit einer stabilisierten Spannung von \( U_\mathrm{A} = \mathrm{8\,V} \) versorgt werden. \\
  Dem Datenblatt der Zenderdiode ist zu entnehmen, dass bei einer Zenerspannung von \( \mathrm{-8\,V} \) ein Zenerstrom von \( \mathrm{-0{,}45\,A} \) fließt.
  
  \begin {figure} [H]
     \centering
     \begin{tikzpicture}
  \draw(0,2) 
    to[V, name=V1, v] (0,0)
    to[short, -o](5,0) -- (7,0)
    to[R, l_=$R_\mathrm{L}$] (7,2)
    to[short, -o](5,2)
    to[short] (4,2)
    to[R, l_=$R_\mathrm{V}$] (0,2);

  \draw(4,0) 
  to[zDo=$D_\mathrm{Z}$, *-*] (4,2) ;

  \draw (5,2) 
  to[open, name=ua, v^] (5,0);
  
  \varrmore{ua}{$U_\mathrm{A}$};
  \varrmore{V1}{$U_\mathrm{E}$};
\end{tikzpicture}
     \label{fig:FigZDiodeStabilisierung}
  \end {figure}

  \begin{itemize}
      \item[a)] Berechnen Sie den Vorwiderstand \( R_\mathrm{V} \), sodass sich bei einer Eingangsspannung von \( U_\mathrm{E} = \mathrm{10\,V} \) eine Ausgangsspannung von \( U_\mathrm{A} = \mathrm{8\,V} \) einstellt.
      \item[b)] Wie groß sind die Leistungen, die an \( R_\mathrm{V} \), \( R_\mathrm{L} \) und an der Diode \( D_\mathrm{Z} \) umgesetzt werden?
  \end{itemize}


%   \speech{ **Aufgabe 6**  
% Mit Hilfe einer **Zenerdiode** soll ein **Lastwiderstand \( R_L = 160Ω \)** mit einer **stabilisierten Spannung \( U_A = 8V \)** versorgt werden.  
% Dem **Datenblatt der Z-Diode** ist zu entnehmen, dass für eine **Zenerspannung von \(-8V\)** ein **Zenerstrom von \(-0,45A\)** fließt.  

% **Fragestellungen:**  
% a) **Berechnen Sie den Vorwiderstand \( R_V \)** so, dass sich bei einer **Eingangsspannung von \( U_E = 10V \)** eine **Ausgangsspannung von \( U_A = 8V \)** einstellt.  
% b) **Wie groß sind die Leistungen, die an \( R_V \), \( R_L \) und \( D_Z \) umgesetzt werden?**  



%  **Beschreibung der Abbildung**  
% Die gezeigte Schaltung ist eine **Spannungsstabilisierungsschaltung mit einer Zenerdiode** und besteht aus folgenden Komponenten:

% 1. **Eingangsspannungsquelle \( U_E \)**  
%    - Symbolisiert durch eine **Gleichspannungsquelle mit \( U_E = 10V \)**.  
%    - Diese **versorgt die gesamte Schaltung** mit Energie.  

% 2. **Vorwiderstand \( R_V \)**  
%    - Platziert **in Serie mit der Zenerdiode**.  
%    - Seine Aufgabe ist die **Begrenzung des Stroms**, um die Zenerdiode nicht zu überlasten.  

% 3. **Zenerdiode \( D_Z \)**  
%    - Schaltet sich bei **Erreichen der Zenerspannung von \(-8V\)** in den **Leitungsmodus**.  
%    - Hält die **Spannung an \( R_L \) konstant auf \( U_A = 8V \)**.  

% 4. **Lastwiderstand \( R_L \) mit \( 160Ω \)**  
%    - Parallel zur **Zenerdiode geschaltet**.  
%    - Die **Ausgangsspannung \( U_A = 8V \)** fällt über \( R_L \) ab.  
%    - Es fließt ein **Laststrom**, der sich aus **\( I_L = \frac{U_A}{R_L} \)** berechnet.  



%  **Funktionsweise der Schaltung**  
% - Wenn die **Eingangsspannung \( U_E = 10V \)** beträgt, sorgt die **Zenerdiode dafür**, dass die **Spannung an \( R_L \) stabil bei 8V bleibt**.  
% - Der **Vorwiderstand \( R_V \)** muss so gewählt werden, dass der **Gesamtstrom aufgeteilt** wird zwischen der **Zenerdiode** und dem **Lastwiderstand**.  
% - Zur **Berechnung der Leistungen** müssen die **verlorene Leistung am Vorwiderstand**, die **Verlustleistung der Zenerdiode** und die **nutzbare Leistung am Lastwiderstand** bestimmt werden.  



%  **Zusammenfassung der Aufgabe**  
% - Die Aufgabe erfordert die **Berechnung des Vorwiderstandes \( R_V \)** zur **Spannungsstabilisierung mit einer Zenerdiode**.  
% - Weiterhin soll bestimmt werden, wie sich die **elektrische Leistung** auf die **verschiedenen Schaltungselemente** verteilt.  
% - Die **Zenerdiode fungiert als Spannungsregler**, indem sie bei überschüssiger Spannung einen **Teil des Stroms aufnimmt**.  
% }

}


\Loesung{
  \begin {figure} [H]
  \centering
  \begin{tikzpicture}
  \draw(0,2) 
    to[V, name=V1, v] (0,0)
    to[short, -o](5,0) -- (8.5,0)
    to[R, l_=$R_\mathrm{L}$, v^<, name=RL] (8.5,2)
    to[short, -o](5,2)
    to[short] (4,2)
    to[R, l_=$R_\mathrm{V}$, i<_, name=RV] (0,2);

  \draw(4,0) 
  to[zDo=$D_\mathrm{D}$, *-*, i_<, name=D] (4,2) ;

  \draw (5,2) 
  to[open, name=ua, v^] (5,0);
  
  \varrmore{ua}{$U_\mathrm{A}$};
  \varrmore{V1}{$U_\mathrm{E}$};
  \varrmore{RL}{$U_\mathrm{RL}$};
  \iarrmore{RV}{${I}$};
  \iarrmore{D}{$I_\mathrm{Z}$};
  
  \draw[->,shift={(2,1)}] (150:0.5) arc (150:-150:0.5) node at(0,0){$M_\mathrm{1}$};  
  \draw[->,shift={(6.5,1)}] (150:0.5) arc (150:-150:0.5) node at(0,0){$M_\mathrm{2}$};
\end{tikzpicture}

  \label{fig:LsgZDiodeStabilisierung}
\end {figure}
\begin{itemize}
\item[a)]
\begin{align*}
  I_\mathrm{RL} &= \frac{\mathrm{8\,V}}{\mathrm{160\,\Omega}} = \mathrm{0{,}05\,A} \\
  I &= \mathrm{0{,}05\,A} + \mathrm{0{,}45\,A} = \mathrm{0{,}5\,A} \\
  U_\mathrm{RV} &= \mathrm{10\,V} - \mathrm{8\,V} = \mathrm{2\,V} \\
  R_\mathrm{V} &= \frac{\mathrm{2\,V}}{\mathrm{0,5\,A}} = \mathrm{4\,\Omega}
\end{align*}

\item[b)]
\begin{align*}
  P_\mathrm{RV} &= \mathrm{2\,V} \cdot \mathrm{0{,}5\,A} = \mathrm{1\,W} \\
  P_\mathrm{RL} &= \mathrm{8\,V} \cdot \mathrm{0{,}05\,A} = \mathrm{0{,}4\,W} \\
  P_\mathrm{DZ} &= \mathrm{8\,V} \cdot \mathrm{0{,}45\,A} = \mathrm{3{,}6\,W}
\end{align*}
\end{itemize}
Siehe: Abschnitt \ref{sec:SpezielleDioden}

% \speech{
% Lösung zu Aufgabe 6:

% Die Aufgabe behandelt eine Schaltung mit einer Zenerdiode, einem Lastwiderstand von 160 Ohm und einer stabilisierten Ausgangsspannung von 8 Volt. Es soll der Vorwiderstand \( R_V \) berechnet werden, sodass sich eine stabile Spannung von 8 Volt einstellt, sowie die Leistungen, die an den Widerständen und der Diode umgesetzt werden.

% ### Teil a: Berechnung des Vorwiderstands \( R_V \)

% Die gegebene Eingangsspannung beträgt 10 Volt, die Zenerspannung der Diode beträgt -8 Volt, und der Strom durch die Zenerdiode ist mit -0,45 Ampere gegeben.

% Zunächst wird der Strom durch den Lastwiderstand \( R_L \) berechnet:
% \[
% I_{R_L} = \frac{U_A}{R_L} = \frac{8V}{160\Omega} = 0,05A
% \]
% Der Gesamtstrom in der Schaltung setzt sich zusammen aus dem Strom durch den Lastwiderstand und dem Strom durch die Zenerdiode:
% \[
% I = I_{R_L} + I_D = 0,05A + 0,45A = 0,5A
% \]
% Die Spannung am Vorwiderstand \( R_V \) ergibt sich als Differenz zwischen Eingangsspannung und Ausgangsspannung:
% \[
% U_{R_V} = U_E - U_A = 10V - 8V = 2V
% \]
% Damit kann der Vorwiderstand berechnet werden:
% \[
% R_V = \frac{U_{R_V}}{I} = \frac{2V}{0,5A} = 4\Omega
% \]

% ### Teil b: Berechnung der Leistungen

% Die Leistung, die am Vorwiderstand umgesetzt wird, ergibt sich aus:
% \[
% P_{R_V} = U_{R_V} \cdot I = 2V \cdot 0,5A = 1W
% \]
% Die Leistung, die am Lastwiderstand \( R_L \) umgesetzt wird:
% \[
% P_{R_L} = U_A \cdot I_{R_L} = 8V \cdot 0,05A = 0,4W
% \]
% Die Verlustleistung der Zenerdiode berechnet sich aus:
% \[
% P_{D_Z} = U_A \cdot I_D = 8V \cdot 0,45A = 3,6W
% \]

% ### Beschreibung der Abbildung:

% Die Schaltung in der Abbildung zeigt eine Spannungsquelle mit 10 Volt, die über den Vorwiderstand \( R_V \) mit einer Zenerdiode in Parallelschaltung mit dem Lastwiderstand \( R_L \) verbunden ist. Der Strom durch den Vorwiderstand wird als Pfeil in die Zenerdiode eingezeichnet, wo sich der Strom in zwei Teilströme aufteilt: ein Teil fließt durch den Lastwiderstand, ein anderer durch die Zenerdiode. Die Spannungen sind ebenfalls eingezeichnet: Die Versorgungsspannung mit 10 Volt, die stabilisierte Spannung von 8 Volt über der Zenerdiode und dem Lastwiderstand sowie die Spannung über dem Vorwiderstand mit 2 Volt.

% Die Berechnungen in der Abbildung sind klar strukturiert und zeigen Schritt für Schritt, wie die Widerstandswerte und Leistungen bestimmt werden. Die Ergebnisse bestätigen die geforderte Ausgangsspannung von 8 Volt und zeigen die Leistungsverteilung in der Schaltung.
% }

}




\subsection{Ermittlung der Betriebsgrenzen einer Zenerdiode\label{Aufg7}}
% Alt: Aufgabe 7
\Aufgabe{
  Die eingesetzte Zenerdiode besitzt eine Zenerspannung von \( U_\mathrm{Z} = \mathrm{8\,V} \). 
  Die maximale Verlustleistung der Diode beträgt \( P_\mathrm{tot}=\mathrm{8\,W} \). 
  Die Diode arbeitet im stabilisierenden Bereich, wenn der Strom durch sie mindestens \( \mathrm{5\,mA} \) beträgt. \\
  Welchen minimalen und maximalen Widerstandswert darf der Lastwiderstand \( R_\mathrm{L} \) nicht unter- bzw. überschreiten, damit die Diode einerseits im stabilisierenden Bereich arbeitet und andererseits nicht überlastet wird?
  
  \begin {figure} [H]
     \centering
     \begin{tikzpicture}
  \draw(0,2) 
    to[V, name=V1, v] (0,0)
    to[short, -o](5,0) -- (7,0)
    to[R, l_=$R_\mathrm{L}$] (7,2)
    to[short, -o](5,2)
    to[short] (4,2)
    to[R, l_=$R_\mathrm{V}$] (0,2);

  \draw(4,0) 
  to[zDo=$D_\mathrm{Z}$, *-*] (4,2) ;

  \draw (5,2) 
  to[open, name=ua, v^] (5,0);
  
  \varrmore{ua}{$U_\mathrm{A}$};
  \varrmore{V1}{$U_\mathrm{E}$};
\end{tikzpicture}
     \label{fig:FigZDiodeGrenzen}
  \end {figure}

%   \speech{ **Aufgabe 7**  
% Die eingesetzte **Zenerdiode** besitzt eine **Zenerspannung von \( U_Z = 8V \)**.  
% - Die **maximale Verlustleistung** der Diode beträgt **8 W**.  
% - Die Diode arbeitet **im stabilisierenden Bereich**, wenn der **Strom durch die Diode größer als 5 mA** ist.  

% **Fragestellung:**  
% - **Welchen minimalen und maximalen Widerstandswert darf der Lastwiderstand \( R_L \) nicht unter- oder überschreiten**, damit die **Diode im stabilisierenden Bereich arbeitet** und nicht überlastet wird?  
%  **Beschreibung der Abbildung**  
% Die gezeigte Schaltung stellt eine **Zener-Spannungsstabilisierungsschaltung** dar und besteht aus folgenden Komponenten:

% 1. **Eingangsspannungsquelle**  
%    - **Nennwert: 20V** mit einer **Toleranz von ±10%**, also zwischen **18V und 22V**.  
%    - Symbolisiert durch einen **Kreis mit Richtungspfeil**.  

% 2. **Serienwiderstand (10Ω)**  
%    - Platziert im **oberen Zweig der Schaltung**.  
%    - Begrenzung des **Gesamtstroms** in der Schaltung.  

% 3. **Zenerdiode \( D_Z \)**  
%    - In **Sperrrichtung** betrieben.  
%    - Hält die **Spannung über \( R_L \) stabil bei 8V**.  
%    - Beginnt **zu leiten**, wenn die Spannung über **8V steigt** und sorgt für eine **Überspannungsbegrenzung**.  

% 4. **Lastwiderstand \( R_L \)**  
%    - Parallel zur **Zenerdiode** geschaltet.  
%    - Soll mit einer **stabilen Spannung von 8V** betrieben werden.  
%    - Der zulässige **Widerstandsbereich für \( R_L \)** ist zu bestimmen, damit die **Diode weder zu wenig noch zu viel Strom führt**.  

%  **Funktionsweise der Schaltung**  
% - Die **Zenerdiode regelt die Ausgangsspannung auf 8V** unabhängig von **Schwankungen der Eingangsspannung**.  
% - Die **minimale Last \( R_L \) darf nicht zu klein** sein, damit die **Diode noch genug Strom aufnimmt**.  
% - Die **maximale Last \( R_L \) darf nicht zu groß** sein, da sonst die **Zenerdiode zu viel Strom aufnehmen müsste**, was zu einer **Überlastung (8W-Grenze) führen könnte**.  
% - Das **Problem wird durch eine Berechnung der Stromverhältnisse** gelöst.  

% **Zusammenfassung der Aufgabe**  
% - Die Aufgabe erfordert die **Bestimmung des minimalen und maximalen Widerstands \( R_L \)**, damit die **Zenerdiode richtig arbeitet**.  
% - Die **Spannungsschwankungen der Quelle** müssen berücksichtigt werden.  
% - Die Berechnung erfolgt unter der **Bedingung, dass der Strom durch die Zenerdiode größer als 5mA bleibt** und die **maximale Leistung von 8W nicht überschritten wird**.  
% }

}


\Loesung{
  \begin{figure}[H]
    \centering
    \begin{tikzpicture}
    \draw(0,2)
    to[V, name=V1, v] (0,0)
    to[short, -o](5,0) -- (7,0)
    to[R, l_=$R_\mathrm{L}$, v^<, name=RL] (7,2)
    to[short, -o](5,2)
    to[short, i<_, name=IL] (4,2)
    to[R, l_=$R_\mathrm{V}$, i<_, name=RV] (0,2);
    
    \draw(4,0)
    to[zDo=$D_\mathrm{Z}$, *-*, i_<, name=D] (4,2) ;
    
    \draw (5,2)
    to[open, name=ua, v^] (5,0);
    
    \varrmore{ua}{$U_\mathrm{A}$};
    \varrmore{V1}{$U_\mathrm{E}$};
    \iarrmore{RV}{${I}$};
    \iarrmore{D}{$I_\mathrm{D}$};
    \iarrmore{IL}{$I_\mathrm{L}$};
    
\end{tikzpicture}
    \label{fig:LsgZDiodeGrenzen}
  \end{figure}

  Gegeben sind:
\begin{align*}
    U_\mathrm{max} &= \mathrm{22\,V} \\
    U_\mathrm{min} &= \mathrm{18\,V}
\end{align*}

\textbf{Fall 1:} Bei \( U_\mathrm{max} = \mathrm{22\,V} \)
\begin{align*}
    U_\mathrm{RV} &= \mathrm{22\,V} - \mathrm{8\,V} = \mathrm{14\,V} \\
    I &= \frac{\mathrm{14\,V}}{\mathrm{10\,\Omega}} = \mathrm{1{,}4\,A} \\
    I &= I_\mathrm{D} + I_\mathrm{L} \\
    I_\mathrm{D,\ max} &= \frac{\mathrm{8\,W}}{\mathrm{8\,V}} = \mathrm{1\,A} \\
    I_\mathrm{L} &= \mathrm{1{,}4\,A} - \mathrm{1\,A} = \mathrm{0{,}4\,A} \\
    R_\mathrm{L} &= \frac{\mathrm{8\,V}}{\mathrm{0{,}4\,A}} = \mathrm{20\,\Omega} \Rightarrow R_\mathrm{L,\ max} = \mathrm{20\,\Omega}
\end{align*}

\textbf{Fall 2:} Bei \( U_\mathrm{min} = \mathrm{18\,V} \)
\begin{align*}
    U_\mathrm{RV} &= \mathrm{18\,V} - \mathrm{8\,V} = \mathrm{10\,V} \\
    I &= \frac{\mathrm{10\,V}}{\mathrm{10\,\Omega}} = \mathrm{1\,A} \\
    I_\mathrm{D,\ min} &= \mathrm{5\,mA} \\
    I_\mathrm{L} &= \mathrm{1\,A} - \mathrm{5\,mA} = \mathrm{0{,}995\,A} \\
    R_\mathrm{L} &= \frac{\mathrm{8\,V}}{\mathrm{0{,}995\,A}} \approx \mathrm{8\,\Omega} \Rightarrow R_\mathrm{L,\ min} \approx \mathrm{8\,\Omega}
\end{align*}
%   \speech{
% Lösung zu Aufgabe 7:

% Gegeben ist eine Schaltung mit einer Spannungsquelle von 20 Volt plus-minus 10 Prozent, einem Vorwiderstand von 10 Ohm, einer Zenerdiode mit einer Zenerspannung von 8 Volt und einem variablen Lastwiderstand R L. Ziel ist es, die minimalen und maximalen Werte für R L zu bestimmen, sodass die Zenerdiode im stabilisierenden Bereich arbeitet, aber nicht überlastet wird.

% Betrachtung für die maximale Eingangsspannung U max gleich 22 Volt:
% - Die Spannung über dem Vorwiderstand beträgt:
%   U R V gleich 22 Volt minus 8 Volt gleich 14 Volt.
% - Der Gesamtstrom durch den Vorwiderstand ist:
%   I gleich U R V geteilt durch R V.
%   I gleich 14 Volt geteilt durch 10 Ohm gleich 1,4 Ampere.
% - Der Gesamtstrom teilt sich auf in den Strom durch die Zenerdiode I D und den Laststrom I L:
%   I gleich I D plus I L.
% - Die maximale Verlustleistung der Zenerdiode ist gegeben mit 8 Watt. Der maximale Strom durch die Zenerdiode ergibt sich durch:
%   I D max gleich P max geteilt durch U Z.
%   I D max gleich 8 Watt geteilt durch 8 Volt gleich 1 Ampere.
% - Daraus ergibt sich der Laststrom:
%   I L gleich I minus I D max.
%   I L gleich 1,4 Ampere minus 1 Ampere gleich 0,4 Ampere.
% - Die Spannung über dem Lastwiderstand beträgt:
%   U L gleich U Z gleich 8 Volt.
% - Der maximale Lastwiderstand ist dann:
%   R L max gleich U L geteilt durch I L.
%   R L max gleich 8 Volt geteilt durch 0,4 Ampere gleich 20 Ohm.

% Betrachtung für die minimale Eingangsspannung U min gleich 18 Volt:
% - Die Spannung über dem Vorwiderstand beträgt:
%   U R V gleich 18 Volt minus 8 Volt gleich 10 Volt.
% - Der Gesamtstrom durch den Vorwiderstand ist:
%   I gleich U R V geteilt durch R V.
%   I gleich 10 Volt geteilt durch 10 Ohm gleich 1 Ampere.
% - Der Gesamtstrom teilt sich wieder auf in den Strom durch die Zenerdiode I D und den Laststrom I L.
% - Die minimale Stromaufnahme der Zenerdiode, damit sie im stabilisierenden Bereich arbeitet, beträgt 5 Milliampere.
% - Daraus ergibt sich der Laststrom:
%   I L gleich I minus I D min.
%   I L gleich 1 Ampere minus 5 Milliampere gleich 995 Milliampere.
% - Die Spannung über dem Lastwiderstand beträgt:
%   U L gleich 8 Volt.
% - Der minimale Lastwiderstand ist dann:
%   R L min gleich U L geteilt durch I L.
%   R L min gleich 8 Volt geteilt durch 0,995 Ampere gleich 8 Ohm.

% Zusammenfassung:
% - Der Lastwiderstand R L muss zwischen 8 Ohm und 20 Ohm liegen, damit die Zenerdiode im stabilisierenden Bereich bleibt und nicht überlastet wird.
% }
Siehe: Abschnitt \ref{sec:SpezielleDioden}
}




\subsection{Einsatz einer Diode zur Leistungsreduzierung\label{Aufg8}}
% Alt: Aufgabe 8
\Aufgabe{
    In einer einfachen Schaltung wird eine Diode zur Leistungsreduktion eingesetzt. Der Widerstand $R_\mathrm{L} = 1\,\mathrm{k\Omega}$ stellt eine Heizlast dar. 
    Die Netzspannung beträgt $230\,\mathrm{V}$ Effektivwert bei einer Frequenz von $50\,\mathrm{Hz}$.
    
    \begin{figure}[H]
        \centering
        \begin{tikzpicture}

    % Batterie links
    \draw
    (0,0) to[sV, v<,name=V1 ] (0,2);
  
    % Widerstand
    \draw
    (0,2) to[D=$D$, v, name=D1] (4,2);

    \draw (0,0) -- (4,0);
  
    % LED rechts
    \draw (4,2) to [R, l=$R_\mathrm{L}$] (4,0);
   

    \varrmore{V1}{$u_\mathrm{E}$};
    \varrmore{D1}{$U_\mathrm{D}$};

  \end{tikzpicture}
        \label{fig:FigDiodeLeistungsreduzierung}
    \end{figure}
    
    \begin{itemize}
        \item[a)] Welche mittlere Leistung wird in $R_\mathrm{L}$ und in der Diode umgesetzt, wenn die Diode als idealer Gleichtrichter wirkt (d.h. nur eine Halbwelle durchlässt)?
        \item[b)] Wie groß wäre die mittlere Leistung bei überbrückter Diode (volle Netzspannung an $R_\mathrm{L}$)?
    \end{itemize}
    
    %   \speech{ **Aufgabe 8**  
    % In einer **einfachen Schaltung** wird eine **Diode zur Leistungsreduktion** eingesetzt.  
    % - Der **Lastwiderstand \( R_L \)** ist ein **Heizwiderstand** mit einem **Widerstand von 1kΩ**.  
    % - Die **Eingangsspannung** hat einen **Effektivwert von 230V bei 50Hz**.  
    
    % **Fragestellungen:**  
    % a) **Welche mittlere Leistung würde in \( R_L \) und in der Diode umgesetzt**, wenn die **Diode ideal als Ventil** funktionieren würde?  
    % b) **Wie groß wäre die mittlere Leistung bei überbrückter Diode?**  
    
    
    
    %  **Beschreibung der Abbildung**  
    % Die gezeigte Schaltung stellt eine **einphasige Halbwellenschaltung mit einer Diode** dar. Die Hauptkomponenten sind:
    
    % 1. **Wechselspannungsquelle \( u_e \)**  
    %    - Symbolisiert durch einen **Sinusgenerator**.  
    %    - Spannung: **230V Effektivwert bei 50Hz**.  
    
    % 2. **Diode \( D \) (Leistungsreduktionselement)**  
    %    - In **Reihe mit dem Lastwiderstand** geschaltet.  
    %    - Funktioniert als **Ventil**, indem sie **eine Halbwelle der Wechselspannung sperrt**.  
    
    % 3. **Lastwiderstand \( R_L = 1kΩ \)**  
    %    - Platziert **nach der Diode**.  
    %    - Ersetzt eine **reale Last, z.B. einen Heizwiderstand**.  
    %    - Umwandlung der elektrischen Leistung in **Wärme**.  
    
    
    
    %  **Funktionsweise der Schaltung**  
    % - Die **Diode sperrt die negative Halbwelle**, sodass nur **eine Halbwelle der Wechselspannung** am **Widerstand \( R_L \)** anliegt.  
    % - Dies führt zu einer **Reduktion der mittleren Leistung im Widerstand**.  
    % - Bei **überbrückter Diode** würde der **volle sinusförmige Wechselstrom** durch den Widerstand fließen, was zu **einer anderen mittleren Leistung führt**.  
    
    
    
    %  **Zusammenfassung der Aufgabe**  
    % - Die Aufgabe erfordert die **Berechnung der mittleren Leistung** in \( R_L \) und der **Diode**, wenn sie **ideal als Ventil** funktioniert.  
    % - Zudem muss untersucht werden, wie sich die Leistung **bei überbrückter Diode** verhält.  
    % - Es wird erwartet, dass sich die **mittlere Leistung um den Faktor 2 unterscheidet**, je nachdem ob die **Diode aktiv ist oder nicht**.  
    % }
    
}


\Loesung{
    \begin{itemize}
        \item[a)]
        Bei einem idealen Ventil: Nach dem Erreichen der Flussspannung verhält sich die Diode wie ein perfekter Leiter ohne Widerstand. \\
        Da nur eine Halbwelle der Wechselspannung durchgelassen wird, halbiert sich die Leistung im Lastwiderstand. Die Diode selbst setzt keine Leistung um:
        \begin{align*}
            P_\mathrm{D} &= 0\,\mathrm{W}
        \end{align*}
        Die mittlere Leistung im Lastwiderstand beträgt:
        \begin{align*}
            P_\mathrm{RL} &= \frac{1}{2} \cdot \frac{(U_\mathrm{eff})^2}{R_\mathrm{L}} = \frac{1}{2} \cdot \frac{(230\,\mathrm{V})^2}{1\,\mathrm{k\Omega}} = 26,45\,\mathrm{W}
        \end{align*}
    
        \item[b)]
        Bei überbrückter Diode liegt die volle Wechselspannung am Lastwiderstand an:
        \begin{align*}
            P_\mathrm{RL} &= \frac{U_\mathrm{eff}^2}{R_\mathrm{L}} = \frac{230\,\mathrm{V}^2}{1\,\mathrm{k\Omega}} = 52,9\,\mathrm{W}
        \end{align*}
    \end{itemize} 
% \speech{
% Lösung zu Aufgabe 8:

% Die gegebene Schaltung enthält eine Diode in Reihe mit einem Lastwiderstand von 1 Kiloohm. Die Eingangsspannung ist eine Wechselspannung mit einer Effektivspannung von 230 Volt und einer Frequenz von 50 Hertz.

% **Teilaufgabe a:**  
% Es wird davon ausgegangen, dass die Diode ein ideales Ventil ist. Das bedeutet, dass sie nach dem Erreichen der Flussspannung wie ein perfekter Leiter ohne Widerstand funktioniert. In diesem Fall beträgt die Verlustleistung der Diode \( P_D = 0 \) Watt.

% Die Leistung am Lastwiderstand kann mit der Formel berechnet werden:

% \[
% P_{RL} = \frac{U^2}{R} = \frac{U_{eff}^2}{R_L}
% \]

% Durch Einsetzen der gegebenen Werte:

% \[
% P_{RL} = \frac{(230V)^2}{1k\Omega} = 52,9 W
% \]

% Das bedeutet, dass die gesamte Leistung am Lastwiderstand umgesetzt wird.

% **Teilaufgabe b:**  
% Wird die Diode überbrückt, dann liegt die gesamte Wechselspannung ungehindert am Lastwiderstand an. Da sich die Schaltung in diesem Fall wie ein reiner Widerstand verhält, bleibt die berechnete Leistung im Lastwiderstand gleich groß.

% Das bedeutet, dass die mittlere Leistung in beiden Fällen \( 52,9 \) Watt beträgt, unabhängig davon, ob die Diode vorhanden ist oder überbrückt wurde.
% }
Siehe: Abschnitt \ref{sec:Einweggleichrichter}
}




\subsection{Spannungsverlauf in einer Brückengleichrichterschaltung\label{Aufg9}}
% Alt: Aufgabe 9
\Aufgabe{
  Skizzieren Sie die Spannung $U_1$ der gegebenen Schaltung.
  Beachten Sie, dass es sich bei $u_\mathrm{E}$ um eine Wechselspannung handelt. 
  \begin {figure} [H]
     \centering
     \begin{tikzpicture}[]

    \draw (6,1.5) to[open,v>, name=u1] (6,-0.5);
    \draw (-0.5,3)
        to[open,v>, name=u0] (-0.5,0);

        \draw (-0.5,3)
        to[short, o-*] (3,3)
        to[D, -*, l=$D_2$] (4.5,1.5)
        to[short, i, -] (4.5,1.5)
        to[short, -o] (6,1.5);
        
    % Verbindungslinie zum unteren Punkt
    \draw (6,-0.5) to[short, o-] (1.5,-0.5) ;
    

    \draw (1.5,-0.5)
        to[short, -*] (1.5,1.5)
        to[D, -*, l=$D_3$] (3,0)
        to[short, -o] (-0.5,0);

    \draw (3,0)
        to[D, -, l=$D_4$] (4.5,1.5);

    \draw (1.5,1.5)
        to[D, -, l=$D_1$] (3,3);


    \varrmore{u0}{$u_\mathrm{E}$};
    \varrmore{u1}{$U_{1}$};
\end{tikzpicture}
     \label{fig:FigDiodeSpannugsverlauf}
  \end {figure}
%   \speech{ **Aufgabe 9**  
% Skizzieren Sie die **Spannung \( U_1 \)** der gegebenen Schaltung.  
% Beachten Sie, dass es sich bei \( u_E \) um eine **Wechselspannung** handelt.  


%  **Beschreibung der Abbildung**  
% Die Abbildung zeigt eine **Brückengleichrichterschaltung** mit folgenden Komponenten:

% 1. **Eingangsspannung \( u_E \)**  
%    - Eine **Wechselspannung**, symbolisiert durch eine **Spannungsquelle mit sinusförmigem Symbol**.  
%    - Der **Eingangsstrom fließt in beide Richtungen** abhängig von der Wechselspannung.  

% 2. **Gleichrichterbrücke mit vier Dioden** (\( D_1, D_2, D_3, D_4 \))  
%    - **Dioden sind in Brückenanordnung** geschaltet.  
%    - Sie sorgen dafür, dass **sowohl die positive als auch die negative Halbwelle der Eingangsspannung gleichgerichtet wird**.  
%    - In jeder Halbwelle **leiten zwei Dioden und sperren zwei Dioden**.  

% 3. **Gleichgerichtete Ausgangsspannung \( U_1 \)**  
%    - Liegt **hinter der Gleichrichterbrücke** an.  
%    - Ist eine **pulsierende Gleichspannung**, da **nur positive Halbwellen durchgelassen werden**.  
%    - **Negative Halbwellen der Wechselspannung werden umgekehrt** und erscheinen ebenfalls **positiv am Ausgang**.  


%  **Funktionsweise der Schaltung**  
% - Während der **positiven Halbwelle** von \( u_E \):  
%   - \( D_1 \) und \( D_2 \) leiten, **\( D_3 \) und \( D_4 \) sperren**.  
%   - Der Strom fließt durch die **Last in eine Richtung**.  

% - Während der **negativen Halbwelle** von \( u_E \):  
%   - \( D_3 \) und \( D_4 \) leiten, **\( D_1 \) und \( D_2 \) sperren**.  
%   - Der Strom fließt erneut durch die **Last in die gleiche Richtung**, da die **negative Spannung umgekehrt wird**.  

% - **Ergebnis:**  
%   - Die **Ausgangsspannung \( U_1 \) hat nur positive Werte** und ist eine **pulsierende Gleichspannung**.  
%   - Die Aufgabe besteht darin, diese **Spannung \( U_1 \) als Skizze** darzustellen.  



%  **Zusammenfassung der Aufgabe**  
% - Die Aufgabe erfordert eine **Skizze der Ausgangsspannung \( U_1 \)**.  
% - Aufgrund der **Brückengleichrichtung** besteht \( U_1 \) aus **positiven Halbwellen** der ursprünglichen Wechselspannung.  
% - Die **negative Halbwelle wird gespiegelt**, sodass **kein negatives Signal am Ausgang erscheint**.  
% - Der Skizze soll der typische Verlauf einer **pulsierenden Gleichspannung** entsprechen.  
% }

}


\Loesung{
  \begin{itemize}
    \item Bei einer negativen Halbwelle fließt der Strom über $D_{\text{2}}$ und $D_{\text{3}}$.
  \end{itemize}
  \begin{figure}[H]
    \centering
    \begin{tikzpicture}
    \draw (6,1.5) to[open,v>, name=u1] (6,-0.5);
    \draw (-0.5,3)
        to[open,v>, name=u0] (-0.5,0);

        \draw (-0.5,3)
        to[short, o-*] (3,3)
        to[D, -*, l=$D_2$] (4.5,1.5)
        to[short, i, -] (4.5,1.5)
        to[short, -o] (6,1.5);
        
    % Verbindungslinie zum unteren Punkt
    \draw (6,-0.5) to[short, o-] (1.5,-0.5) ;
    

    \draw (1.5,-0.5)
        to[short, -*] (1.5,1.5)
        to[D, -*, l=$D_3$] (3,0)
        to[short, -o] (-0.5,0);

        \draw (1.5,3) to[short, i,name=i1](1.6,3);
        \iarrmore{i1}{};
        \draw (3,3) to[short, i,name=i2](4.5,1.5);
        \iarrmore{i2}{};
        \draw (4.5,1.5) to[short, i,name=i3] (6,1.5);
        \iarrmore{i3}{};
        \draw(6,-0.5) to[short,i,name=i4] (1.5,-0.5);
        \iarrmore{i4}{};
        \draw (1.5,-0.5) to[short,i,name=i5](1.5,1);
        \iarrmore{i5}{};
        \draw (1.5,1.5) to[short,i,name=i6](3,0);
        \iarrmore{i6}{};
        \draw (1,0) to[short,i,name=i7] (-0.5,0);
        \iarrmore{i7}{};
        

    \draw (3,0)
        to[D, -, l=$D_4$] (4.5,1.5);

    \draw (1.5,1.5)
        to[D, -, l=$D_1$] (3,3);


    \varrmore{u0}{$u_\mathrm{E}$};
    \varrmore{u1}{$U_{1}$};
   
   \end{tikzpicture}
   
    \label{fig:LsgDiodeSpannungsverlauf1}
  \end{figure}

  \begin{itemize}
    \item Bei einer positiven Halbwelle fließt der Strom über $D_{\text{4}}$ und $D_{\text{1}}$.
  \end{itemize}
  \begin{figure}[H]
    \centering
    \begin{tikzpicture}
    \draw (6,1.5) to[open,v>, name=u1] (6,-0.5);
    \draw (-0.5,3)
        to[open,v>, name=u0] (-0.5,0);

        \draw (-0.5,3)
        to[short, o-*] (3,3)
        to[D, -*, l=$D_2$] (4.5,1.5)
        to[short, i, -] (4.5,1.5)
        to[short, -o] (6,1.5);
        
    % Verbindungslinie zum unteren Punkt
    \draw (6,-0.5) to[short, o-] (1.5,-0.5) ;
    

    \draw (1.5,-0.5)
        to[short, -*] (1.5,1.5)
        to[D, -*, l=$D_3$] (3,0)
        to[short, -o] (-0.5,0);

        \draw (1.6,3) to[short, i,name=i1](1.5,3);
        \iarrmore{i1}{};
        \draw (1.5,1.5) to[short, i,name=i2](3,3);
        \iarrmore{i2}{};
        \draw (4.5,1.5) to[short, i,name=i3] (6,1.5);
        \iarrmore{i3}{};
        \draw(6,-0.5) to[short,i,name=i4] (1.5,-0.5);
        \iarrmore{i4}{};
        \draw (1.5,-0.5) to[short,i,name=i5](1.5,1);
        \iarrmore{i5}{};
        \draw (3,0) to[short,i,name=i6](4.5,1.5);
        \iarrmore{i6}{};
        \draw (1,0) to[short,i,name=i7] (1.5,0);
        \iarrmore{i7}{};
        

    \draw (3,0)
        to[D, -, l=$D_4$] (4.5,1.5);

    \draw (1.5,1.5)
        to[D, -, l=$D_1$] (3,3);


    \varrmore{u0}{$u_\mathrm{E}$};
    \varrmore{u1}{$U_{1}$};
   
\end{tikzpicture}

    \label{fig:LsgDiodeSpannungsverlauf2}
  \end{figure}
%   \speech{
% Lösung zu Aufgabe 9:

% Die gegebene Schaltung stellt eine Gleichrichterschaltung mit vier Dioden in Brückenschaltung dar. Die Eingangsspannung \( u_E \) ist eine Wechselspannung, die gleichgerichtet wird, sodass am Ausgang \( U_1 \) eine pulsierende Gleichspannung entsteht.

% Die Aufgabe analysiert den Stromfluss während der positiven und negativen Halbwelle der Eingangsspannung.

% **Erster Fall: Negative Halbwelle der Eingangsspannung**  
% Wenn die Eingangsspannung negativ ist, dann ist die untere Seite der Spannungsquelle positiver als die obere. Dadurch werden die Dioden \( D_2 \) und \( D_3 \) leitend, während \( D_1 \) und \( D_4 \) sperren.  

% - Der Strom fließt von der positiven Klemme der Spannungsquelle nach oben in die Anode von \( D_2 \).
% - Er passiert \( D_2 \) in Durchlassrichtung, gelangt zur Last und fließt dann weiter über \( D_3 \) in Richtung der negativen Klemme der Spannungsquelle.
% - Dadurch wird die Ausgangsspannung \( U_1 \) in der gleichen Polarität gehalten.

% **Zweiter Fall: Positive Halbwelle der Eingangsspannung**  
% Wenn die Eingangsspannung positiv ist, dann ist die obere Seite der Spannungsquelle positiver als die untere. Dadurch werden die Dioden \( D_1 \) und \( D_4 \) leitend, während \( D_2 \) und \( D_3 \) sperren.  

% - Der Strom fließt nun von der positiven Klemme der Spannungsquelle in die Anode von \( D_1 \).
% - Er passiert \( D_1 \) in Durchlassrichtung, gelangt zur Last und fließt dann über \( D_4 \) zurück zur negativen Klemme der Spannungsquelle.
% - Auch in diesem Fall bleibt die Polarität von \( U_1 \) erhalten, sodass eine gleichgerichtete Spannung entsteht.

% **Fazit:**  
% Diese Brückengleichrichterschaltung sorgt dafür, dass unabhängig von der Polarität der Eingangsspannung immer eine positive Spannung am Ausgang entsteht. Die Strompfade wechseln mit jeder Halbwelle der Wechselspannung, wobei immer zwei Dioden leitend sind, während die anderen beiden sperren.
% }
Siehe: Abschnitt \ref{sec:Brückengleichrichter}
}



 
\subsection{Berechnung des Vorwiderstands einer LED\label{Aufg10}}
% Alt: Aufgabe 10
\Aufgabe{
  Wie bei allen Dioden muss auch bei Leuchtdioden (LEDs) der Strom begrenzt werden. Dazu wird ein Vorwiderstand in Reihe geschaltet, wie in der unten gezeigten Schaltung. \\
  Die eingesetzte LED besitzt eine Durchlassspannung von $U_\mathrm{D} = 2,2\,\mathrm{V}$. Die ideale Helligkeit wird bei einem Strom von $I_\mathrm{D} = 30\,\mathrm{mA}$ erreicht. \\
  Die Eingangsspannung beträgt $U_\mathrm{0} = 5\,\mathrm{V}$.
  
  \begin{itemize}
    \item Welchen der folgenden Widerstandswerte würden Sie für $R$ verwenden, um die LED möglichst nahe an ihrem optimalen Betriebspunkt (d.\,h. mit optimaler Helligkeit) zu betreiben?
    \item $15\,\Omega$, $33\,\Omega$, $68\,\Omega$, $150\,\Omega$, $180\,\Omega$
  \end{itemize}
  
  \begin {figure} [H]
     \centering
     \begin{tikzpicture}
  \draw (0,2) to[V, name=U0] (0,0);
  \draw (0,2) to[R=$R_\mathrm{V}$, i, name=Id] (4,2);
  \draw (4,2) to[leDo, l=LED, v, name=Ud] (4,0) -- (0,0);

  \varrmore{U0}{$U_\mathrm{0}$};
  \varrmore{Ud}{$U_\mathrm{D}$};
  \iarrmore{Id}{$I_\mathrm{D}$};
\end{tikzpicture}
     \label{fig:FigLEDVorwiderstand}
  \end {figure}
%   \speech{ **Aufgabe 10**  
% Wie bei allen **Dioden** ist es wichtig, den **Strom durch das Bauelement zu begrenzen**.  
% Dazu werden **Vorwiderstände** eingesetzt, wie in der unten gezeigten Schaltung.  

% - Die verwendete **LED** hat eine **Durchlassspannung von \( U_D = 2.2V \)**.  
% - Die **optimale Helligkeit** wird bei einem **Strom von \( I_D = 30mA \)** erreicht.  
% - Die **Eingangsspannung** beträgt **\( U_0 = 5V \)**.  

% 

% **Aufgabenstellung:**  
% - Welchen der folgenden **Widerstandswerte** würden Sie für **\( R_V \)** wählen, um die LED möglichst **nahe an ihrem optimalen Betriebspunkt** zu betreiben?  
%   - **15Ω, 33Ω, 68Ω, 150Ω, 180Ω**

% 

%  **Beschreibung der Abbildung**  
% Die gezeigte **elektrische Schaltung** stellt eine **einfache LED-Vorwiderstandsschaltung** dar. Sie besteht aus:

% 1. **Spannungsquelle \( U_0 \)**  
%    - Symbolisiert durch eine **Gleichspannungsquelle mit 5V**.  
%    - Stellt die benötigte **Betriebsspannung für die LED** bereit.  

% 2. **Vorwiderstand \( R_V \)**  
%    - In **Reihe mit der LED geschaltet**.  
%    - Begrenzt den **Stromfluss durch die LED**, um Schäden zu vermeiden.  

% 3. **LED mit Durchlassspannung \( U_D \)**  
%    - Platziert **nach dem Vorwiderstand**.  
%    - Leuchtet, sobald die **Betriebsspannung \( U_0 \)** groß genug ist.  
%    - Hat eine **Durchlassspannung von 2.2V**.  
%    - Der **gewählte Widerstandswert beeinflusst die Stromstärke und damit die Helligkeit**.  

% 

% ### **Funktionsweise der Schaltung**  
% - Die **Gesamtspannung von 5V** teilt sich auf **Vorwiderstand \( R_V \)** und **LED \( U_D \)** auf.  
% - Der **Widerstand \( R_V \)** muss so gewählt werden, dass der **gewünschte Strom von 30mA** fließt.  
% - Durch **Ohm’sches Gesetz** lässt sich die **notwendige Widerstandswert berechnen**.  
% - Zu hohe Werte von **\( R_V \)** würden **die LED dunkler machen**, zu niedrige Werte **könnten die LED beschädigen**.  

% }

}


\Loesung{
Berechnung der Spannung am Vorwiderstand:
\begin{equation}
    U_\mathrm{V} = U_\mathrm{0} - U_\mathrm{D}
\end{equation}
\begin{align*}
    U_\mathrm{V} = 5\,\mathrm{V} - 2,2\,\mathrm{V} = 2,8\,\mathrm{V}
\end{align*}

Berechnung des optimalen Vorwiderstands für \( I_\mathrm{D} = \mathrm{30\,mA} \):
\begin{equation}
    R_\mathrm{V} = \frac{U_\mathrm{V}}{I_\mathrm{D}}
\end{equation}
\begin{align*}
    R_\mathrm{V} = \frac{2,8\,\mathrm{V}}{30\,\mathrm{mA}} = 93,3\,\Omega
\end{align*}

Berechnung des Stroms bei Auswahl eines konkreten Widerstandswertes:
\begin{equation}
    I_\mathrm{D} = \frac{U_\mathrm{V}}{R}
\end{equation}
\begin{align*}
    I_\mathrm{D} &= \frac{2,8\,\mathrm{V}}{150\,\Omega} = 18,7\,\mathrm{mA}
\end{align*}

Berechnung bei zwei Widerständen in Parallelschaltung mit zusätzlichem Serienwiderstand:
\begin{align*}
    R_\mathrm{ges} &= \frac{150\,\Omega \cdot 180\,\Omega}{150\,\Omega + 180\,\Omega} + 15\,\Omega = 96,8\,\Omega \\
    I_\mathrm{D} &= \frac{2,8\,\mathrm{V}}{96,8\,\Omega} =28,9\,\mathrm{mA}
\end{align*}

Alternative Reihenschaltung:
\begin{align*}
    R_\mathrm{ges} &= 4 \cdot 15\,\Omega + 33\,\Omega = 93\,\Omega
\end{align*}

\textbf{Fazit:}  
Ein kombinierter Widerstand aus $4 \times 15\,\Omega + 33\,\Omega = 93\,\Omega$ oder eine Parallelschaltung mit $150\,\Omega \,\|\, 180\,\Omega + 15\,\Omega = 96,8\,\Omega$ führt zu einem Strom nahe dem optimalen Wert von  $30\,\mathrm{mA}$. \\
\textbf{Empfohlene Wahl:} Kombination mit Gesamtwiderstand nahe $93\,\Omega$.

  Siehe: Abschnitt \ref{sec:SpezielleDioden}
%   \speech{
% Lösung zu Aufgabe 10:

% In dieser Aufgabe geht es um eine Schaltung mit einer Leuchtdiode (LED), die durch einen Vorwiderstand geschützt wird. Ziel ist es, den optimalen Widerstandswert für die LED zu bestimmen, sodass sie mit der idealen Helligkeit betrieben wird.

% **Gegebene Werte:**
% - Die LED besitzt eine Durchlassspannung von 2,2 Volt.
% - Der Strom durch die LED sollte 30 Milliampere betragen.
% - Die Eingangsspannung beträgt 5 Volt.

% **Formel zur Berechnung der Spannung über dem Vorwiderstand \(R_V\):**
% Die Spannung über dem Vorwiderstand berechnet sich durch die Differenz zwischen der Eingangsspannung und der Durchlassspannung der LED:
% \[
% U_V = U_0 - U_D = 5V - 2,2V = 2,8V
% \]

% **Formel zur Berechnung des Vorwiderstandes \(R_V\):**
% Der Widerstand kann mit dem Ohmschen Gesetz berechnet werden:
% \[
% R_V = \frac{U_V}{I_D} = \frac{2,8V}{30mA} = 93,3 \Omega
% \]
% Das bedeutet, dass ein Widerstand von etwa 93 Ohm nötig ist, um den gewünschten Strom von 30 Milliampere durch die LED fließen zu lassen.

% **Berechnung für verschiedene Widerstände:**
% Nun wird überprüft, welche der vorgegebenen Widerstandswerte der optimalen Helligkeit am nächsten kommen.

% 1. **Für einen Widerstand von 150 Ohm:**
%    \[
%    I_D = \frac{2,8V}{150 \Omega} = 18,7mA
%    \]
%    Dies ist ein geringerer Strom als ideal, wodurch die LED weniger hell leuchten würde.

% 2. **Für einen Widerstand von 180 Ohm:**
%    \[
%    I_D = \frac{2,8V}{180 \Omega} = 15,6mA
%    \]
%    Auch hier würde die LED dunkler als optimal leuchten.

% 3. **Für eine Parallelschaltung von 150 Ohm und 180 Ohm:**
%    Der Gesamtwiderstand berechnet sich als:
%    \[
%    R_{gesamt} = \frac{150 \cdot 180}{150 + 180} = 96,8 \Omega
%    \]
%    Damit ergibt sich ein Strom durch die LED von:
%    \[
%    I_D = \frac{2,8V}{96,8 \Omega} = 28,9mA
%    \]
%    Dies ist sehr nahe am optimalen Wert von 30 Milliampere.

% **Fazit:**
% Der Widerstand von **93,3 Ohm** wäre optimal, allerdings ist dieser Wert nicht direkt in der Auswahl enthalten. Die beste praktische Wahl ist die Parallelschaltung von **150 Ohm und 180 Ohm**, da sie einen Widerstand von 96,8 Ohm ergibt und somit einen Stromfluss von 28,9 Milliampere ermöglicht, was der gewünschten LED-Helligkeit am nächsten kommt.
% }

}




\subsection{Identifikation eines Bauteils anhand des Schaltzeichens\label{Aufg11}}
% Alt: Aufgabe 11
\Aufgabe{
  Was für ein Bauelement wird mit folgendem Symbol gekennzeichnet?
  \begin {figure} [H]
     \centering
     \begin{tikzpicture}

   \draw
   node[npn, tr circle] (Q1) {}
   (Q1.base) 
   (Q1.collector)  
   (Q1.emitter)  ;
\end{tikzpicture}
     \label{fig:FigSchaltzeichen}
  \end {figure}
%   \speech{ **Aufgabe 11**  
% **Fragestellung:**  
% Was für ein **Bauelement** wird mit folgendem **Schaltzeichen** gekennzeichnet?

% ---

%  **Beschreibung der Abbildung**  
% Das gezeigte **elektrische Symbol** stellt ein **Halbleiterbauelement** dar, das aus drei Anschlüssen besteht:

% 1. **Kreisförmige Darstellung mit drei Anschlüssen:**  
%    - Symbolisiert ein **transistorbasiertes Halbleiterbauelement**.  
   
% 2. **Drei Anschlüsse:**  
%    - Der **obere Anschluss** entspricht dem **Kollektor (C)**.  
%    - Der **mittlere waagerechte Anschluss** entspricht der **Basis (B)**.  
%    - Der **untere Anschluss mit einem Pfeil** entspricht dem **Emitter (E)**.  
   
% }

}


\Loesung{
  Beim dargestellten Bauelement handelt es sich um einen npn-Transistor.
  Siehe: Abschnitt \ref{fig:UebersichtBipolartransistoren} 

  %     \speech{
  % Lösung zu Aufgabe 11:
  
  % Das dargestellte Bauelement in der Abbildung ist ein **npn-Transistor**.  
  % Ein npn-Transistor besteht aus drei Schichten: einer **n-dotierten** Schicht (Emitter), einer **p-dotierten** Schicht (Basis) und einer weiteren **n-dotierten** Schicht (Kollektor).  
  % Der Stromfluss erfolgt, wenn eine positive Spannung an der Basis anliegt und Elektronen vom Emitter zum Kollektor transportiert werden.  
  % Typischerweise wird ein npn-Transistor für Verstärkerschaltungen und als elektronischer Schalter verwendet.
  % }
  
}




\subsection{Ermittlung von Basis-Emitter-Spannung und Basisstrom\label{Aufg12}}
Gegeben ist die abgebildete Transistorschaltung mit den folgenden Parametern: $U_0=\mathrm{5\,V}$ ,
$U_L=\mathrm{15\,V}$, $R_\mathrm{V}=\mathrm{100\,k\Omega}$ und $R_\mathrm{V}=\mathrm{500\,\Omega}$.
% Alt: Aufgabe 12
\Aufgabe{
  \begin{itemize}
    \item[a)] Wie groß ist die Spannung $U_\mathrm{BE}$ näherungsweise, wenn es sich um einen Siliziumtransistor handelt?
    \item[b)] Wie groß ist der Strom $I_\mathrm{B}$?
  \end{itemize}
  \begin{figure}[H]
    \centering
    \begin{tikzpicture}
     \draw (4,2) node[npn, tr circle] (Q1) {};
 
     \draw (Q1.C) -- (4,4) 
     to [R,l=$R_\mathrm{L}$] (7,4)
     to [V, name=Ul, v] (7,0);

     \draw (Q1.E) to [short, -*] (4,0) 
     -- (7,0);

     \draw (Q1.B) to [R,l_=$R_\mathrm{V}$, i<, name=Ib] (0,2)
     to [V, name=U0, v] (0,0)
     to [short] (4,0);

     \draw (Q1.B) ++ (-0.25,0) coordinate(Tb);
     \draw (Q1.E) ++ (0,-0.25) coordinate(Te);
     \draw (Tb) to [open, name=ube, v, voltage=european] (Te);

     \varrmore{ube}{$U_\mathrm{BE}$};
     \varrmore{U0}{$U_\mathrm{0}$};
     \varrmore{Ul}{$U_\mathrm{L}$};
     \iarrmore{Ib}{$I_\mathrm{B}$};
 \end{tikzpicture}
    \label{fig:FigTransistorBasisstrom}
  \end{figure}  
}

\Loesung{
  \begin{itemize}
    \item[a)]
  \end{itemize}
  \begin{equation}
    U_\text{BE} = U_\text{B} - U_\text{E}
  \end{equation}
  Ein npn-Transistor hat im leitenden Zustand eine typische Basis-Emitter-Spannung von:
  \begin{equation}
    U_\text{BE} \approx 0,6\,\text{V} \text{ bis } 0,7\,\text{V}
  \end{equation}
  Diese Spannung ist eine materialbedingte Eigenschaft des Silizium-Transistors und bleibt nahezu konstant, unabhängig von der Basisspannung \( U_\text{B} \).
  - Obwohl \( U_\text{B} = 5\,\text{V} \) ist, begrenzt der Transistor die Spannung \( U_{BE} \) auf ca. \( 0,6\,\text{V} \), sobald der Transistor leitend ist.  
  - Der überschüssige Spannungsabfall (\( U_B - U_{BE} \)) wird durch andere Ströme und Bauteile der Schaltung ausgeglichen.
  \begin{align*}
    {U_{BE} \approx 0,6\,\text{V}}
  \end{align*}
  \begin{itemize}
    \item[b)] $I_\text{B} = \frac{5\,\text{V} - 0,6\,\text{V}}{100\,\text{k}\Omega} = 44\,\mu\text{A}$
  \end{itemize}
  Siehe: Abschnitt \ref{sec:Bipolartransistor} und \ref{fig:StroemeUndSpannungenBeimBipolartransistor}
}

\subsection{Berechnung der Widerstände einer Transistorschaltung\label{Aufg13}}
% Alt: Aufgabe 13
\Aufgabe{
  Gegeben ist die abgebildete Transistorschaltung mit den folgenden Parametern: $U_0=\mathrm{5\,V}$ , 
  $U_\mathrm{R1}=\mathrm{0,6\,V}$, $U_\mathrm{L}=\mathrm{15\,V}$, $R_\mathrm{1}=\mathrm{300\,\Omega}$, $R_\mathrm{2}=\mathrm{2\,k\Omega}$ und $R_\mathrm{V}=\mathrm{500\,\Omega}$.
  Damit der gezeichnete Siliziumtransistor schaltet, braucht er einen Basisstrom von $50 \, \mathrm{\mu A}$ um 
  durchzuschalten. 
  
  \begin{figure}[H]
    \centering
    \begin{tikzpicture}
   \draw (4,2) node[npn, tr circle] (Q1) {};

   \draw (Q1.C) -- (4,4) 
   to [R,l=$R_\mathrm{L}$] (7,4)
   to [V, name=Ul, v] (7,0);

   \draw (Q1.E) to [short, -*, i, name=Ie] (4,0) 
   -- (7,0);

  \draw (Q1.B) to [R,l_=$R_\mathrm{V}$, i<, name=Ib] (0,2)
  to [R,l=$R_\mathrm{2}$] (0,0)
  to [short] (4,0);

  \draw (0,2) to [R,l_=$R_\mathrm{1}$, *-, v^<, name=Ur1] (0,4)
  -- (-2,4)
  to [V, name=U0, v] (-2,0)
  to [short, -*] (0,0) ;

   \varrmore{U0}{$U_\mathrm{0}$};
   \varrmore{Ur1}{$U_\mathrm{R1}$};
   \varrmore{Ul}{$U_\mathrm{L}$};
   \iarrmore{Ib}{$I_\mathrm{B}$};
   \iarrmore{Ie}{$I_\mathrm{E}$};
\end{tikzpicture}
    \label{fig:FigTransistorWiderstaende}
  \end{figure}
  
  \begin{itemize}
    \item[a)] Wie groß muss der Widerstand $R_\mathrm{V}$ gewählt werden, damit der Transistor durchschaltet?
    \item[b)] Angenommen der Spannungsabfall des Transistors $U_\mathrm{CE}$ beträgt $7,5 \, \mathrm{V}$, wie groß ist der Strom
          durch $I_\mathrm{E}$?
  \end{itemize}
  
  
  % \speech{Aufgabe 13  
  
  % Damit der gezeichnete Siliziumtransistor schaltet, braucht er einen Basisstrom von \( 50 \mu A \) um durchzuschalten.  
  
  % a) Wie groß muss der Widerstand \( R_V \) gewählt werden, damit der Transistor durchschaltet?  
  
  % b) Angenommen der Spannungsabfall des Transistors \( U_{CE} \) beträgt \( 7,5V \), wie groß ist der Strom durch \( I_E \)?  
  
  % Beschreibung der Abbildung  
  
  % Die Abbildung zeigt eine **Transistorschaltung** mit einer **Spannungsquelle von 5V** auf der linken Seite, die über einen **300Ω-Widerstand** mit einem **weiteren 300Ω-Widerstand (\( R_V \))** verbunden ist. Parallel zu \( R_V \) ist ein **2kΩ-Widerstand** geschaltet. Diese Schaltung speist den **Basisstrom \( I_B \)** in die Basis eines **npn-Transistors**.  
  
  % Die Basis-Emitter-Spannung \( U_{BE} \) ist mit einem blauen Pfeil markiert. Der **Basisstrom \( I_B \)** ist mit einem roten Pfeil gekennzeichnet.  
  
  % Auf der rechten Seite befindet sich eine **zweite Spannungsquelle mit 15V**, die über einen **500Ω-Kollektorwiderstand** mit dem **Kollektor des Transistors** verbunden ist. Der **Emitterstrom \( I_E \)** ist mit einem roten Pfeil dargestellt.  
  
  % Die Aufgabe besteht darin, \( R_V \) zu berechnen, um den erforderlichen Basisstrom \( I_B \) sicherzustellen, sowie den Emitterstrom \( I_E \) zu bestimmen, wenn der Transistor eine Kollektor-Emitter-Spannung \( U_{CE} = 7,5V \) aufweist.  
  % }  
  
}


\Loesung{
  \begin{itemize}
    \item[a)]
          \[
            \frac{5\,\text{V}}{2\,\text{k}\Omega + 0,3\,\text{k}\Omega} = \frac{U_1}{300\,\Omega}
            \Rightarrow U_\text{1} = 300\,\Omega \cdot \frac{5\,\text{V}}{2\,\text{k}\Omega + 0,3\,\text{k}\Omega} = 0,65\,\text{V}
          \]
          \[
            5\,\text{V} = 0,65\, \text{V} + U_\text{RV} + 0,6\,\text{V}
          \]
          \[
            U_\text{RV} = 5\,\text{V} - 0,65\,\text{V} - 0,6\,\text{V} = 3,75\,\text{V}
          \]
          \[
            R_\text{V} = \frac{3,75\,\text{V}}{50\,\mu \text{A}} = 75\,\text{k}\Omega
          \]
          
          
    \item[b)]
          \[
            I_\text{C} = \frac{15\,\text{V} - 7,5\,\text{V}}{500\,\Omega} = 15\,\text{mA}
          \]
          \[
            I_\text{E} = I_\text{C} + I_\text{B} = 15\,\text{mA} + 50\,\mu \text{A} = 15,05\,\text{mA}
          \]
          \[
            \frac{I_\text{C}}{I_\text{B}} = \beta = 300
          \]
  \end{itemize}
  Siehe: Abschnitt \ref{sec:Bipolartransistor} und \ref{fig:StroemeUndSpannungenBeimBipolartransistor}
}
\subsection{Schaltverhalten eines Transistors\label{Aufg14}}
% Alt: Aufgabe 14
\Aufgabe{
    Gegeben ist die abgebildete Transistorschaltung mit den folgenden Parametern: 
    $U_0=\mathrm{5\,V}$ und $U_L=\mathrm{15\,V}$.
  Damit der gezeigte Siliziumtransistor schaltet, benötigt er einen Basisstrom von \( I_\mathrm{B}=\mathrm{50\,\mu A} \), um in den leitenden Zustand zu gelangen.
  
  \begin{itemize}
    \item[a)] Wie groß muss der Vorwiderstand \( R_\mathrm{V} \) gewählt werden, damit der Transistor durchschaltet?
    \item[b)] Der Transistor hat einen Stromverstärkungsfaktor \( \beta = 180 \) und einen Kollektorwiderstand \( R_\mathrm{C} = \mathrm{200\,\Omega} \). 
    Wie groß sind der Kollektorstrom \( I_\mathrm{C} \) und der Spannungsabfall zwischen Kollektor und Emitter \( U_\mathrm{CE} \)?
  \end{itemize}  
  \begin {figure} [H]
     \centering
     \begin{tikzpicture}
    \draw (4,2) node[npn, tr circle] (Q1) {};

    \draw (Q1.C) -- (4,4) 
    to [R,l=$R_\mathrm{C}$] (7,4)
    to [V, name=Ul, v] (7,0);

    \draw (Q1.E) to [short, -*] (4,0) 
    -- (7,0);

   \draw (Q1.B) to [R,l_=$R_\mathrm{V}$, i<, name=Ib] (0,2)
   to [V, name=U0, v] (0,0)
   to [short] (4,0);

    \varrmore{U0}{$U_\mathrm{0}$};
    \varrmore{Ul}{$U_\mathrm{L}$};
    \iarrmore{Ib}{$I_\mathrm{B}$};
\end{tikzpicture}
     \label{fig:FigTransistorSchaltverhalten}
  \end {figure}
  % \speech{Aufgabe 14  

  % Damit der gezeichnete Siliziumtransistor schaltet, braucht er einen Basisstrom von \( 50 \mu A \) um durchzuschalten.  
  
  % a) Wie groß muss der Widerstand \( R_V \) gewählt werden, damit der Transistor durchschaltet?  
  
  % b) Der Transistor hat eine Stromverstärkungsfaktor \( \beta = 180 \) und einen Kollektorwiderstand \( R_C \) von \( 200 \Omega \).  
  %    Wie groß ist der Kollektorstrom \( I_C \) und der Spannungsabfall zwischen Kollektor-Emitter \( U_{CE} \)?  
  
  % Beschreibung der Abbildung  
  
  % Die Abbildung zeigt eine Transistorschaltung mit einer **Spannungsquelle von \( 5V \)** auf der linken Seite, die über einen **Widerstand \( R_V \)** mit der Basis eines **npn-Transistors** verbunden ist.  
  
  % Die Basis-Emitter-Spannung \( U_{BE} \) ist mit einem blauen Pfeil gekennzeichnet, und der **Basisstrom \( I_B \)** ist mit einem roten Pfeil dargestellt.  
  
  % Auf der rechten Seite der Schaltung befindet sich eine **weitere Spannungsquelle mit \( 15V \)**, die über einen **Kollektorwiderstand \( R_C \)** mit dem **Kollektor des Transistors** verbunden ist. Der **Kollektorstrom \( I_C \)** fließt durch den Widerstand \( R_C \) und ist mit einem roten Pfeil markiert.  
  
  % Die Aufgabe besteht darin, den notwendigen Basiswiderstand \( R_V \) zu berechnen und aus der gegebenen Stromverstärkung \( \beta \) den **Kollektorstrom \( I_C \)** sowie den Spannungsabfall \( U_{CE} \) zu bestimmen.  
  % }  
  

  }


\Loesung{
\begin{itemize}
    \item[a)]
    Berechnung des Vorwiderstands für den Basisstrom:
    \begin{equation}
        R_\mathrm{V} = \frac{U}{I}
    \end{equation}
    \begin{align*}
        R_\mathrm{V} &= \frac{\mathrm{5\,V} - \mathrm{0{,}6\,V}}{\mathrm{50\,\mu A}} = \frac{\mathrm{4{,}4\,V}}{\mathrm{50\,\mu A}} = \mathrm{88\,k\Omega}
    \end{align*}

    \item[b)]
    Berechnung des Kollektorstroms mit Stromverstärkungsfaktor:
    \begin{equation}
        \beta = \frac{I_\mathrm{C}}{I_\mathrm{B}}
    \end{equation}
    \begin{align*}
        I_\mathrm{C} &= \beta \cdot I_\mathrm{B} = 180 \cdot \mathrm{50\,\mu A} = \mathrm{9\,mA}
    \end{align*}

    Berechnung des Spannungsabfalls am Kollektorwiderstand:
    \begin{align*}
        U_\mathrm{RC} &= I_\mathrm{C} \cdot R_\mathrm{C} = \mathrm{9\,mA} \cdot \mathrm{200\,\Omega} = \mathrm{1{,}8\,V}
    \end{align*}

    Berechnung der Spannung zwischen Kollektor und Emitter:
    \begin{align*}
        U_\mathrm{CE} &= \mathrm{15\,V} - \mathrm{1{,}8\,V} = \mathrm{13{,}2\,V}
    \end{align*}
\end{itemize}

    Siehe: Abschnitt \ref{sec:Bipolartransistor} und \ref{fig:StroemeUndSpannungenBeimBipolartransistor}
%     \speech{
% **Aufgabe 14 – Berechnung des Basiswiderstands und der Verstärkung eines npn-Transistors**

% Gegeben ist eine Schaltung mit einem **npn-Transistor**, der durch einen **Basiswiderstand** gesteuert wird. Die Aufgabe besteht darin, den **Basiswiderstand \( R_V \)** sowie die **Stromverstärkung \( \beta \)** und die zugehörigen Spannungen zu berechnen.

% ---

% ### Teil a) Berechnung des Widerstands \( R_V \)
% Die Formel für den Widerstand lautet:

% \[
% R_V = \frac{U}{I}
% \]

% Gegeben sind:
% - Eingangsspannung \( U = 5V \)
% - Basis-Emitter-Spannung \( U_{BE} = 0,6V \)
% - Basisstrom \( I_B = 50 \mu A \)

% Einsetzen der Werte:

% \[
% R_V = \frac{5V - 0,6V}{0,05 mA}
% \]

% \[
% R_V = \frac{4,4V}{0,05 mA} = 88 kΩ
% \]

% **Ergebnis**: Der Basiswiderstand \( R_V \) beträgt **88 kΩ**.

% ---

% ### Teil b) Berechnung der Stromverstärkung \( \beta \) und der zugehörigen Spannungen
% Die Formel für den Verstärkungsfaktor \( \beta \) lautet:

% \[
% \beta = \frac{I_C}{I_B}
% \]

% Gegeben:
% - Basisstrom \( I_B = 50 \mu A \)
% - Stromverstärkungsfaktor \( \beta = 180 \)

% Berechnung des Kollektorstroms \( I_C \):

% \[
% I_C = \beta \cdot I_B
% \]

% \[
% I_C = 180 \cdot 50 \mu A = 9 mA
% \]

% Nun wird die Spannung am **Kollektorwiderstand \( R_C \)** berechnet:

% \[
% U_{RC} = I_C \cdot R_C
% \]

% Gegeben:
% - \( R_C = 200Ω \)
% - \( I_C = 9 mA \)

% \[
% U_{RC} = 9 mA \cdot 200Ω = 1,8V
% \]

% Die **Kollektor-Emitter-Spannung \( U_{CE} \)** ergibt sich aus:

% \[
% U_{CE} = 15V - U_{RC}
% \]

% \[
% U_{CE} = 15V - 1,8V = 13,2V
% \]

% ---

% **Ergebnisse**:
% - Der Kollektorstrom \( I_C \) beträgt **9 mA**.
% - Die Spannung über dem Kollektorwiderstand \( U_{RC} \) beträgt **1,8V**.
% - Die Kollektor-Emitter-Spannung \( U_{CE} \) beträgt **13,2V**.
% }

}




\subsection{Transistorschaltung mit LED\label{Aufg15}}
% Alt: Aufgabe 15
\Aufgabe{
  Mithilfe eines Bipolartransistor soll eine LED betrieben werden. Dabei muss durch die LED ein Strom von $I_\mathrm{D}=30\,\mathrm{mA}$ fließen, damit sie ihre optimale Leuchtkraft erreicht. \\
  Der verwendete Transistor besitzt einen Stromverstärkungsfaktor von $\beta = 50$ und im durchgeschalteten Zustand eine Kollektor-Emitter-Spannung von $U_\mathrm{CE} = 0,3\,\mathrm{V}$. 
  Zusätzlich sind die folgenden Parameter gegeben: $R_1=3,3\,\mathrm{k\Omega}$,  $R_2=270\,\mathrm{\Omega}$ und $U_\mathrm{L}=9\,\mathrm{V}$. \\
  Welche Spannung $U$ muss an der Basis anliegen, damit der Transistor durchschaltet? 
  \begin {figure} [H]
     \centering
     \begin{tikzpicture}
    % Spannungspfeile in blau
    \draw(8,3) to[open, v>, name=9V] (8,0);
    % Batterie links
    \draw
    (0,1.5) to[V, name=U] (0,0);
    % Widerstand RV
    \draw
    (0,1.5) -- (2,1.5)
    to[R=$3.3\,k\Omega$] (3.5,1.5) to[short,i,name=IB] (4.5,1.5)
    -- (5,1.5) ; % Verbindung mit Basis
    \draw(4.7,1.2) to[open, v>, name=ube] (5.2,0.7);
    % Transistor
    \draw
   (5.5,1.5) node[npn, tr circle] (Q1) {}
   (Q1.base) 
   (Q1.collector)  
   (Q1.emitter)  ;
   
   \draw
   (Q1.C) -- (5.5,3)
   to[R=$270\,\Omega$] (7,3);



   \draw (Q1.E) -- (5.5,0) to[short, -o] (8,0);


   \draw (0,0) -- (5.5,0); 

   \draw (8,3) to[leDo, o-] (7,3);

  \varrmore{U}{$U$};
   \varrmore{9V}{$\mathrm{9\,V}$};
   \iarrmore{IB}{$I_\mathrm{B}$};
   \varrmore{ube}{$U_\mathrm{BE}$};
  \end{tikzpicture}
     \label{fig:FigTransistorLED}
  \end {figure}
}


\Loesung{
Berechnung des Kollektorstroms bei bekannter Spannung und Lastwiderstand:
\begin{equation}
    I_\mathrm{C} = \frac{U}{R}
\end{equation}
\begin{align*}
    I_\mathrm{C} &= \frac{9\,\mathrm{V} - 0,3\,\mathrm{V}}{270\,\Omega} = 32,2\,\mathrm{mA}
\end{align*}

Berechnung des benötigten Basisstroms mit dem Stromverstärkungsfaktor $\beta = 50$:
\begin{equation}
    I_\mathrm{B} = \frac{I_\mathrm{C}}{\beta}
\end{equation}
\begin{align*}
    I_\mathrm{B} &= \frac{32,2\,\mathrm{mA}}{50} = 0,644\,\mathrm{mA}
\end{align*}

Spannungsabfall am Vorwiderstand:
\begin{align*}
    U_\mathrm{V} &= 0,644\,\mathrm{mA} \cdot 3,3\,\mathrm{k\Omega} = 2,12\,\mathrm{V}
\end{align*}

Gesamte Basisspannung (inkl. \( U_\mathrm{BE} \)):
\begin{align*}
    U &= U_\mathrm{V} + U_\mathrm{BE} = 2,12\,\mathrm{V} + 0,6\,\mathrm{V} = 2,72\,\mathrm{V}
\end{align*}

\textbf{Antwort:}  
Die Basisspannung muss \( \boxed{2{,}72\,\mathrm{V}} \) betragen, damit der Transistor durchschaltet.

    Siehe: Abschnitt \ref{sec:Bipolartransistor} und \ref{fig:StroemeUndSpannungenBeimBipolartransistor}
}
\subsection{Analyse des Arbeitsbereichs eines Transistors\label{Aufg16}}
% Alt: Aufgabe 16
\Aufgabe{
  Der Stromverstärkungsfaktor des verwendeten Siliziumtransistors beträgt $ \beta = 150$ und die Versorgungsspannung liegt bei $U_\mathrm{V} = 15\,\mathrm{V}$. Die Schaltung soll so dimensioniert werden, dass der Kollektorstrom $I_\mathrm{C} = 2\,\mathrm{mA}$ beträgt.
  
  \begin{itemize}
    \item[a)] Dimensionieren Sie den Widerstand $R_3$ so, dass bei einer Spannung am Ausgang von $3\,\mathrm{V}$ der Strom durch $R_3$ genau $75\,\%$ des Emitterstroms beträgt.
    
    \item[b)] Bestimmen Sie die Widerstände $R_1$ und $R_2$, unter der Annahme, dass die Kollektor-Basis-Spannung $U_\mathrm{BE} = 0,6\,\mathrm{V}$ beträgt und durch $R_2$ etwa das 9-fache des Basisstroms $I_\mathrm{B}$ fließt.
  \end{itemize}
  \begin {figure} [H]
   \centering
   \begin{tikzpicture}
    \draw (0,0) node[npn, tr circle] (Q1) {};
    
    \draw (Q1.C) to[short, -, i<, name=ic] (0,2)
    to[short, -] ++ (-3, 0) 
    to[R=$R_1$]  (-3, 0) 
    to[R=$R_2$]  (-3, -3) 
    node[ground]{};
    
    \draw (-3,2) to[short, *-o] ++(-1,0)
    to [open, v, name=uv] (-4,-3.5); 
    
    \draw (Q1.B) to[short, -*] (-3, 0);
    
    \draw (Q1.E) to[short, -*] ++(0, -0.5) coordinate(E)
    to[R=$R_3$] (0, -3) node[ground]{};
    
    \draw (E) to[short, -] ++(2, 0)
    to[R=$R_\mathrm{L}$] (2, -3) node[ground]{};
    
    \draw (E)++(2, 0) to[short, *-o] ++(1, 0)
    to[open, v^, name=ua] (3,-3.5)    ;
    
    \varrmore{ua}{$U_\mathrm{A}$};
    \varrmore{uv}{$U_\mathrm{B}$};
    \iarrmore{ic}{$I_\mathrm{C}$};
\end{tikzpicture}
   \label{fig:FigTransistorArbeitsbereich}
\end {figure}

% \speech{Aufgabe 16  

% Der Stromverstärkungsfaktor des Siliziumtransistors beträgt **150** und die Versorgungsspannung \( U_V = 15V \).  
% Die Schaltung soll so dimensioniert werden, dass der Kollektorstrom **2 mA** beträgt.  

% **Aufgabenstellung:**  
% a) Dimensionieren Sie \( R_3 \), sodass bei einer Spannung am Ausgang von **3V** der Strom durch \( R_3 \) **75%** des Emitterstroms beträgt.  

% b) Bestimmen Sie \( R_1 \) und \( R_2 \), nehmen Sie dazu an, dass die **Kollektor-Basis-Spannung** **0,6V** beträgt und durch \( R_2 \) etwa **9 \cdot I_B** fließen.  

% **Beschreibung der Abbildung**  

% Die Abbildung zeigt eine **Transistorverstärkerschaltung** mit einem **npn-Transistor**, mehreren Widerständen und einer **Versorgungsspannung von 15V**.  

% - **Oben:** Die **Versorgungsspannung \( U_V = 15V \)** ist an den Kollektorwiderstand \( R_1 \) angeschlossen.  
% - **Transistor:** Der **Kollektor** ist über \( R_1 \) mit \( U_V \) verbunden, während der **Emitter** über \( R_3 \) und den Lastwiderstand \( R_L \) mit Masse verbunden ist.  
% - **Basisbeschaltung:** Zwei Widerstände \( R_1 \) und \( R_2 \) sorgen für eine stabile **Basisspannung** zur Steuerung des Transistors.  
% - **Lastwiderstand:** \( R_L \) ist parallel zum Ausgang \( u_A \) geschaltet, wodurch das Signal abgegriffen wird.  

% Die Aufgabe besteht darin, die Widerstände **\( R_3, R_1 \) und \( R_2 \)** so zu dimensionieren, dass die gegebenen Bedingungen erfüllt werden.  
% }  

}


\Loesung{
\begin{itemize}
    \item[a)] Berechnung des Basisstroms über den Stromverstärkungsfaktor:
\begin{equation}
    I_\mathrm{B} = \frac{I_\mathrm{C}}{\beta}
\end{equation}
\begin{align*}
    I_\mathrm{B} &= \frac{\mathrm{2\,mA}}{150} = \mathrm{13{,}3\,\mu A} \\
    I_\mathrm{E} &= I_\mathrm{C} + I_\mathrm{B} = \mathrm{2\,mA} + \mathrm{13{,}3\,\mu A} = \mathrm{2{,}013\,mA} \\
    I_\mathrm{R_3} &= 0{,}75 \cdot I_\mathrm{E} = 0{,}75 \cdot \mathrm{2{,}013\,mA} = \mathrm{1{,}51\,mA} \\
    R_3 &= \frac{\mathrm{3\,V}}{\mathrm{1{,}51\,mA}} = \mathrm{1{,}99\,k\Omega}
\end{align*}
    \item[b)] Berechnung des Stroms durch \( R_2 \):
\begin{align*}
    I_{R_2} &= 9 \cdot I_\mathrm{B} = 9 \cdot \mathrm{13{,}3\,\mu A} = \mathrm{0{,}12\,mA} \\
    R_2 &= \frac{U_\mathrm{a} + U_\mathrm{BE}}{I_\mathrm{R_2}} = \frac{\mathrm{3\,V} + \mathrm{0{,}6\,V}}{\mathrm{0{,}12\,mA}} = \mathrm{30\,k\Omega}
\end{align*}

Berechnung des Stroms durch $R_1$, welcher gleich $I_\mathrm{R_2}$ ist:
\begin{align*}
    R_1 &= \frac{U_\mathrm{BE}}{I_\mathrm{R_1}} = \frac{0,6\,\mathrm{V}}{0,12\,\mathrm{mA}} = 5,6\,\mathrm{k\Omega}
\end{align*}
\end{itemize}

  Siehe: Abschnitt \ref{sec:ElektrischesVerhaltenBipolartransistor}
%   \speech{
% Aufgabe 16 behandelt die Berechnung verschiedener Widerstände und Ströme in einer Transistorschaltung mit einem Silizium-Transistor.  

% Teil a)  
% Die Berechnung basiert auf dem Verstärkungsfaktor \(\beta\) und den bekannten Strom- und Spannungswerten.  
% - Der Basisstrom \(I_B\) wird mit der Formel \(I_B = \frac{I_C}{\beta}\) berechnet.  
% - Mit \(I_C = 2\) mA und \(\beta = 150\) ergibt sich \(I_B = 13,3\) µA.  
% - Der Emitterstrom ist die Summe aus Kollektor- und Basisstrom:  
%   \(I_E = I_C + I_B = 2,013\) mA.  
% - Der Strom durch \(R_3\) ist definiert als \(I_{R_3} = 0,75 \times I_E = 1,51\) mA.  
% - Mit \(U = 3\) V folgt für den Widerstand \(R_3 = \frac{3V}{1,51mA} = 1,99\) kΩ.  

% Teil b)  
% Hier werden die Widerstände \(R_1\) und \(R_2\) bestimmt.  
% - Der Strom durch \(R_2\) ergibt sich aus \(I_{R_2} = 9 \times I_B = 0,12\) mA.  
% - Mit \(U_A + U_{BE} = 3V + 0,6V\) berechnet sich \(R_2\) als  
%   \(R_2 = \frac{3V + 0,6V}{0,12mA} = 30\) kΩ.  
% - Für \(R_1\) gilt: \(R_1 = \frac{U_{CE}}{I_{R_1}}\), wobei \(I_{R_1} = 9 \times I_B = 0,12\) mA.  
% - Damit folgt \(R_1 = \frac{0,6V}{9 \times 13,3\) µA} = 5,6 kΩ.  
% }

}




\subsection{Berechnungen in einer Transistorschaltungen\label{Aufg17}}
% Alt: Aufgabe 17
\Aufgabe{
    Gegeben ist die folgende Schaltung mit dem Ausgangskennlinienfeld eines Transistors.
    
    \begin{figure}[H]
        \centering
        \begin{subfigure}[b]{0.45\textwidth}
            \centering
            \begin{tikzpicture} 
    \draw (0,0) node[npn] (Q1) {};
    
    \draw (Q1.C)
    to [short] (0,1)
    to[R, l_= $R_\mathrm{C}$] (0,3)
    to [short, -o] (0,3.5) node[right, color=voltage] {$U_\mathrm{V}$};
    
    \draw (0,1)
    to[short, *-o] (2,1);
    
    \draw (Q1.E)
    to [short, -*] (0,-2) node[ground]{};
    
    \draw (-2,-2)
    to[short, o-o] (2,-2);
    
    \draw (2,1)
    to[open,v^, name=ua] (2,-2);
    
    \draw (Q1.B) to[short, -o] (-2,0)
    to[open,v>, name=ue] (-2,-2);
    
    \varrmore{ua}{$U_\mathrm{CE}$};
    \varrmore{ue}{$U_\mathrm{BE}$};
\end{tikzpicture}
        \end{subfigure}
        \begin{subfigure}[b]{0.45\textwidth}
            \centering
            \includesvg[width=0.9\textwidth]{Bilder/Aufgaben/FigTransistorKennlinien}
        \end{subfigure}
        \label{fig:FigTransistorKennlinien}
    \end{figure}
    
    \begin{itemize}
        \item[a)] Wie groß ist der maximale Kollektorstrom des idealen Transistors (bei $R_\mathrm{CE} = 0\,\Omega$), wenn der Kollektorwiderstand $R_\mathrm{C} = 7,5\,\Omega$ beträgt und die Versorgungsspannung $U_\mathrm{V} = 3\,\mathrm{V}$ ist?
        \item[b)] Bestimmen Sie  $U_\mathrm{CE}$, $U_\mathrm{RC}$ und $I_\mathrm{C}$ des realen Transistors bei einem Basisstrom von \( I_\mathrm{B} = \mathrm{2\,mA} \). Tragen Sie dazu die Widerstandsgerade in das Kennlinienfeld ein.
        \item[c)] Für einen neuen Arbeitspunkt fallen nur noch $1,7\,\mathrm{V}$ über dem Kollektorwiderstand $R_\mathrm{C}$ ab, der Basisstrom beträgt weiterhin $2\,\mathrm{mA}$. Bestimmen Sie den neuen Widerstand $R_\mathrm{C}$.
    \end{itemize}
}


\Loesung{
    \begin{itemize}
        \item[a)] Berechnung des maximalen Kollektorstroms bei idealem Transistor (\( R_\mathrm{CE} = 0 \)):
              \begin{equation*}
                  I_\mathrm{C}  = \frac{U_\mathrm{V}}{R_\mathrm{C}} = \frac{3\,\mathrm{V}}{7,5\,\Omega} = 400\,\mathrm{mA}
              \end{equation*}
              
              
        \item[b)] Bestimmung der Spannungen und des Kollektorstroms laut Kennlinienfeld:
              \begin{figure}[H]
                  \centering
                  \includesvg[width=0.7\textwidth]{Bilder/Aufgaben/LsgTransistorKennlinien}
                  \caption{Kennlinienfeld mit eingezeichneter Widerstandsgerade}
              \end{figure}
              \begin{gather*}
                  U_\mathrm{CE} = 1,2\,\mathrm{V} \\
                  U_\mathrm{RC} = 1,8\,\mathrm{V} \\
                  I_\mathrm{C} = \frac{U_\mathrm{RC}}{R_\mathrm{C}} = \frac{1,8\,\mathrm{V}}{7,5\,\Omega} = 240\,\mathrm{mA}
              \end{gather*}
              
        \item[c)] Berechnung des neuen Widerstandswerts bei gleichem $I_\mathrm{C}$ und neuer Spannung $U_\mathrm{RC} = 1,7\,\mathrm{V}$:
              \begin{equation*}
                  R_\mathrm{C} = \frac{U_\mathrm{RC}}{I_\mathrm{C}} = \frac{1,7\,\mathrm{V}}{240\,\mathrm{mA}} = 7,08\,\Omega
              \end{equation*}
    \end{itemize}
    
    Siehe: Abschnitt \ref{sec:ElektrischesVerhaltenBipolartransistor}
}
\subsection{Berechnung der Stromverstärkung einer Transistorschaltung\label{Aufg18}}
% Alt: Aufgabe 18
\Aufgabe{
    Gegeben ist die folgende Schaltung mit dem Kennlinienfeld des Transistors. Die Versorgungsspannung beträgt \( U_\mathrm{V} = \mathrm{6\,V} \). \\
    Der Arbeitspunkt liegt bei der halben Betriebsspannung.\\
    Bestimmen Sie die folgenden Werte aus dem Kennlinienfeld:
    
    \begin{itemize}
        \item Die Paare \( I_\mathrm{B} / U_\mathrm{BE} \) und \( I_\mathrm{C} / U_\mathrm{CE} \) im Arbeitspunkt,
        \item sowie die Stromverstärkung \( \beta \).
    \end{itemize}
    
    \begin{figure}[H]
        % oben
        \begin{minipage}[c]{\textwidth}
            \centering
            \begin{tikzpicture}
\draw
    (0,0) node[npn, tr circle] (Q1) {}
    (Q1.base) 
    (Q1.collector)  
    (Q1.emitter);

\draw (Q1.base) to[short, -o] (-3,0);

\draw (Q1.emitter) to[short, -*] (0,-3)node[ground]{};

\draw (Q1.collector) to[short, -*] (0,1) to[short, R, l=$R_\mathrm{C}$, -*] (0,3);

\draw (-3,3) -- (3,3) node[above]{$U_\mathrm{CC}$};

\draw (0,1) to[short, -o] (3,1);
\draw (-3,-3) to[short, o-o] (3,-3);

\draw (-3,0) to[open,v>, name=ue] (-3,-3);
\draw (3,1) to[open, v>, name=ua] (3,-3);

\varrmore{ue}{$U_\mathrm{E}$};
\varrmore{ua}{$U_\mathrm{A}$};

\draw (-2,3) to[R,l=$R_{2}$, *-*] (-2,0) to[R,l=$R_{1}$, -*] (-2,-3);
\end{tikzpicture}
        \end{minipage}
        \vspace{0.5cm} % Vertikaler Abstand zur zweiten Reihe
        
        % unten
        \begin{minipage}[c]{0.48\textwidth}
            \centering
            \includesvg[width=\textwidth]{Bilder/Aufgaben/FigTransistorStromverstaerkung2}
        \end{minipage}
        \hfill 
        \begin{minipage}[c]{0.48\textwidth}
            \centering
            \includesvg[width=\textwidth]{Bilder/Aufgaben/FigTransistorStromverstaerkung3}
        \end{minipage}
        \label{fig:FigTransistorStromverstaerkung}
    \end{figure}
    
}


\Loesung{
    \begin{figure}[H]
        \centering
        \begin{subfigure}[b]{0.45\textwidth}
            \centering
            \includesvg[width=0.9\textwidth]{Bilder/Aufgaben/LsgTransistorStromverstaerkung1}
            \label{fig:LsgTransistorStromverstaerkung1}
        \end{subfigure}
        \begin{subfigure}[b]{0.45\textwidth}
            \centering
            \includesvg[width=0.9\textwidth]{Bilder/Aufgaben/LsgTransistorStromverstaerkung2}
            \label{fig:LsgTransistorStromverstaerkung2}
        \end{subfigure}
    \end{figure}
    Gegeben sind folgende Werte im Arbeitspunkt:
    \begin{align*}
        U_\mathrm{CC} & = \mathrm{6\,V}      \\
        U_\mathrm{CE} & = \mathrm{3\,V}      \\
        I_\mathrm{C}  & = \mathrm{6\,mA}     \\
        U_\mathrm{BE} & = \mathrm{0{,}62\,V} \\
        I_\mathrm{B}  & = \mathrm{20\,\mu A}
    \end{align*}
    
    Berechnung der Stromverstärkung \( \beta \):
    \begin{align*}
        \beta & = \frac{I_\mathrm{C}}{I_\mathrm{B}} = \frac{\mathrm{6\,mA}}{\mathrm{20\,\mu A}} = \mathrm{300}
    \end{align*}
    
    Siehe: Abschnitt \ref{sec:ElektrischesVerhaltenBipolartransistor}
}
\subsection{Verhalten eines Transistors bei Wechselspannung\label{Aufg19}}
% Alt: Aufgabe 19
\Aufgabe{
    Was passiert wenn an einen Transistor eine reine Wechselspannung 
    angelegt wird?
    \begin {figure} [H]
       \centering
         \begin{tikzpicture}
    \draw (4,2) node[npn, tr circle] (Q1) {};

    \draw (Q1.E) 
    to [short, -*] (4,0) 
    to[short, -] (8,0)
    to[V, name=Ul, v_<] (8,4)
    to[R,l_=$R_\mathrm{2}$] (4,4) 
    -- (Q1.C);

    \draw (Q1.B) 
    to [R,l_=$R_\mathrm{1}$, i<, name=Ib] (0,2)
    to [sV, name=U0, v_] (0,0)
    to [short] (4,0);

     \draw (Q1.C)++(0, 0.25) 
    to[short, *-o] ++ (1, 0)
    to[open, v^, name=ua, -o] (5,0)
    ;

    \varrmore{U0}{$u_\mathrm{E}$};
    \varrmore{Ul}{$U_\mathrm{L}$};
    \varrmore{ua}{$u_\mathrm{A}$};
    \iarrmore{Ib}{$I_\mathrm{B}$};
\end{tikzpicture}
       \label{fig:FigTransistorWechselspannung}
    \end {figure}
%     \speech{
% **Aufgabe 19**

%  **Aufgabenstellung:**
% Die Aufgabe untersucht das Verhalten eines Transistors, wenn eine **reine Wechselspannung** an seine Basis angelegt wird.

% **Fragestellung:**  
% Was passiert, wenn an einen Transistor eine reine Wechselspannung angelegt wird?

%  **Beschreibung der Abbildung:**

% Die gezeigte Schaltung stellt einen **NPN-Transistor in Emitterschaltung** dar.

% - **Eingangssignal $u_E$**:  
%   - Es handelt sich um eine **Wechselspannungsquelle**, die mit einem **100 k\(\Omega\) Widerstand** in Reihe zur **Basis des Transistors** geschaltet ist.
%   - Der Basisstrom $I_B$ ist durch einen roten Pfeil gekennzeichnet und zeigt in Richtung der Basis.

% - **Kollektorkreis:**
%   - Der **Kollektor ist über einen 500 \(\Omega\) Widerstand** mit einer **15V Gleichspannungsquelle** verbunden.
%   - Ein blauer Pfeil zeigt die **Spannung am Punkt A**, die den Kollektor-Emitter-Spannungsabfall symbolisiert.
%   - Der **Emitter ist direkt mit Masse verbunden**.

%  **Zusätzliche Hinweise für eine blinde Person:**
% - Die **Basis erhält eine Wechselspannung**, sodass sich der Basisstrom \(I_B\) periodisch ändert.
% - Durch die Stromverstärkung des Transistors beeinflusst dies **den Kollektorstrom und damit die Ausgangsspannung an Punkt A**.
% - Die **Wechselspannung an der Basis führt zu einer verstärkten Wechselstromkomponente am Kollektor**.
% - Diese Schaltung könnte zur **Wechselspannungsverstärkung** genutzt werden, wobei die Kollektorspannung eine invertierte Version der Eingangsspannung darstellt.
% }

}



\Loesung{
Wenn an einen Transistor eine reine Wechselspannung angelegt wird, tritt kein kontinuierlicher Verstärkungsbetrieb auf.  

Das führt zu folgenden Effekten:
\begin{itemize}
    \item Verzerrungen im Ausgangssignal, da nur die positiven Halbwellen verstärkt werden.
    \item Die negative Halbwelle wird abgeschnitten, was einer Gleichrichtung des Signals ähnelt.
    \item Es erfolgt keine lineare Verstärkung, da der Transistor zwischen Sperr- und Sättigungszustand wechselt.
\end{itemize}
Siehe: Abschnitt \ref{merk:MerksatzWechselspannung}
% \speech{
% **Lösung zu Aufgabe 19**  

% Wenn an einen Transistor eine reine Wechselspannung angelegt wird, tritt kein kontinuierlicher Verstärkungsbetrieb auf.  
% Das bedeutet, dass die Signalverstärkung nicht gleichmäßig erfolgt und es zu unerwünschten Effekten kommt.  

% **Folgende Effekte treten auf:**  

% - **Verzerrungen im Ausgangssignal:**  
%   Da nur die positiven Halbwellen verstärkt werden, während die negativen nicht beeinflusst oder unterdrückt werden,  
%   entsteht eine ungleichmäßige Signalform.  

% - **Die negative Halbwelle wird abgeschnitten:**  
%   Dies ähnelt einer Gleichrichtung des Signals, da der Transistor während der negativen Halbwelle nicht leitet  
%   und kein Verstärkungseffekt auftritt.  

% - **Keine lineare Verstärkung:**  
%   Da der Transistor zwischen Sperr- und Sättigungszustand wechselt, erfolgt keine gleichmäßige Verstärkung über  
%   den gesamten Signalverlauf.  
%   Der Verstärkungsprozess ist also nicht linear, was bedeutet, dass das Ausgangssignal nicht proportional  
%   zum Eingangssignal verstärkt wird.  

% Daraus lässt sich schließen, dass für einen kontinuierlichen Verstärkungsbetrieb eine Gleichspannungskomponente  
% als Arbeitspunkt erforderlich ist, sodass der Transistor nicht in den Sperrbereich wechselt.
% }

}




\subsection{Erstellung eines Kleinsignal-Ersatzschaltbilds\label{Aufg20}}
% Alt: Aufgabe 20
\Aufgabe{
    Zeichnen Sie das Kleinsignalersatzschaltbild der gegebenen Schaltung.
    \begin {figure} [H]
       \centering
       \begin{tikzpicture} 
    \draw (0,0) node[npn] (Q1) {};
    
    \draw (Q1.C)
    to [short] (0,1)
    to[R, l_= $500 \, \Omega$, -*] (0,3)
    to [short, -o] (0,3.5) node[right, color=voltage] {$U_\mathrm{V}$};
    
    \draw (0,1)
    to[short, *-o] (2,1);
    
    \draw (0,3)
    to[short, -]
    (-2,3) to[R=$75 \, k\Omega$] (-2,1)
    to [short, -*] (-2,0)
    to [short, -o] (-3,0);
    
    \draw (-2,0) to (Q1.B);
    
    \draw (Q1.E)
    to [short, -*] (0,-2) node[ground]{};
    
    \draw (-3,-2) to[short, o-o] (2,-2);
    
    \draw (2,1) to[open,v^, name=ua] (2,-2);
    \draw (-3,0) to[open,v>, name=ue] (-3,-2);
    
    \varrmore{ua}{$U_\mathrm{A}$};
    \varrmore{ue}{$U_\mathrm{E}$};
\end{tikzpicture}
       \label{fig:FigErsatzschaltbild1}
    \end {figure}
%     \speech{
% **Aufgabe 20**

%  **Aufgabenstellung:**
% Die Aufgabe fordert, dass das **Kleinsignal-Ersatzschaltbild** der gegebenen Schaltung gezeichnet wird.

%  **Beschreibung der Abbildung:**
% Die dargestellte Schaltung zeigt eine **Transistorverstärkerschaltung in Emitterschaltung** mit Widerständen zur Einstellung des Arbeitspunkts.

% - **Spannungsversorgung**:  
%   - Die Schaltung wird mit einer **Versorgungsspannung \( U_V \)** betrieben.
%   - \( U_V \) liegt am oberen Ende der Schaltung.

% - **Basisbeschaltung**:  
%   - Ein **75 k\(\Omega\) Widerstand** verbindet die **Basis des Transistors** mit der **Versorgungsspannung \( U_V \)**.  
%   - Dieser Widerstand sorgt für eine **Vorspannung der Basis**, um den Transistor in den gewünschten Arbeitspunkt zu bringen.

% - **Kollektorkreis**:  
%   - Der **Kollektor ist über einen 500 \(\Omega\) Widerstand** mit \( U_V \) verbunden.  
%   - Die **Ausgangsspannung \( u_A \)** wird am Kollektor abgegriffen.

% - **Eingangssignal**:  
%   - Das **Eingangssignal \( u_E \)** ist an die **Basis des Transistors** gekoppelt.  
%   - Das Signal wird verstärkt und als **invertiertes Signal am Kollektor** ausgegeben.

% - **Emitterschaltung**:  
%   - Der **Emitter ist direkt mit Masse verbunden**, was typisch für eine **Emitterschaltung** ist.

%  **Zusätzliche Hinweise für eine blinde Person:**
% - Diese Schaltung ist eine **Grundschaltung eines Verstärkers** mit einem **Transistor als aktives Element**.
% - Das **Kleinsignal-Ersatzschaltbild** wird durch Ersetzen des Transistors durch sein **äquivalentes Modell** erstellt.  
%   - Dabei werden Parameter wie **differentieller Widerstand \( r_e \)** und **Stromverstärkung \( \beta \)** berücksichtigt.
% - Das **Eingangssignal \( u_E \)** wird an der **Basis eingespeist**, und die verstärkte Version erscheint an der **Kollektor-Emitter-Strecke**.
% - Der **75 k\(\Omega\) Widerstand** dient zur **Gleichspannungsvorspannung** der Basis, der **500 \(\Omega\) Widerstand** beeinflusst die **Verstärkung und den Arbeitspunkt**.
% }

}



\Loesung{
    \begin {figure} [H]
       \centering
       \begin{tikzpicture}

\draw (0,0) to[short, -o] (1,0)node[above]{$\mathrm{B}$} -- (3,0) to[short, -*,R, l=$r_\mathrm{BE}$] (3,-2) to[short, -*] (2,-2)node[ground]{} to[short, -o] (1,-2) to[short, -*] (0,-2)node[ground]{} to[R,l=$R_\mathrm{2}$](0,0);

\draw (3,-2) to[short, -*] (4,-2) node[ground]{} to[short, -*] (5,-2) to[short, -*] (6.5,-2) to[short, -*] (7,-2)node[ground]{} to[short, -o] (8,-2)node[below]{$\mathrm{E}$} -- (9,-2) ;

%Stromquelle
\draw (5,0) to[short,i,name=i2](5,-0.5) to[I,name=I] (5,-1.5) -- (5,-2);
\draw (5,0) to[short, -*] (6.5,0) to[R,l=$r_\mathrm{CE}$] (6.5,-2);
\draw (6,0) to[short, -o] (8,0)node[above]{$\mathrm{C}$} -- (9,0) to[R,l=$R_\mathrm{C}$] (9,-2);

\draw (1,0)to[open,v>, name=ue](1,-2);
\draw (8,0)to[open,v>, name=ua](8,-2);

\varrmore{ue}{$U_\mathrm{E}$};
\varrmore{ua}{$U_\mathrm{A}$};
\iarrmore{i2}{$\mathrm{B}\cdot I_\mathrm{B}$};

\end{tikzpicture}
       \label{fig:LsgErsatzschaltbild1}
    \end {figure}
    Siehe: Abschnitt \ref{fig:ErsatzschaltbildEinesBipolartransistors} und \ref{fig:ErsatzschaltbilderEmitter-Kollektor-UndBasisschaltung}.
%     \speech{
% **Ersatzschaltbild für Aufgabe 20**  

% Das dargestellte Ersatzschaltbild zeigt die Kleinsignal-Ersatzschaltung eines Transistorverstärkers.  
% Es besteht aus folgenden Bauteilen und Anschlüssen:  

% 1. **Widerstände:**  
%    - Widerstand \( R_2 \) ist mit dem Punkt B verbunden und geht gegen Masse.  
%    - Der Basis-Emitter-Widerstand \( r_{BE} \) ist parallel zur Basis-Emitter-Strecke geschaltet.  
%    - Der Kollektorwiderstand \( R_C \) ist zwischen Punkt C und Masse geschaltet.  

% 2. **Spannungen und Ströme:**  
%    - \( U_E \) ist die Eingangsspannung an Punkt B, die über \( R_2 \) an die Basis des Transistors gelangt.  
%    - \( U_A \) ist die Ausgangsspannung am Punkt C, die über \( R_C \) abgenommen wird.  
%    - Der Basisstrom \( I_B \) fließt in die Basis des Transistors ein.  
%    - Der Kollektorstrom beträgt \( \beta \cdot I_B \), dargestellt durch die gesteuerte Stromquelle im Modell.  

% 3. **Transistormodell:**  
%    - Der Transistor wird durch eine gesteuerte Stromquelle ersetzt, die den Kollektorstrom \( \beta \cdot I_B \) liefert.  
%    - Die Basis-Emitter-Strecke wird durch den differentiellen Widerstand \( r_{BE} \) modelliert.  
%    - Die Kollektor-Emitter-Strecke ist über \( R_C \) mit Masse verbunden, wodurch die Ausgangsspannung \( U_A \) entsteht.  

% **Zusammenfassung:**  
% Das Ersatzschaltbild veranschaulicht das Verhalten eines bipolaren Transistors im Kleinsignalbetrieb.  
% Es zeigt, wie die Basis-Emitter-Spannung \( U_{BE} \) den Stromfluss steuert und wie die Verstärkung durch  
% die Stromquelle \( \beta \cdot I_B \) im Kollektorzweig realisiert wird.  
% }

}




\subsection{Erstellung eines Kleinsignal-Ersatzschaltbilds\label{Aufg21}}
% Alt: Aufgabe 21
\Aufgabe{
    Zeichnen Sie das Kleinsignalersatzschaltbild der gegebenen Schaltung.
    \begin {figure} [H]
       \centering
       \begin{tikzpicture}
    \draw (0,0) node[npn, tr circle] (Q1) {};

    \draw (Q1.C) to[short, -o, name=ic] (0,2.5)
    node[right, color=voltage]{$U_\mathrm{V}$};

    \draw (0,2)
    to[short, *-] (-3, 2) 
    to[R=$R_1$, -*]  (-3, 0) 
    to[R=$R_2$]  (-3, -3) 
    node[ground]{};

    \draw (Q1.B) 
    to[short, -o] (-5, 0)
    to[open, v, name=ue] (-5,-2.75)    
    to[short, o-] (-5, -3)
    node[ground]{};

    \draw (Q1.E) to[short, -*] ++(0, -0.25)
    to[R=$R_3$] (0, -3) node[ground]{};

    \draw (Q1.E)++(0, -0.25) to[short, -o] ++(2, 0)
    to[open, v^, name=ua] (2,-2.75)    
    to[short, o-] (2,-3)
    node[ground]{};
    
    \varrmore{ue}{$U_\mathrm{E}$};
    \varrmore{ua}{$U_\mathrm{A}$};
\end{tikzpicture}
       \label{fig:FigErsatzschaltbild2}
    \end {figure}

%     \speech{
% **A.21 Aufgabe 21**

%  **Aufgabenstellung:**
% Die Aufgabe besteht darin, das **Kleinsignal-Ersatzschaltbild** der gezeigten Transistorschaltung zu zeichnen.



%  **Beschreibung der Abbildung:**
% Die Abbildung zeigt eine **Transistorverstärkerschaltung in Emitterschaltung mit Basisspannungsteiler**.

%  **1. Spannungsversorgung:**
% - Die Schaltung wird mit einer **Versorgungsspannung \( U_V \)** betrieben, die sich am oberen Rand der Schaltung befindet.
% - Diese Spannung versorgt den **Kollektor** des Transistors über einen **Widerstand \( R_1 \)**.

%  **2. Basisbeschaltung mit Spannungsteiler:**
% - Die Basis des Transistors erhält ihre Vorspannung über **zwei Widerstände \( R_1 \) und \( R_2 \)**.  
% - Diese beiden Widerstände bilden einen **Spannungsteiler**, der die **Basisspannung einstellt**.
% - \( R_1 \) ist zwischen \( U_V \) und der Basis geschaltet, während \( R_2 \) zwischen Basis und Masse liegt.

%  **3. Kollektorkreis:**
% - Der **Kollektor ist über den Widerstand \( R_1 \)** mit der **Versorgungsspannung \( U_V \)** verbunden.
% - Die **Ausgangsspannung \( u_A \)** wird am **Kollektor des Transistors** abgegriffen.

%  **4. Emitterschaltung mit Widerstand \( R_3 \):**
% - Der **Emitter des Transistors** ist mit einem **Widerstand \( R_3 \)** gegen Masse verbunden.
% - Dieser Widerstand beeinflusst die **Verstärkung der Schaltung**.

%  **5. Eingangssignal:**
% - Das **Eingangssignal \( u_E \)** wird über eine separate Leitung an die **Basis des Transistors** eingespeist.

%  **Zusätzliche Hinweise für eine blinde Person:**
% - Die gezeigte Schaltung ist eine **Emitterschaltung mit Basisspannungsteiler**.  
% - Das **Kleinsignal-Ersatzschaltbild** wird durch Ersetzen des Transistors durch sein **äquivalentes Modell** erstellt.  
%   - Dabei werden Parameter wie **differentieller Widerstand \( r_e \)** und **Stromverstärkung \( \beta \)** berücksichtigt.
% - Die Spannungsteiler \( R_1 \) und \( R_2 \) bestimmen die **Gleichspannungsvorspannung der Basis**, während \( R_3 \) die **Gegenkopplung am Emitter** definiert.
% - Das **Eingangssignal \( u_E \)** wird an der **Basis eingespeist**, und die verstärkte Version erscheint als **invertiertes Signal am Kollektor**.
% - Der **Widerstand \( R_3 \)** beeinflusst die **Gesamtverstärkung der Schaltung**, indem er die Emitterspannung anhebt.

% Diese Schaltung wird häufig in **Verstärkerschaltungen** verwendet, um ein **stabiles Arbeitspunktverhalten** zu ermöglichen.
% }

}



\Loesung{
    \begin{figure}[H]
        \centering
        \begin{subfigure}[b]{0.8\textwidth}
            \centering
            \begin{tikzpicture}

    \draw (0,0)
    to[short, o-o] (3,0) node[above]{B}
    --(4,0) 
    to[R,l=$r_\mathrm{BE}$] (4,-2)
    to[short,-*] (6,-2) 
    to[I, i_<, name=ib] (6,0) 
    to[short, -*] (8,0) 
    to[R,l=$r_\mathrm{CE}$] (8,-2) 
    to[short, -o] (7,-2) node[above]{E}
    -- (6,-2);
    
    \draw (8,0)
    to[short,-o] (9,0)node[above]{C}
    -- (10,0) 
    -- (10,-4) 
    to[short, -*] (6,-4) node[ground]{}
    to[short, -o] (0,-4);
    
    \draw (7,-2) to[open,v^, name=ua] (7,-4);
    \draw (1,0) to[R,l=$R_\mathrm{1}$, *-*](1,-4);
    \draw (2,0) to[R,l=$R_\mathrm{2}$, *-*](2,-4);
    \draw (6,-2) to[R,l=$R_\mathrm{3}$](6,-4);
    \draw (0,0) to[open, v>, name=ue] (0,-4);
    
    \varrmore{ua}{$u_\mathrm{a}$};
    \varrmore{ue}{$u_\mathrm{e}$};
    \iarrmore{ib}{$\mathrm{B}\cdot i_\mathrm{B}$};
\end{tikzpicture}
            \label{fig:LsgErsatschaltbild2_1}
        \end{subfigure}
        \vspace{0.5cm} 
        \begin{subfigure}[b]{0.8\textwidth}
            \centering
            \begin{tikzpicture}
\draw (0,0) to[short, o-*] (1,0)node[above]{B} to[short, -*] (2,0) to[R,l=$r_\mathrm{BE}$, -*] (4,0) to[short, -*] (6,0)node[above]{E} to[short,-*] (7,0) to[short, -o] (9,0);
\draw (0,-2) to[short, o-*] (1,-2) -- (1.5,-2)node[ground]{};
\draw (1.5,-2) to[short, -*] (2,-2) to[short, -*] (4,-2) to[short, -*] (6,-2) to[short,-*] (7,-2) to[short, -o] (9,-2);


\draw (9,0) to[open,v>, name=ua] (9,-2);

\draw (1,0) to[R,l=$R_\mathrm{1}$](1,-2);
\draw (2,0) to[R,l=$R_\mathrm{2}$](2,-2);
\draw (4,0) to[R,l=$R_\mathrm{3}$](4,-2)node[below]{C};
\draw (6,0) to[I] (6,-2);
\draw (7,0) to[R,l=$r_\mathrm{CE}$](7,-2);

\draw (0,0) to[open, v>, name=ue] (0,-2);

\varrmore{ua}{$U_\mathrm{a}$};
\varrmore{ue}{$U_\mathrm{e}$};

\end{tikzpicture}
            \label{fig:LsgErsatschaltbild2_2}
        \end{subfigure}
        \label{fig:Loesung21}
    \end{figure}
    Siehe: Abschnitt \ref{fig:ErsatzschaltbildEinesBipolartransistors} und \ref{fig:ErsatzschaltbilderEmitter-Kollektor-UndBasisschaltung}.
%     \speech{
% Die Abbildung zeigt zwei verschiedene Darstellungen einer Transistorschaltung. Die obere Abbildung stellt die vollständige Schaltung eines bipolaren Transistors in Emitterschaltung dar, während die untere Abbildung die Klein-Ersatzschaltung dieses Systems zeigt.

% **Beschreibung der oberen Abbildung (vollständige Transistorschaltung):**  
% - Links befindet sich die Eingangsspannung \( U_e \), die in die Schaltung eingespeist wird.  
% - Zwei Widerstände, \( R_1 \) und \( R_2 \), sind parallel zwischen der Eingangsspannung \( U_e \) und Masse geschaltet.  
% - Diese Widerstände \( R_1 \) und \( R_2 \) legen das Basispotenzial des Transistors fest.  
% - Der Transistor befindet sich mittig in der Schaltung und ist mit seinen drei Anschlüssen wie folgt beschaltet:  
%   - **Basis (B):** Die Basis ist über die beiden Widerstände \( R_1 \) und \( R_2 \) an die Eingangsspannung \( U_e \) gekoppelt. Zudem gibt es eine gekennzeichnete Impedanz \( r_{BE} \), die den Basis-Emitter-Widerstand modelliert.  
%   - **Emitter (E):** Der Emitter ist mit einem Widerstand \( R_3 \) verbunden, der nach unten zur Masse führt.  
%   - **Kollektor (C):** Der Kollektor ist mit einem weiteren Widerstand \( r_{CE} \) verbunden, der zum Ausgang \( U_a \) führt.  
% - Die Ausgangsspannung \( U_a \) befindet sich rechts in der Schaltung und ist mit dem Kollektor verbunden.  

% Die Pfeile in der Schaltung zeigen die Richtung der Spannungen:  
% - **\( U_e \) (Eingangsspannung):** Von oben links nach unten gerichtet.  
% - **\( U_a \) (Ausgangsspannung):** Vom Kollektor des Transistors nach unten.

% **Beschreibung der unteren Abbildung (Klein-Ersatzschaltbild):**  
% - Diese Darstellung zeigt eine vereinfachte Version der oberen Schaltung, in der der Transistor durch seine klein-signaltechnischen Ersatzwerte ersetzt wurde.  
% - Die Basis ist weiterhin mit den Widerständen \( R_1 \) und \( R_2 \) verbunden.  
% - Der Basis-Emitter-Übergang wird durch einen Widerstand \( r_{BE} \) modelliert.  
% - Der Kollektor-Emitter-Widerstand \( r_{CE} \) stellt den Durchgangswiderstand des Transistors dar.  
% - Der Emitter ist über \( R_3 \) mit Masse verbunden.  
% - Die Spannungen \( U_e \) und \( U_a \) sind weiterhin links bzw. rechts gekennzeichnet.  

% Zusammengefasst zeigen diese beiden Abbildungen die Entwicklung von einer vollständigen Transistorschaltung hin zu ihrem Klein-Ersatzschaltbild, das für weitergehende Berechnungen und Analysen verwendet wird.
% }

}




\subsection{Bestimmung der Gate-Source-Spannung eines Feldeffekttransistors\label{Aufg22}}
% Alt: Aufgabe 22
\Aufgabe{
    Bei welcher Spannung an Source-Gate leitet ein Sperrschichttransistor 
    den meisten Strom?
%     \speech{
% **Aufgabe 22**

% ### **Aufgabenstellung:**
% Die Frage beschäftigt sich mit der Charakteristik eines **Sperrschicht-Feldeffekttransistors (JFET)**.

% Es soll bestimmt werden:
% - Bei welcher **Source-Gate-Spannung** \( U_{GS} \) ein **JFET den größten Strom leitet**.
%     }
}



\Loesung{
  \begin{center}
    Bei $-0,7\, \mathrm{V}$
  \end{center}
  Siehe: Abschnitt \ref{sec:ElektrischesVerhaltenFET}
%   \speech{
% **Lösung zu Aufgabe 22**
% bei -0,7 Volt
% }
}




\subsection{Untersuchung des Stromflusses im Feldeffekttransistor (FET)\label{Aufg23}}
% Alt: Aufgabe 22
\Aufgabe{
Wo fließt bei dem dargestellten Transistor der meiste Strom?
\begin {figure} [H]
       \centering
       \begin{circuitikz} 
    \draw(0,0) node[njfet] (njfet) {}
    (njfet.G) node[anchor=east] {G}
    (njfet.D) node[anchor=south] {D}
    (njfet.S) node[anchor=north] {S};
\end{circuitikz}
       \label{fig:FigNJFET}
    \end {figure}

% \speech{
% **Aufgabe 23**

%  **Aufgabenstellung:**
% Es soll bestimmt werden, **wo beim dargestellten Transistor der meiste Strom fließt**.

%  **Beschreibung der Abbildung:**
% In der Abbildung ist ein **Feldeffekttransistor (FET)** dargestellt.  
% - Das Schaltzeichen zeigt einen **n-Kanal Sperrschicht-FET (JFET)**.
% - Die Anschlüsse sind:
%   - **Source (S)** unten,
%   - **Drain (D)** oben,
%   - **Gate (G)** links.
% }


}



\Loesung{
      \begin{center}
        Zwischen Gate und Source.
      \end{center}
      Siehe: Abschnitt \ref{sec:ElektrischesVerhaltenFET}
%       \speech{
% **Lösung zu Aufgabe 23**
% Zwischen Gate und Source.
% }
}




\subsection{Interpretation der Kennlinie eines Sperrschichttransistors\label{Aufg24}}
% Alt: Aufgabe 24
\Aufgabe{
   In welche zwei Bereiche lässt sich die Ausgangskennlinie des 
   Sperrschichttransistors unterteilen?
   
   %    \speech{
   % **Aufgabe 24**
   
   %  **Aufgabenstellung:**
   % Es soll bestimmt werden, **in welche zwei Bereiche sich die Ausgangskennlinie eines Sperrschichttransistors unterteilen lässt**.
   %    }
   
}


\Loesung{

   In Sättigungs- und Widerstandsbereich.

   Siehe: Abschnitt \ref{sec:ElektrischesVerhaltenFET}
   
   %    \speech{
   % **Lösung zu Aufgabe 24**
   % In Sättigungs- und Widerstandsbereich.
   %    }
}
\subsection{Erkennung von Dotierungsmustern bei Transistoren\label{Aufg25}}
% Alt: Aufgabe 25
\Aufgabe{
   Ohne das Anlegen einer Spannung ergeben sich die folgenden Verteilungen von 
   p- und n-Gebieten. Um was für einen Typ von Transistor handelt es sich?
   
   \hspace{4ex}
   
   \begin{itemize}
      \begin{minipage}{.45\textwidth}
         \item [a)]
         %{\small \includesvg[width=0.9\textwidth]{Bilder/Aufgaben/FigTransistorDotierung1}}
         \includesvg[width=0.9\textwidth]{Bilder/Aufgaben/FigTransistorDotierung1}
         \vspace{4ex}
         \item [c)]
         \includesvg[width=0.9\textwidth]{Bilder/Aufgaben/FigTransistorDotierung3}
      \end{minipage}
      \begin{minipage}{.45\textwidth}
         \item [b)]
         \includesvg[width=0.9\textwidth]{Bilder/Aufgaben/FigTransistorDotierung2}
         \vspace{4ex}
         \item [b)]
         \includesvg[width=0.9\textwidth]{Bilder/Aufgaben/FigTransistorDotierung4}
      \end{minipage}
   \end{itemize}
}

\Loesung{
   \begin{itemize}
      \item[\bf a)] n-kanal selbstsperrend (Verarmungstyp)     Siehe: Abschnitt \ref{fig:SperrbereichN-KanalMOSFET}
      \item[\bf b)] p-kanal selbstsperrend (Verarmungstyp)     Siehe: Abschnitt \ref{fig:SperrbereichN-KanalMOSFET}
      \item[\bf c)] n-kanal selbstleitend (Anreicherungstyp)   Siehe: Abschnitt \ref{fig:OhmscherBereichN-KanalMOSFET}
      \item[\bf d)] p-kanal selbstleitend (Anreicherungstyp)   Siehe: Abschnitt \ref{fig:OhmscherBereichN-KanalMOSFET}
   \end{itemize}
}
\subsection{Erklärung des Funktionsprinzips der CMOS-Technologie\label{Aufg26}}
% Alt: Aufgabe 26
\Aufgabe{
   Was sagt das C in CMOS aus?
   \begin{itemize}
      \item[a)] Unterscheidung von Anreicherungs- und Verarmungstyp
      \item[b)] Die Verwendung von JFETs und MOSFETS in einer Schaltung
      \item[c)] Spiegelt den Verlauf der Transferkennlinie wieder
      \item[d)] Weißt auf die Spannungsempfindlichkeit von MOSFETs hin
      \item[e)] In einer Schaltung werden sowohl p- als auch n-Kanal Transistoren verwendet 
   \end{itemize}
}


\Loesung{
   \begin{itemize}
      \item [e)] In einer Schaltung werden sowohl p- als auch n-Kanal Transistoren verwendet
   \end{itemize}

   Siehe: Abschnitt \ref{sec:AufbauTransistor}
}
\subsection{Transistor-Verbundverstärker\label{Aufg27}}
% Alt: Klausuraufgabe 2
\Aufgabe{
  Zwei npn-Bipolartransistoren bilden zusammen einen rauscharmen Verstärker für Signale im Bereich unter $1\, \mathrm{\mu V}$. 
  Die Versorgungsspannung beträgt $U_1=10\,\mathrm{V}$. Am Kollektor des Transistors $\mathrm{T_2}$ liegt eine Spannung von $4\,\mathrm{V}$ gegenüber Masse an.
  Im Arbeitspunkt von $T_1$ und $T_2$ kann für die Basis-Emitter-Spannungen jeweils $U_\mathrm{BE}=600\,\mathrm{mV}$ angenommen werden.
  Die Kollektorströme $\mathrm{I_{C}}$ sind sehr groß gegenüber den Basisströmen $I_\mathrm{B}$ anzunehmen, daher gilt näherungsweise $I_\mathrm{C} = I_\mathrm{E}$.
  Zusätzlich sind die folgenden Parameter gegeben:
  
  $R_1 = 10\, \mathrm{k\Omega}$, 
  $R_2 = 10\, \mathrm{k\Omega}$, 
  $R_3 = 2\, \mathrm{k\Omega}$,
  $R_4 = 300\, \mathrm{k\Omega}$,  
  $R_5 = 10\, \mathrm{M\Omega}$ \\
  $C_1 = 1\, \mathrm{\mu F}$,  
  $C_2 = 1\, \mathrm{\mu F}$ \\
  $U_1 = 10\, \mathrm{V}$,  
  $\hat{u}_\mathrm{AC} = 0,1\, \mathrm{mV}$ \\
  
  \begin {figure} [H]
  \centering
  \begin{tikzpicture}
    \draw (0,0)
    node[npn] (Q1) {$T_1$};

    \draw (Q1.collector)
    to[short, -*] (0,1)
    to [R, l_=$R_\mathrm{1}$,] (0,4);

    \draw (Q1.base)
    to[short, -*] (-1,0)
    to[C, l_=$C_\mathrm{1}$](-4,0)
    to[sV, name=UE, v_] (-4,-4)
    node[ground]{};

    \draw (Q1.emitter)
    -- (0,-1)
    node[ground]{};

    \draw (-1,0)
    -- ++ (0,-2)
    to[R,l=$R_\mathrm{4}$, -*] (3,-2);

    \draw (3,0)
    node[npn] (Q2) {$T_2$};

    \draw (Q2.emitter)
    to[short](3,-2)
    to[R,l=$R_\mathrm{3}$] (3,-4)
    node[ground]{};

    \draw (0,1)
    -- (1,1)
    -- (2,0)
    to[short] (Q2.base);

    \draw (Q2.collector)
    to[short,-*](3,1)
    to[R,l_=$R_\mathrm{2}$](3,4)
    -- (6,4)
    to[V,name=U1, v] (6,2)
    node[ground]{};

    \draw (0,4)
    to[short, -*] (3,4);

    \draw (3,1)
    to[C,l=$C_\mathrm{2}$](6,1)
    to[R,l=$R_\mathrm{5}$](6,-1)
    node[ground]{};

    \varrmore{U1}{$U_\mathrm{1}$};
    \varrmore{UE}{$u_\mathrm{E}(t)$};
\end{tikzpicture}
  \label{fig:FigKlausurVerbundverstaerker}
  \end {figure}
  
  \begin{itemize}
    \item[a)]
          Wie groß ist der maximale Strom durch $R_2$ im Arbeitspunkt ($I_\mathrm{R2}=I_\mathrm{C,\,T2}$)?
    \item[b)]
          Wie groß ist der Strom durch $R_3$ näherungsweise, wenn $R_4 >> R_3$?
    \item[c)]
          Wie groß ist die Spannung, die über $R_3$ abfällt? 
    \item[d)]
          Wie groß ist die Gleichspannung an der Basis von $T_2$ ($U_\mathrm{B,\,T2}=U_\mathrm{CE,\,T1}$)?
    \item[e)]
          Wie groß ist der Strom durch $R_1$ ($I_\mathrm{R1}=I_\mathrm{C,\,T1}$)?
    \item[f)]
          Erläutern Sie in Stichworten die Arbeitspunktstabiliserung der gesamten Schaltung auf Grund der Gegenkopplung über $R_4$.
    \item[g)]
          Wie groß ist die Steilheit $S$ von $T_\mathrm{1}$, wenn zusätzlich $U_\mathrm{T} = 26\, \mathrm{mV}$ gegeben ist?
    \item[h)]
          Wie groß ist die Spannungsverstärkung $A_\mathrm{V} = \frac{\Delta U_\mathrm{CE}}{\Delta U_\mathrm{BE}}$, der ersten Transistorstufe?
    \item[i)]
          Wie groß ist die Spannungsverstärkung $A_\mathrm{V}$ der zweiten Transistorstufe, unter Verwendung der Definition $A_\mathrm{V,T2}=\frac{\Delta U_\mathrm{C}}{\Delta U_\mathrm{B}}$?
          Zur Lösung müssen zusätzlich die Widerstände $R_2$ und $R_3$ berücksichtigt werden.
    \item[j)]
          Wie groß ist die Spannungsverstärkung der beiden Stufen $T_\mathrm{1}$ und $T_\mathrm{2}$ zusammen?
    \item[k)]
          Zeichnen Sie das Kleinsignal-Ersatzschaltbild der gesamten Schaltung. Ersetzen Sie darin die beiden Transistoren im Arbeitspunkt jeweils durch eine steuerbare Stromquelle $S=\beta\cdot i_\mathrm{B}$ und einen differentiellen Eingangswiderstand $r_\mathrm{BE}$.
  \end{itemize}
}



\Loesung{
  \begin{itemize}
    \item[a)]
  \end{itemize}
  \begin{align*}
    I_\mathrm{C} & = \frac{6\,\mathrm{V}}{10\,\mathrm{k\Omega}} = 0,6\,\mathrm{mA}
  \end{align*}
  \begin{itemize}
    \item[b)]
  \end{itemize}
  \begin{align*}
    I_\mathrm{C,T2} & = 0,6\,\mathrm{mA}
  \end{align*}
  \begin{itemize}
    \item[c)]
  \end{itemize}
  \begin{align*}
    U_{\mathrm{R3}} & = 0,6\,\mathrm{mA} \cdot 2\,\mathrm{k\Omega} = 1,2\, \mathrm{V}
  \end{align*}
  \begin{itemize}
    \item[d)]
  \end{itemize}
  \begin{align*}
    1,2\, \mathrm{V} + 0,6\,\mathrm{V} = 1,8\,\mathrm{V}
  \end{align*}
  \begin{itemize}
    \item[e)]
  \end{itemize}
  \begin{align*}
    I_\mathrm{C,T1} & = \frac{8,2\,\mathrm{V}}{10\,\mathrm{k\Omega}} = 0,82\,\mathrm{mA}
  \end{align*}
  \begin{itemize}
    \item[f)]
  \end{itemize}
  \begin{itemize}
    \item Erhöhung von $I_\mathrm{C}$ führt zu einem Spannungsanstieg über $R_4$
    \item dadurch sinkt die Basisspannung von $T_\mathrm{1}$, was $I_\mathrm{C}$ wieder reduziert
    \item negative Rückkopplung stabilisiert Arbeitspunkt
  \end{itemize}
  \begin{itemize}
    \item[g)]
  \end{itemize}
  \begin{align*}
    S & = \frac{I_\mathrm{C,T1}}{U_\mathrm{T}} = \frac{0,82\,\mathrm{mA}}{26\,\mathrm{mA}} = 31,54\, \frac{\mathrm{mA}}{\mathrm{V}}
  \end{align*}
  \begin{itemize}
    \item[h)]
  \end{itemize}
  \begin{align*}
    A_\mathrm{V} & = S \cdot (-R_\mathrm{C}) = -315
  \end{align*}
  \begin{itemize}
    \item[i)]
  \end{itemize}
  \begin{align*}
    A_\mathrm{V} & = -\frac{R_\mathrm{C}}{R_\mathrm{E}} = \frac{10\,\mathrm{k\Omega}}{2\,\mathrm{k\Omega}} = -5
  \end{align*}
  \begin{itemize}
    \item[j)]
  \end{itemize}
  \begin{align*}
    -5 \cdot -315 = 1575 \approx 1600
  \end{align*}
  \begin{itemize}
    \item[k)]
  \end{itemize}
  \begin {figure} [H]
  \centering
  \begin{tikzpicture}
    \draw (0,0) to[short, o-*] (1,0) to[short, -o] (3,0)node[above]{B};
    \draw (5,0)node[above]{C} to[short, o-*] (7,0) to[short,-o] (8,0)node[above]{B};
    \draw (10,0)node[above]{C} to[short, o-*] (12,0) to[short, -*] (13,0) to[short, -o] (15,0);
    \draw (0,-2) to[short, o-*] (1,-2) -- (1.5,-2)node[ground]{};
    \draw (1.5,-2) to[short, -] (3,-2) to[short, -*] (4,-2)node[below]{E} to[short, -*] (7,-2) to[short, -*] (9,-2)node[below]{E} to[short, -] (10.5,-2) to[R=$R_3$, -*] (12,-2) to[short, -*] (13,-2) to[short, -o] (15,-2);


    \draw (15,0) to[open,v>, name=ua] (15,-2);
    \draw (0,0) to[open,v>, name=ue] (0,-2);

    \draw (1,0) to[R=$R_4$] (1,-2);
    \draw (3,0) to[R=$r_\mathrm{BE 1}$] (3,-2);
    \draw (5,0) to[I, l=$S_1$] (5,-2);
    \draw (7,0) to[R=$R_1$] (7,-2);
    \draw (8,0) to[R=$r_\mathrm{BE2}$] (8,-2);
    \draw (10,0) to[I, l=$S_2$] (10,-2);
    \draw (12,0) to[R=$R_2$] (12,-2);
    \draw (13,0) to[R=$R_5$] (13,-2);


    \varrmore{ue}{$u_\mathrm{E}$};
    \varrmore{ua}{$u_\mathrm{A}$};

\end{tikzpicture}
  \label{fig:LsgKlausurVerbundverstaerker}
  \end {figure}
  Siehe: Abschnitt \ref{sec:DarlingtonTransistor}
}

\subsection{Gleichrichterschaltung\label{Aufg28}}
\Aufgabe{
  Hier eine Gleichrichterschaltung mit pn-Silizium-Dioden. 
  Die Spannungsquelle $V_1$ liefert sinusförmige $50 \, \mathrm{Hz}$ Wechselspannung mit einer Amplitude von $5\, \mathrm{V}$.
  \begin {figure} [H]
  \centering
  \begin{tikzpicture}
    \draw (-1,0) 
    to[D,l=$D_\mathrm{2}$] (-1,2) -- (-1,4)
    to[D,l=$D_\mathrm{1}$] (-1,6) -- (1,6)
    to[C,l=$C_\mathrm{1}$] (1,4) -- (1,2)
    to[C,l=$C_\mathrm{2}$] (1,0) -- (-1,0);

    \draw (1,0)
    to[short, *-] (3,0)
    to[R, l_=$R_2$] (3,2) -- (3,4)
    to[R, l_=$R_1$] (3,6);  

    \draw (3,0)
    to[short, *-] (5.5,0)
    to[R, l_=$R_3$] (5.5,6)
    to[short, -*] (3,6)
    to[short, -*] (1,6);

    \draw (3,2)
    to[short, *-] (1,2)
    to[short, *-*] (-3,2)    
    node[ground]{}
    to[sV, name=UE, v<] (-3,4)
    to[short, -*] (-1,4); 

    \varrmore{UE}{$u_\mathrm{E}(t)$};
\end{tikzpicture}
  \label{fig:FigKlausurGleichrichter}
  \end {figure}
  \begin{itemize}
    \item[a)]
          Welche Gleichspannungen treten über $R_1$,$R_2$ und $R_3$ auf?
    \item[b)]
          Welche Funktionen haben die Kondensatoren $C_1$ und $C_2$? 
    \item[c)]
          Was ist der besondere Vorteil dieser Gleichrichterschaltung, z.B. für Anwendungen mit Operationsverstärkern?
    \item[d)]
          Skizzieren Sie eine andere Spannungs-Verdopplungs-Schaltung, bei der die maximale positive Gleichsspannung (Ausgangsspannung) gegenüber Massepotential der speisenden Wechselspannung $V_1$ zur Verfügung steht. 
  \end{itemize}
  
  
}

\Loesung{
  \begin{itemize}
    \item[a)]
  \end{itemize}
  \begin{itemize}
    \item $U_\mathrm{R1} = 4,3 \, \mathrm{V} \Rightarrow$ da $0,7 \,\mathrm{V}$ über $D_1$ abfallen \\
    \item $U_\mathrm{R2} = -4,3 \, \mathrm{V} \Rightarrow$ da gleiches für die negative Halbwelle \\
    \item $U_\mathrm{R3} = 8,6 \,\mathrm{V} \Rightarrow$ da $R_3$ beide Spannungswerte aufnimmt \\
  \end{itemize}
  \begin{itemize}
    \item[b)]
  \end{itemize}
  \begin{center}
    Die Kondensatoren $C_1$ und $C_2$ haben einerseits die Funktionen die Signale zu glätten, sowohl von der negativen als auch der positiven Halbwelle. 
  \end{center}
  \begin{itemize}
    \item[c)]
  \end{itemize}
  \begin{center}
    Der Vorteil dieser Gleichrichtung ist, das sowohl positive als auch negative Spannungen abgegriffen werden können.
  \end{center}
  \begin{itemize}
    \item[d)]
  \end{itemize}
  \begin {figure} [H]
  \centering
  \begin{tikzpicture}
    \draw (0,0) to[short, o-] (1,0) to[C,l=$C_\mathrm{1}$](2,0) to[short, *-] (2,0) to[D,l=$D_\mathrm{2}$, -*] (4,0) to[short, -o] (6,0);
    \draw (0,-2) to[short, o-] (1,-2) -- (3,-2)node[ground]{};
    \draw (1.5,-2) to[short, -*] (2,-2) to[short, -*] (4,-2) to[short, -o] (6,-2);
    
    
    \draw (6,0) to[open,v>, name=ua] (6,-2);
    
    
    \draw (2,-2) to[D,l=$D_\mathrm{1}$](2,0);
    \draw (4,-2) to[C,l=$C_\mathrm{2}$](4,0);
    
    
    \draw (0,0) to[open, v>, name=ue] (0,-2);
    
    \varrmore{ua}{$U_\mathrm{a}$};
    \varrmore{ue}{$U_\mathrm{e}$};
    
    \end{tikzpicture}
  \label{fig:LsgKlausurGleichrichter}
  \end {figure}
  Siehe: Abschnitt \ref{fig:BrueckengleichrichterSchaltung}
}
% \subsection{Praktikumsaufgabe 1\label{Aufg29}}
\Aufgabe{
    \begin{figure}[H]
    \centering
    \subsection{Praktikumsaufgabe 1\label{Aufg29}}
\Aufgabe{
    \begin{figure}[H]
    \centering
    \input{Bilder/Aufgaben/Praktikumsaufgabe1.tex}
    \label{fig:Praktikumsaufgabe1}
    \end{figure}

    Berechnen Sie den Widerstand $\mathrm{R_1}$, sodass an $\mathrm{R_2}$ eine Spannung von 1\,V anliegt. Gegeben ist eine Eingangsspannung von $\mathrm{U_E}$ und ein Widerstand $\mathrm{R_2}$.
    
    \begin{itemize}
        \item[a)] Leiten Sie die allgemeine Formel für den Spannungsteiler her.
        \item[b)] Berechnen Sie $\mathrm{R_1}$ für $\mathrm{U_E} = 5\,\text{V}$ und $\mathrm{R_2} = 2\,\text{k}\Omega$.
    \end{itemize}
}


\Loesung{

}
    \label{fig:Praktikumsaufgabe1}
    \end{figure}

    Berechnen Sie den Widerstand $\mathrm{R_1}$, sodass an $\mathrm{R_2}$ eine Spannung von 1\,V anliegt. Gegeben ist eine Eingangsspannung von $\mathrm{U_E}$ und ein Widerstand $\mathrm{R_2}$.
    
    \begin{itemize}
        \item[a)] Leiten Sie die allgemeine Formel für den Spannungsteiler her.
        \item[b)] Berechnen Sie $\mathrm{R_1}$ für $\mathrm{U_E} = 5\,\text{V}$ und $\mathrm{R_2} = 2\,\text{k}\Omega$.
    \end{itemize}
}


\Loesung{

} 
% \begin{tikzpicture}
    \draw (0,0) to[short, -*] (1,0)node[ground]{} to[short, -*] (2.5,0) -- (5,0);
    \draw (5,5) to[C,name=C1, l=$\mathrm{C_1}$](5,3) to[V,name=V2] (5,1) -- (5,0);

    \draw (0,5) to[V, name=V1] (0,0); 
    \draw (0,5) to[R, name=R2, l=$\mathrm{R_2}$, -*] (2.5,5) to[R,name=R3, l=$\mathrm{R_3}$] (5,5);
    
    \draw (2.5,0) to[R,name=R1, l=$\mathrm{R_1}$] (2.5,5);
    
    \varrmore{V1}{$\mathrm{V_1}$};
    \varrmore{V2}{$\mathrm{V_2}$};
\end{tikzpicture}
   
\subsection{Untersuchung eines einfachen Gleichrichters\label{Aufg31}}
% Alt: Praktikumsaufgabe 3
\Aufgabe{

    Analysieren Sie den folgenden Gleichrichter. Zusätzlich sind die folgenden Parameter gegeben:
    
    $\hat{u}_\mathrm{E} = 5\,\mathrm{V}$,
    $R_\mathrm{L} = 1\,\mathrm{k\Omega}$
    und eine Diodendurchlassspannung von $U_\mathrm{f} = 0,7\,\mathrm{V}$.
    
    \begin{figure}[H]
        \centering
        \begin{tikzpicture}
    \draw (4,0)
    to[short, -*] (0,0) node[ground]{}
    to[sV,name=UE, v<] (0,2)
    to[D, l=$D_\mathrm{1}$] (4,2) 
    to[R, l=$R_\mathrm{L}$] (4,0);

    \draw (4,2)
    to[short, *-o] (5.5,2)
    to[open,v^,name=UA] (5.5,0)
    to[short, o-*] (4,0);

    \varrmore{UE}{$u_\mathrm{E}(t)$};
    \varrmore{UA}{$U_\mathrm{A}$};
\end{tikzpicture}
        \label{fig:FigPraktikumGleichrichter}
    \end{figure}
    
    \begin{itemize}
        \item[a)] Berechnen Sie die Ausgangsspannung $U_\mathrm{A}$.
        \item[b)] Bestimmen Sie den durch den Lastwiderstand $R_\mathrm{L}$ fließenden Strom $I_\mathrm{L}$.
    \end{itemize}
}


\Loesung{
    \begin{itemize}
        \item[a)] Berechnung der Ausgangsspannung $U_\mathrm{A}$ \\[0.2cm]
              Die gegebene Schaltung besteht aus einer **Diode in Vorwärtsrichtung** und einem **Lastwiderstand**.  
              Die Ausgangsspannung $U_\mathrm{A}$ entspricht der Eingangsspannung abzüglich der Diodendurchlassspannung:
              \[
                  U_\mathrm{A} = U_\mathrm{E} - U_\mathrm{f}
              \]
              Einsetzen der Werte:
              \[
                  U_\mathrm{A} = 5\,\mathrm{V} - 0{,}7\,\mathrm{V} = 4{,}3\,\mathrm{V}
              \]
              
        \item[b)] Bestimmung des Laststroms $I_\mathrm{L}$ \\[0.2cm]
              Der durch den Lastwiderstand $R_\mathrm{L}$ fließende Strom berechnet sich nach dem Ohmschen Gesetz:
              \[
                  I_\mathrm{L} = \frac{U_\mathrm{A}}{R_\mathrm{L}}
              \]
              Einsetzen der Werte:
              \[
                  I_\mathrm{L} = \frac{4{,}3\,\mathrm{V}}{1\,\mathrm{k\Omega}} = 4{,}3\,\mathrm{mA}
              \]
    \end{itemize}
    Siehe: Abschnitt \ref{sec:Einweggleichrichter}
}

\subsection{Analyse einer Spannungsverdoppler-Schaltung\label{Aufg33}}
% Alt: Praktikumsaufgabe 5
\Aufgabe{
    \begin{figure}[H]
    \centering
    \begin{tikzpicture}
    \draw (0,4)
    to[D,l=$D_\mathrm{1}$, *-*] (2,4)
    to[C,l=$C_\mathrm{1}$, -*] (2,2) 
    to[C,l=$C_\mathrm{2}$, -*] (2,0) 
    to[D,l=$D_\mathrm{2}$] (0,0) -- (0,4); 

    \draw (0,4) -- (-2,4)
    to[sV,name=UE, v_] (-2,2) -- (2,2);

    \draw (2,4) -- (4,4)
     to[R,l=$R_\mathrm{L}$] (4,0) -- (2,0); 
     
    \draw (4,4)
    to[short, *-o] (5.5,4)
    to[open,v^,name=UA] (5.5,0)
    to[short, o-*] (4,0);

    \varrmore{UE}{$u_{\mathrm{E}_i}(t)$};
    \varrmore{UA}{$U_\mathrm{A}$};
\end{tikzpicture}
    \label{fig:FigPraktikumSpannungsverdoppler}
    \end{figure}

        Analysieren Sie die Funktion der Delon-Schaltung zur Spannungsverdopplung und berechnen Sie die Ausgangsspannung für verschiedene Eingangsspannungen.
    
    Gegeben:
    \begin{itemize}
        \item Eingangsspannung $U_\mathrm{E}$ mit Effektivwerten von $110\,\mathrm{V}$ und $230\,\mathrm{V}$
        \item Zwei Dioden mit einer Durchlassspannung von $0,7\,\mathrm{V}$
        \item Zwei Kondensatoren mit der Kapazität $C$
    \end{itemize}
    
    \begin{itemize}
        \item[a)] Erklären Sie das Funktionsprinzip der Delon-Schaltung und wie sie die Eingangsspannung verdoppelt.
        \item[b)] Berechnen Sie die theoretische Leerlauf-Ausgangsspannung $U_\mathrm{A}$ für die gegebenen Eingangsspannungen.
        \item[c)] Diskutieren Sie, wie sich die Ausgangsspannung $U_\mathrm{A}$ unter Lastbedingungen verhält und welche Faktoren die Spannungsstabilität beeinflussen.
    \end{itemize}
}


\Loesung{
    \begin{itemize}
        \item[a)] Funktionsprinzip der Delon-Schaltung  \\
        Die Schaltung arbeitet als Kaskaden-Gleichrichter. Während der positiven Halbwelle der Wechselspannung lädt sich $C_1$ über $D_1$ auf die Eingangsspitzen-Spannung $\hat{U}_\mathrm{E}$ auf. Während der negativen Halbwelle lädt sich $C_2$ über $D_2$ auf. Dadurch addieren sich die Spannungen von $C_1$ und $C_2$, wodurch am Ausgang eine theoretische Gleichspannung von ca. $2 \cdot \hat{U}_\mathrm{E}$ entsteht.
    
        \item[b)] Berechnung der Leerlauf-Ausgangsspannung $U_\mathrm{A}$  
        Die Eingangsspannung ist als Effektivwert $U_\mathrm{E}$ gegeben. Die Spitzenwertspannung beträgt:
        $$ \hat{U}_\mathrm{E} = \sqrt{2} \cdot U_\mathrm{E} $$  
        Da die Schaltung eine Spannungsverdopplung erzeugt, ergibt sich die Leerlauf-Ausgangsspannung:
        $$ U_\mathrm{A} = 2 \cdot (\hat{U}_\mathrm{E} - U_\mathrm{D}) = 2 \cdot (\sqrt{2} \cdot U_E - 0{,}7\,\mathrm{V}) $$  
        Für $ U_\mathrm{E} = 110\, \mathrm{V} $:
        $$ U_\mathrm{A} = 2 \cdot (\sqrt{2} \cdot 110\, \mathrm{V} - 0{,}7\, \mathrm{V}) \approx 308\, \mathrm{V} $$  
        Für $ U_\mathrm{E} = 230\,V $:
        $$ U_\mathrm{A} = 2 \cdot (\sqrt{2} \cdot 230\,\mathrm{V} - 0{,}7\, \mathrm{V}) \approx 644\, \mathrm{V} $$  
    
        \item[c)] Verhalten unter Lastbedingungen  
        Im Leerlauf bleibt die Ausgangsspannung hoch. Unter Last fällt die Spannung ab, abhängig vom Innenwiderstand der Kondensatoren, der Belastung $R_\text{Last}$ und den Diodenverlusten. Der Spannungsabfall hängt von der Restwelligkeit ab, die durch die Lade- und Entladezeiten der Kondensatoren beeinflusst wird. Ein großer Laststrom reduziert die Spannung, da die Kondensatoren sich schneller entladen.
    
        Faktoren, die die Spannung beeinflussen:
        \begin{itemize}
            \item Innenwiderstand der Dioden
            \item Kapazität der Kondensatoren $C$
            \item Belastung durch $R_\text{Last}$
            \item Netzfrequenz ($50\, \text{Hz}$ oder $60 \, \text{Hz}$)
        \end{itemize}
    \end{itemize}
}
\subsection{Differentieller Widerstand einer Diode\label{Aufg34}}
% Alt: Praktikumsaufgabe 7
\Aufgabe{
    \begin{figure}[H]
    \centering
    \begin{tikzpicture}
    \draw (-1,2) to[sinusoidal voltage source,name=V2] (-1,0) -- (0,0);
    \draw  (-1,2) -- (2,2);
    \draw (2,0) to[potentiometer, l=$P_\mathrm{1}$, *-] (2,2);
    \draw (0,0) to[short,-*](1,0)node[ground]{} -- (3.25,0);
    
    \draw (2,1)--(3,1)to[R,l=$R_\mathrm{4}$](5,1) to[C,l=$C_\mathrm{1}$, -*] (5,-1);
    \draw (5,-1) to[D,l=$D_\mathrm{1}$] (7,-1) -- (7,-3) to[short,-*] (2,-3)node[ground]{}--(0,-3);
    
    \draw (0,-1) to[R,l=R1] (5,-1);
    \draw (0,-1) to[V,name=V1](0,-3);
    
    \draw (3,1) to[open,v,name=du](3,0);
    
    \varrmore{du}{$\mathrm{\Delta U}$};
    \varrmore{V1}{$V_\mathrm{1}$};
    \varrmore{V2}{$V_\mathrm{2}$};
\end{tikzpicture}
   
    \label{fig:FigPraktikumBasisschaltung}
    \end{figure}
    
  Analysieren Sie das Grundverhalten eines Bipolartransistors in einer Basisschaltung.
    
    Gegeben:
    \begin{itemize}
        \item NPN-Transistor mit Stromverstärkungsfaktor $\beta = 100$
        \item Basisstrom $I_\mathrm{B} = 20\,\mathrm{\mu A}$
    \end{itemize}
    
    \begin{itemize}
        \item[a)] Berechnen Sie den Kollektorstrom $I_\mathrm{C}$.
        \item[b)] Bestimmen Sie die Kollektor-Emitter-Spannung $U_{CE}$, wenn der Kollektorwiderstand $R_\mathrm{C} = 1\,\mathrm{k\Omega}$ und die Versorgungsspannung $U_\mathrm{CC} = 12\,\mathrm{V}$ beträgt.
    \end{itemize}
}



\Loesung{
    \begin{itemize}
        \item[a)] Berechnung des Kollektorstroms $I_\mathrm{C}$  
        Der Kollektorstrom eines Bipolartransistors ergibt sich aus der Beziehung:
        $$ I_\mathrm{C} = \beta \cdot I_\mathrm{B} $$
        mit den gegebenen Werten:
        $$ I_\mathrm{C} = 100 \cdot 20\,\mathrm{\mu A} $$
        $$ I_\mathrm{C} = 2\,\mathrm{mA} $$
    
        \item[b)] Bestimmung der Kollektor-Emitter-Spannung $U_\mathrm{CE}$  
        Die Kollektor-Emitter-Spannung ergibt sich aus der Gleichung:
        $$ U_\mathrm{CE} = U_\mathrm{CC} - I_\mathrm{C} \cdot R_\mathrm{C} $$
        Einsetzen der Werte:
        $$ U_\mathrm{CE} = 12\,\mathrm{V} - (2\,\mathrm{mA} \cdot 1\,\mathrm{k\Omega}) $$
        $$ U_\mathrm{CE} = 12\,\mathrm{V} - 2\,\mathrm{V} $$
        $$ U_\mathrm{CE} = 10\,\mathrm{V} $$
    \end{itemize}
    Siehe: Abschnitt \ref{fig:DiodenkennlinieUndErsatzschaltbild}
}
\subsection{Grundverhalten eines NPN-Bipolartransistors\label{Aufg35}}
% Alt: Praktikumsaufgabe 8
\Aufgabe{
    \begin{figure}[H]
    \centering
    \begin{tikzpicture}
    \draw
    (0,0) node[npn] (Q1) {}
    (Q1.base) 
    (Q1.collector)  
    (Q1.emitter);

\draw (-3,0) to[R,l=$R_\mathrm{3}$,*-](Q1.B);
\draw (-3,2) to[R,l=$R_1$](-3,0);
\draw (-3,2) -- (-5,2) to[V,name=V1,-*](-5,-2)node[ground]{} to[short,-*](-3,-2) -- (0,-2);
\draw (-3,0) to[R,l=$R_\mathrm{2}$](-3,-2);

\draw (Q1.C) to[R,l=$R_\mathrm{4}$] (0,2.5) -- (2,2.5) to[V,name=V2] (2,-2) -- (0,-2);

\draw (Q1.E) to[short,-*] (0,-2);

\varrmore{V1}{$V_\mathrm{1}$};
\varrmore{V2}{$V_\mathrm{2}$};
\end{tikzpicture}
   
    \label{fig:FigPraktikumNPN}
    \end{figure}
    
    Analysieren Sie das Grundverhalten eines Bipolartransistors in einer Basisschaltung.
    
    Gegeben:
    \begin{itemize}
        \item NPN-Transistor mit Stromverstärkungsfaktor $\beta = 100$
        \item Basisstrom $I_\mathrm{B} = 20\,\mathrm{\mu A}$
    \end{itemize}
    
    \begin{itemize}
        \item[a)] Berechnen Sie den Kollektorstrom $I_\mathrm{C}$.
        \item[b)] Bestimmen Sie die Kollektor-Emitter-Spannung $U_\mathrm{CE}$, wenn der Kollektorwiderstand $R_\mathrm{C} = 1\,\mathrm{k\Omega}$ und die Versorgungsspannung $U_\mathrm{CC} = 12\,\mathrm{V}$ beträgt.
    \end{itemize}
}


\Loesung{
    \begin{itemize}
        \item[a)] Berechnung des Kollektorstroms $I_\mathrm{C}$  
        Der Kollektorstrom eines Bipolartransistors ergibt sich aus der Beziehung:
        $$ I_\mathrm{C} = \beta \cdot I_\mathrm{B} $$
        mit den gegebenen Werten:
        $$ I_\mathrm{C} = 100 \cdot 20\,\mathrm{\mu A} $$
        $$ I_\mathrm{C} = 2\,\mathrm{mA} $$
    
        \item[b)] Bestimmung der Kollektor-Emitter-Spannung $U_\mathrm{CE}$  
        Die Kollektor-Emitter-Spannung ergibt sich aus der Gleichung:
        $$ U_\mathrm{CE} = U_\mathrm{CC} - I_\mathrm{C} \cdot R_\mathrm{C} $$
        Einsetzen der Werte:
        $$ U_\mathrm{CE} = 12\,\mathrm{V} - (2\,\mathrm{mA} \cdot 1\,\mathrm{k\Omega}) $$
        $$ U_\mathrm{CE} = 12\,\mathrm{V} - 2\,\mathrm{V} $$
        $$ U_\mathrm{CE} = 10\,\mathrm{V} $$
    \end{itemize}
    Siehe: Abschnitt \ref{fig:UebersichtBipolartransistoren}
}
\subsection{Zeichnung und Berechnung einer Kollektorschaltung\label{Aufg36}}
% Alt: Praktikumsaufgabe 9
\Aufgabe{
    \begin{itemize}
        \item[a)] Zeichnen Sie die folgende Schaltung:  
        Beschalten Sie einen npn-Transistor mit einem Basisvorwiderstand von  
        $$ R_\mathrm{B} = 4,7\,\mathrm{k\Omega}. $$  
        Der Emitterwiderstand beträgt 
        $$ R_\mathrm{E} = 1\,\mathrm{k\Omega}. $$
        Die Spannungsversorgung beträgt 
        $$ V_\mathrm{0} = 15\,\mathrm{V}. $$
        Schalten Sie eine $10 \,\mathrm{V}$-Z-Diode in Sperrrichtung von der Basis auf Masse.  
        
        Verwenden Sie nur eine Spannungsversorgung, welche sowohl die Basis als auch den Kollektor versorgt.
        \item[b)] Berechnen Sie die Emitterspannung $V_\mathrm{E}$, und den Emitterstrom $I_\mathrm{E}$.
        \item[c)] Berechnen Sie die Leistung am Emitterwiderstand $P_\mathrm{RE}$.
    \end{itemize}
}


\Loesung{
\begin{itemize}
    \item[a)] Schaltung:
    \begin{figure}[H]
        \centering
        \begin{tikzpicture}
    \draw
    (0,0) node[npn] (Q1) {}
    (Q1.base) 
    (Q1.collector)  
    (Q1.emitter);

\draw (Q1.C) -- (0,2);
\draw (Q1.E) to[R,l=$R_\mathrm{E}$] (0,-2.5) -- (-2,-2.5) to[zDo,*-*] (-2,0) to[R,l=$R_\mathrm{B}$] (-2,2);
\draw (Q1.B) -- (-2,0);

\draw (-2,-2.5) -- (-4,-2.5);
\draw (-4,2) to[V,name=V0] (-4,-2.5);
\draw (-4,2)to[short,-*](-2,2)--(0,2);

\varrmore{V0}{$V_\mathrm{0}$};
\end{tikzpicture}
   
        \label{fig:LsgPraktikumZeichnung}
        \end{figure}


    \item[b)] Berechnung der Emitterspannung $V_\mathrm{E}$ \\
    Die Zenerdiode fixiert die Basisspannung des Transistors auf $V_\mathrm{Z} = 10\,\mathrm{V}$. 
    Die Basis-Emitter-Spannung beträgt $V_{BE} = 0,7\,\mathrm{V}$.
    $$ V_\mathrm{E} = V_\mathrm{Z} - V_\mathrm{BE} = 10\,\mathrm{V} - 0,7\,\mathrm{V} = 9,3\,\mathrm{V} $$
    $$ I_\mathrm{E} = \frac{V_\mathrm{E}}{R_\mathrm{E}} = \frac{9,3\,\mathrm{V}}{1\,\mathrm{k\Omega}} = 9,3\,\mathrm{mA} $$

\item[c)] Berechnung der Leistung am Emitterwiderstand $P_{RE}$ \\
    $$ P_\mathrm{RE} = I_\mathrm{E}^2 \cdot R_\mathrm{E} = (9,3\,\mathrm{mA})^2 \cdot 1\,\mathrm{k\Omega} = 86,49\,\mathrm{mW} $$

\end{itemize}
Siehe: Abschnitt \ref{fig:GrundschaltungenBipolartransistor}
}
 

%Beispiel Aufgaben (Skript Aufgabe 4.10)


         


%\section{Lösungen zu den Übungsaufgaben}
%\renewcommand{\Aufgabe}[1]{}
%\renewcommand{\Loesung}[1]{#1}
%\subsection{Beschreibung des Bändermodells\label{Aufg1} }
% Alt: Aufgabe 1
\Aufgabe{
  Beschreiben Sie ausführlich das Bändermodell von Festkörpern und seine Bedeutung für die Unterscheidung zwischen 
  Leiter, Halbleiter und Isolator. Gehen Sie dabei insbesondere auf die folgenden Aspekte ein:
  \begin{itemize}
    \item relevante Bänder und deren Besetzung mit Ladungsträgern.
    \item Bedeutung der Bandlücke und wie diese elektrische Eingeschaften von Materialien beeinflusst.
    \item Einfluss von Temperaturänderungen auf die Leitfähigkeit.
\end{itemize}
% \speech{Die Abbildung zeigt **Aufgabe 1**, die eine ausführliche Beschreibung des **Bändermodells von Festkörpern** erfordert. Dieses Modell ist entscheidend für das Verständnis der elektrischen Eigenschaften von Materialien und ihrer Unterscheidung in **Leiter, Halbleiter und Isolatoren**.  

% ### **Aufgabenstellung**
% Die Aufgabe fordert eine **detaillierte Erläuterung** des Bändermodells unter Berücksichtigung der folgenden Aspekte:

% 1. **Relevante Bänder und deren Besetzung mit Ladungsträgern**  
%    - In Festkörpern sind die **Elektronenenergiezustände** in **Bänder** aufgeteilt:  
%      - **Valenzband**: Enthält die Elektronen, die an chemischen Bindungen beteiligt sind.  
%      - **Leitungsband**: Hier können sich freie Elektronen bewegen und elektrische Leitung ermöglichen.  
%    - Die **Verfügbarkeit von Elektronen im Leitungsband** bestimmt die **Leitfähigkeit des Materials**.  
%    - Bei einem **Leiter (z. B. Metalle)** überlappen Valenz- und Leitungsband, wodurch **freie Ladungsträger ständig vorhanden sind**.  
%    - Bei einem **Halbleiter (z. B. Silizium)** gibt es eine **kleine Bandlücke**, die mit Temperatur oder Dotierung überbrückt werden kann.  
%    - Bei einem **Isolator (z. B. Glas)** ist die Bandlücke sehr groß, wodurch **fast keine Leitung möglich ist**.  

% 2. **Bedeutung der Bandlücke und ihr Einfluss auf elektrische Eigenschaften**  
%    - Die **Bandlücke (\( E_G \))** ist der Energiebereich zwischen Valenz- und Leitungsband.  
%    - Sie bestimmt, ob ein Material **ein Leiter, Halbleiter oder Isolator** ist:  
%      - **Leiter**: \( E_G = 0 \), kontinuierliche elektronische Zustände ermöglichen Leitung.  
%      - **Halbleiter**: \( 0 < E_G < 3eV \), Ladungsträger können durch thermische Anregung ins Leitungsband übergehen.  
%      - **Isolator**: \( E_G > 3eV \), sehr wenige Elektronen im Leitungsband, daher fast keine Leitfähigkeit.  
%    - Die Größe der Bandlücke beeinflusst auch **optische Eigenschaften**:  
%      - **Halbleiter mit kleiner Bandlücke** absorbieren und emittieren Licht (z. B. LEDs, Laser).  
%      - **Isolatoren mit großer Bandlücke** sind meist transparente Materialien (z. B. Quarzglas).  

% 3. **Einfluss von Temperaturänderungen auf die Leitfähigkeit**  
%    - **Mit steigender Temperatur steigt die Elektronenenergie**, wodurch mehr Elektronen die Bandlücke überwinden.  
%    - In **Halbleitern nimmt die Leitfähigkeit mit steigender Temperatur zu**, da mehr Elektronen ins Leitungsband übergehen.  
%    - In **Leitern nimmt die Leitfähigkeit mit steigender Temperatur ab**, da die Bewegung der Ladungsträger durch **Gitterstreuung behindert wird**.  
%    - **Isolatoren zeigen kaum Veränderung**, da ihre Bandlücke zu groß ist.  

% ### **Zusammenfassung der Aufgabe**
% - Das **Bändermodell** erklärt, warum manche Materialien **Strom leiten** und andere nicht.  
% - Die **Bandlücke** ist ein entscheidender Faktor für die **elektrischen und optischen Eigenschaften**.  
% - **Temperaturabhängigkeit** ist für Halbleitertechnologien besonders wichtig.  

% Diese Aufgabe fordert eine umfassende Erklärung der **physikalischen Grundlagen des Bändermodells** sowie dessen Auswirkungen auf **die Klassifikation und das Verhalten von Materialien**.}

}


\Loesung{
    Das Bändermodell erklärt die Energiezustände, die Elektronen in Festkörpern einnehmen können. 
    In einem einzelnen Atom sind die Elektronenzustände diskrete Energielevel.
    Siehe: Abschnitt \ref{sec:Bändermodell}

%     \speech{ 
% Das Bändermodell beschreibt die möglichen Energiezustände, die Elektronen in einem Festkörper einnehmen können. In einem einzelnen Atom sind die Energiezustände der Elektronen diskrete Energieniveaus, also genau bestimmte Werte.  

% In einem Festkörper, der aus vielen Atomen besteht, überlappen diese einzelnen Energieniveaus und bilden sogenannte Energie**bänder**. Die beiden wichtigsten Bänder sind:  

% 1. **Valenzband**: Das energetisch höchste besetzte Band bei niedrigen Temperaturen. Elektronen in diesem Band sind an ihre Atome gebunden und können sich nicht frei bewegen.  
% 2. **Leitungsband**: Ein energetisch höheres Band, in dem sich Elektronen frei bewegen können. Elektronen in diesem Band tragen zum elektrischen Stromfluss bei.  

% Zwischen diesen beiden Bändern befindet sich die **Bandlücke**, auch **Energielücke** genannt. Sie gibt an, wie viel Energie benötigt wird, um ein Elektron vom Valenzband in das Leitungsband zu heben. Die Größe der Bandlücke bestimmt, ob ein Material ein Leiter, Halbleiter oder Isolator ist:  

% - **Leiter** (z. B. Metalle): Keine oder sehr kleine Bandlücke, sodass Elektronen leicht ins Leitungsband übergehen können.  
% - **Halbleiter** (z. B. Silizium, Germanium): Mittlere Bandlücke, Elektronen können durch äußere Energie (z. B. Wärme, Licht) ins Leitungsband angehoben werden.  
% - **Isolatoren** (z. B. Glas, Keramik): Große Bandlücke, Elektronen können nur mit extrem hoher Energie ins Leitungsband gelangen.  

% Das Bändermodell ist essenziell für das Verständnis der elektrischen Eigenschaften von Materialien und die Funktionsweise von Halbleiterbauelementen wie Dioden und Transistoren. 
% }

  }




%{
    %\fta{Lösungen}
    %Lösungen für Beispielaufgaben 4.10
    %\textbf{Aufgabe 1}
    %\newline
    %Spannung über $R_\mathrm{V}$ ($U_\mathrm{V}$): \[ U_\mathrm{V} = U_\mathrm{0} - U_\mathrm{D} = 5\mathrm{V} - 2,2 \mathrm{V} = 2,8\mathrm{V}\]
    %\newline
    %Vorwiderstand:
    %\[
    %  R_\mathrm{V} = \frac{2,8\mathrm{V}}{30\mathrm{mA}} = 93,3\Omega 
    %\]

    %\textbf{Aufgabe 2}
    %\newline
   %a) 
   %\newline
   %\[ 
   %  \frac{5\mathrm{V}}{2\mathrm{k \Omega} + 0,3\mathrm{k \Omega}} = \frac{U_1}{300 \Omega}
   %\]
    % \newline
   %\[
   %  U_1 = 300 \Omega \cdot \frac{5 \mathrm{V}}{2\mathrm{k \Omega} + 0,3\mathrm{k \Omega}} = 0,65\mathrm{V}
   %\]
    % \newline
   %\[
   %  5 \mathrm{V} = 0,65 \mathrm{V} + U_\mathrm{RV} + 0,6 \mathrm{V}
   %\]
   % \newline
   %\[
   %  U_\mathrm{RV}= 5 \mathrm{V} - 0,65 \mathrm{V} - 0,6 \mathrm{V} = 3,75 \mathrm{V}
   %\]
   %\newline 
   %\[
   % R_\mathrm{V} = \frac{3,75 \mathrm{V}}{50 \mathrm{\mu A}} = 75 \mathrm{k \Omega}
   %\]  
   %\newline
   %b)
   %\newline
   %\[ 
   % I_\mathrm{C} = \frac{15 \mathrm{V} - 7,5\mathrm{V}}{500 \Omega} = 15 \mathrm{mA}
   %\]
   %\newline
   %\[
   % I_\mathrm{E} = I_\mathrm{C} + I_\mathrm{B} = 15 \mathrm{mA} + 50 \mathrm{\mu A} = 15,05 \mathrm{mA}
   %\]

%}
\subsection{Die Raumladungszone im pn-Übergang\label{Aufg2}}
% Alt: Aufgabe 2
\Aufgabe{
  Untersuchen Sie die Eigenschaften und die Bildung der Raumladungszone in pn-Übergängen von Halbleitern. 
  Beantworten Sie die folgenden Aspekte in Ihrer Ausarbeitung:
  \begin{itemize}
    \item Was ist die Raumladungszone und wie entsteht diese im pn-Übergang?
    \item Wie ist der Verlauf des elektrischen Feldes im Halbleiter?
    \item Welche inneren Vorgänge wirken auf die freien Ladungsträger?
    \item Wie verändert sich die Raumladungszone bei einer angelegten Spannung in Durchlass- und Sperrrichtung?

    \end{itemize}
%     \speech{Untersuchen Sie die **Eigenschaften und die Bildung der Raumladungszone** in **pn-Übergängen** von Halbleitern.  

% Beantworten Sie die folgenden Aspekte in Ihrer Ausarbeitung:  
% - **Was ist die Raumladungszone** und wie entsteht diese im **pn-Übergang**?  
% - **Wie ist der Verlauf des elektrischen Feldes** im Halbleiter?  
% - **Welche inneren Vorgänge** wirken auf die freien Ladungsträger?  
% - **Wie verändert sich die Raumladungszone** bei einer **angelegten Spannung** in **Durchlass- und Sperrrichtung**?}


}


\Loesung{
  Die RLZ ist eine schmale 
  Region um den pn-Übergang herum, in der positive Ionen aus der n-Schicht und negative Ionen aus der p-Schicht verbleiben, 
  nachdem die Diffusion abgeschlossen ist. In dieser Zone gibt es keine freien Ladungsträger, da die positiven und negativen Ladungen sich 
  gegenseitig neutralisieren. Dadurch entsteht ein elektrisches Feld ($\vec{E}$), das sowohl einen Driftstrom erzeugt als auch die Diffusion von weiteren 
  Ladungsträgern unterdrückt. Die Raumladungszone wirkt wie eine Sperrschicht und verhindert den Stromfluss in Sperrrichtung.
  Siehe: Abschnitt \ref{sec:pn-Übergang}

%  \speech{
% Die Raumladungszone (RLZ) ist eine schmale Region im Bereich des **pn-Übergangs** eines Halbleiters. Sie entsteht, wenn sich die freien Ladungsträger (Elektronen und Löcher) durch **Diffusion** über die Grenze zwischen der p- und n-Schicht hinweg bewegen.  

%  Aufbau der Raumladungszone:
% - In der **n-Schicht** verbleiben **positive Ionen**, weil die beweglichen Elektronen in die p-Schicht diffundiert sind.  
% - In der **p-Schicht** verbleiben **negative Ionen**, da die beweglichen Löcher in die n-Schicht gewandert sind.  

% Diese fixierten Ionen neutralisieren sich **gegenseitig**, wodurch die Raumladungszone frei von beweglichen Ladungsträgern ist.

%  Wirkung der Raumladungszone:
% - Es bildet sich ein **elektrisches Feld**, das die Bewegung weiterer Ladungsträger hemmt.  
% - Ein dadurch entstehender **Driftstrom** wirkt der Diffusion entgegen und stabilisiert die Raumladungszone.  
% - Diese Zone verhält sich wie eine **Sperrschicht** und verhindert, dass Strom in **Sperrrichtung** durch die Diode fließt.  

% Dadurch erhält die Diode ihre richtungsabhängigen Eigenschaften:  
% - In **Durchlassrichtung** (p-Schicht positiver als n-Schicht) kann Strom fließen, wenn eine ausreichend hohe Spannung anliegt.  
% - In **Sperrrichtung** (p-Schicht negativer als n-Schicht) blockiert die Raumladungszone den Stromfluss fast vollständig.  

% Die Raumladungszone ist ein fundamentales Konzept in der Halbleitertechnik und essenziell für das Verständnis von Dioden, Transistoren und anderen elektronischen Bauelementen.  
% }

}





\subsection{Berechnung des Spannungsabfalls einer Diode\label{Aufg3}}
% Alt: Aufgabe 3
\Aufgabe{
  Gegeben ist eine Reihenschaltung aus einer Gleichspannungsquelle, einem Widerstand und einer Siliziumdiode mit einer Schleusenspannung von $U_\mathrm{S}=0,7\,\mathrm{V}$. 
  Welche Spannung fällt näherungsweise über dem Widerstand $R$ ab, wenn die Versorgungsspannung $U_0=5\,\mathrm{V}$ beträgt?
  \begin {figure} [H]
   \centering
   \begin{tikzpicture}
  % Batterie links
  \draw (0,2) to[V, name=U1] (0,0);
  \varrmore{U1}{$U_{0}$}; 

  % Widerstand
  \draw (0,2) to[R=$R$] (4,2);
  \draw (0,0) -- (4,0);

  % Diode rechts
  \draw (4,2) to[D=$D$, v, name=D1] (4,0);
  \varrmore{D1}{$U_\mathrm{D}$};

\end{tikzpicture}
   \label{fig:FigDiodeSpannungsabfall}
\end {figure}

% \speech{ **Aufgabe 3**  
% Gegeben ist eine **Reihenschaltung** aus einer **Gleichspannungsquelle**, einem **Widerstand** und einer **Siliziumdiode**.  
% Beantworten Sie die folgende Frage:  
% - Welche Spannung fällt **näherungsweise** über dem Widerstand \( R \) ab, wenn die **Versorgungsspannung \( U_0 = 5V \)** beträgt?  

%  **Beschreibung der Abbildung**  
% Die Schaltung besteht aus drei Hauptkomponenten:  

% 1. **Gleichspannungsquelle (\( U_0 \))**  
%    - Symbolisiert durch einen Kreis mit einem Richtungspfeil.  
%    - Die Spannung \( U_0 \) beträgt **5V** und versorgt die Schaltung.  

% 2. **Serienwiderstand (\( R \))**  
%    - Der Widerstand ist im oberen Schaltzweig platziert.  
%    - Seine Funktion ist die **Strombegrenzung** in der Schaltung.  

% 3. **Siliziumdiode (Si)**  
%    - In **Durchlassrichtung** geschaltet, mit der **Kathode nach Masse**.  
%    - Sie hat eine typische **Schwellenspannung von ca. 0,7V**, die bei Leitung überschritten werden muss.  

%  **Erwartetes Verhalten der Schaltung**  
% - Die Diode **leitet**, sobald \( U_0 > 0,7V \) beträgt.  
% - Die Spannung über dem Widerstand kann **näherungsweise berechnet werden** als:
%   \[
%   U_R = U_0 - U_D
%   \]
%   wobei \( U_D \approx 0,7V \) die **Durchlassspannung der Siliziumdiode** ist.  
% - Für \( U_0 = 5V \) ergibt sich:
%   \[
%   U_R \approx 5V - 0,7V = 4,3V
%   \]
%   - Diese Spannung fällt über den Widerstand \( R \) ab.  
%   - Der Strom durch die Schaltung kann mit dem **Ohm’schen Gesetz** bestimmt werden:
%     \[
%     I = \frac{U_R}{R}
%     \]
%   - Der Widerstand begrenzt den Strom durch die Diode, um eine Überlastung zu verhindern.  

%  **Zusammenfassung**  
% Die Abbildung zeigt eine **einfache Reihenschaltung** mit einer **Spannungsquelle, einem Widerstand und einer Diode**. Die Aufgabe erfordert die **Berechnung der Spannung über dem Widerstand**, wenn die **Diode in Durchlassrichtung betrieben wird**.}

 }


\Loesung{
  Da es sich um eine Siliziumdiode handelt, beträgt die Schleusenspannung und der daraus resultierende Spannungsabfall
  am Bauteil ungefähr $0,6 \text{ bis } 0,7\,\mathrm{V}$.

  In der Reihenschaltung ergibt sich der folgende Zusammenhang:
    \begin{equation*}
        U_0 = U_\mathrm{R} + U_\mathrm{D}
    \end{equation*}
  \begin{align*}
    U_\mathrm{R} &= U_0 - U_\mathrm{D}\\
    &= 5\,\mathrm{V} - 0,7\,\mathrm{V} \\
    &= 4,3\,\mathrm{V}
  \end{align*}
  Siehe: Abschnitt \ref{sec:ElektrischesVerhalten}
%   \speech{Es handelt sich um eine Siliziumdiode. Die Schleusenspannung der Diode beträgt ungefähr 0,6 bis 0,7 Volt. Dadurch ergibt sich ein entsprechender Spannungsabfall am Bauteil.  

% Da die Diode in einer Reihenschaltung mit einem Widerstand verbunden ist, gilt die Spannungsregel:  

% Die Gesamtspannung U null ist gleich der Summe aus der Spannung am Widerstand U R und der Spannung an der Diode U D.  

% Das bedeutet:  

% U R ist gleich U null minus U D.  

% Setzt man die Werte ein, ergibt sich:  

% 5 Volt minus 0,7 Volt ergibt 4,3 Volt.  

% Die Spannung am Widerstand beträgt also 4,3 Volt.}

}
\subsection{Bestimmung des Arbeitspunkts und der Leistung einer Diode\label{Aufg4}}
% Alt: Aufgabe 4
\Aufgabe{
  Eine Diode hat die folgende Kennlinie im 
  Durchlassbereich. Sie wird in Reihe mit einer 
  idealen Spannungsquelle $U_0=1\,\mathrm{V}$ und einem 
  Widerstand mit dem Wert von $R=200\,\Omega$ geschaltet.

    \begin{figure}[H]
        \centering
        \begin{subfigure}[h]{0.45\textwidth}
            \centering
            \begin{circuitikz}
    
\draw (0,0) to [R,i, name=r0, l=$R$] (4,0) to [D=$D$, v, name=D1] (4,-3);
\draw (4,-3) to (0,-3);
\draw (0,0) to [V, name=U1] (0,-3);


\varrmore{U1}{$U_0$}; 
\varrmore{D1}{$U_\mathrm{D}$};
\iarrmore{r0}{$I$};

\end{circuitikz}
        \end{subfigure}
        \begin{subfigure}[h]{0.45\textwidth}
            \centering
            \includesvg[width=0.9\textwidth]{Bilder/Aufgaben/FigDiodeAP}
        \end{subfigure}
        \label{fig:FigDiodeAP}
    \end{figure}

    \begin{itemize}
        \item[a)]
        Welcher Strom $I$ und welche Spannung $U_\mathrm{D}$ 
        stellen sich ein? Lösen Sie das Problem graphisch.
        \item[b)]
        Welche elektrische Leistung wird an der Diode umgesetzt?
        \item[c)]
        Wie muss $R$ gewählt werden, damit sich ein Strom von $3\,\mathrm{mA}$ einstellt? 
    \end{itemize}
%     \speech{ **Aufgabe 4**  
% Eine **Diode mit gegebener Kennlinie im Durchlassbereich** wird in **Reihe mit einer idealen Spannungsquelle** \( U_0 = 1V \) und einem **Widerstand \( R = 200\Omega \)** geschaltet.

% Beantworten Sie die folgenden Fragen:  
% a) **Welcher Strom \( I \) und welche Spannung \( U_D \) stellen sich ein?**  
%    - Lösen Sie das Problem **graphisch** mit Hilfe der gegebenen **Kennlinie**.  
% b) **Welche elektrische Leistung wird an der Diode umgesetzt?**  
% c) **Wie muss \( R \) gewählt werden, damit sich ein Strom von 3 mA einstellt?**  



%  **Beschreibung der Abbildung**
% Die Abbildung besteht aus zwei Teilen:

% 1. **Schaltungsskizze (links):**  
%    - Eine **ideale Gleichspannungsquelle** mit einer Spannung von \( U_0 = 1V \) ist dargestellt.  
%    - Ein **Serienwiderstand \( R \)** von **200Ω** begrenzt den Stromfluss.  
%    - Eine **Siliziumdiode** ist in **Durchlassrichtung** geschaltet.  
%    - Der Strom \( I \) fließt durch den Widerstand und die Diode.  
%    - Die Spannung **\( U_D \)** fällt über der **Diode** ab.  

% 2. **Diodenkennlinie (rechts):**  
%    - **Auf der x-Achse** ist die **Diodenspannung \( U_D \)** in **Volt (V)** dargestellt.  
%    - **Auf der y-Achse** ist der **Diodenstrom \( I \)** in **Milliampere (mA)** aufgetragen.  
%    - Die **exponentielle Kennlinie** zeigt den **typischen Durchlassbereich einer Diode**.  
%    - Um die Lösung zu bestimmen, muss der **Arbeitspunkt** durch die **Schnittstelle zwischen der Diodenkennlinie und der Lastgeraden** ermittelt werden.  

%  **Zusammenfassung der Aufgabe**
% - Die Aufgabe erfordert die **Bestimmung des Stroms und der Spannung an der Diode**, wobei die **Kennlinie** als Grundlage für die Berechnung dient.  
% - Es soll die **umgesetzte elektrische Leistung** berechnet werden.  
% - Schließlich muss der **Widerstandswert \( R \) angepasst werden**, um einen bestimmten **gewünschten Strom von 3 mA** zu erreichen.  
% }

}
    
\Loesung{
    \begin{itemize}
        \item[a)]
        Im ersten Schritt wird die Wirkung der Spannungsquelle auf den Widerstand $R$ betrachtet. 
        Bei $U_0=1\,\mathrm{V}$ fließen durch den Widerstand $R$ ein Strom von $I_\mathrm{R}=\frac{U_0}{R}=\frac{1\,\mathrm{V}}{200\,\Omega}=5\,\mathrm{mA}$. 
        Es kann die Widerstandgerade vom Ursprung aus zu dem Punkt ($1\,\mathrm{V}$, $5\,\mathrm{mA}$) eingezeichnet werden (linke Abbildung).\\

        Gemäß des Aufbaus der Schaltung muss die Widerstandsgerade im nächsten Schritt gespiegelt werden, es entsteht die rechte Darstellung.
        Der Schnittpunkt beider Linien entspricht dem Arbeitspunkt ($U_\text{AP}$, $I_\text{AP}$), dieser lautet: ($0,75\,\mathrm{V}$, $1,2\mathrm{mA}$)  

        \begin{figure}[H]
            \begin{subfigure}[h]{0.45\linewidth}
                \centering
                \includesvg[width=\textwidth]{Bilder/Aufgaben/LsgDiodeAP1}
            \end{subfigure}
            \begin{subfigure}[h]{0.45\linewidth}
                \centering
                \includesvg[width=\textwidth]{Bilder/Aufgaben/LsgDiodeAP2}
            \end{subfigure}
        \end{figure}

        \item[b)]
        Die Leistung ergibt sich aus dem zuvor ermittelten Arbeitspunkt.
        \begin{equation*}
            P = U_\text{AP} \cdot I_\text{AP} = 0,75 \, \text{V} \cdot 1,2 \, \text{mA} = 0,9 \, \text{mW}
        \end{equation*}
        
        \item[c)]
        Ausgehend vom Punkt ($1\,\mathrm{V}, 0\,\mathrm{mA}$) zum Schnittpunkt mit der Diodenkennlinie bei $I=3\,\mathrm{mA}$ ergibt sich die gesuchte Widerstandsgerade.
        \begin{figure}[H]
            \centering
            \includesvg[width=0.45\linewidth]{Bilder/Aufgaben/LsgDiodeAP3}
        \end{figure}
        Es kann der Schnittpunkt ($0,86\,\mathrm{V}, 3\,\mathrm{mA}$) abegelesen werden.
    

        \begin{equation*}
            U_\mathrm{ges} = U - U_\mathrm{D} = 1 \, \mathrm{V} - 0,86 \, \mathrm{V} = 0,14 \, \mathrm{V}
        \end{equation*}


    \end{itemize}



\begin{itemize}
    \item[\bf c)] 
    \[
    U_{\text{ges}} = U - U_\text{D} = 1 \, \text{V} - 0,86 \, \text{V} = 0,14 \, \text{V}
    \]
    \[
    I = 3 \, \text{mA}
    \]
    \[
    R = \frac{U_{\text{ges}}}{I} = \frac{0,14 \, \text{V}}{3 \, \text{mA}} = 46,67 \, \Omega
    \]
\end{itemize}
Siehe: Abschnitt \ref{fig:Diodenkennlinie}


% \speech{
% **Lösung zu Aufgabe 4**  

% **Teilaufgabe a:**  
% Im ersten Schritt wird die Wirkung der Spannungsquelle auf den Widerstand \( R \) betrachtet.  
% Die gegebene Spannung beträgt \( U_0 = 1V \).  
% Da der Widerstand \( R = 200 \Omega \) beträgt, ergibt sich der Strom durch den Widerstand als:  
% \( I_R = \frac{U_0}{R} = \frac{1V}{200 \Omega} = 5mA \).  

% Die dazugehörige Widerstandsgerade kann im Diagramm als eine Linie von der Ursprungskoordinate \((0V, 0mA)\) bis zum Punkt \((1V, 5mA)\) eingezeichnet werden.  
% Diese Gerade gibt an, wie sich der Strom \( I_R \) in Abhängigkeit der Spannung \( U \) verhält.  

% In der linken Abbildung ist die Diodenkennlinie als eine exponentiell ansteigende blaue Kurve dargestellt.  
% Zusätzlich ist die Widerstandsgerade als schwarze Linie eingezeichnet.  
% Diese verläuft von \((0V, 0mA)\) bis zu \((1V, 5mA)\).  

% Da das Bauteil eine Diode enthält, muss die Widerstandsgerade gespiegelt werden.  
% Dies ist in der zweiten Abbildung auf der rechten Seite ersichtlich.  
% Die Widerstandsgerade wird so verschoben, dass sie nun von einem neuen Ursprungspunkt ausgeht.  
% Der Schnittpunkt dieser Linie mit der Diodenkennlinie entspricht dem Arbeitspunkt der Schaltung.  
% Dort gelten die Werte:  
% \( U_{AP} = 0,75V \) und \( I_{AP} = 1,2mA \).  

% **Teilaufgabe b:**  
% Die elektrische Leistung an der Diode wird aus dem zuvor bestimmten Arbeitspunkt berechnet.  
% Die allgemeine Formel für die Leistung lautet:  
% \( P = U \cdot I \).  
% Hier setzen wir die Werte des Arbeitspunktes ein:  
% \( P = 0,75V \cdot 1,2mA = 0,9mW \).  

% **Teilaufgabe c:**  
% Nun soll der Widerstand \( R \) so bestimmt werden, dass sich ein Strom von \( I = 3mA \) einstellt.  
% In der dritten Abbildung ist erneut eine Diodenkennlinie zu sehen.  
% Zusätzlich wurde eine neue Widerstandsgerade eingezeichnet, die nun durch den Punkt \((1V, 0mA)\) zum Schnittpunkt mit der Diodenkennlinie bei \( I = 3mA \) verläuft.  
% Dieser Schnittpunkt liegt bei der Spannung \( 0,86V \).  

% Der Spannungsabfall am Widerstand ergibt sich aus:  
% \( U_{ges} = U - U_D = 1V - 0,86V = 0,14V \).  
% Da der gewünschte Strom \( I = 3mA \) beträgt, berechnet sich der benötigte Widerstand durch das Ohmsche Gesetz:  
% \( R = \frac{U_{ges}}{I} = \frac{0,14V}{3mA} = 46,67 \Omega \).  

% Das bedeutet, dass für einen Strom von 3mA der Widerstand \( R \) auf 46,67 Ohm angepasst werden muss.  
% }

}

\subsection{Dimensionierung des Vorwiderstands einer Zenerdiode\label{Aufg5}}
% Alt: Aufgabe 5
\Aufgabe{
  Die Zenerspannung der Diode beträgt $U_\mathrm{D}=6\,\mathrm{V}$ und die Diode darf eine maximale Verlustleistung von \(P_\mathrm{tot}=\mathrm{400\,mW} \) nicht überschreiten. 
  Die maximale Eingangsspannung beträgt \( U_0 = \mathrm{12\,V} \). \\
  Wie groß muss der Widerstand \( R \) mindestens sein, damit die Diode nicht zerstört wird?

\begin {figure} [H]
   \centering
   \begin{tikzpicture}

    \draw (0,2) to[V, name=U0] (0,0);
    \draw (0,2) to[R=$R$] (4,2);
    \draw (0,0) -- (4,0);
    \draw (4,0) to[zDo, l_=$D_\mathrm{z}$, v^<, name=Uz] (4,2);

    \varrmore{U0}{$U_0$}; 
    \varrmore{Uz}{$U_\mathrm{D}$}; 
  
  \end{tikzpicture}
   \label{fig:ZDiodeVorwiderstand}
\end {figure}

% \speech{ **Aufgabe 5**  
% Eine **Zenerdiode** mit einer **Zenerspannung von 6V** wird in einer Schaltung verwendet, um **Überspannung zu begrenzen**.  
% Die **maximale Verlustleistung** der Diode beträgt **400 mW**, und die **maximale Eingangsspannung \( U_0 \)** ist **12V**.  

% **Fragestellung:**  
% - Wie groß muss der **Widerstand \( R \)** mindestens sein, damit die **Diode niemals zerstört** wird?  

%  **Beschreibung der Abbildung**  
% Die Schaltung besteht aus den folgenden Komponenten:

% 1. **Gleichspannungsquelle \( U_0 \)**  
%    - Symbolisiert durch einen Kreis mit einem Pfeil.  
%    - Die **maximale Spannung beträgt 12V**.  

% 2. **Serienwiderstand \( R \)**  
%    - Platziert im oberen Zweig der Schaltung.  
%    - Begrenzung des **Stroms durch die Zenerdiode**.  

% 3. **Zenerdiode**  
%    - In **Sperrrichtung** geschaltet.  
%    - Sobald die **Spannung \( U_0 > 6V \)** erreicht, beginnt die Diode zu **leiten** und hält die Spannung auf **6V konstant**.  

%  **Funktionsweise der Schaltung**  
% - Wenn **\( U_0 < 6V \)**, ist die **Diode nicht leitend**, und es fließt kein Strom durch sie.  
% - Wenn **\( U_0 > 6V \)**, hält die **Zenerdiode die Spannung bei 6V**, indem sie den überschüssigen Strom ableitet.  
% - Der Widerstand \( R \) muss so **gewählt werden**, dass der **Strom durch die Diode die maximale Verlustleistung von 400 mW nicht überschreitet**.  


%  **Zusammenfassung der Aufgabe**  
% - Die Aufgabe erfordert die **Berechnung des minimalen Widerstandswerts \( R \)**, um die **maximale Verlustleistung der Zenerdiode nicht zu überschreiten**.  
% - Die Schaltung dient als **Spannungsregler**, der eine **konstante Spannung von 6V gewährleistet**, wenn die Eingangsspannung zu hoch ist.  
% - Die Berechnung erfolgt mit **Leistungsgesetzen und dem Ohm’schen Gesetz**, um den maximal zulässigen Strom durch die Diode zu bestimmen.  
% }

}


\Loesung{
    Die Spannung am Widerstand ergibt sich zu:
    \begin{align*}
        U_\mathrm{R} &= U_0 - U_\mathrm{D} = \mathrm{12\,V} - \mathrm{6\,V} = \mathrm{6\,V}
    \end{align*}
    Die maximale Verlustleistung der Diode ist:
    \begin{align*}
        P_\mathrm{tot} &= U_\mathrm{D} \cdot I = \mathrm{400\,mW}
    \end{align*}
    Daraus folgt der maximale Strom durch die Diode:
    \begin{align*}
        I &= \frac{\mathrm{400\,mW}}{\mathrm{6\,V}} = \mathrm{66{,}7\,mA}
    \end{align*}
    Um diesen Strom nicht zu überschreiten, ergibt sich der Mindestwert für den Widerstand:
    \begin{align*}
        R &= \frac{U_\mathrm{R}}{I} = \frac{\mathrm{6\,V}}{\mathrm{66{,}7\,mA}} \approx \mathrm{90\,\Omega}
    \end{align*}
  Siehe: Abschnitt \ref{sec:SpezielleDioden}

%   \speech{
% **Lösung zu Aufgabe 5**  

% Die gegebene Schaltung besteht aus einer Spannungsquelle mit **12 Volt**, einem Widerstand **R**, und einer **Zenerdiode**.  
% Die **Zenerdiode** begrenzt die Spannung an ihren Anschlüssen auf **6 Volt**.  
% Das bedeutet, dass die restlichen **6 Volt** über den Widerstand **R** abfallen müssen.  

% ### Berechnungen  

% **Berechnung des Spannungsabfalls über den Widerstand**  
% Da die **Gesamtspannung** der Quelle **12 Volt** beträgt und die Zenerdiode **6 Volt** benötigt, ergibt sich die Spannung über **R** als:  

% \[
% U_R = 12V - 6V = 6V
% \]  

% **Berechnung der Leistung**  
% Die gegebene **maximale Verlustleistung** der Zenerdiode beträgt **0,4 Milliwatt**.  
% Die elektrische Leistung wird nach der Formel berechnet:  

% \[
% P = U \cdot I
% \]  

% Einsetzen der bekannten Werte:  

% \[
% 0,4 mW = 6V \cdot I
% \]  

% Nach **Umstellen nach I** ergibt sich:  

% \[
% I = \frac{0,4 mW}{6V} = 66,7mA
% \]  

% **Berechnung des Widerstands \( R \)**  
% Nach dem **Ohmschen Gesetz** gilt:  

% \[
% U = R \cdot I
% \]  

% Umstellen nach **R**:  

% \[
% R = \frac{U}{I} = \frac{6V}{66,7mA} = 90 \Omega
% \]  

% ### Interpretation  

% Damit die **Zenerdiode** nicht beschädigt wird, muss der Widerstand **R** mindestens **90 Ohm** betragen.  
% Dieser Widerstand begrenzt den Strom auf **66,7 Milliamper**, sodass die **maximale Verlustleistung der Diode** nicht überschritten wird.  

% }

}




\subsection{Einsatz einer Zenerdiode zur Spannungsstabilisierung\label{Aufg6}}
% Alt: Aufgabe 6
\Aufgabe{
  Mit Hilfe einer Zenerdiode soll ein Lastwiderstand \( R_\mathrm{L} = \mathrm{160\,\Omega} \) mit einer stabilisierten Spannung von \( U_\mathrm{A} = \mathrm{8\,V} \) versorgt werden. \\
  Dem Datenblatt der Zenderdiode ist zu entnehmen, dass bei einer Zenerspannung von \( \mathrm{-8\,V} \) ein Zenerstrom von \( \mathrm{-0{,}45\,A} \) fließt.
  
  \begin {figure} [H]
     \centering
     \begin{tikzpicture}
  \draw(0,2) 
    to[V, name=V1, v] (0,0)
    to[short, -o](5,0) -- (7,0)
    to[R, l_=$R_\mathrm{L}$] (7,2)
    to[short, -o](5,2)
    to[short] (4,2)
    to[R, l_=$R_\mathrm{V}$] (0,2);

  \draw(4,0) 
  to[zDo=$D_\mathrm{Z}$, *-*] (4,2) ;

  \draw (5,2) 
  to[open, name=ua, v^] (5,0);
  
  \varrmore{ua}{$U_\mathrm{A}$};
  \varrmore{V1}{$U_\mathrm{E}$};
\end{tikzpicture}
     \label{fig:FigZDiodeStabilisierung}
  \end {figure}

  \begin{itemize}
      \item[a)] Berechnen Sie den Vorwiderstand \( R_\mathrm{V} \), sodass sich bei einer Eingangsspannung von \( U_\mathrm{E} = \mathrm{10\,V} \) eine Ausgangsspannung von \( U_\mathrm{A} = \mathrm{8\,V} \) einstellt.
      \item[b)] Wie groß sind die Leistungen, die an \( R_\mathrm{V} \), \( R_\mathrm{L} \) und an der Diode \( D_\mathrm{Z} \) umgesetzt werden?
  \end{itemize}


%   \speech{ **Aufgabe 6**  
% Mit Hilfe einer **Zenerdiode** soll ein **Lastwiderstand \( R_L = 160Ω \)** mit einer **stabilisierten Spannung \( U_A = 8V \)** versorgt werden.  
% Dem **Datenblatt der Z-Diode** ist zu entnehmen, dass für eine **Zenerspannung von \(-8V\)** ein **Zenerstrom von \(-0,45A\)** fließt.  

% **Fragestellungen:**  
% a) **Berechnen Sie den Vorwiderstand \( R_V \)** so, dass sich bei einer **Eingangsspannung von \( U_E = 10V \)** eine **Ausgangsspannung von \( U_A = 8V \)** einstellt.  
% b) **Wie groß sind die Leistungen, die an \( R_V \), \( R_L \) und \( D_Z \) umgesetzt werden?**  



%  **Beschreibung der Abbildung**  
% Die gezeigte Schaltung ist eine **Spannungsstabilisierungsschaltung mit einer Zenerdiode** und besteht aus folgenden Komponenten:

% 1. **Eingangsspannungsquelle \( U_E \)**  
%    - Symbolisiert durch eine **Gleichspannungsquelle mit \( U_E = 10V \)**.  
%    - Diese **versorgt die gesamte Schaltung** mit Energie.  

% 2. **Vorwiderstand \( R_V \)**  
%    - Platziert **in Serie mit der Zenerdiode**.  
%    - Seine Aufgabe ist die **Begrenzung des Stroms**, um die Zenerdiode nicht zu überlasten.  

% 3. **Zenerdiode \( D_Z \)**  
%    - Schaltet sich bei **Erreichen der Zenerspannung von \(-8V\)** in den **Leitungsmodus**.  
%    - Hält die **Spannung an \( R_L \) konstant auf \( U_A = 8V \)**.  

% 4. **Lastwiderstand \( R_L \) mit \( 160Ω \)**  
%    - Parallel zur **Zenerdiode geschaltet**.  
%    - Die **Ausgangsspannung \( U_A = 8V \)** fällt über \( R_L \) ab.  
%    - Es fließt ein **Laststrom**, der sich aus **\( I_L = \frac{U_A}{R_L} \)** berechnet.  



%  **Funktionsweise der Schaltung**  
% - Wenn die **Eingangsspannung \( U_E = 10V \)** beträgt, sorgt die **Zenerdiode dafür**, dass die **Spannung an \( R_L \) stabil bei 8V bleibt**.  
% - Der **Vorwiderstand \( R_V \)** muss so gewählt werden, dass der **Gesamtstrom aufgeteilt** wird zwischen der **Zenerdiode** und dem **Lastwiderstand**.  
% - Zur **Berechnung der Leistungen** müssen die **verlorene Leistung am Vorwiderstand**, die **Verlustleistung der Zenerdiode** und die **nutzbare Leistung am Lastwiderstand** bestimmt werden.  



%  **Zusammenfassung der Aufgabe**  
% - Die Aufgabe erfordert die **Berechnung des Vorwiderstandes \( R_V \)** zur **Spannungsstabilisierung mit einer Zenerdiode**.  
% - Weiterhin soll bestimmt werden, wie sich die **elektrische Leistung** auf die **verschiedenen Schaltungselemente** verteilt.  
% - Die **Zenerdiode fungiert als Spannungsregler**, indem sie bei überschüssiger Spannung einen **Teil des Stroms aufnimmt**.  
% }

}


\Loesung{
  \begin {figure} [H]
  \centering
  \begin{tikzpicture}
  \draw(0,2) 
    to[V, name=V1, v] (0,0)
    to[short, -o](5,0) -- (8.5,0)
    to[R, l_=$R_\mathrm{L}$, v^<, name=RL] (8.5,2)
    to[short, -o](5,2)
    to[short] (4,2)
    to[R, l_=$R_\mathrm{V}$, i<_, name=RV] (0,2);

  \draw(4,0) 
  to[zDo=$D_\mathrm{D}$, *-*, i_<, name=D] (4,2) ;

  \draw (5,2) 
  to[open, name=ua, v^] (5,0);
  
  \varrmore{ua}{$U_\mathrm{A}$};
  \varrmore{V1}{$U_\mathrm{E}$};
  \varrmore{RL}{$U_\mathrm{RL}$};
  \iarrmore{RV}{${I}$};
  \iarrmore{D}{$I_\mathrm{Z}$};
  
  \draw[->,shift={(2,1)}] (150:0.5) arc (150:-150:0.5) node at(0,0){$M_\mathrm{1}$};  
  \draw[->,shift={(6.5,1)}] (150:0.5) arc (150:-150:0.5) node at(0,0){$M_\mathrm{2}$};
\end{tikzpicture}

  \label{fig:LsgZDiodeStabilisierung}
\end {figure}
\begin{itemize}
\item[a)]
\begin{align*}
  I_\mathrm{RL} &= \frac{\mathrm{8\,V}}{\mathrm{160\,\Omega}} = \mathrm{0{,}05\,A} \\
  I &= \mathrm{0{,}05\,A} + \mathrm{0{,}45\,A} = \mathrm{0{,}5\,A} \\
  U_\mathrm{RV} &= \mathrm{10\,V} - \mathrm{8\,V} = \mathrm{2\,V} \\
  R_\mathrm{V} &= \frac{\mathrm{2\,V}}{\mathrm{0,5\,A}} = \mathrm{4\,\Omega}
\end{align*}

\item[b)]
\begin{align*}
  P_\mathrm{RV} &= \mathrm{2\,V} \cdot \mathrm{0{,}5\,A} = \mathrm{1\,W} \\
  P_\mathrm{RL} &= \mathrm{8\,V} \cdot \mathrm{0{,}05\,A} = \mathrm{0{,}4\,W} \\
  P_\mathrm{DZ} &= \mathrm{8\,V} \cdot \mathrm{0{,}45\,A} = \mathrm{3{,}6\,W}
\end{align*}
\end{itemize}
Siehe: Abschnitt \ref{sec:SpezielleDioden}

% \speech{
% Lösung zu Aufgabe 6:

% Die Aufgabe behandelt eine Schaltung mit einer Zenerdiode, einem Lastwiderstand von 160 Ohm und einer stabilisierten Ausgangsspannung von 8 Volt. Es soll der Vorwiderstand \( R_V \) berechnet werden, sodass sich eine stabile Spannung von 8 Volt einstellt, sowie die Leistungen, die an den Widerständen und der Diode umgesetzt werden.

% ### Teil a: Berechnung des Vorwiderstands \( R_V \)

% Die gegebene Eingangsspannung beträgt 10 Volt, die Zenerspannung der Diode beträgt -8 Volt, und der Strom durch die Zenerdiode ist mit -0,45 Ampere gegeben.

% Zunächst wird der Strom durch den Lastwiderstand \( R_L \) berechnet:
% \[
% I_{R_L} = \frac{U_A}{R_L} = \frac{8V}{160\Omega} = 0,05A
% \]
% Der Gesamtstrom in der Schaltung setzt sich zusammen aus dem Strom durch den Lastwiderstand und dem Strom durch die Zenerdiode:
% \[
% I = I_{R_L} + I_D = 0,05A + 0,45A = 0,5A
% \]
% Die Spannung am Vorwiderstand \( R_V \) ergibt sich als Differenz zwischen Eingangsspannung und Ausgangsspannung:
% \[
% U_{R_V} = U_E - U_A = 10V - 8V = 2V
% \]
% Damit kann der Vorwiderstand berechnet werden:
% \[
% R_V = \frac{U_{R_V}}{I} = \frac{2V}{0,5A} = 4\Omega
% \]

% ### Teil b: Berechnung der Leistungen

% Die Leistung, die am Vorwiderstand umgesetzt wird, ergibt sich aus:
% \[
% P_{R_V} = U_{R_V} \cdot I = 2V \cdot 0,5A = 1W
% \]
% Die Leistung, die am Lastwiderstand \( R_L \) umgesetzt wird:
% \[
% P_{R_L} = U_A \cdot I_{R_L} = 8V \cdot 0,05A = 0,4W
% \]
% Die Verlustleistung der Zenerdiode berechnet sich aus:
% \[
% P_{D_Z} = U_A \cdot I_D = 8V \cdot 0,45A = 3,6W
% \]

% ### Beschreibung der Abbildung:

% Die Schaltung in der Abbildung zeigt eine Spannungsquelle mit 10 Volt, die über den Vorwiderstand \( R_V \) mit einer Zenerdiode in Parallelschaltung mit dem Lastwiderstand \( R_L \) verbunden ist. Der Strom durch den Vorwiderstand wird als Pfeil in die Zenerdiode eingezeichnet, wo sich der Strom in zwei Teilströme aufteilt: ein Teil fließt durch den Lastwiderstand, ein anderer durch die Zenerdiode. Die Spannungen sind ebenfalls eingezeichnet: Die Versorgungsspannung mit 10 Volt, die stabilisierte Spannung von 8 Volt über der Zenerdiode und dem Lastwiderstand sowie die Spannung über dem Vorwiderstand mit 2 Volt.

% Die Berechnungen in der Abbildung sind klar strukturiert und zeigen Schritt für Schritt, wie die Widerstandswerte und Leistungen bestimmt werden. Die Ergebnisse bestätigen die geforderte Ausgangsspannung von 8 Volt und zeigen die Leistungsverteilung in der Schaltung.
% }

}




\subsection{Ermittlung der Betriebsgrenzen einer Zenerdiode\label{Aufg7}}
% Alt: Aufgabe 7
\Aufgabe{
  Die eingesetzte Zenerdiode besitzt eine Zenerspannung von \( U_\mathrm{Z} = \mathrm{8\,V} \). 
  Die maximale Verlustleistung der Diode beträgt \( P_\mathrm{tot}=\mathrm{8\,W} \). 
  Die Diode arbeitet im stabilisierenden Bereich, wenn der Strom durch sie mindestens \( \mathrm{5\,mA} \) beträgt. \\
  Welchen minimalen und maximalen Widerstandswert darf der Lastwiderstand \( R_\mathrm{L} \) nicht unter- bzw. überschreiten, damit die Diode einerseits im stabilisierenden Bereich arbeitet und andererseits nicht überlastet wird?
  
  \begin {figure} [H]
     \centering
     \begin{tikzpicture}
  \draw(0,2) 
    to[V, name=V1, v] (0,0)
    to[short, -o](5,0) -- (7,0)
    to[R, l_=$R_\mathrm{L}$] (7,2)
    to[short, -o](5,2)
    to[short] (4,2)
    to[R, l_=$R_\mathrm{V}$] (0,2);

  \draw(4,0) 
  to[zDo=$D_\mathrm{Z}$, *-*] (4,2) ;

  \draw (5,2) 
  to[open, name=ua, v^] (5,0);
  
  \varrmore{ua}{$U_\mathrm{A}$};
  \varrmore{V1}{$U_\mathrm{E}$};
\end{tikzpicture}
     \label{fig:FigZDiodeGrenzen}
  \end {figure}

%   \speech{ **Aufgabe 7**  
% Die eingesetzte **Zenerdiode** besitzt eine **Zenerspannung von \( U_Z = 8V \)**.  
% - Die **maximale Verlustleistung** der Diode beträgt **8 W**.  
% - Die Diode arbeitet **im stabilisierenden Bereich**, wenn der **Strom durch die Diode größer als 5 mA** ist.  

% **Fragestellung:**  
% - **Welchen minimalen und maximalen Widerstandswert darf der Lastwiderstand \( R_L \) nicht unter- oder überschreiten**, damit die **Diode im stabilisierenden Bereich arbeitet** und nicht überlastet wird?  
%  **Beschreibung der Abbildung**  
% Die gezeigte Schaltung stellt eine **Zener-Spannungsstabilisierungsschaltung** dar und besteht aus folgenden Komponenten:

% 1. **Eingangsspannungsquelle**  
%    - **Nennwert: 20V** mit einer **Toleranz von ±10%**, also zwischen **18V und 22V**.  
%    - Symbolisiert durch einen **Kreis mit Richtungspfeil**.  

% 2. **Serienwiderstand (10Ω)**  
%    - Platziert im **oberen Zweig der Schaltung**.  
%    - Begrenzung des **Gesamtstroms** in der Schaltung.  

% 3. **Zenerdiode \( D_Z \)**  
%    - In **Sperrrichtung** betrieben.  
%    - Hält die **Spannung über \( R_L \) stabil bei 8V**.  
%    - Beginnt **zu leiten**, wenn die Spannung über **8V steigt** und sorgt für eine **Überspannungsbegrenzung**.  

% 4. **Lastwiderstand \( R_L \)**  
%    - Parallel zur **Zenerdiode** geschaltet.  
%    - Soll mit einer **stabilen Spannung von 8V** betrieben werden.  
%    - Der zulässige **Widerstandsbereich für \( R_L \)** ist zu bestimmen, damit die **Diode weder zu wenig noch zu viel Strom führt**.  

%  **Funktionsweise der Schaltung**  
% - Die **Zenerdiode regelt die Ausgangsspannung auf 8V** unabhängig von **Schwankungen der Eingangsspannung**.  
% - Die **minimale Last \( R_L \) darf nicht zu klein** sein, damit die **Diode noch genug Strom aufnimmt**.  
% - Die **maximale Last \( R_L \) darf nicht zu groß** sein, da sonst die **Zenerdiode zu viel Strom aufnehmen müsste**, was zu einer **Überlastung (8W-Grenze) führen könnte**.  
% - Das **Problem wird durch eine Berechnung der Stromverhältnisse** gelöst.  

% **Zusammenfassung der Aufgabe**  
% - Die Aufgabe erfordert die **Bestimmung des minimalen und maximalen Widerstands \( R_L \)**, damit die **Zenerdiode richtig arbeitet**.  
% - Die **Spannungsschwankungen der Quelle** müssen berücksichtigt werden.  
% - Die Berechnung erfolgt unter der **Bedingung, dass der Strom durch die Zenerdiode größer als 5mA bleibt** und die **maximale Leistung von 8W nicht überschritten wird**.  
% }

}


\Loesung{
  \begin{figure}[H]
    \centering
    \begin{tikzpicture}
    \draw(0,2)
    to[V, name=V1, v] (0,0)
    to[short, -o](5,0) -- (7,0)
    to[R, l_=$R_\mathrm{L}$, v^<, name=RL] (7,2)
    to[short, -o](5,2)
    to[short, i<_, name=IL] (4,2)
    to[R, l_=$R_\mathrm{V}$, i<_, name=RV] (0,2);
    
    \draw(4,0)
    to[zDo=$D_\mathrm{Z}$, *-*, i_<, name=D] (4,2) ;
    
    \draw (5,2)
    to[open, name=ua, v^] (5,0);
    
    \varrmore{ua}{$U_\mathrm{A}$};
    \varrmore{V1}{$U_\mathrm{E}$};
    \iarrmore{RV}{${I}$};
    \iarrmore{D}{$I_\mathrm{D}$};
    \iarrmore{IL}{$I_\mathrm{L}$};
    
\end{tikzpicture}
    \label{fig:LsgZDiodeGrenzen}
  \end{figure}

  Gegeben sind:
\begin{align*}
    U_\mathrm{max} &= \mathrm{22\,V} \\
    U_\mathrm{min} &= \mathrm{18\,V}
\end{align*}

\textbf{Fall 1:} Bei \( U_\mathrm{max} = \mathrm{22\,V} \)
\begin{align*}
    U_\mathrm{RV} &= \mathrm{22\,V} - \mathrm{8\,V} = \mathrm{14\,V} \\
    I &= \frac{\mathrm{14\,V}}{\mathrm{10\,\Omega}} = \mathrm{1{,}4\,A} \\
    I &= I_\mathrm{D} + I_\mathrm{L} \\
    I_\mathrm{D,\ max} &= \frac{\mathrm{8\,W}}{\mathrm{8\,V}} = \mathrm{1\,A} \\
    I_\mathrm{L} &= \mathrm{1{,}4\,A} - \mathrm{1\,A} = \mathrm{0{,}4\,A} \\
    R_\mathrm{L} &= \frac{\mathrm{8\,V}}{\mathrm{0{,}4\,A}} = \mathrm{20\,\Omega} \Rightarrow R_\mathrm{L,\ max} = \mathrm{20\,\Omega}
\end{align*}

\textbf{Fall 2:} Bei \( U_\mathrm{min} = \mathrm{18\,V} \)
\begin{align*}
    U_\mathrm{RV} &= \mathrm{18\,V} - \mathrm{8\,V} = \mathrm{10\,V} \\
    I &= \frac{\mathrm{10\,V}}{\mathrm{10\,\Omega}} = \mathrm{1\,A} \\
    I_\mathrm{D,\ min} &= \mathrm{5\,mA} \\
    I_\mathrm{L} &= \mathrm{1\,A} - \mathrm{5\,mA} = \mathrm{0{,}995\,A} \\
    R_\mathrm{L} &= \frac{\mathrm{8\,V}}{\mathrm{0{,}995\,A}} \approx \mathrm{8\,\Omega} \Rightarrow R_\mathrm{L,\ min} \approx \mathrm{8\,\Omega}
\end{align*}
%   \speech{
% Lösung zu Aufgabe 7:

% Gegeben ist eine Schaltung mit einer Spannungsquelle von 20 Volt plus-minus 10 Prozent, einem Vorwiderstand von 10 Ohm, einer Zenerdiode mit einer Zenerspannung von 8 Volt und einem variablen Lastwiderstand R L. Ziel ist es, die minimalen und maximalen Werte für R L zu bestimmen, sodass die Zenerdiode im stabilisierenden Bereich arbeitet, aber nicht überlastet wird.

% Betrachtung für die maximale Eingangsspannung U max gleich 22 Volt:
% - Die Spannung über dem Vorwiderstand beträgt:
%   U R V gleich 22 Volt minus 8 Volt gleich 14 Volt.
% - Der Gesamtstrom durch den Vorwiderstand ist:
%   I gleich U R V geteilt durch R V.
%   I gleich 14 Volt geteilt durch 10 Ohm gleich 1,4 Ampere.
% - Der Gesamtstrom teilt sich auf in den Strom durch die Zenerdiode I D und den Laststrom I L:
%   I gleich I D plus I L.
% - Die maximale Verlustleistung der Zenerdiode ist gegeben mit 8 Watt. Der maximale Strom durch die Zenerdiode ergibt sich durch:
%   I D max gleich P max geteilt durch U Z.
%   I D max gleich 8 Watt geteilt durch 8 Volt gleich 1 Ampere.
% - Daraus ergibt sich der Laststrom:
%   I L gleich I minus I D max.
%   I L gleich 1,4 Ampere minus 1 Ampere gleich 0,4 Ampere.
% - Die Spannung über dem Lastwiderstand beträgt:
%   U L gleich U Z gleich 8 Volt.
% - Der maximale Lastwiderstand ist dann:
%   R L max gleich U L geteilt durch I L.
%   R L max gleich 8 Volt geteilt durch 0,4 Ampere gleich 20 Ohm.

% Betrachtung für die minimale Eingangsspannung U min gleich 18 Volt:
% - Die Spannung über dem Vorwiderstand beträgt:
%   U R V gleich 18 Volt minus 8 Volt gleich 10 Volt.
% - Der Gesamtstrom durch den Vorwiderstand ist:
%   I gleich U R V geteilt durch R V.
%   I gleich 10 Volt geteilt durch 10 Ohm gleich 1 Ampere.
% - Der Gesamtstrom teilt sich wieder auf in den Strom durch die Zenerdiode I D und den Laststrom I L.
% - Die minimale Stromaufnahme der Zenerdiode, damit sie im stabilisierenden Bereich arbeitet, beträgt 5 Milliampere.
% - Daraus ergibt sich der Laststrom:
%   I L gleich I minus I D min.
%   I L gleich 1 Ampere minus 5 Milliampere gleich 995 Milliampere.
% - Die Spannung über dem Lastwiderstand beträgt:
%   U L gleich 8 Volt.
% - Der minimale Lastwiderstand ist dann:
%   R L min gleich U L geteilt durch I L.
%   R L min gleich 8 Volt geteilt durch 0,995 Ampere gleich 8 Ohm.

% Zusammenfassung:
% - Der Lastwiderstand R L muss zwischen 8 Ohm und 20 Ohm liegen, damit die Zenerdiode im stabilisierenden Bereich bleibt und nicht überlastet wird.
% }
Siehe: Abschnitt \ref{sec:SpezielleDioden}
}




\subsection{Einsatz einer Diode zur Leistungsreduzierung\label{Aufg8}}
% Alt: Aufgabe 8
\Aufgabe{
    In einer einfachen Schaltung wird eine Diode zur Leistungsreduktion eingesetzt. Der Widerstand $R_\mathrm{L} = 1\,\mathrm{k\Omega}$ stellt eine Heizlast dar. 
    Die Netzspannung beträgt $230\,\mathrm{V}$ Effektivwert bei einer Frequenz von $50\,\mathrm{Hz}$.
    
    \begin{figure}[H]
        \centering
        \begin{tikzpicture}

    % Batterie links
    \draw
    (0,0) to[sV, v<,name=V1 ] (0,2);
  
    % Widerstand
    \draw
    (0,2) to[D=$D$, v, name=D1] (4,2);

    \draw (0,0) -- (4,0);
  
    % LED rechts
    \draw (4,2) to [R, l=$R_\mathrm{L}$] (4,0);
   

    \varrmore{V1}{$u_\mathrm{E}$};
    \varrmore{D1}{$U_\mathrm{D}$};

  \end{tikzpicture}
        \label{fig:FigDiodeLeistungsreduzierung}
    \end{figure}
    
    \begin{itemize}
        \item[a)] Welche mittlere Leistung wird in $R_\mathrm{L}$ und in der Diode umgesetzt, wenn die Diode als idealer Gleichtrichter wirkt (d.h. nur eine Halbwelle durchlässt)?
        \item[b)] Wie groß wäre die mittlere Leistung bei überbrückter Diode (volle Netzspannung an $R_\mathrm{L}$)?
    \end{itemize}
    
    %   \speech{ **Aufgabe 8**  
    % In einer **einfachen Schaltung** wird eine **Diode zur Leistungsreduktion** eingesetzt.  
    % - Der **Lastwiderstand \( R_L \)** ist ein **Heizwiderstand** mit einem **Widerstand von 1kΩ**.  
    % - Die **Eingangsspannung** hat einen **Effektivwert von 230V bei 50Hz**.  
    
    % **Fragestellungen:**  
    % a) **Welche mittlere Leistung würde in \( R_L \) und in der Diode umgesetzt**, wenn die **Diode ideal als Ventil** funktionieren würde?  
    % b) **Wie groß wäre die mittlere Leistung bei überbrückter Diode?**  
    
    
    
    %  **Beschreibung der Abbildung**  
    % Die gezeigte Schaltung stellt eine **einphasige Halbwellenschaltung mit einer Diode** dar. Die Hauptkomponenten sind:
    
    % 1. **Wechselspannungsquelle \( u_e \)**  
    %    - Symbolisiert durch einen **Sinusgenerator**.  
    %    - Spannung: **230V Effektivwert bei 50Hz**.  
    
    % 2. **Diode \( D \) (Leistungsreduktionselement)**  
    %    - In **Reihe mit dem Lastwiderstand** geschaltet.  
    %    - Funktioniert als **Ventil**, indem sie **eine Halbwelle der Wechselspannung sperrt**.  
    
    % 3. **Lastwiderstand \( R_L = 1kΩ \)**  
    %    - Platziert **nach der Diode**.  
    %    - Ersetzt eine **reale Last, z.B. einen Heizwiderstand**.  
    %    - Umwandlung der elektrischen Leistung in **Wärme**.  
    
    
    
    %  **Funktionsweise der Schaltung**  
    % - Die **Diode sperrt die negative Halbwelle**, sodass nur **eine Halbwelle der Wechselspannung** am **Widerstand \( R_L \)** anliegt.  
    % - Dies führt zu einer **Reduktion der mittleren Leistung im Widerstand**.  
    % - Bei **überbrückter Diode** würde der **volle sinusförmige Wechselstrom** durch den Widerstand fließen, was zu **einer anderen mittleren Leistung führt**.  
    
    
    
    %  **Zusammenfassung der Aufgabe**  
    % - Die Aufgabe erfordert die **Berechnung der mittleren Leistung** in \( R_L \) und der **Diode**, wenn sie **ideal als Ventil** funktioniert.  
    % - Zudem muss untersucht werden, wie sich die Leistung **bei überbrückter Diode** verhält.  
    % - Es wird erwartet, dass sich die **mittlere Leistung um den Faktor 2 unterscheidet**, je nachdem ob die **Diode aktiv ist oder nicht**.  
    % }
    
}


\Loesung{
    \begin{itemize}
        \item[a)]
        Bei einem idealen Ventil: Nach dem Erreichen der Flussspannung verhält sich die Diode wie ein perfekter Leiter ohne Widerstand. \\
        Da nur eine Halbwelle der Wechselspannung durchgelassen wird, halbiert sich die Leistung im Lastwiderstand. Die Diode selbst setzt keine Leistung um:
        \begin{align*}
            P_\mathrm{D} &= 0\,\mathrm{W}
        \end{align*}
        Die mittlere Leistung im Lastwiderstand beträgt:
        \begin{align*}
            P_\mathrm{RL} &= \frac{1}{2} \cdot \frac{(U_\mathrm{eff})^2}{R_\mathrm{L}} = \frac{1}{2} \cdot \frac{(230\,\mathrm{V})^2}{1\,\mathrm{k\Omega}} = 26,45\,\mathrm{W}
        \end{align*}
    
        \item[b)]
        Bei überbrückter Diode liegt die volle Wechselspannung am Lastwiderstand an:
        \begin{align*}
            P_\mathrm{RL} &= \frac{U_\mathrm{eff}^2}{R_\mathrm{L}} = \frac{230\,\mathrm{V}^2}{1\,\mathrm{k\Omega}} = 52,9\,\mathrm{W}
        \end{align*}
    \end{itemize} 
% \speech{
% Lösung zu Aufgabe 8:

% Die gegebene Schaltung enthält eine Diode in Reihe mit einem Lastwiderstand von 1 Kiloohm. Die Eingangsspannung ist eine Wechselspannung mit einer Effektivspannung von 230 Volt und einer Frequenz von 50 Hertz.

% **Teilaufgabe a:**  
% Es wird davon ausgegangen, dass die Diode ein ideales Ventil ist. Das bedeutet, dass sie nach dem Erreichen der Flussspannung wie ein perfekter Leiter ohne Widerstand funktioniert. In diesem Fall beträgt die Verlustleistung der Diode \( P_D = 0 \) Watt.

% Die Leistung am Lastwiderstand kann mit der Formel berechnet werden:

% \[
% P_{RL} = \frac{U^2}{R} = \frac{U_{eff}^2}{R_L}
% \]

% Durch Einsetzen der gegebenen Werte:

% \[
% P_{RL} = \frac{(230V)^2}{1k\Omega} = 52,9 W
% \]

% Das bedeutet, dass die gesamte Leistung am Lastwiderstand umgesetzt wird.

% **Teilaufgabe b:**  
% Wird die Diode überbrückt, dann liegt die gesamte Wechselspannung ungehindert am Lastwiderstand an. Da sich die Schaltung in diesem Fall wie ein reiner Widerstand verhält, bleibt die berechnete Leistung im Lastwiderstand gleich groß.

% Das bedeutet, dass die mittlere Leistung in beiden Fällen \( 52,9 \) Watt beträgt, unabhängig davon, ob die Diode vorhanden ist oder überbrückt wurde.
% }
Siehe: Abschnitt \ref{sec:Einweggleichrichter}
}




\subsection{Spannungsverlauf in einer Brückengleichrichterschaltung\label{Aufg9}}
% Alt: Aufgabe 9
\Aufgabe{
  Skizzieren Sie die Spannung $U_1$ der gegebenen Schaltung.
  Beachten Sie, dass es sich bei $u_\mathrm{E}$ um eine Wechselspannung handelt. 
  \begin {figure} [H]
     \centering
     \begin{tikzpicture}[]

    \draw (6,1.5) to[open,v>, name=u1] (6,-0.5);
    \draw (-0.5,3)
        to[open,v>, name=u0] (-0.5,0);

        \draw (-0.5,3)
        to[short, o-*] (3,3)
        to[D, -*, l=$D_2$] (4.5,1.5)
        to[short, i, -] (4.5,1.5)
        to[short, -o] (6,1.5);
        
    % Verbindungslinie zum unteren Punkt
    \draw (6,-0.5) to[short, o-] (1.5,-0.5) ;
    

    \draw (1.5,-0.5)
        to[short, -*] (1.5,1.5)
        to[D, -*, l=$D_3$] (3,0)
        to[short, -o] (-0.5,0);

    \draw (3,0)
        to[D, -, l=$D_4$] (4.5,1.5);

    \draw (1.5,1.5)
        to[D, -, l=$D_1$] (3,3);


    \varrmore{u0}{$u_\mathrm{E}$};
    \varrmore{u1}{$U_{1}$};
\end{tikzpicture}
     \label{fig:FigDiodeSpannugsverlauf}
  \end {figure}
%   \speech{ **Aufgabe 9**  
% Skizzieren Sie die **Spannung \( U_1 \)** der gegebenen Schaltung.  
% Beachten Sie, dass es sich bei \( u_E \) um eine **Wechselspannung** handelt.  


%  **Beschreibung der Abbildung**  
% Die Abbildung zeigt eine **Brückengleichrichterschaltung** mit folgenden Komponenten:

% 1. **Eingangsspannung \( u_E \)**  
%    - Eine **Wechselspannung**, symbolisiert durch eine **Spannungsquelle mit sinusförmigem Symbol**.  
%    - Der **Eingangsstrom fließt in beide Richtungen** abhängig von der Wechselspannung.  

% 2. **Gleichrichterbrücke mit vier Dioden** (\( D_1, D_2, D_3, D_4 \))  
%    - **Dioden sind in Brückenanordnung** geschaltet.  
%    - Sie sorgen dafür, dass **sowohl die positive als auch die negative Halbwelle der Eingangsspannung gleichgerichtet wird**.  
%    - In jeder Halbwelle **leiten zwei Dioden und sperren zwei Dioden**.  

% 3. **Gleichgerichtete Ausgangsspannung \( U_1 \)**  
%    - Liegt **hinter der Gleichrichterbrücke** an.  
%    - Ist eine **pulsierende Gleichspannung**, da **nur positive Halbwellen durchgelassen werden**.  
%    - **Negative Halbwellen der Wechselspannung werden umgekehrt** und erscheinen ebenfalls **positiv am Ausgang**.  


%  **Funktionsweise der Schaltung**  
% - Während der **positiven Halbwelle** von \( u_E \):  
%   - \( D_1 \) und \( D_2 \) leiten, **\( D_3 \) und \( D_4 \) sperren**.  
%   - Der Strom fließt durch die **Last in eine Richtung**.  

% - Während der **negativen Halbwelle** von \( u_E \):  
%   - \( D_3 \) und \( D_4 \) leiten, **\( D_1 \) und \( D_2 \) sperren**.  
%   - Der Strom fließt erneut durch die **Last in die gleiche Richtung**, da die **negative Spannung umgekehrt wird**.  

% - **Ergebnis:**  
%   - Die **Ausgangsspannung \( U_1 \) hat nur positive Werte** und ist eine **pulsierende Gleichspannung**.  
%   - Die Aufgabe besteht darin, diese **Spannung \( U_1 \) als Skizze** darzustellen.  



%  **Zusammenfassung der Aufgabe**  
% - Die Aufgabe erfordert eine **Skizze der Ausgangsspannung \( U_1 \)**.  
% - Aufgrund der **Brückengleichrichtung** besteht \( U_1 \) aus **positiven Halbwellen** der ursprünglichen Wechselspannung.  
% - Die **negative Halbwelle wird gespiegelt**, sodass **kein negatives Signal am Ausgang erscheint**.  
% - Der Skizze soll der typische Verlauf einer **pulsierenden Gleichspannung** entsprechen.  
% }

}


\Loesung{
  \begin{itemize}
    \item Bei einer negativen Halbwelle fließt der Strom über $D_{\text{2}}$ und $D_{\text{3}}$.
  \end{itemize}
  \begin{figure}[H]
    \centering
    \begin{tikzpicture}
    \draw (6,1.5) to[open,v>, name=u1] (6,-0.5);
    \draw (-0.5,3)
        to[open,v>, name=u0] (-0.5,0);

        \draw (-0.5,3)
        to[short, o-*] (3,3)
        to[D, -*, l=$D_2$] (4.5,1.5)
        to[short, i, -] (4.5,1.5)
        to[short, -o] (6,1.5);
        
    % Verbindungslinie zum unteren Punkt
    \draw (6,-0.5) to[short, o-] (1.5,-0.5) ;
    

    \draw (1.5,-0.5)
        to[short, -*] (1.5,1.5)
        to[D, -*, l=$D_3$] (3,0)
        to[short, -o] (-0.5,0);

        \draw (1.5,3) to[short, i,name=i1](1.6,3);
        \iarrmore{i1}{};
        \draw (3,3) to[short, i,name=i2](4.5,1.5);
        \iarrmore{i2}{};
        \draw (4.5,1.5) to[short, i,name=i3] (6,1.5);
        \iarrmore{i3}{};
        \draw(6,-0.5) to[short,i,name=i4] (1.5,-0.5);
        \iarrmore{i4}{};
        \draw (1.5,-0.5) to[short,i,name=i5](1.5,1);
        \iarrmore{i5}{};
        \draw (1.5,1.5) to[short,i,name=i6](3,0);
        \iarrmore{i6}{};
        \draw (1,0) to[short,i,name=i7] (-0.5,0);
        \iarrmore{i7}{};
        

    \draw (3,0)
        to[D, -, l=$D_4$] (4.5,1.5);

    \draw (1.5,1.5)
        to[D, -, l=$D_1$] (3,3);


    \varrmore{u0}{$u_\mathrm{E}$};
    \varrmore{u1}{$U_{1}$};
   
   \end{tikzpicture}
   
    \label{fig:LsgDiodeSpannungsverlauf1}
  \end{figure}

  \begin{itemize}
    \item Bei einer positiven Halbwelle fließt der Strom über $D_{\text{4}}$ und $D_{\text{1}}$.
  \end{itemize}
  \begin{figure}[H]
    \centering
    \begin{tikzpicture}
    \draw (6,1.5) to[open,v>, name=u1] (6,-0.5);
    \draw (-0.5,3)
        to[open,v>, name=u0] (-0.5,0);

        \draw (-0.5,3)
        to[short, o-*] (3,3)
        to[D, -*, l=$D_2$] (4.5,1.5)
        to[short, i, -] (4.5,1.5)
        to[short, -o] (6,1.5);
        
    % Verbindungslinie zum unteren Punkt
    \draw (6,-0.5) to[short, o-] (1.5,-0.5) ;
    

    \draw (1.5,-0.5)
        to[short, -*] (1.5,1.5)
        to[D, -*, l=$D_3$] (3,0)
        to[short, -o] (-0.5,0);

        \draw (1.6,3) to[short, i,name=i1](1.5,3);
        \iarrmore{i1}{};
        \draw (1.5,1.5) to[short, i,name=i2](3,3);
        \iarrmore{i2}{};
        \draw (4.5,1.5) to[short, i,name=i3] (6,1.5);
        \iarrmore{i3}{};
        \draw(6,-0.5) to[short,i,name=i4] (1.5,-0.5);
        \iarrmore{i4}{};
        \draw (1.5,-0.5) to[short,i,name=i5](1.5,1);
        \iarrmore{i5}{};
        \draw (3,0) to[short,i,name=i6](4.5,1.5);
        \iarrmore{i6}{};
        \draw (1,0) to[short,i,name=i7] (1.5,0);
        \iarrmore{i7}{};
        

    \draw (3,0)
        to[D, -, l=$D_4$] (4.5,1.5);

    \draw (1.5,1.5)
        to[D, -, l=$D_1$] (3,3);


    \varrmore{u0}{$u_\mathrm{E}$};
    \varrmore{u1}{$U_{1}$};
   
\end{tikzpicture}

    \label{fig:LsgDiodeSpannungsverlauf2}
  \end{figure}
%   \speech{
% Lösung zu Aufgabe 9:

% Die gegebene Schaltung stellt eine Gleichrichterschaltung mit vier Dioden in Brückenschaltung dar. Die Eingangsspannung \( u_E \) ist eine Wechselspannung, die gleichgerichtet wird, sodass am Ausgang \( U_1 \) eine pulsierende Gleichspannung entsteht.

% Die Aufgabe analysiert den Stromfluss während der positiven und negativen Halbwelle der Eingangsspannung.

% **Erster Fall: Negative Halbwelle der Eingangsspannung**  
% Wenn die Eingangsspannung negativ ist, dann ist die untere Seite der Spannungsquelle positiver als die obere. Dadurch werden die Dioden \( D_2 \) und \( D_3 \) leitend, während \( D_1 \) und \( D_4 \) sperren.  

% - Der Strom fließt von der positiven Klemme der Spannungsquelle nach oben in die Anode von \( D_2 \).
% - Er passiert \( D_2 \) in Durchlassrichtung, gelangt zur Last und fließt dann weiter über \( D_3 \) in Richtung der negativen Klemme der Spannungsquelle.
% - Dadurch wird die Ausgangsspannung \( U_1 \) in der gleichen Polarität gehalten.

% **Zweiter Fall: Positive Halbwelle der Eingangsspannung**  
% Wenn die Eingangsspannung positiv ist, dann ist die obere Seite der Spannungsquelle positiver als die untere. Dadurch werden die Dioden \( D_1 \) und \( D_4 \) leitend, während \( D_2 \) und \( D_3 \) sperren.  

% - Der Strom fließt nun von der positiven Klemme der Spannungsquelle in die Anode von \( D_1 \).
% - Er passiert \( D_1 \) in Durchlassrichtung, gelangt zur Last und fließt dann über \( D_4 \) zurück zur negativen Klemme der Spannungsquelle.
% - Auch in diesem Fall bleibt die Polarität von \( U_1 \) erhalten, sodass eine gleichgerichtete Spannung entsteht.

% **Fazit:**  
% Diese Brückengleichrichterschaltung sorgt dafür, dass unabhängig von der Polarität der Eingangsspannung immer eine positive Spannung am Ausgang entsteht. Die Strompfade wechseln mit jeder Halbwelle der Wechselspannung, wobei immer zwei Dioden leitend sind, während die anderen beiden sperren.
% }
Siehe: Abschnitt \ref{sec:Brückengleichrichter}
}



 
\subsection{Berechnung des Vorwiderstands einer LED\label{Aufg10}}
% Alt: Aufgabe 10
\Aufgabe{
  Wie bei allen Dioden muss auch bei Leuchtdioden (LEDs) der Strom begrenzt werden. Dazu wird ein Vorwiderstand in Reihe geschaltet, wie in der unten gezeigten Schaltung. \\
  Die eingesetzte LED besitzt eine Durchlassspannung von $U_\mathrm{D} = 2,2\,\mathrm{V}$. Die ideale Helligkeit wird bei einem Strom von $I_\mathrm{D} = 30\,\mathrm{mA}$ erreicht. \\
  Die Eingangsspannung beträgt $U_\mathrm{0} = 5\,\mathrm{V}$.
  
  \begin{itemize}
    \item Welchen der folgenden Widerstandswerte würden Sie für $R$ verwenden, um die LED möglichst nahe an ihrem optimalen Betriebspunkt (d.\,h. mit optimaler Helligkeit) zu betreiben?
    \item $15\,\Omega$, $33\,\Omega$, $68\,\Omega$, $150\,\Omega$, $180\,\Omega$
  \end{itemize}
  
  \begin {figure} [H]
     \centering
     \begin{tikzpicture}
  \draw (0,2) to[V, name=U0] (0,0);
  \draw (0,2) to[R=$R_\mathrm{V}$, i, name=Id] (4,2);
  \draw (4,2) to[leDo, l=LED, v, name=Ud] (4,0) -- (0,0);

  \varrmore{U0}{$U_\mathrm{0}$};
  \varrmore{Ud}{$U_\mathrm{D}$};
  \iarrmore{Id}{$I_\mathrm{D}$};
\end{tikzpicture}
     \label{fig:FigLEDVorwiderstand}
  \end {figure}
%   \speech{ **Aufgabe 10**  
% Wie bei allen **Dioden** ist es wichtig, den **Strom durch das Bauelement zu begrenzen**.  
% Dazu werden **Vorwiderstände** eingesetzt, wie in der unten gezeigten Schaltung.  

% - Die verwendete **LED** hat eine **Durchlassspannung von \( U_D = 2.2V \)**.  
% - Die **optimale Helligkeit** wird bei einem **Strom von \( I_D = 30mA \)** erreicht.  
% - Die **Eingangsspannung** beträgt **\( U_0 = 5V \)**.  

% 

% **Aufgabenstellung:**  
% - Welchen der folgenden **Widerstandswerte** würden Sie für **\( R_V \)** wählen, um die LED möglichst **nahe an ihrem optimalen Betriebspunkt** zu betreiben?  
%   - **15Ω, 33Ω, 68Ω, 150Ω, 180Ω**

% 

%  **Beschreibung der Abbildung**  
% Die gezeigte **elektrische Schaltung** stellt eine **einfache LED-Vorwiderstandsschaltung** dar. Sie besteht aus:

% 1. **Spannungsquelle \( U_0 \)**  
%    - Symbolisiert durch eine **Gleichspannungsquelle mit 5V**.  
%    - Stellt die benötigte **Betriebsspannung für die LED** bereit.  

% 2. **Vorwiderstand \( R_V \)**  
%    - In **Reihe mit der LED geschaltet**.  
%    - Begrenzt den **Stromfluss durch die LED**, um Schäden zu vermeiden.  

% 3. **LED mit Durchlassspannung \( U_D \)**  
%    - Platziert **nach dem Vorwiderstand**.  
%    - Leuchtet, sobald die **Betriebsspannung \( U_0 \)** groß genug ist.  
%    - Hat eine **Durchlassspannung von 2.2V**.  
%    - Der **gewählte Widerstandswert beeinflusst die Stromstärke und damit die Helligkeit**.  

% 

% ### **Funktionsweise der Schaltung**  
% - Die **Gesamtspannung von 5V** teilt sich auf **Vorwiderstand \( R_V \)** und **LED \( U_D \)** auf.  
% - Der **Widerstand \( R_V \)** muss so gewählt werden, dass der **gewünschte Strom von 30mA** fließt.  
% - Durch **Ohm’sches Gesetz** lässt sich die **notwendige Widerstandswert berechnen**.  
% - Zu hohe Werte von **\( R_V \)** würden **die LED dunkler machen**, zu niedrige Werte **könnten die LED beschädigen**.  

% }

}


\Loesung{
Berechnung der Spannung am Vorwiderstand:
\begin{equation}
    U_\mathrm{V} = U_\mathrm{0} - U_\mathrm{D}
\end{equation}
\begin{align*}
    U_\mathrm{V} = 5\,\mathrm{V} - 2,2\,\mathrm{V} = 2,8\,\mathrm{V}
\end{align*}

Berechnung des optimalen Vorwiderstands für \( I_\mathrm{D} = \mathrm{30\,mA} \):
\begin{equation}
    R_\mathrm{V} = \frac{U_\mathrm{V}}{I_\mathrm{D}}
\end{equation}
\begin{align*}
    R_\mathrm{V} = \frac{2,8\,\mathrm{V}}{30\,\mathrm{mA}} = 93,3\,\Omega
\end{align*}

Berechnung des Stroms bei Auswahl eines konkreten Widerstandswertes:
\begin{equation}
    I_\mathrm{D} = \frac{U_\mathrm{V}}{R}
\end{equation}
\begin{align*}
    I_\mathrm{D} &= \frac{2,8\,\mathrm{V}}{150\,\Omega} = 18,7\,\mathrm{mA}
\end{align*}

Berechnung bei zwei Widerständen in Parallelschaltung mit zusätzlichem Serienwiderstand:
\begin{align*}
    R_\mathrm{ges} &= \frac{150\,\Omega \cdot 180\,\Omega}{150\,\Omega + 180\,\Omega} + 15\,\Omega = 96,8\,\Omega \\
    I_\mathrm{D} &= \frac{2,8\,\mathrm{V}}{96,8\,\Omega} =28,9\,\mathrm{mA}
\end{align*}

Alternative Reihenschaltung:
\begin{align*}
    R_\mathrm{ges} &= 4 \cdot 15\,\Omega + 33\,\Omega = 93\,\Omega
\end{align*}

\textbf{Fazit:}  
Ein kombinierter Widerstand aus $4 \times 15\,\Omega + 33\,\Omega = 93\,\Omega$ oder eine Parallelschaltung mit $150\,\Omega \,\|\, 180\,\Omega + 15\,\Omega = 96,8\,\Omega$ führt zu einem Strom nahe dem optimalen Wert von  $30\,\mathrm{mA}$. \\
\textbf{Empfohlene Wahl:} Kombination mit Gesamtwiderstand nahe $93\,\Omega$.

  Siehe: Abschnitt \ref{sec:SpezielleDioden}
%   \speech{
% Lösung zu Aufgabe 10:

% In dieser Aufgabe geht es um eine Schaltung mit einer Leuchtdiode (LED), die durch einen Vorwiderstand geschützt wird. Ziel ist es, den optimalen Widerstandswert für die LED zu bestimmen, sodass sie mit der idealen Helligkeit betrieben wird.

% **Gegebene Werte:**
% - Die LED besitzt eine Durchlassspannung von 2,2 Volt.
% - Der Strom durch die LED sollte 30 Milliampere betragen.
% - Die Eingangsspannung beträgt 5 Volt.

% **Formel zur Berechnung der Spannung über dem Vorwiderstand \(R_V\):**
% Die Spannung über dem Vorwiderstand berechnet sich durch die Differenz zwischen der Eingangsspannung und der Durchlassspannung der LED:
% \[
% U_V = U_0 - U_D = 5V - 2,2V = 2,8V
% \]

% **Formel zur Berechnung des Vorwiderstandes \(R_V\):**
% Der Widerstand kann mit dem Ohmschen Gesetz berechnet werden:
% \[
% R_V = \frac{U_V}{I_D} = \frac{2,8V}{30mA} = 93,3 \Omega
% \]
% Das bedeutet, dass ein Widerstand von etwa 93 Ohm nötig ist, um den gewünschten Strom von 30 Milliampere durch die LED fließen zu lassen.

% **Berechnung für verschiedene Widerstände:**
% Nun wird überprüft, welche der vorgegebenen Widerstandswerte der optimalen Helligkeit am nächsten kommen.

% 1. **Für einen Widerstand von 150 Ohm:**
%    \[
%    I_D = \frac{2,8V}{150 \Omega} = 18,7mA
%    \]
%    Dies ist ein geringerer Strom als ideal, wodurch die LED weniger hell leuchten würde.

% 2. **Für einen Widerstand von 180 Ohm:**
%    \[
%    I_D = \frac{2,8V}{180 \Omega} = 15,6mA
%    \]
%    Auch hier würde die LED dunkler als optimal leuchten.

% 3. **Für eine Parallelschaltung von 150 Ohm und 180 Ohm:**
%    Der Gesamtwiderstand berechnet sich als:
%    \[
%    R_{gesamt} = \frac{150 \cdot 180}{150 + 180} = 96,8 \Omega
%    \]
%    Damit ergibt sich ein Strom durch die LED von:
%    \[
%    I_D = \frac{2,8V}{96,8 \Omega} = 28,9mA
%    \]
%    Dies ist sehr nahe am optimalen Wert von 30 Milliampere.

% **Fazit:**
% Der Widerstand von **93,3 Ohm** wäre optimal, allerdings ist dieser Wert nicht direkt in der Auswahl enthalten. Die beste praktische Wahl ist die Parallelschaltung von **150 Ohm und 180 Ohm**, da sie einen Widerstand von 96,8 Ohm ergibt und somit einen Stromfluss von 28,9 Milliampere ermöglicht, was der gewünschten LED-Helligkeit am nächsten kommt.
% }

}




\subsection{Identifikation eines Bauteils anhand des Schaltzeichens\label{Aufg11}}
% Alt: Aufgabe 11
\Aufgabe{
  Was für ein Bauelement wird mit folgendem Symbol gekennzeichnet?
  \begin {figure} [H]
     \centering
     \begin{tikzpicture}

   \draw
   node[npn, tr circle] (Q1) {}
   (Q1.base) 
   (Q1.collector)  
   (Q1.emitter)  ;
\end{tikzpicture}
     \label{fig:FigSchaltzeichen}
  \end {figure}
%   \speech{ **Aufgabe 11**  
% **Fragestellung:**  
% Was für ein **Bauelement** wird mit folgendem **Schaltzeichen** gekennzeichnet?

% ---

%  **Beschreibung der Abbildung**  
% Das gezeigte **elektrische Symbol** stellt ein **Halbleiterbauelement** dar, das aus drei Anschlüssen besteht:

% 1. **Kreisförmige Darstellung mit drei Anschlüssen:**  
%    - Symbolisiert ein **transistorbasiertes Halbleiterbauelement**.  
   
% 2. **Drei Anschlüsse:**  
%    - Der **obere Anschluss** entspricht dem **Kollektor (C)**.  
%    - Der **mittlere waagerechte Anschluss** entspricht der **Basis (B)**.  
%    - Der **untere Anschluss mit einem Pfeil** entspricht dem **Emitter (E)**.  
   
% }

}


\Loesung{
  Beim dargestellten Bauelement handelt es sich um einen npn-Transistor.
  Siehe: Abschnitt \ref{fig:UebersichtBipolartransistoren} 

  %     \speech{
  % Lösung zu Aufgabe 11:
  
  % Das dargestellte Bauelement in der Abbildung ist ein **npn-Transistor**.  
  % Ein npn-Transistor besteht aus drei Schichten: einer **n-dotierten** Schicht (Emitter), einer **p-dotierten** Schicht (Basis) und einer weiteren **n-dotierten** Schicht (Kollektor).  
  % Der Stromfluss erfolgt, wenn eine positive Spannung an der Basis anliegt und Elektronen vom Emitter zum Kollektor transportiert werden.  
  % Typischerweise wird ein npn-Transistor für Verstärkerschaltungen und als elektronischer Schalter verwendet.
  % }
  
}




\subsection{Ermittlung von Basis-Emitter-Spannung und Basisstrom\label{Aufg12}}
Gegeben ist die abgebildete Transistorschaltung mit den folgenden Parametern: $U_0=\mathrm{5\,V}$ ,
$U_L=\mathrm{15\,V}$, $R_\mathrm{V}=\mathrm{100\,k\Omega}$ und $R_\mathrm{V}=\mathrm{500\,\Omega}$.
% Alt: Aufgabe 12
\Aufgabe{
  \begin{itemize}
    \item[a)] Wie groß ist die Spannung $U_\mathrm{BE}$ näherungsweise, wenn es sich um einen Siliziumtransistor handelt?
    \item[b)] Wie groß ist der Strom $I_\mathrm{B}$?
  \end{itemize}
  \begin{figure}[H]
    \centering
    \begin{tikzpicture}
     \draw (4,2) node[npn, tr circle] (Q1) {};
 
     \draw (Q1.C) -- (4,4) 
     to [R,l=$R_\mathrm{L}$] (7,4)
     to [V, name=Ul, v] (7,0);

     \draw (Q1.E) to [short, -*] (4,0) 
     -- (7,0);

     \draw (Q1.B) to [R,l_=$R_\mathrm{V}$, i<, name=Ib] (0,2)
     to [V, name=U0, v] (0,0)
     to [short] (4,0);

     \draw (Q1.B) ++ (-0.25,0) coordinate(Tb);
     \draw (Q1.E) ++ (0,-0.25) coordinate(Te);
     \draw (Tb) to [open, name=ube, v, voltage=european] (Te);

     \varrmore{ube}{$U_\mathrm{BE}$};
     \varrmore{U0}{$U_\mathrm{0}$};
     \varrmore{Ul}{$U_\mathrm{L}$};
     \iarrmore{Ib}{$I_\mathrm{B}$};
 \end{tikzpicture}
    \label{fig:FigTransistorBasisstrom}
  \end{figure}  
}

\Loesung{
  \begin{itemize}
    \item[a)]
  \end{itemize}
  \begin{equation}
    U_\text{BE} = U_\text{B} - U_\text{E}
  \end{equation}
  Ein npn-Transistor hat im leitenden Zustand eine typische Basis-Emitter-Spannung von:
  \begin{equation}
    U_\text{BE} \approx 0,6\,\text{V} \text{ bis } 0,7\,\text{V}
  \end{equation}
  Diese Spannung ist eine materialbedingte Eigenschaft des Silizium-Transistors und bleibt nahezu konstant, unabhängig von der Basisspannung \( U_\text{B} \).
  - Obwohl \( U_\text{B} = 5\,\text{V} \) ist, begrenzt der Transistor die Spannung \( U_{BE} \) auf ca. \( 0,6\,\text{V} \), sobald der Transistor leitend ist.  
  - Der überschüssige Spannungsabfall (\( U_B - U_{BE} \)) wird durch andere Ströme und Bauteile der Schaltung ausgeglichen.
  \begin{align*}
    {U_{BE} \approx 0,6\,\text{V}}
  \end{align*}
  \begin{itemize}
    \item[b)] $I_\text{B} = \frac{5\,\text{V} - 0,6\,\text{V}}{100\,\text{k}\Omega} = 44\,\mu\text{A}$
  \end{itemize}
  Siehe: Abschnitt \ref{sec:Bipolartransistor} und \ref{fig:StroemeUndSpannungenBeimBipolartransistor}
}

\subsection{Berechnung der Widerstände einer Transistorschaltung\label{Aufg13}}
% Alt: Aufgabe 13
\Aufgabe{
  Gegeben ist die abgebildete Transistorschaltung mit den folgenden Parametern: $U_0=\mathrm{5\,V}$ , 
  $U_\mathrm{R1}=\mathrm{0,6\,V}$, $U_\mathrm{L}=\mathrm{15\,V}$, $R_\mathrm{1}=\mathrm{300\,\Omega}$, $R_\mathrm{2}=\mathrm{2\,k\Omega}$ und $R_\mathrm{V}=\mathrm{500\,\Omega}$.
  Damit der gezeichnete Siliziumtransistor schaltet, braucht er einen Basisstrom von $50 \, \mathrm{\mu A}$ um 
  durchzuschalten. 
  
  \begin{figure}[H]
    \centering
    \begin{tikzpicture}
   \draw (4,2) node[npn, tr circle] (Q1) {};

   \draw (Q1.C) -- (4,4) 
   to [R,l=$R_\mathrm{L}$] (7,4)
   to [V, name=Ul, v] (7,0);

   \draw (Q1.E) to [short, -*, i, name=Ie] (4,0) 
   -- (7,0);

  \draw (Q1.B) to [R,l_=$R_\mathrm{V}$, i<, name=Ib] (0,2)
  to [R,l=$R_\mathrm{2}$] (0,0)
  to [short] (4,0);

  \draw (0,2) to [R,l_=$R_\mathrm{1}$, *-, v^<, name=Ur1] (0,4)
  -- (-2,4)
  to [V, name=U0, v] (-2,0)
  to [short, -*] (0,0) ;

   \varrmore{U0}{$U_\mathrm{0}$};
   \varrmore{Ur1}{$U_\mathrm{R1}$};
   \varrmore{Ul}{$U_\mathrm{L}$};
   \iarrmore{Ib}{$I_\mathrm{B}$};
   \iarrmore{Ie}{$I_\mathrm{E}$};
\end{tikzpicture}
    \label{fig:FigTransistorWiderstaende}
  \end{figure}
  
  \begin{itemize}
    \item[a)] Wie groß muss der Widerstand $R_\mathrm{V}$ gewählt werden, damit der Transistor durchschaltet?
    \item[b)] Angenommen der Spannungsabfall des Transistors $U_\mathrm{CE}$ beträgt $7,5 \, \mathrm{V}$, wie groß ist der Strom
          durch $I_\mathrm{E}$?
  \end{itemize}
  
  
  % \speech{Aufgabe 13  
  
  % Damit der gezeichnete Siliziumtransistor schaltet, braucht er einen Basisstrom von \( 50 \mu A \) um durchzuschalten.  
  
  % a) Wie groß muss der Widerstand \( R_V \) gewählt werden, damit der Transistor durchschaltet?  
  
  % b) Angenommen der Spannungsabfall des Transistors \( U_{CE} \) beträgt \( 7,5V \), wie groß ist der Strom durch \( I_E \)?  
  
  % Beschreibung der Abbildung  
  
  % Die Abbildung zeigt eine **Transistorschaltung** mit einer **Spannungsquelle von 5V** auf der linken Seite, die über einen **300Ω-Widerstand** mit einem **weiteren 300Ω-Widerstand (\( R_V \))** verbunden ist. Parallel zu \( R_V \) ist ein **2kΩ-Widerstand** geschaltet. Diese Schaltung speist den **Basisstrom \( I_B \)** in die Basis eines **npn-Transistors**.  
  
  % Die Basis-Emitter-Spannung \( U_{BE} \) ist mit einem blauen Pfeil markiert. Der **Basisstrom \( I_B \)** ist mit einem roten Pfeil gekennzeichnet.  
  
  % Auf der rechten Seite befindet sich eine **zweite Spannungsquelle mit 15V**, die über einen **500Ω-Kollektorwiderstand** mit dem **Kollektor des Transistors** verbunden ist. Der **Emitterstrom \( I_E \)** ist mit einem roten Pfeil dargestellt.  
  
  % Die Aufgabe besteht darin, \( R_V \) zu berechnen, um den erforderlichen Basisstrom \( I_B \) sicherzustellen, sowie den Emitterstrom \( I_E \) zu bestimmen, wenn der Transistor eine Kollektor-Emitter-Spannung \( U_{CE} = 7,5V \) aufweist.  
  % }  
  
}


\Loesung{
  \begin{itemize}
    \item[a)]
          \[
            \frac{5\,\text{V}}{2\,\text{k}\Omega + 0,3\,\text{k}\Omega} = \frac{U_1}{300\,\Omega}
            \Rightarrow U_\text{1} = 300\,\Omega \cdot \frac{5\,\text{V}}{2\,\text{k}\Omega + 0,3\,\text{k}\Omega} = 0,65\,\text{V}
          \]
          \[
            5\,\text{V} = 0,65\, \text{V} + U_\text{RV} + 0,6\,\text{V}
          \]
          \[
            U_\text{RV} = 5\,\text{V} - 0,65\,\text{V} - 0,6\,\text{V} = 3,75\,\text{V}
          \]
          \[
            R_\text{V} = \frac{3,75\,\text{V}}{50\,\mu \text{A}} = 75\,\text{k}\Omega
          \]
          
          
    \item[b)]
          \[
            I_\text{C} = \frac{15\,\text{V} - 7,5\,\text{V}}{500\,\Omega} = 15\,\text{mA}
          \]
          \[
            I_\text{E} = I_\text{C} + I_\text{B} = 15\,\text{mA} + 50\,\mu \text{A} = 15,05\,\text{mA}
          \]
          \[
            \frac{I_\text{C}}{I_\text{B}} = \beta = 300
          \]
  \end{itemize}
  Siehe: Abschnitt \ref{sec:Bipolartransistor} und \ref{fig:StroemeUndSpannungenBeimBipolartransistor}
}
\subsection{Schaltverhalten eines Transistors\label{Aufg14}}
% Alt: Aufgabe 14
\Aufgabe{
    Gegeben ist die abgebildete Transistorschaltung mit den folgenden Parametern: 
    $U_0=\mathrm{5\,V}$ und $U_L=\mathrm{15\,V}$.
  Damit der gezeigte Siliziumtransistor schaltet, benötigt er einen Basisstrom von \( I_\mathrm{B}=\mathrm{50\,\mu A} \), um in den leitenden Zustand zu gelangen.
  
  \begin{itemize}
    \item[a)] Wie groß muss der Vorwiderstand \( R_\mathrm{V} \) gewählt werden, damit der Transistor durchschaltet?
    \item[b)] Der Transistor hat einen Stromverstärkungsfaktor \( \beta = 180 \) und einen Kollektorwiderstand \( R_\mathrm{C} = \mathrm{200\,\Omega} \). 
    Wie groß sind der Kollektorstrom \( I_\mathrm{C} \) und der Spannungsabfall zwischen Kollektor und Emitter \( U_\mathrm{CE} \)?
  \end{itemize}  
  \begin {figure} [H]
     \centering
     \begin{tikzpicture}
    \draw (4,2) node[npn, tr circle] (Q1) {};

    \draw (Q1.C) -- (4,4) 
    to [R,l=$R_\mathrm{C}$] (7,4)
    to [V, name=Ul, v] (7,0);

    \draw (Q1.E) to [short, -*] (4,0) 
    -- (7,0);

   \draw (Q1.B) to [R,l_=$R_\mathrm{V}$, i<, name=Ib] (0,2)
   to [V, name=U0, v] (0,0)
   to [short] (4,0);

    \varrmore{U0}{$U_\mathrm{0}$};
    \varrmore{Ul}{$U_\mathrm{L}$};
    \iarrmore{Ib}{$I_\mathrm{B}$};
\end{tikzpicture}
     \label{fig:FigTransistorSchaltverhalten}
  \end {figure}
  % \speech{Aufgabe 14  

  % Damit der gezeichnete Siliziumtransistor schaltet, braucht er einen Basisstrom von \( 50 \mu A \) um durchzuschalten.  
  
  % a) Wie groß muss der Widerstand \( R_V \) gewählt werden, damit der Transistor durchschaltet?  
  
  % b) Der Transistor hat eine Stromverstärkungsfaktor \( \beta = 180 \) und einen Kollektorwiderstand \( R_C \) von \( 200 \Omega \).  
  %    Wie groß ist der Kollektorstrom \( I_C \) und der Spannungsabfall zwischen Kollektor-Emitter \( U_{CE} \)?  
  
  % Beschreibung der Abbildung  
  
  % Die Abbildung zeigt eine Transistorschaltung mit einer **Spannungsquelle von \( 5V \)** auf der linken Seite, die über einen **Widerstand \( R_V \)** mit der Basis eines **npn-Transistors** verbunden ist.  
  
  % Die Basis-Emitter-Spannung \( U_{BE} \) ist mit einem blauen Pfeil gekennzeichnet, und der **Basisstrom \( I_B \)** ist mit einem roten Pfeil dargestellt.  
  
  % Auf der rechten Seite der Schaltung befindet sich eine **weitere Spannungsquelle mit \( 15V \)**, die über einen **Kollektorwiderstand \( R_C \)** mit dem **Kollektor des Transistors** verbunden ist. Der **Kollektorstrom \( I_C \)** fließt durch den Widerstand \( R_C \) und ist mit einem roten Pfeil markiert.  
  
  % Die Aufgabe besteht darin, den notwendigen Basiswiderstand \( R_V \) zu berechnen und aus der gegebenen Stromverstärkung \( \beta \) den **Kollektorstrom \( I_C \)** sowie den Spannungsabfall \( U_{CE} \) zu bestimmen.  
  % }  
  

  }


\Loesung{
\begin{itemize}
    \item[a)]
    Berechnung des Vorwiderstands für den Basisstrom:
    \begin{equation}
        R_\mathrm{V} = \frac{U}{I}
    \end{equation}
    \begin{align*}
        R_\mathrm{V} &= \frac{\mathrm{5\,V} - \mathrm{0{,}6\,V}}{\mathrm{50\,\mu A}} = \frac{\mathrm{4{,}4\,V}}{\mathrm{50\,\mu A}} = \mathrm{88\,k\Omega}
    \end{align*}

    \item[b)]
    Berechnung des Kollektorstroms mit Stromverstärkungsfaktor:
    \begin{equation}
        \beta = \frac{I_\mathrm{C}}{I_\mathrm{B}}
    \end{equation}
    \begin{align*}
        I_\mathrm{C} &= \beta \cdot I_\mathrm{B} = 180 \cdot \mathrm{50\,\mu A} = \mathrm{9\,mA}
    \end{align*}

    Berechnung des Spannungsabfalls am Kollektorwiderstand:
    \begin{align*}
        U_\mathrm{RC} &= I_\mathrm{C} \cdot R_\mathrm{C} = \mathrm{9\,mA} \cdot \mathrm{200\,\Omega} = \mathrm{1{,}8\,V}
    \end{align*}

    Berechnung der Spannung zwischen Kollektor und Emitter:
    \begin{align*}
        U_\mathrm{CE} &= \mathrm{15\,V} - \mathrm{1{,}8\,V} = \mathrm{13{,}2\,V}
    \end{align*}
\end{itemize}

    Siehe: Abschnitt \ref{sec:Bipolartransistor} und \ref{fig:StroemeUndSpannungenBeimBipolartransistor}
%     \speech{
% **Aufgabe 14 – Berechnung des Basiswiderstands und der Verstärkung eines npn-Transistors**

% Gegeben ist eine Schaltung mit einem **npn-Transistor**, der durch einen **Basiswiderstand** gesteuert wird. Die Aufgabe besteht darin, den **Basiswiderstand \( R_V \)** sowie die **Stromverstärkung \( \beta \)** und die zugehörigen Spannungen zu berechnen.

% ---

% ### Teil a) Berechnung des Widerstands \( R_V \)
% Die Formel für den Widerstand lautet:

% \[
% R_V = \frac{U}{I}
% \]

% Gegeben sind:
% - Eingangsspannung \( U = 5V \)
% - Basis-Emitter-Spannung \( U_{BE} = 0,6V \)
% - Basisstrom \( I_B = 50 \mu A \)

% Einsetzen der Werte:

% \[
% R_V = \frac{5V - 0,6V}{0,05 mA}
% \]

% \[
% R_V = \frac{4,4V}{0,05 mA} = 88 kΩ
% \]

% **Ergebnis**: Der Basiswiderstand \( R_V \) beträgt **88 kΩ**.

% ---

% ### Teil b) Berechnung der Stromverstärkung \( \beta \) und der zugehörigen Spannungen
% Die Formel für den Verstärkungsfaktor \( \beta \) lautet:

% \[
% \beta = \frac{I_C}{I_B}
% \]

% Gegeben:
% - Basisstrom \( I_B = 50 \mu A \)
% - Stromverstärkungsfaktor \( \beta = 180 \)

% Berechnung des Kollektorstroms \( I_C \):

% \[
% I_C = \beta \cdot I_B
% \]

% \[
% I_C = 180 \cdot 50 \mu A = 9 mA
% \]

% Nun wird die Spannung am **Kollektorwiderstand \( R_C \)** berechnet:

% \[
% U_{RC} = I_C \cdot R_C
% \]

% Gegeben:
% - \( R_C = 200Ω \)
% - \( I_C = 9 mA \)

% \[
% U_{RC} = 9 mA \cdot 200Ω = 1,8V
% \]

% Die **Kollektor-Emitter-Spannung \( U_{CE} \)** ergibt sich aus:

% \[
% U_{CE} = 15V - U_{RC}
% \]

% \[
% U_{CE} = 15V - 1,8V = 13,2V
% \]

% ---

% **Ergebnisse**:
% - Der Kollektorstrom \( I_C \) beträgt **9 mA**.
% - Die Spannung über dem Kollektorwiderstand \( U_{RC} \) beträgt **1,8V**.
% - Die Kollektor-Emitter-Spannung \( U_{CE} \) beträgt **13,2V**.
% }

}




\subsection{Transistorschaltung mit LED\label{Aufg15}}
% Alt: Aufgabe 15
\Aufgabe{
  Mithilfe eines Bipolartransistor soll eine LED betrieben werden. Dabei muss durch die LED ein Strom von $I_\mathrm{D}=30\,\mathrm{mA}$ fließen, damit sie ihre optimale Leuchtkraft erreicht. \\
  Der verwendete Transistor besitzt einen Stromverstärkungsfaktor von $\beta = 50$ und im durchgeschalteten Zustand eine Kollektor-Emitter-Spannung von $U_\mathrm{CE} = 0,3\,\mathrm{V}$. 
  Zusätzlich sind die folgenden Parameter gegeben: $R_1=3,3\,\mathrm{k\Omega}$,  $R_2=270\,\mathrm{\Omega}$ und $U_\mathrm{L}=9\,\mathrm{V}$. \\
  Welche Spannung $U$ muss an der Basis anliegen, damit der Transistor durchschaltet? 
  \begin {figure} [H]
     \centering
     \begin{tikzpicture}
    % Spannungspfeile in blau
    \draw(8,3) to[open, v>, name=9V] (8,0);
    % Batterie links
    \draw
    (0,1.5) to[V, name=U] (0,0);
    % Widerstand RV
    \draw
    (0,1.5) -- (2,1.5)
    to[R=$3.3\,k\Omega$] (3.5,1.5) to[short,i,name=IB] (4.5,1.5)
    -- (5,1.5) ; % Verbindung mit Basis
    \draw(4.7,1.2) to[open, v>, name=ube] (5.2,0.7);
    % Transistor
    \draw
   (5.5,1.5) node[npn, tr circle] (Q1) {}
   (Q1.base) 
   (Q1.collector)  
   (Q1.emitter)  ;
   
   \draw
   (Q1.C) -- (5.5,3)
   to[R=$270\,\Omega$] (7,3);



   \draw (Q1.E) -- (5.5,0) to[short, -o] (8,0);


   \draw (0,0) -- (5.5,0); 

   \draw (8,3) to[leDo, o-] (7,3);

  \varrmore{U}{$U$};
   \varrmore{9V}{$\mathrm{9\,V}$};
   \iarrmore{IB}{$I_\mathrm{B}$};
   \varrmore{ube}{$U_\mathrm{BE}$};
  \end{tikzpicture}
     \label{fig:FigTransistorLED}
  \end {figure}
}


\Loesung{
Berechnung des Kollektorstroms bei bekannter Spannung und Lastwiderstand:
\begin{equation}
    I_\mathrm{C} = \frac{U}{R}
\end{equation}
\begin{align*}
    I_\mathrm{C} &= \frac{9\,\mathrm{V} - 0,3\,\mathrm{V}}{270\,\Omega} = 32,2\,\mathrm{mA}
\end{align*}

Berechnung des benötigten Basisstroms mit dem Stromverstärkungsfaktor $\beta = 50$:
\begin{equation}
    I_\mathrm{B} = \frac{I_\mathrm{C}}{\beta}
\end{equation}
\begin{align*}
    I_\mathrm{B} &= \frac{32,2\,\mathrm{mA}}{50} = 0,644\,\mathrm{mA}
\end{align*}

Spannungsabfall am Vorwiderstand:
\begin{align*}
    U_\mathrm{V} &= 0,644\,\mathrm{mA} \cdot 3,3\,\mathrm{k\Omega} = 2,12\,\mathrm{V}
\end{align*}

Gesamte Basisspannung (inkl. \( U_\mathrm{BE} \)):
\begin{align*}
    U &= U_\mathrm{V} + U_\mathrm{BE} = 2,12\,\mathrm{V} + 0,6\,\mathrm{V} = 2,72\,\mathrm{V}
\end{align*}

\textbf{Antwort:}  
Die Basisspannung muss \( \boxed{2{,}72\,\mathrm{V}} \) betragen, damit der Transistor durchschaltet.

    Siehe: Abschnitt \ref{sec:Bipolartransistor} und \ref{fig:StroemeUndSpannungenBeimBipolartransistor}
}
\subsection{Analyse des Arbeitsbereichs eines Transistors\label{Aufg16}}
% Alt: Aufgabe 16
\Aufgabe{
  Der Stromverstärkungsfaktor des verwendeten Siliziumtransistors beträgt $ \beta = 150$ und die Versorgungsspannung liegt bei $U_\mathrm{V} = 15\,\mathrm{V}$. Die Schaltung soll so dimensioniert werden, dass der Kollektorstrom $I_\mathrm{C} = 2\,\mathrm{mA}$ beträgt.
  
  \begin{itemize}
    \item[a)] Dimensionieren Sie den Widerstand $R_3$ so, dass bei einer Spannung am Ausgang von $3\,\mathrm{V}$ der Strom durch $R_3$ genau $75\,\%$ des Emitterstroms beträgt.
    
    \item[b)] Bestimmen Sie die Widerstände $R_1$ und $R_2$, unter der Annahme, dass die Kollektor-Basis-Spannung $U_\mathrm{BE} = 0,6\,\mathrm{V}$ beträgt und durch $R_2$ etwa das 9-fache des Basisstroms $I_\mathrm{B}$ fließt.
  \end{itemize}
  \begin {figure} [H]
   \centering
   \begin{tikzpicture}
    \draw (0,0) node[npn, tr circle] (Q1) {};
    
    \draw (Q1.C) to[short, -, i<, name=ic] (0,2)
    to[short, -] ++ (-3, 0) 
    to[R=$R_1$]  (-3, 0) 
    to[R=$R_2$]  (-3, -3) 
    node[ground]{};
    
    \draw (-3,2) to[short, *-o] ++(-1,0)
    to [open, v, name=uv] (-4,-3.5); 
    
    \draw (Q1.B) to[short, -*] (-3, 0);
    
    \draw (Q1.E) to[short, -*] ++(0, -0.5) coordinate(E)
    to[R=$R_3$] (0, -3) node[ground]{};
    
    \draw (E) to[short, -] ++(2, 0)
    to[R=$R_\mathrm{L}$] (2, -3) node[ground]{};
    
    \draw (E)++(2, 0) to[short, *-o] ++(1, 0)
    to[open, v^, name=ua] (3,-3.5)    ;
    
    \varrmore{ua}{$U_\mathrm{A}$};
    \varrmore{uv}{$U_\mathrm{B}$};
    \iarrmore{ic}{$I_\mathrm{C}$};
\end{tikzpicture}
   \label{fig:FigTransistorArbeitsbereich}
\end {figure}

% \speech{Aufgabe 16  

% Der Stromverstärkungsfaktor des Siliziumtransistors beträgt **150** und die Versorgungsspannung \( U_V = 15V \).  
% Die Schaltung soll so dimensioniert werden, dass der Kollektorstrom **2 mA** beträgt.  

% **Aufgabenstellung:**  
% a) Dimensionieren Sie \( R_3 \), sodass bei einer Spannung am Ausgang von **3V** der Strom durch \( R_3 \) **75%** des Emitterstroms beträgt.  

% b) Bestimmen Sie \( R_1 \) und \( R_2 \), nehmen Sie dazu an, dass die **Kollektor-Basis-Spannung** **0,6V** beträgt und durch \( R_2 \) etwa **9 \cdot I_B** fließen.  

% **Beschreibung der Abbildung**  

% Die Abbildung zeigt eine **Transistorverstärkerschaltung** mit einem **npn-Transistor**, mehreren Widerständen und einer **Versorgungsspannung von 15V**.  

% - **Oben:** Die **Versorgungsspannung \( U_V = 15V \)** ist an den Kollektorwiderstand \( R_1 \) angeschlossen.  
% - **Transistor:** Der **Kollektor** ist über \( R_1 \) mit \( U_V \) verbunden, während der **Emitter** über \( R_3 \) und den Lastwiderstand \( R_L \) mit Masse verbunden ist.  
% - **Basisbeschaltung:** Zwei Widerstände \( R_1 \) und \( R_2 \) sorgen für eine stabile **Basisspannung** zur Steuerung des Transistors.  
% - **Lastwiderstand:** \( R_L \) ist parallel zum Ausgang \( u_A \) geschaltet, wodurch das Signal abgegriffen wird.  

% Die Aufgabe besteht darin, die Widerstände **\( R_3, R_1 \) und \( R_2 \)** so zu dimensionieren, dass die gegebenen Bedingungen erfüllt werden.  
% }  

}


\Loesung{
\begin{itemize}
    \item[a)] Berechnung des Basisstroms über den Stromverstärkungsfaktor:
\begin{equation}
    I_\mathrm{B} = \frac{I_\mathrm{C}}{\beta}
\end{equation}
\begin{align*}
    I_\mathrm{B} &= \frac{\mathrm{2\,mA}}{150} = \mathrm{13{,}3\,\mu A} \\
    I_\mathrm{E} &= I_\mathrm{C} + I_\mathrm{B} = \mathrm{2\,mA} + \mathrm{13{,}3\,\mu A} = \mathrm{2{,}013\,mA} \\
    I_\mathrm{R_3} &= 0{,}75 \cdot I_\mathrm{E} = 0{,}75 \cdot \mathrm{2{,}013\,mA} = \mathrm{1{,}51\,mA} \\
    R_3 &= \frac{\mathrm{3\,V}}{\mathrm{1{,}51\,mA}} = \mathrm{1{,}99\,k\Omega}
\end{align*}
    \item[b)] Berechnung des Stroms durch \( R_2 \):
\begin{align*}
    I_{R_2} &= 9 \cdot I_\mathrm{B} = 9 \cdot \mathrm{13{,}3\,\mu A} = \mathrm{0{,}12\,mA} \\
    R_2 &= \frac{U_\mathrm{a} + U_\mathrm{BE}}{I_\mathrm{R_2}} = \frac{\mathrm{3\,V} + \mathrm{0{,}6\,V}}{\mathrm{0{,}12\,mA}} = \mathrm{30\,k\Omega}
\end{align*}

Berechnung des Stroms durch $R_1$, welcher gleich $I_\mathrm{R_2}$ ist:
\begin{align*}
    R_1 &= \frac{U_\mathrm{BE}}{I_\mathrm{R_1}} = \frac{0,6\,\mathrm{V}}{0,12\,\mathrm{mA}} = 5,6\,\mathrm{k\Omega}
\end{align*}
\end{itemize}

  Siehe: Abschnitt \ref{sec:ElektrischesVerhaltenBipolartransistor}
%   \speech{
% Aufgabe 16 behandelt die Berechnung verschiedener Widerstände und Ströme in einer Transistorschaltung mit einem Silizium-Transistor.  

% Teil a)  
% Die Berechnung basiert auf dem Verstärkungsfaktor \(\beta\) und den bekannten Strom- und Spannungswerten.  
% - Der Basisstrom \(I_B\) wird mit der Formel \(I_B = \frac{I_C}{\beta}\) berechnet.  
% - Mit \(I_C = 2\) mA und \(\beta = 150\) ergibt sich \(I_B = 13,3\) µA.  
% - Der Emitterstrom ist die Summe aus Kollektor- und Basisstrom:  
%   \(I_E = I_C + I_B = 2,013\) mA.  
% - Der Strom durch \(R_3\) ist definiert als \(I_{R_3} = 0,75 \times I_E = 1,51\) mA.  
% - Mit \(U = 3\) V folgt für den Widerstand \(R_3 = \frac{3V}{1,51mA} = 1,99\) kΩ.  

% Teil b)  
% Hier werden die Widerstände \(R_1\) und \(R_2\) bestimmt.  
% - Der Strom durch \(R_2\) ergibt sich aus \(I_{R_2} = 9 \times I_B = 0,12\) mA.  
% - Mit \(U_A + U_{BE} = 3V + 0,6V\) berechnet sich \(R_2\) als  
%   \(R_2 = \frac{3V + 0,6V}{0,12mA} = 30\) kΩ.  
% - Für \(R_1\) gilt: \(R_1 = \frac{U_{CE}}{I_{R_1}}\), wobei \(I_{R_1} = 9 \times I_B = 0,12\) mA.  
% - Damit folgt \(R_1 = \frac{0,6V}{9 \times 13,3\) µA} = 5,6 kΩ.  
% }

}




\subsection{Berechnungen in einer Transistorschaltungen\label{Aufg17}}
% Alt: Aufgabe 17
\Aufgabe{
    Gegeben ist die folgende Schaltung mit dem Ausgangskennlinienfeld eines Transistors.
    
    \begin{figure}[H]
        \centering
        \begin{subfigure}[b]{0.45\textwidth}
            \centering
            \begin{tikzpicture} 
    \draw (0,0) node[npn] (Q1) {};
    
    \draw (Q1.C)
    to [short] (0,1)
    to[R, l_= $R_\mathrm{C}$] (0,3)
    to [short, -o] (0,3.5) node[right, color=voltage] {$U_\mathrm{V}$};
    
    \draw (0,1)
    to[short, *-o] (2,1);
    
    \draw (Q1.E)
    to [short, -*] (0,-2) node[ground]{};
    
    \draw (-2,-2)
    to[short, o-o] (2,-2);
    
    \draw (2,1)
    to[open,v^, name=ua] (2,-2);
    
    \draw (Q1.B) to[short, -o] (-2,0)
    to[open,v>, name=ue] (-2,-2);
    
    \varrmore{ua}{$U_\mathrm{CE}$};
    \varrmore{ue}{$U_\mathrm{BE}$};
\end{tikzpicture}
        \end{subfigure}
        \begin{subfigure}[b]{0.45\textwidth}
            \centering
            \includesvg[width=0.9\textwidth]{Bilder/Aufgaben/FigTransistorKennlinien}
        \end{subfigure}
        \label{fig:FigTransistorKennlinien}
    \end{figure}
    
    \begin{itemize}
        \item[a)] Wie groß ist der maximale Kollektorstrom des idealen Transistors (bei $R_\mathrm{CE} = 0\,\Omega$), wenn der Kollektorwiderstand $R_\mathrm{C} = 7,5\,\Omega$ beträgt und die Versorgungsspannung $U_\mathrm{V} = 3\,\mathrm{V}$ ist?
        \item[b)] Bestimmen Sie  $U_\mathrm{CE}$, $U_\mathrm{RC}$ und $I_\mathrm{C}$ des realen Transistors bei einem Basisstrom von \( I_\mathrm{B} = \mathrm{2\,mA} \). Tragen Sie dazu die Widerstandsgerade in das Kennlinienfeld ein.
        \item[c)] Für einen neuen Arbeitspunkt fallen nur noch $1,7\,\mathrm{V}$ über dem Kollektorwiderstand $R_\mathrm{C}$ ab, der Basisstrom beträgt weiterhin $2\,\mathrm{mA}$. Bestimmen Sie den neuen Widerstand $R_\mathrm{C}$.
    \end{itemize}
}


\Loesung{
    \begin{itemize}
        \item[a)] Berechnung des maximalen Kollektorstroms bei idealem Transistor (\( R_\mathrm{CE} = 0 \)):
              \begin{equation*}
                  I_\mathrm{C}  = \frac{U_\mathrm{V}}{R_\mathrm{C}} = \frac{3\,\mathrm{V}}{7,5\,\Omega} = 400\,\mathrm{mA}
              \end{equation*}
              
              
        \item[b)] Bestimmung der Spannungen und des Kollektorstroms laut Kennlinienfeld:
              \begin{figure}[H]
                  \centering
                  \includesvg[width=0.7\textwidth]{Bilder/Aufgaben/LsgTransistorKennlinien}
                  \caption{Kennlinienfeld mit eingezeichneter Widerstandsgerade}
              \end{figure}
              \begin{gather*}
                  U_\mathrm{CE} = 1,2\,\mathrm{V} \\
                  U_\mathrm{RC} = 1,8\,\mathrm{V} \\
                  I_\mathrm{C} = \frac{U_\mathrm{RC}}{R_\mathrm{C}} = \frac{1,8\,\mathrm{V}}{7,5\,\Omega} = 240\,\mathrm{mA}
              \end{gather*}
              
        \item[c)] Berechnung des neuen Widerstandswerts bei gleichem $I_\mathrm{C}$ und neuer Spannung $U_\mathrm{RC} = 1,7\,\mathrm{V}$:
              \begin{equation*}
                  R_\mathrm{C} = \frac{U_\mathrm{RC}}{I_\mathrm{C}} = \frac{1,7\,\mathrm{V}}{240\,\mathrm{mA}} = 7,08\,\Omega
              \end{equation*}
    \end{itemize}
    
    Siehe: Abschnitt \ref{sec:ElektrischesVerhaltenBipolartransistor}
}
\subsection{Berechnung der Stromverstärkung einer Transistorschaltung\label{Aufg18}}
% Alt: Aufgabe 18
\Aufgabe{
    Gegeben ist die folgende Schaltung mit dem Kennlinienfeld des Transistors. Die Versorgungsspannung beträgt \( U_\mathrm{V} = \mathrm{6\,V} \). \\
    Der Arbeitspunkt liegt bei der halben Betriebsspannung.\\
    Bestimmen Sie die folgenden Werte aus dem Kennlinienfeld:
    
    \begin{itemize}
        \item Die Paare \( I_\mathrm{B} / U_\mathrm{BE} \) und \( I_\mathrm{C} / U_\mathrm{CE} \) im Arbeitspunkt,
        \item sowie die Stromverstärkung \( \beta \).
    \end{itemize}
    
    \begin{figure}[H]
        % oben
        \begin{minipage}[c]{\textwidth}
            \centering
            \begin{tikzpicture}
\draw
    (0,0) node[npn, tr circle] (Q1) {}
    (Q1.base) 
    (Q1.collector)  
    (Q1.emitter);

\draw (Q1.base) to[short, -o] (-3,0);

\draw (Q1.emitter) to[short, -*] (0,-3)node[ground]{};

\draw (Q1.collector) to[short, -*] (0,1) to[short, R, l=$R_\mathrm{C}$, -*] (0,3);

\draw (-3,3) -- (3,3) node[above]{$U_\mathrm{CC}$};

\draw (0,1) to[short, -o] (3,1);
\draw (-3,-3) to[short, o-o] (3,-3);

\draw (-3,0) to[open,v>, name=ue] (-3,-3);
\draw (3,1) to[open, v>, name=ua] (3,-3);

\varrmore{ue}{$U_\mathrm{E}$};
\varrmore{ua}{$U_\mathrm{A}$};

\draw (-2,3) to[R,l=$R_{2}$, *-*] (-2,0) to[R,l=$R_{1}$, -*] (-2,-3);
\end{tikzpicture}
        \end{minipage}
        \vspace{0.5cm} % Vertikaler Abstand zur zweiten Reihe
        
        % unten
        \begin{minipage}[c]{0.48\textwidth}
            \centering
            \includesvg[width=\textwidth]{Bilder/Aufgaben/FigTransistorStromverstaerkung2}
        \end{minipage}
        \hfill 
        \begin{minipage}[c]{0.48\textwidth}
            \centering
            \includesvg[width=\textwidth]{Bilder/Aufgaben/FigTransistorStromverstaerkung3}
        \end{minipage}
        \label{fig:FigTransistorStromverstaerkung}
    \end{figure}
    
}


\Loesung{
    \begin{figure}[H]
        \centering
        \begin{subfigure}[b]{0.45\textwidth}
            \centering
            \includesvg[width=0.9\textwidth]{Bilder/Aufgaben/LsgTransistorStromverstaerkung1}
            \label{fig:LsgTransistorStromverstaerkung1}
        \end{subfigure}
        \begin{subfigure}[b]{0.45\textwidth}
            \centering
            \includesvg[width=0.9\textwidth]{Bilder/Aufgaben/LsgTransistorStromverstaerkung2}
            \label{fig:LsgTransistorStromverstaerkung2}
        \end{subfigure}
    \end{figure}
    Gegeben sind folgende Werte im Arbeitspunkt:
    \begin{align*}
        U_\mathrm{CC} & = \mathrm{6\,V}      \\
        U_\mathrm{CE} & = \mathrm{3\,V}      \\
        I_\mathrm{C}  & = \mathrm{6\,mA}     \\
        U_\mathrm{BE} & = \mathrm{0{,}62\,V} \\
        I_\mathrm{B}  & = \mathrm{20\,\mu A}
    \end{align*}
    
    Berechnung der Stromverstärkung \( \beta \):
    \begin{align*}
        \beta & = \frac{I_\mathrm{C}}{I_\mathrm{B}} = \frac{\mathrm{6\,mA}}{\mathrm{20\,\mu A}} = \mathrm{300}
    \end{align*}
    
    Siehe: Abschnitt \ref{sec:ElektrischesVerhaltenBipolartransistor}
}
\subsection{Verhalten eines Transistors bei Wechselspannung\label{Aufg19}}
% Alt: Aufgabe 19
\Aufgabe{
    Was passiert wenn an einen Transistor eine reine Wechselspannung 
    angelegt wird?
    \begin {figure} [H]
       \centering
         \begin{tikzpicture}
    \draw (4,2) node[npn, tr circle] (Q1) {};

    \draw (Q1.E) 
    to [short, -*] (4,0) 
    to[short, -] (8,0)
    to[V, name=Ul, v_<] (8,4)
    to[R,l_=$R_\mathrm{2}$] (4,4) 
    -- (Q1.C);

    \draw (Q1.B) 
    to [R,l_=$R_\mathrm{1}$, i<, name=Ib] (0,2)
    to [sV, name=U0, v_] (0,0)
    to [short] (4,0);

     \draw (Q1.C)++(0, 0.25) 
    to[short, *-o] ++ (1, 0)
    to[open, v^, name=ua, -o] (5,0)
    ;

    \varrmore{U0}{$u_\mathrm{E}$};
    \varrmore{Ul}{$U_\mathrm{L}$};
    \varrmore{ua}{$u_\mathrm{A}$};
    \iarrmore{Ib}{$I_\mathrm{B}$};
\end{tikzpicture}
       \label{fig:FigTransistorWechselspannung}
    \end {figure}
%     \speech{
% **Aufgabe 19**

%  **Aufgabenstellung:**
% Die Aufgabe untersucht das Verhalten eines Transistors, wenn eine **reine Wechselspannung** an seine Basis angelegt wird.

% **Fragestellung:**  
% Was passiert, wenn an einen Transistor eine reine Wechselspannung angelegt wird?

%  **Beschreibung der Abbildung:**

% Die gezeigte Schaltung stellt einen **NPN-Transistor in Emitterschaltung** dar.

% - **Eingangssignal $u_E$**:  
%   - Es handelt sich um eine **Wechselspannungsquelle**, die mit einem **100 k\(\Omega\) Widerstand** in Reihe zur **Basis des Transistors** geschaltet ist.
%   - Der Basisstrom $I_B$ ist durch einen roten Pfeil gekennzeichnet und zeigt in Richtung der Basis.

% - **Kollektorkreis:**
%   - Der **Kollektor ist über einen 500 \(\Omega\) Widerstand** mit einer **15V Gleichspannungsquelle** verbunden.
%   - Ein blauer Pfeil zeigt die **Spannung am Punkt A**, die den Kollektor-Emitter-Spannungsabfall symbolisiert.
%   - Der **Emitter ist direkt mit Masse verbunden**.

%  **Zusätzliche Hinweise für eine blinde Person:**
% - Die **Basis erhält eine Wechselspannung**, sodass sich der Basisstrom \(I_B\) periodisch ändert.
% - Durch die Stromverstärkung des Transistors beeinflusst dies **den Kollektorstrom und damit die Ausgangsspannung an Punkt A**.
% - Die **Wechselspannung an der Basis führt zu einer verstärkten Wechselstromkomponente am Kollektor**.
% - Diese Schaltung könnte zur **Wechselspannungsverstärkung** genutzt werden, wobei die Kollektorspannung eine invertierte Version der Eingangsspannung darstellt.
% }

}



\Loesung{
Wenn an einen Transistor eine reine Wechselspannung angelegt wird, tritt kein kontinuierlicher Verstärkungsbetrieb auf.  

Das führt zu folgenden Effekten:
\begin{itemize}
    \item Verzerrungen im Ausgangssignal, da nur die positiven Halbwellen verstärkt werden.
    \item Die negative Halbwelle wird abgeschnitten, was einer Gleichrichtung des Signals ähnelt.
    \item Es erfolgt keine lineare Verstärkung, da der Transistor zwischen Sperr- und Sättigungszustand wechselt.
\end{itemize}
Siehe: Abschnitt \ref{merk:MerksatzWechselspannung}
% \speech{
% **Lösung zu Aufgabe 19**  

% Wenn an einen Transistor eine reine Wechselspannung angelegt wird, tritt kein kontinuierlicher Verstärkungsbetrieb auf.  
% Das bedeutet, dass die Signalverstärkung nicht gleichmäßig erfolgt und es zu unerwünschten Effekten kommt.  

% **Folgende Effekte treten auf:**  

% - **Verzerrungen im Ausgangssignal:**  
%   Da nur die positiven Halbwellen verstärkt werden, während die negativen nicht beeinflusst oder unterdrückt werden,  
%   entsteht eine ungleichmäßige Signalform.  

% - **Die negative Halbwelle wird abgeschnitten:**  
%   Dies ähnelt einer Gleichrichtung des Signals, da der Transistor während der negativen Halbwelle nicht leitet  
%   und kein Verstärkungseffekt auftritt.  

% - **Keine lineare Verstärkung:**  
%   Da der Transistor zwischen Sperr- und Sättigungszustand wechselt, erfolgt keine gleichmäßige Verstärkung über  
%   den gesamten Signalverlauf.  
%   Der Verstärkungsprozess ist also nicht linear, was bedeutet, dass das Ausgangssignal nicht proportional  
%   zum Eingangssignal verstärkt wird.  

% Daraus lässt sich schließen, dass für einen kontinuierlichen Verstärkungsbetrieb eine Gleichspannungskomponente  
% als Arbeitspunkt erforderlich ist, sodass der Transistor nicht in den Sperrbereich wechselt.
% }

}




\subsection{Erstellung eines Kleinsignal-Ersatzschaltbilds\label{Aufg20}}
% Alt: Aufgabe 20
\Aufgabe{
    Zeichnen Sie das Kleinsignalersatzschaltbild der gegebenen Schaltung.
    \begin {figure} [H]
       \centering
       \begin{tikzpicture} 
    \draw (0,0) node[npn] (Q1) {};
    
    \draw (Q1.C)
    to [short] (0,1)
    to[R, l_= $500 \, \Omega$, -*] (0,3)
    to [short, -o] (0,3.5) node[right, color=voltage] {$U_\mathrm{V}$};
    
    \draw (0,1)
    to[short, *-o] (2,1);
    
    \draw (0,3)
    to[short, -]
    (-2,3) to[R=$75 \, k\Omega$] (-2,1)
    to [short, -*] (-2,0)
    to [short, -o] (-3,0);
    
    \draw (-2,0) to (Q1.B);
    
    \draw (Q1.E)
    to [short, -*] (0,-2) node[ground]{};
    
    \draw (-3,-2) to[short, o-o] (2,-2);
    
    \draw (2,1) to[open,v^, name=ua] (2,-2);
    \draw (-3,0) to[open,v>, name=ue] (-3,-2);
    
    \varrmore{ua}{$U_\mathrm{A}$};
    \varrmore{ue}{$U_\mathrm{E}$};
\end{tikzpicture}
       \label{fig:FigErsatzschaltbild1}
    \end {figure}
%     \speech{
% **Aufgabe 20**

%  **Aufgabenstellung:**
% Die Aufgabe fordert, dass das **Kleinsignal-Ersatzschaltbild** der gegebenen Schaltung gezeichnet wird.

%  **Beschreibung der Abbildung:**
% Die dargestellte Schaltung zeigt eine **Transistorverstärkerschaltung in Emitterschaltung** mit Widerständen zur Einstellung des Arbeitspunkts.

% - **Spannungsversorgung**:  
%   - Die Schaltung wird mit einer **Versorgungsspannung \( U_V \)** betrieben.
%   - \( U_V \) liegt am oberen Ende der Schaltung.

% - **Basisbeschaltung**:  
%   - Ein **75 k\(\Omega\) Widerstand** verbindet die **Basis des Transistors** mit der **Versorgungsspannung \( U_V \)**.  
%   - Dieser Widerstand sorgt für eine **Vorspannung der Basis**, um den Transistor in den gewünschten Arbeitspunkt zu bringen.

% - **Kollektorkreis**:  
%   - Der **Kollektor ist über einen 500 \(\Omega\) Widerstand** mit \( U_V \) verbunden.  
%   - Die **Ausgangsspannung \( u_A \)** wird am Kollektor abgegriffen.

% - **Eingangssignal**:  
%   - Das **Eingangssignal \( u_E \)** ist an die **Basis des Transistors** gekoppelt.  
%   - Das Signal wird verstärkt und als **invertiertes Signal am Kollektor** ausgegeben.

% - **Emitterschaltung**:  
%   - Der **Emitter ist direkt mit Masse verbunden**, was typisch für eine **Emitterschaltung** ist.

%  **Zusätzliche Hinweise für eine blinde Person:**
% - Diese Schaltung ist eine **Grundschaltung eines Verstärkers** mit einem **Transistor als aktives Element**.
% - Das **Kleinsignal-Ersatzschaltbild** wird durch Ersetzen des Transistors durch sein **äquivalentes Modell** erstellt.  
%   - Dabei werden Parameter wie **differentieller Widerstand \( r_e \)** und **Stromverstärkung \( \beta \)** berücksichtigt.
% - Das **Eingangssignal \( u_E \)** wird an der **Basis eingespeist**, und die verstärkte Version erscheint an der **Kollektor-Emitter-Strecke**.
% - Der **75 k\(\Omega\) Widerstand** dient zur **Gleichspannungsvorspannung** der Basis, der **500 \(\Omega\) Widerstand** beeinflusst die **Verstärkung und den Arbeitspunkt**.
% }

}



\Loesung{
    \begin {figure} [H]
       \centering
       \begin{tikzpicture}

\draw (0,0) to[short, -o] (1,0)node[above]{$\mathrm{B}$} -- (3,0) to[short, -*,R, l=$r_\mathrm{BE}$] (3,-2) to[short, -*] (2,-2)node[ground]{} to[short, -o] (1,-2) to[short, -*] (0,-2)node[ground]{} to[R,l=$R_\mathrm{2}$](0,0);

\draw (3,-2) to[short, -*] (4,-2) node[ground]{} to[short, -*] (5,-2) to[short, -*] (6.5,-2) to[short, -*] (7,-2)node[ground]{} to[short, -o] (8,-2)node[below]{$\mathrm{E}$} -- (9,-2) ;

%Stromquelle
\draw (5,0) to[short,i,name=i2](5,-0.5) to[I,name=I] (5,-1.5) -- (5,-2);
\draw (5,0) to[short, -*] (6.5,0) to[R,l=$r_\mathrm{CE}$] (6.5,-2);
\draw (6,0) to[short, -o] (8,0)node[above]{$\mathrm{C}$} -- (9,0) to[R,l=$R_\mathrm{C}$] (9,-2);

\draw (1,0)to[open,v>, name=ue](1,-2);
\draw (8,0)to[open,v>, name=ua](8,-2);

\varrmore{ue}{$U_\mathrm{E}$};
\varrmore{ua}{$U_\mathrm{A}$};
\iarrmore{i2}{$\mathrm{B}\cdot I_\mathrm{B}$};

\end{tikzpicture}
       \label{fig:LsgErsatzschaltbild1}
    \end {figure}
    Siehe: Abschnitt \ref{fig:ErsatzschaltbildEinesBipolartransistors} und \ref{fig:ErsatzschaltbilderEmitter-Kollektor-UndBasisschaltung}.
%     \speech{
% **Ersatzschaltbild für Aufgabe 20**  

% Das dargestellte Ersatzschaltbild zeigt die Kleinsignal-Ersatzschaltung eines Transistorverstärkers.  
% Es besteht aus folgenden Bauteilen und Anschlüssen:  

% 1. **Widerstände:**  
%    - Widerstand \( R_2 \) ist mit dem Punkt B verbunden und geht gegen Masse.  
%    - Der Basis-Emitter-Widerstand \( r_{BE} \) ist parallel zur Basis-Emitter-Strecke geschaltet.  
%    - Der Kollektorwiderstand \( R_C \) ist zwischen Punkt C und Masse geschaltet.  

% 2. **Spannungen und Ströme:**  
%    - \( U_E \) ist die Eingangsspannung an Punkt B, die über \( R_2 \) an die Basis des Transistors gelangt.  
%    - \( U_A \) ist die Ausgangsspannung am Punkt C, die über \( R_C \) abgenommen wird.  
%    - Der Basisstrom \( I_B \) fließt in die Basis des Transistors ein.  
%    - Der Kollektorstrom beträgt \( \beta \cdot I_B \), dargestellt durch die gesteuerte Stromquelle im Modell.  

% 3. **Transistormodell:**  
%    - Der Transistor wird durch eine gesteuerte Stromquelle ersetzt, die den Kollektorstrom \( \beta \cdot I_B \) liefert.  
%    - Die Basis-Emitter-Strecke wird durch den differentiellen Widerstand \( r_{BE} \) modelliert.  
%    - Die Kollektor-Emitter-Strecke ist über \( R_C \) mit Masse verbunden, wodurch die Ausgangsspannung \( U_A \) entsteht.  

% **Zusammenfassung:**  
% Das Ersatzschaltbild veranschaulicht das Verhalten eines bipolaren Transistors im Kleinsignalbetrieb.  
% Es zeigt, wie die Basis-Emitter-Spannung \( U_{BE} \) den Stromfluss steuert und wie die Verstärkung durch  
% die Stromquelle \( \beta \cdot I_B \) im Kollektorzweig realisiert wird.  
% }

}




\subsection{Erstellung eines Kleinsignal-Ersatzschaltbilds\label{Aufg21}}
% Alt: Aufgabe 21
\Aufgabe{
    Zeichnen Sie das Kleinsignalersatzschaltbild der gegebenen Schaltung.
    \begin {figure} [H]
       \centering
       \begin{tikzpicture}
    \draw (0,0) node[npn, tr circle] (Q1) {};

    \draw (Q1.C) to[short, -o, name=ic] (0,2.5)
    node[right, color=voltage]{$U_\mathrm{V}$};

    \draw (0,2)
    to[short, *-] (-3, 2) 
    to[R=$R_1$, -*]  (-3, 0) 
    to[R=$R_2$]  (-3, -3) 
    node[ground]{};

    \draw (Q1.B) 
    to[short, -o] (-5, 0)
    to[open, v, name=ue] (-5,-2.75)    
    to[short, o-] (-5, -3)
    node[ground]{};

    \draw (Q1.E) to[short, -*] ++(0, -0.25)
    to[R=$R_3$] (0, -3) node[ground]{};

    \draw (Q1.E)++(0, -0.25) to[short, -o] ++(2, 0)
    to[open, v^, name=ua] (2,-2.75)    
    to[short, o-] (2,-3)
    node[ground]{};
    
    \varrmore{ue}{$U_\mathrm{E}$};
    \varrmore{ua}{$U_\mathrm{A}$};
\end{tikzpicture}
       \label{fig:FigErsatzschaltbild2}
    \end {figure}

%     \speech{
% **A.21 Aufgabe 21**

%  **Aufgabenstellung:**
% Die Aufgabe besteht darin, das **Kleinsignal-Ersatzschaltbild** der gezeigten Transistorschaltung zu zeichnen.



%  **Beschreibung der Abbildung:**
% Die Abbildung zeigt eine **Transistorverstärkerschaltung in Emitterschaltung mit Basisspannungsteiler**.

%  **1. Spannungsversorgung:**
% - Die Schaltung wird mit einer **Versorgungsspannung \( U_V \)** betrieben, die sich am oberen Rand der Schaltung befindet.
% - Diese Spannung versorgt den **Kollektor** des Transistors über einen **Widerstand \( R_1 \)**.

%  **2. Basisbeschaltung mit Spannungsteiler:**
% - Die Basis des Transistors erhält ihre Vorspannung über **zwei Widerstände \( R_1 \) und \( R_2 \)**.  
% - Diese beiden Widerstände bilden einen **Spannungsteiler**, der die **Basisspannung einstellt**.
% - \( R_1 \) ist zwischen \( U_V \) und der Basis geschaltet, während \( R_2 \) zwischen Basis und Masse liegt.

%  **3. Kollektorkreis:**
% - Der **Kollektor ist über den Widerstand \( R_1 \)** mit der **Versorgungsspannung \( U_V \)** verbunden.
% - Die **Ausgangsspannung \( u_A \)** wird am **Kollektor des Transistors** abgegriffen.

%  **4. Emitterschaltung mit Widerstand \( R_3 \):**
% - Der **Emitter des Transistors** ist mit einem **Widerstand \( R_3 \)** gegen Masse verbunden.
% - Dieser Widerstand beeinflusst die **Verstärkung der Schaltung**.

%  **5. Eingangssignal:**
% - Das **Eingangssignal \( u_E \)** wird über eine separate Leitung an die **Basis des Transistors** eingespeist.

%  **Zusätzliche Hinweise für eine blinde Person:**
% - Die gezeigte Schaltung ist eine **Emitterschaltung mit Basisspannungsteiler**.  
% - Das **Kleinsignal-Ersatzschaltbild** wird durch Ersetzen des Transistors durch sein **äquivalentes Modell** erstellt.  
%   - Dabei werden Parameter wie **differentieller Widerstand \( r_e \)** und **Stromverstärkung \( \beta \)** berücksichtigt.
% - Die Spannungsteiler \( R_1 \) und \( R_2 \) bestimmen die **Gleichspannungsvorspannung der Basis**, während \( R_3 \) die **Gegenkopplung am Emitter** definiert.
% - Das **Eingangssignal \( u_E \)** wird an der **Basis eingespeist**, und die verstärkte Version erscheint als **invertiertes Signal am Kollektor**.
% - Der **Widerstand \( R_3 \)** beeinflusst die **Gesamtverstärkung der Schaltung**, indem er die Emitterspannung anhebt.

% Diese Schaltung wird häufig in **Verstärkerschaltungen** verwendet, um ein **stabiles Arbeitspunktverhalten** zu ermöglichen.
% }

}



\Loesung{
    \begin{figure}[H]
        \centering
        \begin{subfigure}[b]{0.8\textwidth}
            \centering
            \begin{tikzpicture}

    \draw (0,0)
    to[short, o-o] (3,0) node[above]{B}
    --(4,0) 
    to[R,l=$r_\mathrm{BE}$] (4,-2)
    to[short,-*] (6,-2) 
    to[I, i_<, name=ib] (6,0) 
    to[short, -*] (8,0) 
    to[R,l=$r_\mathrm{CE}$] (8,-2) 
    to[short, -o] (7,-2) node[above]{E}
    -- (6,-2);
    
    \draw (8,0)
    to[short,-o] (9,0)node[above]{C}
    -- (10,0) 
    -- (10,-4) 
    to[short, -*] (6,-4) node[ground]{}
    to[short, -o] (0,-4);
    
    \draw (7,-2) to[open,v^, name=ua] (7,-4);
    \draw (1,0) to[R,l=$R_\mathrm{1}$, *-*](1,-4);
    \draw (2,0) to[R,l=$R_\mathrm{2}$, *-*](2,-4);
    \draw (6,-2) to[R,l=$R_\mathrm{3}$](6,-4);
    \draw (0,0) to[open, v>, name=ue] (0,-4);
    
    \varrmore{ua}{$u_\mathrm{a}$};
    \varrmore{ue}{$u_\mathrm{e}$};
    \iarrmore{ib}{$\mathrm{B}\cdot i_\mathrm{B}$};
\end{tikzpicture}
            \label{fig:LsgErsatschaltbild2_1}
        \end{subfigure}
        \vspace{0.5cm} 
        \begin{subfigure}[b]{0.8\textwidth}
            \centering
            \begin{tikzpicture}
\draw (0,0) to[short, o-*] (1,0)node[above]{B} to[short, -*] (2,0) to[R,l=$r_\mathrm{BE}$, -*] (4,0) to[short, -*] (6,0)node[above]{E} to[short,-*] (7,0) to[short, -o] (9,0);
\draw (0,-2) to[short, o-*] (1,-2) -- (1.5,-2)node[ground]{};
\draw (1.5,-2) to[short, -*] (2,-2) to[short, -*] (4,-2) to[short, -*] (6,-2) to[short,-*] (7,-2) to[short, -o] (9,-2);


\draw (9,0) to[open,v>, name=ua] (9,-2);

\draw (1,0) to[R,l=$R_\mathrm{1}$](1,-2);
\draw (2,0) to[R,l=$R_\mathrm{2}$](2,-2);
\draw (4,0) to[R,l=$R_\mathrm{3}$](4,-2)node[below]{C};
\draw (6,0) to[I] (6,-2);
\draw (7,0) to[R,l=$r_\mathrm{CE}$](7,-2);

\draw (0,0) to[open, v>, name=ue] (0,-2);

\varrmore{ua}{$U_\mathrm{a}$};
\varrmore{ue}{$U_\mathrm{e}$};

\end{tikzpicture}
            \label{fig:LsgErsatschaltbild2_2}
        \end{subfigure}
        \label{fig:Loesung21}
    \end{figure}
    Siehe: Abschnitt \ref{fig:ErsatzschaltbildEinesBipolartransistors} und \ref{fig:ErsatzschaltbilderEmitter-Kollektor-UndBasisschaltung}.
%     \speech{
% Die Abbildung zeigt zwei verschiedene Darstellungen einer Transistorschaltung. Die obere Abbildung stellt die vollständige Schaltung eines bipolaren Transistors in Emitterschaltung dar, während die untere Abbildung die Klein-Ersatzschaltung dieses Systems zeigt.

% **Beschreibung der oberen Abbildung (vollständige Transistorschaltung):**  
% - Links befindet sich die Eingangsspannung \( U_e \), die in die Schaltung eingespeist wird.  
% - Zwei Widerstände, \( R_1 \) und \( R_2 \), sind parallel zwischen der Eingangsspannung \( U_e \) und Masse geschaltet.  
% - Diese Widerstände \( R_1 \) und \( R_2 \) legen das Basispotenzial des Transistors fest.  
% - Der Transistor befindet sich mittig in der Schaltung und ist mit seinen drei Anschlüssen wie folgt beschaltet:  
%   - **Basis (B):** Die Basis ist über die beiden Widerstände \( R_1 \) und \( R_2 \) an die Eingangsspannung \( U_e \) gekoppelt. Zudem gibt es eine gekennzeichnete Impedanz \( r_{BE} \), die den Basis-Emitter-Widerstand modelliert.  
%   - **Emitter (E):** Der Emitter ist mit einem Widerstand \( R_3 \) verbunden, der nach unten zur Masse führt.  
%   - **Kollektor (C):** Der Kollektor ist mit einem weiteren Widerstand \( r_{CE} \) verbunden, der zum Ausgang \( U_a \) führt.  
% - Die Ausgangsspannung \( U_a \) befindet sich rechts in der Schaltung und ist mit dem Kollektor verbunden.  

% Die Pfeile in der Schaltung zeigen die Richtung der Spannungen:  
% - **\( U_e \) (Eingangsspannung):** Von oben links nach unten gerichtet.  
% - **\( U_a \) (Ausgangsspannung):** Vom Kollektor des Transistors nach unten.

% **Beschreibung der unteren Abbildung (Klein-Ersatzschaltbild):**  
% - Diese Darstellung zeigt eine vereinfachte Version der oberen Schaltung, in der der Transistor durch seine klein-signaltechnischen Ersatzwerte ersetzt wurde.  
% - Die Basis ist weiterhin mit den Widerständen \( R_1 \) und \( R_2 \) verbunden.  
% - Der Basis-Emitter-Übergang wird durch einen Widerstand \( r_{BE} \) modelliert.  
% - Der Kollektor-Emitter-Widerstand \( r_{CE} \) stellt den Durchgangswiderstand des Transistors dar.  
% - Der Emitter ist über \( R_3 \) mit Masse verbunden.  
% - Die Spannungen \( U_e \) und \( U_a \) sind weiterhin links bzw. rechts gekennzeichnet.  

% Zusammengefasst zeigen diese beiden Abbildungen die Entwicklung von einer vollständigen Transistorschaltung hin zu ihrem Klein-Ersatzschaltbild, das für weitergehende Berechnungen und Analysen verwendet wird.
% }

}




\subsection{Bestimmung der Gate-Source-Spannung eines Feldeffekttransistors\label{Aufg22}}
% Alt: Aufgabe 22
\Aufgabe{
    Bei welcher Spannung an Source-Gate leitet ein Sperrschichttransistor 
    den meisten Strom?
%     \speech{
% **Aufgabe 22**

% ### **Aufgabenstellung:**
% Die Frage beschäftigt sich mit der Charakteristik eines **Sperrschicht-Feldeffekttransistors (JFET)**.

% Es soll bestimmt werden:
% - Bei welcher **Source-Gate-Spannung** \( U_{GS} \) ein **JFET den größten Strom leitet**.
%     }
}



\Loesung{
  \begin{center}
    Bei $-0,7\, \mathrm{V}$
  \end{center}
  Siehe: Abschnitt \ref{sec:ElektrischesVerhaltenFET}
%   \speech{
% **Lösung zu Aufgabe 22**
% bei -0,7 Volt
% }
}




\subsection{Untersuchung des Stromflusses im Feldeffekttransistor (FET)\label{Aufg23}}
% Alt: Aufgabe 22
\Aufgabe{
Wo fließt bei dem dargestellten Transistor der meiste Strom?
\begin {figure} [H]
       \centering
       \begin{circuitikz} 
    \draw(0,0) node[njfet] (njfet) {}
    (njfet.G) node[anchor=east] {G}
    (njfet.D) node[anchor=south] {D}
    (njfet.S) node[anchor=north] {S};
\end{circuitikz}
       \label{fig:FigNJFET}
    \end {figure}

% \speech{
% **Aufgabe 23**

%  **Aufgabenstellung:**
% Es soll bestimmt werden, **wo beim dargestellten Transistor der meiste Strom fließt**.

%  **Beschreibung der Abbildung:**
% In der Abbildung ist ein **Feldeffekttransistor (FET)** dargestellt.  
% - Das Schaltzeichen zeigt einen **n-Kanal Sperrschicht-FET (JFET)**.
% - Die Anschlüsse sind:
%   - **Source (S)** unten,
%   - **Drain (D)** oben,
%   - **Gate (G)** links.
% }


}



\Loesung{
      \begin{center}
        Zwischen Gate und Source.
      \end{center}
      Siehe: Abschnitt \ref{sec:ElektrischesVerhaltenFET}
%       \speech{
% **Lösung zu Aufgabe 23**
% Zwischen Gate und Source.
% }
}




\subsection{Interpretation der Kennlinie eines Sperrschichttransistors\label{Aufg24}}
% Alt: Aufgabe 24
\Aufgabe{
   In welche zwei Bereiche lässt sich die Ausgangskennlinie des 
   Sperrschichttransistors unterteilen?
   
   %    \speech{
   % **Aufgabe 24**
   
   %  **Aufgabenstellung:**
   % Es soll bestimmt werden, **in welche zwei Bereiche sich die Ausgangskennlinie eines Sperrschichttransistors unterteilen lässt**.
   %    }
   
}


\Loesung{

   In Sättigungs- und Widerstandsbereich.

   Siehe: Abschnitt \ref{sec:ElektrischesVerhaltenFET}
   
   %    \speech{
   % **Lösung zu Aufgabe 24**
   % In Sättigungs- und Widerstandsbereich.
   %    }
}
\subsection{Erkennung von Dotierungsmustern bei Transistoren\label{Aufg25}}
% Alt: Aufgabe 25
\Aufgabe{
   Ohne das Anlegen einer Spannung ergeben sich die folgenden Verteilungen von 
   p- und n-Gebieten. Um was für einen Typ von Transistor handelt es sich?
   
   \hspace{4ex}
   
   \begin{itemize}
      \begin{minipage}{.45\textwidth}
         \item [a)]
         %{\small \includesvg[width=0.9\textwidth]{Bilder/Aufgaben/FigTransistorDotierung1}}
         \includesvg[width=0.9\textwidth]{Bilder/Aufgaben/FigTransistorDotierung1}
         \vspace{4ex}
         \item [c)]
         \includesvg[width=0.9\textwidth]{Bilder/Aufgaben/FigTransistorDotierung3}
      \end{minipage}
      \begin{minipage}{.45\textwidth}
         \item [b)]
         \includesvg[width=0.9\textwidth]{Bilder/Aufgaben/FigTransistorDotierung2}
         \vspace{4ex}
         \item [b)]
         \includesvg[width=0.9\textwidth]{Bilder/Aufgaben/FigTransistorDotierung4}
      \end{minipage}
   \end{itemize}
}

\Loesung{
   \begin{itemize}
      \item[\bf a)] n-kanal selbstsperrend (Verarmungstyp)     Siehe: Abschnitt \ref{fig:SperrbereichN-KanalMOSFET}
      \item[\bf b)] p-kanal selbstsperrend (Verarmungstyp)     Siehe: Abschnitt \ref{fig:SperrbereichN-KanalMOSFET}
      \item[\bf c)] n-kanal selbstleitend (Anreicherungstyp)   Siehe: Abschnitt \ref{fig:OhmscherBereichN-KanalMOSFET}
      \item[\bf d)] p-kanal selbstleitend (Anreicherungstyp)   Siehe: Abschnitt \ref{fig:OhmscherBereichN-KanalMOSFET}
   \end{itemize}
}
\subsection{Erklärung des Funktionsprinzips der CMOS-Technologie\label{Aufg26}}
% Alt: Aufgabe 26
\Aufgabe{
   Was sagt das C in CMOS aus?
   \begin{itemize}
      \item[a)] Unterscheidung von Anreicherungs- und Verarmungstyp
      \item[b)] Die Verwendung von JFETs und MOSFETS in einer Schaltung
      \item[c)] Spiegelt den Verlauf der Transferkennlinie wieder
      \item[d)] Weißt auf die Spannungsempfindlichkeit von MOSFETs hin
      \item[e)] In einer Schaltung werden sowohl p- als auch n-Kanal Transistoren verwendet 
   \end{itemize}
}


\Loesung{
   \begin{itemize}
      \item [e)] In einer Schaltung werden sowohl p- als auch n-Kanal Transistoren verwendet
   \end{itemize}

   Siehe: Abschnitt \ref{sec:AufbauTransistor}
}
\subsection{Transistor-Verbundverstärker\label{Aufg27}}
% Alt: Klausuraufgabe 2
\Aufgabe{
  Zwei npn-Bipolartransistoren bilden zusammen einen rauscharmen Verstärker für Signale im Bereich unter $1\, \mathrm{\mu V}$. 
  Die Versorgungsspannung beträgt $U_1=10\,\mathrm{V}$. Am Kollektor des Transistors $\mathrm{T_2}$ liegt eine Spannung von $4\,\mathrm{V}$ gegenüber Masse an.
  Im Arbeitspunkt von $T_1$ und $T_2$ kann für die Basis-Emitter-Spannungen jeweils $U_\mathrm{BE}=600\,\mathrm{mV}$ angenommen werden.
  Die Kollektorströme $\mathrm{I_{C}}$ sind sehr groß gegenüber den Basisströmen $I_\mathrm{B}$ anzunehmen, daher gilt näherungsweise $I_\mathrm{C} = I_\mathrm{E}$.
  Zusätzlich sind die folgenden Parameter gegeben:
  
  $R_1 = 10\, \mathrm{k\Omega}$, 
  $R_2 = 10\, \mathrm{k\Omega}$, 
  $R_3 = 2\, \mathrm{k\Omega}$,
  $R_4 = 300\, \mathrm{k\Omega}$,  
  $R_5 = 10\, \mathrm{M\Omega}$ \\
  $C_1 = 1\, \mathrm{\mu F}$,  
  $C_2 = 1\, \mathrm{\mu F}$ \\
  $U_1 = 10\, \mathrm{V}$,  
  $\hat{u}_\mathrm{AC} = 0,1\, \mathrm{mV}$ \\
  
  \begin {figure} [H]
  \centering
  \begin{tikzpicture}
    \draw (0,0)
    node[npn] (Q1) {$T_1$};

    \draw (Q1.collector)
    to[short, -*] (0,1)
    to [R, l_=$R_\mathrm{1}$,] (0,4);

    \draw (Q1.base)
    to[short, -*] (-1,0)
    to[C, l_=$C_\mathrm{1}$](-4,0)
    to[sV, name=UE, v_] (-4,-4)
    node[ground]{};

    \draw (Q1.emitter)
    -- (0,-1)
    node[ground]{};

    \draw (-1,0)
    -- ++ (0,-2)
    to[R,l=$R_\mathrm{4}$, -*] (3,-2);

    \draw (3,0)
    node[npn] (Q2) {$T_2$};

    \draw (Q2.emitter)
    to[short](3,-2)
    to[R,l=$R_\mathrm{3}$] (3,-4)
    node[ground]{};

    \draw (0,1)
    -- (1,1)
    -- (2,0)
    to[short] (Q2.base);

    \draw (Q2.collector)
    to[short,-*](3,1)
    to[R,l_=$R_\mathrm{2}$](3,4)
    -- (6,4)
    to[V,name=U1, v] (6,2)
    node[ground]{};

    \draw (0,4)
    to[short, -*] (3,4);

    \draw (3,1)
    to[C,l=$C_\mathrm{2}$](6,1)
    to[R,l=$R_\mathrm{5}$](6,-1)
    node[ground]{};

    \varrmore{U1}{$U_\mathrm{1}$};
    \varrmore{UE}{$u_\mathrm{E}(t)$};
\end{tikzpicture}
  \label{fig:FigKlausurVerbundverstaerker}
  \end {figure}
  
  \begin{itemize}
    \item[a)]
          Wie groß ist der maximale Strom durch $R_2$ im Arbeitspunkt ($I_\mathrm{R2}=I_\mathrm{C,\,T2}$)?
    \item[b)]
          Wie groß ist der Strom durch $R_3$ näherungsweise, wenn $R_4 >> R_3$?
    \item[c)]
          Wie groß ist die Spannung, die über $R_3$ abfällt? 
    \item[d)]
          Wie groß ist die Gleichspannung an der Basis von $T_2$ ($U_\mathrm{B,\,T2}=U_\mathrm{CE,\,T1}$)?
    \item[e)]
          Wie groß ist der Strom durch $R_1$ ($I_\mathrm{R1}=I_\mathrm{C,\,T1}$)?
    \item[f)]
          Erläutern Sie in Stichworten die Arbeitspunktstabiliserung der gesamten Schaltung auf Grund der Gegenkopplung über $R_4$.
    \item[g)]
          Wie groß ist die Steilheit $S$ von $T_\mathrm{1}$, wenn zusätzlich $U_\mathrm{T} = 26\, \mathrm{mV}$ gegeben ist?
    \item[h)]
          Wie groß ist die Spannungsverstärkung $A_\mathrm{V} = \frac{\Delta U_\mathrm{CE}}{\Delta U_\mathrm{BE}}$, der ersten Transistorstufe?
    \item[i)]
          Wie groß ist die Spannungsverstärkung $A_\mathrm{V}$ der zweiten Transistorstufe, unter Verwendung der Definition $A_\mathrm{V,T2}=\frac{\Delta U_\mathrm{C}}{\Delta U_\mathrm{B}}$?
          Zur Lösung müssen zusätzlich die Widerstände $R_2$ und $R_3$ berücksichtigt werden.
    \item[j)]
          Wie groß ist die Spannungsverstärkung der beiden Stufen $T_\mathrm{1}$ und $T_\mathrm{2}$ zusammen?
    \item[k)]
          Zeichnen Sie das Kleinsignal-Ersatzschaltbild der gesamten Schaltung. Ersetzen Sie darin die beiden Transistoren im Arbeitspunkt jeweils durch eine steuerbare Stromquelle $S=\beta\cdot i_\mathrm{B}$ und einen differentiellen Eingangswiderstand $r_\mathrm{BE}$.
  \end{itemize}
}



\Loesung{
  \begin{itemize}
    \item[a)]
  \end{itemize}
  \begin{align*}
    I_\mathrm{C} & = \frac{6\,\mathrm{V}}{10\,\mathrm{k\Omega}} = 0,6\,\mathrm{mA}
  \end{align*}
  \begin{itemize}
    \item[b)]
  \end{itemize}
  \begin{align*}
    I_\mathrm{C,T2} & = 0,6\,\mathrm{mA}
  \end{align*}
  \begin{itemize}
    \item[c)]
  \end{itemize}
  \begin{align*}
    U_{\mathrm{R3}} & = 0,6\,\mathrm{mA} \cdot 2\,\mathrm{k\Omega} = 1,2\, \mathrm{V}
  \end{align*}
  \begin{itemize}
    \item[d)]
  \end{itemize}
  \begin{align*}
    1,2\, \mathrm{V} + 0,6\,\mathrm{V} = 1,8\,\mathrm{V}
  \end{align*}
  \begin{itemize}
    \item[e)]
  \end{itemize}
  \begin{align*}
    I_\mathrm{C,T1} & = \frac{8,2\,\mathrm{V}}{10\,\mathrm{k\Omega}} = 0,82\,\mathrm{mA}
  \end{align*}
  \begin{itemize}
    \item[f)]
  \end{itemize}
  \begin{itemize}
    \item Erhöhung von $I_\mathrm{C}$ führt zu einem Spannungsanstieg über $R_4$
    \item dadurch sinkt die Basisspannung von $T_\mathrm{1}$, was $I_\mathrm{C}$ wieder reduziert
    \item negative Rückkopplung stabilisiert Arbeitspunkt
  \end{itemize}
  \begin{itemize}
    \item[g)]
  \end{itemize}
  \begin{align*}
    S & = \frac{I_\mathrm{C,T1}}{U_\mathrm{T}} = \frac{0,82\,\mathrm{mA}}{26\,\mathrm{mA}} = 31,54\, \frac{\mathrm{mA}}{\mathrm{V}}
  \end{align*}
  \begin{itemize}
    \item[h)]
  \end{itemize}
  \begin{align*}
    A_\mathrm{V} & = S \cdot (-R_\mathrm{C}) = -315
  \end{align*}
  \begin{itemize}
    \item[i)]
  \end{itemize}
  \begin{align*}
    A_\mathrm{V} & = -\frac{R_\mathrm{C}}{R_\mathrm{E}} = \frac{10\,\mathrm{k\Omega}}{2\,\mathrm{k\Omega}} = -5
  \end{align*}
  \begin{itemize}
    \item[j)]
  \end{itemize}
  \begin{align*}
    -5 \cdot -315 = 1575 \approx 1600
  \end{align*}
  \begin{itemize}
    \item[k)]
  \end{itemize}
  \begin {figure} [H]
  \centering
  \begin{tikzpicture}
    \draw (0,0) to[short, o-*] (1,0) to[short, -o] (3,0)node[above]{B};
    \draw (5,0)node[above]{C} to[short, o-*] (7,0) to[short,-o] (8,0)node[above]{B};
    \draw (10,0)node[above]{C} to[short, o-*] (12,0) to[short, -*] (13,0) to[short, -o] (15,0);
    \draw (0,-2) to[short, o-*] (1,-2) -- (1.5,-2)node[ground]{};
    \draw (1.5,-2) to[short, -] (3,-2) to[short, -*] (4,-2)node[below]{E} to[short, -*] (7,-2) to[short, -*] (9,-2)node[below]{E} to[short, -] (10.5,-2) to[R=$R_3$, -*] (12,-2) to[short, -*] (13,-2) to[short, -o] (15,-2);


    \draw (15,0) to[open,v>, name=ua] (15,-2);
    \draw (0,0) to[open,v>, name=ue] (0,-2);

    \draw (1,0) to[R=$R_4$] (1,-2);
    \draw (3,0) to[R=$r_\mathrm{BE 1}$] (3,-2);
    \draw (5,0) to[I, l=$S_1$] (5,-2);
    \draw (7,0) to[R=$R_1$] (7,-2);
    \draw (8,0) to[R=$r_\mathrm{BE2}$] (8,-2);
    \draw (10,0) to[I, l=$S_2$] (10,-2);
    \draw (12,0) to[R=$R_2$] (12,-2);
    \draw (13,0) to[R=$R_5$] (13,-2);


    \varrmore{ue}{$u_\mathrm{E}$};
    \varrmore{ua}{$u_\mathrm{A}$};

\end{tikzpicture}
  \label{fig:LsgKlausurVerbundverstaerker}
  \end {figure}
  Siehe: Abschnitt \ref{sec:DarlingtonTransistor}
}

\subsection{Gleichrichterschaltung\label{Aufg28}}
\Aufgabe{
  Hier eine Gleichrichterschaltung mit pn-Silizium-Dioden. 
  Die Spannungsquelle $V_1$ liefert sinusförmige $50 \, \mathrm{Hz}$ Wechselspannung mit einer Amplitude von $5\, \mathrm{V}$.
  \begin {figure} [H]
  \centering
  \begin{tikzpicture}
    \draw (-1,0) 
    to[D,l=$D_\mathrm{2}$] (-1,2) -- (-1,4)
    to[D,l=$D_\mathrm{1}$] (-1,6) -- (1,6)
    to[C,l=$C_\mathrm{1}$] (1,4) -- (1,2)
    to[C,l=$C_\mathrm{2}$] (1,0) -- (-1,0);

    \draw (1,0)
    to[short, *-] (3,0)
    to[R, l_=$R_2$] (3,2) -- (3,4)
    to[R, l_=$R_1$] (3,6);  

    \draw (3,0)
    to[short, *-] (5.5,0)
    to[R, l_=$R_3$] (5.5,6)
    to[short, -*] (3,6)
    to[short, -*] (1,6);

    \draw (3,2)
    to[short, *-] (1,2)
    to[short, *-*] (-3,2)    
    node[ground]{}
    to[sV, name=UE, v<] (-3,4)
    to[short, -*] (-1,4); 

    \varrmore{UE}{$u_\mathrm{E}(t)$};
\end{tikzpicture}
  \label{fig:FigKlausurGleichrichter}
  \end {figure}
  \begin{itemize}
    \item[a)]
          Welche Gleichspannungen treten über $R_1$,$R_2$ und $R_3$ auf?
    \item[b)]
          Welche Funktionen haben die Kondensatoren $C_1$ und $C_2$? 
    \item[c)]
          Was ist der besondere Vorteil dieser Gleichrichterschaltung, z.B. für Anwendungen mit Operationsverstärkern?
    \item[d)]
          Skizzieren Sie eine andere Spannungs-Verdopplungs-Schaltung, bei der die maximale positive Gleichsspannung (Ausgangsspannung) gegenüber Massepotential der speisenden Wechselspannung $V_1$ zur Verfügung steht. 
  \end{itemize}
  
  
}

\Loesung{
  \begin{itemize}
    \item[a)]
  \end{itemize}
  \begin{itemize}
    \item $U_\mathrm{R1} = 4,3 \, \mathrm{V} \Rightarrow$ da $0,7 \,\mathrm{V}$ über $D_1$ abfallen \\
    \item $U_\mathrm{R2} = -4,3 \, \mathrm{V} \Rightarrow$ da gleiches für die negative Halbwelle \\
    \item $U_\mathrm{R3} = 8,6 \,\mathrm{V} \Rightarrow$ da $R_3$ beide Spannungswerte aufnimmt \\
  \end{itemize}
  \begin{itemize}
    \item[b)]
  \end{itemize}
  \begin{center}
    Die Kondensatoren $C_1$ und $C_2$ haben einerseits die Funktionen die Signale zu glätten, sowohl von der negativen als auch der positiven Halbwelle. 
  \end{center}
  \begin{itemize}
    \item[c)]
  \end{itemize}
  \begin{center}
    Der Vorteil dieser Gleichrichtung ist, das sowohl positive als auch negative Spannungen abgegriffen werden können.
  \end{center}
  \begin{itemize}
    \item[d)]
  \end{itemize}
  \begin {figure} [H]
  \centering
  \begin{tikzpicture}
    \draw (0,0) to[short, o-] (1,0) to[C,l=$C_\mathrm{1}$](2,0) to[short, *-] (2,0) to[D,l=$D_\mathrm{2}$, -*] (4,0) to[short, -o] (6,0);
    \draw (0,-2) to[short, o-] (1,-2) -- (3,-2)node[ground]{};
    \draw (1.5,-2) to[short, -*] (2,-2) to[short, -*] (4,-2) to[short, -o] (6,-2);
    
    
    \draw (6,0) to[open,v>, name=ua] (6,-2);
    
    
    \draw (2,-2) to[D,l=$D_\mathrm{1}$](2,0);
    \draw (4,-2) to[C,l=$C_\mathrm{2}$](4,0);
    
    
    \draw (0,0) to[open, v>, name=ue] (0,-2);
    
    \varrmore{ua}{$U_\mathrm{a}$};
    \varrmore{ue}{$U_\mathrm{e}$};
    
    \end{tikzpicture}
  \label{fig:LsgKlausurGleichrichter}
  \end {figure}
  Siehe: Abschnitt \ref{fig:BrueckengleichrichterSchaltung}
}
% \subsection{Praktikumsaufgabe 1\label{Aufg29}}
\Aufgabe{
    \begin{figure}[H]
    \centering
    \subsection{Praktikumsaufgabe 1\label{Aufg29}}
\Aufgabe{
    \begin{figure}[H]
    \centering
    \input{Bilder/Aufgaben/Praktikumsaufgabe1.tex}
    \label{fig:Praktikumsaufgabe1}
    \end{figure}

    Berechnen Sie den Widerstand $\mathrm{R_1}$, sodass an $\mathrm{R_2}$ eine Spannung von 1\,V anliegt. Gegeben ist eine Eingangsspannung von $\mathrm{U_E}$ und ein Widerstand $\mathrm{R_2}$.
    
    \begin{itemize}
        \item[a)] Leiten Sie die allgemeine Formel für den Spannungsteiler her.
        \item[b)] Berechnen Sie $\mathrm{R_1}$ für $\mathrm{U_E} = 5\,\text{V}$ und $\mathrm{R_2} = 2\,\text{k}\Omega$.
    \end{itemize}
}


\Loesung{

}
    \label{fig:Praktikumsaufgabe1}
    \end{figure}

    Berechnen Sie den Widerstand $\mathrm{R_1}$, sodass an $\mathrm{R_2}$ eine Spannung von 1\,V anliegt. Gegeben ist eine Eingangsspannung von $\mathrm{U_E}$ und ein Widerstand $\mathrm{R_2}$.
    
    \begin{itemize}
        \item[a)] Leiten Sie die allgemeine Formel für den Spannungsteiler her.
        \item[b)] Berechnen Sie $\mathrm{R_1}$ für $\mathrm{U_E} = 5\,\text{V}$ und $\mathrm{R_2} = 2\,\text{k}\Omega$.
    \end{itemize}
}


\Loesung{

} 
% \begin{tikzpicture}
    \draw (0,0) to[short, -*] (1,0)node[ground]{} to[short, -*] (2.5,0) -- (5,0);
    \draw (5,5) to[C,name=C1, l=$\mathrm{C_1}$](5,3) to[V,name=V2] (5,1) -- (5,0);

    \draw (0,5) to[V, name=V1] (0,0); 
    \draw (0,5) to[R, name=R2, l=$\mathrm{R_2}$, -*] (2.5,5) to[R,name=R3, l=$\mathrm{R_3}$] (5,5);
    
    \draw (2.5,0) to[R,name=R1, l=$\mathrm{R_1}$] (2.5,5);
    
    \varrmore{V1}{$\mathrm{V_1}$};
    \varrmore{V2}{$\mathrm{V_2}$};
\end{tikzpicture}
   
\subsection{Untersuchung eines einfachen Gleichrichters\label{Aufg31}}
% Alt: Praktikumsaufgabe 3
\Aufgabe{

    Analysieren Sie den folgenden Gleichrichter. Zusätzlich sind die folgenden Parameter gegeben:
    
    $\hat{u}_\mathrm{E} = 5\,\mathrm{V}$,
    $R_\mathrm{L} = 1\,\mathrm{k\Omega}$
    und eine Diodendurchlassspannung von $U_\mathrm{f} = 0,7\,\mathrm{V}$.
    
    \begin{figure}[H]
        \centering
        \begin{tikzpicture}
    \draw (4,0)
    to[short, -*] (0,0) node[ground]{}
    to[sV,name=UE, v<] (0,2)
    to[D, l=$D_\mathrm{1}$] (4,2) 
    to[R, l=$R_\mathrm{L}$] (4,0);

    \draw (4,2)
    to[short, *-o] (5.5,2)
    to[open,v^,name=UA] (5.5,0)
    to[short, o-*] (4,0);

    \varrmore{UE}{$u_\mathrm{E}(t)$};
    \varrmore{UA}{$U_\mathrm{A}$};
\end{tikzpicture}
        \label{fig:FigPraktikumGleichrichter}
    \end{figure}
    
    \begin{itemize}
        \item[a)] Berechnen Sie die Ausgangsspannung $U_\mathrm{A}$.
        \item[b)] Bestimmen Sie den durch den Lastwiderstand $R_\mathrm{L}$ fließenden Strom $I_\mathrm{L}$.
    \end{itemize}
}


\Loesung{
    \begin{itemize}
        \item[a)] Berechnung der Ausgangsspannung $U_\mathrm{A}$ \\[0.2cm]
              Die gegebene Schaltung besteht aus einer **Diode in Vorwärtsrichtung** und einem **Lastwiderstand**.  
              Die Ausgangsspannung $U_\mathrm{A}$ entspricht der Eingangsspannung abzüglich der Diodendurchlassspannung:
              \[
                  U_\mathrm{A} = U_\mathrm{E} - U_\mathrm{f}
              \]
              Einsetzen der Werte:
              \[
                  U_\mathrm{A} = 5\,\mathrm{V} - 0{,}7\,\mathrm{V} = 4{,}3\,\mathrm{V}
              \]
              
        \item[b)] Bestimmung des Laststroms $I_\mathrm{L}$ \\[0.2cm]
              Der durch den Lastwiderstand $R_\mathrm{L}$ fließende Strom berechnet sich nach dem Ohmschen Gesetz:
              \[
                  I_\mathrm{L} = \frac{U_\mathrm{A}}{R_\mathrm{L}}
              \]
              Einsetzen der Werte:
              \[
                  I_\mathrm{L} = \frac{4{,}3\,\mathrm{V}}{1\,\mathrm{k\Omega}} = 4{,}3\,\mathrm{mA}
              \]
    \end{itemize}
    Siehe: Abschnitt \ref{sec:Einweggleichrichter}
}

\subsection{Analyse einer Spannungsverdoppler-Schaltung\label{Aufg33}}
% Alt: Praktikumsaufgabe 5
\Aufgabe{
    \begin{figure}[H]
    \centering
    \begin{tikzpicture}
    \draw (0,4)
    to[D,l=$D_\mathrm{1}$, *-*] (2,4)
    to[C,l=$C_\mathrm{1}$, -*] (2,2) 
    to[C,l=$C_\mathrm{2}$, -*] (2,0) 
    to[D,l=$D_\mathrm{2}$] (0,0) -- (0,4); 

    \draw (0,4) -- (-2,4)
    to[sV,name=UE, v_] (-2,2) -- (2,2);

    \draw (2,4) -- (4,4)
     to[R,l=$R_\mathrm{L}$] (4,0) -- (2,0); 
     
    \draw (4,4)
    to[short, *-o] (5.5,4)
    to[open,v^,name=UA] (5.5,0)
    to[short, o-*] (4,0);

    \varrmore{UE}{$u_{\mathrm{E}_i}(t)$};
    \varrmore{UA}{$U_\mathrm{A}$};
\end{tikzpicture}
    \label{fig:FigPraktikumSpannungsverdoppler}
    \end{figure}

        Analysieren Sie die Funktion der Delon-Schaltung zur Spannungsverdopplung und berechnen Sie die Ausgangsspannung für verschiedene Eingangsspannungen.
    
    Gegeben:
    \begin{itemize}
        \item Eingangsspannung $U_\mathrm{E}$ mit Effektivwerten von $110\,\mathrm{V}$ und $230\,\mathrm{V}$
        \item Zwei Dioden mit einer Durchlassspannung von $0,7\,\mathrm{V}$
        \item Zwei Kondensatoren mit der Kapazität $C$
    \end{itemize}
    
    \begin{itemize}
        \item[a)] Erklären Sie das Funktionsprinzip der Delon-Schaltung und wie sie die Eingangsspannung verdoppelt.
        \item[b)] Berechnen Sie die theoretische Leerlauf-Ausgangsspannung $U_\mathrm{A}$ für die gegebenen Eingangsspannungen.
        \item[c)] Diskutieren Sie, wie sich die Ausgangsspannung $U_\mathrm{A}$ unter Lastbedingungen verhält und welche Faktoren die Spannungsstabilität beeinflussen.
    \end{itemize}
}


\Loesung{
    \begin{itemize}
        \item[a)] Funktionsprinzip der Delon-Schaltung  \\
        Die Schaltung arbeitet als Kaskaden-Gleichrichter. Während der positiven Halbwelle der Wechselspannung lädt sich $C_1$ über $D_1$ auf die Eingangsspitzen-Spannung $\hat{U}_\mathrm{E}$ auf. Während der negativen Halbwelle lädt sich $C_2$ über $D_2$ auf. Dadurch addieren sich die Spannungen von $C_1$ und $C_2$, wodurch am Ausgang eine theoretische Gleichspannung von ca. $2 \cdot \hat{U}_\mathrm{E}$ entsteht.
    
        \item[b)] Berechnung der Leerlauf-Ausgangsspannung $U_\mathrm{A}$  
        Die Eingangsspannung ist als Effektivwert $U_\mathrm{E}$ gegeben. Die Spitzenwertspannung beträgt:
        $$ \hat{U}_\mathrm{E} = \sqrt{2} \cdot U_\mathrm{E} $$  
        Da die Schaltung eine Spannungsverdopplung erzeugt, ergibt sich die Leerlauf-Ausgangsspannung:
        $$ U_\mathrm{A} = 2 \cdot (\hat{U}_\mathrm{E} - U_\mathrm{D}) = 2 \cdot (\sqrt{2} \cdot U_E - 0{,}7\,\mathrm{V}) $$  
        Für $ U_\mathrm{E} = 110\, \mathrm{V} $:
        $$ U_\mathrm{A} = 2 \cdot (\sqrt{2} \cdot 110\, \mathrm{V} - 0{,}7\, \mathrm{V}) \approx 308\, \mathrm{V} $$  
        Für $ U_\mathrm{E} = 230\,V $:
        $$ U_\mathrm{A} = 2 \cdot (\sqrt{2} \cdot 230\,\mathrm{V} - 0{,}7\, \mathrm{V}) \approx 644\, \mathrm{V} $$  
    
        \item[c)] Verhalten unter Lastbedingungen  
        Im Leerlauf bleibt die Ausgangsspannung hoch. Unter Last fällt die Spannung ab, abhängig vom Innenwiderstand der Kondensatoren, der Belastung $R_\text{Last}$ und den Diodenverlusten. Der Spannungsabfall hängt von der Restwelligkeit ab, die durch die Lade- und Entladezeiten der Kondensatoren beeinflusst wird. Ein großer Laststrom reduziert die Spannung, da die Kondensatoren sich schneller entladen.
    
        Faktoren, die die Spannung beeinflussen:
        \begin{itemize}
            \item Innenwiderstand der Dioden
            \item Kapazität der Kondensatoren $C$
            \item Belastung durch $R_\text{Last}$
            \item Netzfrequenz ($50\, \text{Hz}$ oder $60 \, \text{Hz}$)
        \end{itemize}
    \end{itemize}
}
\subsection{Differentieller Widerstand einer Diode\label{Aufg34}}
% Alt: Praktikumsaufgabe 7
\Aufgabe{
    \begin{figure}[H]
    \centering
    \begin{tikzpicture}
    \draw (-1,2) to[sinusoidal voltage source,name=V2] (-1,0) -- (0,0);
    \draw  (-1,2) -- (2,2);
    \draw (2,0) to[potentiometer, l=$P_\mathrm{1}$, *-] (2,2);
    \draw (0,0) to[short,-*](1,0)node[ground]{} -- (3.25,0);
    
    \draw (2,1)--(3,1)to[R,l=$R_\mathrm{4}$](5,1) to[C,l=$C_\mathrm{1}$, -*] (5,-1);
    \draw (5,-1) to[D,l=$D_\mathrm{1}$] (7,-1) -- (7,-3) to[short,-*] (2,-3)node[ground]{}--(0,-3);
    
    \draw (0,-1) to[R,l=R1] (5,-1);
    \draw (0,-1) to[V,name=V1](0,-3);
    
    \draw (3,1) to[open,v,name=du](3,0);
    
    \varrmore{du}{$\mathrm{\Delta U}$};
    \varrmore{V1}{$V_\mathrm{1}$};
    \varrmore{V2}{$V_\mathrm{2}$};
\end{tikzpicture}
   
    \label{fig:FigPraktikumBasisschaltung}
    \end{figure}
    
  Analysieren Sie das Grundverhalten eines Bipolartransistors in einer Basisschaltung.
    
    Gegeben:
    \begin{itemize}
        \item NPN-Transistor mit Stromverstärkungsfaktor $\beta = 100$
        \item Basisstrom $I_\mathrm{B} = 20\,\mathrm{\mu A}$
    \end{itemize}
    
    \begin{itemize}
        \item[a)] Berechnen Sie den Kollektorstrom $I_\mathrm{C}$.
        \item[b)] Bestimmen Sie die Kollektor-Emitter-Spannung $U_{CE}$, wenn der Kollektorwiderstand $R_\mathrm{C} = 1\,\mathrm{k\Omega}$ und die Versorgungsspannung $U_\mathrm{CC} = 12\,\mathrm{V}$ beträgt.
    \end{itemize}
}



\Loesung{
    \begin{itemize}
        \item[a)] Berechnung des Kollektorstroms $I_\mathrm{C}$  
        Der Kollektorstrom eines Bipolartransistors ergibt sich aus der Beziehung:
        $$ I_\mathrm{C} = \beta \cdot I_\mathrm{B} $$
        mit den gegebenen Werten:
        $$ I_\mathrm{C} = 100 \cdot 20\,\mathrm{\mu A} $$
        $$ I_\mathrm{C} = 2\,\mathrm{mA} $$
    
        \item[b)] Bestimmung der Kollektor-Emitter-Spannung $U_\mathrm{CE}$  
        Die Kollektor-Emitter-Spannung ergibt sich aus der Gleichung:
        $$ U_\mathrm{CE} = U_\mathrm{CC} - I_\mathrm{C} \cdot R_\mathrm{C} $$
        Einsetzen der Werte:
        $$ U_\mathrm{CE} = 12\,\mathrm{V} - (2\,\mathrm{mA} \cdot 1\,\mathrm{k\Omega}) $$
        $$ U_\mathrm{CE} = 12\,\mathrm{V} - 2\,\mathrm{V} $$
        $$ U_\mathrm{CE} = 10\,\mathrm{V} $$
    \end{itemize}
    Siehe: Abschnitt \ref{fig:DiodenkennlinieUndErsatzschaltbild}
}
\subsection{Grundverhalten eines NPN-Bipolartransistors\label{Aufg35}}
% Alt: Praktikumsaufgabe 8
\Aufgabe{
    \begin{figure}[H]
    \centering
    \begin{tikzpicture}
    \draw
    (0,0) node[npn] (Q1) {}
    (Q1.base) 
    (Q1.collector)  
    (Q1.emitter);

\draw (-3,0) to[R,l=$R_\mathrm{3}$,*-](Q1.B);
\draw (-3,2) to[R,l=$R_1$](-3,0);
\draw (-3,2) -- (-5,2) to[V,name=V1,-*](-5,-2)node[ground]{} to[short,-*](-3,-2) -- (0,-2);
\draw (-3,0) to[R,l=$R_\mathrm{2}$](-3,-2);

\draw (Q1.C) to[R,l=$R_\mathrm{4}$] (0,2.5) -- (2,2.5) to[V,name=V2] (2,-2) -- (0,-2);

\draw (Q1.E) to[short,-*] (0,-2);

\varrmore{V1}{$V_\mathrm{1}$};
\varrmore{V2}{$V_\mathrm{2}$};
\end{tikzpicture}
   
    \label{fig:FigPraktikumNPN}
    \end{figure}
    
    Analysieren Sie das Grundverhalten eines Bipolartransistors in einer Basisschaltung.
    
    Gegeben:
    \begin{itemize}
        \item NPN-Transistor mit Stromverstärkungsfaktor $\beta = 100$
        \item Basisstrom $I_\mathrm{B} = 20\,\mathrm{\mu A}$
    \end{itemize}
    
    \begin{itemize}
        \item[a)] Berechnen Sie den Kollektorstrom $I_\mathrm{C}$.
        \item[b)] Bestimmen Sie die Kollektor-Emitter-Spannung $U_\mathrm{CE}$, wenn der Kollektorwiderstand $R_\mathrm{C} = 1\,\mathrm{k\Omega}$ und die Versorgungsspannung $U_\mathrm{CC} = 12\,\mathrm{V}$ beträgt.
    \end{itemize}
}


\Loesung{
    \begin{itemize}
        \item[a)] Berechnung des Kollektorstroms $I_\mathrm{C}$  
        Der Kollektorstrom eines Bipolartransistors ergibt sich aus der Beziehung:
        $$ I_\mathrm{C} = \beta \cdot I_\mathrm{B} $$
        mit den gegebenen Werten:
        $$ I_\mathrm{C} = 100 \cdot 20\,\mathrm{\mu A} $$
        $$ I_\mathrm{C} = 2\,\mathrm{mA} $$
    
        \item[b)] Bestimmung der Kollektor-Emitter-Spannung $U_\mathrm{CE}$  
        Die Kollektor-Emitter-Spannung ergibt sich aus der Gleichung:
        $$ U_\mathrm{CE} = U_\mathrm{CC} - I_\mathrm{C} \cdot R_\mathrm{C} $$
        Einsetzen der Werte:
        $$ U_\mathrm{CE} = 12\,\mathrm{V} - (2\,\mathrm{mA} \cdot 1\,\mathrm{k\Omega}) $$
        $$ U_\mathrm{CE} = 12\,\mathrm{V} - 2\,\mathrm{V} $$
        $$ U_\mathrm{CE} = 10\,\mathrm{V} $$
    \end{itemize}
    Siehe: Abschnitt \ref{fig:UebersichtBipolartransistoren}
}
\subsection{Zeichnung und Berechnung einer Kollektorschaltung\label{Aufg36}}
% Alt: Praktikumsaufgabe 9
\Aufgabe{
    \begin{itemize}
        \item[a)] Zeichnen Sie die folgende Schaltung:  
        Beschalten Sie einen npn-Transistor mit einem Basisvorwiderstand von  
        $$ R_\mathrm{B} = 4,7\,\mathrm{k\Omega}. $$  
        Der Emitterwiderstand beträgt 
        $$ R_\mathrm{E} = 1\,\mathrm{k\Omega}. $$
        Die Spannungsversorgung beträgt 
        $$ V_\mathrm{0} = 15\,\mathrm{V}. $$
        Schalten Sie eine $10 \,\mathrm{V}$-Z-Diode in Sperrrichtung von der Basis auf Masse.  
        
        Verwenden Sie nur eine Spannungsversorgung, welche sowohl die Basis als auch den Kollektor versorgt.
        \item[b)] Berechnen Sie die Emitterspannung $V_\mathrm{E}$, und den Emitterstrom $I_\mathrm{E}$.
        \item[c)] Berechnen Sie die Leistung am Emitterwiderstand $P_\mathrm{RE}$.
    \end{itemize}
}


\Loesung{
\begin{itemize}
    \item[a)] Schaltung:
    \begin{figure}[H]
        \centering
        \begin{tikzpicture}
    \draw
    (0,0) node[npn] (Q1) {}
    (Q1.base) 
    (Q1.collector)  
    (Q1.emitter);

\draw (Q1.C) -- (0,2);
\draw (Q1.E) to[R,l=$R_\mathrm{E}$] (0,-2.5) -- (-2,-2.5) to[zDo,*-*] (-2,0) to[R,l=$R_\mathrm{B}$] (-2,2);
\draw (Q1.B) -- (-2,0);

\draw (-2,-2.5) -- (-4,-2.5);
\draw (-4,2) to[V,name=V0] (-4,-2.5);
\draw (-4,2)to[short,-*](-2,2)--(0,2);

\varrmore{V0}{$V_\mathrm{0}$};
\end{tikzpicture}
   
        \label{fig:LsgPraktikumZeichnung}
        \end{figure}


    \item[b)] Berechnung der Emitterspannung $V_\mathrm{E}$ \\
    Die Zenerdiode fixiert die Basisspannung des Transistors auf $V_\mathrm{Z} = 10\,\mathrm{V}$. 
    Die Basis-Emitter-Spannung beträgt $V_{BE} = 0,7\,\mathrm{V}$.
    $$ V_\mathrm{E} = V_\mathrm{Z} - V_\mathrm{BE} = 10\,\mathrm{V} - 0,7\,\mathrm{V} = 9,3\,\mathrm{V} $$
    $$ I_\mathrm{E} = \frac{V_\mathrm{E}}{R_\mathrm{E}} = \frac{9,3\,\mathrm{V}}{1\,\mathrm{k\Omega}} = 9,3\,\mathrm{mA} $$

\item[c)] Berechnung der Leistung am Emitterwiderstand $P_{RE}$ \\
    $$ P_\mathrm{RE} = I_\mathrm{E}^2 \cdot R_\mathrm{E} = (9,3\,\mathrm{mA})^2 \cdot 1\,\mathrm{k\Omega} = 86,49\,\mathrm{mW} $$

\end{itemize}
Siehe: Abschnitt \ref{fig:GrundschaltungenBipolartransistor}
}
 

%Beispiel Aufgaben (Skript Aufgabe 4.10)


         


%\clearpage
%\fancyhead[OL]{\textsc{Bildernachweis}}
%\listoffigures

% Literaturverzeichnis
%\clearpage
%\fancyhead[OL]{\textsc{Literaturverzeichnis}}
%\input{Literatur}

% Index erstellen
\clearpage
\fancyhead[OL]{\textsc{Index}}
\addcontentsline{toc}{section}{Index}
\printindex


\end{document}
