\subsection{Erkennung von Dotierungsmustern bei Transistoren\label{Aufg25}}
% Alt: Aufgabe 25
\Aufgabe{
   Ohne das Anlegen einer Spannung ergeben sich die folgenden Verteilungen von 
   p- und n-Gebieten. Um was für einen Typ von Transistor handelt es sich?
   
   \hspace{4ex}
   
   \begin{itemize}
      \begin{minipage}{.45\textwidth}
         \item [a)]
         %{\small \includesvg[width=0.9\textwidth]{Bilder/Aufgaben/FigTransistorDotierung1.svg}}
         \includesvg[width=0.9\textwidth]{Bilder/Aufgaben/FigTransistorDotierung1.svg}
         \vspace{4ex}
         \item [c)]
         \includesvg[width=0.9\textwidth]{Bilder/Aufgaben/FigTransistorDotierung3.svg}
      \end{minipage}
      \begin{minipage}{.45\textwidth}
         \item [b)]
         \includesvg[width=0.9\textwidth]{Bilder/Aufgaben/FigTransistorDotierung2.svg}
         \vspace{4ex}
         \item [b)]
         \includesvg[width=0.9\textwidth]{Bilder/Aufgaben/FigTransistorDotierung4.svg}
      \end{minipage}
   \end{itemize}
}

\Loesung{
   \begin{itemize}
      \item[\bf a)] n-kanal selbstsperrend (Verarmungstyp)     Siehe: Abschnitt \ref{fig:SperrbereichN-KanalMOSFET}
      \item[\bf b)] p-kanal selbstsperrend (Verarmungstyp)     Siehe: Abschnitt \ref{fig:SperrbereichN-KanalMOSFET}
      \item[\bf c)] n-kanal selbstleitend (Anreicherungstyp)   Siehe: Abschnitt \ref{fig:OhmscherBereichN-KanalMOSFET}
      \item[\bf d)] p-kanal selbstleitend (Anreicherungstyp)   Siehe: Abschnitt \ref{fig:OhmscherBereichN-KanalMOSFET}
   \end{itemize}
}