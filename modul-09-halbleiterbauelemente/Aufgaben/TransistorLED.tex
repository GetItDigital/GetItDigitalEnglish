\subsection{Transistorschaltung mit LED\label{Aufg15}}
% Alt: Aufgabe 15
\Aufgabe{
  Mithilfe eines Transistors soll eine LED betrieben werden. Dabei muss durch die LED ein Strom von $30\,\mathrm{mA}$ fließen, damit sie ihre optimale Leuchtkraft erreicht. \\
  Der verwendete Siliziumtransistor besitzt einen Stromverstärkungsfaktor von $\beta = 50$ und im durchgeschalteten Zustand eine Kollektor-Emitter-Spannung von $U_\mathrm{CE} = 0,3\,\mathrm{V}$. \\
  Welche Spannung $U$ muss an der Basis anliegen, damit der Transistor durchschaltet? 
  \begin {figure} [H]
     \centering
     \begin{tikzpicture}
    % Spannungspfeile in blau
    \draw(8,3) to[open, v>, name=9V] (8,0);
    % Batterie links
    \draw
    (0,1.5) to[V, name=U] (0,0);
    % Widerstand RV
    \draw
    (0,1.5) -- (2,1.5)
    to[R=$3.3\,k\Omega$] (3.5,1.5) to[short,i,name=IB] (4.5,1.5)
    -- (5,1.5) ; % Verbindung mit Basis
    \draw(4.7,1.2) to[open, v>, name=ube] (5.2,0.7);
    % Transistor
    \draw
   (5.5,1.5) node[npn, tr circle] (Q1) {}
   (Q1.base) 
   (Q1.collector)  
   (Q1.emitter)  ;
   
   \draw
   (Q1.C) -- (5.5,3)
   to[R=$270\,\Omega$] (7,3);



   \draw (Q1.E) -- (5.5,0) to[short, -o] (8,0);


   \draw (0,0) -- (5.5,0); 

   \draw (8,3) to[leDo, o-] (7,3);

  \varrmore{U}{$U$};
   \varrmore{9V}{$\mathrm{9\,V}$};
   \iarrmore{IB}{$I_\mathrm{B}$};
   \varrmore{ube}{$U_\mathrm{BE}$};
  \end{tikzpicture}
     \label{fig:FigTransistorLED}
  \end {figure}

%   \speech{Aufgabe 15  

% Mithilfe eines Transistors soll eine LED betrieben werden. Dabei muss durch die LED ein Strom von **30mA** fließen, damit sie ihre optimale Leuchtkraft hat. Der Siliziumtransistor hat einen Verstärkungsfaktor \( \beta = 50 \) und im durchgeschalteten Zustand ist die Kollektor-Emitter-Spannung \( U_{CE} = 0.3V \).  

% Was für eine Spannung \( U \) muss an der Basis anliegen, damit der Transistor durchschaltet?  

% Beschreibung der Abbildung  

% Die Abbildung zeigt eine **Transistorschaltung zur Ansteuerung einer LED**.  

% - Rechts oben ist eine **LED** mit einem **Vorwiderstand von 270Ω** an einer **9V-Spannungsquelle** angeschlossen.  
% - Die LED ist in Reihe mit dem **Kollektor des npn-Transistors** geschaltet.  
% - Der **Emitter des Transistors** ist direkt mit Masse (0V) verbunden.  
% - Der **Basisstrom \( I_B \)** fließt durch einen **3.3kΩ-Widerstand** und wird über die Spannung \( U \) an die Basis des Transistors angelegt.  
% - Die Basis-Emitter-Spannung \( U_{BE} \) ist in der Abbildung mit einem blauen Pfeil gekennzeichnet.  

% Die Aufgabe besteht darin, die **Spannung \( U \)** an der Basis zu berechnen, sodass der Transistor sicher durchschaltet und der gewünschte LED-Strom von **30mA** fließen kann.  
% }  

}


\Loesung{
Berechnung des Kollektorstroms bei bekannter Spannung und Lastwiderstand:
\begin{equation}
    I_\mathrm{C} = \frac{U}{R}
\end{equation}
\begin{align*}
    I_\mathrm{C} &= \frac{9\,\mathrm{V} - 0,3\,\mathrm{V}}{270\,\Omega} = 32,2\,\mathrm{mA}
\end{align*}

Berechnung des benötigten Basisstroms mit dem Stromverstärkungsfaktor $\beta = 50$:
\begin{equation}
    I_\mathrm{B} = \frac{I_\mathrm{C}}{\beta}
\end{equation}
\begin{align*}
    I_\mathrm{B} &= \frac{32,2\,\mathrm{mA}}{50} = 0,644\,\mathrm{mA}
\end{align*}

Spannungsabfall am Vorwiderstand:
\begin{align*}
    U_\mathrm{V} &= 0,644\,\mathrm{mA} \cdot 3,3\,\mathrm{k\Omega} = 2,12\,\mathrm{V}
\end{align*}

Gesamte Basisspannung (inkl. \( U_\mathrm{BE} \)):
\begin{align*}
    U &= U_\mathrm{V} + U_\mathrm{BE} = 2,12\,\mathrm{V} + 0,6\,\mathrm{V} = 2,72\,\mathrm{V}
\end{align*}

\textbf{Antwort:}  
Die Basisspannung muss \( \boxed{2{,}72\,\mathrm{V}} \) betragen, damit der Transistor durchschaltet.

    Siehe: Abschnitt \ref{sec:Bipolartransistor} und \ref{fig:StroemeUndSpannungenBeimBipolartransistor}

%     \speech{
% **Aufgabe 15 – Berechnung des Basisstroms, der Verstärkung und der Eingangsspannung eines npn-Transistors**

% Gegeben ist eine Schaltung mit einem **npn-Transistor**, der eine **LED mit 270 Ω Vorwiderstand** ansteuert. Die Aufgabe besteht darin, den **Kollektorstrom \( I_C \)**, den **Basisstrom \( I_B \)**, die **Verstärkung \( \beta \)** sowie die benötigte Eingangsspannung \( U \) zu berechnen.

% ---

% ### Berechnung des Kollektorstroms \( I_C \)
% Die Formel für den Strom lautet:

% \[
% I_C = \frac{U}{R}
% \]

% Gegeben sind:
% - Versorgungsspannung \( U = 9V \)
% - Sättigungsspannung des Transistors \( U_{CE,sat} = 0,3V \)
% - Widerstand \( R = 270Ω \)

% Einsetzen der Werte:

% \[
% I_C = \frac{9V - 0,3V}{270Ω}
% \]

% \[
% I_C = \frac{8,7V}{270Ω} = 32,2 mA
% \]

% **Ergebnis**: Der Kollektorstrom \( I_C \) beträgt **32,2 mA**.

% ---

% ### Berechnung des Basisstroms \( I_B \)
% Die Formel für den Basisstrom lautet:

% \[
% I_B = \frac{I_C}{\beta}
% \]

% Gegeben ist der **Verstärkungsfaktor**:

% - \( \beta = 50 \)

% Berechnung:

% \[
% I_B = \frac{32,2 mA}{50}
% \]

% \[
% I_B = 0,644 mA
% \]

% **Ergebnis**: Der Basisstrom \( I_B \) beträgt **0,644 mA**.

% ---

% ### Berechnung der Basiswiderstandsspannung \( U_V \)
% Der Basiswiderstand \( R_V = 3,3 kΩ \) wird mit \( I_B \) multipliziert:

% \[
% U_V = I_B \cdot R_V
% \]

% \[
% U_V = 0,644 mA \cdot 3,3 kΩ = 2,12V
% \]

% ---

% ### Berechnung der Eingangsspannung \( U \)
% Die Eingangsspannung \( U \) setzt sich zusammen aus:

% \[
% U = U_V + U_{BE}
% \]

% Die Basis-Emitter-Spannung eines **leitenden npn-Transistors** beträgt etwa:

% \[
% U_{BE} = 0,6V
% \]

% \[
% U = 2,12V + 0,6V = 2,72V
% \]

% **Ergebnis**: Die Eingangsspannung \( U \) beträgt **2,72V**.
% }

}



