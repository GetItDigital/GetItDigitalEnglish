\subsection{Gleichrichterschaltung\label{Aufg28}}
\Aufgabe{
  Hier eine Gleichrichterschaltung mit pn-Silizium-Dioden. 
  Die Spannungsquelle $V_1$ liefert sinusförmige $50 \, \mathrm{Hz}$ Wechselspannung mit einer Amplitude von $5\, \mathrm{V}$.
  \begin {figure} [H]
  \centering
  \begin{tikzpicture}
    \draw (-1,0) 
    to[D,l=$D_\mathrm{2}$] (-1,2) -- (-1,4)
    to[D,l=$D_\mathrm{1}$] (-1,6) -- (1,6)
    to[C,l=$C_\mathrm{1}$] (1,4) -- (1,2)
    to[C,l=$C_\mathrm{2}$] (1,0) -- (-1,0);

    \draw (1,0)
    to[short, *-] (3,0)
    to[R, l_=$R_2$] (3,2) -- (3,4)
    to[R, l_=$R_1$] (3,6);  

    \draw (3,0)
    to[short, *-] (5.5,0)
    to[R, l_=$R_3$] (5.5,6)
    to[short, -*] (3,6)
    to[short, -*] (1,6);

    \draw (3,2)
    to[short, *-] (1,2)
    to[short, *-*] (-3,2)    
    node[ground]{}
    to[sV, name=UE, v<] (-3,4)
    to[short, -*] (-1,4); 

    \varrmore{UE}{$u_\mathrm{E}(t)$};
\end{tikzpicture}
  \label{fig:FigKlausurGleichrichter}
  \end {figure}
  \begin{itemize}
    \item[a)]
          Welche Gleichspannungen treten über $R_1$,$R_2$ und $R_3$ auf?
    \item[b)]
          Welche Funktionen haben die Kondensatoren $C_1$ und $C_2$? 
    \item[c)]
          Was ist der besondere Vorteil dieser Gleichrichterschaltung, z.B. für Anwendungen mit Operationsverstärkern?
    \item[d)]
          Skizzieren Sie eine andere Spannungs-Verdopplungs-Schaltung, bei der die maximale positive Gleichsspannung (Ausgangsspannung) gegenüber Massepotential der speisenden Wechselspannung $V_1$ zur Verfügung steht. 
  \end{itemize}
  
  
}

\Loesung{
  \begin{itemize}
    \item[a)]
  \end{itemize}
  \begin{itemize}
    \item $U_\mathrm{R1} = 4,3 \, \mathrm{V} \Rightarrow$ da $0,7 \,\mathrm{V}$ über $D_1$ abfallen \\
    \item $U_\mathrm{R2} = -4,3 \, \mathrm{V} \Rightarrow$ da gleiches für die negative Halbwelle \\
    \item $U_\mathrm{R3} = 8,6 \,\mathrm{V} \Rightarrow$ da $R_3$ beide Spannungswerte aufnimmt \\
  \end{itemize}
  \begin{itemize}
    \item[b)]
  \end{itemize}
  \begin{center}
    Die Kondensatoren $C_1$ und $C_2$ haben einerseits die Funktionen die Signale zu glätten, sowohl von der negativen als auch der positiven Halbwelle. 
  \end{center}
  \begin{itemize}
    \item[c)]
  \end{itemize}
  \begin{center}
    Der Vorteil dieser Gleichrichtung ist, das sowohl positive als auch negative Spannungen abgegriffen werden können.
  \end{center}
  \begin{itemize}
    \item[d)]
  \end{itemize}
  \begin {figure} [H]
  \centering
  \begin{tikzpicture}
    \draw (0,0) to[short, o-] (1,0) to[C,l=$C_\mathrm{1}$](2,0) to[short, *-] (2,0) to[D,l=$D_\mathrm{2}$, -*] (4,0) to[short, -o] (6,0);
    \draw (0,-2) to[short, o-] (1,-2) -- (3,-2)node[ground]{};
    \draw (1.5,-2) to[short, -*] (2,-2) to[short, -*] (4,-2) to[short, -o] (6,-2);
    
    
    \draw (6,0) to[open,v>, name=ua] (6,-2);
    
    
    \draw (2,-2) to[D,l=$D_\mathrm{1}$](2,0);
    \draw (4,-2) to[C,l=$C_\mathrm{2}$](4,0);
    
    
    \draw (0,0) to[open, v>, name=ue] (0,-2);
    
    \varrmore{ua}{$U_\mathrm{a}$};
    \varrmore{ue}{$U_\mathrm{e}$};
    
    \end{tikzpicture}
  \label{fig:LsgKlausurGleichrichter}
  \end {figure}
  Siehe: Abschnitt \ref{fig:BrueckengleichrichterSchaltung}
}