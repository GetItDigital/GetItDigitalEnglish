

\subsection{RC-Tiefpass 1. Ordnung}\label{ueb:rctiefpass}
% Übernommen von Prof. Exknowski, FH SWF Hagen, 2024, ET3 (Übung 9, Aufgabe 2), leicht verändert
\Aufgabe{
Die Ausgangsspannung eines RC-Tiefpasses soll bei der Frequenz $f=1\,\mathrm{kHz}$ nur noch $10\,\%$ der Eingangspannung betragen.
Wie groß ist die Grenzfrequenz zu wählen?
}

\Loesung{
%F(jw)=Ua/Ue=1/(1+jwRC)
%A(w)=|F(jw)|=1/sqrt(1+(wRC)^2)
%A(w_g)=1/sqrt(1+(w_gRC)^2)=1/sqrt(2) => (w_gRC)^2=1 => w_g=1/RC
%A(w)=0,1 => sqrt(1+(wRC)^2)=10 => (wRC)^2=99 => w_g=w/sqrt(99)
%fg=f/sqrt(99)=1kHz/sqrt(99)=100,5Hz
\begin{align*}
    \underline{F}(\mathrm{j}\omega)&=\frac{\underline{U}_a}{\underline{U}_e} = \frac{\frac{1}{j\omega C}}{\frac{1}{j\omega C} + R} = \frac{1}{1 + j\omega RC}\\
    A(\omega)&=\frac{U_a}{U_e} = \frac{1}{\sqrt{1+(\omega RC)^2}}\\
    A(\omega_g)&=\frac{1}{\sqrt{1+(\omega_g RC)^2}} \overset{!}{=} \frac{1}{\sqrt{2}} \quad \Rightarrow \quad \omega_g = \frac{1}{RC} \\
    A(\omega)&\overset{!}{=} 0,1 \quad \Rightarrow \quad \sqrt{1+(\omega RC)^2} \overset{!}{=} 10\\
    \Leftrightarrow\quad 1 + (\omega RC)^2 &= 100 \quad \Leftrightarrow\quad (\omega RC)^2 = 99 \quad \Rightarrow\quad\omega_g = \frac{\omega}{\sqrt{99}}\\
    \Rightarrow\quad f_g &= \frac{f}{\sqrt{99}} = \frac{1\,\mathrm{kHz}}{\sqrt{99}} = 100,5\,\mathrm{Hz}
\end{align*}
}

%%%%%%%%%%%%%%%%%%%%%%%%%%%%%%%%%%%%%%%%%%%%%%%%%%%%%%%%%%%%%%%%%%%%%%%%%%%%%%%%%%%%%%%%

\subsection{RC-Hochpass 1. Ordnung}\label{ueb:rchochpass}
% Übernommen von Prof. Exknowski, FH SWF Hagen, 2024, ET3 (Übung 9, Aufgabe 3), leicht verändert
\Aufgabe{
Wie groß ist die Phasenverschiebung $\varphi$ bei einem RC-Hochpass 1. Ordnung, wenn die Sperrdämpfung (Verhältnis von Ausgangs- zu Eingangsspannung) $-6dB$ beträgt?
}

\Loesung{
%F(jw)=Ua/Ue=jwRC/(1+jwRC)
%A(w)=|F(jw)|=wRC/sqrt(1+(wRC)^2)
\begin{align*}
    -6\,\dB &= 20\cdot \lg\left(\frac{U_a}{U_e}\right)\,\dB \\
    \Rightarrow\quad -0,3 &= \lg\left(\frac{U_a}{U_e}\right)\\
    \Rightarrow\quad\frac{U_a}{U_e} &= 10^{-0,3} = 0,501\\
    \frac{\underline{U}_a}{\underline{U}_e} &= \frac{R}{R+\frac{1}{j\omega C}} = \frac{1}{1-j\frac{1}{\omega CR}}\\
    \frac{U_a}{U_e} &= \frac{1}{\sqrt{1+\frac{1}{(\omega CR)^2}}} \overset{!}{=} 0,501\\
    \Leftrightarrow\quad \sqrt{1 + \frac{1}{(\omega CR)^2}} &= \frac{1}{0,501}\\
    \Leftrightarrow\quad \frac{1}{(\omega CR)^2} &= 2,981 \quad\Leftrightarrow\quad \frac{1}{\omega CR} = \sqrt{2,981} = 1,727\\
    \frac{\underline{U}_a}{\underline{U}_e} &= \frac{1+j\frac{1}{\omega CR}}{1+\frac{1}{(\omega CR)^2}}\\[0.75em]
    \varphi &= \arctan\left(\frac{1}{\omega CR}\right) = \arctan\left(\sqrt{2,981}\right) = 59,92\degree
\end{align*}
}

%%%%%%%%%%%%%%%%%%%%%%%%%%%%%%%%%%%%%%%%%%%%%%%%%%%%%%%%%%%%%%%%%%%%%%%%%%%%%%%%%%%%%%%%

\subsection{Frequenzgang eines RLC-Serienschwingkreises\label{ueb:rlcs:schwingkreis}}
\Aufgabe{
% Übernommen von Prof. Exknowski, FH SWF Hagen, 2024, ET3 (Übung 4, Aufgabe 1), leicht verändert, Vgl. Prof. Ackers Übung 4.4
Gegeben ist die rechts dargestellte Schaltung mit $L = 20\,\mathrm{mH}$ , einem Widerstand $R = 150\,\Omega$ und einem Kondensator mit $ C = 1 \mu \mathrm{F}$.\\
\begin{minipage}{0.45\textwidth}
\begin{itemize}
    \item[a)] Geben Sie die Resonanzfrequenz der Schaltung an. Handelt es dich um Strom- oder Spannungsresonanz?
    \item[b)] Leiten Sie den Frequenzgang $\frac{\underline{U}_2}{\underline{U}_1}$ allgemein her.
\end{itemize}
\end{minipage}\hspace{0.05\textwidth}%
\begin{minipage}{0.5\textwidth}\centering%
    \includegraphics{Tikz/pdf/Uebungen/circ_uebung_rlcs_schwingkreis.pdf}% RLC-Reihenschaltung, u1 über RLC, u2 über C, Vgl. Prof. Ackers Übung 4.4
\end{minipage}
\begin{itemize}
    \item[c)] Berechnen und skizzieren Sie den Betrag des Frequenzganges $\frac{\underline{U}_2}{\underline{U}_1}$ in halblogarithmischer Darstellung. Geben Sie den Betrag für $\omega= 0, 100\,\mathrm{s}^{-1}$, $2000\,\mathrm{s}^{-1}$, $3500\,\mathrm{s}^{-1}$, $\omega_0$ , $10000\,\mathrm{s}^{-1}$ und $\infty$ an.
\end{itemize}
\begin{itemize}
    \item [d)]Kommt es bei einer Induktivität von 10 mH im Resonanzfall zu einer Spannungsüberhöhung?
\end{itemize}
}

\Loesung{
    % Übernommen von Prof. Exknowski, FH SWF Hagen, 2024, ET3 (Übung 4, Aufgabe 1), leicht verändert, Vgl. Prof. Ackers Übung 4.4
\begin{itemize}
    \item[a)] Die Resonanzkreisfrequenz $\omega_0$ ergibt sich durch die Resonanzbedingung $\Im\{\underline{Z}=0\}$.
    \begin{align*}
        \underline{Z} &= R + \mathrm{j}\omega L + \frac{1}{\mathrm{j} \omega C} = R + \mathrm{j}\cdot\left(\omega L + \frac{1}{\omega C}\right)
        &&\Rightarrow\quad \omega_0 L - \frac{1}{\omega_0 C} \overset{!}= 0 \\
        \Rightarrow\quad\omega_0 &= \frac{1}{\sqrt{LC}} = \frac{1}{\sqrt{20\,\mathrm{mH} \cdot 1\,\mu\mathrm{F}}} = 7071,07\,\mathrm{s}^{-1} 
        &&\Rightarrow\quad f_0 = \frac{\omega_0}{2\pi} = 1125,4\,\mathrm{Hz}
    \end{align*}
    Da die Schaltung eine Reihenschaltung aus einem Widerstand, Kondensator und einer Spule ist, handelt es sich um eine \underline{Spannungsresonanz}.

    \item[b)] Die Spannungsteilerregel liefert den Frequenzgang:
    \begin{align*}
        \frac{\underline{U}_2}{\underline{U}_1} = \underline{F}(\mathrm{j}\omega) 
            &= \frac{\underline{Z}_{\mathrm{C}}}{\underline{Z}_{\mathrm{R}} + \underline{Z}_{\mathrm{L}} + \underline{Z}_{\mathrm{C}}} \\ 
            &= \frac{\frac{1}{\mathrm{j}\omega C}}{R + \mathrm{j}\omega L + \frac{1}{\mathrm{j}\omega C}} &&\bigg| \cdot \frac{\mathrm{j}\omega C}{\mathrm{j}\omega C} \\ 
            &= \frac{1}{\mathrm{j}\omega CR + \mathrm{j}^2 \omega^2 LC + 1} \\
            &= \frac{1}{1 - \omega^2 LC + \mathrm{j} \omega R C}
    \end{align*}

    \item[c)] Der Betrag des Frequenzganges (Amplitudengang $A(\omega)$) ergibt sich zu:
    \begin{align*}
        \left| \frac{\underline{U}_2}{\underline{U}_1} \right| = A(\omega) &= \frac{1}{\sqrt{(1-\omega^2 LC)^2 + (\omega R C)^2 }}
    \end{align*}
    Der Betrag des Frequenzganges ist in folger Abbildung dargestellt.\\
    \includegraphics{Tikz/pdf/Uebungen/plot_uebung_rlcs_schwingkreis_amplitudengang.pdf}

    Die zugehörigen Werte sind:\\
    % A(100)=1.000
    % A(1000)=1.009
    % A(2000)=1.033
    % A(3500)=1.087
    % A(7071)=0.943
    % A(10000)=0.555
    % A(w->inf)=0
    \begin{tabular}{cccccccc}
    \toprule
        $\omega$ & $0$ & $100\,\mathrm{s}^{-1}$ & $2000\,\mathrm{s}^{-1}$ & $3500\,\mathrm{s}^{-1}$ & $\omega_0$ & $10000\,\mathrm{s}^{-1}$ & $\omega\to\infty$\\
    \midrule
        $A(\omega)$ & $1,000$ & $1,009$ & $1,033$ & $1,087$ & $0,943$ & $0,555$ & $0$ \\
    \bottomrule
    \end{tabular}    
    \item[d)] Die Güte eines Reihenschwingkreises gibt die Spannungsüberhöhung über $L$ beziehungsweise $C$ 
    im Resonanzfall an. 
    \begin{align*}
        \frac{U_2(\omega_0)}{U_1(\omega_0)} = Q_S &= \frac{X_k}{R} = \frac{\omega_0 L}{R} = \frac{1}{R}\cdot\sqrt{\frac{L}{C}}\\
        &= \frac{1}{150\,\Omega}\cdot\sqrt{\frac{10\,\mathrm{mH}}{1\,\mu\mathrm{F}}}= 0,\overline{66} % 0,666... 
    \end{align*}
    Für $L=10\,\mathrm{mH}$ ist $Q_S < 1$, es kommt also zu keiner Spannungsüberhöhung im Resonanzfall.
\end{itemize}
}      

%%%%%%%%%%%%%%%%%%%%%%%%%%%%%%%%%%%%%%%%%%%%%%%%%%%%%%%%%%%%%%%%%%%%%%%%%%%%%%%%%%%%%%%%

\subsection{Frequenzgang eines Vierpols aus R, L und C}\label{ueb:rrlc:frequenzgang}
\Aufgabe{
% Übernommen von Prof. Exknowski, FH SWF Hagen, 2024, ET3 (Übung 4, Aufgabe 2), leicht verändert, Vgl. Prof. Ackers Übung 4.5
Die rechts dargestellte Schaltung soll untersucht werden.
Hierbei sollen für die Bauelemente die allgemeinen Werte
$\mathrm{R}_1$, $\mathrm{R}_2$, L, C verwendet werden.
% Leerzeile

\begin{minipage}{0.45\textwidth}
\begin{itemize}
    \item[a)] Bestimmen Sie das Spannungsteilerverhältnis
    \begin{align*}
        \underline{F}(\mathrm{j}\omega) = \frac{\underline{U}_2}{\underline{U}_1}
    \end{align*}
    \item[b)] Berechnen und skizzieren Sie den Betrag und die Phase von $\underline{F}(\mathrm{j}\omega)$. Kennzeichnen
    Sie charakteristische Punkte.
    \item[c)] Berechnen Sie die Ausgangsspannung $\mathrm{u}_2 (t)$, wenn für die Eingangsspannung
    \begin{align*}
        \mathrm{u}_1 (t) = 5 \mathrm{V} \cdot \mathrm{cos} (2\pi \cdot 10 \mathrm{kHz} \cdot t)
    \end{align*}
        gilt.
\end{itemize}
\end{minipage}%
\begin{minipage}{0.55\textwidth}\centering
    \includegraphics{Tikz/pdf/Uebungen/circ_uebung_vierpol_r_lcp_r.pdf}%  u1 über R1+(L||C)+R2, u2 über R2, Vgl. Prof. Ackers Übung 4.5
\end{minipage}
}

\Loesung{
% Übernommen von Prof. Exknowski, FH SWF Hagen, 2024, ET3 (Übung 4, Aufgabe 2), Vgl. Prof. Ackers Übung 4.5
\begin{itemize}
    \item[a)] Das Spannungsteilerverhältnis ergibt sich zu:
    \begin{align*}
        \underline{F}(\mathrm{j}\omega) & = \frac{\underline{U}_2}{\underline{U}_1} \\
        &= \frac{R_2}{R_1 + \underline{Z}_L || \underline{Z}_C + R_2} 
        &&\text{mit}\quad \underline{Z}_L || \underline{Z}_C = \frac{\mathrm{j}\omega L \cdot \frac{1}{\mathrm{j}\omega C}}{\mathrm{j}\omega L - \mathrm{j}\frac{1}{\omega C}} = -\mathrm{j}\cdot\left(\frac{1}{\omega C - \frac{1}{\omega L}}\right)\\
        &= \frac{R_2}{R_1 + R_2 - \mathrm{j}\left(\frac{1}{\omega C - \frac{1}{\omega L}}\right)}
    \end{align*}

    \item[b)] Der Betrag des Spannungsteilerverhältnisses ergibt sich allgemein zu:
    \begin{equation*}
        \left| \underline{F}(\mathrm{j}\omega) \right| = A(\omega) 
        = \frac{\mathrm{R}_2}{\sqrt{(\mathrm{R}_1 + \mathrm{R}_2)^2 + \left(\frac{1}{\omega \mathrm{C} - \frac{1}{\omega \mathrm{L}}}\right)^2}} 
    \end{equation*}
    Die Parallelschaltung aus $L$ und $C$ verhält sich für $\omega \to 0$ und für $\omega \to \infty$ wie ein Kurzschluss, 
    da jeweils eins der beiden Bauelemente in beiden Fällen eine Impedanz gegen Null besitzt.
    Daraus folgt für $\omega \to 0$ und $\omega \to \infty$:
    \begin{equation*}
        A(\omega \to 0) = A(\omega \to \infty) = \frac{R_2}{R_1+R_2}
    \end{equation*}
    Im Resonanzfall ($\omega = \omega_0 = \sqrt{\frac{1}{LC}}$) heben sich die Admittanzen von $L$ und $C$ (Parallel-Schwingkreis) gegenseitig auf. 
    Mit $Y_C + Y_L \to 0$ folgt auch $Z_L||Z_C \to \infty$ für $\omega \to \omega_0$.
    \begin{equation*}
        A(\omega_0) = \frac{R_2}{\sqrt{(\mathrm{R}_1 + \mathrm{R}_2)^2 + \infty^2}} = \frac{1}{\infty} = 0
    \end{equation*}

    Die Phase ergibt sich zu:
    \begin{align*}
        \varphi(\omega)&= 
            \underbrace{
                \arctan\left(\frac{0}{R_2}\right)
            }_{\varphi_{\text{Zähler}}} - 
            \underbrace{
                \arctan\left(\frac{-\left(\frac{1}{\omega C - \frac{1}{\omega L}}\right)}{R_1 + R_2}\right)
            }_{\varphi_{\text{Nenner}}}
            = \arctan\left(\frac{\frac{1}{\omega C - \frac{1}{\omega L}}}{R_1+R_2}\right)
    \end{align*}
    Mit:
    \begin{align*}
        \varphi(\omega \to 0) &= \arctan\left(0\right) = 0\\
        \varphi(\omega \to \infty) &= \arctan\left(0\right) = 0\\
        \varphi(\omega_0) &= \arctan\left(\infty\right) = \pm\frac{\pi}{2} \\
        \varphi(\omega < \omega_0) &< 0\\
        \varphi(\omega > \omega_0) &> 0
    \end{align*}

    Der Amplitudengang und Phasengang sind in folgendem Bode-Diagramm dargestellt.

    \vspace{0.5cm}
    \includegraphics{Tikz/pdf/Uebungen/plot_uebung_vierpol_r_lcp_r_bodediagramm.pdf}

    \item[c)] Die Ausgangsspannung ergibt sich zu:
    \begin{align*}
        \underline{U}_2 &= \underline{F}(\mathrm{j}\omega) \cdot \underline{U}_1 \\
        u_2(t) &=  A(\omega) \cdot 5\,\mathrm{V} \cdot \cos\left(\omega t + \varphi(\omega)\right) 
        &&\text{mit}\quad \omega = 2\pi\cdot10\,\mathrm{kHz}\cdot
    \end{align*}
\end{itemize}
}

%%%%%%%%%%%%%%%%%%%%%%%%%%%%%%%%%%%%%%%%%%%%%%%%%%%%%%%%%%%%%%%%%%%%%%%%%%%%%%%%%%%%%%%%

\subsection{RLC-Parallelschwingkreis}\label{ueb:rlcp:schwingkreis}
\Aufgabe{
% Übernommen von Prof. Exknowski, FH SWF Hagen, 2024, ET3 (Übung 5, Aufgabe 1), Vgl. Prof. Ackers Übung 5.4
\begin{minipage}{0.6\textwidth}
    Der rechts dargestellte Schwingkreis soll untersucht werden.
    \begin{itemize}%\itemsep1mm
        \item[a)] Bestimmen Sie die komplexe Gesamtimpedanz $\underline{Z}$ in allgemeiner Form.
        \item[b)] Skizzieren Sie den Betrag der komplexen Impedanz in linearer Darstellung und kennzeichnen Sie charakteristische Punkte.
    \end{itemize}
\end{minipage}
\hspace{0.1\textwidth}
\begin{minipage}{0.3\textwidth}\centering%
    \includegraphics{Tikz/pdf/Uebungen/circ_uebung_rlcp_schwingkreis.pdf}% RLC-Parallelschwingkreis, Ackers Übung 5.4
\end{minipage}
\begin{itemize}%\itemsep1mm
    \item[c)] Berechnen Sie die Resonanzfrequenz $f_0$, sowie die Frequenzen $f_{\mathrm{u}}$ und $f_{\mathrm{o}}$, bei denen der Betrag von $\underline{Z}(\omega)$ auf $\frac{Z_0}{\sqrt{2}}$ gefallen ist, wobei $Z_0$ der Betrag der Gesamtimpedanz im Resonanzfall ist. Bestimmen Sie auch die Differenz der beiden Frequenzen. Diese gibt die Bandbreite $B$ an. Für diesen Aufgabenteil gilt $R = 300\,\Omega$, $L = 200\,\mathrm{mH}$ und $C = 1\,\mu\mathrm{F}$.
    \item[d)] Die Bandbreite wird auch mit der Formel $B = \frac{f_0}{Q}$ bestimmt. Vergleichen Sie das Ergebnis mit dem aus Aufgabenteil c).
    \item[e)] Liegen die obere und die untere Grenzfrequenz symmetrisch um $f_0$?
\end{itemize}
}

\Loesung{
% Übernommen von Prof. Exknowski, FH SWF Hagen, 2024, ET3 (Übung 5, Aufgabe 1), Vgl. Prof. Ackers Übung 5.4
\begin{itemize}
    \item[a)] Die Gesamtimpedanz des Schwingkreises ergibt sich zu:
    \begin{align*}
        \underline{Y} &= \frac{1}{R} + \mathrm{j}\omega C + \frac{1}{\mathrm{j}\omega L} \\[2pt]
        \underline{Z} &= \frac{1}{\underline{Y}} = \frac{1}{\frac{1}{R} + \mathrm{j}\omega C + \frac{1}{\mathrm{j}\omega L}} \\[2pt]
        &= \frac{\frac{1}{R}-\mathrm{j}\cdot\left(\omega C - \frac{1}{\omega L}\right)}{\frac{1}{R^2} + \left(\omega C - \frac{1}{\omega L}\right)^2}
    \end{align*}
    \item[b)] Betrag der Gesamptimpedanz in halblogarithmischer Darstellung:
    \fu{\includegraphics{Tikz/pdf/Uebungen/plot_uebung_rlcp_schwingkreis_gesamtimpedanz.pdf}}{}% no caption/label
    Die Gesamptimpedanz ist maximal bei Resonanzkreisfrequenz mit $\underline{Z}(\omega_0) = R$.

    \item[c)] Resonanz:
    \begin{align*}
        \Im\{\underline{Y}\} = \omega_0 C - \frac{1}{\omega_0 L} \overset{!}{=} 0 \quad \Leftrightarrow \omega_0 &= \frac{1}{\sqrt{LC}} \quad \Rightarrow f_0 = \frac{\omega_0}{2\pi}\\
        |\underline{Z}(\omega_0)| = \frac{1}{%
            \sqrt{\frac{1}{R^2} + 
                \Big(\smash{\underbrace{
                    \omega_0 C - \frac{1}{\omega_0 L}
                }_{={}\mathrlap{0}}}\Big)^2
            }
        } &= \frac{1}{\sqrt{\frac{1}{R^2}}} = R
    \end{align*}% ev. Seitenumbruch
    Grenzfrequenzen, Bandbreite:
    \begin{align*}
        \mathllap{|\underline{Z}(\omega)| \overset{!}{=} \frac{R}{\sqrt{2}} \quad \Leftrightarrow \quad}
        \frac{1}{\sqrt{\frac{1}{R^2} + \left(\omega C - \frac{1}{\omega L}\right)^2}} &= \frac{R}{\sqrt{2}} \\
        \frac{R}{\sqrt{1 + R^2\cdot\left(\omega C - \frac{1}{\omega L}\right)^2}} &= \frac{R}{\sqrt{2}} \\
        1 + R^2\cdot\left(\omega C - \frac{1}{\omega L}\right)^2 &= 2 \\
        \left(\omega C - \frac{1}{\omega L}\right)^2 &= \frac{1}{R^2} \\
        \omega C - \frac{1}{\omega L} &= \pm\frac{1}{R} \\
        %\omega^2 LC - 1 \pm \omega \cdot \frac{L}{R} &= 0 \\ 
        \omega^2 LC \pm \omega \cdot \frac{L}{R} - 1 &= 0 \\
        \omega^2 \pm \omega \cdot \frac{1}{RC} - \frac{1}{LC} &= 0 \\
        x^2 + p \cdot x + q &= 0 \vphantom{\frac{1}{LC}}% p-q Form
    \end{align*}
    \begin{align*}
        x_{1,2} &= -\frac{p}{2} \pm \sqrt{\left(\frac{p}{2}\right)^2 - q} \\ % p-q Formel
        \Rightarrow \omega &= \pm \frac{1}{2 \cdot RC} \pm \sqrt{\frac{1}{(2 \cdot RC)^2} + \frac{1}{LC}}\\
        \frac{1}{2 \cdot RC} &= 1,6667 \cdot 10^3\ \mathrm{s}^{-1} \qquad
        \frac{1}{LC} = 5 \cdot 10^6\ \mathrm{s}^{-2} \\[4pt]
        \omega &= \pm 1,6667 \cdot 10^3\ \mathrm{s}^{-1} \pm 2,7889 \cdot 10^3\ \mathrm{s}^{-1} \\[4pt]
        \omega{g,o} &= 4,4556 \cdot 10^3 \mathrm{s}^{-1} \qquad f_{g,o} = 709,12\ \mathrm{Hz}\\
        \omega{g,u} &= 1,1222 \cdot 10^3 \mathrm{s}^{-1} \qquad f_{g,u} = 178,60\ \mathrm{Hz}\\[4pt]
        B &= f_{g,o} - f_{g,u} = 530,5\,\mathrm{Hz}
    \end{align*}
    \item[d)] Bestimmung der Bandbreite über die Resonanzfrequenz und die Güte:
    \begin{align*}
        B &= \frac{f_0}{Q} 
            & \text{mit}\quad Q &= \sqrt{\frac{C}{L}}\cdot R = \sqrt{\frac{1\ \mu\mathrm{F}}{200\ \mathrm{mH}}} \cdot 300\ \Omega = 0,6708 \\
        B &= \frac{355,88\ \mathrm{Hz}}{0,6708} = 530,5\ \mathrm{Hz}
            & \text{mit}\quad f_0 &= \frac{1}{2\pi\sqrt{LC}} = \frac{1}{2\pi\sqrt{200\ \mathrm{mH} \cdot 1\ \mu\mathrm{F}}} = 355,88\ \mathrm{Hz} \\ 
    \end{align*}
    \item[e)] Bei logarithmischer Frequenzachse verläuft die Kurve des Impedanzbetrags $|\underline{Z}(\omega)|$  
    symmetrisch um die Resonanzkreisfrequenz $\omega_0$, das heißt $f_{g,o}$ und $f_{g,u}$ liegen dann symmetrisch um $f_0$.
    \begin{equation*}
        \left(\omega C - \frac{1}{\omega L}\right)^2 = \bigg(\underbrace{
            \frac{\omega}{\omega_0} - \frac{\omega_0}{\omega}
        }_{=\vartheta(\omega)\mathrlap{,\ \vartheta^2\ \text{symmetrisch um }\omega_0\text{ bei faktorieller Änderung}}}
        \bigg)^2\cdot\frac{C}{L}
        \qquad \Longrightarrow \qquad \vartheta^2(n\cdot\omega_0) = \vartheta^2\left(\frac{1}{n}\cdot\omega_0\right)
    \end{equation*}
\end{itemize}
}

%%%%%%%%%%%%%%%%%%%%%%%%%%%%%%%%%%%%%%%%%%%%%%%%%%%%%%%%%%%%%%%%%%%%%%%%%%%%%%%%%%%%%%%%

\subsection{RC-Filter zweiter Ordnung}\label{ueb:rcfilter:zweiterordnung}
\Aufgabe{
% Übernommen von Prof. Exknowski, FH SWF Hagen, 2024, ET3 (Übung 5, Aufgabe 2)
Für die untenstehende hintereinandergeschalteten RC-Glieder gilt $R_1 = R_2 = R = 1000\,\Omega$ und $C_1 = C_2 = C = 1\,\mu\mathrm F$.
\begin{figure}[H]
    \centering
    \includegraphics{Tikz/pdf/Uebungen/circ_uebung_rc_filter_2_ord.pdf}% 2x RC-Tiefpass 1. Ordnung in Reihe
\end{figure}
\begin{itemize}%\itemsep1mm
    \item[a)] Bestimmen Sie den Frequenzgang der Schaltung $\underline{F}(\mathrm{j}\omega) = \frac{\underline{U}_2}{\underline{U}_1}$.
    \item[b)] Zeichnen Sie den Amplitudengang und den Phasengang in logarithmischer Darstellung. Vergleichen Sie das Ergebnis mit dem einfachen RC-Glied mit denselben Bauelementen.
\end{itemize}
}

\Loesung{
\begin{itemize}
    \item[a)] Frequenzgang:
    \begin{align*}
        \underline{F}(\mathrm{j}\omega) 
            &= \frac{\underline{U}_2}{\underline{U}_1} = \frac{\underline{U}_2}{\underline{U}_m} \cdot \frac{\underline{U}_m}{\underline{U}_1}\\
        \frac{\underline{U}_m}{\underline{U}_1} 
            &= \frac{\underline{Z}_{\mathrm{ers}}}{R + \underline{Z}_{\mathrm{ers}}} = \frac{1}{\frac{R}{\underline{Z}_{\mathrm{ers}}} + 1}\\
        \underline{Z}_{\mathrm{ers}} 
            &= \left( R + \frac{1}{\mathrm{j}\omega C} \right) \bigg|\bigg|\, \frac{1}{\mathrm{j}\omega C} 
            %= \left(\frac{ \left( R + \frac{1}{\mathrm{j}\omega C} \right) + \frac{1}{\mathrm{j}\omega C}}{ \left( R + \frac{1}{\mathrm{j}\omega C} \right) \cdot \frac{1}{\mathrm{j}\omega C}} \right)^{-1}
            = \left( \frac{1}{R + \frac{1}{\mathrm{j}\omega C}} + \mathrm{j}\omega C\right)^{-1}\\
        \frac{\underline{U}_m}{\underline{U}_1} 
            &= \frac{1}{R \cdot \left(\frac{1}{R + \frac{1}{\mathrm{j}\omega C}} + \mathrm{j}\omega C\right) + 1} \\
            &= \frac{1}{\frac{R\cdot\mathrm{j}\omega C}{R\cdot\mathrm{j}\omega C + 1} + R\cdot\mathrm{j}\omega C + 1} \\
            &= \frac{R\cdot\mathrm{j}\omega C + 1}{R\cdot\mathrm{j}\omega C + \left( R\cdot\mathrm{j}\omega C + 1 \right)^2} \\
            &= \frac{R\cdot\mathrm{j}\omega C + 1}{R\cdot\mathrm{j}\omega C - R^2\omega^2 C^2 + 2\cdot\mathrm{j}\omega C R + 1 } \\
            &= \frac{1}{1 - \omega^2 C^2 R^2 + \mathrm{j}3\omega C R} \\
            &= \frac{1 + \mathrm{j}\Omega}{1 - \Omega^2 + \mathrm{j}3\Omega}
            \qquad \text{mit} \qquad \Omega = \omega C R\\[4pt]
        \frac{\underline{U}_2}{\underline{U}_m}
            &= \frac{\frac{1}{\mathrm{j}\omega C}}{R + \frac{1}{\mathrm{j}\omega C}} 
            = \frac{1}{1 + \mathrm{j}\omega C R} = \frac{1}{1 + \mathrm{j}\Omega}\\[4pt]
        \frac{\underline{U}_2}{\underline{U}_1} 
            &= \frac{\underline{U}_2}{\underline{U}_m} \cdot \frac{\underline{U}_m}{\underline{U}_1} 
            = \frac{1}{1 + \mathrm{j}\omega C R} = \frac{1}{1 + \mathrm{j}\Omega} \cdot \frac{1 + \mathrm{j}\Omega}{1 - \Omega^2 + \mathrm{j}3\Omega} \\
            &= \frac{1}{1 - \Omega^2 + \mathrm{j}3\Omega} = \frac{1}{1 - (\omega C R)^2 + \mathrm{j}3\omega C R}
    \end{align*}
    Vergleich einfacher Tiefpass:
    \begin{align*}
        \underline{F}(\mathrm{j}\omega) = \frac{\frac{1}{\mathrm{j}\omega C}}{R + \frac{1}{\mathrm{j}\omega C}} = \frac{1}{1 + \mathrm{j}\omega C R} = \frac{1}{1 + \mathrm{j}\Omega}
    \end{align*}
    Wegen Kopplungseffekten kann der Frequenzgang des einfachen Tiefpasses bei Serienschaltung 
    nicht einfach multipliziert werden.

    \item[b)] Bode-Diagramm (Amplituden- und Phasengang) für RC-Tiefpassfilter 1. und 2. Ordnung:
    \begin{figure}[H]\centering
        \includegraphics{Tikz/pdf/Uebungen/plot_uebung_rc_filter_2_ord_bodediagramm.pdf}% RC-Tiefpass Vgl. 1. Ordnung, 2. Ordnung
    \end{figure}
    Die Dämpfung im Sperrbereich beträgt bei 2. Ordnung (blau) das doppelte ($40\,\dB/\mathrm{Dek}$) 
    im Vergleich zur 1. Ordnung (rot) ($20\,\dB/\mathrm{Dek}$). 
    Die Phasenverschiebung reicht von $0\degree$ bis $-180\degree$ bei 2. Ordnung, 
    im Gegensatz zur 1. Ordnung, die nur von $0\degree$ bis $-90\degree$ reicht.
    Die Grenzfrequenz verschiebt sich durch die höhere Dämpfung nach links (kleiner für 2. Ordnung).
\end{itemize}
}

%%%%%%%%%%%%%%%%%%%%%%%%%%%%%%%%%%%%%%%%%%%%%%%%%%%%%%%%%%%%%%%%%%%%%%%%%%%%%%%%%%%%%%%%

\subsection{Impedanz- und Admittanzortskurve eines RL-Gliedes\label{ueb:ortskurve:rl}}
\Aufgabe{
% Übernommen von Prof. Exknowski, FH SWF Hagen, 2024, ET3 (Übung 6, Aufgabe 1), leicht verändert
Gegeben ist die Serienschaltung aus einer Induktivität $L = 10\,\mathrm{mH}$ und einem Widerstand $R = 5\,\mathrm{k}\Omega$ bei einer variablen Frequenz mit $f_\mathrm{min} = 5\,\mathrm{kHz}$ und $f_\mathrm{max} = 50\,\mathrm{kHz}$.
\begin{itemize}
    \item[a)] Bestimmen Sie die Impedanzortskurve und stellen Sie diese quantitativ dar. Kennzeichnen Sie den Parameter an mindestens drei sinnvollen Stellen.
    \item[b)] Zeichnen Sie die invertierte Ortskurve (Admittanzortskurve). Überlegen Sie sich zunächst den prinzipiellen Verlauf und benutzen Sie dann einige Zahlenwerte aus dem vorherigen Aufgabenteil a) für die Bestimmung der Ortskurvenpunkte.
\end{itemize}
}

\Loesung{
% Übernommen von Prof. Exknowski, FH SWF Hagen, 2024, ET3 (Übung 6, Aufgabe 1), leicht verändert
\begin{minipage}{0.58\textwidth}
    a) \hspace{2cm}%
    % Schaltbild: RL in Serie
    \includegraphics{Tikz/pdf/Uebungen/circ_uebung_impedanzortskurve_admittanzortskurve_rls.pdf}%
    % Rechnung
    \begin{align*}
        f&=5\,\mathrm{kHz} ... 50\,\mathrm{kHz} & \underline{Z} &= R + \mathrm{j}\omega L \\
        f_1 &=  5\,\mathrm{kHz} & \underline{Z}_1 &= 5\,\mathrm{k}\Omega + \mathrm{j} 314,16\,\Omega \\ 
        f_2 &= 25\,\mathrm{kHz} & \underline{Z}_2 &= 5\,\mathrm{k}\Omega + \mathrm{j} 1570,8\,\Omega \\ 
        f_3 &= 50\,\mathrm{kHz} & \underline{Z}_3 &= 5\,\mathrm{k}\Omega + \mathrm{j} 3141,6\,\Omega    
    \end{align*}

$\underline{Z}$-Ortskurve: $\Re\{\underline{Z}\}=konst.,\quad \Im\{\underline{Z}\}=var.\\ 
\Rightarrow$ Gerade parallel zur Imaginärachse im I. Quadranten.\\

b) %
Inversion von Geraden im I. Quadranten: \\ Halbkreis im IV. Quadranten.     
\begin{align*}
    \underline{Y}&=\frac{1}{\underline{Z}} & 
    \underline{Y}&= \frac{1}{|\underline{Z}|} \cdot \mathrm{e}^{-\mathrm{j}\varphi_Z}
\end{align*}
Maßstab so gewählt, dass $\underline{Z}(0) = 5\,\mathrm{k}\Omega {}\hat{=}{} \underline{Y}(0) = 0,2\,\mathrm{mS}$. \\

Start des Halbkreises bei $0,2\mathrm{mS}$ ($\omega=0$), Ende bei $0\,\mathrm{mS}$ ($\omega \to \infty$) und Mittelpunkt bei $0,1\,\mathrm{mS}$.\\

Die Admittanzen sind abzulesen am jeweiligen Schnittpunkt des gespiegelten Impedanzzeiger mit dem Halbkreis (Winkel negiert, Spiegelung an x-Achse).\\

Die Admittanzkurve entspricht dem Kreisabschnitt zwischen $\underline{Y}(f=5\,\mathrm{kHz})$ und $\underline{Y}(f=50\,\mathrm{kHz})$.
\end{minipage}%
\begin{minipage}{0.38\textwidth}
    \includegraphics{Tikz/pdf/Uebungen/plot_uebung_impedanzortskurve_admittanzortskurve_rls.pdf}
\end{minipage}
}

%%%%%%%%%%%%%%%%%%%%%%%%%%%%%%%%%%%%%%%%%%%%%%%%%%%%%%%%%%%%%%%%%%%%%%%%%%%%%%%%%%%%%%%%

\subsection{Admittanzortskurve gemischter Reihen-/Parallelschaltung}\label{ueb:ortskurve:rrc}
\Aufgabe{
\begin{minipage}{0.7\textwidth}
    Die Admittanz der rechts gezeigten Schaltung soll untersucht werden.\\[5pt]
    Parallel zu einer Reihenschaltung aus dem Widerstand $R_1$ und der Kapazität $C$ 
    befindet sich ein zweiter Widerstand $R_2$.\\[5pt]
    Die Bauteilwerte sind gegeben durch:\\[2pt] % R1=p*R0, R0=5 Ohm, R2=20 Ohm, omega*C=5 Ohm
    $R_1(p) = p \cdot R_0$,\quad $R_0 = 5\,\Omega$,\quad $R_2 = 20\,\Omega$\quad und \quad $\omega C = 5\,\Omega$
\end{minipage}%
\begin{minipage}{0.3\textwidth}
    % Schaltbild: R1 in Reihe zu C, zusammen parallel zu R2
    \includegraphics{Tikz/pdf/Uebungen/circ_uebung_admittanzortskurve_rcs_rp.pdf}% 
\end{minipage}
\begin{itemize}
    \item[a)] Bestimmen Sie die Admittanz für $p=0$, $p=1$, $p=2$ und $p\to\infty$.
    \item[b)] Skizzieren Sie den Verlauf der Admittanzortskurve und zeichnen Sie die Werte ein. 
    (Empfohlener Darstellungsbereich: $0 + \mathrm{j}0$ bis $0,25\,\mathrm{S}+ \mathrm{j}0,25\,\mathrm{S}$)
    \item[c)] Für welchen Wert von $p$ eilt der Gesamtstrom $\underline{I}$ einer angelegten Wechselspannung $\underline{U}$ um $45\degree$ voraus?
    Bestimmen Sie zeichnerisch den Wert für $\underline{Y}$. 
    \item[d)] Überprüfen Sie das Ergebnis aus c) rechnerisch.
\end{itemize}
}

\Loesung{
\begin{itemize}
    \item[a)] Admittanz $\underline{Y}$ für $p=0$, $p=1$, $p=2$ und $p\to\infty$:
\end{itemize}
\begin{minipage}{0.4\textwidth}
    \begin{align*}
        \underline{Y} &= \frac{1}{R_2} +  \frac{1}{R_1 + \frac{1}{\mathrm{j}\omega C}} \\[2pt]
        &= \frac{1}{R_2} + \frac{R_1 - \frac{1}{\mathrm{j}\omega C}}{R_1^2 - \left(\frac{1}{\mathrm{j}\omega C}\right)^2} \\[2pt]
        &= \frac{1}{R_2} + \frac{p \cdot R_0 + \mathrm{j} \frac{1}{\omega C}}{p^2 \cdot R_0^2 + \left(\frac{1}{\omega C}\right)^2} \\[2pt]
        &= \frac{1}{20\,\Omega} + \frac{p \cdot 5\,\Omega + \mathrm{j} \cdot 5\,\Omega}{p^2 \cdot 25\,\Omega^2 + 25\,\Omega^2} \\[2pt]
        &= 0,05\,\mathrm{S} + \frac{p + \mathrm{j}}{p^2 + 1} \cdot 0,2\,\mathrm{S}
    \end{align*}
\end{minipage}%
\begin{minipage}{0.6\textwidth}
    \begin{align*}
        \underline{Y}(p) &= \left(0,05 + \frac{p + \mathrm{j}}{p^2 + 1} \cdot 0,2\right)\,\mathrm{S}\\
        \underline{Y}(p \to 0) &= \left(0,05 + \frac{0 + \mathrm{j}}{0 + 1} \cdot 0,2\right)\,\mathrm{S} = (0,05 + \mathrm{j}\cdot 0,2)\,\mathrm{S} \\
        \underline{Y}(p = 1) &= \left(0,05 + \frac{1 + \mathrm{j}}{2} \cdot 0,2\right)\,\mathrm{S} = (0,15 + \mathrm{j} \cdot 0,1)\,\mathrm{S}\\
        \underline{Y}(p = 2) &= \left(0,05 + \frac{2 + \mathrm{j}}{5} \cdot 0,2\right)\,\mathrm{S} = (0,13 + \mathrm{j} \cdot 0,04)\,\mathrm{S}\\
        \underline{Y}(p \to \infty) &= \left(0,05 + \frac{\infty + \mathrm{j}}{\infty^2 + 1} \cdot 0,2\right)\,\mathrm{S} = (0,05 + \mathrm{j} \cdot 0)\,\mathrm{S}
    \end{align*}
\end{minipage}\hfill%

\begin{minipage}{0.4\textwidth}
\begin{itemize}
\item[b)] Skizze der Admittanzortskurve:
\item[] \includegraphics{Tikz/pdf/Uebungen/plot_uebung_admittanzortskurve_rcs_rp.pdf}
\item[c)] Strom eilt um $45\degree$ voraus für $\varphi=45\degree$.
    Schnittpunkt aus Gerade und Ortskurve.
\end{itemize}
\end{minipage}%
\begin{minipage}{0.6\textwidth}
\begin{itemize}
    \item[d)] Mit $\arctan(45\degree)=\frac{\Im\{\underline{Y}\}}{\Re\{\underline{Y}\}} = 1$ folgt:
    \begin{align*}
        \Re\{\underline{Y}\} &\overset{!}{=} \Im\{\underline{Y}\} \\
        0,05 + 0,2 \cdot \frac{p}{p^2+1} &\overset{!}{=} 0,2 \cdot \frac{1}{p^2+1}\\ 
        0,05 \cdot p^2 + 0,2 \cdot p - 0,15 &= 0 \\
        p_{1/2} &= -2 \pm \sqrt{4 + 3} \\
        \text{mit }R_1>0: \qquad p_{45\degree} &= -2 + \sqrt{7} \\
        \underline{Y}(\varphi=45\degree) &= 0,05 + 0,2 \cdot \frac{-2 + \sqrt{7} + \mathrm{j}}{(-2 + \sqrt{7})^2 + 1} \\
        &= (0,14114 + \mathrm{j} \cdot 0,14114)\,\mathrm{S}
    \end{align*}
\end{itemize}
\end{minipage}\hfill
}

%%%%%%%%%%%%%%%%%%%%%%%%%%%%%%%%%%%%%%%%%%%%%%%%%%%%%%%%%%%%%%%%%%%%%%%%%%%%%%%%%%%%%%%%

\subsection{Wechselstrommessbrücke}
\Aufgabe{
% Schaltung: U0 an Halbbrücke
% U0 über Z1 in Reihe mit Z2, beide parallel zu Z3 in Reihe mit Z4.
% Messspannung von mittig zwischen Z1 und Z2 über offen bis mittig zwischen Z3 und Z4
\begin{minipage}{0.5\textwidth}%
    \begin{itemize}
        \item[a)] Bestimmen Sie allgemein die Gleichung für die Messspannung $\underline{U}_B$. 
            Bringen Sie das Ergebnis auf einen gemeinsamen Nenner und kürzen Sie, wenn möglich.
        \item[b)] Im Ausschlagsverfahren soll ein Blindwiderstand gemessen werden. 
            Im gegebenen Messbereich seien $\underline{Z}_1=\mathrm{j}X_1$ und $\underline{Z}_2=\mathrm{j}X_2$ 
            (reine Blindwiderstände) und $\underline{Z}_3 = \underline{Z}_4 = R$ (reine Wirkwiderstände). 
            Bestimmen Sie die Gleichung für die Messspannung $\underline{U}_B$.
    \end{itemize}
\end{minipage}%
\begin{minipage}{0.5\textwidth}\centering%
\includegraphics{Tikz/pdf/Uebungen/circ_uebung_wechselstrommessbruecke_allgemein.pdf}%
\end{minipage}
\begin{itemize}
    \item[c)] Ist die Messspannung $\underline{U}_B$ aus Aufgabe b) frequenzabhängig? Begründen Sie Ihre Antwort.
    \item[d)] Wie verhält sich die Messspannung $\underline{U}_B$ bei $X_1 = X_2$?
    \item[e)] Wie verhält sich die Messspannung $\underline{U}_B$ bei $X_1 = -X_2$?
\end{itemize}
}

\Loesung{
\begin{itemize}
    \item[a)] Messspannung aus Potentialdifferenz von Spannungsteilern:
    % UB=U0*(Z2/(Z1+Z2)-Z4/(Z3+Z4))
    % UB=U0*(Z2*Z3+Z2*Z4-Z1*Z4)/(Z1+Z2)*(Z3+Z4)
    % UB=U0*(Z2*Z3-Z1*Z4)/(Z1+Z2)*(Z3+Z4)
    \begin{align*}
        \underline{U}_B &= U_0 \cdot \left( \frac{\underline{Z}_2}{\underline{Z}_1 + \underline{Z}_2} - \frac{\underline{Z}_4}{\underline{Z}_3 + \underline{Z}_4} \right)\\
        &= U_0 \cdot \left( \frac{\underline{Z}_2\cdot\underline{Z}_3+\redcancel{\underline{Z}_2\cdot\underline{Z}_4}-\underline{Z}_1\cdot\underline{Z}_4-\redcancel{\underline{Z}_2\cdot\underline{Z}_4}}{\left(\underline{Z}_1+\underline{Z}_2\right)\cdot\left(\underline{Z}_3+\underline{Z}_4\right)} \right) \\
        &= U_0 \cdot \left( \frac{\underline{Z}_2\cdot\underline{Z}_3\ -\ \underline{Z}_1\cdot\underline{Z}_4}{\left(\underline{Z}_1+\underline{Z}_2\right)\cdot\left(\underline{Z}_3+\underline{Z}_4\right)} \right) 
    \end{align*}
    \item[b)] wie a) nur mit $\underline{Z}_1 = \mathrm{j}X_1$, $\underline{Z}_2 = \mathrm{j}X_2$, $\underline{Z}_3 = \underline{Z}_4 = R$.
    % UB=U0*(jX2*R-jX1*R)/(jX1+jX2)*(R+R)
    % UB=U0* (jR)/(jR) * (X2-X1) / ((X1+X2)*(1+1))
    % UB=U0/2 * (X2-X1)/(X1+X2)
    \begin{align*}
        \underline{U}_B &= U_0 \cdot \left( \frac{X_2 - X_1}{\left(X_1+X_2\right)\cdot\left(1+1\right)} \right) \\
        \underline{U}_B &= U_0 \cdot \left( \redcancel{\frac{\mathrm{j}R}{\mathrm{j}R}} \cdot \frac{\mathrm{j}X_2\cdot R - \mathrm{j}X_1\cdot R}{\left(\mathrm{j}X_1+\mathrm{j}X_2\right)\cdot\left(R+R\right)} \right) \\
        &= \frac{U_0}{2}\cdot\frac{X_2-X_1}{X_2+X_1}
    \end{align*}
    \item[c)] Die Messpannung $\underline{U}_B$ ist unabhängig von der Frequenz, \underline{vorrausgesetzt} 
    es handelt sich um zwei kapazitive ($X_C \sim \frac{1}{\omega}$) oder um zwei induktive Blindwiderstände ($X_L \sim \omega$). 
    Dann heben sich die frequenzabhängigen Terme der Reaktanzen $X_1$ und $X_2$ gegenseitig auf. 
    \item[d)] Für $X_1=X_2$ ist $\underline{U}_B=0$. Beide Spannungsteiler (für $\underline{U}_B^+$ und $\underline{U}_B^-$) ergeben $U_0/2$.
    \item[e)] Für $X_1=-X_2$ ist die Resonanzbedingung erfüllt (des idealen LC-Reihenschwingkreises aus $X_1$ und $X_2$). 
            Die Messspannung $\underline{U}_B$ geht gegen unendlich (, da ungedämpft). 
\end{itemize}
}