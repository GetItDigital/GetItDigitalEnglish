% plot u_C(t), u_2(t) und i(t) ab t>=0
% u1 = Uq = konst. für t>=0
% u1 über Serienschaltung R1 + R2 + C
% u_2 über R_2 und C 
% u_C(t=0) = 0
\begin{tikzpicture}[x=1cm,y=1cm]
\begin{axis}[
    axis lines=center,
    xlabel={$t$},
    xmin=-1.25, xmax=5.5,
    ymin=-0.25, ymax=1.25,
    xtick={0,1,2,3,4,5},
    ytick={0,0.4,1},
    xticklabels={$0$,$\tau$,,,,$5\tau$},
    yticklabels={$0$,$\scriptstyle \frac{R_2}{R_1+R_2}\cdot \ecv{U_q}$,$\ecv{U_q}$},
    grid=none,
    width=8cm,
    height=5cm,
]
    % Hilfslinien und Grenzwerte
    \addplot[gray, dashed, domain=0:5.5]{1};% Asymptote für t>5tau 
    \addplot[black,dotted] coordinates {(0.0,0) (1,1)};% für tau bei u_C
    \addplot[black,dotted] coordinates {(0,0.4) (1,0)};% für tau bei u_{R_2}
    \addplot[black,dotted] coordinates {(0,0.4) (1,1)};% für tau bei u_2
    \addplot[gray, dashed] coordinates {(1.0,0) (1,1)};% tau

    % u_C(t)
    \addplot[blue, thick, domain=-1.25:0]{0};                               % t<=0
    \addplot[blue, thick, domain=-0:5.5, samples=100]{1-exp(-x)}
        node[pos=0.4,below right]{$\ec[blue]{u_C}$};                        % t>=0

    % u_2(t)
    \addplot[voltage, thick, domain=-1.25:0, densely dashed]{0};            % t<=0
    \addplot[voltage, thick] coordinates {(0,0) (0,0.4)};                   % t=0, Sprung
    \addplot[voltage, thick, domain=-0:5.5, samples=100]{1-0.6*exp(-x)}     % t>=0
        node[pos=0.4, above right, yshift=5pt]{$\ecv{u_2}\,\ec[black]{=}\,\ec[blue]{u_C}\,\ec[black]{+}\,\ec[purple]{u_{R_2}}$};

    % u_{R_1}(t) = i(t)*R_1
    \addplot[purple, thick, domain=-1.25:0, dashed]{0};                     % t<0
    \addplot[purple, thick, dashed] coordinates {(0,0) (0,0.4)};            % t=0, Sprung
    \addplot[purple, thick, domain=-0:5.5, samples=100]{0.4*exp(-x)}        % t>0 
        node[pos=0.4, above right]{$\ec[purple]{u_{R_2}}\,\ec[black]{=}\,\eci{i}\,\ec[black]{\cdot R_2}$};
        % Note: \ec[color]{...} broken in node's inlinemath. Does not return to default color after command argument is colored
\end{axis}
\end{tikzpicture}
