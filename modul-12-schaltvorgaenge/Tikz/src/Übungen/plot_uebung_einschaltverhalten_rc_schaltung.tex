% Spannung u_C(t), Spannung u_R1(t) und Strom i_C(t) für u_C(0)=0, t>=0
% ESB: Uq über R1 in Reihe mit (R2 parallel zu C)
\begin{tikzpicture}[x=1cm,y=1cm]
    \draw[draw=none] (-0.75,-6.25) rectangle (5.75,2.5); % Boundbox
    \begin{axis}[
        name=AxisuC,% u_C(t)
        xmin=-1, xmax=5.5,
        ymin=-0.125, ymax=1.35,
        xlabel={$t$},
        ylabel={$u(t)$},
        axis x line=center,
        axis y line=center,
        ylabel style={yshift=5pt},
        xtick={1,3,5},
        ytick={1},
        yticklabels={${\frac{U_q \cdot R_2}{R_1+R_2}}$},
        xticklabels={$\tau$,3$\tau$,5$\tau$},
        mark=none,
        grid=none,
        width=7cm,
        height=3.5cm,
        clip=false,
    ]
        \addplot[color=black,dashed,	domain=0:5.4] {1};          			% Asymptote für t gegen unendlich
        \addplot[color=voltage,very thick,	domain=-1:0] {0};           		% t<0
        \addplot[color=voltage,very thick,	domain=0:5.4,samples=61] {1-e^(-x)};% t>0 
        \addplot[color=lightgray, very thick] coordinates {(0,0)(1,1)}; 		% Tangente 
        \addplot[color=lightgray, thick, dotted] coordinates {(1,1)(1,0)}; 		% Dotted line
        \node at (1,0.5)[voltage,anchor=west]{$u_C$};
    \end{axis}
    \begin{axis}[
        at={($(AxisuC.south west)+(0cm,-2.75cm)$)},
        name=AxisuRone,% u_R1(t)
        xmin=-1, xmax=5.5,
        ymin=-0.125, ymax=1.35,
        xlabel={$t$},
        ylabel={$u(t)$},
        axis x line=center,
        axis y line=center,
        ylabel style={yshift=5pt},
        xtick={1,3,5},
        xticklabels={$\tau$,3$\tau$,5$\tau$},
        ytick={0.3,1},
        yticklabels={$\frac{U_q \cdot R_1}{R_1+R_2}$,$U_q$},
        mark=none,
        grid=none,
        width=7cm,
        height=3.5cm,
        clip=false,
    ]
        \addplot[color=black,dashed,	domain=-0:5.4] {1};          			% Max-Wert
        \addplot[color=black,dashed,	domain=-0:5.4] {0.3};          			% Asymptote für t gegen unendlich
        \addplot[color=voltage,very thick,	domain=-1:0] {0};           		% t<0
        \addplot[color=voltage,very thick] coordinates {(0,0)(0,1)}; 		    % t=0
        \addplot[color=voltage,very thick,	domain=0:5.4,samples=61] {1-0.7*(1-e^(-x))};% t>0
        \addplot[color=lightgray, very thick] coordinates {(0,1)(1,0.3)}; 		% Tangente 
        \addplot[color=lightgray, thick, dotted] coordinates {(1,1)(1,0)}; 		% Dotted line
        \node at (1,0.6)[voltage,anchor=west]{$u_{R1}$};
    \end{axis}
    \begin{axis}[
        at={($(AxisuRone.south west)+(0cm,-2.75cm)$)},
        name=AxisiC,% i_C(t)
        xmin=-1, xmax=5.5,
        ymin=-0.125, ymax=1.35,
        xlabel={$t$},
        ylabel={$i(t)$},
        axis x line=center,
        axis y line=center,
        ylabel style={yshift=5pt},
        xtick={1,3,5},
        ytick={1},
        yticklabels={$\frac{U_q}{R_1}$},
        xticklabels={$\tau$,3$\tau$,5$\tau$},
        mark=none,
        grid=none,
        width=7cm,
        height=3.5cm,
        clip=false,
    ]
        \addplot[color=black,dashed,	domain=0:5.4] {1};          			% Asymptote für t gegen unendlich
        \addplot[color=current,very thick,	domain=-1:0] {0};           		% t<0
        \addplot[color=current,very thick] coordinates {(0,0)(0,1)}; 		    % t=0
        \addplot[color=current,very thick,	domain=0:5.4,samples=61] {e^(-x)};  % t>0 
        \addplot[color=lightgray, very thick] coordinates {(0,1)(1,0)}; 		% Tangente 
        \addplot[color=lightgray, thick, dotted] coordinates {(1,1)(1,0)}; 		% Dotted line
        \node at (1,0.5)[current,anchor=west]{$i_C$};
    \end{axis}
\end{tikzpicture}
