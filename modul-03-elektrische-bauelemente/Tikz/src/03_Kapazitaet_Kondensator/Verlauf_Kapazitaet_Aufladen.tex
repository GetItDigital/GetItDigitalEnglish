\begin{tikzpicture}
    \begin{axis} [
            xlabel={Zeit},
            ylabel={\textcolor{cyan}{$U_\mathrm{q}$}/ \textcolor<3->{voltage}{$u_\mathrm{C}(t)$}/ \textcolor<4->{current}{$i(t)$}},
            % ylabel={
            % 	\textcolor{cyan}{$U_\mathrm{q}$}}
            % 	\only<3->{/ \textcolor{voltage}{$u_\mathrm{C}(t)$}}
            % 	\only<4->{/ \textcolor{current}{$i(t)$}},
            xmin=0, xmax=12.5,
            ymin=-0.2, ymax=1.2,
            ytick={0, 0.37, 0.63, 1},
            yticklabels={%
                    0,%
                    \only<4->{37\%},%
                    \only<3->{63\%},%
                    100\%
                },
            %yticklabels={0,37\%, 63\%, 100\%},
            xtick={0, 1, 2, 3, 4, 5, 6, 7, 8, 9, 10, 11, 12},
            xticklabels={0, \(\tau\), \(2\tau\), \(3\tau\), \(4\tau\), \(5\tau\), \(6\tau\), \(7\tau\), \(8\tau\), \(9\tau\), \(10\tau\), \(11\tau\), \(12\tau\)},
            width=10cm,
            height=3.5cm,
            axis lines=middle,
            every axis x label/.style={at={(current axis.right of origin)},anchor=west},
            every axis y label/.style={at={(current axis.above origin)},anchor=south},
            font=\small,
            legend style={font=\scriptsize}
        ]

        % Rechtecksignal für die Spannung
        \addplot [cyan, thick] coordinates {(0,1) (5,1) (5,0) (12,0)};

        % Spannungskurve eines Kondensators (Ladekurve)
        \only<3->{\addplot [voltage, domain=0:12, samples=100] {ifthenelse(x<0, 0, 1 - exp(-(x)))};}
        \only<4->{\addplot [current, domain=0:12, samples=100] {ifthenelse(x<0, 0,     exp(-(x)))};}


        % Legenden separat für Spannung und Strom
        \only<2->{\legend{ $U_\mathrm{q}$}}
        \only<3->{\legend{ $U_\mathrm{q}$, $u_\mathrm{C}(t)$}}
        \only<4->{\legend{ $U_\mathrm{q}$, $u_\mathrm{C}(t)$, $i(t)$}}


        % Markierung der Zeitkonstanten
        \only<3->{\draw[dashed] (axis cs:1,0) -- (axis cs:1,0.63) node[right] {};}
        \only<4->{\draw[dashed] (axis cs:1,0) -- (axis cs:1,0.37) node[right] {};}
        \only<4->{\draw[dashed] (axis cs:0,0.37) -- (axis cs:1,0.37) node[right] {};}
        \only<3->{\draw[dashed] (axis cs:0,0.63) -- (axis cs:1,0.63) node[right] {};}
        \node[right] at (5,0.5) {$t_\mathrm{0}$};

        % Punkte zur Hervorhebung
        \only<3->{\addplot[
                only marks,
                mark=*,
                mark options={scale=0.7, fill=black}
            ]  coordinates {
                    (1, 0.63)

                };}
        \only<4->{\addplot[
                only marks,
                mark=*,
                mark options={scale=0.7, fill=black}
            ]  coordinates {
                    (1, 0.37)

                };}


    \end{axis}
\end{tikzpicture}