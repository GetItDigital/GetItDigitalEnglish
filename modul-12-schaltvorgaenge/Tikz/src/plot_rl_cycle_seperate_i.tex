% Auf- und Entladevorgang einer Spule mittels Schalter
% AxisEnv 1: Spannung der Spannungsquelle
% AxisEnv 2: Spannung an der Spule
% AxisEnv 3: Strom am Kondensator
\begin{tikzpicture}[x=1cm,y=1cm]
	\begin{axis}[
		name=AxisQuelle,
		%%%%%%% START SAME OPTIONS %%%%%%%
		xmin=0, xmax=13,
		ymin=-0.125, ymax=1.25,
		xlabel={$t$},
		ylabel={$u(t)$},
		axis x line=center,
		axis y line=center,
		ylabel style={anchor=east, at={(ticklabel cs:0.95)}},
		ticks=none,
		mark=none,
		grid=none,
		width=7cm,
		height=3.5cm,
		%%%%%%% END OF SAME OPTIONS %%%%%%%
	]
		\addplot[color=voltage,very thick,	domain=0:1.05] {0};           		
		\addplot[color=voltage, very thick] coordinates {(1,1.014)(1,0)}; 
		\addplot[color=voltage,very thick,	domain=1:7] {1};
		\addplot[color=voltage, very thick] coordinates {(7,1.014)(7,0)};
		\addplot[color=voltage,very thick,	domain=6.95:12.8] {0};
		%\node at (1,0.5)[voltage,anchor=west]{$u_q$};
		%\addplot[color=black,dashed, domain=0:1]{x}; % Gradient in t=0, schneidet Asymptote bei t=tau
	\end{axis}
	\begin{axis}[
		at={($(AxisQuelle.south west)+(0,-0.25cm)$)}, % Position 2. Plot
		anchor=north west,
		name=AxisStrom,
		%%%%%%% START SAME OPTIONS %%%%%%%
		xmin=0, xmax=13,
		ymin=-0.125, ymax=1.25,
		xlabel={$t$},
		ylabel={$i(t)$},
		axis x line=center,
		axis y line=center,
		ylabel style={anchor=east, at={(ticklabel cs:0.95)}},
		ticks=none,
		mark=none,
		grid=none,
		width=7cm,
		height=3.5cm,
		%%%%%%% END OF SAME OPTIONS %%%%%%%
	]
		%\addplot[color=black,dashed,	domain=0:12.9] {1};        				% Asymptote für t gegen unendlich (U_q/R)
		\addplot[color=red,very thick,	domain=0:1.05] {0};           			% t < t_0
		\addplot[color=red,very thick,	domain=1:7.04,samples=61] {1-e^(-x+1)};	% t_0 < t < t_1 
		\addplot[color=red,very thick,	domain=7:12.8,samples=61] {e^(-x+7)};	% t_1 < t
		%\node at (11,1)[black,anchor=south]{$U_q/R$};					% Beschriftung Asymptote
		%\addplot[color=black,dashed, domain=0:1]{x}; % Gradient in t=0, schneidet Asymptote bei t=tau
	\end{axis}
	\begin{axis}[
		at={($(AxisStrom.south west)+(0,-0.25cm)$)}, % Position 3. Plot
		anchor=north west,
		name=AxisSpannung,
		%%%%%%% START SAME OPTIONS %%%%%%%
		xmin=0, xmax=13,
		ymin=-1.25, ymax=1.25,
		xlabel={$t$},
		ylabel={$u_L(t)$},
		axis x line=center,
		axis y line=center,
		ylabel style={anchor=east, at={(ticklabel cs:0.95)}},
		%ylabel style={anchor=east, yshift=8pt},
		ticks=none,
		%extra y ticks={1},
		%extra y tick labels={$\frac{U_q}{R}$},
		mark=none,
		grid=none,
		width=7cm,
		height=5cm,
		%%%%%%% END OF SAME OPTIONS %%%%%%%
		clip=false,
	]
		\addplot[color=voltage,very thick,	domain=0:1] {0}; 						% t < t_0
		\addplot[color=voltage, very thick] coordinates {(1,1)(1,0)}; 				% Sprung
		\addplot[color=voltage,very thick,	domain=1:7,samples=61] {e^(-x+1)}; 		% t_0 < t < t_1 
		\addplot[color=voltage, very thick] coordinates {(7,-1)(7,0)}; 				% Sprung
		\addplot[color=voltage,very thick,	domain=7:12.8,samples=61] {-e^(-x+7)}; 	% t_1 < t
		%\node at (1,0.5)[red,anchor=west]{$i$}; 
	\end{axis}
\end{tikzpicture}