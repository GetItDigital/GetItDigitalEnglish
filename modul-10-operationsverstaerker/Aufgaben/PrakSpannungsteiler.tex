\subsection{Analyse eines belasteten Spannungsteilers\label{Aufg26}}
% Alt: Praktikumsaufgabe 12
\Aufgabe{
        Zeichnen Sie einen Schaltplan bestehend aus einem belasteten Spannungsteiler.  
        Die Bauteilwerte sind wie folgt gegeben:
        
        \begin{itemize}
            \item Widerstände: $R_1 = R_2 = 10\,k\Omega$
            \item Lastwiderstand: $R_\mathrm{L} = 2{,}2\,\mathrm{k\Omega}$
            \item Versorgungsspannung: $U_\mathrm{CC} = +15\,\mathrm{V}$
        \end{itemize}
        
        \begin{itemize}
            \item[a)] Berechnen Sie die Ausgangsspannung $U_\mathrm{C}$ ohne Lastwiderstand.
            \item[b)] Berechnen Sie die Ausgangsspannung $U_\mathrm{C}$ mit dem angeschlossenen Lastwiderstand $\mathrm{R_L}$.
            \item[c)] Erklären Sie, warum die Spannung $U_\mathrm{C}$ durch den Lastwiderstand absinkt.
        \end{itemize}
    
}

\Loesung{
    \begin{figure}[H]
        \centering
            \begin{tikzpicture}
    \draw (2,4) to[R,l=$R_\mathrm{1}$, -*] (2,2) to[R,l=$R_\mathrm{2}$,-*] (2,0);
    \draw (2,2) -- (3.5,2) to[R,l=$R_\mathrm{L}$] (3.5,0) -- (2,0);
    \draw (2,0) -- (0,0);
    \draw (2,4) -- (0,4) to[V,name=V] (0,0);
    
    \varrmore{V}{$V_\mathrm{0}$};
\end{tikzpicture}
   
            \label{fig:LsgPrakSpannungsteiler}
    \end{figure}
    \begin{itemize}
        \item[a)] Berechnung der Ausgangsspannung $U_\mathrm{C}$ ohne Lastwiderstand:
        \begin{align*}
            U_\mathrm{C} &= \frac{R_2}{R_1 + R_2} \cdot U_\mathrm{CC} \\
            U_\mathrm{C} &= \frac{10\,k\Omega}{10\,k\Omega + 10\,\mathrm{k\Omega}} \cdot 15\,\mathrm{V} \\
            U_\mathrm{C} &= \frac{1}{2} \cdot 15\,\mathrm{V} \\
            U_\mathrm{C} &= 7{,}5\,\mathrm{V}
        \end{align*}
        \item[b)] Berechnung der Ausgangsspannung $\mathrm{U_C}$ mit dem angeschlossenen Lastwiderstand $\mathrm{R_L}$:
        \begin{align*}
            U_\mathrm{C} &= \frac{R_2}{R_1 + R_2 + R_\mathrm{L}} \cdot U_\mathrm{CC} \\
            U_\mathrm{C} &= \frac{10\,\mathrm{k\Omega}}{10\,\mathrm{k\Omega} + 10\,\mathrm{k\Omega} + 2{,}2\,\mathrm{k\Omega}} \cdot 15\,\mathrm{V} \\
            U_\mathrm{C} &= \frac{10\,\mathrm{k\Omega}}{22\,\mathrm{k\Omega}} \cdot 15\,\mathrm{V} \\
            U_\mathrm{C} &= \frac{10}{22} \cdot 15\,\mathrm{V} \\
            U_\mathrm{C} &= \frac{150}{22}\,\mathrm{V} \\
            U_\mathrm{C} &\approx 6{,}82\,\mathrm{V}
        \end{align*}
        \item[c)] Die Spannung $U_\mathrm{C}$ sinkt durch den Lastwiderstand $R_\mathrm{L}$ ab, da dieser einen Spannungsteiler mit den Widerständen $R_1$ und $R_2$ bildet. Der Lastwiderstand $R_\mathrm{L}$ entzieht der Schaltung einen Teil der
        Spannung, wodurch die Ausgangsspannung $U_\mathrm{C}$ sinkt.
    \end{itemize}
    Siehe: Abschnitt \ref{fig: Verschaltung Verstaerker} und Tabelle \ref{tab:Grundschaltungen}
}