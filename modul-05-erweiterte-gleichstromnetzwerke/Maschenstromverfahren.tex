\newpage

\newvideofile{Mastrom}{Maschenstromverfahren}
\begin{frame}
	\ftx{Lernziele Maschenstromverfahren}
	\begin{Lernziele}{Maschenstromverfahren}
        Die Studierenden 
        \begin{itemize}
			\item kennen die Schritte des Maschenstromverfahrens.
            \item können mit Hilfe des Maschenstromverfahrens elektrische Netzwerke analysieren. 
        \end{itemize}                   
    \end{Lernziele}

	\speech{MastromLernz}{1}{Neben dem Knotenpotentialverfahren bietet das Maschenstromverfahren eine weitere Möglichkeit, 
    um elektrische Netzwerke zu analysieren. 
    Die für die Durchführung des Maschenstromverfahrens erforderlichen Schritte werden folgend vorgestellt.} 
\end{frame}

\begin{frame}
    \fta{Maschenstromverfahren}

    \s{
        Neben der Knotenpotentialanalyse zur Untersuchung von elektrischen Netzwerken kann auch das Maschenstromverfahren
        zur Berechnung herangezogen werden. Beim Maschenstromverfahren wird ein Gleichungssystem 
        aufgestellt, mit welchem die Maschenströme ermittelt werden können. Um das Gleichungssystem des Maschenstromverfahrens 
        aufstellen zu können, müssen einige Vorbereitungen getroffen werden. In den folgenden Schritten soll das elektrische 
        Netzwerk aus der Abbildung \ref{BildMaschenstromverfahren} mittels des Maschenstromverfahrens berechnet werden. 
    }

    \b{
		\begin{itemize}
			\item Vorbereitung des Netzwerkes
			\item Maschen definieren
			\item Widerstandsmatrix bestimmen
			\item Quellenspannungen zuordnen
			\item Gleichungssystem aufstellen
		\end{itemize}
	}

	\fu{   
        \resizebox{0.7\textwidth}{!}{%                                         
        \begin{circuitikz}
    \draw (0,0) [V^<, i, name=U0] to (0,3);
    \draw (0,3) [R, l_=$R_0$, v^>, name=R0] to (0,6) ;
    \draw (0,6) to [short, -*] (5,6)
    to [R, l=$R_1$, v, i, name=R1, -*] (5,0)
    to (0,0);

    \draw (5,0) to (10,0)
    to (10,1.5);
    \draw (10,4.5) to [I, i, name=Iq,*-*] (10,1.5);
    \draw (10,4.5) to (10,6)
    to (5,6);
    \draw (10,1.5) to (12,1.5);
    \draw (10,4.5) to (12,4.5);
    \draw (12,1.5) [R, l=$R_2$, v, i, name=R2] to (12,4.5);

    \varrmore{R0}{$U_\mathrm{R0}$};
    \iarrmore{R1}{$I_1$};
    \varrmore{R1}{$U_\mathrm{R1}$};
    \iarrmore{R2}{$I_2$};
    \varrmore{R2}{$U_\mathrm{R2}$};
    \iarrmore{U0}{$I_0$};
    \varrmore{U0}{$U_0$};
    \iarrmore{Iq}{$I_\mathrm{q}$};
\end{circuitikz}
        }                                                  
    }{{\bf Netzwerk für das Maschenstromverfahren.} Elektrisches Netzwerk mit einer Spannungsquelle, einer Stromquelle und drei Widerständen. Anhand des Netzwerkes wird 
    das Maschenstromverfahren erläutert. \label{BildMaschenstromverfahren}}   
    
    \speech{MastromSchritte}{1}{Zu beginn sollte wie auch beim Knotenpotentialverfahren das elektrische Netzwerk vorbereitet werden.
    Das beinhaltet unter anderem das Zusammenfassen von Reihen und Parallelschaltungen wie auch das eventuelle Umwandeln der Quellen. 
    Folgend werden die Maschen und die zugehörigen Maschenströme festgelegt, die Widerstandsmatrix erstellt, die Quellspannungen zugeordnet 
    und daraus ein Gleichungssystem aufgestellt.} 
\end{frame}




\begin{frame}
\ftb{Vorbereitung des Netzwerkes}

    \s{
        Das Maschenstromverfahren arbeitet mit Spannungsquellen und Widerstandswerten. Hierfür müssen eventuelle Stromquellen
        in Spannungsquellen transformiert werden. Diese Art der Umwandlung einer Stromquelle in eine Spannungsquelle wird noch
        einmal in der Abbildung \ref{BildMasch1} dargestellt. Hier wird aus der Stromquelle und dem parallel liegenden 
        Innenwiderstand eine Spannungsquelle mit einem in reihe liegenden Innenwiderstand. 
    }

	\b{
		\begin{itemize}
			\item Umwandlung der Stromquellen in Spannungsquellen:
		\end{itemize}
	}    

    \fu{                                     
        \begin{circuitikz}
    \draw (0,0) [I, name=Iq] to (0,3);
    \draw (0,3) to [short, -*] (2,3)
    to [R=$R_\mathrm{i}$, -*] (2,0)
    to (0,0);
    \draw (2,0) to [short, -o] (4,0);
    \draw (2,3) to [short, -o] (4,3);
    \draw (8,0) [V^<, name=Uq] to (8,3);
    \draw (8,3) to [R, l=$R_\mathrm{i}$, v, name=Ri, -o] (12,3);
    \draw (8,0) to [short, -o] (12,0);
    \draw [ultra thick,->] (4.8,1.5) -- (6.5,1.5);

    \iarrmore{Iq}{$I_\mathrm{q}$};
    \varrmore{Uq}{$U_\mathrm{q}$};
    \varrmore{Ri}{$U_\mathrm{Ri}$};
\end{circuitikz}                                                   
    }{{\bf Umwandlung einer Stromquelle.} Umwandlung einer Stromquelle mit Parallelwiderstand zur Spannungsquelle mit Serienwiderstand. \label{BildMasch1}}   

    \s{
        Die Gleichung \ref{GleichungUmrechnungStromquelle} zeigt die Berechnung des Spannungswertes der Spannungsquelle 
        aus dem angegebenen Stromwert der Stromquelle und dem Widerstandswert des Innenwiderstands. 
    }

    \onslide<2->{
	\b{
		\begin{itemize}
			\item Umrechnung der Stromwerte zu Spannungswerten:
		\end{itemize}
	}

    \begin{eq}
         U_\mathrm{q} = R_\mathrm{i} \cdot I_\mathrm{q} \label{GleichungUmrechnungStromquelle}
    \end{eq}}

    \s{
        Außerdem müssen mögliche Leitwerte von elektrischen Widerständen in Widerstandswerte geändert werden. In der Gleichung 
        \ref{GleichungUmrechnungLeitwert} wird gezeigt, wie der Widerstandswert aus dem Kehrwert des Leitwerts ermittelt wird. 
    }

    \onslide<3->{
	\b{
		\begin{itemize}
			\item Berechnung der Leitwerte:
		\end{itemize}
	}

    \begin{eq}               
         R_\mathrm{i} = \frac{1}{G_\mathrm{i}} \label{GleichungUmrechnungLeitwert}
    \end{eq}}

    \speech{MastromVorb}{1}{Eventuelle Stromquellen müssen für das Maschenstromverfahren in Spannungsquellen umgewandelt werden.
    Wie hier dargestellt werden die Stromquellen mit Parallelwiderstand in Spannungsquellen mit Reihenwiderstand umgewandelt.} 
    \speech{MastromVorb}{2}{Die Stromwerte der Stromquellen werden über das ohmsche Gesetz in Spannungswerte überführt.} 
    \speech{MastromVorb}{3}{Falls Leitwerte anstatt von Widerstandswerten angegeben werden, können diese über den Kehrwert ermittelt werden,
    da beim Maschenstromverfahren mit den Widerstandswerten gearbeitet wird.} 
\end{frame}

\begin{frame}
    \ftx{Vorbereitung des Netzwerkes}

    \s{
        In der Abbildung \ref{BildMasch2} wird die Umwandlung der Stromquelle des vorgestellten Netzwerkes in eine Spannungsquelle
        durchgeführt. Aus der Parallelschaltung der Stromquelle und des Widerstandes $R_2$ wird eine Reihenschaltung bestehend
        aus der neuen Spannungsquelle und dem Widerstand $R_2$.
    }

    \b{
        Umwandlung der Stromquelle in eine Spannungsquelle:
        \begin{itemize}
            \item Die Stromquelle mit einer Spannungsquelle ersetzen
            \item Aus dem Parallelwiderstand wird ein Reihenwiderstand
        \end{itemize}
    }

    \fu{                                     % Beginn Bild Netzwerk 1 mit rotem Kreis
		\resizebox{\textwidth}{!}{%
        \begin{circuitikz}                                        
            \draw (0,0) [V^<, i, name=U0] to (0,3);            
            \draw (0,3) [R, l_=$R_0$, v^>, name=R0] to (0,6) ;          
            \draw (0,6) to [short, -*] (5,6)                                          
                to [R, l=$R_1$, v, i, name=R1, -*] (5,0)                 
                to (0,0);                                                    
    
            \draw (5,0) to (10,0)                                        
                to (10,1.5);                                                  
            \draw (10,4.5) to [I, i, name=Iq,*-*] (10,1.5);                         
            \draw (10,4.5) to (10,6)                                      
                to (5,6);                                  
            \draw (10,1.5) to (12,1.5);                                   
            \draw (10,4.5) to (12,4.5);                                   
            \draw (12,1.5) [R, l=$R_2$, v, i, name=R2] to (12,4.5); 
            \draw[red,line width= 0.1 mm] (11,3) circle[radius = 2.6 cm];

            \varrmore{R0}{$U_\mathrm{R0}$};
            \iarrmore{R1}{$I_1$};
            \varrmore{R1}{$U_\mathrm{R1}$};
            \iarrmore{R2}{$I_2$};
            \varrmore{R2}{$U_\mathrm{R2}$};
            \iarrmore{U0}{$I_0$};
            \varrmore{U0}{$U_0$};
            \iarrmore{Iq}{$I_\mathrm{q}$};
        
            \draw [ultra thick,->] (15,3) -- (17,3); 

            \draw (20,0) [V^<, i, name=U02] to (20,3);            
            \draw (20,3) [R, l_=$R_0$, v^>, name=R02] to (20,6) ;          
            \draw (20,6) to [short, -*] (25,6)                                          
                to [R, l=$R_1$, v, i, name=R12, -*] (25,0)                 
                to (20,0);                                                    
            \draw (25,0) to (30,0)
            to [V^<, i, name=U1] (30,3)
            to [R, l=$R_2$, v, name=R22] (30,6)
            to (25,6);                
            \draw[red,line width= 0.1 mm] (30,3) circle[radius = 2.6 cm];

            \varrmore{R02}{$U_\mathrm{R0}$};
            \iarrmore{R12}{$I_1$};
            \varrmore{R12}{$U_\mathrm{R1}$};
            \varrmore{R22}{$U_\mathrm{R2}$};
            \iarrmore{U02}{$I_0$};
            \varrmore{U02}{$U_0$};
            \iarrmore{U1}{$I_2$};
            \varrmore{U1}{$U_1$};
        \end{circuitikz}  
        }                                           
    }{{\bf Umwandlung einer Stromquelle in einem elektrischen Netzwerk.} Umwandlung der Stromquelle mit Parallelwiderstand 
    des zu berechnenden Netzwerkes in eine Spannungsquelle mit Reihenwiderstand. Die Transformation der Quelle findet im roten 
    Kreis statt. \label{BildMasch2}}  
    
    \speech{MastromUmw}{1}{Die Umwandlung der Stromquelle in eine Spannungsquelle wird für das angegebene Netzwerk dargestellt.
    Aus dem Parallelwiderstand der Stromquelle wird der Reihenwiderstand für die Spannungsquelle.} 
\end{frame}

\begin{frame}
    \ftx{Maschenstromverfahren}

    \begin{Merksatz}{Maschenstromverfahren}
        Für das Maschenstromverfahren werden zur Analyse eines elektrischen Netzwerkes Spannungsquellen und Widerstandswerte 
        benötigt.
    \end{Merksatz}

    \speech{MastromMrk}{1}{Für das Maschenstromverfahren werden zur Analyse eines elektrischen Netzwerkes 
    Spannungsquellen und Widerstandswerte benötigt.} 
\end{frame}


\begin{frame}
\ftb{Maschen definieren}

    \s{
        Für die Maschenstromanalyse müssen die Maschen festgelegt werden. Über den Maschen werden die Maschenströme
        festgelegt. Außerdem werden den Maschen die Widerstände zugeordnet. In der Abbildung \ref{BildMasch3} wird 
        das vorgestellte Netzwerk mit den eingezeichneten Maschen $M_1$ und $M_2$ dargestellt. 
    } 
    
    \b{
		\begin{itemize}
			\item Festlegung der Maschen und Maschenströme
		\end{itemize}
	}

    \fu{    
        \resizebox{0.7\textwidth}{!}{%                                                 
        \begin{circuitikz}
    \draw[blue,->,shift={(2.5,3)}] (120:1 cm) arc (120:-90:1 cm) node at(0,0){$M_1$};
    \draw[blue,->,shift={(7.5,3)}] (120:1 cm) arc (120:-90:1 cm) node at(0,0){$M_2$};

    \draw (0,0) [V^<, i, name=U0] to (0,3);
    \draw (0,3) [R, l_=$R_0$, v^>, name=R0] to (0,6) ;
    \draw (0,6) to [short, -*] (5,6)
    to [R, l=$R_1$, v, i, name=R1, -*] (5,0)
    to (0,0);
    \draw (5,0) to (10,0)
    to [V^<, i, name=U1] (10,3)
    to [R, l=$R_2$, v, name=R2] (10,6)
    to (5,6);

    \varrmore{R0}{$U_\mathrm{R0}$};
    \iarrmore{R1}{$I_1$};
    \varrmore{R1}{$U_\mathrm{R1}$};
    \varrmore{R2}{$U_\mathrm{R2}$};
    \iarrmore{U0}{$I_0$};
    \varrmore{U0}{$U_0$};
    \iarrmore{U1}{$I_2$};
    \varrmore{U1}{$U_1$};

\end{circuitikz}
        }                                                            
    }{{\bf Maschen im elektrischen Netzwerk.} Festlegung der Maschen $M_1$ und $M_2$ sowie der zugehörigen Maschenströme. 
    \label{BildMasch3}}                                                

    \s{
        
        Durch alle Komponenten welche sich an der Masche $M_1$ befinden, fließt der Maschenstrom $I_\mathrm{M1}$. So wie auch 
        alle Komponenten an der Masche $M_2$ von dem Maschenstrom $I_\mathrm{M2}$ durchflossen werden. Die beiden Maschenströme
        $I_\mathrm{M1}$ und $I_\mathrm{M2}$ werden vektoriell in der Gleichung \ref{GleichungMaschenströme} notiert. 
    }

    \begin{eq}
        I_\mathrm{M} = 
        \begin{bmatrix}
            I_\mathrm{M1} \\
            \\
            I_\mathrm{M2} \\
        \end{bmatrix} \label{GleichungMaschenströme}
    \end{eq}      

    \speech{MastromMasch}{1}{Die Maschen M 1 und M 2 werden entsprechend definiert. 
    Ihnen zugehörig werden die Maschenströme IM eins und IM zwei der beiden Maschen im Vektor der Maschenströme IM festgehalten.}     
\end{frame}


\begin{frame}
    \ftb{Widerstandsmatrix bestimmen}

    \s{
        In der Masche $M_1$ trifft der Maschenstrom $I_{M1}$ auf die beiden Widerstände $R_0$ und $R_1$. Die Masche $M_1$ wird in 
        der Abbildung \ref{BildWiderstandsmatrix} in blau dargestellt. Der Maschenstrom $I_{M2}$ der Masche $M_2$ durchfließt die 
        Widerstände $R_1$ und $R_2$. Der Widerstand $R_1$ ist somit teil der beiden Maschen $M_1$ und $M_2$ und wird von beiden 
        Maschenströmen durchflossen. Wird ein Widerstand von mehreren Maschenströmen durchflossen gilt dieser als Kopplungswiderstand 
        zwischen den Maschen. 

        In der Widerstandsmatrix nach Gleichung
        \ref{GleichungWiderstandsmatrix} werden die Widerstände der Maschen und die Kopplungswiderstände zugeordnet. Hier werden
        die Umlaufwiderstände aus den Maschen in die Hauptdiagonale eingetragen. Das heißt, dass die Reihenschaltung $R_0 + R_1$ 
        das Element der ersten Spalte und Zeile für die erste Masche wird. Die Reihenschaltung $R_1 + R_2$ wird in das Element 
        der zweiten Spalte und Zeile für die zweite Masche eingetragen. In die übrigen Elemente der Widerstandsmatrix werden 
        die Kopplungswiderstände zwischen den jeweiligen Maschen eingetragen. Der Kopplungswiderstand bekommt ein positives 
        Vorzeichen, wenn beide Maschenströme in die selbe Richtung fließen und ein negatives, wenn diese entgegengesetzt fließen. 
        Die Widerstandsmatrix wird in der Wechselstromtechnik auch als Maschenimpedanzmatrix bezeichnet. 
    }

    \b{
		\begin{itemize}
			\item<2-> Hauptdiagonale mit den Widerstandswerten, welche direkt an den Maschen liegen
			\item<3-> Übrige Elemente mit den Widerstandswerten, welche direkt zwischen beteiligten Maschen liegen
		\end{itemize}
	}

    \fu{    
        \resizebox{0.7\textwidth}{!}{%                                                 
        \begin{circuitikz}
    \draw[blue,->,shift={(2.5,3)}] (120:1 cm) arc (120:-90:1 cm) node at(0,0){$M_1$};
    \draw[blue,->,shift={(7.5,3)}] (120:1 cm) arc (120:-90:1 cm) node at(0,0){$M_2$};

    \draw (0,0) [V^<, i, name=U0] to (0,3);
    \draw (0,3) [R, l_=$R_0$, v^>, name=R0] to (0,6) ;
    \draw (0,6) to [short, -*] (5,6)
    to [R, l=$R_1$, v, i, name=R1, -*] (5,0)
    to (0,0);
    \draw (5,0) to (10,0)
    to [V^<, i, name=U1] (10,3)
    to [R, l=$R_2$, v, name=R2] (10,6)
    to (5,6);

    \varrmore{R0}{$U_\mathrm{R0}$};
    \iarrmore{R1}{$I_1$};
    \varrmore{R1}{$U_\mathrm{R1}$};
    \varrmore{R2}{$U_\mathrm{R2}$};
    \iarrmore{U0}{$I_0$};
    \varrmore{U0}{$U_0$};
    \iarrmore{U1}{$I_2$};
    \varrmore{U1}{$U_1$};

    \pause
    \draw[blue, very thick](0,5) to (0,6) to (5,6) to (5,3.5);
    \draw[blue, very thick](5,2.5) to (5,0) to (0,0) to (0,1);
    \draw[blue, very thick](0,2) to (0,4);

    \pause
    \draw[green, very thick, dashed](5,6) to (5,3.5);
    \draw[green, very thick, dashed](5,0) to (5,2.5);

\end{circuitikz}
        }                                                            
    }{{\bf Festlegung der Zweige.} Elektrisches Netzwerk mit hervorgehobenen Zweigen zur Erstellung der Widerstandsmatrix für das Maschenstromverfahren
    \label{BildWiderstandsmatrix}}

    \b{
		\only<1>{
		\begin{eq}
			R_\mathrm{M} =
			\begin{bmatrix} 
				0 & 0 \\
				\\
				0 & 0 \\
			\end{bmatrix} 
		\end{eq}
		}
		\only<2>{
		\begin{eq}
			R_\mathrm{M} =
			\begin{bmatrix} 
				\color{blue}{R_0 + R_1} & 0 \\
				\\
				0 & R_1 + R_2 \\
			\end{bmatrix} 
		\end{eq}
		}
		\only<3>{
		\begin{eq}
			R_\mathrm{M} =
			\begin{bmatrix} 
				\color{blue}{R_0 + R_1} & \color{green}{-R_1} \\
				\\
				\color{green}{-R_1} & R_1 + R_2 \\
			\end{bmatrix} 
		\end{eq}
		}
	}

    \s{
    \begin{eq}
        R_\mathrm{M} =
        \begin{bmatrix} 
            \color{blue}{R_0 + R_1} & \color{green}{-R_1} \\
			\\
			\color{green}{-R_1} & R_1 + R_2 \\
        \end{bmatrix} \label{GleichungWiderstandsmatrix}
    \end{eq}
    }

    \speech{MastromWiderstands}{1}{Als nächstes wird die Widerstandsmatrix für das angegebene elektrische Netzwerk bestimmt.}
    \speech{MastromWiderstands}{2}{Die an den Maschen beteiligten Widerstände werden auf der Hauptdiagonalen der Matrix eingetragen.
    Für die Masche M 1 sind das die in blau gekennzeichneten Widerstände R null und R eins.
    An der Masche M 2 sind die Widerstände R eins und R 2 beteiligt. } 
    \speech{MastromWiderstands}{3}{Die Widerstände welche an zwei oder mehr Maschen beteiligt sind, 
    werden zwischen den Maschen eingetragen. 
    In diesem Fall handelt es sich hier um den Widerstand R eins. 
    Die Widerstände zwischen den Maschen werden mit einem negativen Vorzeichen versehen.}  
\end{frame}


\begin{frame}
    \ftb{Quellenspannungen zuordnen}

    \s{
        Den Maschen müssen die Spannungsquellen, die sogennanten Quellspannungen zugeordnet werden. In der Masche $M_1$ liegt die 
        Quellenspannung $U_0$. Die Masche $M_2$ weist die Quellenspannung $U_1$ auf. Bei übereinstimmender Richtung von Spannungspfeil 
        und Maschenrichtung ergibt sich ein negatives Vorzeichen für die Quellspannungen. Sind die Richtungen von Spannungspfeil und
        dem Maschenumlauf entgegengesetzt wird das Vorzeichen der Quellspannung positiv. Für den Fall, dass in einer Masche keine Quelle 
        vohanden ist, wird eine $0$ eingetragen.
    }

    \b{
        \fu{    
        \resizebox{0.7\textwidth}{!}{%                                                 
        \begin{circuitikz}
    \draw[blue,->,shift={(2.5,3)}] (120:1 cm) arc (120:-90:1 cm) node at(0,0){$M_1$};
    \draw[blue,->,shift={(7.5,3)}] (120:1 cm) arc (120:-90:1 cm) node at(0,0){$M_2$};

    \draw (0,0) [V^<, i, name=U0] to (0,3);
    \draw (0,3) [R, l_=$R_0$, v^>, name=R0] to (0,6) ;
    \draw (0,6) to [short, -*] (5,6)
    to [R, l=$R_1$, v, i, name=R1, -*] (5,0)
    to (0,0);
    \draw (5,0) to (10,0)
    to [V^<, i, name=U1] (10,3)
    to [R, l=$R_2$, v, name=R2] (10,6)
    to (5,6);

    \varrmore{R0}{$U_\mathrm{R0}$};
    \iarrmore{R1}{$I_1$};
    \varrmore{R1}{$U_\mathrm{R1}$};
    \varrmore{R2}{$U_\mathrm{R2}$};
    \iarrmore{U0}{$I_0$};
    \varrmore{U0}{$U_0$};
    \iarrmore{U1}{$I_2$};
    \varrmore{U1}{$U_1$};

\end{circuitikz}
        }                                                            
    }    
    }
   
    \begin{eq}
        U_\mathrm{M} = 
        \begin{bmatrix}
            U_0 \\
            \\
            -U_1 \\
        \end{bmatrix} \label{GleichungQuellspannungen}
    \end{eq}

    \speech{MastromQuell}{1}{In dem Vektor U M werden die Quellströme festgehalten. 
    An der Masche M 1 befindet sich die Spannungsquelle U null. 
    An der Masche M 2 befindet sich die Spannungsquelle U eins.
    Verläuft der Spannungspfeil der Spanungsquelle mit dem Maschenumlauf,
    wird ein negatives Vorzeichen, 
    bei einem entgegengesetzten Spannungspfeil ein positives Vorzeichen gewählt.}  
\end{frame}


\begin{frame}
\ftb{Gleichungssystem aufstellen}

\s{
    Nach dem ohmschen Gesetz wird die elektrische Spannung als Produkt aus dem elektrischen Widerstand und Strom berechnet. 
    Bei der Maschenstromanalyse werden zur Berechnung des Vektors der Maschenströme $I_\mathrm{M}$ die Widerstandsmatrix $R_\mathrm{M}$ und der 
    Vektor der Quellspannungen $U_\mathrm{M}$ verwendet. Das Gleichungssystem wird in der Gleichung \ref{GleichungMasche1} abgebildet.  
    }

\begin{eq}
\begin{bmatrix} 
    R_0 + R_1 & -R_1 \\
    \\
    -R_1 & R_1 + R_2 \\
\end{bmatrix}   
\cdot
\begin{bmatrix}
    I_\mathrm{M1} \\
    \\
    I_\mathrm{M2} \\
\end{bmatrix}
=
\begin{bmatrix}
    U_0 \\
    \\
    -U_1 \\
\end{bmatrix} \label{GleichungMasche1}
\end{eq}

\s{
    Die gesuchte Größen werden aus den Maschenströmen berechnet. Beispielsweise werden die Ströme $I_0$, $I_1$ und $I_2$ gesucht.
    Die Ergebnisse der Maschenströme sind mit den Strömen gleichzusetzen, welche lediglich von einem Maschenstrom durchflossen werden. 
    In dem vorgestellten elektrischen Netzwerk beträgt der Strom der Masche $M_1: I_\mathrm{M1} = I_0$ und der Masche $M_2: I_\mathrm{M2} = -I_2$. Der
    Strom durch den Widerstand $R_1$ lässt sich beispielsweise über die Knotengleichung berechnen: $I_1 = I_0 + I_2$.
}

\speech{MastromGleich}{1}{Nun kann aus der Widerstandsmatrix, 
dem Vektor der Maschenströme 
und dem Vektor der Quellspannungen 
das notwendige Gleichungssystem aufgestellt und gelöst werden.
Aus den so zu bestimmenden Maschenströmen können anschließend die gesuchten Ströme bestimmt werden.
Hier ist beispielsweise der Strom I 0 gleich dem Maschenstrom I emm eins.
Der Maschenstrom I emm 2 ist gleich dem negativen Wert von I zwei, 
da die Verläufe der beiden Ströme entgegengesetzt sind.
Zum Schluss kann der Strom I eins über die Knotenregel aus den Strömen I 0 und I 2 berechnet werden.
Somit sind dann alle Größen des Netzwerkes identifiziert worden.}
\end{frame}
\newpage


%%-------------------------NETZWERK2------------------------------%%



\begin{frame}
    \ftx{Beispiel Maschenstromverfahren}
    \begin{bsp}{Maschenstromverfahren}{}

    \s{
        Analyse des elektrischen Netzwerk aus der Abbildung \ref{BildMasch4} mittels der Maschenstromanalyse. 
    }

    \begin{enumerate}
        \only<1>{	
            \item[]

    Für die Maschenstromanalyse sollen die folgenden Schritte bearbeitet werden:

    \begin{itemize}
        \item[a)] Vorbereitung des Netzwerkes
		\item[b)] Maschen und Maschenströme definieren
		\item[c)] Widerstandsmatrix bestimmen
		\item[d)] Quellspannungen zuordnen
		\item[e)] Gleichungssystem aufstellen
    \end{itemize}

    \fu{  
        \resizebox{0.4\textwidth}{!}{%                                     
        \begin{circuitikz}
    \draw (-1,2) to [R=$R_0$, v_<, name=R0] (-1,-2)
    to [V_>, i, name=U0] (-1,-6)
    to (3,-6)
    to [R=$R_4$, i^<, v_<, name=R4, -*] (3,-2)
    to [R=$R_3$, i^<, v_<, name=R3, -*] (3,2)
    to [V^<, i_>, name=U1] (7,2)
    to [R=$R_1$ , v, name=R1] (11,2);
    \draw (-1,2) to (3,2);
    \draw (3,-2) to [R=$R_5$ ,i, v_>, name=R5, -*] (11,-2);
    \draw (11,2) to (11,-6)
    to  (9,-6);
    \draw (9,-6) to  [I, name=Iq,*-*] (5,-6)
    to (3,-6);
    \draw (5,-6) to (5,-8)
    to [R=$R_2$, i, v_>, name=R2] (9,-8)
    to (9,-6);

    \varrmore{U0}{$U_0$};
    \iarrmore{U0}{$I_0$};
    \varrmore{U1}{$U_1$};
    \iarrmore{U1}{$I_1$};
    \iarrmore{Iq}{$I_\mathrm{q}$};
    \varrmore{R0}{$U_\mathrm{R0}$};
    \varrmore{R1}{$U_\mathrm{R1}$};
    \varrmore{R2}{$U_\mathrm{R2}$};
    \iarrmore{R2}{$I_2$};
    \varrmore{R3}{$U_\mathrm{R3}$};
    \iarrmore{R3}{$I_3$};
    \varrmore{R4}{$U_\mathrm{R4}$};
    \iarrmore{R4}{$I_4$};
    \varrmore{R5}{$U_\mathrm{R5}$};
    \iarrmore{R5}{$I_5$};
\end{circuitikz}
        }                  
    }{{\bf Beispiel.} Maschenstromanalyse an einem elektrischen Netzwerk. \label{BildMasch4}}                                            
    
        }


        \only<2>{			
            \item[a)] Vorbereitung des Netzwerkes:
			\begin{figure}[H]
                \resizebox{0.7\textwidth}{!}{%
                    \begin{circuitikz}
    \draw (-1,2) to [R=$R_0$, v_<, name=R0] (-1,-2)
    to [V_>, i, name=U0] (-1,-6)
    to (3,-6)
    to [R=$R_4$, i^<, v_<, name=R4, -*] (3,-2)
    to [R=$R_3$, i^<, v_<, name=R3, -*] (3,2)
    to [V^<, i_>, name=U1] (7,2)
    to [R=$R_1$ , v, name=R1] (11,2);
    \draw (-1,2) to (3,2);
    \draw (3,-2) to [R=$R_5$ ,i, v_>, name=R5, -*] (11,-2);
    \draw (11,2) to (11,-6)
    to [R=$R_2$ , v_<, name=R2]  (7,-6)
    to [V, i, name=U2] (3,-6);
    \draw[red,line width= 0.1 mm] (7.5,-6) circle[radius=3cm];

    \varrmore{U0}{$U_0$};
    \iarrmore{U0}{$I_0$};
    \varrmore{U1}{$U_1$};
    \iarrmore{U1}{$I_1$};
    \varrmore{U2}{$U_2$};
    \iarrmore{U2}{$I_2$};
    \varrmore{R0}{$U_\mathrm{R0}$};
    \varrmore{R1}{$U_\mathrm{R1}$};
    \varrmore{R2}{$U_\mathrm{R2}$};
    \varrmore{R3}{$U_\mathrm{R3}$};
    \iarrmore{R3}{$I_3$};
    \varrmore{R4}{$U_\mathrm{R4}$};
    \iarrmore{R4}{$I_4$};
    \varrmore{R5}{$U_\mathrm{R5}$};
    \iarrmore{R5}{$I_5$};
\end{circuitikz}	
                    }\centering                                                           
                \end{figure}
            }
        \only<3>{
            \item[b)] Maschen und Maschenströme definieren:
            \onslide<3>{
                \begin{figure}[H]
                \resizebox{0.4\textwidth}{!}{%                                                        
                    \begin{circuitikz}
    \draw[blue,->,shift={(1,-2)}] (120:1 cm) arc (120:-90:1 cm) node at(0,0){$M_1$};
    \draw[blue,->,shift={(7,0)}] (120:1 cm) arc (120:-90:1 cm) node at(0,0){$M_2$};
    \draw[blue,->,shift={(7,-4)}] (120:1 cm) arc (120:-90:1 cm) node at(0,0){$M_3$};

    \draw (-1,2) to [R=$R_0$, v_<, name=R0] (-1,-2)
    to [V_>, i, name=U0] (-1,-6)
    to (3,-6)
    to [R=$R_4$, i^<, v_<, name=R4, -*] (3,-2)
    to [R=$R_3$, i^<, v_<, name=R3, -*] (3,2)
    to [V^<, i_>, name=U1] (7,2)
    to [R=$R_1$ , v, name=R1] (11,2);
    \draw (-1,2) to (3,2);
    \draw (3,-2) to [R=$R_5$ ,i, v_>, name=R5, -*] (11,-2);
    \draw (11,2) to (11,-6)
    to [R=$R_2$ , v_<, name=R2]  (7,-6)
    to [V, i, name=U2] (3,-6);

    \varrmore{U0}{$U_0$};
    \iarrmore{U0}{$I_0$};
    \varrmore{U1}{$U_1$};
    \iarrmore{U1}{$I_1$};
    \varrmore{U2}{$U_2$};
    \iarrmore{U2}{$I_2$};
    \varrmore{R0}{$U_\mathrm{R0}$};
    \varrmore{R1}{$U_\mathrm{R1}$};
    \varrmore{R2}{$U_\mathrm{R2}$};
    \varrmore{R3}{$U_\mathrm{R3}$};
    \iarrmore{R3}{$I_3$};
    \varrmore{R4}{$U_\mathrm{R4}$};
    \iarrmore{R4}{$I_4$};
    \varrmore{R5}{$U_\mathrm{R5}$};
    \iarrmore{R5}{$I_5$};
\end{circuitikz}
                    }\centering                                                           
                \end{figure}
            \pause
            
            \begin{eq}
                I_\mathrm{M} = 
                \begin{bmatrix}  
                    I_\mathrm{M1} \\
                    \\
                    I_\mathrm{M2} \\
                    \\
                    I_\mathrm{M3} \\ 
                \end{bmatrix}\nonumber
            \end{eq}

            }
        }
        \only<4>{
            \item[c)] Widerstandsmatrix bestimmen:
            \onslide<4>{
                \begin{eq}
                    R_\mathrm{M} =
                    \begin{bmatrix}                    
                        R_0 + R_3 + R_4 & -R_3 & -R_4 \\
                        \\
                        -R_3 & R_1 + R_3 + R_5 & -R_5\\
                        \\
                        -R_4 & -R_5 & R_2 + R_4 + R_5 \\
                    \end{bmatrix} \nonumber
                \end{eq} 
            }
        }
        \only<5>{
            \item[d)] Quellspannungen zuordnen:
            \onslide<5>{
                \begin{eq}
                    U_\mathrm{M} = 
                    \begin{bmatrix} 
                        U_0 \\
                        \\
                        U_1 \\
                        \\
                        -U_2\\
                    \end{bmatrix} \nonumber
                \end{eq} 
            }
        }
        \only<6>{
            \item[e)] Gleichungssystem aufstellen:
            \onslide<6>{
                \begin{eq}
                    \begin{bmatrix} 
                        R_0 + R_3 + R_4 & -R_3 & -R_4 \\
                        \\
                        -R_3 & R_1 + R_3 + R_5 & -R_5\\
                        \\
                        -R_4 & -R_5 & R_2 + R_4 + R_5 \\
                    \end{bmatrix}   
                    \cdot
                    \begin{bmatrix}
                        I_\mathrm{M1} \\
                        \\
                        I_\mathrm{M2} \\
                        \\
                        I_\mathrm{M3} \\
                    \end{bmatrix}
                    =
                    \begin{bmatrix}
                        U_0 \\
                        \\
                        U_1 \\
                        \\
                        -U_2\\
                    \end{bmatrix}\nonumber
                \end{eq} 
                \pause
                Ströme Berechnen: 
                \begin{eqa}   
                    I_3 = I_0 - I_1\nonumber\\
                    I_4 = I_3 - I_5\nonumber\\
                    I_0 = I_4 - I_2\nonumber\\
                    -I_5 = I_2 + I_1\nonumber
                \end{eqa}  
            }
        }
    \end{enumerate}
\end{bsp}{}{}

\speech{MastromBsp}{1}{Die Schritte zur Anwendung des Maschenstromverfahrens wurden nun erläutert. 
Folgend wird anhand des abgebildeten Beispielnetzwerkes noch einmal die grundsätzliche Arbeitsweise wiederholt.
Hierzu wird das Netzwerk enstprechend für das Maschenstromverfahren vorbereitet, die Maschen und Maschenströme definiert,
die Widerstandsmatrix bestimmt, die Quellspannungen zugeordnet und anschließend das Gleichungssystem aufgestellt.}
\speech{MastromBsp}{2}{Bei dem Beispielnetzwerk müssen keine Widerstände zusammengefasst werden.
Jedoch wird die Stromquelle mit Parallelwiderstand in eine Spannungsquelle mit Serienwiderstand umgewandelt.}
\speech{MastromBsp}{3}{Für das Beispielnetzwerk werden drei Maschen M 1, M 2 und M 3 mit ihren zugehörigen 
Maschenströmen I emm eins, I emm zwei und I emm drei definiert.}
\speech{MastromBsp}{4}{An der Masche M 1 sind die Widerstände R null, R drei und R 4 beteiligt. Die Widerstände R eins, R drei und R fünf liegen an der Masche M 2. Die Masche M 3 verfügt über die Widerstände R zwei, R vier und R fünf. 
Zwischen den Maschen M eins und M zwei liegt der Widerstand R drei. Zwischen den Maschen M eins und M 3 liegt der Widerstand R vier und zwischen den Maschen M 2 und M 3 liegt der Widerstand R fünf. Die Widerstände werden in die Widerstandsmatrix R emm eingetragen.}
\speech{MastromBsp}{5}{Der Vektor der Quellspannungen ergibt sich nach U emm.
Für die Masche M 1 wird die Spannungsquelle U null aufgeschrieben.
Die Spannungsquelle U eins ist an der Masche M 2 beteiligt 
und für die Masche M 3 wird die Spannungsquelle minus U zwei notiert.}
\speech{MastromBsp}{6}{Mit der Widerstandsmatrix, dem Vektor der Maschenströme 
und dem Vektor der Quellspannungen kann das Gleichungssystem aufgestellt werden.
Nach dem Lösen der Gleichung sind die Maschenströme bekannt und können den Strömen zugeordnet werden.
Hier entspricht der Maschenstrom I emm 1 dem Strom I null.
Der Maschenstrom I emm 2 entspricht dem Strom I eins
und der Maschenstrom I emm 3 entspricht dem Strom minus I zwei.
Die übrigen Ströme des Netzwerkes können anschließend über die Knotenregel bestimmt werden,
bis letztendlich alle Größen des Netzwerkes bekannt sein sollten.}
\end{frame}
