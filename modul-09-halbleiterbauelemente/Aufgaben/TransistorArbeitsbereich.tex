\subsection{Analyse des Arbeitsbereichs eines Transistors\label{Aufg16}}
% Alt: Aufgabe 16
\Aufgabe{
  Der Stromverstärkungsfaktor des verwendeten Siliziumtransistors beträgt $ \beta = 150$ und die Versorgungsspannung liegt bei $U_\mathrm{V} = 15\,\mathrm{V}$. Die Schaltung soll so dimensioniert werden, dass der Kollektorstrom $I_\mathrm{C} = 2\,\mathrm{mA}$ beträgt.
  
  \begin{itemize}
    \item[a)] Dimensionieren Sie den Widerstand $R_3$ so, dass bei einer Spannung am Ausgang von $3\,\mathrm{V}$ der Strom durch $R_3$ genau $75\,\%$ des Emitterstroms beträgt.
    
    \item[b)] Bestimmen Sie die Widerstände $R_1$ und $R_2$, unter der Annahme, dass die Kollektor-Basis-Spannung $U_\mathrm{BE} = 0,6\,\mathrm{V}$ beträgt und durch $R_2$ etwa das 9-fache des Basisstroms $I_\mathrm{B}$ fließt.
  \end{itemize}
  \begin {figure} [H]
   \centering
   \begin{tikzpicture}
    \draw (0,0) node[npn, tr circle] (Q1) {};
    
    \draw (Q1.C) to[short, -, i<, name=ic] (0,2)
    to[short, -] ++ (-3, 0) 
    to[R=$R_1$]  (-3, 0) 
    to[R=$R_2$]  (-3, -3) 
    node[ground]{};
    
    \draw (-3,2) to[short, *-o] ++(-1,0)
    to [open, v, name=uv] (-4,-3.5); 
    
    \draw (Q1.B) to[short, -*] (-3, 0);
    
    \draw (Q1.E) to[short, -*] ++(0, -0.5) coordinate(E)
    to[R=$R_3$] (0, -3) node[ground]{};
    
    \draw (E) to[short, -] ++(2, 0)
    to[R=$R_\mathrm{L}$] (2, -3) node[ground]{};
    
    \draw (E)++(2, 0) to[short, *-o] ++(1, 0)
    to[open, v^, name=ua] (3,-3.5)    ;
    
    \varrmore{ua}{$U_\mathrm{A}$};
    \varrmore{uv}{$U_\mathrm{B}$};
    \iarrmore{ic}{$I_\mathrm{C}$};
\end{tikzpicture}
   \label{fig:FigTransistorArbeitsbereich}
\end {figure}

% \speech{Aufgabe 16  

% Der Stromverstärkungsfaktor des Siliziumtransistors beträgt **150** und die Versorgungsspannung \( U_V = 15V \).  
% Die Schaltung soll so dimensioniert werden, dass der Kollektorstrom **2 mA** beträgt.  

% **Aufgabenstellung:**  
% a) Dimensionieren Sie \( R_3 \), sodass bei einer Spannung am Ausgang von **3V** der Strom durch \( R_3 \) **75%** des Emitterstroms beträgt.  

% b) Bestimmen Sie \( R_1 \) und \( R_2 \), nehmen Sie dazu an, dass die **Kollektor-Basis-Spannung** **0,6V** beträgt und durch \( R_2 \) etwa **9 \cdot I_B** fließen.  

% **Beschreibung der Abbildung**  

% Die Abbildung zeigt eine **Transistorverstärkerschaltung** mit einem **npn-Transistor**, mehreren Widerständen und einer **Versorgungsspannung von 15V**.  

% - **Oben:** Die **Versorgungsspannung \( U_V = 15V \)** ist an den Kollektorwiderstand \( R_1 \) angeschlossen.  
% - **Transistor:** Der **Kollektor** ist über \( R_1 \) mit \( U_V \) verbunden, während der **Emitter** über \( R_3 \) und den Lastwiderstand \( R_L \) mit Masse verbunden ist.  
% - **Basisbeschaltung:** Zwei Widerstände \( R_1 \) und \( R_2 \) sorgen für eine stabile **Basisspannung** zur Steuerung des Transistors.  
% - **Lastwiderstand:** \( R_L \) ist parallel zum Ausgang \( u_A \) geschaltet, wodurch das Signal abgegriffen wird.  

% Die Aufgabe besteht darin, die Widerstände **\( R_3, R_1 \) und \( R_2 \)** so zu dimensionieren, dass die gegebenen Bedingungen erfüllt werden.  
% }  

}


\Loesung{
\begin{itemize}
    \item[a)] Berechnung des Basisstroms über den Stromverstärkungsfaktor:
\begin{equation}
    I_\mathrm{B} = \frac{I_\mathrm{C}}{\beta}
\end{equation}
\begin{align*}
    I_\mathrm{B} &= \frac{\mathrm{2\,mA}}{150} = \mathrm{13{,}3\,\mu A} \\
    I_\mathrm{E} &= I_\mathrm{C} + I_\mathrm{B} = \mathrm{2\,mA} + \mathrm{13{,}3\,\mu A} = \mathrm{2{,}013\,mA} \\
    I_\mathrm{R_3} &= 0{,}75 \cdot I_\mathrm{E} = 0{,}75 \cdot \mathrm{2{,}013\,mA} = \mathrm{1{,}51\,mA} \\
    R_3 &= \frac{\mathrm{3\,V}}{\mathrm{1{,}51\,mA}} = \mathrm{1{,}99\,k\Omega}
\end{align*}
    \item[b)] Berechnung des Stroms durch \( R_2 \):
\begin{align*}
    I_{R_2} &= 9 \cdot I_\mathrm{B} = 9 \cdot \mathrm{13{,}3\,\mu A} = \mathrm{0{,}12\,mA} \\
    R_2 &= \frac{U_\mathrm{a} + U_\mathrm{BE}}{I_\mathrm{R_2}} = \frac{\mathrm{3\,V} + \mathrm{0{,}6\,V}}{\mathrm{0{,}12\,mA}} = \mathrm{30\,k\Omega}
\end{align*}

Berechnung des Stroms durch $R_1$, welcher gleich $I_\mathrm{R_2}$ ist:
\begin{align*}
    R_1 &= \frac{U_\mathrm{BE}}{I_\mathrm{R_1}} = \frac{0,6\,\mathrm{V}}{0,12\,\mathrm{mA}} = 5,6\,\mathrm{k\Omega}
\end{align*}
\end{itemize}

  Siehe: Abschnitt \ref{sec:ElektrischesVerhaltenBipolartransistor}
%   \speech{
% Aufgabe 16 behandelt die Berechnung verschiedener Widerstände und Ströme in einer Transistorschaltung mit einem Silizium-Transistor.  

% Teil a)  
% Die Berechnung basiert auf dem Verstärkungsfaktor \(\beta\) und den bekannten Strom- und Spannungswerten.  
% - Der Basisstrom \(I_B\) wird mit der Formel \(I_B = \frac{I_C}{\beta}\) berechnet.  
% - Mit \(I_C = 2\) mA und \(\beta = 150\) ergibt sich \(I_B = 13,3\) µA.  
% - Der Emitterstrom ist die Summe aus Kollektor- und Basisstrom:  
%   \(I_E = I_C + I_B = 2,013\) mA.  
% - Der Strom durch \(R_3\) ist definiert als \(I_{R_3} = 0,75 \times I_E = 1,51\) mA.  
% - Mit \(U = 3\) V folgt für den Widerstand \(R_3 = \frac{3V}{1,51mA} = 1,99\) kΩ.  

% Teil b)  
% Hier werden die Widerstände \(R_1\) und \(R_2\) bestimmt.  
% - Der Strom durch \(R_2\) ergibt sich aus \(I_{R_2} = 9 \times I_B = 0,12\) mA.  
% - Mit \(U_A + U_{BE} = 3V + 0,6V\) berechnet sich \(R_2\) als  
%   \(R_2 = \frac{3V + 0,6V}{0,12mA} = 30\) kΩ.  
% - Für \(R_1\) gilt: \(R_1 = \frac{U_{CE}}{I_{R_1}}\), wobei \(I_{R_1} = 9 \times I_B = 0,12\) mA.  
% - Damit folgt \(R_1 = \frac{0,6V}{9 \times 13,3\) µA} = 5,6 kΩ.  
% }

}



