\subsection{Entwurf einer Verstärkerschaltung mit unterschiedlichen Eingangsverstärkungen\label{Aufg12}}
% Alt: Aufgabe 12
\Aufgabe{
    Zwischen der Ausgangsspannung $U_\mathrm{a}$ und dem Eingangsspannungen $U_\mathrm{1}$ und $U_\mathrm{2}$ besteht der folgende 
    funktionale Zusammenhang:
    $U_\mathrm{a} = 2 \ U_\mathrm{1} - 0,5 \ U_\mathrm{2}$
    Geben Sie eine mit Operationsverstärker(n) und Widerständen aufgebaute Schaltung zur Realisierung 
    dieses funktionalen Zusammenhangs an. Die Operationsverstärker dürfen als ideale Operationsverstärker 
    angenommen werden. Geben Sie Werte der verwendeten Widerstände an! Typische Widerstände für OP-Schaltungen liegen im 
    Bereich von $\mathrm{1\ bis\ 100\,k\Omega}$.

  %  \speech{
  %      Aufgabe 12: Funktionale Umsetzung mit Operationsverstärkern  
  %      
  %      Gegeben ist eine funktionale Beziehung zwischen der Ausgangsspannung U A  
  %      und den Eingangsspannungen U1 und U2:  
  %      
  %      U A = 2 U1 - 0,5 U2  
  %      
  %      Es soll eine Schaltung mit Operationsverstärker(n) und Widerständen entworfen werden,  
  %      die genau diese mathematische Beziehung realisiert.  
  %      Dabei dürfen die Operationsverstärker als ideal angenommen werden.  
  %      
  %      Zusätzlich sollen die Werte der verwendeten Widerstände angegeben werden.  
  %      Typische Widerstandswerte für Operationsverstärkerschaltungen liegen im Bereich von 1 bis 100 Kiloohm.  
  %      
  %      Die Aufgabe besteht darin, die geeignete Verschaltung der Bauelemente zu finden,  
 %       um die gewünschte Ausgangsspannung zu erzeugen.  
%        }
        
}

\Loesung{
	\begin {figure} [H]
\centering
\begin{tikzpicture}
    \ctikzset{tripoles/en amp/input height=0.45}
    \draw (0,0)node[en amp, en amp text A](E){}
        (E.out)
        (E.-)
        (E.+);
        \draw (E.out) to[short, -*] (2,0) to[short, -o] (2.5,0);
        \draw (E.-) to[short, -*] (-2,0.5) to[R, l_=$R_\mathrm{1}$] (-4,0.5) to[short, -o] (-5,0.5);
        \draw (-2,0.5) to[short] (-2,2.5) to[R, name=R, l=$R_\mathrm{3}$] (2,2.5) -- (2,0);
    
    
        
        \draw (-5,-2) to[short, o-o] (-4,-2) -- (-2,-2);
        \draw (E.+) -- (-2,-0.5) to[R, l=$R_\mathrm{4}$, -*] (-2,-2)node[ground]{};
        \draw (-2,-0.5) to[R,l_=$R_\mathrm{2}$, *-o] (-4,-0.5);
        \draw (2.5,-2) to[short, o-] (-2,-2);
        
        \draw (2.5,0) to[open, v, name=ua] (2.5,-2);
        \draw (-4,-0.5) to[open, v, name=ue2] (-4,-2);
        \draw (-5,0.5) to[open, v, name=ue1] (-5,-2);
        
        
    
        \varrmore{ue1}{$U_\mathrm{e1}$};
        \varrmore{ue2}{$U_\mathrm{e2}$};
        \varrmore{ua}{$U_\mathrm{a}$};   

\end{tikzpicture}
\label{LsgOPV2Signale2}
 \end {figure}

Der funktionale Zusammenhang zwischen der Ausgangsspannung \( U_a \) und den Eingangsspannungen \( U_{e1} \) und \( U_{e2} \) lautet:
\[
U_\mathrm{a} = 2 U_\mathrm{e1} - 0{,}5 U_\mathrm{e2}
\]

\textbf{1. Allgemeine Formel für die Ausgangsspannung eines Subtrahierers:}
\[
U_\mathrm{a} = U_\mathrm{e2} \cdot \frac{R_1 + R_3}{R_2 + R_4} \cdot \frac{R_4}{R_1} - U_\mathrm{e1} \cdot \frac{R_3}{R_1}
\]
Vergleich mit der Vorgabe \( U_\mathrm{a} = 2 U_\mathrm{e1} - 0{,}5 U_\mathrm{e2} \) ergibt folgende Bedingungen: \\
- Verstärkungsfaktor für \( U_\mathrm{e1} \): \( \frac{R_3}{R_1} = 0{,}5 \) \\
- Verstärkungsfaktor für \( U_\mathrm{e2} \): \( \frac{R_1 + R_3}{R_2 + R_4} \cdot \frac{R_4}{R_1} = 2 \)



\textbf{2. Vorgehen:} \\
- Setze \( R_3 = 1\,\mathrm{k}\Omega \) (typischer Wert). \\
- Berechne die restlichen Widerstände.



\textbf{3. Berechnung der Widerstände:}

\textbf{Berechnung von \( R_1 \) (aus \( \frac{R_3}{R_1} = 0{,}5 \)):}
\[
\frac{R_3}{R_1} = 0{,}5 \quad \Rightarrow \quad R_1 = \frac{R_3}{0{,}5} = \frac{1\,\mathrm{k}\Omega}{0{,}5} = 2\,\mathrm{k}\Omega
\]

\textbf{Berechnung von \( R_2 \) und \( R_4 \) (aus \( \frac{R_1 + R_3}{R_2 + R_4} \cdot \frac{R_4}{R_1} = 2 \)):}
Setze \( R_4 = 1\,\mathrm{k}\Omega \):
\[
\frac{R_1 + R_3}{R_2 + R_4} \cdot \frac{R_4}{R_1} = 2 \quad \Rightarrow \quad \frac{3\,\mathrm{k}\Omega}{R_2 + 1\,\mathrm{k}\Omega} \cdot \frac{1\,\mathrm{k}\Omega}{2\,\mathrm{k}\Omega} = 2
\]
Multipliziere beide Seiten:
\[
\frac{3}{2} = 2 \cdot (R_2 + 1\,\mathrm{k}\Omega)
\]
Umstellen nach \( R_2 \):
\[
R_2^2 + R_2 - \frac{3}{2} = 0
\]



\textbf{4. Lösung mit der PQ-Formel:}

Die quadratische Gleichung hat die Form:
\[
R_2^2 + p \cdot R_2 + q = 0 \quad \text{mit} \quad p = 1, \, q = -\frac{3}{2}
\]
PQ-Formel:
\[
R_2 = -\frac{p}{2} \pm \sqrt{\left( \frac{p}{2} \right)^2 - q}
\]
Einsetzen der Werte:
\[
R_2 = -\frac{1}{2} \pm \sqrt{\left( \frac{1}{2} \right)^2 - \left( -\frac{3}{2} \right)} = -\frac{1}{2} \pm \sqrt{\frac{1}{4} + \frac{3}{2}}
\]
\[
R_2 = -\frac{1}{2} \pm \sqrt{\frac{7}{4}} = -\frac{1}{2} \pm \frac{\sqrt{7}}{2}
\]
Da \( R_2 \) positiv sein muss:
\[
R_2 = -\frac{1}{2} + \frac{\sqrt{7}}{2} \approx 1{,}82\,\mathrm{k}\Omega
\]


\textbf{5. Ergebnis der Widerstände:}
\begin{align*}
R_1 &= 2\,\mathrm{k}\Omega \\
R_2 &= 1{,}82\,\mathrm{k}\Omega \\
R_3 &= 1\,\mathrm{k}\Omega \\
R_4 &= 1\,\mathrm{k}\Omega
\end{align*}
Siehe: Abschnitt \ref{fig: Verschaltung Verstaerker} und Tabelle \ref{tab:Grundschaltungen}
}