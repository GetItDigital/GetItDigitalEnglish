% R in Reihe mit (R parallel C), Ladeschaltung mit U_q=konst. 
\begin{tikzpicture}
    
    \ctikzset{bipoles/cuteswitch/thickness=0.3} %Liniendicke für bessere Lesbarkeit reduzieren
    
    %Switch Circuit
    \draw(2,1.69)
        node[cute spdt up arrow, rotate = 180](s1){}% Cute Switch, Position des Schalters mit up/down veränderbar
        (s1.center) node[left, xshift= -5]{t=0};%Label 
    \draw(s1.out 2) -- (0.5,2.11);% nach Uq+
    \draw([yshift=-0.5mm] s1.cout 1) to[short,-*] (1.70,-0.5);%nach Uq-
    %R+(R||C) Circuit
    \draw(0.5,2.11) 
        to[V, name=Uq, v_>=$U_q$, !v](0.5,-0.5)
        to[short](6.25,-0.5) 
        to[C, l=$C$, name=C, v<=$u_C$, !vi](6.25,1.685)
        to[short, name=iC, i<_=$i_C$, -*, !i](5,1.685)
        to[R, l=$R_1$, name=R1, i^<=$i_1$, !i](s1.in);
    \draw(5,1.685)
        to[R, l_=$R_2$, name=R2, i>_=$i_2$, -*, !i](5,-0.5);
    % Spannungspfeile
    \varronly{Uq};
    \varronly{C};
    \iarronly{R1};
    \iarronly{R2};
    \iarronly{iC};
\end{tikzpicture} 