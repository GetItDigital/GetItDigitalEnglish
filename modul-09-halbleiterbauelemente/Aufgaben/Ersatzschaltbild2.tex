\subsection{Erstellung eines Kleinsignal-Ersatzschaltbilds\label{Aufg21}}
% Alt: Aufgabe 21
\Aufgabe{
    Zeichnen Sie das Kleinsignalersatzschaltbild der gegebenen Schaltung.
    \begin {figure} [H]
       \centering
       \begin{tikzpicture}
    \draw (0,0) node[npn, tr circle] (Q1) {};

    \draw (Q1.C) to[short, -o, name=ic] (0,2.5)
    node[right, color=voltage]{$U_\mathrm{V}$};

    \draw (0,2)
    to[short, *-] (-3, 2) 
    to[R=$R_1$, -*]  (-3, 0) 
    to[R=$R_2$]  (-3, -3) 
    node[ground]{};

    \draw (Q1.B) 
    to[short, -o] (-5, 0)
    to[open, v, name=ue] (-5,-2.75)    
    to[short, o-] (-5, -3)
    node[ground]{};

    \draw (Q1.E) to[short, -*] ++(0, -0.25)
    to[R=$R_3$] (0, -3) node[ground]{};

    \draw (Q1.E)++(0, -0.25) to[short, -o] ++(2, 0)
    to[open, v^, name=ua] (2,-2.75)    
    to[short, o-] (2,-3)
    node[ground]{};
    
    \varrmore{ue}{$U_\mathrm{E}$};
    \varrmore{ua}{$U_\mathrm{A}$};
\end{tikzpicture}
       \label{fig:FigErsatzschaltbild2}
    \end {figure}

%     \speech{
% **A.21 Aufgabe 21**

%  **Aufgabenstellung:**
% Die Aufgabe besteht darin, das **Kleinsignal-Ersatzschaltbild** der gezeigten Transistorschaltung zu zeichnen.



%  **Beschreibung der Abbildung:**
% Die Abbildung zeigt eine **Transistorverstärkerschaltung in Emitterschaltung mit Basisspannungsteiler**.

%  **1. Spannungsversorgung:**
% - Die Schaltung wird mit einer **Versorgungsspannung \( U_V \)** betrieben, die sich am oberen Rand der Schaltung befindet.
% - Diese Spannung versorgt den **Kollektor** des Transistors über einen **Widerstand \( R_1 \)**.

%  **2. Basisbeschaltung mit Spannungsteiler:**
% - Die Basis des Transistors erhält ihre Vorspannung über **zwei Widerstände \( R_1 \) und \( R_2 \)**.  
% - Diese beiden Widerstände bilden einen **Spannungsteiler**, der die **Basisspannung einstellt**.
% - \( R_1 \) ist zwischen \( U_V \) und der Basis geschaltet, während \( R_2 \) zwischen Basis und Masse liegt.

%  **3. Kollektorkreis:**
% - Der **Kollektor ist über den Widerstand \( R_1 \)** mit der **Versorgungsspannung \( U_V \)** verbunden.
% - Die **Ausgangsspannung \( u_A \)** wird am **Kollektor des Transistors** abgegriffen.

%  **4. Emitterschaltung mit Widerstand \( R_3 \):**
% - Der **Emitter des Transistors** ist mit einem **Widerstand \( R_3 \)** gegen Masse verbunden.
% - Dieser Widerstand beeinflusst die **Verstärkung der Schaltung**.

%  **5. Eingangssignal:**
% - Das **Eingangssignal \( u_E \)** wird über eine separate Leitung an die **Basis des Transistors** eingespeist.

%  **Zusätzliche Hinweise für eine blinde Person:**
% - Die gezeigte Schaltung ist eine **Emitterschaltung mit Basisspannungsteiler**.  
% - Das **Kleinsignal-Ersatzschaltbild** wird durch Ersetzen des Transistors durch sein **äquivalentes Modell** erstellt.  
%   - Dabei werden Parameter wie **differentieller Widerstand \( r_e \)** und **Stromverstärkung \( \beta \)** berücksichtigt.
% - Die Spannungsteiler \( R_1 \) und \( R_2 \) bestimmen die **Gleichspannungsvorspannung der Basis**, während \( R_3 \) die **Gegenkopplung am Emitter** definiert.
% - Das **Eingangssignal \( u_E \)** wird an der **Basis eingespeist**, und die verstärkte Version erscheint als **invertiertes Signal am Kollektor**.
% - Der **Widerstand \( R_3 \)** beeinflusst die **Gesamtverstärkung der Schaltung**, indem er die Emitterspannung anhebt.

% Diese Schaltung wird häufig in **Verstärkerschaltungen** verwendet, um ein **stabiles Arbeitspunktverhalten** zu ermöglichen.
% }

}



\Loesung{
    \begin{figure}[H]
        \centering
        \begin{subfigure}[b]{0.8\textwidth}
            \centering
            \begin{tikzpicture}

\draw (0,0) to[short, o-*] (1,0) to[short, -*] (2,0) to[short, -o] (3,0)node[above]{B} --(4,0) to[R,l=$r_\mathrm{BE}$] (4,-2)node[below]{E} to[short,-*] (6,-2) to[I] (6,0) to[short, -*] (7,0) to[R,l=$r_\mathrm{CE}$] (7,-2) -- (6,-2);
\draw (7,0) to[short,-o] (8,0)node[above]{C} -- (9,0) --(9,-4) to[short, -o] (8,-4) to[short, -*] (6,-4) -- (3,-4);

\draw (0,-2) to[short, o-*] (1,-2)node[ground]{} to[short, -*] (2,-2) -- (3,-2) -- (3,-4);
\draw (8,0) to[open,v>, name=ua] (8,-4);

\draw (1,0) to[R,l=$R_\mathrm{1}$](1,-2);
\draw (2,0) to[R,l=$R_\mathrm{2}$](2,-2);
\draw (6,-2) to[R,l=$R_\mathrm{3}$](6,-4);
\draw (0,0) to[open, v>, name=ue] (0,-2);

\varrmore{ua}{$U_\mathrm{a}$};
\varrmore{ue}{$U_\mathrm{e}$};

\end{tikzpicture}
            \label{fig:LsgErsatschaltbild2_1}
        \end{subfigure}
        \vspace{0.5cm} 
        \begin{subfigure}[b]{0.8\textwidth}
            \centering
            \begin{tikzpicture}
\draw (0,0) to[short, o-*] (1,0)node[above]{B} to[short, -*] (2,0) to[R,l=$r_\mathrm{BE}$, -*] (4,0) to[short, -*] (6,0)node[above]{E} to[short,-*] (7,0) to[short, -o] (9,0);
\draw (0,-2) to[short, o-*] (1,-2) -- (1.5,-2)node[ground]{};
\draw (1.5,-2) to[short, -*] (2,-2) to[short, -*] (4,-2) to[short, -*] (6,-2) to[short,-*] (7,-2) to[short, -o] (9,-2);


\draw (9,0) to[open,v>, name=ua] (9,-2);

\draw (1,0) to[R,l=$R_\mathrm{1}$](1,-2);
\draw (2,0) to[R,l=$R_\mathrm{2}$](2,-2);
\draw (4,0) to[R,l=$R_\mathrm{3}$](4,-2)node[below]{C};
\draw (6,0) to[I] (6,-2);
\draw (7,0) to[R,l=$r_\mathrm{CE}$](7,-2);

\draw (0,0) to[open, v>, name=ue] (0,-2);

\varrmore{ua}{$U_\mathrm{a}$};
\varrmore{ue}{$U_\mathrm{e}$};

\end{tikzpicture}
            \label{fig:LsgErsatschaltbild2_2}
        \end{subfigure}
        \label{fig:Loesung21}
    \end{figure}
    Siehe: Abschnitt \ref{fig:ErsatzschaltbildEinesBipolartransistors} und \ref{fig:ErsatzschaltbilderEmitter-Kollektor-UndBasisschaltung}.
%     \speech{
% Die Abbildung zeigt zwei verschiedene Darstellungen einer Transistorschaltung. Die obere Abbildung stellt die vollständige Schaltung eines bipolaren Transistors in Emitterschaltung dar, während die untere Abbildung die Klein-Ersatzschaltung dieses Systems zeigt.

% **Beschreibung der oberen Abbildung (vollständige Transistorschaltung):**  
% - Links befindet sich die Eingangsspannung \( U_e \), die in die Schaltung eingespeist wird.  
% - Zwei Widerstände, \( R_1 \) und \( R_2 \), sind parallel zwischen der Eingangsspannung \( U_e \) und Masse geschaltet.  
% - Diese Widerstände \( R_1 \) und \( R_2 \) legen das Basispotenzial des Transistors fest.  
% - Der Transistor befindet sich mittig in der Schaltung und ist mit seinen drei Anschlüssen wie folgt beschaltet:  
%   - **Basis (B):** Die Basis ist über die beiden Widerstände \( R_1 \) und \( R_2 \) an die Eingangsspannung \( U_e \) gekoppelt. Zudem gibt es eine gekennzeichnete Impedanz \( r_{BE} \), die den Basis-Emitter-Widerstand modelliert.  
%   - **Emitter (E):** Der Emitter ist mit einem Widerstand \( R_3 \) verbunden, der nach unten zur Masse führt.  
%   - **Kollektor (C):** Der Kollektor ist mit einem weiteren Widerstand \( r_{CE} \) verbunden, der zum Ausgang \( U_a \) führt.  
% - Die Ausgangsspannung \( U_a \) befindet sich rechts in der Schaltung und ist mit dem Kollektor verbunden.  

% Die Pfeile in der Schaltung zeigen die Richtung der Spannungen:  
% - **\( U_e \) (Eingangsspannung):** Von oben links nach unten gerichtet.  
% - **\( U_a \) (Ausgangsspannung):** Vom Kollektor des Transistors nach unten.

% **Beschreibung der unteren Abbildung (Klein-Ersatzschaltbild):**  
% - Diese Darstellung zeigt eine vereinfachte Version der oberen Schaltung, in der der Transistor durch seine klein-signaltechnischen Ersatzwerte ersetzt wurde.  
% - Die Basis ist weiterhin mit den Widerständen \( R_1 \) und \( R_2 \) verbunden.  
% - Der Basis-Emitter-Übergang wird durch einen Widerstand \( r_{BE} \) modelliert.  
% - Der Kollektor-Emitter-Widerstand \( r_{CE} \) stellt den Durchgangswiderstand des Transistors dar.  
% - Der Emitter ist über \( R_3 \) mit Masse verbunden.  
% - Die Spannungen \( U_e \) und \( U_a \) sind weiterhin links bzw. rechts gekennzeichnet.  

% Zusammengefasst zeigen diese beiden Abbildungen die Entwicklung von einer vollständigen Transistorschaltung hin zu ihrem Klein-Ersatzschaltbild, das für weitergehende Berechnungen und Analysen verwendet wird.
% }

}



