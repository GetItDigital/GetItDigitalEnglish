
\subsection{Arbeit und Energie\label{Arbeit und Energie}}
\Aufgabe{
	
    Ein Elektromotor hebt eine Masse von $m = 10kg$ um eine Höhe von $5m$.

    \begin{itemize}
        \item[\bf a)]
        Berechnen Sie die verrichtete Arbeit.
        \item[\bf b)]
        Wieviel Energie wird benötigt, wenn der Wirkungsgrad des Motors $\eta = 80\%$ beträgt? 
    \end{itemize}
}

\Loesung{
	\begin{itemize}
	
		\item[\bf a)]
        Berechnen Sie die verrichtete Arbeit:
		\begin{equation*}
		    W = m\cdot g \cdot h 
		\end{equation*}
		\begin{equation*}
		    W = 10 kg \cdot 9,81 \frac{m}{s^2}\cdot 5 m
		\end{equation*}
		\begin{equation*}
			W = 490,5J
		\end{equation*}
		\item[\bf b)]
		Wieviel Energie wird benötigt, wenn der Wirkungsgrad des Motors $\eta = 80\%$ beträgt?
		\begin{equation*}
		    Wirkungsgrad = \frac{genutzte Energie}{aufgewendete Energie} = \eta 
		\end{equation*}
		\begin{equation*}
		    E_{zu}=\frac{W}{\eta}=\frac{490,5J}{80\%}=613,125J
		\end{equation*}
	\end{itemize}
	}

\subsection{Energie und Leistung\label{Energie und Leistung}}
\Aufgabe{
    Ein Heizgerät hat eine Leistungsaufnahme von $P = 2kW$ und wird für $t = 3h$ betrieben. 

    \begin{itemize}
        \item[\bf a)]
        Wie viel Energie wird verbraucht?  
        \item[\bf b)]
        Wie hoch sind die Kosten, wenn der Strompreis bei $0,30\,Euro/kWh$ liegt?
    \end{itemize}
}

\Loesung{
	\begin{itemize}
		
		\item[\bf a)]
        Wie viel Energie wird verbraucht?  
		\begin{equation*}
		    E = P\cdot t
		\end{equation*}
		\begin{equation*}
			E=2kW \cdot 3h=6kWh
		\end{equation*}

		\item[\bf b)]
		Wie hoch sind die Kosten, wenn der Strompreis bei $0,30\,Euro/kWh$ liegt?
		\begin{equation*}
			C=E \cdot p
		\end{equation*}
		\begin{equation*}
			C=6kWh \cdot 0,30 \frac{Euro}{kWh} = 1,80\,Euro
		\end{equation*}
	\end{itemize}
	}

\subsection{Wirkungsgrad\label{Wirkungsgrad}}
\Aufgabe{
    Ein Generator erzeugt eine elektrische Leistung von $P_el = 1kW$, indem er eine mechanische Leistung von $P_mech = 1,5kW$ aufnimmt.

    \begin{itemize}
        \item[\bf a)]
        Berechnen Sie den Wirkungsgrad.
        \item[\bf b)]
        Wie groß muss $P_mech$ sein, um $P_el = 2kW$ zu erzeugen?
    \end{itemize}
}

\Loesung{
	\begin{itemize}
		
		\item[\bf a)]
        Berechnen Sie den Wirkungsgrad.
		\begin{equation*}
			\eta=\frac{P_{mech}}{P_{el}}
		\end{equation*}
		\begin{equation*}
			\eta=\frac{1kW}{1,5kW}=\frac{2}{3}\approx 66,67 \%
		\end{equation*}
		\item[\bf b)]
		Wie groß muss $P_mech$ sein, um $P_el = 2kW$ zu erzeugen?
		\begin{equation*}
			P_{mech}=\frac{P_{el}}{\eta}
		\end{equation*}
		\begin{equation*}
			P_{mech}=\frac{2kW}{0,6667} \approx 3kW
		\end{equation*}
	\end{itemize}
	}

	\subsection{Leitfähigkeit und Widerstand\label{Leitfähigkeit und Widerstand}}
\Aufgabe{
	
Drei Widerstände $R_1=4\Omega, R_2=6\Omega, R_3=12\Omega$ sind folgendermaßen geschaltet: $R_1$ parallel zu $R_2$, der Ersatzwiderstand in Reihe zu $R_3$.

    \begin{itemize}
        \item[\bf a)]
        Wie lautet der Rest der Frage?
        \item[\bf b)]
        Was könnte man dazu fragen? 
    \end{itemize}
}

\Loesung{
	\begin{itemize}
		
		\item[\bf a)]
        Wie lautet der Rest der Frage?
		\begin{equation*}
			R=\rho \cdot \frac{l}{A}
		\end{equation*}
		\begin{equation*}
			R=1,68 \cdot10^{-8}Ωm \cdot \frac{5m}{1 \cdot 10^{-6} m^2}
		\end{equation*}
		\begin{equation*}
			R=0,084 \Omega
		\end{equation*}
		\item[\bf b)]
		Was könnte man dazu fragen? 
		\begin{equation*}
			P_{mech}=\frac{P_{el}}{\eta}
		\end{equation*}
		\begin{equation*}
			P_{mech}=\frac{2kW}{0,6667} \approx 3kW
		\end{equation*}
	\end{itemize}
	}

\subsection{Materialabhängigkeit des Widerstandes\label{Materialabhängigkeit des Widerstandes}}
\Aufgabe{
    Ein Draht aus Kupfer hat eine Länge von $l = 5m$ und einen Querschnitt von $A = 1mm^2$. Kupfer hat einen spezifischen Widerstand von $\rho = 1,68 \cdot10^{-8}\Omega m$.

    \begin{itemize}
        \item[\bf a)]
        Berechnen Sie den Widerstand $R$ des Drahtes.
        \item[\bf b)]
        Was kann man hier noch fragen?
    \end{itemize}
}

\Loesung{
	\begin{itemize}
		
		\item[\bf a)]
        Berechnen Sie den Widerstand $R$ des Drahtes.
		\begin{equation*}
			R=\rho \cdot \frac{l}{A}
		\end{equation*}
		\begin{equation*}
			R=1,68 \cdot 10^{-8}\Omega m \cdot \frac{5m}{1 \cdot 10^{-6} m^2}
		\end{equation*}
		\begin{equation*}
			R=0,084 \Omega
		\end{equation*}
		\item[\bf b)]
		Was kann man noch fragen?
		\begin{equation*}
			P_{mech}=\frac{P_{el}}{\eta}
		\end{equation*}
		\begin{equation*}
			P_{mech}=\frac{2kW}{0,6667} \approx 3kW
		\end{equation*}
	\end{itemize}
	}

	\subsection{Spannungs- und Stromquellen\label{Spannungs und Stromquellen}}
\Aufgabe{
    Eine Spannungsquelle liefert $U=12V$ an einen Widerstand $R = 4\Omega$.

    \begin{itemize}
        \item[\bf a)]
        Berechnen Sie den Strom $I$.
        \item[\bf b)]
        Berechnen Sie die am Widerstand umgesetzte Leistung $P$.
    \end{itemize}
}

\Loesung{
	\begin{itemize}
		
		\item[\bf a)]
        Berechnen Sie den Strom $I$.
		\begin{equation*}
			I=\frac{U}{R}
		\end{equation*}
		\begin{equation*}
			I=\frac{12V}{4\Omega}=3A
		\end{equation*}
		\item[\bf b)]
		Berechnen Sie die am Widerstand umgesetzte Leistung $P$.
		\begin{equation*}
			P=U \cdot I
		\end{equation*}
		\begin{equation*}
			P=\frac{U^2}{R}
		\end{equation*}
		\begin{equation*}
			P=12V \cdot 3A = 36W
		\end{equation*}
	\end{itemize}
	}

	\subsection{Spannungs- und Stromquellen\label{Spannungs und Stromquellen}}
\Aufgabe{
    Eine reale Stromquelle hat eine Leerlaufspannung $U_0 = 9V$ und einen Innenwiderstand $R_i = 1\Omega$.

    \begin{itemize}
        \item[\bf a)]
        Wie groß ist die Klemmspannung, wenn ein Lastwiderstand von $R_L = 4\Omega$ angeschlossen wird?
        \item[\bf b)]
        Wie groß ist der Strom durch die Last?
    \end{itemize}
}

\Loesung{
	\begin{itemize}
		
		\item[\bf a)]
        Wie groß ist die Klemmspannung, wenn ein Lastwiderstand von $R_L = 4\Omega$ angeschlossen wird?
		\begin{equation*}
			U_{Kl}=U_0 \cdot \frac{R_L}{R_i+R_L}
		\end{equation*}
		\begin{equation*}
			U_{Kl}=9V \cdot \frac{4\Omega}{1\Omega+4\Omega}=7,2\Omega
		\end{equation*}
		\item[\bf b)]
		Wie groß ist der Strom durch die Last?

	\end{itemize}
	}


	\subsection{Spannungs- und Stromquellen\label{Spannungs und Stromquellen}}
\Aufgabe{
    Gegeben ist ein Netzwerk aus einer Spannungsquelle mit $U=10V$, einem Reihenwiderstand $R_1 = 2\Omega$ und einem parallelen Widerstand $R_2 = 4\Omega$.

    \begin{itemize}
        \item[\bf a)]
        Berechnen Sie alle Ströme und Spannungen im Netzwerk.

    \end{itemize}
}

\Loesung{
	\begin{itemize}
		
		\item[\bf a)]
		Berechnen Sie alle Ströme und Spannungen im Netzwerk.
		
	\end{itemize}
	}

	\subsection{Kapazität und Kondensator\label{Kapazitaet und Kondensator}}
\Aufgabe{
    Ein Kondensator mit einer Kapazität von $C=10pF$ wird von einer $5V$ Spannungsquelle getrennt.

    \begin{itemize}
        \item[\bf a)]
        Wie groß ist die Ladung $Q$ auf den Kondensatorplatten?
        \item[\bf b)]
        Wie groß ist die im Kondensator gespeicherte Energie?
    \end{itemize}
}

\Loesung{
	\begin{itemize}
		
		\item[\bf a)]
        Wie groß ist die Ladung $Q$ auf den Kondensatorplatten?
		\begin{equation*}
			I=\frac{U}{R}
		\end{equation*}
		\begin{equation*}
			I=\frac{12V}{4\Omega}=3A
		\end{equation*}
		\item[\bf b)]
		Wie groß ist die im Kondensator gespeicherte Energie?
		\begin{equation*}
			P=U \cdot I
		\end{equation*}
		\begin{equation*}
			P=\frac{U^2}{R}
		\end{equation*}
		\begin{equation*}
			P=12V \cdot 3A = 36W
		\end{equation*}
	\end{itemize}
	}

	\subsection{Spule und Induktivität\label{Spule und Induktivitaet}}
	\Aufgabe{
		Eine Spule mit $L=2H$ trägt Strom von $I=3A$.
	
		\begin{itemize}
			\item[\bf a)]
			Wie hoch ist die magnetische Energie in der Spule?
			\item[\bf b)]
			Wie ändert sich die Energie bei einem Strom von $I=5A$?
		\end{itemize}
	}
	
	\Loesung{
		\begin{itemize}
			
			\item[\bf a)]
			Wie hoch ist die magnetische Energie in der Spule?
			\begin{equation*}
				I=\frac{U}{R}
			\end{equation*}
			\begin{equation*}
				I=\frac{12V}{4\Omega}=3A
			\end{equation*}
			\item[\bf b)]
			Wie ändert sich die Energie bei einem Strom von $I=5A$?
			\begin{equation*}
				P=U \cdot I
			\end{equation*}
			\begin{equation*}
				P=\frac{U^2}{R}
			\end{equation*}
			\begin{equation*}
				P=12V \cdot 3A = 36W
			\end{equation*}
		\end{itemize}
		}

		\subsection{Spule und Induktivität\label{Spule und Induktivitaet}}
		\Aufgabe{
			Eine zylindrische Spule hat $N=500$ Wicklungen, eine Länge von $l=20cm$ und einen Querschnitt von $A=5cm^2$. Das innere der Spule ist Luftgefüllt.
		
			\begin{itemize}
				\item[\bf a)]
				Wie groß ist die Induktivität $L$?
				\item[\bf b)]
				Wie hoch ist die magnetische Energie in der Spule?
			\end{itemize}
		}
		
		\Loesung{
			\begin{itemize}
				
				\item[\bf a)]
				Wie groß ist die Induktivität $L$?
				\begin{equation*}
					L=\mu_0 \cdot \frac{N^2\cdot A}{l}
				\end{equation*}
				\begin{equation*}
					L=4 pi \cdot 10^{-7} \frac{H}{m} \cdot \frac{500^2\cdot 5 \cdot 10^{-4}m^2}{0,2m}=0,785mH
				\end{equation*}
				\item[\bf b)]
				Wie hoch ist die magnetische Energie in der Spule?
				\begin{equation*}
					P=U \cdot I
				\end{equation*}
				\begin{equation*}
					P=\frac{U^2}{R}
				\end{equation*}
				\begin{equation*}
					P=12V \cdot 3A = 36W
				\end{equation*}
			\end{itemize}
			}

			\subsection{Leitfähigkeit\label{Leit}}
			\Aufgabe{
				Ein runder Aluminiumdraht besitzt einen Radius von $a=0,5mm$. Er ist mit einer dünnen Goldschicht von $b=0,1mm$ Dicke überzogen. Für die spezifische elektrische Leitfähigkeit und den Temperaturkoeffizienten der beiden Metalle gelten folgende Werte:
			
				\begin{itemize}
					\item $\kappa_{Alu}=35\frac{m}{\Omega \cdot mm^2}$
					\item $\alpha_{Alu}=3,5 \cdot 10^{-3}K^{-1}$
				\end{itemize}
				\begin{itemize}
					\item $\kappa_{Gold}=44\frac{m}{\Omega \cdot mm^2}$
					\item $\alpha_{Gold}=3,6 \cdot 10^{-3}K^{-1}$
				\end{itemize}

				\begin{itemize}
					\item[\bf a)]
					Berechnen Sie den elektrischen Gleichstromwiderstand des Aluminiumkerns und der Goldummantelung bei einer Temperatur von $25C$, wenn die Drahtlänge $l=1,5m$.
					\item[\bf b)]
					Wie verändert sich der jeweilige Gleichstromwiderstand, wenn die Temperatur auf $90C$ ansteigt?
				\end{itemize}
			}
			
			\Loesung{
				\begin{itemize}
					
					\item[\bf a)]
					Wie groß ist die Induktivität $L$?
					\begin{equation*}
						L=\mu_0 \cdot \frac{N^2\cdot A}{l}
					\end{equation*}
					\begin{equation*}
						L=4 pi \cdot 10^{-7} \frac{H}{m} \cdot \frac{500^2\cdot 5 \cdot 10^{-4}m^2}{0,2m}=0,785mH
					\end{equation*}
					\item[\bf b)]
					Wie hoch ist die magnetische Energie in der Spule?
					\begin{equation*}
						P=U \cdot I
					\end{equation*}
					\begin{equation*}
						P=\frac{U^2}{R}
					\end{equation*}
					\begin{equation*}
						P=12V \cdot 3A = 36W
					\end{equation*}
				\end{itemize}
				}