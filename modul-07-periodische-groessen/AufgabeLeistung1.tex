\subsection{Leistungsberechnung 1}
\Aufgabe{

    Text
    
    \begin{figure}[H]
        \centering
        \resizebox{0.7\textwidth}{!}{
            \begin{circuitikz}[domain=0:8,samples=100]
    %Gitter
    \draw[very thin,color=gray] (-0.1,-2.1) grid (8.2,2.1);
    \draw[->] (-0.2,0) -- (8.5,0) node[right] {t in ms};
    \draw[->] (0,-2.1) -- (0,2.2) node[above] {};
    \draw (0,-2.1) node[below] {$0$};
    \draw (2,-2.1) node[below] {$25$};
    \draw (4,-2.1) node[below] {$50$};
    \draw (6,-2.1) node[below] {$75$};
    \draw (8,-2.1) node[below] {$100$};
    \draw[color=blue] (-0.5,1.5) node[] {$u(t)$};
    \draw[color=red] (-0.5,-1) node[] {$i(t)$};
    \draw[color=blue,smooth] plot[id=Sinus14] function{1.5 * sin(pi/2 * x)} node[right] {};
    \draw[color=red,smooth] plot[id=Sinus15] function{2 * sin((pi/2) * x - (pi/3))} node[right] {};
\end{circuitikz}
        }
    \end{figure}
    
    \begin{itemize}
        \item[\bf a)] Ablesen des Betrages der Phasenwinkel von Spannung und Strom. Bennenung des Verhaltens.
        \item[\bf b)] Angabe von Periodendauer, Frequenz und Kreisfrequenz.
        \item[\bf c)] Berechnung der Augenblicksleistung zum Zeitpunkt t = 12,5\ ms.
        \item[\bf d)] Berechnung von Wirkleistung und Blindleistung.
        \item[\bf e)] Berechnung der Scheinleistung mit dem konjugierten komplexen Strom.
    \end{itemize}
    
}

\Loesung{
    \begin{itemize}

        \item[\bf a)]
              Die Spannung weist einen Phasenwinkel von $\varphi_\mathrm{u}=0$ und der Strom weist einen Phasenwinkel von $\varphi_\mathrm{i}=\pi/3$ auf. \\
              Das Nacheilen des Stromes deutet auf ein {\bf induktives Verhalten} hin. 
              
        \item[\bf b)]
              Die Periodendauer T beträgt $50\ ms$. \\
              Die Frequenz beträgt:
              \begin{eq}
                  f=\frac{1}{T}=\frac{1}{50\ ms}= 20\ Hz   \nonumber
              \end{eq}
              Die Kreisfrequenz beträgt:
              \begin{eq}
                  \omega = 2\pi \cdot f = 2\pi \cdot 20\ Hz = 125,66\ Hz  \nonumber
              \end{eq}
              
        \item[\bf c)]
              Die Augenblicksleistung zum Zeitpunkt t = 12,5\ ms beträgt:
              \begin{eqa}
                  p(t) &= u(t) \cdot i(t) = 12\ V \cdot \sin(2\pi\cdot20\ Hz\cdot 0,0125\ s) \cdot 2\ A \cdot \sin(2\pi\cdot20\ Hz\cdot 0,0125\ s+\frac{\pi}{3}) \nonumber \\
                  p(t) &= 12\ W \nonumber
              \end{eqa}
              
        \item[\bf d)]
              Die Wirkleistung beträgt:
              \begin{eqa}
                  P &= U \cdot I \cdot \cos(\varphi) = \frac{\hat{U}}{\sqrt{2}} \cdot \frac{\hat{I}}{\sqrt{2}} \cdot \cos(\varphi)
                  = \frac{12\ V}{\sqrt{2}} \cdot \frac{2\ A}{\sqrt{2}} \cdot \cos(60^o) \nonumber \\
                  P &= 6\ W    \nonumber
              \end{eqa}
              Die Blindleistung beträgt:
              \begin{eqa}
                  Q &= U \cdot I \cdot \sin(\varphi) = \frac{\hat{U}}{\sqrt{2}} \cdot \frac{\hat{I}}{\sqrt{2}} \cdot \sin(\varphi)
                  = \frac{12\ V}{\sqrt{2}} \cdot \frac{2\ A}{\sqrt{2}} \cdot \sin(60^o) \nonumber \\
                  Q &= 10,392\ var    \nonumber
              \end{eqa}
              
        \item[\bf e)]
              Die Scheinleistung beträgt:
              \begin{eqa}
                  \underline{S} &= \underline{U} \cdot \underline{I}^* = 
                  \frac{\underline{\hat{U}}}{\sqrt{2}} \cdot \frac{\underline{\hat{I}}^*}{\sqrt{2}} =
                  \frac{12\ V}{\sqrt{2}} \cdot e^{j(2\pi\cdot20\ Hz)} \cdot \frac{2\ A}{\sqrt{2}} \cdot e^{j(2\pi\cdot20\ Hz+\frac{\pi}{3})} \nonumber \\
                  \underline{S} &= 12\ VA \cdot e^{j\frac{\pi}{3}} = 6\ W + j10,392\ var    \nonumber
              \end{eqa}
              
              
    \end{itemize}
    
}