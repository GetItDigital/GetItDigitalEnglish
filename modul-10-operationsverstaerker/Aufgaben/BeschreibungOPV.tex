\subsection{Beschreibung der idealen Eigenschaften eines Operationsverstärkers\label{Aufg8}}
% Alt: Aufgabe 8
\Aufgabe{
    Beschreiben Sie folgende Kennzeichen eines idealen Operationsverstärkers:
    \begin{itemize}
        \item [ \bf a)] Eingangswiderstand
        \item [ \bf b)] Ausgangswiderstand
        \item [ \bf c)] Differenzverstärkung
        \item [ \bf d)] Eingangsoffsetspannung
    \end{itemize}
 %   \speech{
%Aufgabe 8: Kennzeichen eines idealen Operationsverstärkers  
%
%Es sollen die folgenden Eigenschaften eines idealen Operationsverstärkers beschrieben werden:  
%
%a) Eingangswiderstand:  
%   Dieser beschreibt, wie viel Strom an den Eingängen des Operationsverstärkers fließt.  
%
%b) Ausgangswiderstand:  
%   Dieser gibt an, wie stark die Ausgangsspannung durch eine angeschlossene Last beeinflusst wird.  
%
%c) Differenzverstärkung:  
%   Diese beschreibt, wie stark die Differenz zwischen den beiden Eingängen verstärkt wird.  
%
%d) Eingangsoffsetspannung:  
%   Dies ist eine kleine unerwünschte Spannung, die auch bei null Eingangssignal vorhanden sein kann.  
%}

}

\Loesung{
    \begin{itemize}
        \item [ \bf a)] Eingangswiderstand: unendlich, um die Spannungsquelle wenig zu belasten
        \item [ \bf b)] Ausgangswiderstand: $0\,\Omega$, da OP so viel Leistung wie möglich abgeben soll
        \item [ \bf c)] Differenzverstärkung: unendlich, um kleine Signale zu verstärken
        \item [ \bf d)] Eingangsoffsetspannung: $0\,\mathrm{V}$, damit OP bei keinem Eingangssignal auch keine Ausgangsspannung ausgibt
    \end{itemize}
	Siehe: Tabelle \ref{tab:Ideale und typische Operationsverstärkereigenschaften}
%    \speech{
%        Eigenschaften eines idealen Operationsverstärkers  
%        
%        a) Der Eingangswiderstand eines idealen Operationsverstärkers ist unendlich.  
%           Dadurch wird die Spannungsquelle kaum belastet, und es fließt nahezu kein Strom in die Eingänge des Operationsverstärkers.  
%        
%        b) Der Ausgangswiderstand eines idealen Operationsverstärkers beträgt null Ohm.  
%           Dadurch kann der OP so viel Leistung wie möglich an die Last abgeben,  
%           ohne dass ein nennenswerter Spannungsabfall am Ausgang auftritt.  
%        
%        c) Die Differenzverstärkung eines idealen Operationsverstärkers ist unendlich.  
%           Dies bedeutet, dass selbst kleinste Spannungsdifferenzen am Eingang stark verstärkt werden,  
%           wodurch der Operationsverstärker sehr empfindlich auf Eingangssignale reagiert.  
%        
%        d) Die Eingangsoffsetspannung beträgt null Volt.  
%           Dadurch gibt der Operationsverstärker bei einem Null-Eingangssignal auch eine exakt null Volt Ausgangsspannung aus,  
%           was für eine perfekte Signaltreue sorgt.  
%        }
        
    }