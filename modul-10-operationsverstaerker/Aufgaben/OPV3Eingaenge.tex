\subsection{Berechnung einer OP-Schaltung mit drei Eingängen\label{Aufg16}}
% Alt: Aufgabe 16
\Aufgabe{
    Gegeben ist die unten abgebildete Schaltung. Bestimmen Sie die Ausgangsspannung \( U_\mathrm{A} \). \\
    Die Widerstandswerte lauten: \\
    $
    R_\mathrm{1} = \mathrm{10\,k\Omega},\\
    R_\mathrm{2} = \mathrm{20\,k\Omega},\\
    R_\mathrm{3} = \mathrm{30\,k\Omega},\\
    R_\mathrm{4} = \mathrm{10\,k\Omega},\\
    R_\mathrm{5} = \mathrm{20\,k\Omega},\\
    R_\mathrm{6} = \mathrm{40\,k\Omega}
    $
    \begin {figure} [H]
    \centering
    \begin{tikzpicture}
\ctikzset{tripoles/en amp/input height=0.45}
\draw (0,0)node[en amp](E){N1}
    (E.out)
    (E.-)
    (E.+);
    \draw (E.out) to[short, -*] (2,0) to[R,l=$R_5$] (4,0) to[short, -*] (4,0.5);
    \draw (E.-) to[short, -*] (-2,0.5) to[short, -*] (-3,0.5) to[short, -*] (-3,1.5) -- (-3,2.5)
    to[R,l_=$R_1$] (-5,2.5) to[short, -o] (-7,2.5);
    \draw (-2,0.5) to[short] (-2,2.5) to[R, name=R, l=$R_4$] (2,2.5) -- (2,0);
    \draw (-3,1.5) to[R, l_=$R_2$] (-5,1.5) to[short, -o] (-6,1.5);
    \draw (-3,0.5) to[R,l_=$R_3$,-o] (-5,0.5);

    \draw (-7,-2) to[short, o-o] (-6,-2) to[short, -o](-5,-2);
    \draw (E.+) -- (-2,-0.5) to[short, -*] (-2,-2);
    \draw (2.5,-2) to[short, -o] (-5,-2);
    \draw (2.5,-2) to[short, -*] (4,-2) to[short, -o](8.5,-2);

        \draw (6,0)node[en amp](E){N2}
    (E.out)
    (E.-)
    (E.+);
    \draw (E.-) -- (4,0.5) -- (4,2.5) to[R,l=$R_6$] (8,2.5) to[short, -*] (8,0);
    \draw (E.out) to[short, -o] (8.5,0);
    \draw (E.+) -- (4,-0.5) -- (4,-2);
    
    \draw (8.5,0) to[open, v, name=ua] (8.5,-2);
    \draw (-7,2.5) to[open, v, name=ue1] (-7,-2);
    \draw (-6,1.5) to[open, v, name=ue2] (-6,-2);
    \draw (-5,0.5) to[open, v, name=ue3] (-5,-2);
    
    \varrmore{ue1}{${U_\mathrm{E1}}$};
    \varrmore{ue2}{${U_\mathrm{E2}}$};
    \varrmore{ue3}{${U_\mathrm{E3}}$};
    \varrmore{ua}{${U_\mathrm{A}}$};
\end{tikzpicture}
    \label{fig:FigOPV3Eingaenge}
    \end {figure}
%    \speech{
%        Aufgabe 16: Zweistufige Operationsverstärkerschaltung mit drei Eingängen  
%        
%        Gegeben ist eine Schaltung mit zwei Operationsverstärkern N1 und N2.  
%        Die Widerstände haben folgende Werte:  
%        R1 = 10 Kiloohm, R2 = 20 Kiloohm, R3 = 30 Kiloohm,  
%        R4 = 10 Kiloohm, R5 = 20 Kiloohm, R6 = 40 Kiloohm.  
%        
%        Die Eingangsspannungen U E1, U E2 und U E3 werden über die Widerstände R1, R2 und R3  
%        an den invertierenden Eingang des ersten Operationsverstärkers N1 geführt.  
%        Der nichtinvertierende Eingang von N1 ist mit Masse verbunden.  
%        Die Rückkopplung erfolgt über den Widerstand R4.  
%        
%        Der Ausgang des ersten Operationsverstärkers ist über den Widerstand R5  
%        mit dem invertierenden Eingang des zweiten Operationsverstärkers N2 verbunden.  
%        Der nichtinvertierende Eingang von N2 ist ebenfalls mit Masse verbunden.  
%        Die Rückkopplung erfolgt über den Widerstand R6.  
%        
%        Gesucht ist die Ausgangsspannung U A in Abhängigkeit der Eingangsspannungen U E1, U E2 und U E3.  
%        }
        
}

\Loesung{
	Ausgangsspannung nach 1. OPV mit folgender Formel berechnen ($U_{a1}$):
    \begin{align*}
        U_\mathrm{A1} = - (U_\mathrm{E1} \cdot \frac{R_4}{R_1} + U_\mathrm{E2} \cdot \frac{R_4}{R_2} + U_\mathrm{E3} \cdot \frac{R_4}{R_3}) \\
        U_\mathrm{A1} = - (U_\mathrm{E1} \cdot \frac{10}{10} + U_\mathrm{E2} \cdot \frac{10}{20} + U_\mathrm{E3} \cdot \frac{10}{30}) \\
        U_\mathrm{A1} = - (U_\mathrm{E1} + \frac{1}{2} \cdot U_\mathrm{E2} + \frac{1}{3} \cdot U_\mathrm{E3}) \\
    \end{align*}
    Daraus egibt sich für die gesamte Ausgangsspannung ($U_\mathrm{A}$):
    \begin{align*}
        U_\mathrm{A} = - \frac{R_6}{R_5} \cdot U_\mathrm{a1} \\
        U_\mathrm{A} = - \frac{40}{20} \cdot (- (U_\mathrm{E1} + \frac{1}{2} \cdot U_\mathrm{E2} + \frac{1}{3} \cdot U_\mathrm{E3})) \\
        U_\mathrm{A} = 2 \cdot U_\mathrm{E1} + U_\mathrm{E2} + \frac{2}{3} \cdot U_\mathrm{E3} 
    \end{align*}

  %  \speech{
  %      Lösung für Aufgabe 16.
  %      
  %      Gegeben ist eine zweistufige Operationsverstärkerschaltung. Die Ausgangsspannung des ersten Operationsverstärkers, U A1, wird mit folgender Formel berechnet:
  %      
  %      U A1 ist gleich minus Klammer auf U E1 mal R 4 durch R 1 plus U E2 mal R 4 durch R 2 plus U E3 mal R 4 durch R 3 Klammer zu.
  %      
  %      Einsetzen der Werte. U A1 ist gleich minus Klammer auf U E1 mal 10 durch 10 plus U E2 mal 10 durch 20 plus U E3 mal 10 durch 30 Klammer zu.
  %      
  %      Nach Kürzen bleibt. U A1 ist gleich minus Klammer auf U E1 plus ein halb mal U E2 plus ein Drittel mal U E3 Klammer zu.
  %      
  %      Daraus ergibt sich für die gesamte Ausgangsspannung U A:
  %      
  %      U A ist gleich minus R 6 durch R 5 mal U A1.
  %      
  %      Einsetzen der Werte. U A ist gleich 40 durch 20 mal Klammer auf minus Klammer auf U E1 plus ein halb mal U E2 plus ein Drittel mal U E3 Klammer zu Klammer zu.
  %      
  %      Nach Vereinfachung bleibt. U A ist gleich zwei mal U E1 plus U E2 plus zwei Drittel mal U E3.
  %      
 %       Die Ausgangsspannung ist somit eine gewichtete Summe der Eingangsspannungen U E1, U E2 und U E3 mit den entsprechenden Verstärkungsfaktoren.
%        }
Siehe: Abschnitt \ref{fig: Verschaltung Verstaerker} und Tabelle \ref{tab:Grundschaltungen}
}