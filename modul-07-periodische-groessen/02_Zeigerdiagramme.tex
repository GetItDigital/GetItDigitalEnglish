%Anwendung komplexe Zahlen in der Wechselstromtechnik

\newvideofile{Zeiger}{Zeigerdiagramme}

\begin{frame}
    \fta{Zeigerdiagramme in der Wechselstromtechnik}
    
    \s{
        In der Wechselstromtechnik finden die komplexen Zahlen unter Anderem als Zeiger Verwendung.  
        In linearen Netzwerken haben sinusförmige und monofrequente Quellengrößen nur sinusförmige und monofrequente 
        Spannungen und Ströme zur Folge. Im Stationären Zustand unterscheiden sich die Spannungen und Ströme nur in 
        ihrer Amplitude und der zugehörigen Phase. Alle Signale haben dabei eine identische Frequenz. Die Stationären
        und harmonischen Schwingungen können durch ihren Zeitverlauf im Liniendiagramm oder durch komplexe Zahlen durch
        Zeiger dargestellt werden. Die Beschreibung von Schwingungen durch komplexe Zahlen ermöglicht die Darstellung 
        durch ruhende Zeiger, wobei nur die relative Lage zueinander ausgedrückt wird. 
    }
    
    \begin{Lernziele}{Zeigerdiagramme}
        Die Studierenden
        \begin{itemize}
            \item verstehen die Kenngrößen von periodischen Wechselspannungen.
            \item können mit den komplexen Drehzeigern der Amplituden umgehen.
        \end{itemize}
    \end{Lernziele}
    
    \speech{ZeigerLrnz}{1}{Zeigerdiagramme dienen in der Wechselstromtechnik der vereinfachten Berechnung von sinusförmigen Wechselgrößen.
    In diesem Kapitel sollen die Kenngrößen von periodischen Wechselgrößen vermittelt 
    und die zugehörigen komplexen Drehzeiger erläutert werden.} 
\end{frame}

\begin{frame}
    \ftb{Periodische Wechselspannung}
    
    \s{
        In der Abbildung \ref{BildWechselspannungssignal} wird ein Wechselspannungssignal mit dem Scheitelwert (Amplitudenwert) $\hat{U}$
        dargestellt. Der Versatz des Wechselspannungssignals auf der Abszisse wird durch den Zeitpunkt $t_\mathrm{0}$
        ausgedrückt. Der Punkt in dem das Wechselspannungssignal die Ordinate kreuzt gibt den Wechselspannungswert 
        $U$ zum Zeitpunkt $t=0$ an. 
        
        Das Wechselspannungssignal ist periodisch in $T$. Eine Periode wird bei dem angegebenen Wechselspannungsignal
        durch einen kompletten Durchlauf einer positiven und negativen Halbwelle definiert. So wird ab $t_\mathrm{0}$ nach einer 
        Periode der Zeitpunkt $t_0+T$ und nach zwei Perioden der Zeitpunkt $t_\mathrm{0+2T}$ erreicht. 
        
        Mit der Übertragung der angegebenen Werte auf ein Kreisdiagramm lässt sich der Zusammenhang der Wechselspannung 
        mit der komplexen Darstellung erklären. Auf diese Weise kann die Wechselspannungsgröße $U(t=0)$ zum Zeitpunkt $0$ 
        durch die Länge und Ausrichtung des Zeigers beschrieben werden. Der Scheitelwert enspricht der Länge des Zeigers
        im Kreisdiagramm. Der Winkel $\varphi_\mathrm{u}$ beschreibt den Phasenwinkel zum Zeitpunkt $t=0$. 
    }
    
    \b{
        Anwendung von komplexen Zahlen in der Wechselstromtechnik
        
        \begin{itemize}
            \item <1-> Sinusförmige und harmonische Quellengrößen, Scheitelwert $\hat{U}$ und Phasenverschiebung $t_\mathrm{0}$
            \item <2-> Periodendauer $T$, Wiederholung von Perioden
            \item <3-> Stationärer Zustand $\rightarrow$ Spannungen und Ströme unterscheiden sich nur in Amplitude und Phase
            \item <4-> Schwingungen können durch Zeiger dargestellt werden
        \end{itemize}
    }
    
    \fu{
        %vergrößern auf Folienbreite
        \resizebox{0.95\textwidth}{!}{
            \begin{circuitikz}[domain=-1:8.5,samples=100]
    %Gitter
    \draw[very thin,color=gray] (-1.1,-2.1) grid (8.7,2.1);
    \draw[->] (-1.2,0) -- (8.9,0) node[right] {$t$};
    \draw[->] (0,-2.1) -- (0,2.2) node[above] {};
    \draw (0,-2.1) node[below] {$0\ ms$};
    \draw (2,-2.1) node[below] {$10\ ms$};
    \draw (4,-2.1) node[below] {$20\ ms$};
    \draw (6,-2.1) node[below] {$30\ ms$};
    \draw (8,-2.1) node[below] {$40\ ms$};
    \draw[color=voltage] (0,2.2) node[above] {$u(t)$};
    \draw[color=voltage, very thick] plot[id=Sinus6] function{1.5 * sin((pi/2) * x + (pi/4))} node[right] {};
    \draw(0,-2.1) -- (0,2.2) node[above] {};
    \draw[dashed](0,1.5) -- (1.5,1.5) node[above] {$\hat{U}$};
    \draw[dashed](0.2,1) -- (-0.2,1) node[above left] {$u(t=0)$};
    \draw (-0.5,0.2) -- (-0.5,-0.2) node[below] {$t_\mathrm{0}$};
    \pause

    %Periodendauer
    \draw (3.5,0.2) -- (3.5,-0.2) node[below right] {$t_\mathrm{0+T}$};
    \draw (7.5,0.2) -- (7.5,-0.2) node[below right] {$t_\mathrm{0+2T}$};
    \draw[dashed](3.5,0) -- (3.5,2);
    \draw[dashed](7.5,0) -- (7.5,2);
    \draw[<->] (3.5,2) -- (7.5,2);
    \draw (5.5,2) node[above] {T};
    \pause

    %Kreisdiagramm
    \draw (11.5,-2) -- (11.5,2);
    \draw (9.5,0) -- (13.5,0);
    \draw (11.5,0) circle (1.5);
    \draw[dashed](8.5,1.5) -- (11.5,1.5);
    \draw (9.5,1.5) node[above] {$\hat{U}$};
    \draw[dashed](7.8,1) -- (12.5,1);
    \draw (9,0.6) node {$u(t=0)$};
    \pause

    %Zeiger und Winkel
    \draw[->] (11.5,0) -- (12.55,1.05);
    \draw (13,0) coordinate (3);
    \draw (11.5,0) coordinate (4);
    \draw (13,1.5) coordinate (5);
    \draw pic[draw, very thick, angle radius = 0.75cm, ->] {angle = 3--4--5};
    \draw (12.5,0.5) node[very thick]{$\varphi_\mathrm{u}$};
    \draw (13.5,1) coordinate (6);
    \draw (11.5,0) coordinate (7);
    \draw (12.5,2) coordinate (8);
    \draw pic[draw, thick, angle radius = 1.7cm, ->] {angle = 6--7--8};
    \draw (12.8,1.7) node[very thick]{$\omega=\frac{2 \pi}{T}$};
    \def\StartAngle{0}
    \def\EndAngle{359}
    \def\Radius{1.5}
    \draw([shift=(\StartAngle : \Radius)]11.5,0)  arc[start angle=\StartAngle, end angle=\EndAngle, radius=\Radius];
    %\draw (9.3,0.9) node[right] {$\varphi_i=\frac{\pi}{2}$};									

    %ohne diesen Befehl macht das \pause in der Tikz Umgebung die Footline kaputt
    \onslide<1->
\end{circuitikz}
        }
    }{{\bf Periodische Wechselspannung über der Zeit mit zwei Perioden.} Anhand der Wechselspannung lassen sich z.B. der
        Amplitudenwert und die Periodedauer ermitteln.   \label{BildWechselspannungssignal}}

    \speech{ZeigerPerd}{1}{Folgend wird eine Sinusförmige und harmonische, also gleichförmige Wechselspannung dargstellt. 
    Der größtmögliche Ausschlag des Spannungssignals wird beim sogenannten Scheitelwert U Dach erreicht.
    Außerdem kann ein Stromsignal oder Spannungssignal eine Phasenverschiebung T null aufweisen.}
    \speech{ZeigerPerd}{2}{Die Periodendauer T gibt hierbei die Länge einer Periode an. Das Spannungssignal wird periodisch wiederholt.}
    \speech{ZeigerPerd}{3}{Neben dieser Darstellung einer Wechselgröße, kann in einem Stationären Zustand, also zu einem festgesetzten Zeitpunkt,
    quasi ein Momentwert einer Wechselspannung oder eines Wechselstroms beschrieben werden. 
    Die Beschreibung von sinusförmigen Wechselgrößen erfolgt hier mit dem Amplitudenwert und der Phase.}
    \speech{ZeigerPerd}{4}{So lässt sich die Schwingung der Wechselgrößen auch als Zeiger darstellen.
    Der Zeiger dreht sich hierbei um den Kreis mit der Kreisfrequenz.}
\end{frame}

\begin{frame}
    \ftx{Umrechnung der Kreisfrequenz}
    
    \s{
        Die Kreisfrequenz $\omega$ kann mithilfe der Periodendauer $T$ oder der Frequenz $f$ angegeben werden. Weiter wird die 
        Kreisfrequenz auch als Winkelgeschwindigkeit beschrieben. Über den 
        Zusammenhang zwischen der Periodendauer und der Frequenz können die Größen ineinander umgerechnet werden. 
        Die erläuterten Zusammenhänge werden in den Gleichungen \ref{GleichungKreisfrequenz} beschrieben.
    }
    
    \b{
        \begin{itemize}
            \item Kreisfrequenz $\omega$ beschrieben durch die Periodendauer $T$
            \item<2-> Umrechnung von Periodendauer und Frequenz $f$
            \item<3-> Kreisfrequenz $\omega$ beschrieben durch die Frequenz
        \end{itemize}
    }
    
    \begin{eq}
        \omega = \frac{2\pi}{T} 
        \onslide<2->{\text{\quad mit \quad} f=\frac{1}{T} }
        \onslide<3->{\text{\quad wird \quad} \omega = 2\pi f \label{GleichungKreisfrequenz}}
    \end{eq}

    \speech{ZeigerKreis}{1}{Die Kreisfrequenz Omega lässt sich wie angegeben aus der Periodendauer T errechnen.
    Der Ausdruck zwei Pi erfolgt aus dem Umlauf eines Kreises im Bogenmaß.}
    \speech{ZeigerKreis}{2}{Über die Periodendauer kann auch eine Verbindung zur Frequenz f hergeleitet werden.
    Die Frequenz einer Wechselspannung lässt sich aus dem Kehrwert der Periodendauer bestimmen.}
    \speech{ZeigerKreis}{3}{So kann die Kreisfrequenz auch über die frequenz errechnet werden.}
\end{frame}

\begin{frame}
    \ftx{Harmonische Wechselgrößen}
    
    \s{
        Angenommen der Durchlauf einer kompletten Periode bei einer Frequenz von $50 Hz$ dauert $20 ms$ (vgl. Abbildung \ref{BildZeiten}). 
        Übertragen auf das Gradmaß und das Radmaß ergeben sich so Winkelangaben für den Durchlauf einer Periode. Der 
        Winkel nach der kompletten Periode T entspricht dem Umlauf eines vollen Kreises. Der Umlauf eines vollen Kreises
        entspricht dazu 
        Die Phasenverschieben $t_\mathrm{0}$
        lässt sich auch in der Winkeldarstellung als Phasenwinkel $\varphi_\mathrm{N}$ beschreiben. 
    }
    
    \b{
        \begin{itemize}
            \item <1-> Eine Periodendauer entspricht in Winkelgrößen einem Kreisumlauf von $360°$ oder $2\pi$
            \item <2-> Die Phasenverschiebung $t_\mathrm{0}$ entspricht in der Winkeldarstellung dem Phasenwinkel $\varphi_\mathrm{N}$
        \end{itemize}
    }
    
    \fu{
        \begin{circuitikz}
    \draw[very thin,color=gray] (-1.1,-0.1) grid (8.7,0.3);
    \draw[->] (-1.2,0) -- (8.9,0) node[right] {t in ms};
    \draw (0,-0.1) node[below] {$0$};
    \draw (2,-0.1) node[below] {$5$};
    \draw (4,-0.1) node[below] {$10$};
    \draw (6,-0.1) node[below] {$15$};
    \draw (8,-0.1) node[below] {$20$};
    \draw[dashed,GETgreen](-1.2,-0.6) -- (10.3,-0.6);
    \draw[very thin,color=gray] (-1.1,-1.1) grid (8.7,-0.7);
    \draw[->] (-1.2,-1) -- (8.9,-1) node[right] {Grad};
    \draw (0,-1.1) node[below] {$0$};
    \draw (2,-1.1) node[below] {$90$};
    \draw (4,-1.1) node[below] {$180$};
    \draw (6,-1.1) node[below] {$270$};
    \draw (8,-1.1) node[below] {$360$};
    \draw[very thin,color=gray] (-1.1,-2.1) grid (8.7,-1.7);
    \draw[->] (-1.2,-2) -- (8.9,-2) node[right] {Rad};
    \draw (0,-2.1) node[below] {$0$};
    \draw (2,-2.1) node[below] {$\pi/2$};
    \draw (4,-2.1) node[below] {$\pi$};
    \draw (6,-2.1) node[below] {$3/2 \pi$};
    \draw (8,-2.1) node[below] {$2\pi$};
    \draw[dashed](-0.5,0.3) -- (-0.5,-2.2);
    \draw (-0.5,0.3) node[above] {$t_\mathrm{0}$};
    \draw (-0.5,-2.2) node[below] {$\varphi_\mathrm{N}$};
\end{circuitikz}
    }{{\bf Verschiedene Maßangaben.} Vegleich zwischen Gradmaß und Radmaß, der Umlauf eines Kreises entspricht $360^o$ oder
        $2\pi$. \label{BildZeiten}}

    \speech{ZeigerHarm}{1}{Wie bereits angemerkt entspricht die Periodendauer einem Kreisumlauf von zwei Pi oder Dreihundertsechzig grad.}
    \speech{ZeigerHarm}{2}{Die zuvor kennengelernte Phasenverschiebung T null, welche in der Zeiteinteilung und der Berechnung mit der Periodendauer zum Einsatz kommt,
    entspricht in der Winkeldarstellung Vieh N dem Phasenwinkel.}
\end{frame}



\begin{frame}
    \ftx{Wechselgrößen}
    
    \begin{Merksatz}{Wechselgrößen}
        Eine Wechselspannung erklärt einen regelmäßigen Polaritätswechsel mit dem {\bf Amplitudenwert $\hat{U}$} und der 
            {\bf Periodendauer $T$}.
    \end{Merksatz}
    
    \speech{ZeigerMrk1}{1}{Eine Wechselspannung erklärt einen regelmäßigen Polaritätswechsel mit dem Amplitudenwert U Dach und der Periodendauer Tee.}
\end{frame}




\begin{frame}
    \ftx{Zeitabhängige Wechselspannung}
    
    \s{
        Der zeitabhängige Spannungswert $u(t)$ einer Wechselspannung ist von mehreren Einflussgrößen abhängig.
        Dazu gehören der Scheitelwert und die Frequenz der Wechselspannung sowie der dazugehörige Phasenwinkel.
        Über das Verhältnis vom Zeitpunkt t zur Periodendauer lässt sich $u(t)$ bestimmen.
    }
    
    \b{
        \begin{itemize}
            \item $u(t)$ beschrieben durch die Kreisfrequenz $\omega$
            \item $u(t)$ zum Zeitpunkt $t$ bei einer Periodendauer von $T$
        \end{itemize}
    }
    
    \begin{eq}
        u(t) = \hat{U} \cdot \sin(\omega t + \varphi_\mathrm{N}) = \hat{U} \cdot \sin(2\pi f t + \varphi_\mathrm{N}) = \hat{U} \cdot \sin(2\pi \frac{t}{T} + \varphi_N)       \label{GleichungSpannung}
    \end{eq}

    \speech{ZeigerZeitab}{1}{Je nachdem zu welchen Zeitpunkt klein T hat der Drehzeiger eine andere Ausrichtung.
    Der zeitabhängige Spannungswerte U von T wird in der Abhängigkeit von der Kreisfrequenz Omega beschrieben.}
\end{frame}




\begin{frame}
    \ftb{Komplexer Drehzeiger der Amplitude}
    
    \s{
        Der komplexe Spannungswert sowie der komplexe Stromwert können als Zeiger dargestellt werden. 
        Hier wird auf die Darstellungsart der Polar-Koordinaten zurückgegriffen. 
    }
    
    \b{
        \begin{itemize}
            \item lineare Netzwerke $\rightarrow$ Sinusförmige und harmonische Quellengrößen
            \item Stationärer Zustand $\rightarrow$ Spannungen und Ströme unterscheiden sich nur in Amplitude und Phase
            \item Schwingungen können durch Zeiger dargestellt werden
            \item Ruhende Zeiger drücken die relative Lage von Schwingungen zueinander aus
        \end{itemize}
    }
    
    \fu{
        \begin{tikzpicture}
    \draw (0,0) coordinate (U);
    %\draw[very thin,color=gray] (-0.1,-0.1) grid (3.2,2.1);
    \draw[->] (-0.2,0) -- (3.4,0) node[right] {$\Re$};
    \draw[->] (0,-0.1) -- (0,2.2) node[above] {$\Im$};
    \draw (3,0.02) coordinate (R);
    \draw (3,2.02) coordinate (Z);
    \draw[->, very thick, voltage] (0,0.02) -- (3,2.02);
    \draw pic[draw, very thick, angle radius = 2cm, ->, voltage] {angle = R--U--Z};
    \draw(1.4,0.5) node [voltage] {$\varphi_\mathrm{U}$};
    \draw(1,1)node [voltage] {$\hat{U}$};

    \draw (6,0) coordinate (U);
    %\draw[very thin,color=gray] (5.9,-0.1) grid (9.2,2.1);
    \draw[->] (5.8,0) -- (9.4,0) node[right] {$\Re$};
    \draw[->] (6,-0.1) -- (6,2.2) node[above] {$\Im$};
    \draw (9,0.02) coordinate (R);
    \draw (9,2.02) coordinate (Z);
    \draw[->, very thick, red] (6,0.02) -- (9,2.02);
    \draw pic[draw, very thick, angle radius = 2cm, ->, red] {angle = R--U--Z};
    \draw(7.4,0.5) node [red] {$\varphi_\mathrm{I}$};
    \draw(7,1)node [red] {$\hat{I}$};
\end{tikzpicture}
    }{{\bf Spannungs- und Stromzeiger.} Ein Spannungszeiger mit dazugehörigen Phasenwinkel und äquivalent dazu ein Stromzeiger mit Phasenwinkel\label{BildSpannung}}

    \speech{ZeigerZsm}{1}{Zusammenfassend kann gesagt werden, dass für lienare Netzwerke Sinusförmige und harmonische Quellengrößen gelten.
    In stationären Zuständen unterscheiden sich Spannungen und Ströme lediglich in ihrer Amplitude und Phase.
    Die Schwingungen können durch Zeiger dargestellt werden. Die Zeiger beruhen auf der Darstellung der Polarkoordinaten.
    Mithilfe von ruhenden Zeigern kann die relative Lage von Schwingungen zueinander ausgedrückt werden.}
\end{frame}

\begin{frame}
    \ftx{Komplexer Drehzeiger der Amplitude}
    
    \s{
        Eine harmonische Wechselgröße kann als mit konstanter Winkelgeschwindigkeit $\omega$ rotierender Zeiger dargestellt
        werden. Die Zeitfunktion ist die Projektion des Zeigers auf die reelle Achse. Der rotierende Zeiger wird durch 
        eine zeitabhängige komplexe Zahl beschrieben. Wie bei den komplexen Zahlen erklärt, können der Imaginärteil 
        und der Realteil jeweils sparat bestimmt werden. Über die Winkelfunktionen kann der Realteil und der 
        Imaginärteil der komplexen Spannung berechnet werden. 
    }
    
    \b{
        Komplexer Drehzeiger einer elektrischen Spannung:
        \begin{itemize}
            \item<1-> Amplitudenwert der Spannung: $\hat{U}$
            \item<2-> Phasenwinkel der Spannung: $\varphi_\mathrm{U}$
            \item<3-> Komplexe Spannungszeiger: $\underline{U}=\hat{U} \cdot e^{j \varphi_\mathrm{U}}$
            \item<4-> Realanteil der Spannung
            \item<5-> Imaginäranteil der Spannung
        \end{itemize}
    }
    
    \fu{
        \begin{tikzpicture}
    \draw (0,0) coordinate (U);
    %\draw[very thin,color=gray] (-0.1,-0.1) grid (3.2,3.1);
    \draw[->] (-0.2,0) -- (3.4,0) node[right] {$\Re(U)$};
    \draw[->] (0,-0.1) -- (0,3.2) node[above] {$\Im(U)$};
    \draw (3,0) coordinate (R);
    \draw (3,3) coordinate (Z);
    \draw[->, voltage] (0,0) -- (2.5,2.5);
    \draw(1,2) node [voltage] {$\hat{U}$};
    \pause

    \draw pic[draw, angle radius = 1cm, ->, voltage] {angle = R--U--Z};
    \draw(1.7,0.5) node [voltage] {$\varphi_\mathrm{U}$};
    \pause

    \draw(3,2.5) node [right,voltage] {$\underline{U}=\hat{U} \cdot e^{j \varphi_\mathrm{U}}$};
    \pause

    \draw[dashed](2.5,2.5) -- (2.5,-0.2) node[below] {$\hat{U} \cdot \cos(\omega t + \varphi_\mathrm{u})$};
    \pause

    \draw[dashed](2.5,2.5) -- (-0.2,2.5) node[left] {$\hat{U} \cdot \sin(\omega t + \varphi_\mathrm{u})$};
\end{tikzpicture}
    }{{\bf Zeigerdarstellung einer komplexen Wechselspannung.} Wechselspannungszeiger mit der Aufteilung in Realteil und
        Imaginärteil.}

    \speech{ZeigerDreh}{1}{Folgend wird beispielhaft der komplexe Drehzeiger einer elektrischen Spannung näher erläutert.
    Es werden der Imaginärteil und der Realteil in der komplexen Ebene der Spannung dargestellt.
    Außerdem ist der Drehzeiger mit der Information U Dach, also dem Amplitudenwert, eingezeichnet.}
    \speech{ZeigerDreh}{2}{Für die Darstellung in Polarkoordinaten fehlt noch der Phasenwinkel Vieh U in Relation zur Abszisse.}
    \speech{ZeigerDreh}{3}{Der komplexe Drehzeiger für die Komplexe Spannung U kann somit vollständig beschrieben werden.}
    \speech{ZeigerDreh}{4}{Der Anteil des Realteiles an der komplexen Spannung lässt sich über die trigonometrische Funktion
    des Cosinus bestimmen.}
    \speech{ZeigerDreh}{5}{Vergleichbar lässt sich der Imaginärteil mithilfe der Sinusfunktion berechnen.}
\end{frame}


\begin{frame}
    \ftx{Komplexe Drehzeiger von Spannung und Strom}
    
    \begin{Merksatz}{Komplexe Drehzeiger von Spannung und Strom}
        Die Wechselgrößen Spannung und Strom werden als komplexe Drehzeiger dargestellt. Die Drehzeiger beschreiben dabei 
            {\bf Augenblickswerte $u(t)$}, welche sich mit konstanter {\bf Winkelgeschwindigkeit $\omega$} verändern. 
    \end{Merksatz}

    \speech{ZeigerMrk2}{1}{Die Wechselgrößen Spannung und Strom werden als komplexe Drehzeiger dargestellt.
    Die Drehzeiger beschreiben dabei Augenblickswerte wie U von T, 
    welche sich mit konstanter Winkelgeschwindigkeit Omega verändern.}
\end{frame}



\begin{frame}
    \ftx{Spannungszeiger und Stromzeiger von elektrischen Komponenten}
    
    \s{
        Die Beschreibung von periodischen Größen an elektrischen Komponenten erfolgt durch feste Spannungszeiger und Stromzeiger. 
        Das Verhalten der verschiedenen Komponenten ist dabei jeweils unterschiedlich. In der Abbildung \ref{BildBauteileZeiger}
        werden ein Widerstand, ein Kondensator als Kapazität und eine Spule als Induktivität mit ihren zugehörigen Spannungs- und 
        Stromzeigern dargestellt. Das genaue Verhalten der drei Komponenten wird im folgenden Kapitel näher betrachtet. 
    }
    
    \b{
        Spannungszeiger und Stromzeiger für verschiedene Komponenten: 
        \begin{itemize}
            \item <1-> Elektrischer Widerstand $\rightarrow$ Spannungszeiger und Stromzeiger in Phase
            \item <2-> Kapazität $\rightarrow$ Stromzeiger eilt dem Spannungszeiger vor
            \item <3-> Induktivität $\rightarrow$ Stromzeiger folgt dem Spannungszeiger
        \end{itemize}
    }
    
    \fu{
        \resizebox{\textwidth}{!}{
            \begin{tikzpicture}
    \draw (0,0) coordinate (U);
    %\draw[very thin,color=gray] (-0.1,-0.1) grid (3.2,2.1);
    \draw[->] (-0.2,0) -- (3.4,0) node[right] {$\Re$};
    \draw[->] (0,-2.2) -- (0,2.2) node[above] {$\Im$};
    \draw (3,0.02) coordinate (R);
    \draw (3,2.02) coordinate (Z);
    \draw[->, voltage] (0,0) -- (3,2);
    \draw pic[draw, angle radius = 2.5cm, ->, voltage] {angle = R--U--Z};
    \draw(2.7,1.3) node [voltage] {$\varphi_\mathrm{UR}$};
    \draw(3,2)node [voltage,above] {$\hat{U}_\mathrm{R}$};
    \draw[->, red] (0,0.1) -- (1.5,1.1);
    \draw pic[draw, angle radius = 1.5cm, ->, red] {angle = R--U--Z};
    \draw(2,0.7) node [red] {$\varphi_\mathrm{IR}$};
    \draw(1.5,1.5)node [red] {$\hat{I}_\mathrm{R}$};
    \draw(0.5,4) to [R=$R$, v, i, name=R] (3.5,4);
    \varrmore{R}{$\underline{U}_\mathrm{R}$};
    \iarrmore{R}{$\underline{I}_\mathrm{R}$};
    \pause

    \draw (6,0) coordinate (U);
    %\draw[very thin,color=gray] (5.9,-0.1) grid (9.2,2.1);
    \draw[->] (5.8,0) -- (9.4,0) node[right] {$\Re$};
    \draw[->] (6,-2.2) -- (6,2.2) node[above] {$\Im$};
    \draw (9,0.02) coordinate (R);
    \draw (9,2.02) coordinate (Z);
    \draw[->, voltage] (6,0.02) -- (9,2.02);
    \draw pic[draw, angle radius = 2.5cm, ->, voltage] {angle = R--U--Z};
    \draw(8.7,1.3) node [voltage] {$\varphi_\mathrm{UC}$};
    \draw(9,2)node [voltage,above] {$\hat{U}_\mathrm{C}$};
    \draw[->, red] (6,0) -- (5,1.5) coordinate (C);
    \draw pic[draw, angle radius = 1.5cm, ->, red] {angle = R--U--C};
    \draw(6.5,1.7) node [red] {$\varphi_\mathrm{IC}$};
    \draw(5,1.5)node [red,above] {$\hat{I}_\mathrm{C}$};
    \draw(6.5,4) to [C=$C$, v, i, name=C] (9.5,4);
    \varrmore{C}{$\underline{U}_\mathrm{C}$};
    \iarrmore{C}{$\underline{I}_\mathrm{C}$};
    \draw (5,1.5) coordinate (K);
    \draw pic[draw, angle radius = 1cm] {angle = Z--U--K};
    \draw (6.1,0.5)[very thick] node {$\cdot$};
    \pause

    \draw (12,0) coordinate (U);
    %\draw[very thin,color=gray] (11.9,-0.1) grid (15.2,2.1);
    \draw[->] (11.8,0) -- (15.4,0) node[right] {$\Re$};
    \draw[->] (12,-2.2) -- (12,2.2) node[above] {$\Im$};
    \draw (15,0.02) coordinate (R);
    \draw (15,2.02) coordinate (Z);
    \draw[->, voltage] (12,0.02) -- (15,2.02);
    \draw pic[draw, angle radius = 2.5cm, ->, voltage] {angle = R--U--Z};
    \draw(14.7,1.3) node [voltage] {$\varphi_\mathrm{UL}$};
    \draw(15,2)node [voltage,above] {$\hat{U}_\mathrm{L}$};
    \draw[->, red] (12,0) -- (13,-1.5) coordinate (L);
    \draw pic[draw, angle radius = 1.5cm, <-, red] {angle = L--U--R};
    \draw(12.3,-1) node [red] {$\varphi_\mathrm{IL}$};
    \draw(13.5,-1.5)node [red,above] {$\hat{I}_\mathrm{L}$};
    \draw(12.5,4) to [L=$L$, v, i, name=L] (15.5,4);
    \varrmore{L}{$\underline{U}_\mathrm{L}$};
    \iarrmore{L}{$\underline{I}_\mathrm{L}$};
    \draw (13,-1.5) coordinate (M);
    \draw pic[draw, angle radius = 1cm] {angle = M--U--Z};
    \draw (12.5,-0.1)[very thick] node {$\cdot$};
\end{tikzpicture}
        }
    }{{\bf Verschiedene Bauteile und die Wechselspannungszeiger.} Die Bauteile Widerstand, Kondensator und Spule mit zugehörigen Spannungs- und Stromzeigern. \label{BildBauteileZeiger}}

    \speech{ZeigerKomp}{1}{Folgend ein Überblick über die Spannungszeiger und die Stromzeiger von drei Komponenten eines elektrischen Netzwerkes verschafft.
    Angefangen mit dem elektrischen Widerstand. Die komplexe Spannung und der komplexe Strom zeigen mit einem identischen Phasenwinkel in dieselbe Richtung.
    Der Spannungszeiger und der Stromzeiger sind somit in Phase.}
    \speech{ZeigerKomp}{2}{Bei einer Kapazität, also einem Kondensator, sind der Spannungszeiger und der Stromzeiger nicht in Phase, zeigen dementsprechend in unterschiedliche Richtungen.
    Der Stromzeiger eilt dem Spannungszeiger an einer Kapazität vor.}
    \speech{ZeigerKomp}{3}{An einer Induktivität, wie einer Spule, sind die Zeiger ebenfalls nicht in Phase. 
    Hier folgt der Stromzeiger dem Spannungszeiger nach.
    Wie es zu dieser Phasenverschiebung an Kapazitäten und Induktivitäten kommt, wird im Themengebiet der komplexen Wechselstromrechnung behandelt.}
\end{frame}


\newpage