\subsection{Berechnung der Stromverstärkung einer Transistorschaltung\label{Aufg18}}
% Alt: Aufgabe 18
\Aufgabe{
    Gegeben ist die folgende Schaltung mit dem Kennlinienfeld des Transistors. Die Versorgungsspannung beträgt \( U_\mathrm{V} = \mathrm{6\,V} \). Der Arbeitspunkt liegt bei der halben Betriebsspannung.\\
    
    \begin{figure}[H]
        \begin{minipage}[c]{\textwidth}
            \centering
            \begin{tikzpicture}
\draw
    (0,0) node[npn, tr circle] (Q1) {}
    (Q1.base) 
    (Q1.collector)  
    (Q1.emitter);

\draw (Q1.base) to[short, -o] (-3,0);

\draw (Q1.emitter) to[short, -*] (0,-3)node[ground]{};

\draw (Q1.collector) to[short, -*] (0,1) to[short, R, l=$R_\mathrm{C}$, -*] (0,3);

\draw (-3,3) -- (3,3) node[above]{$U_\mathrm{CC}$};

\draw (0,1) to[short, -o] (3,1);
\draw (-3,-3) to[short, o-o] (3,-3);

\draw (-3,0) to[open,v>, name=ue] (-3,-3);
\draw (3,1) to[open, v>, name=ua] (3,-3);

\varrmore{ue}{$U_\mathrm{E}$};
\varrmore{ua}{$U_\mathrm{A}$};

\draw (-2,3) to[R,l=$R_{2}$, *-*] (-2,0) to[R,l=$R_{1}$, -*] (-2,-3);
\end{tikzpicture}
        \end{minipage}
        \vspace{0.5cm} % Vertikaler Abstand zur zweiten Reihe
    \end{figure}
    
    \begin{figure}[H]
        \begin{minipage}[c]{0.49\textwidth}
            \centering
            \includesvg[width=.8\textwidth]{Bilder/Aufgaben/FigTransistorStromverstaerkung2}
        \end{minipage}
        \hfill 
        \begin{minipage}[c]{0.49\textwidth}
            \centering
            \includesvg[width=.8\textwidth]{Bilder/Aufgaben/FigTransistorStromverstaerkung3}
        \end{minipage}
        \label{fig:FigTransistorStromverstaerkung}
    \end{figure}
    
    Bestimmen Sie die folgenden Werte aus dem Kennlinienfeld:
    
    \begin{itemize}
        \item Die Paare \( I_\mathrm{B} / U_\mathrm{BE} \) und \( I_\mathrm{C} / U_\mathrm{CE} \) im Arbeitspunkt
        \item Die Stromverstärkung \( \beta \)
    \end{itemize}
}


\Loesung{
    \begin{figure}[H]
        \centering
        \begin{subfigure}[b]{0.45\textwidth}
            \centering
            \includesvg[width=0.9\textwidth]{Bilder/Aufgaben/LsgTransistorStromverstaerkung1}
            \label{fig:LsgTransistorStromverstaerkung1}
        \end{subfigure}
        \begin{subfigure}[b]{0.45\textwidth}
            \centering
            \includesvg[width=0.9\textwidth]{Bilder/Aufgaben/LsgTransistorStromverstaerkung2}
            \label{fig:LsgTransistorStromverstaerkung2}
        \end{subfigure}
    \end{figure}
    Gegeben sind folgende Werte im Arbeitspunkt:
    \begin{gather*}
        U_\mathrm{CC} = \mathrm{6\,V}      \\
        U_\mathrm{CE} = \mathrm{3\,V}      \\
        I_\mathrm{C} = \mathrm{6\,mA}      \\
        U_\mathrm{BE} = \mathrm{0{,}62\,V} \\
        I_\mathrm{B} = \mathrm{20\,\mu A}
    \end{gather*}
    
    Berechnung der Stromverstärkung \( \beta \):
    \begin{equation*}
        \beta = \frac{I_\mathrm{C}}{I_\mathrm{B}} = \frac{\mathrm{6\,mA}}{\mathrm{20\,\mu A}} = \mathrm{300}
    \end{equation*}
    
    Siehe: Abschnitt \ref{sec:ElektrischesVerhaltenBipolartransistor}
}