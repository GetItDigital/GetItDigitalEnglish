\subsection{Kirchhoffsche Gesetze\label{AufgabeKirchhoff1}}
\Aufgabe{
    Berechnen Sie für die Schaltungen die Ströme, Spannungen und Leistungen an den Widerständen. Zeichnen Sie zu Ihren Ergebnissen
    alle Strom- und Spannungspfeile ein. Folgende Werte sind gegeben: \newline
    \phantom{text}\\
    $U_0 = 12 \ V; I_0 = 3 \ A; R_1 = 2 \ \Omega; R_2 = 4 \ \Omega; R_3 = 3 \ \Omega; R_4 = 3 \ \Omega$
    \newline
    \phantom{text}\\
    a)
    %   \newline
    \begin{center}


        \begin{tikzpicture}
    \draw(0,0) to[V,v<,i_,name=U0,-*] (0,3)
    to[short,-*] (2,3)
    to[short,-*] (4,3)
    to[short] (6,3)
    to[R,i_,l=$R_3$,name=R3](6,0)
    to[short,-*] (4,0)
    to[short,-*](2,0)
    to[short,-*](0,0);
    \draw(2,3) to[R,v,i_,l=$R_1$,name=R1](2,0);
    \draw(4,3) to[R,i_,l=$R_2$,name=R2](4,0);

    \varrmore{U0}{$U_0$};
    \varrmore{R1}{$U_1$};
    \iarrmore{U0}{$I_{\mathrm{ges}}$};
    \iarrmore{R3}{$I_3$};
    \iarrmore{R1}{$I_1$};
    \iarrmore{R2}{$I_2$};

    \node at (-0.3,0) {B};
    \node at (-0.3,3) {A};
\end{tikzpicture}
        
    \end{center}
    % \newline
    b)
    % \newline
    \begin{center}


       \begin{tikzpicture}
    \draw(0,0) to[I,i_,name=I0,-*] (0,3)
    to[R,i_,l=$R_1$,name=R1,-*] (3,3)
    to[short] (6,3)
    to[R,i_,l=$R_3$,name=R3](6,0)
    to[R,i_,l=$R_4$,name=R4,-*](3,0)
    to[short,-*](0,0);
    \draw(3,3) to[R,i_,l=$R_2$,name=R2](3,0);

    \iarrmore{I0}{$I_0$};

    \node at (-0.3,0) {B};
    \node at (-0.3,3) {A};
\end{tikzpicture}
    \end{center}
}

\Loesung{
    Hier entsteht eine Musterlösung...
}