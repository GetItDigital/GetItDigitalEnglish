\subsection{Bestimmung des Arbeitspunkts und der Leistung einer Diode\label{Aufg4}}
% Alt: Aufgabe 4
\Aufgabe{
  Eine Diode hat die folgende Kennlinie im 
  Durchlassbereich. Sie wird in Reihe mit einer 
  idealen Spannungsquelle $U_0=1\,\mathrm{V}$ und einem 
  Widerstand mit dem Wert von $R=200\,\Omega$ geschaltet.

    \begin{figure}[H]
        \centering
        \begin{subfigure}[h]{0.45\textwidth}
            \centering
            \begin{circuitikz}
    
\draw (0,0) to [R,i, name=r0, l=$R$] (4,0) to [D=$D$, v, name=D1] (4,-3);
\draw (4,-3) to (0,-3);
\draw (0,0) to [V, name=U1] (0,-3);


\varrmore{U1}{$U_0$}; 
\varrmore{D1}{$U_\mathrm{D}$};
\iarrmore{r0}{$I$};

\end{circuitikz}
        \end{subfigure}
        \begin{subfigure}[h]{0.45\textwidth}
            \centering
            \includesvg[width=0.9\textwidth]{Bilder/Aufgaben/FigDiodeAP}
        \end{subfigure}
        \label{fig:FigDiodeAP}
    \end{figure}

    \begin{itemize}
        \item[a)]
        Welcher Strom $I$ und welche Spannung $U_\mathrm{D}$ 
        stellen sich ein? Lösen Sie das Problem graphisch.
        \item[b)]
        Welche elektrische Leistung wird an der Diode umgesetzt?
        \item[c)]
        Wie muss $R$ gewählt werden, damit sich ein Strom von $3\,\mathrm{mA}$ einstellt? 
    \end{itemize}
%     \speech{ **Aufgabe 4**  
% Eine **Diode mit gegebener Kennlinie im Durchlassbereich** wird in **Reihe mit einer idealen Spannungsquelle** \( U_0 = 1V \) und einem **Widerstand \( R = 200\Omega \)** geschaltet.

% Beantworten Sie die folgenden Fragen:  
% a) **Welcher Strom \( I \) und welche Spannung \( U_D \) stellen sich ein?**  
%    - Lösen Sie das Problem **graphisch** mit Hilfe der gegebenen **Kennlinie**.  
% b) **Welche elektrische Leistung wird an der Diode umgesetzt?**  
% c) **Wie muss \( R \) gewählt werden, damit sich ein Strom von 3 mA einstellt?**  



%  **Beschreibung der Abbildung**
% Die Abbildung besteht aus zwei Teilen:

% 1. **Schaltungsskizze (links):**  
%    - Eine **ideale Gleichspannungsquelle** mit einer Spannung von \( U_0 = 1V \) ist dargestellt.  
%    - Ein **Serienwiderstand \( R \)** von **200Ω** begrenzt den Stromfluss.  
%    - Eine **Siliziumdiode** ist in **Durchlassrichtung** geschaltet.  
%    - Der Strom \( I \) fließt durch den Widerstand und die Diode.  
%    - Die Spannung **\( U_D \)** fällt über der **Diode** ab.  

% 2. **Diodenkennlinie (rechts):**  
%    - **Auf der x-Achse** ist die **Diodenspannung \( U_D \)** in **Volt (V)** dargestellt.  
%    - **Auf der y-Achse** ist der **Diodenstrom \( I \)** in **Milliampere (mA)** aufgetragen.  
%    - Die **exponentielle Kennlinie** zeigt den **typischen Durchlassbereich einer Diode**.  
%    - Um die Lösung zu bestimmen, muss der **Arbeitspunkt** durch die **Schnittstelle zwischen der Diodenkennlinie und der Lastgeraden** ermittelt werden.  

%  **Zusammenfassung der Aufgabe**
% - Die Aufgabe erfordert die **Bestimmung des Stroms und der Spannung an der Diode**, wobei die **Kennlinie** als Grundlage für die Berechnung dient.  
% - Es soll die **umgesetzte elektrische Leistung** berechnet werden.  
% - Schließlich muss der **Widerstandswert \( R \) angepasst werden**, um einen bestimmten **gewünschten Strom von 3 mA** zu erreichen.  
% }

}
    
\Loesung{
    \begin{itemize}
        \item[a)]
        Im ersten Schritt wird die Wirkung der Spannungsquelle auf den Widerstand $R$ betrachtet. 
        Bei $U_0=1\,\mathrm{V}$ fließen durch den Widerstand $R$ ein Strom von $I_\mathrm{R}=\frac{U_0}{R}=\frac{1\,\mathrm{V}}{200\,\Omega}=5\,\mathrm{mA}$. 
        Es kann die Widerstandgerade vom Ursprung aus zu dem Punkt ($1\,\mathrm{V}$, $5\,\mathrm{mA}$) eingezeichnet werden (linke Abbildung).\\

        Gemäß des Aufbaus der Schaltung muss die Widerstandsgerade im nächsten Schritt gespiegelt werden, es entsteht die rechte Darstellung.
        Der Schnittpunkt beider Linien entspricht dem Arbeitspunkt ($U_\text{AP}$, $I_\text{AP}$), dieser lautet: ($0,75\,\mathrm{V}$, $1,2\mathrm{mA}$)  

        \begin{figure}[H]
            \begin{subfigure}[h]{0.45\linewidth}
                \centering
                \includesvg[width=\textwidth]{Bilder/Aufgaben/LsgDiodeAP1}
            \end{subfigure}
            \begin{subfigure}[h]{0.45\linewidth}
                \centering
                \includesvg[width=\textwidth]{Bilder/Aufgaben/LsgDiodeAP2}
            \end{subfigure}
        \end{figure}

        \item[b)]
        Die Leistung ergibt sich aus dem zuvor ermittelten Arbeitspunkt.
        \begin{equation*}
            P = U_\text{AP} \cdot I_\text{AP} = 0,75 \, \text{V} \cdot 1,2 \, \text{mA} = 0,9 \, \text{mW}
        \end{equation*}
        
        \item[c)]
        Ausgehend vom Punkt ($1\,\mathrm{V}, 0\,\mathrm{mA}$) zum Schnittpunkt mit der Diodenkennlinie bei $I=3\,\mathrm{mA}$ ergibt sich die gesuchte Widerstandsgerade.
        \begin{figure}[H]
            \centering
            \includesvg[width=0.45\linewidth]{Bilder/Aufgaben/LsgDiodeAP3}
        \end{figure}
        Es kann der Schnittpunkt ($0,86\,\mathrm{V}, 3\,\mathrm{mA}$) abegelesen werden.
    

        \begin{equation*}
            U_\mathrm{ges} = U - U_\mathrm{D} = 1 \, \mathrm{V} - 0,86 \, \mathrm{V} = 0,14 \, \mathrm{V}
        \end{equation*}


    \end{itemize}



\begin{itemize}
    \item[\bf c)] 
    \[
    U_{\text{ges}} = U - U_\text{D} = 1 \, \text{V} - 0,86 \, \text{V} = 0,14 \, \text{V}
    \]
    \[
    I = 3 \, \text{mA}
    \]
    \[
    R = \frac{U_{\text{ges}}}{I} = \frac{0,14 \, \text{V}}{3 \, \text{mA}} = 46,67 \, \Omega
    \]
\end{itemize}
Siehe: Abschnitt \ref{fig:Diodenkennlinie}


% \speech{
% **Lösung zu Aufgabe 4**  

% **Teilaufgabe a:**  
% Im ersten Schritt wird die Wirkung der Spannungsquelle auf den Widerstand \( R \) betrachtet.  
% Die gegebene Spannung beträgt \( U_0 = 1V \).  
% Da der Widerstand \( R = 200 \Omega \) beträgt, ergibt sich der Strom durch den Widerstand als:  
% \( I_R = \frac{U_0}{R} = \frac{1V}{200 \Omega} = 5mA \).  

% Die dazugehörige Widerstandsgerade kann im Diagramm als eine Linie von der Ursprungskoordinate \((0V, 0mA)\) bis zum Punkt \((1V, 5mA)\) eingezeichnet werden.  
% Diese Gerade gibt an, wie sich der Strom \( I_R \) in Abhängigkeit der Spannung \( U \) verhält.  

% In der linken Abbildung ist die Diodenkennlinie als eine exponentiell ansteigende blaue Kurve dargestellt.  
% Zusätzlich ist die Widerstandsgerade als schwarze Linie eingezeichnet.  
% Diese verläuft von \((0V, 0mA)\) bis zu \((1V, 5mA)\).  

% Da das Bauteil eine Diode enthält, muss die Widerstandsgerade gespiegelt werden.  
% Dies ist in der zweiten Abbildung auf der rechten Seite ersichtlich.  
% Die Widerstandsgerade wird so verschoben, dass sie nun von einem neuen Ursprungspunkt ausgeht.  
% Der Schnittpunkt dieser Linie mit der Diodenkennlinie entspricht dem Arbeitspunkt der Schaltung.  
% Dort gelten die Werte:  
% \( U_{AP} = 0,75V \) und \( I_{AP} = 1,2mA \).  

% **Teilaufgabe b:**  
% Die elektrische Leistung an der Diode wird aus dem zuvor bestimmten Arbeitspunkt berechnet.  
% Die allgemeine Formel für die Leistung lautet:  
% \( P = U \cdot I \).  
% Hier setzen wir die Werte des Arbeitspunktes ein:  
% \( P = 0,75V \cdot 1,2mA = 0,9mW \).  

% **Teilaufgabe c:**  
% Nun soll der Widerstand \( R \) so bestimmt werden, dass sich ein Strom von \( I = 3mA \) einstellt.  
% In der dritten Abbildung ist erneut eine Diodenkennlinie zu sehen.  
% Zusätzlich wurde eine neue Widerstandsgerade eingezeichnet, die nun durch den Punkt \((1V, 0mA)\) zum Schnittpunkt mit der Diodenkennlinie bei \( I = 3mA \) verläuft.  
% Dieser Schnittpunkt liegt bei der Spannung \( 0,86V \).  

% Der Spannungsabfall am Widerstand ergibt sich aus:  
% \( U_{ges} = U - U_D = 1V - 0,86V = 0,14V \).  
% Da der gewünschte Strom \( I = 3mA \) beträgt, berechnet sich der benötigte Widerstand durch das Ohmsche Gesetz:  
% \( R = \frac{U_{ges}}{I} = \frac{0,14V}{3mA} = 46,67 \Omega \).  

% Das bedeutet, dass für einen Strom von 3mA der Widerstand \( R \) auf 46,67 Ohm angepasst werden muss.  
% }

}
