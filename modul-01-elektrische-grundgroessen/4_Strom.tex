\newvideofile{strom}{Die elektrische Stromstärke}


\section{Die elektrische Stromstärke}

\begin{frame}
	\ftx{Lernziele: Die Elektrische Stromstärke}

	\begin{Lernziele}{Die Elektrische Stromstärke}
        Die Studierenden können
        \begin{itemize}
			\item die Begriffe der elektrischen Stromstärke und der elektrischen Stromdichte erläutern und
			anwenden
            \item die elektrische Stromstärke sowie die elektrische Stromdichte in einfachen Anordnungen berechnen
            \item die Driftgeschwindigkeit von Elektronen in einfachen Anordnungen bestimmen

        \end{itemize}
    \end{Lernziele}
		\speech{lernziele_strom}{1}{In diesem Video werden wir uns die elektrische Stromstärke einmal genauer anschauen.
			Am Ende dieser Lerneinheit solltet ihr die Begriffe der elektrischen Stromstärke und der elektrischen Stromdichte erläutern und
			anwenden können, aber auch die elektrische Stromstärke sowie die elektrische Stromdichte in einfachen Anordnungen berechnen. 
			Weiterhin werden wir uns mit der Driftgeschwindigkeit von Elektronen in einfachen Anordnungen beschäftigen.}

\end{frame}


\subsection{Das elektrische Strömungsfeld}

\s{

Die vorherigen Kapitel beziehen sich ausschließlich auf ruhende Ladungsträger.
Folglich wird dieser Teilbereich der Elektrotechnik als Elektrostatik bezeichnet.



Auf molekularer Ebene bewegen sich die freien Elektronen in einem elektrischen Leiter ständig in zufällige Richtungen.
Im Mittel gleicht sich die Bewegung der einzelnen Elektronen jedoch aus. 
Liegt an einem Leiter ein elektrisches Feld $\vec{E}$ an, so werden die frei 
beweglichen Elektronen in diesem in Richtung der positiv geladenen Elektrode angezogen (siehe Abbildung \ref{fig:stromfeld}).
Zusätzlich zur zufälligen, ungerichteten Bewegung wird eine \textbf{gerichtete Bewegungskomponente} hinzugefügt.
Liegt eine solche gerichtete Bewegung von Teilchen vor, wird von einem \textbf{Strömungsfeld} gesprochen.


 \begin{figure}[h!]
	\centering
	\includesvg[width=0.6\textwidth]{strömungsfeld}
	\caption{Elektrischer Leiter mit eingeprägter elektrischer Feldstärke $\vec{E}$. Die Driftbewegung der Elektronen $\vec{v}_\mathrm{el}$ verläuft in entgegengesetzter Richtung zur Feldstärke.}
	\label{fig:stromfeld}
\end{figure}


 Neben Elektronen in Leitern kann ein elektrisches Strömungsfeld auch durch geladene Ionen in Gasen
  oder Flüssigkeiten hervorgerufen werden.

Im Fall einer im zeitlichen Mittel konstanten Ladungsträgerbewegung, was im abgebildeten
 Leiter durch ein konstantes elektrisches Feld erreicht wird, liegt ein stationäres elektrisches
  Strömungsfeld vor.




}


\begin{frame}
	\ftx{Das elektrische Strömungsfeld}
	\b{

	
	\begin{columns}
       \column[c]{0.5\textwidth}
	   Strömungsfeld: \textbf{gerichtete} Bewegung von Teilchen\\
	   \phantom{Hallo Welt!}\\
	   \phantom{Hallo Welt!}\\
	
	   Stationäres Stromungsfeld: \textbf{zeitlich konstanter} Teilchenstrom\\
	   
	   \phantom{Hallo Welt!}\\
	   \phantom{Hallo Welt!}\\
	
	   Stationäres elektrisches Strömungsfeld: strömende Teilchen \textbf{elektrische Ladungsträger}
	   
	   
	   
	

		\column[c]{0.5\textwidth}

		\begin{figure}[h!]
			\centering
			\includesvg[width=\textwidth]{strömungsfeld}
			
		\end{figure}

    \end{columns}

	\speech{stroemungsfeld}{1}{
		In den vorherigen Videos haben wir uns ausschließlich ruhende Ladungsträger angeschaut, weswegen dieser
Teilbereich der Elektrotechnik als Elektrostatik bezeichnet wird. Nun wollen wir uns jedoch mit der Bewegung von Ladungsträgern beschäftigen. 
Auf molekularer Ebene bewegen sich die freien Elektronen in einem elektrischen Leiter ständig in
zufällige Richtungen. Im Mittel gleicht sich die Bewegung der einzelnen Elektronen jedoch aus. Liegt wie in diesem Beispiel hier
an einem Leiter ein elektrisches Feld an, so werden die frei beweglichen Elektronen in diesem in
Richtung der positiv geladenen Elektrode angezogen. 
Dadurch wird der zufälligen, ungerichteten Bewegung eine gerichtete Bewegungskomponente hinzugefügt, und es wird von einem Strömungsfeld gesprochen.
Neben Elektronen in Leitern kann ein solches elektrisches Strömungsfeld auch durch geladene Ionen in Gasen
oder Flüssigkeiten hervorgerufen werden.
Im Fall einer im zeitlichen Mittel konstanten Ladungsträgerbewegung, was im hier abgebildeten Leiter
durch ein konstantes elektrisches Feld erreicht wird, liegt ein stationäres elektrisches Strömungsfeld vor.}
}
\end{frame}



\subsection{Der elektrische Strom}


\s{
Als elektrische Stromstärke $I$ wird die gerichtete Bewegung (Driftbewegung) von elektrischen Ladungsträgern bezeichnet.
Die Richtung der elektrischen Stromstärke ist so definiert, dass sie vom höheren Potential (positiv geladene Elektrode)
zum niedrigeren Potential verläuft. Damit verläuft sie in die gleiche Richtung wie die elektrische Feldstärke,
jedoch in die zur Driftgeschwindigkeit der Elektronen entgegengesetzen Richtung (siehe Abbildung \ref{fig:strom}). 


 \begin{figure}[h!]
	\centering
	\includesvg[width=0.6\textwidth]{strom}
	\caption{Elektrischer Leiter mit elektrischer Feldstärke $\vec{E}$ und resultierender elektrischer Stromstärke $I$.}
	\label{fig:strom}
\end{figure}


Die Elektronen erreichen eine feldstärkeabhängige
Driftgeschwindigkeit $\vec{v}_\mathrm{el}$. Zusammenstöße mit den Metallatomen aus der Gitterstruktur behindern
die freie Bewegung der Elektronen, und wirken dieser als eine Art Widerstand entgegen. 


\begin{equation}
	\vec{v}_\mathrm{el} = - b_\mathrm{el} \cdot \vec{E}
	\label{eq:driftgeschw}
\end{equation}




\begin{equation*}
	[ \vec{v}_\mathrm{el} ]  = \frac{\mathrm{m}}{\mathrm{s}}
\end{equation*}

Der Proportionalitätsfaktor $b_\mathrm{el}$ zwischen der 
Driftgeschwindigkeit $\vec{v}_\mathrm{el}$ und der elektrischen Feldstärke $\vec{E}$ wird als 
Elektronenbeweglichkeit bezeichnet.

\begin{equation*}
	[b_\mathrm{el}] = \frac{\mathrm{m}^2}{\mathrm{Vs}}
\end{equation*}




Neben der Driftgeschwindigkeit $\vec{v}_\mathrm{el}$ ist auch die Anzahl der Elektronen $N_\mathrm{el}$
sowie deren Ladungsmenge $\Delta Q$ zur Ermittlung der elektrischen Stromstärke $I$ relevant.

\begin{figure}[h!]
	\centering
	\includesvg[width=0.6\textwidth]{strom1}
	\caption{Illustration der Elektronenbewegung durch einen elektrischen Leiter}
	\label{fig:strom1}
\end{figure}

 
Die Ladungsmenge $\Delta Q$ entspricht

\begin{equation*}	
	\Delta Q = e \cdot N_\mathrm{el} = e \cdot n_\mathrm{el} \cdot V_1 \,
\end{equation*}

wobei $n_\mathrm{el}$ die Dichte der Ladungsträger im Volumen $V_1$ angibt.

Mit $V_1 = \Delta s \cdot A$ und $\Delta s = v_\mathrm{el}  \cdot \Delta t$ ergibt sich: 

\begin{equation*}	
	\Delta Q =  e \cdot n_\mathrm{el} \cdot v_\mathrm{el} \cdot \Delta t \cdot A
\end{equation*}

Über die Definition der Driftgeschwindigkeit (\ref{eq:driftgeschw}) führt dies zu:

\begin{equation*}	
	\Delta Q =  e \cdot n_\mathrm{el} \cdot b_\mathrm{el} \cdot E \cdot \Delta t \cdot A \, ,
\end{equation*}

woraus sich die Änderung der Ladungsmenge $\Delta Q$ pro Zeiteinheit $\Delta t$ herleiten lässt:

\begin{equation*}	
	\frac{\Delta Q}{\Delta t} =  e \cdot n_\mathrm{el} \cdot b_\mathrm{el} \cdot E \cdot A
\end{equation*}




\subsection{Definition der elektrischen Stromstärke}

Diese so ermittelte Ladungsänderung über die Zeit wird als \textbf{elektrische Stromstärke} $I$ bezeichnet:

\begin{equation}
	I = \frac{\Delta Q}{\Delta t}
	\label{eq:stromdef}
\end{equation}



\begin{equation*}
	[I] = \mathrm{Ampere} = \mathrm{A}
\end{equation*}



Aus dieser Definition der Stromstärke folgt, dass eine über einen Zeitraum aufaddierte elektrische 
Stromstärke die in dieser Zeit transporte Ladungsmenge $Q$ ergibt:

\begin{equation}
	Q = \int_{t_1}^{t_2} i(t) \, \mathrm{d} t	
\end{equation}

Neben der gesamten elektrischen Stromstärke $I$ ist oft auch der auf den Leiterquerschnitt bezogene Strom von Interesse.
Zur Berechnung dieser \textbf{Stromdichte} $J$ wird die Stromstärke $\Delta I$ durch ein elementar kleines Flächenelement $\Delta A$
betrachtet: 

\begin{equation}
	J = \frac{\Delta I}{\Delta A}
\end{equation}

Ist die elektrische Stromstärke $I$ über eine Fläche gleichmäßig verteilt, lässt sich dies vereinfachen:

\begin{equation}
	J = \frac{I}{A}
\end{equation}

Da die Stromstärke $I$ über den gesamten Leiter hinweg identisch bleibt, ändert sich bei einer 
Querschnittsverkleinerung des Leiters auch die resultierende elektrische Stromdichte $J$ (siehe Abbildung \ref{fig:stromdichte}).

\begin{figure}[h!]
	\centering
	\includesvg[width=0.6\textwidth]{stromdichte}
	\caption{Zusammenhang zwischen der Stromdichte $J_1$ bei Querschnitt $A_1$ und $J_2$ bei $A_2$}
	\label{fig:stromdichte}
\end{figure}

Hier gilt:

\begin{equation*}
	J_1 \cdot A_1 = J_2 \cdot A_2 = I
\end{equation*}





\begin{bsp}{}{}

	Durch eine Kupferleitung mit dem Querschnitt $A = 1 \mathrm{mm}^2$ fließt ein Strom von $I = 8 \mathrm{A}$. 
	Ein $\mathrm{mm}^3$ enthält etwa $ 8,5 \cdot 10^{19}$ Atome. Jeweils 1 Elektron pro Atom sei am Ladungstransport beteiligt.

	\begin{center}
		\includesvg[width=0.5\textwidth]{radialfeld}
	\end{center}
	

	

	Mit welcher Driftgeschwindigkeit bewegen sich die Elektronen im Mittel durch die Leitung?

	\begin{equation*}
		\vec{v}_\mathrm{el} = - b_\mathrm{el} \cdot \vec{E}
	\end{equation*}

	Da sich die negativ geladenen Elektronen entgegen der Feldrichtung $\vec{E}$ (und somit auch entgegen der
	 technischen Stromrichtung $I$) bewegen, ist die Richtung der Driftgeschwindigkeit ausreichend beschrieben,
	 und es genügt, den Betrag $v_\mathrm{el}$ zu berechen.


	mit dem Zusammenhang

	\begin{equation*}
	 I = e \cdot n_\mathrm{el} \cdot b_\mathrm{el} \cdot E  \cdot A\\
\rightarrow E = \frac{I}{e \cdot n_\mathrm{el} \cdot b_\mathrm{el} \cdot A}
	\end{equation*}

	

	ergibt sich:

\begin{equation*}
	v_\mathrm{el} = \cancel{b_\mathrm{el}} \cdot \frac{I}{e \cdot n_\mathrm{el} \cdot \cancel{b_\mathrm{el}} \cdot A}
\end{equation*}	

Einsetzen der Zahlenwerte:

\begin{equation*}
	v_\mathrm{el} = \frac{8 \mathrm{A}}{1,602 \cdot 10^{-19} \, \mathrm{As} \cdot 8,5 \cdot 10^{19} \, \mathrm{mm}^{-3} \cdot 1 \, \mathrm{mm}^2 }
\end{equation*}	


\begin{equation*}
	v_\mathrm{el} = 0,59 \frac{\mathrm{mm}}{s} 
\end{equation*}	

Die Elektronen bewegen sich im Mittel also mit 0,59 mm/s. Da sich innerhalb des Leiters die elektrische Feldstärke
jedoch mit Lichtgeschwindigkeit fortbewegt, setzen sich alle Elektronen des Leiters praktisch gleichzeitig in Bewegen.
Vergleichbar ist dies mit einem vollständig mit Murmeln gefüllten Rohr. Im Moment des hereinschiebens einer weiteren 
Murmel fällt auf der anderen Seite eine Murmel heraus, obwohl sich die einzelnen Murmeln nur langsam bewegen. 






\end{bsp}

}


\newpage









\begin{frame}
	\b{
	\ftx{Elektrischer Strom I}

	

	

	
	\begin{columns}
       \column[c]{0.6\textwidth}

	   Elektrischer Strom: gerichtete Driftbewegung von Ladungsträgern\\

	   \phantom{text}\\
	   \phantom{text}\\


	   Feldstärkeabhängige Driftgeschwindigkeit $\vec{v}_\mathrm{el}$:

		\phantom{.}\\
		\phantom{text}\\

		
		$\vec{v}_\mathrm{el} = - b_\mathrm{el} \cdot \vec{E}$\\

		\phantom{.}\\

		

		mit:\\
		 $\vec{v}_\mathrm{el}$ = Driftgeschwindigkeit der Elektronen in $\mathrm{m/s}$\\

		 $b_\mathrm{el}$ = Elektronenbeweglichkeit in $\mathrm{cm}^2/\mathrm{Vs}$





	

		
		\column[c]{0.4\textwidth}


		\begin{figure}[h!]
						\centering
						\includesvg[width=\textwidth]{strömungsfeld}
						
					\end{figure}
    \end{columns}
		\speech{strom1}{1}{
		Diese gerichtete Bewegung von elektrischen Ladungsträgern wird als elektrische Stromstärke  mit dem Formelbuchstaben Ih bezeichnet. Dabei ist ihre Richtung so definiert, 
		dass sie vom höheren Potential zum niedrigeren Potential, also von Plus nach Minus verläuft. 
Damit verläuft sie in die gleiche Richtung wie die bereits bekannte elektrische Feldstärke.
Da Elektronen jedoch negativ geladen sind, also vom Pluspol angezogen werden, bewegen diese sich im Mittel in die entgegengesetzte Richtung.
Um auf diesen Fakt aufmerksam zu machen, wird in einigen Zusammenhängen inoffiziell zwischen der technischen Stromrichtung und der physikalischen Stromrichtung,
 welche die Elektronenbewegung beschreibt, unterschieden.
In Abhängigkeit von der eletrischen Feldstärke erreichen die Elektronen eine mittlere Geschwindigkeit, die sogenannte Driftgeschwindigkeit, welche wir hier mit V E L bezeichnen. 
Zwar werden die Elektronen durchgehend vom elektrischen Feld beschleunigt, jedoch behindern Kollisionen der Elektronen mit 
Metallatomen aus der Gitterstruktur die freie Bewegung der Elektronen, und wirken dieser
als eine Art Widerstand entgegen.
Dabei wird der Proportionalitätsfaktor B E L zwischen der Driftgeschwindigkeit V EL und der elektrischen Feldstärke als Elektronenbeweglichkeit bezeichnet.}
	}
\end{frame}

\begin{frame}
	\b{
	\ftx{Elektrischer Strom II}

	
	\begin{columns}
       \column[c]{0.6\textwidth}
	  Ladungsträgertransport innerhalb eines Leiters:\\

		\phantom{.}\\

		\phantom{text}\\
		$\Delta Q = e \cdot N_\mathrm{el} $\\

		\phantom{text}\\

		$\Delta Q = e \cdot n_\mathrm{el} \cdot V_1$\\

		\phantom{text}\\

	$\Delta Q = e \cdot n_\mathrm{el} \cdot b_\mathrm{el} \cdot E \cdot \Delta t \cdot A$

		\phantom{.}\\

		mit:\\
		$N_\mathrm{el} =$ Anzahl Ladungsträger in betrachtetem Volumen\\

		$n_\mathrm{el} = $  Ladungsträgerdichte in $1/\mathrm{m}^3$






	

		
		\column[c]{0.4\textwidth}
		


		\begin{figure}[h!]
			\centering
			\includesvg[width=\textwidth]{strom1}
			
		\end{figure}

    \end{columns}
	\speech{strom2}{1}{Neben der Driftgeschwindigkeit ist auch die Anzahl der Elektronen sowie die damit verbundene Ladungsmenge
 zur Ermittlung der elektrischen Stromstärke relevant.
 Die gesamte Ladungsmenge, die sich bewegt, wird als Delta Kuh, also der Anderung von Kuh bezeichnet. 
 Sie ist das Produkt der Elementarladung E mit der Anzahl der freien Ladungsträger im betrachteten Raumabschnitt. Bei vielen Metallen wie beispielsweise Kupfer 
 geht man in der Regel davon aus, dass es pro Kupferatom im Schnitt etwa Ein freies, schwach gebundenes Elektron gibt, welches zum Ladungstransport genutzt werden kann.
 Die absolute Anzahl der Ladungsträger, Groß N, erhält man über die Dichte dieser Ladungsträger pro Volumen, multipliziert mit dem betrachteten Volumen, hier also das dunkelgelb
  hinterlegte Volumen V eins. Die Ladungsträgerdichte ist eine Materialabhängige Eigenschaft, welche beispielsweise in Tabellenbüchern nachgeschlagen werden kann.
  Das Volumen V eins kann als Produkt der Fläche A mit der betrachteten Strecke Delta S gesehen werden, die das Produkt der Driftgeschwindigkeit V EL mit dem betrachteten
   Zeitabschnitt Delta T repräsentiert. Mit dem Wissen aus der vorherigen Folie, dass die Driftgeschwindigkeit aus dem Produkt der Elektronenbeweglichkeit B EL mit dem elektrischen Feld
   berechnet werden kann, erhalten wir für die Änderung der Ladungsmenge Delta Kuh gleich die Elementarladung 
   mal der Ladungsträgerdichte und Elektronenbeweglichkeit, multipliziert mit dem E Feld, der vergangenen Zeit und der Querschnittsfläche des Leiters.
	}




	}
\end{frame}






\begin{frame}
	\b{
	\ftx{Definition der elektrischen Stromstärke}

	\begin{columns}
		\column[c]{0.5\textwidth}
	
		Stromstärke $I:$ Ladungsmenge $\Delta Q$ in $\Delta t$ durch Leiter \\

		\phantom{.}\\



\begin{Merksatz}{}
	$I = \frac{\Delta Q}{\Delta t}$\\
\end{Merksatz}

\begin{equation*}
	[ I ] = \mathrm{Ampere} = \mathrm{A}
\end{equation*}


\phantom{text}\\



$ I = e \cdot n_\mathrm{el} \cdot b_\mathrm{el} \cdot E  \cdot A$\\





		
	

		\column[c]{0.5\textwidth}


		\begin{figure}[h!]
			\centering
			\includesvg[width=\textwidth]{strom}
			
		\end{figure}

	

	\end{columns}
	\speech{stromstaerke}{1}{
		Die elektrische Stromstärke ist also die Ladungsänderung pro Zeitänderung, oder mathematisch ausgedrückt die zeitliche Ableitung der Ladung.
		Daraus folgt, dass man aus dem Aufsummieren beziehungsweise Integrieren der elektrischen Ströme die transportierte Ladungsmenge Q ermitteln kann, welche beispielsweise in einem Bauteil gespeichert wird.
		Die elektrische Stromstärke wird in Ampere angegeben, und ist eine unserer sieben S Ih Basiseinheiten, aus denen alle anderen Größen abgeleitet werden können.}
	}
\end{frame}


\begin{frame}
	\b{
	\ftx{Definition der elektrischen Stromdichte}

	\begin{columns}
		\column[c]{0.5\textwidth}
	
		Stromstärke $J:$ Stromstärke $\Delta I$ pro Leiterquerschnitt $\Delta A$ \\

		\phantom{.}\\



\begin{Merksatz}{}
	$J = \frac{\Delta I}{\Delta A}$\\
\end{Merksatz}

\begin{equation*}
	[ J ] = \frac{\mathrm{A}}{\mathrm{m}^2}
\end{equation*}


\phantom{text}\\

\begin{equation*}
	J_1 \cdot A_1 = J_2 \cdot A_2 = I
\end{equation*}

	\column[c]{0.5\textwidth}

	\begin{figure}[h!]
		\centering
		\includesvg[width=\textwidth]{stromdichte}	
	\end{figure}

	\end{columns}
	\speech{stromdichte}{1}{Neben der gesamten elektrischen Stromstärke ist oft auch der auf den Leiterquerschnitt bezogene
Strom von Interesse. 
Zur Berechnung dieser sogenannten Stromdichte mit dem Formelzeichen J wird die Stromstärke Delta I durch ein
elementar kleines Flächenelement Delta A betrachtet. 
Ist die elektrische Stromstärke über die betrachtete Fläche gleichmäßig verteilt, wie es in vielen der von uns betrachteten Anwendungen der Fall ist, lässt sich dies vereinfachen, und 
die Stromdichte kann über J gleich I durch A ermittelt werden.
Zur Verdeutlichung des Konzepts schauen wir uns den hier gezeigten Leiter mit einer Querschnittsänderung rechts an.
Da die Stromstärke I über den gesamten Leiter hinweg identisch bleibt, haben die Elektronen nach der Verkleinerung des Querschnitts weniger Platz, sie rücken also näher zusammen.
Als Ergebnis steigt die Stromdichte in diesem Fall.}
}
\end{frame}



