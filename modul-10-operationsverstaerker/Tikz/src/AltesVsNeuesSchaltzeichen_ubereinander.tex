\begin{circuitikz}
   \ctikzset{tripoles/en amp/input height=0.45}

   % Altes Schaltzeichen
   \draw (0,0) node[op amp] (amerOpAmp) {};
   \draw (amerOpAmp.up) -- ++(0,0.2) node[ocirc]{} node[above]{$+U_B$};
   \draw (amerOpAmp.down) -- ++(0,-0.2) node[ocirc]{} node[below]{$-U_B$};
   \draw (amerOpAmp.out) node[ocirc](oldOut){} node[right,yshift=-0.5cm]{};
   \draw (amerOpAmp.-) node[ocirc](oldMinus){} node[left]{};
   \draw (amerOpAmp.+) node[ocirc](oldPlus){} node[left]{};
   \node at (0,-1.5) {\textbf{altes Schaltzeichen}};
   
   % Ground-Symbol und Spannungspfeil für altes Schaltzeichen
   \draw[-{Triangle[width=3pt,length=4pt]}, color=spannung] ($(oldOut) + (0, -0.1)$) -- ($(oldOut) + (0, -1)$) node[midway, right] {$U_A$};
   \draw[black] ($(oldOut) + (0, -1.05)$) -- ++(-0.2, 0) -- ++(0.4,0);
   % Spannungspfeil U_Diff für altes Schaltzeichen
   \draw[-{Triangle[width=3pt,length=4pt]}, color=spannung] ($(oldPlus) + (0, 0.1)$) -- ($(oldMinus) + (0, -0.1)$) node[midway, left] {$U_{\text{Diff}}$};

   % Neues Schaltzeichen
   \draw (0,-4) node[en amp](E){};
   \draw (E.out)  node[ocirc](newOut){} node[right,yshift=-0.5cm]{};
   \draw (E.-) node[ocirc](newMinus){} node[left]{};
   \draw (E.+) node[ocirc](newPlus){} node[left]{};
   \node at (0,-6) {\textbf{neues Schaltzeichen}};
   \draw (E.up) -- ++(0,0.2) node[ocirc]{} node[above]{$+U_B$};
   \draw (E.down) -- ++(0,-0.2) node[ocirc]{} node[below]{$-U_B$};
   
   % Ground-Symbol und Spannungspfeil für neues Schaltzeichen
   \draw[-{Triangle[width=3pt,length=4pt]}, color=spannung] ($(newOut) + (0, -0.1)$) -- ($(newOut) + (0, -1)$) node[midway, right] {$U_A$};
   \draw[black] ($(newOut) + (0, -1.05)$) -- ++(-0.2, 0) -- ++(0.4,0);

   % Spannungspfeil U_Diff für neues Schaltzeichen
   \draw[-{Triangle[width=3pt,length=4pt]}, color=spannung] ($(newPlus) + (0, 0.1)$) -- ($(newMinus) + (0, -0.1)$) node[midway, left] {$U_{\text{Diff}}$};

\end{circuitikz}