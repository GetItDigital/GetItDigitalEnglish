\subsection{Untersuchung der Verstärkung eines OPs mit variablen Widerständen\label{Aufg24}}
% Alt: Praktikumsaufgabe 10
\Aufgabe{
    Untersuchen Sie die Verstärkung eines nicht-invertierenden Operationsverstärkers mit variablen Widerständen und berechnen Sie die Ausgangsspannung $U_\mathrm{A}$.
    Gegeben:
    \begin{itemize}
        \item Versorgungsspannung: $\pm 15\,\mathrm{V}$
        \item Eingangsspannung: $U_\mathrm{E} = 1\,\mathrm{V}$
        \item Rückkopplungswiderstand: $R_\mathrm{R} = 100\,\Omega$
        \item Eingangswiderstand: $R_\mathrm{E} = 680\,\Omega$
        \item Weitere mögliche Widerstandswerte: $R_\mathrm{R} = 220\,\Omega$, $R_\mathrm{E} = 680\,\Omega$
        \item Lastwiderstand: $R_\mathrm{Last} = 10\,\mathrm{k\Omega}$
    \end{itemize}
    
    \begin{itemize}
        \item[a)] Berechnen Sie die Verstärkung $V$ und die Ausgangsspannung $U_\mathrm{A}$ für die gegebenen Widerstandswerte.
        \item[b)] Wie ändert sich die Verstärkung, wenn $R_\mathrm{R}$ von $100\,\Omega$ auf $220\,\Omega$ erhöht wird?
        \item[c)] Welche Auswirkungen hat eine Änderung von $R_\mathrm{E}$ auf die Verstärkung und die Ausgangsspannung?
        \item[d)] Welche Begrenzungen für $U_\mathrm{A}$ ergeben sich durch die Versorgungsspannung?
    \end{itemize}
    \begin {figure} [H]
    \centering
    \begin{tikzpicture}
    \draw
    (0,0) node[op amp, anchor=+] (opamp) {}
    (opamp.+) 
    (opamp.-) 
    (opamp.out) ;
    \draw
        (opamp.up) -- ++(0,0.5) node[right] {$+$}; % Text leicht rechts
        \draw
        (opamp.down) -- ++(0,-0.5) node[right] {$-$}; % Text leicht rechts
    
    \draw (opamp.out) to[short, -*] (4,0.5)to[R,l=$R_\mathrm{Last}$] (4,-1.5) to[short,-*] (-1,-1.5)node[ground]{} -- (-1,0) to (opamp.+);
    
    \draw (-1,-1.5) -- (-3.5,-1.5);
    
    \draw (opamp.-) to[short, -*] (-1,1) -- (-1,2) -- (1,2) to[R,l=$R_\mathrm{R}$](4,2)--(4,0.5);
    \draw (-3.8,1) to[R,l=$R_\mathrm{E}$] (-1,1);
    
    \draw (opamp.up) -- (1.1,4) to[short, -*] (-4,4) -- (-6,4) to[V,name=V1] (-6,2)node[ground]{};
    
    \draw (opamp.down) -- (1.1,-3) to[short,-*](-4,-3) -- (-6,-3) to[V,name=V2] (-6,-5)node[ground]{};
    \draw (-4,-3) to[R,l=$R_\mathrm{2}$] (-4,0) to[potentiometer, l=$P_\mathrm{1}$] (-4,2) to [R,l=$R_\mathrm{1}$](-4,4);
    
    \draw (-3.1,1) to[open,v,name=ue](-3.1,-1.5);
    \draw (3,0.5) to[open,v,name=ua](3,-1.5);
    
    \varrmore{ua}{$U_\mathrm{A}$};
    \varrmore{ue}{$U_\mathrm{E}$};
    \varrmore{V1}{$U_\mathrm{1}$};
    \varrmore{V2}{$U_\mathrm{2}$};
\end{tikzpicture}
   
    \label{fig:FigPrakNIOPV}
    \end {figure}
}

\Loesung{
        \textbf{a) Berechnung der Ausgangsspannung $U_\mathrm{A}$ für gegebene Werte} \\
        Gegeben sind:
        \[
        U_\mathrm{E} = 1\,\mathrm{V}, \quad R_\mathrm{R} = 100\,\Omega, \quad R_\mathrm{E} = 680\,\Omega
        \]
        Einsetzen in die Verstärkungsformel:
        \[
        V = 1 + \frac{100\,\Omega}{680\,\Omega} = 1 + 0{,}147 = 1{,}147
        \]
        Die Ausgangsspannung berechnet sich zu:
        \[
        U_\mathrm{A} = V \cdot U_\mathrm{E} = 1{,}147 \cdot 1\,\mathrm{V} = 1{,}147\,\mathrm{V}
        \]
    
        \textbf{b) Änderung von $R_\mathrm{R}$ auf $220\,\Omega$} \\[0.2cm]
        Wenn $R_\mathrm{R} = 220\,\Omega$ gewählt wird:
        \[
        V = 1 + \frac{220\,\Omega}{680\,\Omega} = 1 + 0{,}323 = 1{,}323
        \]
        \[
        U_\mathrm{A} = 1{,}323 \cdot 1\,\mathrm{V} = 1{,}323\,\mathrm{V}
        \]
    
        \textbf{c) Einfluss von $R_\mathrm{E}$ auf die Verstärkung} \\[0.2cm]
        Eine Vergrößerung von $R_\mathrm{E}$ führt zu einer kleineren Verstärkung $V$, während eine Verkleinerung von $R_\mathrm{E}$ die Verstärkung erhöht.  
        In der Praxis sollte $R_\mathrm{E}$ so gewählt werden, dass das Ausgangssignal nicht zu stark schwankt.
    
        \textbf{d) Begrenzung von $U_\mathrm{A}$ durch die Versorgungsspannung} \\[0.2cm]
        Die Ausgangsspannung $U_\mathrm{A}$ kann niemals über die Versorgungsspannung $\pm 15\,\mathrm{V}$ hinausgehen.  
        Falls $U_\mathrm{A}$ rechnerisch diesen Bereich überschreiten würde, wird die Ausgangsspannung auf maximal $15\,\mathrm{V}$ begrenzt.
        Siehe: Abschnitt \ref{fig: Verschaltung Verstaerker} und Tabelle \ref{tab:Grundschaltungen}
}