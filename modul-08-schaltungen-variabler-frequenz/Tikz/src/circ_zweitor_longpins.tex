% Schaltsymbol eines Zweitors (Vierpols)
\begin{tikzpicture}%
%   invisible boundbox
    \draw[draw=none](-2.5,-1.25) rectangle (+2.5,+1.25);% gleiche Höhe wie bei circ_zweitor.tex und circ_zweitor_quelle_last.tex
%   Zweitor (Vierpol)
    \node[draw=none] {Zweitor};%
    \draw(0,0) node[dipchip, hide numbers, num pins=6, no topmark, external pins width=0](C){};% Mittelpunkt des Vierpols
    \draw(C.bpin 1) to[short,-o, name=i11, i_<, !i] ++(-1,0) coordinate(In1);%
    \draw(C.bpin 3) to[short,-o, name=i12, i_>, !i] ++(-1,0) coordinate(In2);%
    \draw(C.bpin 6) to[short,-o, name=i21,  i>, !i] ++(1,0) coordinate(Out1);%
    \draw(C.bpin 4) to[short,-o, name=i22,  i<, !i] ++(1,0) coordinate(Out2);%
    \draw(In1) to[open, name=u1, v, !v] (In2);%
    \draw(Out1) to[open, name=u2, v^, !v] (Out2);%

    %draw arrows
    \varrmore{u1}{$u_1$};
    \varrmore{u2}{$u_2$};
    %\iarrmore{i11}{$I_1$};
    %\iarrmore{i12}$I_1$;
    %\iarrmore{i21}$I_2$;
    %\iarrmore{i22}$I_2$;
\end{tikzpicture}%
