\subsection{Berechnung der Stromverstärkung einer Transistorschaltung\label{Aufg18}}
% Alt: Aufgabe 18
\Aufgabe{
    Gegeben ist die folgende Schaltung mit dem Kennlinienfeld des Transistors. Die Versorgungsspannung beträgt \( U_\mathrm{CC} = \mathrm{6\,V} \). \\
    Der Arbeitspunkt liegt bei der halben Betriebsspannung.\\
    Bestimmen Sie die folgenden Werte aus dem Kennlinienfeld:
    
    \begin{itemize}
      \item Die Paare \( I_\mathrm{B} / U_\mathrm{BE} \) und \( I_\mathrm{C} / U_\mathrm{CE} \) im Arbeitspunkt,
      \item sowie die Stromverstärkung \( \beta \).
    \end{itemize}
    
  \begin{figure}[H]
    % oben
    \begin{minipage}[c]{\textwidth} 
        \centering
        \begin{tikzpicture}
\draw
    (0,0) node[npn, tr circle] (Q1) {}
    (Q1.base) 
    (Q1.collector)  
    (Q1.emitter);

\draw (Q1.base) to[short, -o] (-3,0);

\draw (Q1.emitter) to[short, -*] (0,-3)node[ground]{};

\draw (Q1.collector) to[short, -*] (0,1) to[short, R, l=$R_\mathrm{C}$, -*] (0,3);

\draw (-3,3) -- (3,3) node[above]{$U_\mathrm{CC}$};

\draw (0,1) to[short, -o] (3,1);
\draw (-3,-3) to[short, o-o] (3,-3);

\draw (-3,0) to[open,v>, name=ue] (-3,-3);
\draw (3,1) to[open, v>, name=ua] (3,-3);

\varrmore{ue}{$U_\mathrm{E}$};
\varrmore{ua}{$U_\mathrm{A}$};

\draw (-2,3) to[R,l=$R_{2}$, *-*] (-2,0) to[R,l=$R_{1}$, -*] (-2,-3);
\end{tikzpicture}
    \end{minipage}
    \vspace{0.5cm} % Vertikaler Abstand zur zweiten Reihe

    % unten
    \begin{minipage}[c]{0.48\textwidth} 
        \centering
        \includesvg[width=\textwidth]{Bilder/Aufgaben/FigTransistorStromverstaerkung2.svg}
    \end{minipage}
    \hfill 
    \begin{minipage}[c]{0.48\textwidth} 
        \centering
        \includesvg[width=\textwidth]{Bilder/Aufgaben/FigTransistorStromverstaerkung3.svg}
    \end{minipage}
    \label{fig:FigTransistorStromverstaerkung}
\end{figure}

% \speech{
% Aufgabe 18

% Diese Aufgabe behandelt die Bestimmung der Stromverstärkung $\beta$ eines Transistors anhand der gegebenen Kennlinien und einer bestimmten Schaltung.

%  **Aufgabenstellung:**
% Es ist eine Transistorschaltung gegeben, bei der die **Versorgungsspannung $U_{CC} = 6V$** beträgt. Der **Arbeitspunkt** des Transistors soll sich auf **die halbe Betriebsspannung** einstellen.

% Gefragt sind die folgenden Bestimmungen:
% 1. Die **Wertepaare** für den Basisstrom $I_B$ und die Basis-Emitter-Spannung $U_{BE}$.
% 2. Die **Wertepaare** für den Kollektorstrom $I_C$ und die Kollektor-Emitter-Spannung $U_{CE}$.
% 3. Die Berechnung der **Stromverstärkung $\beta$** des Transistors.


%  **Beschreibung der Abbildungen:**
%  **Obere Abbildung: Transistorschaltung**
% - Es handelt sich um eine **Spannungsteiler-Basisbeschaltung eines NPN-Transistors**.
% - Die **Versorgungsspannung $U_{CC}$** befindet sich oben an der Schaltung.
% - Der **Kollektorwiderstand $R_C$** ist zwischen den Kollektor des Transistors und $U_{CC}$ geschaltet.
% - Zwei Widerstände **$R_1$ und $R_2$ bilden einen Spannungsteiler**, um die Basisspannung des Transistors festzulegen.
% - Der **Emitter des Transistors ist auf Masse geschaltet**.
% - Die **Ausgangsspannung $U_A$ wird zwischen Kollektor und Masse gemessen**.

%  **Linke untere Abbildung: Ausgangskennlinienfeld des Transistors**
% - Die **x-Achse stellt die Kollektor-Emitter-Spannung $U_{CE}$ in Volt dar**.
% - Die **y-Achse stellt den Kollektorstrom $I_C$ in mA dar**.
% - Es sind **mehrere rote Kennlinien** eingezeichnet, die den Zusammenhang zwischen $I_C$ und $U_{CE}$ für verschiedene Basisströme $I_B$ darstellen.
% - Die Basisströme $I_B$ sind in Stufen von **5 μA bis 30 μA** gegeben.
% - Die Linien verlaufen steil ansteigend bei niedrigen $U_{CE}$-Werten und flachen im Sättigungsbereich ab.

%  **Rechte untere Abbildung: Eingangskennlinie des Transistors**
% - Die **x-Achse stellt die Basis-Emitter-Spannung $U_{BE}$ in Volt dar**.
% - Die **y-Achse stellt den Basisstrom $I_B$ in μA dar**.
% - Die Kennlinie verläuft exponentiell steigend ab etwa **0,6V**.
% - Dies zeigt das **typische Verhalten einer pn-Übergangsdiode**, bei der der Basisstrom ab einer bestimmten Schwellenspannung rapide ansteigt.



%  **Zusätzliche Hinweise für eine blinde Person**
% - Die Schaltung verwendet eine **Spannungsteilerschaltung** zur Basisspannungseinstellung.
% - Die Werte für **$I_B$ und $U_{BE}$** können aus der Eingangskennlinie abgelesen werden.
% - Die Werte für **$I_C$ und $U_{CE}$** können aus der Ausgangskennlinie bestimmt werden.
% - Die **Stromverstärkung $\beta$** ergibt sich aus dem Verhältnis **$I_C / I_B$**.
% }


}



\Loesung{
    \begin{figure}[H]
        \centering
        \begin{subfigure}[b]{0.45\textwidth}
            \centering
            \includesvg[width=0.9\textwidth]{Bilder/Aufgaben/LsgTransistorStromverstaerkung1.svg}
            \label{fig:LsgTransistorStromverstaerkung1}
        \end{subfigure}
        \begin{subfigure}[b]{0.45\textwidth}
            \centering
            \includesvg[width=0.9\textwidth]{Bilder/Aufgaben/LsgTransistorStromverstaerkung2.svg}
            \label{fig:LsgTransistorStromverstaerkung2}
        \end{subfigure}
    \end{figure}
        Gegeben sind folgende Werte im Arbeitspunkt:
        \begin{align*}
            U_\mathrm{CC} &= \mathrm{6\,V} \\
            U_\mathrm{CE} &= \mathrm{3\,V} \\
            I_\mathrm{C} &= \mathrm{6\,mA} \\
            U_\mathrm{BE} &= \mathrm{0{,}62\,V} \\
            I_\mathrm{B} &= \mathrm{20\,\mu A}
        \end{align*}
        
        Berechnung der Stromverstärkung \( \beta \):
        \begin{align*}
            \beta &= \frac{I_\mathrm{C}}{I_\mathrm{B}} = \frac{\mathrm{6\,mA}}{\mathrm{20\,\mu A}} = \mathrm{300}
        \end{align*}
        
    Siehe: Abschnitt \ref{sec:ElektrischesVerhaltenBipolartransistor}
%     \speech{
% Die Lösung zu Aufgabe 18 besteht aus mehreren Teilen.

% **Teil a)**  
% Im ersten Schritt wird der Arbeitspunkt (AP) anhand der gegebenen Werte bestimmt.  
% Dazu werden zwei Diagramme betrachtet:

% **Linkes Diagramm:**  
% - Es zeigt die Ausgangskennlinien des Transistors.  
% - Die x-Achse stellt die Kollektor-Emitter-Spannung \( U_{CE} \) dar, während die y-Achse den Kollektorstrom \( I_C \) anzeigt.  
% - Mehrere rote Kurven repräsentieren verschiedene Basisströme \( I_B \) in Mikroampere (µA).  
% - Die gestrichelte schwarze Linie stellt die Widerstandsgerade dar, die vom Ursprung aus zum Arbeitspunkt (AP) verläuft.  
% - Der Arbeitspunkt AP ist durch einen schwarzen Kreis markiert und liegt bei einer Kollektor-Emitter-Spannung von \( U_{CE} = 3V \) und einem Kollektorstrom \( I_C = 6mA \).

% **Rechtes Diagramm:**  
% - Es zeigt die Kennlinie der Basis-Emitter-Spannung \( U_{BE} \).  
% - Die x-Achse stellt die Spannung \( U \) in Volt dar, während die y-Achse den Basisstrom \( I_B \) in Mikroampere (µA) angibt.  
% - Die blaue Kurve beschreibt den exponentiellen Anstieg des Basisstroms mit zunehmender Spannung.  
% - Der relevante Punkt liegt bei einer Basis-Emitter-Spannung \( U_{BE} = 0,62V \) und einem Basisstrom \( I_B = 20\mu A \).

% **Berechnungen:**  
% - Die Versorgungsspannung beträgt \( U_{CC} = 6V \).  
% - Die Kollektor-Emitter-Spannung beträgt \( U_{CE} = 3V \).  
% - Der Kollektorstrom ist \( I_C = 6mA \).  
% - Die Basis-Emitter-Spannung beträgt \( U_{BE} = 0,62V \).  
% - Der Basisstrom beträgt \( I_B = 20\mu A \).  
% - Der Verstärkungsfaktor \( \beta \) wird mit der Formel \( \beta = \frac{I_C}{I_B} \) berechnet.  
%   Durch Einsetzen der Werte \( I_C = 6mA \) und \( I_B = 20\mu A \) ergibt sich:  
%   \( \beta = \frac{6mA}{20\mu A} = 300 \).

% Damit ist die Lösung zu Aufgabe 18 abgeschlossen.
% }

}