\subsection{Knoten- und Maschenanalyse 3}
\Aufgabe{
    Gegeben ist unten stehendes Netzwerk.

    \begin{figure}[H]
        \resizebox{\textwidth}{!}{%  
        \begin{circuitikz}
            \draw (0,2) to [V] (0,0)
            (0,2) to [R=$R_\mathrm{i}$] (0,4)
            to [short] (2,4)
            to [R=$R_\mathrm{1}$,-*] (4,4)
            to (4,3) to [R=$R_\mathrm{3}$] (4,1) to [short,-*] (4,0)
            to [R=$R_\mathrm{2}$] (2,0) to (0,0);
            \draw (4,0) to (5,0) to [R=$R_\mathrm{5}$] (7,0) to [short,-*] (8,0)
            to (8,1) to [R=$R_\mathrm{6}$] (8,3) to [short,-*] (8,4)
            to (7,4) to [R=$R_\mathrm{4}$] (5,4) to (4,4);
            \draw (8,0) to (9,0) to [R=$R_\mathrm{7}$] (11,0) to [short,-*] (12,0)
            (12,4) to (12,3) to [V] (12,1) to [short,-*] (12,0) 
            (12,4) to (11,4) to [R=$R_\mathrm{9}$] (9,4) to (8,4);
            \draw (12,4) to [short,*-] (16,4) to (16,3) to [R=$R_\mathrm{8}$] (16,1) to (16,0) to (12,0);
        \end{circuitikz}
        }\centering                                                           
    \end{figure}

    

    \begin{itemize}
        \item[\bf a)]
            Markieren und bennen Sie alle Ströme und Spannungen!
        \item[\bf b)]
            Stellen Sie die Gleichungen für alle Knoten auf!
        \item[\bf c)]
            Stellen Sie die Gleichungen für alle Maschen auf!
    \end{itemize}
}

\Loesung{
	\begin{itemize}
        \item[\bf a)]
        Netzwerk mit Strömen und Spannungen:\\
        \begin{figure}[H]
            \resizebox{\textwidth}{!}{%  
            \begin{circuitikz}
                \draw (0,2) to [V, name=U1] (0,0)
                (0,2) to [R=$R_\mathrm{i}$, i, v, name=Ri] (0,4)
                to [short] (2,4)
                to [R=$R_\mathrm{1}$, i, v, name=R1,-*] (4,4)
                to (4,3) to [R=$R_\mathrm{3}$, i, v, name=R3] (4,1) to [short,-*] (4,0)
                to [R=$R_\mathrm{2}$, i, v, name=R2] (2,0) to (0,0);
                \draw (4,0) to (5,0) to [R=$R_\mathrm{5}$, i, v, name=R5] (7,0) to [short,-*] (8,0)
                to (8,1) to [R=$R_\mathrm{6}$, i, v, name=R6] (8,3) to [short,-*] (8,4)
                to (7,4) to [R=$R_\mathrm{4}$, i, v, name=R4] (5,4) to (4,4);
                \draw (8,0) to (9,0) to [R=$R_\mathrm{7}$, i, v, name=R7] (11,0) to [short,-*] (12,0)
                (12,4) to (12,3) to [V, i, name=U2] (12,1) to [short,-*] (12,0) 
                (12,4) to (11,4) to [R=$R_\mathrm{9}$, i, v, name=R9] (9,4) to (8,4);
                \draw (12,4) to [short,*-] (16,4) to (16,3) to [R=$R_\mathrm{8}$, i, v, name=R8] (16,1) to (16,0) to (12,0);
                
                \varrmore{U1}{$U_1$};
                \varrmore{U2}{$U_2$};
                \iarrmore{U2}{$I_\mathrm{2}$};
                \varrmore{Ri}{$U_\mathrm{Ri}$};
                \iarrmore{Ri}{$I_\mathrm{1}$};
                \varrmore{R1}{$U_\mathrm{R1}$};
                \varrmore{R2}{$U_\mathrm{R2}$};
                \varrmore{R3}{$U_\mathrm{R3}$};
                \iarrmore{R3}{$I_\mathrm{3}$};
                \varrmore{R4}{$U_\mathrm{R4}$};
                \iarrmore{R4}{$I_\mathrm{4}$};
                \varrmore{R5}{$U_\mathrm{R5}$};
                \iarrmore{R5}{$I_\mathrm{5}$};
                \varrmore{R6}{$U_\mathrm{R6}$};
                \iarrmore{R6}{$I_\mathrm{6}$};
                \varrmore{R7}{$U_\mathrm{R7}$};
                \iarrmore{R7}{$I_\mathrm{7}$};
                \varrmore{R8}{$U_\mathrm{R8}$};
                \iarrmore{R8}{$I_\mathrm{8}$};
                \varrmore{R9}{$U_\mathrm{R9}$};
                \iarrmore{R9}{$I_\mathrm{9}$};
                \draw[<-,shift={(2,2)},blue] (150:0.8) arc (150:-150:0.8) node at(0,0){$M_1$};
                \draw[<-,shift={(6,2)},blue] (150:0.8) arc (150:-150:0.8) node at(0,0){$M_2$};
                \draw[<-,shift={(10,2)},blue] (150:0.8) arc (150:-150:0.8) node at(0,0){$M_3$};
                \draw[<-,shift={(14,2)},blue] (150:0.8) arc (150:-150:0.8) node at(0,0){$M_4$};
                \draw (4,0) node[below] {$K_0$};
                \draw (4,4) node[above] {$K_1$};
                \draw (8,4) node[above] {$K_2$};
                \draw (12,4) node[above] {$K_3$};
                \draw (8,0) node[below] {$K_5$};
                \draw (12,0) node[below] {$K_4$};
            \end{circuitikz}
            }\centering                                                           
        \end{figure}

		\item[\bf b)]
		Knotengleichungen:
        \begin{eqa}
            K_0 : -I_1+I_3-I_5 &=0 \nonumber\\
            K_1 : +I_1-I_3+I_4 &=0 \nonumber\\
            K_2 : -I_4+I_6+I_9 &=0 \nonumber\\
            K_3 : +I_2-I_8-I_9 &=0 \nonumber\\
            K_4 : -I_2-I_7+I_8 &=0 \nonumber\\
            K_5 : +I_5-I_6-I_7 &=0 \nonumber
        \end{eqa}

        \item[\bf c)]
        Maschengleichungen:
        \begin{eqa}
            M_1 : U_1-U_\mathrm{Ri}-U_\mathrm{R1}-U_\mathrm{R2}-U_\mathrm{R3} &=0 \nonumber\\
            M_2 : U_\mathrm{R3}+U_\mathrm{R4}+U_\mathrm{R5}+U_\mathrm{R6} &=0 \nonumber\\
            M_3 : -U_\mathrm{R6}+U_\mathrm{R7}-U_2+U_\mathrm{R9} &=0 \nonumber\\
            M_4 : U_2-U_\mathrm{R8} &=0 \nonumber
        \end{eqa}
	\end{itemize}
}