\subsection{Materialhabhängigkeit des Widerstandes R\label{Materialhabhängigkeit des Widerstandes R}}
\Loesung{
\[
R = \rho \cdot \frac{l}{A} = 1{,}68 \cdot 10^{-8} \cdot \frac{5}{1 \cdot 10^{-6}} = \SI{0.084}{\ohm}
\]
}

\subsection{Elektrischer Widerstand\label{Elektrischer}}

\Loesung{

Da das System radial-symmetrisch ist, fließt der Strom radial von innen nach außen (oder umgekehrt), und das elektrische Feld hängt nur vom Radius \( r \) ab.

Die Stromdichte ergibt sich über das Ohm’sche Gesetz in Differentialform:
\[
\vec{j}(r) = \kappa \cdot \vec{E}(r)
\]

Die Fläche, durch die der Strom bei Radius \( r \) fließt, ist die Oberfläche einer Kugel:
\[
A(r) = 4 \pi r^2
\]

Der Gesamtstrom ist konstant durch alle Kugelflächen:
\[
I = j(r) \cdot A(r) = \kappa \cdot E(r) \cdot 4 \pi r^2 \quad \Rightarrow \quad E(r) = \frac{I}{4 \pi \kappa r^2}
\]

\textbf{Spannung berechnen:}

Die Spannung ergibt sich durch das Integral über das elektrische Feld:
\[
U = \int_a^b E(r) \, dr = \int_a^b \frac{I}{4 \pi \kappa r^2} \, dr = \frac{I}{4 \pi \kappa} \int_a^b \frac{1}{r^2} \, dr = \frac{I}{4 \pi \kappa} \left[-\frac{1}{r}\right]_a^b
\]
\[
U = \frac{I}{4 \pi \kappa} \left(\frac{1}{a} - \frac{1}{b}\right)
\]

\textbf{Widerstand berechnen:}

Da \( U = R \cdot I \) gilt, folgt
\[
R = \frac{U}{I} = \frac{1}{4 \pi \kappa} \left(\frac{1}{a} - \frac{1}{b}\right)
\]

Der elektrische Widerstand zwischen den beiden Schalen beträgt somit:
\[
\boxed{ R = \frac{1}{4 \pi \kappa} \left(\frac{1}{a} - \frac{1}{b}\right) }
\]

}

\subsection{Elektrischer Widerstand eines beschichteten Drahts\label{Elektrischer Widerstand eines beschichteten Drahts}}
\Loesung{

\begin{align*}
&\text{Radius Aluminiumkern:} \quad a = \SI{0.5}{\milli\meter} \\
&\text{Dicke der Goldschicht:} \quad b = \SI{0.1}{\milli\meter} \\
&\Rightarrow \text{Gesamtradius:} \quad r = a + b = \SI{0.6}{\milli\meter} \\
&\text{Länge des Leiters:} \quad l = \SI{1.5}{\meter} = \SI{1500}{\milli\meter} \\
&\text{Aluminium:} \quad \kappa_\text{Alu} = \SI{35}{\meter/(\ohm \milli\meter^2)}, \quad \alpha_\text{Alu} = 3{,}5 \times 10^{-3}\, \si{\per\kelvin} \\
&\text{Gold:} \quad \kappa_\text{Gold} = \SI{44}{\meter/(\ohm \milli\meter^2)}, \quad \alpha_\text{Gold} = 3{,}6 \times 10^{-3}\, \si{\per\kelvin} \\
&\text{Temperaturänderung:} \\
&T_1 = \SI{25}{\celsius} = \SI{298}{\kelvin} \\
&T_2 = \SI{90}{\celsius} = \SI{363}{\kelvin} \\
&\Delta T = T_2 - T_1 = \SI{65}{\kelvin}
\end{align*}

Querschnittsflächen berechnen

\begin{align*}
&\text{Aluminiumkern:} \quad A_\text{Alu} = \pi a^2 = \pi \times (0.5)^2 = \pi \times 0.25 = 0.7854\, \si{\milli\meter^2} \\
&\text{Goldmantel (Ringfläche zwischen } r = 0.6 \text{ mm und } a = 0.5 \text{ mm):} \\
&A_\text{Gold} = \pi (r^2 - a^2) = \pi (0.6^2 - 0.5^2) = \pi (0.36 - 0.25) = \pi \times 0.11 \approx 0.3456\, \si{\milli\meter^2}
\end{align*}

Widerstände bei \(\SI{25}{\celsius}\)

Formel:
\[
R = \frac{l}{\kappa \cdot A}
\]

\begin{align*}
&R_\text{Alu} = \frac{1500}{35 \times 0.7854} = \frac{1500}{27.489} \approx \SI{54.58}{\milli\ohm} \\
&R_\text{Gold} = \frac{1500}{44 \times 0.3456} = \frac{1500}{15.206} \approx \SI{98.61}{\milli\ohm}
\end{align*}

Widerstände bei \(\SI{90}{\celsius}\)

Temperaturabhängiger Widerstand:
\[
R(T) = R_0 \cdot \left( 1 + \alpha \cdot \Delta T \right)
\]

\begin{align*}
&R_{\text{Alu},90^\circ C} = 54.58 \times (1 + 3.5 \times 10^{-3} \times 65) = 54.58 \times 1.2275 \approx \SI{66.97}{\milli\ohm} \\
&R_{\text{Gold},90^\circ C} = 98.61 \times (1 + 3.6 \times 10^{-3} \times 65) = 98.61 \times 1.234 \approx \SI{121.64}{\milli\ohm}
\end{align*}


\begin{tabular}{lcc}
\toprule
Material & Widerstand bei \(\SI{25}{\celsius}\) & Widerstand bei \(\SI{90}{\celsius}\) \\
\midrule
Aluminium & \(\SI{54.58}{\milli\ohm}\) & \(\SI{66.97}{\milli\ohm}\) \\
Gold & \(\SI{98.61}{\milli\ohm}\) & \(\SI{121.64}{\milli\ohm}\) \\
\bottomrule
\end{tabular}




}

\subsection{Reale Stromquellen\label{Reale Stromquellen}}
\Loesung{
\textbf{a) Klemmenspannung \( U_\text{k} \):}

Die Klemmenspannung berechnet sich mit dem Spannungsteiler:

\[
U_\text{k} = U_0 \cdot \frac{R_\text{l}}{R_\text{i} + R_\text{l}} = 9\,\si{\volt} \cdot \frac{4\,\si{\ohm}}{1\,\si{\ohm} + 4\,\si{\ohm}} = 9 \cdot \frac{4}{5} = \SI{7.2}{\volt}
\]

\textbf{b) Strom durch die Last \( I \):}

\[
I = \frac{U_0}{R_\text{i} + R_\text{l}} = \frac{9}{1 + 4} = \frac{9}{5} = \SI{1.8}{\ampere}
\]
	}


\subsection{Spannungsquelle\label{Spannungsquelle}}
\Loesung{
\[
R_\text{ges} = R_1 + R_2 = 2 + 4 = \SI{6}{\ohm}, \quad
I = \frac{U}{R_\text{ges}} = \frac{10}{6} = \SI{1.67}{\ampere}
\]

\[
U_{R_1} = I \cdot R_1 = 1.67 \cdot 2 = \SI{3.33}{\volt}, \quad
U_{R_2} = U - U_{R_1} = \SI{6.67}{\volt}
\]

\[
I_{R_2} = \frac{U_{R_2}}{R_2} = \frac{6.67}{4} = \SI{1.67}{\ampere}
\]
}

\subsection{Kapazität und Kondensator\label{Kapazität und Kondensator}}
\Loesung{
\begin{enumerate}
  \item[a)] \textbf{Ladung:}

  \[
  Q = C \cdot U = 10 \cdot 10^{-12} \cdot 5 = \SI{50}{\pico\coulomb}
  \]

  \item[b)] \textbf{Gespeicherte Energie:}

  \[
  E = \frac{1}{2} C U^2 = \frac{1}{2} \cdot 10 \cdot 10^{-12} \cdot 25 = \SI{125}{\pico\joule}
  \]

  \item[c)] \textbf{Gesamtkapazität der Reihenschaltung:}

  \[
  \frac{1}{C_\text{ges}} = \frac{1}{C_1} + \frac{1}{C_2} = \frac{1}{4} + \frac{1}{6} = \frac{5}{12}
  \Rightarrow C_\text{ges} = \SI{2.4}{\micro\farad}
  \]

  \item[d)] \textbf{Spannung über \( C_1 \):}

  \[
  Q = C_\text{ges} \cdot U_\text{ges} = 2.4 \cdot 10^{-6} \cdot 12 = \SI{28.8}{\micro\coulomb}
  \]
  \[
  U_1 = \frac{Q}{C_1} = \frac{28.8 \cdot 10^{-6}}{4 \cdot 10^{-6}} = \SI{7.2}{\volt}
  \]
\end{enumerate}
}

\subsection{Induktivität und Spüle\label{Induktivität und Spüle}}
\Loesung{

\begin{enumerate}
  \item[a)] \textbf{Energie bei \( I = \SI{3}{\ampere} \):}

  \[
  E = \frac{1}{2} L I^2 = \frac{1}{2} \cdot 2 \cdot 3^2 = \SI{9}{\joule}
  \]

  \item[b)] \textbf{Energie bei \( I = \SI{0.5}{\ampere} \):}

  \[
  E = \frac{1}{2} \cdot 2 \cdot 0.5^2 = \SI{0.25}{\joule}
  \]

  \item[c)] \textbf{Berechnung der Induktivität \( L \) der zylindrischen Spule:}

  Zunächst die Umrechnung der Einheiten:
  \[
  l = \SI{20}{\centi\meter} = \SI{0.2}{\meter}, \quad A = \SI{5}{\centi\meter\squared} = 5 \times 10^{-4} \, \si{\meter\squared}
  \]

  Die Induktivität einer luftgefüllten zylindrischen Spule berechnet sich durch:
  \[
  L = \mu_0 \frac{N^2 A}{l}
  \]
  mit der magnetischen Feldkonstanten
  \[
  \mu_0 = 4\pi \times 10^{-7} \, \si{\henry\per\meter}
  \]

  Einsetzen:
  \[
  L = 4\pi \times 10^{-7} \cdot \frac{500^2 \cdot 5 \times 10^{-4}}{0.2} = 7.85 \times 10^{-2} \, \si{\henry} = \SI{78.5}{\milli\henry}
  \]
\end{enumerate}

}


\subsection{Induktivität, Magnetische Flussdichte und Magnetfeldenergie\label{Induktivität, Magnetische Flussdichte und Magnetfeldenergie}}

\Loesung{

\textbf{a) Magnetische Flussdichte \( \vec{B} \):}

Für eine lange, luftgefüllte Zylinderspule gilt:
\[
B = \mu_0 \cdot \frac{N}{l} \cdot I
\]

Einsetzen der Werte:
\[
B = 4\pi \cdot 10^{-7} \cdot \frac{500}{0.5} \cdot 2.5 = 4\pi \cdot 10^{-7} \cdot 1000 \cdot 2.5
\]
\[
B = 4\pi \cdot 10^{-7} \cdot 2500 = \pi \cdot 10^{-3} \, \si{\tesla}
\]
\[
\Rightarrow B \approx 3.14 \cdot 10^{-3} \, \si{\tesla} = \SI{3.14}{\milli\tesla}
\]

\vspace{0.5em}
\textbf{b) Induktivität \( L \):}

Die Induktivität einer langen luftgefüllten Spule ergibt sich aus:
\[
L = \mu_0 \cdot \frac{N^2 \cdot A}{l}
\]

Einsetzen der Werte:
\[
L = 4\pi \cdot 10^{-7} \cdot \frac{500^2 \cdot 4 \cdot 10^{-4}}{0.5}
\]
\[
L = 4\pi \cdot 10^{-7} \cdot \frac{250000 \cdot 4 \cdot 10^{-4}}{0.5}
= 4\pi \cdot 10^{-7} \cdot \frac{100}{0.5}
= 4\pi \cdot 10^{-7} \cdot 200
\]
\[
L = 8\pi \cdot 10^{-5} \, \si{\henry}
\Rightarrow L \approx \SI{2.51e-4}{\henry} = \SI{251}{\micro\henry}
\]

\vspace{0.5em}
\textbf{c) Gespeicherte Energie im Magnetfeld \( E_m \):}

Die Energie im Magnetfeld einer Spule berechnet sich mit:
\[
E_m = \frac{1}{2} L I^2
\]

Einsetzen:
\[
E_m = \frac{1}{2} \cdot 2.51 \cdot 10^{-4} \cdot (2.5)^2
= 0.5 \cdot 2.51 \cdot 10^{-4} \cdot 6.25
\]
\[
E_m = 7.84 \cdot 10^{-4} \, \si{\joule}
= \SI{0.784}{\milli\joule}
\]

}





