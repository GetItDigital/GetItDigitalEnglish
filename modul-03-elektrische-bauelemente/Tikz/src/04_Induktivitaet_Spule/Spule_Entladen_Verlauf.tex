\begin{tikzpicture}
		\begin{axis} [
			xlabel={Zeit},
				ylabel={\textcolor{cyan}{$u(t)$}/ \textcolor{current}{$i(t)$}/ \textcolor{voltage}{$u_\mathrm{L}(t)$} },
				xmin=0, xmax=12.5,
				ymin=-1.2, ymax=1.2,
				ytick={-1,-0.63,-0.37,0, 0.37, 0.63, 1},
				yticklabels={-100\%,-63\%,-37\%,0,37\%, 63\%, 100\%},
				xtick={0, 1, 2, 3, 4, 5, 6, 7, 8, 9, 10, 11, 12},
				xticklabels={0, \(\tau\), \(2\tau\), \(3\tau\), \(4\tau\), \(5\tau\), \(6\tau\), \(7\tau\), \(8\tau\), \(9\tau\), \(10\tau\), \(11\tau\), \(12\tau\)},
				width=10cm,
				height=5cm,
				axis lines=middle,
				every axis x label/.style={at={(current axis.right of origin)},anchor=west},
				every axis y label/.style={at={(current axis.above origin)},anchor=south},
				font=\small,
				legend style={font=\scriptsize}
		]

		% Rechtecksignal für die Spannung
		\addplot [cyan, thick] coordinates {(0,1) (5,1) (5,0) (12,0)};

		% Stromkurve (Ladekurve)
		\addplot [current, domain=0:5, samples=100] {ifthenelse(x<0, 0, 1 - exp(-(x)))};
	    \addplot [voltage, domain=0:5, samples=100] {ifthenelse(x<0, 0,     exp(-(x)))};

		% Spannungskurve (Entladekurve)
		\addplot [current, domain=5:12, samples=100] {ifthenelse(x<5, 0, exp(-(x-5)))};
	    \addplot [voltage, domain=5:12, samples=100] {ifthenelse(x<5, 0,-exp(-(x-5)))};

		% Markierung der Zeitkonstanten

	        \legend{$u(t)$ , $i(t)$ , $u_\mathrm{L}(t)$}
	        \draw[dashed] (axis cs:1,0) -- (axis cs:1,0.63) node[right] {};
			\draw[dashed] (axis cs:6,0) -- (axis cs:6,0.37) node[right] {};
			\draw[dashed] (axis cs:1,0) -- (axis cs:1,0.37) node[right] {};
			\draw[dashed] (axis cs:6,-1) -- (axis cs:6,-0.37) node[right] {};
			\draw[dashed] (axis cs:0,0.37) -- (axis cs:1,0.37) node[right] {};
			\draw[dashed] (axis cs:0,0.63) -- (axis cs:1,0.63) node[right] {};
			\draw[dashed] (axis cs:0,-0.37) -- (axis cs:6,-0.37) node[right] {};
			\draw[dashed] (axis cs:0,0.37) -- (axis cs:6,0.37) node[right] {};
			\draw [dashed] (axis cs:0,-1) -- (axis cs:12,-1);
            \draw [dashed] (axis cs:5,1) -- (axis cs:12,1);

	% Punkte zur Hervorhebung
		\addplot[
			only marks,
			mark=*,
			mark options={scale=0.7, fill=black}
		] coordinates {
			(1, 0.63)
			(6, 0.37)
			(1, 0.37)
			(6, -0.37)
		};

		\end{axis}
		\end{tikzpicture}