\subsection{Identifikation eines Bauteils anhand des Schaltzeichens\label{Aufg11}}
% Alt: Aufgabe 11
\Aufgabe{
  Was für ein Bauelement wird mit folgendem Symbol gekennzeichnet?
  \begin {figure} [H]
     \centering
     \begin{tikzpicture}

   \draw
   node[npn, tr circle] (Q1) {}
   (Q1.base) 
   (Q1.collector)  
   (Q1.emitter)  ;
\end{tikzpicture}
     \label{fig:FigSchaltzeichen}
  \end {figure}
%   \speech{ **Aufgabe 11**  
% **Fragestellung:**  
% Was für ein **Bauelement** wird mit folgendem **Schaltzeichen** gekennzeichnet?

% ---

%  **Beschreibung der Abbildung**  
% Das gezeigte **elektrische Symbol** stellt ein **Halbleiterbauelement** dar, das aus drei Anschlüssen besteht:

% 1. **Kreisförmige Darstellung mit drei Anschlüssen:**  
%    - Symbolisiert ein **transistorbasiertes Halbleiterbauelement**.  
   
% 2. **Drei Anschlüsse:**  
%    - Der **obere Anschluss** entspricht dem **Kollektor (C)**.  
%    - Der **mittlere waagerechte Anschluss** entspricht der **Basis (B)**.  
%    - Der **untere Anschluss mit einem Pfeil** entspricht dem **Emitter (E)**.  
   
% }

}


\Loesung{
  Beim dargestellten Bauelement handelt es sich um einen npn-Transistor.
  Siehe: Abschnitt \ref{fig:UebersichtBipolartransistoren} 

  %     \speech{
  % Lösung zu Aufgabe 11:
  
  % Das dargestellte Bauelement in der Abbildung ist ein **npn-Transistor**.  
  % Ein npn-Transistor besteht aus drei Schichten: einer **n-dotierten** Schicht (Emitter), einer **p-dotierten** Schicht (Basis) und einer weiteren **n-dotierten** Schicht (Kollektor).  
  % Der Stromfluss erfolgt, wenn eine positive Spannung an der Basis anliegt und Elektronen vom Emitter zum Kollektor transportiert werden.  
  % Typischerweise wird ein npn-Transistor für Verstärkerschaltungen und als elektronischer Schalter verwendet.
  % }
  
}



