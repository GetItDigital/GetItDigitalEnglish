\subsection{Widerstandsberechnung für invertierenden Verstärker\label{Aufg3}}
% Alt: Aufgabe 3
\Aufgabe{
	Gegeben ist die abgebildete Schaltung mit einem invertierenden 
    Verstärker. Dabei ist der Widerstand $R_\mathrm{2} = 1\,\mathrm{k \Omega}$. Was für Widerstandswerte 
    muss $R_\mathrm{1}$ aufweisen, damit die Verstärkung -10, -50 und -100 beträgt?

    \begin {figure} [H]
   \centering
   \begin{tikzpicture}
   \ctikzset{tripoles/en amp/input height=0.45} % Setzt die Höhe des OPVs auf 0.45
\draw (0,0)node[en amp, en amp text A](E){}
(E.out)
(E.-)
(E.+);
    \draw (E.out) to[short, -*] (2,0) to[short, -o] (2.5,0);
    \draw (E.-) to[short, -*] (-2,0.5);
    \draw (-5,0.5) to[R,l=$R_1$, o-o] (-2,0.5);
    \draw (-2,0.5) to[short, o-] (-2,2) to[R,l=$R_2$] (2,2) -- (2,0);

    
    \draw [short, -o](-5,-2);
    \draw (E.+) -- (-2,-0.5) to[short, -*] (-2,-2)node[ground]{};
    \draw (2.5,-2) to[short, o-o] (-5,-2);
    
    \draw (2.5,0) to[open, v, name=ua] (2.5,-2);
    \draw (-5,0.5) to[open, v, name=ue3] (-5,-2);
    
    

    \varrmore{ue3}{$U_\mathrm{E}$};
    \varrmore{ua}{$U_\mathrm{A}$};
\end{tikzpicture}
   \label{fig:FigWiderstandIV}
    \end {figure}
    %\speech{
        %Aufgabe 3: Invertierender Verstärker  
        %
        %Gegeben ist eine Schaltung mit einem invertierenden Operationsverstärker.  
        %Der Widerstand R2 hat einen Wert von 1 Kiloohm.  
        %Gesucht sind die passenden Widerstandswerte für R1, sodass die Verstärkung jeweils -10, -50 und -100 beträgt.  
        %
        %Die Schaltung besteht aus einem Operationsverstärker mit einer Rückkopplung über die Widerstände R1 und R2.  
        %Die Eingangsspannung U E wird über R1 an den invertierenden Eingang des Operationsverstärkers angelegt.  
        %Der nichtinvertierende Eingang ist mit Masse verbunden.  
       % Der Ausgang U A ist über den Widerstand R2 mit dem invertierenden Eingang verbunden.  
      %  
     %   Die Verstärkung V eines invertierenden Verstärkers wird mit folgender Formel berechnet:  
    %    
   %     V = - (R2 / R1)  
  %      
 %       Um die gesuchten Widerstandswerte zu bestimmen, muss diese Gleichung nach R1 umgestellt werden.  
%        }
        
}

\Loesung{
	Gemäß Formel invertierender Verstärker:
    \begin{eqa}
        V = - \frac{R_2}{R_1}
    \end{eqa}
    Formel umstellen nach $R_\mathrm{1}$:
    \begin{eqa}
        R_1 = - \frac{R_2}{V}
    \end{eqa}
    \begin{itemize}
        \item $V = - 10$
        \begin{align*}
            R_1 =\mathrm{- \frac{1\,k\Omega}{- 10} = 100\,\Omega}
        \end{align*}
        \item $V = - 50$ 
        \begin{align*}
            R_1 =\mathrm{- \frac{1\,k\Omega}{- 50} = 20\,\Omega}
        \end{align*}
        \item $V = - 100$
        \begin{align*}
            R_1 =\mathrm{- \frac{1\,k\Omega}{- 100} = 10\,\Omega}
        \end{align*}
    \end{itemize}
   % \speech{
   %     Berechnung der Widerstände für einen invertierenden Verstärker  
   %     
   %     Die Verstärkung eines invertierenden Operationsverstärkers wird durch die Formel bestimmt:  
   %     
   %     V = - R2 durch R1  
   %     
   %     Umstellen der Formel nach R1 ergibt:  
   %     
   %     R1 = - R2 durch V  
   %     
   %     Die Aufgabe besteht darin, den Widerstand R1 für verschiedene Verstärkungswerte zu berechnen,  
   %     wenn der Widerstand R2 gleich 1 Kiloohm ist.  
   %     
   %     Für eine Verstärkung von -10:  
   %     R1 = - 1 Kiloohm durch -10 = 100 Ohm  
   %     
   %     Für eine Verstärkung von -50:  
   %     R1 = - 1 Kiloohm durch -50 = 20 Ohm  
   %     
   %     Für eine Verstärkung von -100:  
   %     R1 = - 1 Kiloohm durch -100 = 10 Ohm  
   %     
  %      Die berechneten Widerstandswerte zeigen, dass mit zunehmender Verstärkung  
 %       der Wert von R1 abnimmt.  
%        }
Siehe: Abschnitt \ref{fig: Verschaltung Verstaerker} und Tabelle \ref{tab:Grundschaltungen}
}