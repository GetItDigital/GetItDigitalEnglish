% Beispiel Einschwingvorgang (Kriechfall, aperiodisch) (Übergang)
% AxisEnv 1: DC
% AxisEnv 2: AC
\def\amplitude{0.5} % für AC
\begin{tikzpicture}[x=1cm,y=1cm]
	\draw[draw=none] (-0.25,-4.45) rectangle (11,2.55); % boundbox für lable außerhalb der Achsen
	\begin{axis}[
		name=AxisDC,
		%%%%%%% START SAME OPTIONS %%%%%%%
		xmin=-1, xmax=10,
		ymin=-0.125, ymax=1.25,
		xlabel={$t$},
		ylabel={$s(t)$},
		axis x line=center,
		axis y line=center,
		ylabel style={anchor=east, at={(ticklabel cs:0.95)}},
		ticks=none,
		mark=none,
		grid=none,
		width=12cm,
		height=4cm, % 4cm/1.375 y-scale
		%%%%%%% END OF SAME OPTIONS %%%%%%%
	]
		\addplot[color=black,dashed,	domain=-1:10] {1};          % Asymptote
		\addplot[color=red,very thick,	domain=-1:0] {0};           % t<0
		\addplot[color=red,very thick,	domain=0:5, samples=61] {1-e^(-x)};    % DC t>0
		\addplot[color=red,very thick,	domain=5:10,samples=61] {1-e^(-x)};    % DC t>0
		% nodes
		\node (DC) at (7.5,0.5){};
		\node (endwert) at (10,1){};
		\node (grenzeDCeins) at (0,1.125){};
		\node (grenzeDCzwei) at (5,1.125){};
		\node (randDClinks) at (-1,1){};
		\node (randDCrechts) at (10,1){};
	\end{axis}%
	\onslide<2->{%
	\begin{axis}[
		at={($(AxisDC.south west)+(0,-1cm)$)},
		anchor=north west,
		name=AxisAC,
		%%%%%%% START SAME OPTIONS %%%%%%%
		xmin=-1, xmax=10,
		ymin=-0.75, ymax=+0.75, % ymax=1.25 mit empty node, damit Achsenspitze näher bei Scheitelwert
		xlabel={$t$},
		ylabel={$s(t)$},
		axis x line=center,
		axis y line=center,
		ylabel style={anchor=east, at={(ticklabel cs:0.95)}},
		ticks=none,
		mark=none,
		grid=none,
		width=12cm,
		height=5cm, % 4cm/1.375 y-scale -> 6cm/2.0625 y-scale
		%%%%%%% END OF SAME OPTIONS %%%%%%%
		clip=false,
	]
		\addplot[color=black,dashed,	domain=-1:10,clip=true] {+\amplitude};    % Scheitelwert
		\addplot[color=black,dashed,	domain=-1:10,clip=true] {-\amplitude};    % Scheitelwert
		\onslide<3->{\addplot[color=black,dotted,	domain=0:10,samples=61] {-\amplitude*sin(-90)*e^(-x)};} % flüchtig
		\onslide<3->{\addplot[color=black,dotted,	domain=0:10,samples=61] {+\amplitude*sin(x*180-90)};} 	% eingeschwungen
		\addplot[color=red,very thick,	domain=-1:0] {0};               % t<0
		\addplot[color=red,very thick,	domain=0:5, samples=61] {-\amplitude*sin(-90)*e^(-x)+\amplitude*sin(x*180-90)}; 	% AC t>0
		\addplot[color=red,very thick,	domain=5:10,samples=61] {-\amplitude*sin(-90)*e^(-x)+\amplitude*sin(x*180-90)}; 	% AC t>0
		
		% nodes
		\node (AC) at (7.5,\amplitude*0.5){};
		\node (scheitelwert)	at (10,\amplitude){};
		\node (transient)		at (2.5,\amplitude*1.25){}; 
		\node (persistent)		at (7.5,\amplitude*1.25){}; 
		\node (grenzeACeins)	at (0,-\amplitude*1.2){};
		\node (grenzeACzwei)	at (5,-\amplitude*1.2){};
		\node (randAClinks)		at (-1,-\amplitude){};
		\node (randACrechts)	at (10,-\amplitude){};
	\end{axis}
	}% end of \onslide
	% Beschriftungen
	\node<1->[draw=black,fill=white,text=black] at (DC) {\textbf{DC}};
	\node<2->[draw=black,fill=white,text=black] at (AC) {\textbf{AC}};
	\node<1->[anchor=west] at (endwert) {$S$};
	\node<2->[anchor=west] at (scheitelwert) {$\hat{S}$};
	
	% Trennlinien (stationär | Übergang | stationär) und Beschriftungen für Bereich mittig und rechts
	\draw<3->[color=black, dashed, thick] (grenzeDCeins) -- (grenzeACeins); % Trennlinie: stationär | Übergang
	\draw<3->[color=black, dashed, thick] (grenzeDCzwei) -- (grenzeACzwei); % Trennlinie: Übergang | stationär
	\node<3->[align=center, anchor=south] at (transient) 	{\textbf{Übergang}$\vphantom{\big|}$\\transient$\vphantom{\big|}$};
	\node<3->[align=center, anchor=south] at (persistent) {\textbf{stationär}$\vphantom{\big|}$\\persistent$\vphantom{\big|}$};

	% mit randAClinks und ... ev. Bereiche semi-transparent überdecken ... 
\end{tikzpicture}