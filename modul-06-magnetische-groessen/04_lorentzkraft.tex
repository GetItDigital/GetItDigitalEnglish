\newvideofile{04.Lorentzkraft}{Lorentzkraft}
\section{Lorentzkraft\label{KapLorentzkraft}}
\v{
	%Videoframe 
	\begin{frame}{Lernziele}
		\begin{Lernziele}{Lorentzkraft}
			Du kannst ...
			\begin{itemize}
			\item ... das Phänomen der Lorentzkraft erklären.
			\item ... die Richtung der Lorentzkraft mit der Drei-Finger-Regel bestimmen.
			\end{itemize}
		\end{Lernziele}
	\speech{LernzieleLorentzkraft}{1}{In den letzten Videos habt ihr bereits die magnetischen Größen kennengelernt. Jetzt wollen wir uns genauer mit ihren Wirkungen beschäftigen. Eine der bedeutesten Wirkung ist die Lorentzkraft. Sie ist nicht nur ein physikalisches Phänomen, sondern die Grundlage fast aller elektromechanischen Energiewandler – insbesondere elektrischer Maschinen. Immer wenn in einem Motor elektrische Energie in mechanische Bewegung umgewandelt wird, oder in einem Generator Bewegung in elektrische Energie, spielt die Lorentzkraft eine zentrale Rolle. In diesem Video erfährt ihr, auf welchen physikalischen Grundlagen die Lorentzkraft beruht und wie sie sich auf verschiedene Objekte auswirkt. Euch wird die  Drei-Finger-Regel näher gebracht – Eine Regel, mit der ihr die Richtung der Lorentzkraft sicher bestimmen. Anschließend schauen wir uns die mathematische Herleitung zur Bestimmung der Lorentzkraft an und wenden das Wissen in einem Rechenbeispiel an.}
\end{frame}
	}
\begin{frame} \ftx{Lorentzkraft} \index{Lorentzkraft}
	\s{Befindet sich in einem Magnetfeld ein Leiter, der mit Strom durchflossen wird, wirkt auf ihn eine Kraft, die senkrecht zum Leiter und dem Magnetfeld steht. Diese Kraft wird Lorentzkraft genannt. Verläuft der Leiter senkrecht zu den magnetischen Feldlinien, ist die Kraft am größten.}

	\fo{width=0.45\textwidth}{height=0.8\textheight}{Lorentzkraft}{
		\put(9.2, 88){\color{current}$I$}
		\put(50, 90){\color{current}$I$}
		\put(15, 34){\color{gray}$F$}
		\put(78, 72){\color{magnetfeld}$N$}
		\put(78, 33){\color{magnetfeld}$S$}
	}{{\bf Versuchsaufbau zur Lorentzkraft.} Ablenkung eines freischwingenden stromdurchflossenen Leiters im Feld eines Permanentmagneten, wodurch sich der Leiter in die Richtung der Kraft bewegt. Die Richtung kann mit der Drei-Finger-Regel der rechten Hand bestimmt werden. Der Daumen orientiert sich dabei an der Richtung des Stromflusses, der Zeigefinger an der Richtung der magnetischen Feldlinien. Der Mittelfinger zeigt dann in die Richtung, in die die Kraft wirkt. \label{Lorentz1}}
	\speech{LorentzkraftEinfuehrung}{1}{Die Lorentzkraft beschreibt die Kraft, die auf einen stromdurchflossenen Leiter in einem Magnetfeld wirkt. Sie steht immer senkrecht sowohl zur Stromrichtung als auch zur Richtung des Magnetfeldes. Die Stärke der Lorentzkraft hängt im Wesentlichen von drei Größen ab: Erstens von der Stromstärke im Leiter. Je größer der Strom, desto stärker wirkt auch die Lorentzkraft. Zweitens von der Länge des Leiters im Magnetfeld. Je länger der Leiter in das Feld hineinragt, desto größer ist die gesamte Kraftwirkung. Und drittens von der Stärke des Magnetfeldes selbst. Eine höhere magnetische Flussdichte Be führt zu einer entsprechend stärkeren Lorentzkraft. Außerdem spielt der Winkel zwischen Leiter und Magnetfeld eine Rolle. Die Kraft ist am größten, wenn Stromrichtung und Magnetfeld senkrecht aufeinander stehen. Sind sie parallel, wirkt keine Lorentzkraft.}
\end{frame}

\begin{frame} \ftx{Lorentzkraft}
	\s{ % Abbildungen für Skript
		Schrittweise Erklärung zum in Abbildung \ref{Lorentz1} gezeigten Versuchsaufbau mit einem stromdurchflossenen Leiter im Feld eines Permanentmagneten:

		\begin{minipage}{0.25\textwidth}
			\foo{width=0.9\textwidth}{}{Lorentzkraft_Feldverlaeufe1}{
				\put(85, 78){\color{magnetfeld}N}
				\put(85, 15){\color{magnetfeld}S}
				\put(40, -8){\color{gray}+}
			}{\label{Lorentz2}}
		\end{minipage}
		\begin{minipage}{0.74\textwidth}
			\begin{itemize}
				\item Ein Leiter befindet sich zwischen den Polen eines Hufeisenmagneten.
				\item Der Leiter ist hier nicht stromdurchflossen, sodass das Feld des Permanentmagneten einzeln betrachtet werden kann.
				\item Die Magnetfeldlinien verlaufen außerhalb des Permanentmagneten vom Nord- zum Südpol und schließen sich innerhalb des Magneten.
			\end{itemize}
		\end{minipage}
		\begin{minipage}{0.25\textwidth}
			\foo{width=0.9\textwidth}{}{Lorentzkraft_Feldverlaeufe2}{
				\put(40, -8){\color{gray}$\Downarrow$}
			}{}
		\end{minipage}
		\begin{minipage}{0.74\textwidth}
			\begin{itemize}
				\item Nun wird der Leiter stromdurchflossen betrachtet, ohne das Feld des Permanentmagneten zu berücksichtigen.
				\item Um den stromdurchflossenen Leiter entsteht ein magnetisches Feld in Abhängikeit der Flussrichtung des Stroms (Rechte-Hand-Regel).
			\end{itemize}
		\end{minipage}
		\begin{minipage}{0.25\textwidth}
			\foo{width=0.9\textwidth}{}{Lorentzkraft_Feldverlaeufe3}{
				\put(85, 78){\color{magnetfeld}N}
				\put(85, 15){\color{magnetfeld}S}
				\put(5, 55){\color{gray}$F$}
			}{}
		\end{minipage}
		\begin{minipage}{0.74\textwidth}
			\begin{itemize}
				\item Das Feld des Permanentmagneten und das magnetische Feld des stromdurchflossenen Leiters werden nun zusammen betrachtet.
				\item Beide Felder überlagern sich, sodass auf der Rechten Seite eine Feldverstärkung und links eine Feldschwächung auftritt.
				\item Das System versucht diese Feldverzerrung auszugleichen, wodurch eine Kraft auf den Leiter wirkt, die senkrecht zum Leiter und dem Magnetfeld steht (Drei-Finger-Regel).
				\item Der Leiter bewegt sich in die Richtung, in die die Kraft wirkt.
			\end{itemize}
		\end{minipage}
		\begin{minipage}{0.25\textwidth}
			\foo{width=0.9\textwidth}{}{Lorentzkraft_Feldverlaeufe4}{
				\put(85, 78){\color{magnetfeld}N}
				\put(85, 15){\color{magnetfeld}S}
				\put(54.6, 55){\color{gray}$F$}
			}{}
		\end{minipage}
		\begin{minipage}{0.74\textwidth}
			\begin{itemize}
				\item Hier wird die Flussrichtung des stromdurchflossenen Leiters umgekehrt.
				\item Die Lorentzkraft wirkt entsprechend in die umgekehrte Richtung (Drei-Finger-Regel).
				\item Der Leiter bewegt sich in die Richtung, in die die Kraft wirkt.
			\end{itemize}
		\end{minipage}
	}
	\b{ % Abbildungen für Folien
		\begin{columns}
			\column[c]{0.25\textwidth}
			\foo{}{width=0.8\textwidth}{Lorentzkraft_Feldverlaeufe1}{
				\put(85, 78){\color{magnetfeld}N}
				\put(85, 15){\color{magnetfeld}S}
			}{}
			\pause
			\column[c]{0.25\textwidth}
			\foo{}{width=0.8\textwidth}{Lorentzkraft_Feldverlaeufe2}{
				\put(-22, 47){\color{gray}+}
			}{}
			\pause
			\column[c]{0.25\textwidth}
			\foo{}{width=0.8\textwidth}{Lorentzkraft_Feldverlaeufe3}{
				\put(85, 78){\color{magnetfeld}N}
				\put(85, 15){\color{magnetfeld}S}
				\put(5, 55){\color{gray}$F$}
				\put(-22, 47){\color{gray}$\Rightarrow$}
			}{}
			\pause
			\column[c]{0.25\textwidth}
			\foo{}{width=0.8\textwidth}{Lorentzkraft_Feldverlaeufe4}{
				\put(85, 78){\color{magnetfeld}N}
				\put(85, 15){\color{magnetfeld}S}
				\put(54.6, 55){\color{gray}$F$}
			}{}
		\end{columns}
	\speech{LorentzkraftPrinzip}{1}{Um das Phänomen besser zu verstehen, betrachten wir es in einem kleinen Versuch. Beginnen wir mit den magnetischen Kräften. Wir nehmen dazu einen Hufeisenmagneten. Er eignet sich besonders gut für unsere Veranschaulichung, denn zwischen seinen beiden Schenkeln bildet sich ein gleichmäßiges, also homogenes Magnetfeld. Die Feldlinien verlaufen dort alle parallel und geordnet vom Nordpol zum Südpol. Dieses Magnetfeld bezeichnen wir auch als B-Feld. Es bezeichnet die zuvor behandelte magnetische Flussdichte Be. Diese Größe werden wir später bei der Berechnung der Lorentzkraft benötigen.}
   	\silence{2}
	\speech{LorentzkraftPrinzip}{2}{Betrachten wir nun die zweite entscheidende Kraft, nämlich das elektromagnetische Feld, das durch einen elektrischen Strom entsteht. Im vorangegangenen Video über die Durchflutung, haben wir die Entstehung dieses Feldes näher besprochen. Falls ihr euch zurück erinnern könnt, bildet sich um ihn herum ein kreisförmiges Magnetfeld, wenn ein Leiter mit Strom durchflossen wird. Die Richtung dieses elektromagnetischen Feldes können wir mit der Rechten-Hand-Regel bestimmen: Zeigt der Daumen in die Richtung des Stroms, so geben die gekrümmten Finger die Richtung der Feldlinien an. In unserem Beispiel fließt der Strom in die Bildschirmebene rein, symbolisiert mit dem Kreuz im Leiter, sodass ein Magnetfeld in Uhrzeigersinn um den Leiter entsteht. Die Richtung ist eine wichtige Komponente des Feldes. Die zweite wichtige Komponente ist die Stromstärke Iii, die den Leiter durchfließt. Je größer der Strom I, desto stärker wird auch das entstehende elektromagnetische Magnetfeld und desto stärker die anschließend entstehende Lorentzkraft.}
	\silence{2}
	\speech{LorentzkraftPrinzip}{3}{Schauen wir uns nun das Zusammenwirken beider Felder, so wird die Lorentzkraft anhand der Bewegung des Leiters sichtbar. Im Schaubild sehen wir, wie sich das Magnetfeld des Leiters mit dem Magnetfeld des Permanentmagneten überlagert. Auf der linken Seite zeigen die Feldlinien in entgegengesetzte Richtungen und schwächen sich gegenseitig ab. Auf der rechten Seite verlaufen sie dagegen parallel und verstärken sich. Die Feldlinien werden dadurch gekrümmt, was zu einer Krafteinwirkung auf der Leiter führt. Der Leiter  wird durch die Einwirkung nach links verschoben. Vereinfacht lässt sich sagen, dass die Feldlinien möglichst gerade verlaufen möchten. Steht ein Leiter im Weg, wird er von den Feldlinien verdrängt. Diese einwirkende Kraft auf den Leiter bezeichnen wir als Lorentzkraft.}
	\silence{2}
	\speech{LorentzkraftPrinzip}{4}{Drehen wir nun die Stromrichtung um, fließt der Strom nicht mehr in die Bildebene hinein, sondern heraus. Das wird im Schaubild durch einen Punkt im Leiter symbolisiert. Logischer weise ändert sich auch die Richtung des Magnetfeldes nun entgegen des Uhrzeigersinns. Durch diese Umkehrung verschiebt sich auch die Überlagerung der Feldlinien auf die linke Seite des Leiters. Die Folge ist, dass der Leiter nach rechts herausgedrängt wird.}
	}
\end{frame}
	
\v{
\begin{frame} \ftx{Lorentzkraft} \index{Lorentzkraft}

	\fo{width=0.45\textwidth}{height=0.8\textheight}{Lorentzkraft}{
		\put(9.2, 88){\color{current}$I$}
		\put(50, 90){\color{current}$I$}
		\put(15, 34){\color{gray}$F$}
		\put(78, 72){\color{magnetfeld}$N$}
		\put(78, 33){\color{magnetfeld}$S$}
	}{{\bf Versuchsaufbau zur Lorentzkraft.} Ablenkung eines freischwingenden stromdurchflossenen Leiters im Feld eines Permanentmagneten, wodurch sich der Leiter in die Richtung der Kraft bewegt. Die Richtung kann mit der Drei-Finger-Regel der rechten Hand bestimmt werden. Der Daumen orientiert sich dabei an der Richtung des Stromflusses, der Zeigefinger an der Richtung der magnetischen Feldlinien. Der Mittelfinger zeigt dann in die Richtung, in die die Kraft wirkt. \label{Lorentz1}}
\speech{DreiFingerRegel}{1}{Wir hatten bereits eine rechte Handregel kennengelernt. Dabei zeigt der Daumen die Stromrichtung an, und die gekrümmten Finger beschreiben die Richtung des Magnetfeldes um den Leiter. Diese Vorstellung erweitern wir jetzt zur sogenannten Drei-Finger-Regel. Dazu strecken wir Daumen, Zeigefinger und Mittelfinger so, dass sie wie die drei Achsen im Raum jeweils im rechten Winkel zueinander stehen. Man kann sich die Hand also wie ein kleines räumliches Koordinatensystem vorstellen. Der Daumen symbolisiert die Richtung des elektrischen Stroms. Der Zeigefinger weist die Richtung des Magnetfeldes. Und der Mittelfinger zeigt schließlich die Richtung der Kraft an, die wir als Lorentzkraft bezeichnen.}

\end{frame}
}

\begin{frame} \ftx{Lorentzkraft} \index{Lorentzkraft}
	\begin{minipage}{0.6\textwidth}
	\s{Zur Berechnung der Lorentzkraft kann die vektorielle oder die skalare Darstellung genutzt werden. Bei der vektoriellen Berechnung (Gleichung \ref{GlLorentzkraftvek}) bewirkt eine bewegte Einheitsladung $q$ mit der Geschwindigkeit $\vec{v}$ im Magnetfeld $\vec{B}$ die Kraft $\vec{F}$:}	
	\b{Auf eine mit der Geschwindigkeit $\vec{v}$ bewegte Einheitsladung $q$ im Magnetfeld $\vec{B}$ wirkt eine Kraft:}
		\begin{eqa}
			\vec{F} = q \cdot (\vec{v} \cdot \vec{B}) \label{GlLorentzkraftvek}
		\end{eqa}
	\end{minipage}
	\hfill%
	\begin{minipage}{0.35\textwidth}
		\foo{width=\textwidth}{width=\textwidth}{Lorentzkraft_Berechnung}{
			\put(12, 31){\color{gray}$\vec{F}$}
			\put(78, 72){\color{magnetfeld}$N$}
			\put(78, 33){\color{magnetfeld}$S$}
			\put(18, 22){\color{magnetfeld}$\vec{B}$}
			\put(74, 18){\color{current}$\vec{v}$}
		}{} 
	\end{minipage}
	\speech{LorentzkraftGrundformel}{1}{Versuchen wir das gerade beschriebene Prinzip in eine mathematische Formel zu fassen. Zunächst betrachten wir eine einzelne elektrische Ladung q, die sich mit einer Geschwindigkeit v in einem Magnetfeld B bewegt. In Formeln ausgedrückt ist die Kraft F gleich der Ladung q, multipliziert mit dem Kreuzprodukt aus der Geschwindigkeit v und der magnetischen Flussdichte B. Das Kreuzprodukt kennt ihr sicher noch aus der Schule: Wenn man zwei Vektoren miteinander multipliziert, entsteht ein neuer Vektor. Dieser steht rechtwinklig auf den beiden Ausgangsgrößen. Das passt genau zur Drei-Finger-Regel: Strom, Magnetfeld und Kraft stehen senkrecht zueinander.}
\end{frame}

\begin{frame} \ftx{Lorentzkraft}
	\s{Bei der skalaren Berechnung entfallen die Vektoren, stattdessen wird der Winkel sin ${\alpha}$ im Bezug auf die Geschwindigkeit $v$ und das Magnetfeld $B$ in die Rechnung einbezogen. Dabei gilt, dass die Lorentzkraft sowohl zu der Flussdichte, als auch auf der Geschwindigkeit der Ladung senkrecht steht. Der Betrag der Kraft ergibt sich bei geraden Leitern und konstantem Magnetfeld zu:}
	\b{
		\begin{eqa}
			F &= q \cdot v \cdot B  \cdot \sin {\alpha}\\ 
			F &=I\cdot \ell\cdot N \cdot B\cdot \sin{\alpha}
		\end{eqa}
	}
	\s{
		\begin{eqa}
			F &= q \cdot v \cdot B  \cdot \sin {\alpha}\\ 
			F &=I\cdot \ell\cdot N \cdot B\cdot \sin{\alpha}\label{GlLorentzkraft}
		\end{eqa}

	}
	
	\s{Die Gleichung \ref{GlLorentzkraft} illustriert die skalare Berechung in einem stromdurchflossenen Leiter. $\ell$ ist die wirksame Leiterlänge mit der Anzahl von $N$ parallelen Leitern, die mit dem Strom $I$ durchflossen werden. 
	$\alpha$ ist der Winkel zwischen dem Leiter und der Richtung der magnetischen Flussdichte $\vec{B}$. Stehen Leiter und Magnetfeld senkrecht zueinander, entfällt $\sin{\alpha}$.}
	\end{frame}

\begin{frame} \ftx{Lorentzkraft}
	\v{
		\only<1>{
			\begin{eqa}
			F &= q \cdot v \cdot B  \cdot \sin {\alpha}
		\end{eqa}
		}
		\pause
		\only<2>{
		\begin{eqa}
			F &= q \cdot v \cdot B  \cdot \sin {\alpha}\\ 
			F &=I\cdot \ell\cdot N \cdot B\cdot \sin{\alpha} 
		\end{eqa}
		}
		\pause
		\only<3>{
		\begin{eqa}
			F &= q \cdot v \cdot B  \cdot \sin {\alpha}\\ 
			F &=I\cdot \ell\cdot N \cdot B\cdot \sin{\alpha} \\
			F &=I\cdot \ell\cdot N \cdot B
		\end{eqa}
		}
	}
	\speech{LorentzkraftFormeln}{1}{Wir können aber auch die Kraft mittels des Skalarprodukts als Zahl, rein nach ihrer Größe berechnen errechnen. Hierzu müssen wir in Erfahrung bringen in welchen Winkel , alfa, die Magnetische Flussdichte Be und der Geschwindigkeitsvektor V zu einander liegen. Ist der Winkel bekann,t ist die Kraft F gleich der elektrischen Ladung q, multipliziert mit der Geschwindigkeit v, multipliziert mit der magnetischen Flussdichte B und multipliziert mit dem Sinus des Winkels Alpha zwischen Bewegungsrichtung und Magnetfeld. Im Idealfall ist der Winkel senkrech. Dann haben wir einen Sinus von 90 Grad. In dem Fall wirkt die Kraft maximal. Würden die magnetische Flussdichte B und das die elektrische Ladung parallel zueinander bewegen, würde der Winkel gleich 0 sein, was einen Sinus von 0 grad ergibt. Und eine Multiplikation mit einem Faktor gleich 0 ergibt als Ergebnis immer 0. Es wirkt folglich keine Kraft.}
	\silence{2}
	\speech{LorentzkraftFormeln}{2}{Nun übertragen wir diese Formel vom einzelnen Elektron auf einen ganzen Leiter. Ein Strom besteht ja aus vielen bewegten Ladungen und diese können wir auch zusammenfassen. Aus der ursprünglichen Formel tauschen wir nun die elektrische Ladung und ihre Geschwindigkeit gegen Stromstärke I, die wirksame Leiterlänge L im Magnetfeld und der Anzahl der stromdurchflossenen Leiter innerhalb des Magnetfelds, gekennzeichnet mit den Formelzeichen N. So ergibt in einem stromdurchflossenen Leiter nun eine Lorentzkraft, die gleich der magnetischen Flussdichte B, multipliziert mit der Stromstärke I, multipliziert mit der wirksamen Leiterlänge L, multipliziert mit der Anzahl der Leiter N und schließlich multipliziert mit dem Sinus des Winkels Alpha zwischen Leiter und Magnetfeld.}
	\silence{2}
	\speech{LorentzkraftFormeln}{3}{In der realen Welt richten wir Strom und Magnetfeld in den elektrischen Maschinen fast immer senkrecht zueinander aus. Denn die Maschinen sind so gebaut, dass sie eine optimale Stromwirkung erzielen – es macht ja keinen Sinn, Leistung oder Energie zu verschenken. Deshalb werden Magnetfeld und Stromfluss in einem 90-Grad-Winkel angeordnet. In diesem Fall ist der Sinus von 90 Grad gleich eins, und die Formel für die Lorentzkraft vereinfacht sich entsprechend, gleich der magnetischen Flussdichte B, multipliziert mit der Stromstärke I, multipliziert mit der wirksamen Leiterlänge L und multipliziert mit der Anzahl der Leiter N.}
\end{frame}

	
\s{\begin{Merksatz}{Lorentzkraft}
	Die Lorentzkraft beschreibt die Kraft, die auf eine bewegte Ladung in einem Magnetfeld wirkt. Fließt die Ladung durch einen Leiter, so wirkt die Lorentzkraft senkrecht sowohl zum Stromfluss als auch zur magnetischen Flussdichte.
\end{Merksatz}}
\begin{frame} \ftx{Lorentzkraft}
	\s{ % Abbildungen für Skript
		Zur Veranschaulichung des Prinzips eines Elektromotors wird eine stromdurchflossene Leiterschleife, die beweglich auf einem drehbar gelagerten Rotor befestigt ist, im Feld eines Permanentmagneten betrachtet:

		\begin{minipage}{0.25\textwidth}
			\foo{width=0.9\textwidth}{}{Lorentzkraft_Feldverlaeufe5}{
				\put(85, 78){\color{magnetfeld}N}
				\put(85, 15){\color{magnetfeld}S}
				\put(40, -8){\color{gray}+}
			}{\label{Lorentz4}}
		\end{minipage}
		\begin{minipage}{0.74\textwidth}
			\begin{itemize}
				\item Eine Leiterschleife befindet sich auf einem drehbaren Element, dem Rotor, im Magnetfeld eines Hufeisenmagneten.
				\item Der Leiter ist hier noch nicht stromdurchflossen, sodass das Feld des Permanentmagneten einzeln betrachtet werden kann.
				\item Die Magnetfeldlinien verlaufen außerhalb des Permanentmagneten vom Nord- zum Südpol.
			\end{itemize}
		\end{minipage}
		\begin{minipage}{0.25\textwidth}
			\foo{width=0.9\textwidth}{}{Lorentzkraft_Feldverlaeufe6}{
				\put(5, 46.5){\color{magnetfeld}N}
				\put(55, 46.5){\color{magnetfeld}S}
				\put(40, -8){\color{gray}$\Downarrow$}
			}{}
		\end{minipage}
		\begin{minipage}{0.74\textwidth}
			\begin{itemize}
				\item Nun wird die Leiterschleife stromdurchflossen betrachtet, ohne das Feld des Permanentmagneten zu berücksichtigen.
				\item Um die stromdurchflossene Leiterschleife entsteht ein magnetisches Feld in Abhängikeit der Flussrichtung des Stroms (Rechte-Hand-Regel).
			\end{itemize}
		\end{minipage}
		\begin{minipage}{0.25\textwidth}
			\foo{width=0.9\textwidth}{}{Lorentzkraft_Feldverlaeufe7}{
				\put(85, 78){\color{magnetfeld}N}
				\put(85, 15){\color{magnetfeld}S}
				\put(8, 52){\color{gray}$F$}
				\put(54, 42){\color{gray}$F$}
			}{}
		\end{minipage}
		\begin{minipage}{0.74\textwidth}
			\begin{itemize}
				\item Das Feld des Permanentmagneten und das magnetische Feld der stromdurchflossenen Leiterschleife werden nun zusammen betrachtet.
				\item Die Lorentzkraft, die je Leiter senkrecht zum Leiter und dem Magnetfeld steht, wirkt auf die Leiterschleife. 
				\item Die Leiterschleife beginnt sich aufgrund der Lorentzkraft zu drehen.%
			\end{itemize}%
		\end{minipage}
		\begin{minipage}{0.25\textwidth}
			\foo{width=0.9\textwidth}{}{Lorentzkraft_Feldverlaeufe8}{
				\put(85, 78){\color{magnetfeld}N}
				\put(85, 15){\color{magnetfeld}S}
				\put(8, 40){\color{gray}$F$}
				\put(54, 56){\color{gray}$F$}
			}{}
		\end{minipage}
		\begin{minipage}{0.74\textwidth}
			\begin{itemize}
				\item Hier wird die Flussrichtung des stromdurchflossen Leiters umgekehrt.
				\item Entsprechend kehrt sich auch die Polung der stromdurchflossenen Leiterschleife und damit die Drehrichtung der Leiterschleife um (Drei-Finger-Regel).
			\end{itemize}
		\end{minipage}
	}
	\b{ % Abbildungen für Folien
		\begin{columns}
			\column[c]{0.25\textwidth}
			\foo{}{width=0.8\textwidth}{Lorentzkraft_Feldverlaeufe5}{
				\put(85, 78){\color{magnetfeld}N}
				\put(85, 15){\color{magnetfeld}S}
			}{}
			\pause
			\column[c]{0.25\textwidth}
			\foo{}{width=0.8\textwidth}{Lorentzkraft_Feldverlaeufe6}{
				\put(4, 46){\color{magnetfeld}N}
				\put(55, 46){\color{magnetfeld}S}
				\put(-22, 47){\color{gray}+}
			}{}
			\pause
			\column[c]{0.25\textwidth}
			\foo{}{width=0.8\textwidth}{Lorentzkraft_Feldverlaeufe7}{
				\put(85, 78){\color{magnetfeld}N}
				\put(85, 15){\color{magnetfeld}S}
				\put(6.5, 52){\color{gray}$F$}
				\put(53, 42){\color{gray}$F$}
				\put(-22, 47){\color{gray}$\Rightarrow$}
			}{}
			\pause
			\column[c]{0.25\textwidth}
			\foo{}{width=0.8\textwidth}{Lorentzkraft_Feldverlaeufe8}{
				\put(85, 78){\color{magnetfeld}N}
				\put(85, 15){\color{magnetfeld}S}
				\put(6.5, 40){\color{gray}$F$}
				\put(53, 56){\color{gray}$F$}
			}{}
		\end{columns}
	}
	\speech{LorentzkraftSpule}{1}{Das folgende Beispiel veranschaulicht sehr gut die Nutzung der Lorentzkraft in der Realität. In der Abbildung sehen wir wieder einen Permanentmagneten, dessen magnetischer Fluss homogen vom Nordpol zum Südpol verläuft. In dieses Magnetfeld setzen wir nun eine Leiterschleife ein, die an eine Stromquelle angeschlossen ist. Die Schleife ist fest an einer Drehachse verankert, sodass sie sich nicht seitlich verschieben kann, sondern nur um diese Achse drehen kann.}
	\silence{2}
	\speech{LorentzkraftSpule}{2}{Betrachten wir nun die das Magnetfeld der Schleife für sich. Sobald wir jedoch Strom durch die Spule schicken, fließt er im oberen Leiter in die Bildfläche rein und im unterm aus der Bildebene heraus. Jeder dieser Leiter bildet ein eigenes Magnetfeld aus. Diese orientieren sich nach der rechten Handregel.}
	\silence{2}
	\speech{LorentzkraftSpule}{3}{Im folgenden Schaubild überlagern sich nun das Magnetfeld des Permanentmagneten und die magnetfelder der Schleife. Wie wir zuvor gesehen haben, suchen die Feldlinien den kürzesten Weg. Jedoch werden sie in der Nähe der Leiter aufgrund des elektromagnetischen Feldes stark verbogen. Deshalb üben sie eine Kraft auf die Leiter aus – der eine wird nach links gedrückt, der andere nach rechts. Da die Leiterschleife jedoch fest an einer Drehachse verankert ist, können die Leiter nicht seitlich ausweichen. Stattdessen dreht sich die gesamte Schleife. Diese Drehung hält so lange an, bis die beiden Leiter waagerecht stehen. In dieser Stellung gleichen sich die Kräfte aus, und die Drehbewegung würde in dieser Position zum Stillstand kommen.}
	\silence{2}
	\speech{LorentzkraftSpule}{4}{Allein mit einer Spule lässt sich also noch kein Motor betreiben. Doch genau hier setzt das Prinzip des Gleichstrommotors an: Sobald die Spule die waagerechte Position erreicht, wird die Stromrichtung umgekehrt. Damit kehrt sich auch die Richtung der Lorentzkraft um. Würde die Spule exakt stillstehen, hätte das keinen Effekt. Doch durch ihre Massenträgheit dreht sie sich etwas weiter als 90 Grad. Ab diesem Moment greift die neue Kraft erneut an und beschleunigt die Drehung. So entsteht eine fortlaufende Rotation. Genau auf diesem Prinzip funktioniert die Gleichstrommaschine: Mit jeder halben Umdrehung wird die Stromrichtung gewechselt, wodurch sich die Spule immer weiter dreht.}
\end{frame}

\begin{frame} \ftx{Beispiel: Lorentzkraft}
	\begin{bsp}{Lorentzkraft}{}
		%\todo{Quelle: \cite{Tkotz}}
		Ein Gleichstrommotor hat im Luftspalt eine magnetische Flussdichte von $B=0,8\,\mathrm{T}$. Unter den Polen befinden sich insgesamt $N=400$ Wicklungen, die mit einem Strom von $I=10\,\mathrm{A}$ durchflossen werden. Die wirksame Leiterlänge ist $\ell=150\,\mathrm{mm}$.

		Berechnen Sie die Kraft $F$ am Umfang des Ankers.\pause
		\begin{align*}
			F & =B\cdot I\cdot \ell\cdot N                                                                                                               \\
			  & =0,8\,\tfrac{\mathrm{Vs}}{\mathrm{m}^2}\cdot 10\,\mathrm{A}\cdot0,15\,\mathrm{m}\cdot 400                                                        \\
			  & =480\,\frac{\mathrm{kg}\cdot\mathrm{m}^2\cdot\mathrm{s}\cdot\mathrm{A}\cdot\mathrm{m}}{\mathrm{s}^3\cdot\mathrm{A}\cdot\mathrm{m}^2}     = 480\,\mathrm{N}
		\end{align*}

	\end{bsp}
	\speech{LorentzkraftBeispiel}{1}{Versuchen wir das Ganze nun in einem Beispiel anzuwenden. Wir betrachten einen Gleichstrommotor. Im Luftspalt dieses Motors herrscht eine magnetische Flussdichte B von null Komma acht Tesla. Unter den Polen befinden sich insgesamt vierhundert Wicklungen, die mit einem Strom von zehn Ampere durchflossen werden. Die wirksame Leiterlänge beträgt einhundertfünfzig Millimeter, also null Komma eins fünf Meter. Setzen wir diese Werte in die Formel ein: Damit wir die Berechnung mit anderen SI-Einheiten verknüpfen können, schreiben wir Tesla als null Komma acht Voltsekunden pro Quadratmeter. Multiplizieren wir das mit zehn Ampere, mit null Komma eins fünf Meter Leiterlänge und mit vierhundert Wicklungen, so erhalten wir eine Kraft von vierhundertachtzig Newton. Falls ihr euch einmal unsicher bei den Einheiten seid, könnt ihr sie durch Kürzen herleiten. Die Quadratmeter kürzen sich heraus, ebenso ein Teil der Sekunden und die Ampere. Am Ende bleibt Kilogramm mal Meter durch Sekunde zum Quadrat – und das ist, wie ihr sicher wisst, genau die Einheit Newton.}

\end{frame}

\begin{frame} \ftx{Lorentzkraft}
	\s{Zwei parallele Leiter, die von Strom durchflossen werden, üben ebenfalls eine Lorentzkraft aufeinander aus, da jeder stromdurchflossene Leiter um sich herum ein Magnetfeld aufbaut (siehe Abbildung \ref{Lorentz5}). Ist die Stromrichtung in beiden Leitern gleich, entsteht eine Anziehungskraft, bei umgekehrter Stromrichtung wirkt die Kraft abstoßend.}
	\b{
		\begin{minipage}{0.48\textwidth}
			Kraft zwischen zwei geraden, parallelen, dünnen und unendlich langen Leitern:
			\begin{eq}
				F_{12} = \frac{\ell\cdot \mu_0\cdot I_1\cdot I_2}{2\cdot \pi\cdot r}
			\end{eq}
		\end{minipage}
		\hfill%
		\begin{minipage}{0.48\textwidth}
			\foo{}{width=\textwidth}{Lorentzkraft_parallel}{
			\put(13.5, 94){\color{current}$I$}
			\put(13.5, 32){\color{current}$I$}
			\put(5, 66){\color{gray}$F$}
			\put(31, 66){\color{gray}$F$}

			\put(57.5, 90.5){\color{current}$I$}
			\put(83.5, 94){\color{current}$I$}
			\put(51, 66){\color{gray}$F$}
			\put(89, 66){\color{gray}$F$}
		}{}
		\end{minipage}
	}
	\s{
		\fo{width=0.65\textwidth}{}{Lorentzkraft_parallel}{
			\put(14.5, 94){\color{current}$I$}
			\put(14.5, 32){\color{current}$I$}
			\put(5, 66){\color{gray}$F$}
			\put(32, 66){\color{gray}$F$}

			\put(58.5, 91){\color{current}$I$}
			\put(83.5, 94){\color{current}$I$}
			\put(52, 66){\color{gray}$F$}
			\put(90, 66){\color{gray}$F$}
		}{{\bf Lorentzkraft zwischen zwei parallelen Leitern.} Ist die Flussrichtung des Stroms gleich, ziehen sie sich an. Bei gegensätzlicher Flussrichtung stoßen sich die Leiter ab. \label{Lorentz5}}


		Für den theoretischen Spezialfall zweier geraden, parallelen, dünnen und unendlich langen Drähte gilt das Verhältnis:
		
		\begin{eq}
			F_{1,2} = \frac{\ell\cdot \mu_0\cdot I_1\cdot I_2}{2\cdot \pi\cdot r}
		\end{eq}

		Da in der Praxis diese speziellen Voraussetzungen nicht zutreffen, dient die Formel lediglich als  Näherung an reale Fälle, um die Krafteinwirkung der Lorentzkraft abschätzen zu können.
	}
\end{frame}