% Aufladevorgang RLC-Serienschwingkreis an Gleichspannung
% AxisEnv 1: Spannung am Kondensator
% AxisEnv 2: Strom am Kondensator
%
% Vergleich der Fälle: 1) aperiodischer Fall, 2) aperiodischer Grenzfall und 3) periodischer Fall
% Anfangsbedingungen: u_C(0) = 0, i_L(0) = 0
%
\pgfmathsetmacro\Uq{1} % U_q
% delta = R/(2L) 
\pgfmathsetmacro\deltaA{1} % aperiodischer Fall
\pgfmathsetmacro\deltaB{1} % aperiodischer Grenzfall
\pgfmathsetmacro\deltaC{1} % periodischer Fall
% omega_0 = 1/sqrt(LC)
\pgfmathsetmacro\omeganullA{0.75} % aperiodischer Fall
\pgfmathsetmacro\omeganullB{1} % aperiodischer Grenzfall
\pgfmathsetmacro\omeganullC{3} % periodischer Fall
%
% abgeleitete Werte
% lambda = -R/(2L) +/- sqrt( (R/(2L))^2 - 1/(LC) ) 
% 		= -delta +/- sqrt( delta^2 - omega_0^2 )  	mit delta = R/(2L) und omega_0 = 1/sqrt(LC)
% 		= -delta +/- kappa							mit kappa = sqrt( delta^2 - omega_0^2 )
%
% A: aperiodischer Fall (delta > omega_0)
\pgfmathsetmacro\kappaA{sqrt(\deltaA^2-\omeganullA^2)} 
\pgfmathsetmacro\KeinsA{\Uq*1/((-\deltaA+\kappaA)/(-\deltaA-\kappaA)-1)} 
\pgfmathsetmacro\KzweiA{\Uq*1/((-\deltaA-\kappaA)/(-\deltaA+\kappaA)-1)}
%
% B: aperiodischer Grenzfall (delta = omega_0)
\pgfmathsetmacro\kappaB{0}
\pgfmathsetmacro\KeinsB{-\Uq}
\pgfmathsetmacro\KzweiB{-\Uq*\deltaB}
%
% C: periodischer Fall (delta < omega_0)
\pgfmathsetmacro\omegad{sqrt(\omeganullC^2-\deltaC^2)}%
%\pgfmathsetmacro\phiC{atan(\omegad/\deltaC)}% Möglichkeit A: Falsch: mit  sin(x)/cos(x) = tan(x) = sin(180-x)/cos(180-x) !!! 2 Möglichkeiten wegen Periodizität/Symmetrie
\pgfmathsetmacro\phiC{atan(\omegad/\deltaC)}% Möglichkeit B: erfüllt du/dt(0)=0 ... : mit x = 180 - atan(sin(x)/cos(x)) OR x = atan(sin(x)/cos(x))
% test
\pgfmathsetmacro\Cnull{-\Uq/sin(\phiC)}
%
\begin{tikzpicture}[x=1cm,y=1cm]
	\begin{axis}[
		name=AxisSpannung,
		%%%%%%% START SAME OPTIONS %%%%%%%
		xmin=-0.01, xmax=7.5,
		ymin=-0.25, ymax=2, % 2.5 auf 5cm, vorher 1.475 auf 3.5cm 
		xlabel={$t$},
		ylabel={$u_C$},
		axis x line=center,
		axis y line=center,
		ylabel style={anchor=east, at={(ticklabel cs:0.95)}},
		ticks=none,
		mark=none,
		grid=none,
		width=7cm,
		height=5cm,
		%%%%%%% END OF SAME OPTIONS %%%%%%%
	]
		\node at (0,\Uq)[black,anchor=east,yshift=-1pt]{$U_q$};				% u_max label
		\addplot[color=black,dashed,	domain=-0.1:7.4] {\Uq};          	% u_max Grenze

		\node at (5,1.5)[draw=none,align=left,scale=0.75]{%
		$\ec[red]{u_C(t)}$: period. Fall\\%
		$\ec[violet]{u_C(t)}$: aperiod. Grenzf.\\%
		$\ec[blue]{u_C(t)}$: aperiod. Fall%
		}; % u_L label

		% Fall 1: aperiodischer Fall
		% u_C(t) = U_q + (K_1*e^(lambda_1*t) + K_2*e^(lambda_2*t))
		\addplot[thick, color=blue, domain=0:7.4, samples=61]{\Uq + \KeinsA*e^((-\deltaA+\kappaA)*x) + \KzweiA*e^((-\deltaA-\kappaA)*x)}; % uc = Uq + K1 * e^(lambda1*t) + K2 * e^(lambda2*t) 
		%\addplot[thick, dotted, color=orange, domain=0:5.4, samples=61]{\Uq + \KeinsA*e^((-\deltaA+\kappaA)*x)}; % Uq + K1 * e^(lambda1*t) 
		%\addplot[thick, dashed, color=orange, domain=0:5.4, samples=61]{\Uq + \KzweiA*e^((-\deltaA-\kappaA)*x)}; % Uq + K2 * e^(lambda2*t)

		% Fall 2: aperiodischer Grenzfall
		% u_C(t) = U_q + (K_1 + K_2*t) * e^(lambda*t)
		\addplot[thick, color=violet, domain=0:7.4,samples=61]{\Uq + (\KeinsB + \KzweiB*x) * e^(-\deltaB*x)}; % uc = Uq + (K1 + K2*t) * e^(lambda*t)

		% Fall 3: periodischer Fall
		% u_C(t) = U_q + C_0 * sin(omega*t+phi_0) * e^(-delta*t)
		\addplot[thick, color=red, domain=0:7.4,samples=256, smooth]{\Uq + \Cnull * sin(\omegad*(180/pi)*x+\phiC) * e^(-\deltaC*x)}; % uc = Uq + C0 * sin(omega*t+phi0) * e^(-delta*t) <-- ??? 
		%\addplot[thick, color=red, domain=0:7.4,samples=151]{\Uq + ( -\Uq*(\deltaC/\omegad)*sin(\omegad*x*360) - \Uq*cos(\omegad*x*360) ) * e^(-\deltaC*x)}; % uc = Uq + [C1*sin(omega*t) + C2*cos(omega*t)] * e^(-delta*t)

		% Güte Q
		\draw [->](0.8,0.04) .. controls (0.6, 0.25) and (0.2,0.275) .. (0.1,0.3)
		node[pos=0.3,above,scale=0.75]{$Q$}; % Q-Pfeil
	
	\end{axis}
\end{tikzpicture}
