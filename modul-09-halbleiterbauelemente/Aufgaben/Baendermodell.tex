\subsection{Beschreibung des Bändermodells\label{Aufg1} }
% Alt: Aufgabe 1
\Aufgabe{
  Beschreiben Sie ausführlich das Bändermodell von Festkörpern und seine Bedeutung für die Unterscheidung zwischen 
  Leiter, Halbleiter und Isolator. Gehen Sie dabei insbesondere auf die folgenden Aspekte ein:
  \begin{itemize}
    \item relevante Bänder und deren Besetzung mit Ladungsträgern.
    \item Bedeutung der Bandlücke und wie diese elektrische Eingeschaften von Materialien beeinflusst.
    \item Einfluss von Temperaturänderungen auf die Leitfähigkeit.
\end{itemize}
% \speech{Die Abbildung zeigt **Aufgabe 1**, die eine ausführliche Beschreibung des **Bändermodells von Festkörpern** erfordert. Dieses Modell ist entscheidend für das Verständnis der elektrischen Eigenschaften von Materialien und ihrer Unterscheidung in **Leiter, Halbleiter und Isolatoren**.  

% ### **Aufgabenstellung**
% Die Aufgabe fordert eine **detaillierte Erläuterung** des Bändermodells unter Berücksichtigung der folgenden Aspekte:

% 1. **Relevante Bänder und deren Besetzung mit Ladungsträgern**  
%    - In Festkörpern sind die **Elektronenenergiezustände** in **Bänder** aufgeteilt:  
%      - **Valenzband**: Enthält die Elektronen, die an chemischen Bindungen beteiligt sind.  
%      - **Leitungsband**: Hier können sich freie Elektronen bewegen und elektrische Leitung ermöglichen.  
%    - Die **Verfügbarkeit von Elektronen im Leitungsband** bestimmt die **Leitfähigkeit des Materials**.  
%    - Bei einem **Leiter (z. B. Metalle)** überlappen Valenz- und Leitungsband, wodurch **freie Ladungsträger ständig vorhanden sind**.  
%    - Bei einem **Halbleiter (z. B. Silizium)** gibt es eine **kleine Bandlücke**, die mit Temperatur oder Dotierung überbrückt werden kann.  
%    - Bei einem **Isolator (z. B. Glas)** ist die Bandlücke sehr groß, wodurch **fast keine Leitung möglich ist**.  

% 2. **Bedeutung der Bandlücke und ihr Einfluss auf elektrische Eigenschaften**  
%    - Die **Bandlücke (\( E_G \))** ist der Energiebereich zwischen Valenz- und Leitungsband.  
%    - Sie bestimmt, ob ein Material **ein Leiter, Halbleiter oder Isolator** ist:  
%      - **Leiter**: \( E_G = 0 \), kontinuierliche elektronische Zustände ermöglichen Leitung.  
%      - **Halbleiter**: \( 0 < E_G < 3eV \), Ladungsträger können durch thermische Anregung ins Leitungsband übergehen.  
%      - **Isolator**: \( E_G > 3eV \), sehr wenige Elektronen im Leitungsband, daher fast keine Leitfähigkeit.  
%    - Die Größe der Bandlücke beeinflusst auch **optische Eigenschaften**:  
%      - **Halbleiter mit kleiner Bandlücke** absorbieren und emittieren Licht (z. B. LEDs, Laser).  
%      - **Isolatoren mit großer Bandlücke** sind meist transparente Materialien (z. B. Quarzglas).  

% 3. **Einfluss von Temperaturänderungen auf die Leitfähigkeit**  
%    - **Mit steigender Temperatur steigt die Elektronenenergie**, wodurch mehr Elektronen die Bandlücke überwinden.  
%    - In **Halbleitern nimmt die Leitfähigkeit mit steigender Temperatur zu**, da mehr Elektronen ins Leitungsband übergehen.  
%    - In **Leitern nimmt die Leitfähigkeit mit steigender Temperatur ab**, da die Bewegung der Ladungsträger durch **Gitterstreuung behindert wird**.  
%    - **Isolatoren zeigen kaum Veränderung**, da ihre Bandlücke zu groß ist.  

% ### **Zusammenfassung der Aufgabe**
% - Das **Bändermodell** erklärt, warum manche Materialien **Strom leiten** und andere nicht.  
% - Die **Bandlücke** ist ein entscheidender Faktor für die **elektrischen und optischen Eigenschaften**.  
% - **Temperaturabhängigkeit** ist für Halbleitertechnologien besonders wichtig.  

% Diese Aufgabe fordert eine umfassende Erklärung der **physikalischen Grundlagen des Bändermodells** sowie dessen Auswirkungen auf **die Klassifikation und das Verhalten von Materialien**.}

}


\Loesung{
    Das Bändermodell erklärt die Energiezustände, die Elektronen in Festkörpern einnehmen können. 
    In einem einzelnen Atom sind die Elektronenzustände diskrete Energielevel.
    Siehe: Abschnitt \ref{sec:Bändermodell}

%     \speech{ 
% Das Bändermodell beschreibt die möglichen Energiezustände, die Elektronen in einem Festkörper einnehmen können. In einem einzelnen Atom sind die Energiezustände der Elektronen diskrete Energieniveaus, also genau bestimmte Werte.  

% In einem Festkörper, der aus vielen Atomen besteht, überlappen diese einzelnen Energieniveaus und bilden sogenannte Energie**bänder**. Die beiden wichtigsten Bänder sind:  

% 1. **Valenzband**: Das energetisch höchste besetzte Band bei niedrigen Temperaturen. Elektronen in diesem Band sind an ihre Atome gebunden und können sich nicht frei bewegen.  
% 2. **Leitungsband**: Ein energetisch höheres Band, in dem sich Elektronen frei bewegen können. Elektronen in diesem Band tragen zum elektrischen Stromfluss bei.  

% Zwischen diesen beiden Bändern befindet sich die **Bandlücke**, auch **Energielücke** genannt. Sie gibt an, wie viel Energie benötigt wird, um ein Elektron vom Valenzband in das Leitungsband zu heben. Die Größe der Bandlücke bestimmt, ob ein Material ein Leiter, Halbleiter oder Isolator ist:  

% - **Leiter** (z. B. Metalle): Keine oder sehr kleine Bandlücke, sodass Elektronen leicht ins Leitungsband übergehen können.  
% - **Halbleiter** (z. B. Silizium, Germanium): Mittlere Bandlücke, Elektronen können durch äußere Energie (z. B. Wärme, Licht) ins Leitungsband angehoben werden.  
% - **Isolatoren** (z. B. Glas, Keramik): Große Bandlücke, Elektronen können nur mit extrem hoher Energie ins Leitungsband gelangen.  

% Das Bändermodell ist essenziell für das Verständnis der elektrischen Eigenschaften von Materialien und die Funktionsweise von Halbleiterbauelementen wie Dioden und Transistoren. 
% }

  }




%{
    %\fta{Lösungen}
    %Lösungen für Beispielaufgaben 4.10
    %\textbf{Aufgabe 1}
    %\newline
    %Spannung über $R_\mathrm{V}$ ($U_\mathrm{V}$): \[ U_\mathrm{V} = U_\mathrm{0} - U_\mathrm{D} = 5\mathrm{V} - 2,2 \mathrm{V} = 2,8\mathrm{V}\]
    %\newline
    %Vorwiderstand:
    %\[
    %  R_\mathrm{V} = \frac{2,8\mathrm{V}}{30\mathrm{mA}} = 93,3\Omega 
    %\]

    %\textbf{Aufgabe 2}
    %\newline
   %a) 
   %\newline
   %\[ 
   %  \frac{5\mathrm{V}}{2\mathrm{k \Omega} + 0,3\mathrm{k \Omega}} = \frac{U_1}{300 \Omega}
   %\]
    % \newline
   %\[
   %  U_1 = 300 \Omega \cdot \frac{5 \mathrm{V}}{2\mathrm{k \Omega} + 0,3\mathrm{k \Omega}} = 0,65\mathrm{V}
   %\]
    % \newline
   %\[
   %  5 \mathrm{V} = 0,65 \mathrm{V} + U_\mathrm{RV} + 0,6 \mathrm{V}
   %\]
   % \newline
   %\[
   %  U_\mathrm{RV}= 5 \mathrm{V} - 0,65 \mathrm{V} - 0,6 \mathrm{V} = 3,75 \mathrm{V}
   %\]
   %\newline 
   %\[
   % R_\mathrm{V} = \frac{3,75 \mathrm{V}}{50 \mathrm{\mu A}} = 75 \mathrm{k \Omega}
   %\]  
   %\newline
   %b)
   %\newline
   %\[ 
   % I_\mathrm{C} = \frac{15 \mathrm{V} - 7,5\mathrm{V}}{500 \Omega} = 15 \mathrm{mA}
   %\]
   %\newline
   %\[
   % I_\mathrm{E} = I_\mathrm{C} + I_\mathrm{B} = 15 \mathrm{mA} + 50 \mathrm{\mu A} = 15,05 \mathrm{mA}
   %\]

%}