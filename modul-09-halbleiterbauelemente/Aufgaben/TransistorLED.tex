\subsection{Transistorschaltung mit LED\label{Aufg15}}
% Alt: Aufgabe 15
\Aufgabe{
  Mithilfe eines Bipolartransistor soll eine LED betrieben werden. Dabei muss durch die LED ein Strom von $I_\mathrm{D}=30\,\mathrm{mA}$ fließen, damit sie ihre optimale Leuchtkraft erreicht. \\
  Der verwendete Transistor besitzt einen Stromverstärkungsfaktor von $\beta = 50$ und im durchgeschalteten Zustand eine Kollektor-Emitter-Spannung von $U_\mathrm{CE} = 0,3\,\mathrm{V}$. 
  Zusätzlich sind die folgenden Parameter gegeben: $R_1=3,3\,\mathrm{k\Omega}$,  $R_2=270\,\mathrm{\Omega}$ und $U_\mathrm{L}=9\,\mathrm{V}$. \\
  Welche Spannung $U$ muss an der Basis anliegen, damit der Transistor durchschaltet? 
  \begin {figure} [H]
     \centering
     \begin{tikzpicture}
    % Spannungspfeile in blau
    \draw(8,3) to[open, v>, name=9V] (8,0);
    % Batterie links
    \draw
    (0,1.5) to[V, name=U] (0,0);
    % Widerstand RV
    \draw
    (0,1.5) -- (2,1.5)
    to[R=$3.3\,k\Omega$] (3.5,1.5) to[short,i,name=IB] (4.5,1.5)
    -- (5,1.5) ; % Verbindung mit Basis
    \draw(4.7,1.2) to[open, v>, name=ube] (5.2,0.7);
    % Transistor
    \draw
   (5.5,1.5) node[npn, tr circle] (Q1) {}
   (Q1.base) 
   (Q1.collector)  
   (Q1.emitter)  ;
   
   \draw
   (Q1.C) -- (5.5,3)
   to[R=$270\,\Omega$] (7,3);



   \draw (Q1.E) -- (5.5,0) to[short, -o] (8,0);


   \draw (0,0) -- (5.5,0); 

   \draw (8,3) to[leDo, o-] (7,3);

  \varrmore{U}{$U$};
   \varrmore{9V}{$\mathrm{9\,V}$};
   \iarrmore{IB}{$I_\mathrm{B}$};
   \varrmore{ube}{$U_\mathrm{BE}$};
  \end{tikzpicture}
     \label{fig:FigTransistorLED}
  \end {figure}
}


\Loesung{
Berechnung des Kollektorstroms bei bekannter Spannung und Lastwiderstand:
\begin{equation}
    I_\mathrm{C} = \frac{U}{R}
\end{equation}
\begin{align*}
    I_\mathrm{C} &= \frac{9\,\mathrm{V} - 0,3\,\mathrm{V}}{270\,\Omega} = 32,2\,\mathrm{mA}
\end{align*}

Berechnung des benötigten Basisstroms mit dem Stromverstärkungsfaktor $\beta = 50$:
\begin{equation}
    I_\mathrm{B} = \frac{I_\mathrm{C}}{\beta}
\end{equation}
\begin{align*}
    I_\mathrm{B} &= \frac{32,2\,\mathrm{mA}}{50} = 0,644\,\mathrm{mA}
\end{align*}

Spannungsabfall am Vorwiderstand:
\begin{align*}
    U_\mathrm{V} &= 0,644\,\mathrm{mA} \cdot 3,3\,\mathrm{k\Omega} = 2,12\,\mathrm{V}
\end{align*}

Gesamte Basisspannung (inkl. \( U_\mathrm{BE} \)):
\begin{align*}
    U &= U_\mathrm{V} + U_\mathrm{BE} = 2,12\,\mathrm{V} + 0,6\,\mathrm{V} = 2,72\,\mathrm{V}
\end{align*}

\textbf{Antwort:}  
Die Basisspannung muss \( \boxed{2{,}72\,\mathrm{V}} \) betragen, damit der Transistor durchschaltet.

    Siehe: Abschnitt \ref{sec:Bipolartransistor} und \ref{fig:StroemeUndSpannungenBeimBipolartransistor}
}