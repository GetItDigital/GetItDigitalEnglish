\subsection{Berechnung der Stromverstärkung einer Transistorschaltung\label{Aufg18}}
% Alt: Aufgabe 18
\Aufgabe{
    Gegeben ist die folgende Schaltung mit dem Kennlinienfeld des Transistors. Die Versorgungsspannung beträgt \( U_\mathrm{V} = \mathrm{6\,V} \). \\
    Der Arbeitspunkt liegt bei der halben Betriebsspannung.\\
    Bestimmen Sie die folgenden Werte aus dem Kennlinienfeld:
    
    \begin{itemize}
        \item Die Paare \( I_\mathrm{B} / U_\mathrm{BE} \) und \( I_\mathrm{C} / U_\mathrm{CE} \) im Arbeitspunkt,
        \item sowie die Stromverstärkung \( \beta \).
    \end{itemize}
    
    \begin{figure}[H]
        % oben
        \begin{minipage}[c]{\textwidth}
            \centering
            \begin{tikzpicture}
    \draw (0,0) node[npn, tr circle] (Q1) {};
    
    \draw (Q1.C) to[R, l_=$R_\mathrm{C}$] (0,3)
    to [short, -o] (0,3.5) node[right, color=voltage] {$U_\mathrm{V}$};
    
    \draw (0,3)
    to[short, *-] (-2, 3) 
    to[R=$R_2$, -*]  (-2, 0) 
    to[R=$R_1$]  (-2, -2) 
    node[ground]{};
    
    \draw (Q1.B)
    to[short, -o] (-4, 0)
    to[open, v, name=ue] (-4,-1.75)    
    to[short, o-] (-4, -2)
    node[ground]{};
    
    \draw (Q1.E)
    to[short, -] (0, -2) node[ground]{};
    
    \draw (Q1.C)++(0, 0.25) to[short, *-o] ++(2, 0)
    to[open, v^, name=ua] (2,-1.75)    
    to[short, o-] (2,-2)
    node[ground]{};
    
    \varrmore{ue}{$U_\mathrm{E}$};
    \varrmore{ua}{$U_\mathrm{A}$};
\end{tikzpicture}
        \end{minipage}
        \vspace{0.5cm} % Vertikaler Abstand zur zweiten Reihe
        
        % unten
        \begin{minipage}[c]{0.48\textwidth}
            \centering
            \includesvg[width=\textwidth]{Bilder/Aufgaben/FigTransistorStromverstaerkung2}
        \end{minipage}
        \hfill 
        \begin{minipage}[c]{0.48\textwidth}
            \centering
            \includesvg[width=\textwidth]{Bilder/Aufgaben/FigTransistorStromverstaerkung3}
        \end{minipage}
        \label{fig:FigTransistorStromverstaerkung}
    \end{figure}
    
}


\Loesung{
    \begin{figure}[H]
        \centering
        \begin{subfigure}[b]{0.45\textwidth}
            \centering
            \includesvg[width=0.9\textwidth]{Bilder/Aufgaben/LsgTransistorStromverstaerkung1}
            \label{fig:LsgTransistorStromverstaerkung1}
        \end{subfigure}
        \begin{subfigure}[b]{0.45\textwidth}
            \centering
            \includesvg[width=0.9\textwidth]{Bilder/Aufgaben/LsgTransistorStromverstaerkung2}
            \label{fig:LsgTransistorStromverstaerkung2}
        \end{subfigure}
    \end{figure}
    Gegeben sind folgende Werte im Arbeitspunkt:
    \begin{align*}
        U_\mathrm{CC} & = \mathrm{6\,V}      \\
        U_\mathrm{CE} & = \mathrm{3\,V}      \\
        I_\mathrm{C}  & = \mathrm{6\,mA}     \\
        U_\mathrm{BE} & = \mathrm{0{,}62\,V} \\
        I_\mathrm{B}  & = \mathrm{20\,\mu A}
    \end{align*}
    
    Berechnung der Stromverstärkung \( \beta \):
    \begin{align*}
        \beta & = \frac{I_\mathrm{C}}{I_\mathrm{B}} = \frac{\mathrm{6\,mA}}{\mathrm{20\,\mu A}} = \mathrm{300}
    \end{align*}
    
    Siehe: Abschnitt \ref{sec:ElektrischesVerhaltenBipolartransistor}
}