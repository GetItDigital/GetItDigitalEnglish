\immediate\write18{makeindex -g -s Templates/Arbeit.ist TeXAux/Skript.idx}
\immediate\write18{cat TeXAux/Skript.nlo | sort | uniq -w 55 | makeindex -i -s nomencl.ist -o TeXAux/Skript.nls}
\immediate\write18{cd TeXAux;gnuplot *.gnuplot;cd ..;}
\immediate\write18{bash modul-08-schaltungen-variabler-frequenz/Tikz/sh/tikz_externalize.sh}

\documentclass[twoside, a4paper]{article}
\usepackage{beamerarticle}
\pdfminorversion=6


% Modulspezifische Settings einbinden
% Projekt-Definitionen
\newcommand{\GetItDigitalModulnumber}{5}
\newcommand{\GetItDigitalModulname}{Erweiterte Gleichstromnetzwerke}
\newcommand{\GetItDigitalAutoren}{Torben Meibeck M.Eng.\\Dr.-Ing. Martin Kiel}
\newcommand{\GetItDigitalAutorenListe}{T. Meibeck, M. Kiel}
\newcommand{\GetItDigitalLink}{https://getitdigital.uni-wuppertal.de/module/modul-5-erweiterte-gleichstromnetzwerke}

\usetheme{GetItDigital}
\usepackage[utf8]{inputenc}         % Zeichenkodierung auf UTF-8
\usepackage[ngerman]{babel}         % Deutsche Texte, Sonderzeichen
\usepackage{color, colortbl}		% Farben, Tabellenfarben
\usepackage{multirow}			   	% Tabellen
\usepackage{booktabs}				% better tables (\toprule, \midrule, \bottomrule)
\usepackage{ifthen}                 % Konditionen
\usepackage{makeidx}                % automatische Index-Generierung
\usepackage[german]{nomencl}        % Nomenklatur (Symbolverzeichnis)
\usepackage{etoolbox}               % Gliederung von der Nomenklatur
\usepackage{icomma}				    % kein Leerzeichen hinter einem Komma, wenn es in einer Zahl steht
\usepackage{amsmath}                % Mathematische Symbole
\usepackage{amsfonts}               % Mathematische Symbole
\usepackage{amssymb}                % Mathematische Symbole
\usepackage{mathtools}				% Addon for amsmath (z.B. für \mathclap)
\usepackage{cancel}				    % Durchstreichen von Formelteilen
\usepackage{textcomp, gensymb}	    % Grad-Zeichen
\usepackage{xargs}					% Eigene Kommandos mit optionalen Argumenten definieren \newcommandx* für eigenes \highlight command basierend auf hf-tikz
\usepackage{booktabs}	   			% Für schönere horizontale Linien (FH Aachen)
\usepackage[intlimits]{esint}		% Schönere Integrale
\usepackage{siunitx}	   			% Für SI-Einheiten (FH Aachen)
\usepackage{colortbl}	   			% Farben (FH Aachen)
\usepackage{subcaption}             % Für Unterabbildungen (FH Aachen)
%\usepackage{caption}                % Caption (FH Aachen)
\usepackage[nice]{nicefrac}			% Math fracs in nice!
\usepackage{xfrac}					% newer than nicefrac \sfrac{}{} > \nicefrac{}{}
\usepackage{bm}						% Hervorhebung von Teilen einer Formel

% float-Positionierungen:
\usepackage{float}  			    % Gleitobjekte mit H genau an die angegebene Stelle positionieren.
\usepackage{placeins} 			    % neuer Befehl: \FloatBarrier  zu dem Zeitpunkt werden alle Floats gesetzt.

% Schaltpläne und LaTeX-Grafiken
\usepackage[nosiunitx,european,straightvoltages]{circuitikz}
\usepackage{tikz}
\usepackage{siunitx}
\usepackage{pgfplots}			    % Bode Diagramme
\usepackage{mathrsfs}				% Laplace Zeichen
%tabellen damit schaltungen ausgerrichtet sind
\usepackage{array}
\setlength{\marginparwidth}{2cm}	%Eingefügt von Jonas, um die To-Dos richtig darzustellen

%Lernziele, Merksätze und Beispiele
\usepackage{varwidth}
\usepackage[most,skins,breakable]{tcolorbox}
% Grid Overlay zur Positionierungshilfe - DEBUGGING ONLY
%\usepackage[texcoord,grid,gridcolor=red!60,subgridcolor=green!60,gridunit=mm]{eso-pic}
\usepackage{graphicx}				% for subfigures and resizeboxtikz
\usepackage{adjustbox}[export]		% crop images relative to image width/height
\usepackage{tcolorbox}				% Farbige Boxen für Definitionen, Beispiele etc.
\tcbuselibrary{skins}
\tcbuselibrary{breakable}			% für Seitenumbrüche in tcolorbox
%\usepackage{varwidth}

\usepackage{svg}					%Einbindung von svg-Grafiken

% Optionen des Grafikpakets TikZ
\usetikzlibrary{arrows}				% schönere Pfeilspitzen
\usetikzlibrary{datavisualization}
\usetikzlibrary{angles}				% Winkelberechnung und Darstellung
\usetikzlibrary{decorations.markings}
\usetikzlibrary{patterns}           % gefüllte Flächen
\usetikzlibrary{shapes.geometric}	% Geometrische Formen (z.B. Dreieck für Verstärker)
\usepgfplotslibrary{fillbetween}	% Fläche zwischen zwei Kurven
\tikzset{>=stealth'}				% schönere Pfeilspitzen 

% Bilder in getrenntem Pfad:
\graphicspath{{Bilder/}}


% eigene Befehle
\newcommand{\red}[1]{\textcolor[rgb]{1.00,0.00,0.00}{#1}}
\newcommand{\blue}[1]{\textcolor[rgb]{0.00,0.00,1.00}{#1}}
\newcommand{\Hz}{\operatorname{Hz}} % für gerades Hz
\newcommand{\dB}{\operatorname{dB}} % für gerades dB (Dezibel)
\newcommand{\dBm}{\operatorname{dBm}} % für gerades dBm {Dezibel auf 1 mW bezogen}
\renewcommand{\Re}{\operatorname{Re}} % für gerades Re für Realteil wie in physics package (statt Fraktur-R wie in plain TeX)
\renewcommand{\Im}{\operatorname{Im}} % für gerades Im für Imaginärteil wie in phycis package (statt Fraktur-I wie in plain TeX)
\newcommand{\dt}[1][{}]{\frac{\mathrm{d}^{#1}}{\mathrm{d}t^{#1}}} % \dt = d/dt ; \dt[2] = d^2/dt^2
% Note: \Re, \Im in Superscript/Subscript immer mit {}, Bsp. ^{\Im} oder _{\Re} statt ^\Im oder _\Re damit beide styles funktionieren
\newcommand{\alttexxt}[1]{} % Alternativtext für Bilder. Wird zur Zeit noch ignoeriert.

% Durchstreichen von Formelteilen
\definecolor{Red}{rgb}{1,0,0}
\newcommand{\colorcancel}[2]{\renewcommand{\CancelColor}{\color{#2}}\cancel{#1}}	% \colorcancel{text to cancel}{color}
\newcommand{\redcancel}[2][red]{\renewcommand{\CancelColor}{\color{#1}}\cancel{#2}} % \redcancel[optional color]{text to cancel} <=compatible=> \cancel{text to cancel}

% Multi Footnotes (gleiche Nummer, gleicher Text)
\newcommand{\footm}{\footnotemark} %https://tex.stackexchange.com/questions/86650/how-to-display-the-footnote-in-the-bottom-of-the-slide-while-using-columns
\newcommand*{\footmc}[1][0]{\addtocounter{footnote}{#1}\footnotemark[\value{footnote}]{}}% footmark with either same number (default) or by optional argument incremented number
\newcommand{\foott}[1]{\footnotetext{#1}}
% 	blabla\footm         ... undso\footmc\\ % 1...1
% 	blabla\footmc[-1]    ... undso\footmc\\ % 2...2
%	blabla\footmc[-1]    ... undso\footmc 	% 3...3
%	\foott{1st info} \foott{2nd info} \foott{3rd info}	 % footnotetexts
\newcommand\CMT[1]\null% Intra-Line Comments à la \CMT{this is a comment} https://www.reddit.com/r/LaTeX/comments/1cmq3v/intraline_comments_comment_without_commenting_out/

% Symbole für Stern- und Dreieckschaltung
\newcommand{\Stern}{{\ifmmode%
		\text{\tikz \draw[] (0,0) -- (90:.9ex) -- (0,0) -- (-45:1.1ex) -- (0,0) -- (225:1.1ex);}%
		\else%
		\tikz \draw[] (0,0) -- (90:.9ex) -- (0,0) -- (-45:1.1ex) -- (0,0) -- (225:1.1ex);%
		\fi}}
\newcommand{\Dreieck}{{\ifmmode%
		\text{\tikz \draw[] (0,0) -- (65:1.5ex) --++ (-65:1.5ex) --cycle;}%
		\else%
		\tikz \draw[] (0,0) -- (65:1.5ex) --++ (-65:1.5ex) --cycle;%
		\fi}}

% Farbige Spannungs und Strompfeile 
% Ref: Circuitikz Manual, Kapitel Advanced voltages, currents and flows https://texdoc.org/serve/circuitikz/0#subsection.5.8 (v. 1.6.7)
\ctikzset{bipole voltage style/.style={color=voltage}} % voltage label color
\ctikzset{bipole current style/.style={color=red}} 	% current label color
\ctikzset{!vi/.style={no v symbols, no i symbols}}	% add option for argument "!vi" to remove voltage and current arrows
\ctikzset{!v/.style={no v symbols}}	% add option for argument "!v" to remove voltage arrows
\ctikzset{!i/.style={no i symbols}}	% add option for argument "!i" to remove current arrows

\newcommand{\varronly}[1]{% {node}
	\draw [color=voltage] (#1-Vfrom) .. controls (#1-Vcont1) and (#1-Vcont2) .. (#1-Vto) node [currarrow, sloped, anchor=tip, allow upside down, pos=1]{} % arrow
}
\newcommand{\iarronly}[1]{% {node}
	%\draw [red, -] (#1-Ifrom) -- (#1-Ipos); % arrow shaft (optional)
	\node [color=red, currarrow, anchor=center,	rotate=\ctikzgetdirection{#1-Iarrow}] at (#1-Ipos) {} % arrow tip
}

% modified versions of above commands to draw arrows and labels together in one command with optional color argument
\newcommand{\varrmore}[3][voltage]{% [color]{node}{label}
    \draw[color=#1] (#2-Vfrom) .. controls (#2-Vcont1) and (#2-Vcont2) .. (#2-Vto) node [currarrow, sloped, anchor=tip, allow upside down, pos=1]{}; % arrow
    \node[color=#1, anchor=\ctikzgetanchor{#2}{Vlab}, inner sep=2pt] at (#2-Vlab) {#3} % label
	% note: label offset [...,inner sep=2pt] copied from internal macro definition, s. /usr/share/texlive/texmf-dist/tex/latex/circuitikz/circuitikz-1.2.7-body.tex line 23009)
}
\newcommand{\iarrmore}[3][red]{% [color]{node}{label}
	%\draw[#1,-] (#2-Ifrom) -- (#2-Ipos); % arrow shaft (optional)
	\node[color=#1, currarrow, anchor=center, rotate=\ctikzgetdirection{#2-Iarrow}] at (#2-Ipos) {}; % arrow tip
	\node[color=#1, anchor=\ctikzgetanchor{#2}{Ilab}] at (#2-Ipos) {#3} % label
}



% logarithmische Achsen für den Bode-Plot
\pgfplotsset{
	compat=1.18,
	log x ticks with fixed point/.style={
		xticklabel={
			\pgfkeys{/pgf/fpu=true}
			\pgfmathparse{exp(\tick)}%
			\pgfmathprintnumber[fixed relative, precision=3]{\pgfmathresult}
			\pgfkeys{/pgf/fpu=false}
		}
	},
	log y ticks with fixed point/.style={
		yticklabel={
			\pgfkeys{/pgf/fpu=true}{\tiny }
			\pgfmathparse{exp(\tick)}%
			\pgfmathprintnumber[fixed relative, precision=3]{\pgfmathresult}
			\pgfkeys{/pgf/fpu=false}
		}
	}
}


% Für minipage, weil \columns nur in Beamer funktioniert
\newlength\Colsep
\setlength\Colsep{10pt}

% Für hf-tikz, Offsets to adjust vertical bounds highlight boxes for typical equation sizes
%\usepackage{xparse} % needed for hf-tikz
\def\hfoffupper{0.3} 	% e.g. for 1
\def\hfoffuppera{0.45} 	% e.g. for \frac{1}{1}
\def\hfoffupperb{0.7}	% e.g. for \frac{\frac{\int^1_0 11}{\int^1_0 12}}{\frac{\int^1_0 21}{\int^1_0 22}}
\def\hfoffupperc{0.9}
\def\hfofflower{-0.12}	% analog zu upper offset
\def\hfofflowera{-0.21}
\def\hfofflowerb{-0.55}
\def\hfofflowerc{-0.7}

%Farbdefinitionen
\definecolor{GETgreen}{RGB}{25,185,145}				%Titelstreifen der Folien
\definecolor{leiter}{RGB}{207,163,118}				%elektrischer Leiter
\definecolor{magnetfeld}{RGB}{255,128,0}			%magnetisches Feld
\definecolor{flaeche}{RGB}{121,179,205}				%Querschnittsfläche
\definecolor{voltage}{RGB}{26, 91, 144}				%Spannungspfeile
\definecolor{spannung}{RGB}{26, 91, 144}			%farbe für spannung definieren
\definecolor{current}{RGB}{200, 0, 0}				%Strompfeile
\definecolor{strom}{RGB}{200, 0, 0}					%farbe für strom definieren
\definecolor{durchflutung}{RGB}{25, 210, 220}		%magnetiche Durchflutung Theta
\definecolor{Red}{rgb}{255, 0, 0}					%Reines rot z.B. zum durchstreichen von Elementen die sich wegkürzen
\definecolor{plotgreen}{RGB}{0,200,0}				% green
\definecolor{green1}{RGB}{0,200,0}					% green

%Farbkommandos für Gleichungen (equation color)
% usage: \ec[optional color]{mathsymbols} for any color or e.g. \ecg{mathsymbols} for green mathsymbols
% ref: https://tex.stackexchange.com/questions/21598/how-to-color-math-symbols
\newcommand{\ec}[2][black]{{\color{#1}#2}}	% set text/equation coloring, default black
\newcommand{\eci}[1]{{\color{red}#1}}		% current color
\newcommand{\ecv}[1]{{\color{voltage}#1}} 	% voltage color
\newcommand{\ecr}[1]{{\color{red}#1}} 		% red color
\newcommand{\ecg}[1]{{\color{green}#1}} 	% green color
\newcommand{\ecb}[1]{{\color{blue}#1}} 		% blue color
\newcommand{\ecp}[1]{{\color{purple}#1}} 	% purple color
\newcommand{\ecy}[1]{{\color{yellow}#1}} 	% yellow color
\newcommand{\ecc}[1]{{\color{cyan}#1}} 		% cyan color
\newcommand{\eco}[1]{{\color{orange}#1}} 	% orange color

%Lernziel
\newtcolorbox{Lernziele}[1]{enhanced,
	before skip=1cm,
	after skip=1cm,
	colframe=GETgreen,
	colbacktitle=GETgreen,
	colback=white!95!black,
	boxrule=0.5mm,
	title={Lernziele: #1},
	attach boxed title to top left={xshift=1.14cm,yshift*=-\tcboxedtitleheight/2}, varwidth boxed title*=-3cm,
	boxed title style={
			frame code={
				\path[left color=tcbcolback,right color=tcbcolback,
					middle color=tcbcolback]
					([xshift=1.6mm]frame.north west)[rounded corners=0.5mm]
					-- ([xshift=1.6mm]frame.north east)
					-- ([xshift=-1mm]frame.south east) 
					-- ([xshift=-1mm]frame.south west)
					-- cycle;
					\path[left color=tcbcolback,right color=tcbcolback,
					middle color=tcbcolback]
					([xshift=-3mm]frame.north west)[rounded corners=0.5mm]
					-- ([xshift=0.3mm]frame.north west)
					-- ([xshift=-2.2mm]frame.south west) 
					-- ([xshift=-5.5mm]frame.south west)
					-- cycle;
					\path[left color=tcbcolback,right color=tcbcolback,
					middle color=tcbcolback]
					([xshift=-7.6mm]frame.north west)[rounded corners=0.5mm]
					-- ([xshift=-4.3mm]frame.north west)
					-- ([xshift=-6.8mm]frame.south west) 
					-- ([xshift=-10.1mm]frame.south west)
					-- cycle;
				},interior engine=empty,
		},fonttitle=\bfseries
	}

%Merksatz
\newtcolorbox{Merksatz}[1]{enhanced,
	before skip=1cm,
	after skip=1cm,
	colframe=GETgreen,
	colbacktitle=GETgreen,
	colback=GETgreen!5!white,
	boxrule=0.5mm,
	title={Merke: #1},
	attach boxed title to top left={xshift=1.14cm,yshift*=-\tcboxedtitleheight/2}, varwidth boxed title*=-3cm,
	boxed title style={
			frame code={
				\path[left color=tcbcolback,right color=tcbcolback,
					middle color=tcbcolback]
					([xshift=1.6mm]frame.north west)[rounded corners=0.5mm]
					-- ([xshift=1.6mm]frame.north east)
					-- ([xshift=-1mm]frame.south east) 
					-- ([xshift=-1mm]frame.south west)
					-- cycle;
					\path[left color=tcbcolback,right color=tcbcolback,
					middle color=tcbcolback]
					([xshift=-3mm]frame.north west)[rounded corners=0.5mm]
					-- ([xshift=0.3mm]frame.north west)
					-- ([xshift=-2.2mm]frame.south west) 
					-- ([xshift=-5.5mm]frame.south west)
					-- cycle;
					\path[left color=tcbcolback,right color=tcbcolback,
					middle color=tcbcolback]
					([xshift=-7.6mm]frame.north west)[rounded corners=0.5mm]
					-- ([xshift=-4.3mm]frame.north west)
					-- ([xshift=-6.8mm]frame.south west) 
					-- ([xshift=-10.1mm]frame.south west)
					-- cycle;
				},interior engine=empty,
		},fonttitle=\bfseries
	}


% logarithmische Achsen für den Bode-Plot
\pgfplotsset{
	compat=1.18,
	log x ticks with fixed point/.style={
		xticklabel={
			\pgfkeys{/pgf/fpu=true}
			\pgfmathparse{exp(\tick)}%
			\pgfmathprintnumber[fixed relative, precision=3]{\pgfmathresult}
			\pgfkeys{/pgf/fpu=false}
		}
	},
	log y ticks with fixed point/.style={
		yticklabel={
			\pgfkeys{/pgf/fpu=true}{\tiny }
			\pgfmathparse{exp(\tick)}%
			\pgfmathprintnumber[fixed relative, precision=3]{\pgfmathresult}
			\pgfkeys{/pgf/fpu=false}
		}
	}
}


% Farbige Spannungs und Strompfeile 
% Ref: Circuitikz Manual, Kapitel Advanced voltages, currents and flows https://texdoc.org/serve/circuitikz/0#subsection.5.8 (v. 1.6.7)
\ctikzset{!vi/.style={no v symbols, no i symbols}}  		% add option for argument "!vi" to remove voltage and current arrows

%Tabellenbefehle
\usepackage{ragged2e}
\newcolumntype{L}[1]{>{\raggedright\arraybackslash}p{#1}}
\newcolumntype{C}[1]{>{\centering\arraybackslash}p{#1}}
\newcolumntype{R}[1]{>{\raggedleft\arraybackslash}p{#1}}
\newcolumntype{J}[1]{>{\justifying\arraybackslash}p{#1}}

	 		% Allgemeine Settings einbinden
% Pakete nur für das Skript
\usepackage{fancyhdr}      		% Unterstrichene Kopfzeilen: selbstdefinierbar
\usepackage{sanitize-umlaut}	   % Umlaute im Index
\usepackage[pdftex,bookmarks,bookmarksopen=false,colorlinks,bookmarksnumbered]{hyperref} % Links im PDF anklickbar
\hypersetup{% \ref Links im Skript umfärben
    ,urlcolor=blue
    ,citecolor=blue
    ,linkcolor=blue
    }
\usepackage{hf-tikz}		                        % Highlighting Text/Math, Ref: https://tex.stackexchange.com/questions/52598/beamer-highlighting-aligned-math-with-overlay
% example1: \tikzmarkin<2->[above offset={\hfoffuppera},below offset={\hfofflowera},blend mode=multiply]{a} R = \frac{U}{I} \tikzmarkend{a}
% example2: \tikzmarkin<2-3>[above left offset={0,\hfoffuppera}, below right offset={0,\hfofflowera}]{markerid} \int^{\pi}_{0}f(x)\d x \tikzmarkend{markerid}
% option: blend mode=multiply (to not overshadow stuff behind the highlight box) and (in beamer mode) <n>, <n->, <n-m> for overlay (ignored in article)

\usepackage{enumitem} % Aufzählungen nummerisch oder alphabetisch
\usepackage{todonotes}	%Anmerkungen und to-dos erstellen
	 	% Allgemeine Foliensettings einbinden
\usepackage[disabled]{../Video/beamervideo}  % Videoerstellung deaktiviert


\newcommand{\GetItDigitalModulnumber}{0}
\usepackage{import}				% relatives input

%PDF Eigenschaften
\hypersetup{
	pdftitle={GET it digital},
	pdfauthor={verschiedene Autoren}
}

\graphicspath{
	{Templates/Bilder/}
	{modul-01-elektrische-grundgroessen/Bilder/}
	{modul-02-energie-und-leistung/Bilder/}
	{modul-03-elektrische-bauelemente/Bilder/}
	{modul-04-grundlegende-gleichstromnetzwerke/Bilder/}
	{modul-05-erweiterte-gleichstromnetzwerke/Bilder/}
	{modul-06-magnetische-groessen/Bilder/}
	{modul-07-periodische-groessen/Bilder/}
	{modul-08-schaltungen-variabler-frequenz/Bilder/}
	{modul-09-halbleiterbauelemente/Bilder/}
	{modul-10-operationsverstaerker/Bilder/}
	{modul-11-elektrische-maschinen/Bilder/}
	{modul-12-schaltvorgaenge/Bilder/}
}

\institute{GET it digital}
\date{\today}


\begin{document}
\renewcommand{\partname}{Modul}
\thispagestyle{empty}
\begin{center}

	\begin{minipage}[c][2cm][c]{0.3\textwidth}
		\includegraphics[angle=270,width=\linewidth]{FHAachen-logo2010}
	\end{minipage}
	\hfill
	\begin{minipage}[c][2cm][c]{0.3\textwidth}
		\includegraphics[width=\linewidth]{Logo_Uni_Paderborn}
	\end{minipage}
	\hfill
	\begin{minipage}[c][2cm][c]{0.3\textwidth}
		\includegraphics[width=\linewidth]{BUW_Logo-weiss-auf-gruen-rgb}
	\end{minipage}
	\\
	\begin{minipage}[c][2cm][c]{0.3\textwidth}
		\includegraphics[width=\linewidth]{Fachhochschule_Südwestfalen_logo}
	\end{minipage}
	\hfill
	\begin{minipage}[c][2cm][c]{0.3\textwidth}
		\includegraphics[width=\linewidth]{logo-hrw}
	\end{minipage}
	\hfill
	\begin{minipage}[c][2cm][c]{0.3\textwidth}
		\begin{center}
			\includegraphics[width=0.7\linewidth]{FH_Dortmund-logo}
		\end{center}
	\end{minipage}

	\vfill
	%\vspace{23ex}
	{\large GET it digital} \\[10ex]
	{\large Grundlagen der Elektrotechnik}\\[10ex]
	\ \\
\end{center}
\vfill

\begin{minipage}[t]{0.36\textwidth}
	Ein Kooperationsvorhaben \\
	empfohlen durch die\\[0.3cm]
	\includegraphics[width=\linewidth]{dh_nrw.jpg}
\end{minipage}
\hfill
\begin{minipage}[t]{0.5\textwidth}
	gefördert durch\\[1cm]
	\includegraphics[width=\linewidth]{mkw.jpg}
\end{minipage}



\newpage
\pagestyle{fancyplain}

Die einzelnen Module wurden erstellt von:
\begin{itemize}
	\item Modul 1 und 4: Universität Paderborn, Henrik Bode
	\item Modul 2 und 3: Fachhochschule Aachen, Michael Hillgärtner, Sven Micun, Filimon Stergianos
	\item Modul 5: Fachhochschule Dortmund, Martin Kiel, Torben Meibeck
	\item Modul 6 und 11: Bergische Universität Wuppertal, Ralf Wegener, Grzegorz Lisicki, Josef Kirschner
	\item Modul 7: Fachhochschule Dortmund, Stefan Kempen, Torben Meibeck
	\item Modul 8 und 12: Fachhochschule Südwestfalen, Matthias Werle
	\item Modul 9 und 10: Hochschule Ruhr West, Marvin Kaminski, Kerstin Siebert, Jonas Brodmann, Dominik Thiem
\end{itemize}

\vfill
{\hfill Stand: \today}\\
{\includegraphics[width=0.2\linewidth]{by.png}}\\
Weiternutzung als OER ausdrücklich erlaubt: Dieses Werk und dessen Inhalte sind lizenziert unter CC BY 4.0. Ausgenommen von der Lizenz
sind die verwendeten Logos sowie alle anders gekennzeichneten Elemente. Nennung gemäß \href{https://open-educational-resources.de/oer-tullu-regel/}{TULLU-Regel} bitte wie folgt: \glqq GET it digital \grqq\ von verschiedenen Autoren \\ Lizenz: CC BY 4.0.

Der Lizenzvertrag ist hier abrufbar:\\
\url{https://creativecommons.org/licenses/by/4.0/deed.de}\\
Das Werk ist online verfügbar unter:\\
\url{https://getitdigital.uni-wuppertal.de/module/komplett/Skript.pdf}
\newpage

{\setlength{\parskip}{0.05ex}    %Verringerter Zeilenabstand für Verzeichnisse
	\tableofcontents
	\newpage
}

\printnomenclature
\cleardoublepage

\pagestyle{fancyplain}
\pagenumbering{arabic}
\fancyhead[OL]{\textsc{Kapitel \thesection : \sectiontitle}}
\fancyhead[ER]{GET it digital}
\fancyfoot[EL,OR]{}

\cleardoublepage
\part{Elektrische Grundgrößen}
\import{modul-01-elektrische-grundgroessen/}{Inhalt}

\cleardoublepage
\part{Energie und Leistung}
\import{modul-02-energie-und-leistung/}{Inhalt}

\cleardoublepage
\part{Elektrische Bauelemente}
\import{modul-03-elektrische-bauelemente/}{Inhalt}

\cleardoublepage
\part{Grundlegende Gleichstromnetzwerke}
\import{modul-04-grundlegende-gleichstromnetzwerke/}{Inhalt}

\cleardoublepage
\part{Erweiterte Gleichstromnetzwerke}
\import{modul-05-erweiterte-gleichstromnetzwerke/}{Inhalt}

\cleardoublepage
\part{Magnetische Größen}
\import{modul-06-magnetische-groessen/}{Inhalt}

\cleardoublepage
\part{Periodische Größen}
\import{modul-07-periodische-groessen/}{Inhalt}

\cleardoublepage
\part{Schaltungen variabler Frequenz}
\import{modul-08-schaltungen-variabler-frequenz/}{Inhalt}

\cleardoublepage
\part{Halbleiterbauelemente}
%\import{modul-09-halbleiterbauelemente/}{Inhalt}

\cleardoublepage
\part{Operationsverstärker}
%\import{modul-10-operationsverstaerker/}{Inhalt}

\cleardoublepage
\part{Elektrische Maschinen}
\import{modul-11-elektrische-maschinen/}{Inhalt}

\cleardoublepage
\part{Schaltvorgänge}
%\import{modul-12-schaltvorgaenge/}{Inhalt}


\cleardoublepage
\appendix
\setheadanhang

%\section{Übungsaufgaben}
%\renewcommand{\Aufgabe}[1]{#1}
%\renewcommand{\Loesung}[1]{}
%%hier jede Aufgabe als eigene Datei mit dem Input-Befehl einfügen
%\subsection{Elektrisches Homogenfeld 1\label{ElHomFeld}}
\Aufgabe{
    Mit einem Dielektrikum, in dem maximal eine Feldstärke $E = 30 \ \frac{kV}{cm}$ zulässig ist, soll 
	ein Kondensator für den Spannungsbereich von $100 \ kV$ realisiert werden.\\

       \underline{Aufgabenteil 1}:\\
		\phantom{text}\\

        Das Dielektrikum wird zunächst in einem (idealen) \underline{Plattenkondensator} eingesetzt.\\
        \begin{enumerate}[label=\alph*1)]
            \item Skizzieren Sie die Anordnung.
            \item Skizzieren Sie die Feldlinien in der Anordnung.
            \item Wo ist der Ort der maximalen Feldstärke?
            \item Skizzieren Sie den Verlauf des Betrags der elektrischen Feldstärke von einer Kondensatorplatte zur anderen $\left| E(x) \right|$, parallel zu den Kondensatorplatten
            $\left| E_{Diel}(z) \right|$ und in den Kondensatorplatten $\left| E_{Pl}(z) \right|$.
            \item Zeichnen Sie die 50\%-Äquipotentiallinie ein.
            \item Berechnen Sie den minimalen Plattenabstand $d_{Pl,min}$ für die o.a. Randbedingungen.            
        \end{enumerate} 

       \underline{Aufgabenteil} 2:\\
	   \phantom{text}\\

        Nun soll das \underline{Dielektrikum zwischen koaxialen Zylindern} eingesetzt werden.\\
        $\left(E_r(r)=\frac{U}{r\cdot\ln\left(\frac{a}{b}\right)};\ r_a: Außenradius; r_i: Radius \ des \ Innenzylinders\right)$
        \begin{enumerate}[label=\alph*2)]
            \item Skizzieren Sie die Anordnung.
            \item Skizzieren Sie die Feldlinien.
            \item Wo ist der Ort der maximalen Feldstärke?
            \item Skizzieren Sie den Verlauf der elektrischen Feldstärke $E_r(r)$ von $r=r_i$ bis $r=r_a$.
            \item Zeichnen Sie die 50\%-Äquipotentiallinie ein.
            \item Berechnen Sie den minimalen Außenradius $r_a$ für $r_i = 1;\ 2;\ 3;\ 4;\ 5;\ 10 \ cm$ und
            die zugehörige Dicke des Dielektrikums $d_i$.                        
        \end{enumerate}
}

\Loesung{
	Hier entsteht eine Musterlösung...
}
%\subsection{Elektrisches Homogenfeld 2\label{ElHomFeld2}}
\Aufgabe{
	Zwischen den Platten des nebenstehenden Kondensators befinden sich zwei Isolierstoffe mit den Dielektrizitätszahlen $\varepsilon_{\mathrm{r1}}$ und $\varepsilon_{\mathrm{r2}}$. Der Kondensator ist über den
    Schalter $S$ mit einer Batterie verbunden.\\
    \newline
    Gegeben sind die Größen: Plattenteilflächen $A_1 = A_2 = 100 \ cm^2; U_{\mathrm{B}}=100 \ V;d = 1 \ cm$.
    \begin{enumerate}[label=\alph*)]
            \item  Der Schalter ist geschlossen. Es gelte $\varepsilon_{\mathrm{r1}}=\varepsilon_{\mathrm{r2}}=1$. Wie groß sind die Kondensatorladung $Q$, die elektrische Flussdichte $D$, die elektrische Feldstärke $E$
            sowie die Kapazität $C_{\mathrm{AB}}$?
            \item Der Schalter ist geschlossen. Es gelte $\varepsilon_{\mathrm{r1}}=1, \varepsilon_{\mathrm{r2}}=3$. Wie hat sich die Gesamtkapazität allgemein im Vergleich zu Punkt a) verändert? Zu bestimmen
            sind außerdem die Werte $Q_1$ und $Q_2$, $D_1$ und $D_2$ sowie $E_1$ und $E_2$. Welche Spannung $U_{AB}$ stellt sich am Kondensator ein?
            \item  In die Anordnung gemäß Punkt a) wird \underline{nach} dem Öffnen des Schalters $S$ ein anderes Dielektrikum eingeführt, so dass gilt: $\varepsilon_{\mathrm{r1}}=1, \varepsilon_{\mathrm{r2}}=3$.
            Welche Spannung $U_{\mathrm{AB}}$ stellt sich am Kondensator ein?
    \end{enumerate}
}

\Loesung{
	Hier entsteht eine Musterlösung...
}
%....

%Das ist nur Beispielsinhalt, bitte auskommentieren!
%\subsection{Gleichstrommaschine 1\label{AufgGM1}}
\Aufgabe{
    Eine fremderregte Gleichstrommaschine hat die folgenden Angaben auf ihrem Typenschild:
    \begin{itemize}
        \item Nennleistung: $41,8\,\text{kW}$
        \item Nenndrehzahl: $1900\,\frac{1}{\text{min}}$
        \item Ankernennspannung: $440\,\text{V}$
        \item Ankernennstrom: $100\,\text{A}$
        \item Erregung: $240\,\text{V}$
        \item Erregerstrom: $10\,\text{A}$
        \item Leerlaufdrehzahl: $2000\,\frac{1}{\text{min}}$
    \end{itemize}

    Hinweis: Die Nennleistung auf einem Motortypenschild bezeichnet immer die abgegebene mechanische Leistung im Nennpunkt.

    \begin{itemize}
        \item[\bf a)]
        Wie groß ist das Drehmoment der Maschine im Nennpunkt?
        \item[\bf b)]
        Berechnen Sie den Wirkungsgrad der Maschine im Nennpunkt.	
        \item[\bf c)]
        Berechnen Sie das Produkt aus Erregerfluss und der Ankerkonstanten $K\cdot \Phi$
        \item[\bf d)]
        Wie groß ist der Ankerwiderstand $R_A$?
    \end{itemize}
}

\Loesung{
	\begin{itemize}
		
		\item[\bf a)]
		Aus der Nennleistung und der Nenndrehzahl ergibt sich direkt das Nenndrehmoment:
		\begin{align*}
		    P_\text{mech} &= M\cdot \omega\\
		    M &= \frac{P_\text{mech}}{\omega} = \frac{41,8\,\text{kW}}{2\pi\cdot 1900\,\frac{1}{\text{min}} \cdot \frac{1\,\text{min}}{60\,\text{s}}} = 210,08\,\text{Nm}
		\end{align*}
		\item[\bf b)]
		Im Motorbetrieb ist der Wirkungsgrad der Quotient aus mechanischer bezogen auf die elektrische Leistung (siehe Gleichung \ref{GlLeistung}). Hierbei muss sowohl die Ankerleistung, als auch die Erregerleistung berücksichtigt werden.
		\begin{eqa}
			\eta &= \frac{P_\text{mech}}{P_\text{elektr}} = \frac{41,8\,\text{kW}}{440\,\text{V}\cdot 100\,\text{A} + 240\,\text{V}\cdot 10\,\text{A}} = 90,08\,\%\nonumber
		\end{eqa}
		
		\item[\bf c)]
		Nach Gleichung \ref{GlInduzierteSpannungGM} wird die induzierte Spannung berechnet. Im Leerlauf muss die induzierte Spannung der Ankerspannung entsprechen. Gleiches geht aus Gleichung \ref{GlfremderregteGM1} hervor, da im Leerlauf auch das Drehmoment Null sein muss.
		\begin{eqa}
			U_q &= K \cdot \Phi \cdot \omega\tag{\ref{GlInduzierteSpannungGM}}\\
			K\cdot \Phi &= \frac{U_q}{\omega} = \frac{440\,\text{V}}{2\pi\cdot 2000\,\frac{1}{\text{min}} \cdot \frac{1\,\text{min}}{60\,\text{s}}} = 2,1\,\text{Vs}\nonumber
		\end{eqa}
		\item[\bf d)]
		Hier sind zwei Ansätze möglich. Es kann Gleichung \ref{GlfremderregteGM1} auf den Ankerwiderstand umgeformt werden:
		\begin{eqa}
			n &= \frac{U_A}{2\pi K\cdot \Phi} - \frac{R_A\cdot M_i}{2\pi(K\cdot \Phi)^2}\tag{\ref{GlfremderregteGM1}}\\
			R_A &= \left(\frac{U_A}{2\pi K\cdot \Phi} - n\right) \cdot \frac{2\pi(K\cdot \Phi)^2}{M_i}\nonumber\\
			&= \left(\frac{440\,\text{V}}{2\pi\cdot 2,1\,\text{Vs}} - \frac{1900}{60\,\text{s}}\right)\cdot \frac{2\pi\cdot (2,1\,\text{Vs})^2}{210,08\,\text{Nm}} = 220\,\text{m}\Omega\nonumber
		\end{eqa}
		Die andere Berechnungsvariante führt über das Ersatzschaltbild in Abbildung \ref{AbbGMFremderregt}. Im Nennbetrieb können wir die Quellenspannung $U_q$ berechnen. Diese muss kleiner als die Quellenspannung im Leerlauf sein.
		\begin{eqa}
			U_q &= K \cdot \Phi \cdot \omega \tag{\ref{GlInduzierteSpannungGM}}\\
			U_{q,n}&=  2,1\,\text{Vs} \cdot 2\pi\cdot 1900\,\frac{1}{\text{min}} \cdot \frac{1\,\text{min}}{60\,\text{s}} = 418\,\text{V}\nonumber
		\end{eqa}
		Der Spannungsabfall am Widerstand $R_A$ muss durch dem Maschenumlauf die Differenzspannung zwischen Ankerspannung und induzierter Spannung betragen. Da der Ankerstrom im Nennbetrieb bekannt ist, kann der Widerstand über das Ohmsche Gesetz berechnet werden.
		\begin{equation*}
		    R_A = \frac{U_A - U_q}{I_A} = \frac{440\,\text{V} - 418\,\text{V}}{100\,\text{A}} = 220\,\text{m}\Omega
		\end{equation*}
	\end{itemize}
}
\subsection{Elektrisches Homogenfeld 1\label{ElHomFeld}}
\Aufgabe{
    Mit einem Dielektrikum, in dem maximal eine Feldstärke $E = 30 \ \frac{kV}{cm}$ zulässig ist, soll 
	ein Kondensator für den Spannungsbereich von $100 \ kV$ realisiert werden.\\

       \underline{Aufgabenteil 1}:\\
		\phantom{text}\\

        Das Dielektrikum wird zunächst in einem (idealen) \underline{Plattenkondensator} eingesetzt.\\
        \begin{enumerate}[label=\alph*1)]
            \item Skizzieren Sie die Anordnung.
            \item Skizzieren Sie die Feldlinien in der Anordnung.
            \item Wo ist der Ort der maximalen Feldstärke?
            \item Skizzieren Sie den Verlauf des Betrags der elektrischen Feldstärke von einer Kondensatorplatte zur anderen $\left| E(x) \right|$, parallel zu den Kondensatorplatten
            $\left| E_{Diel}(z) \right|$ und in den Kondensatorplatten $\left| E_{Pl}(z) \right|$.
            \item Zeichnen Sie die 50\%-Äquipotentiallinie ein.
            \item Berechnen Sie den minimalen Plattenabstand $d_{Pl,min}$ für die o.a. Randbedingungen.            
        \end{enumerate} 

       \underline{Aufgabenteil} 2:\\
	   \phantom{text}\\

        Nun soll das \underline{Dielektrikum zwischen koaxialen Zylindern} eingesetzt werden.\\
        $\left(E_r(r)=\frac{U}{r\cdot\ln\left(\frac{a}{b}\right)};\ r_a: Außenradius; r_i: Radius \ des \ Innenzylinders\right)$
        \begin{enumerate}[label=\alph*2)]
            \item Skizzieren Sie die Anordnung.
            \item Skizzieren Sie die Feldlinien.
            \item Wo ist der Ort der maximalen Feldstärke?
            \item Skizzieren Sie den Verlauf der elektrischen Feldstärke $E_r(r)$ von $r=r_i$ bis $r=r_a$.
            \item Zeichnen Sie die 50\%-Äquipotentiallinie ein.
            \item Berechnen Sie den minimalen Außenradius $r_a$ für $r_i = 1;\ 2;\ 3;\ 4;\ 5;\ 10 \ cm$ und
            die zugehörige Dicke des Dielektrikums $d_i$.                        
        \end{enumerate}
}

\Loesung{
	Hier entsteht eine Musterlösung...
}
\subsection{Elektrisches Homogenfeld 2\label{ElHomFeld2}}
\Aufgabe{
	Zwischen den Platten des nebenstehenden Kondensators befinden sich zwei Isolierstoffe mit den Dielektrizitätszahlen $\varepsilon_{\mathrm{r1}}$ und $\varepsilon_{\mathrm{r2}}$. Der Kondensator ist über den
    Schalter $S$ mit einer Batterie verbunden.\\
    \newline
    Gegeben sind die Größen: Plattenteilflächen $A_1 = A_2 = 100 \ cm^2; U_{\mathrm{B}}=100 \ V;d = 1 \ cm$.
    \begin{enumerate}[label=\alph*)]
            \item  Der Schalter ist geschlossen. Es gelte $\varepsilon_{\mathrm{r1}}=\varepsilon_{\mathrm{r2}}=1$. Wie groß sind die Kondensatorladung $Q$, die elektrische Flussdichte $D$, die elektrische Feldstärke $E$
            sowie die Kapazität $C_{\mathrm{AB}}$?
            \item Der Schalter ist geschlossen. Es gelte $\varepsilon_{\mathrm{r1}}=1, \varepsilon_{\mathrm{r2}}=3$. Wie hat sich die Gesamtkapazität allgemein im Vergleich zu Punkt a) verändert? Zu bestimmen
            sind außerdem die Werte $Q_1$ und $Q_2$, $D_1$ und $D_2$ sowie $E_1$ und $E_2$. Welche Spannung $U_{AB}$ stellt sich am Kondensator ein?
            \item  In die Anordnung gemäß Punkt a) wird \underline{nach} dem Öffnen des Schalters $S$ ein anderes Dielektrikum eingeführt, so dass gilt: $\varepsilon_{\mathrm{r1}}=1, \varepsilon_{\mathrm{r2}}=3$.
            Welche Spannung $U_{\mathrm{AB}}$ stellt sich am Kondensator ein?
    \end{enumerate}
}

\Loesung{
	Hier entsteht eine Musterlösung...
}
\subsection{Elektrisches Homogenfeld 3\label{ElHomFeld 3}}
\Aufgabe{
    Mit einem Dielektrikum, in dem maximal eine Feldstärke $E = 30 \ \frac{kV}{cm}$ zulässig ist, soll ein Kondensator für den Spannungsbereich von $100 \ kV$ realisiert werden.\\
    \newline
    Das Dielektrikum wird zunächst in einem (idealen) \underline{Plattenkondensator} eingesetzt.\\
    \begin{enumerate}[label=\alph*1)]
        \item Skizzieren Sie die Anordnung.
        \item Skizzieren Sie die Feldlinien in der Anordnung.
        \item Wo ist der Ort der maximalen Feldstärke?
        \item Skizzieren Sie den Verlauf des Betrags der elektrischen Feldstärke von einer Kondensatorplatte zur anderen $\left| E(x) \right|$, parallel zu den Kondensatorplatten
        $\left| E_{\mathrm{Diel}}(z) \right|$ und in den Kondensatorplatten $\left| E_{\mathrm{Pl}}(z) \right|$.
        \item Zeichnen Sie die 50\%-Äquipotentiallinie ein.
        \item Berechnen Sie den minimalen Plattenabstand $d_{\mathrm{Pl,min}}$ für die o.a. Randbedingungen.            
    \end{enumerate}
    Nun soll das \underline{Dielektrikum zwischen koaxialen Zylindern} eingesetzt werden.\\
    \newline
    $\left(E_{\mathrm{r}}(r)=\frac{U}{r\cdot\ln\left(\frac{a}{b}\right)};\ r_{\mathrm{a}}: Außenradius; r_{\mathrm{i}}: Radius \ des \ Innenzylinders\right)$
    \begin{enumerate}[label=\alph*2)]
        \item Skizzieren Sie die Anordnung.
        \item Skizzieren Sie die Feldlinien.
        \item Wo ist der Ort der maximalen Feldstärke?
        \item Skizzieren Sie den Verlauf der elektrischen Feldstärke $E_{\mathrm{r}}(r)$ von $r=r_{\mathrm{i}}$ bis $r=r_{\mathrm{a}}$.
        \item Zeichnen Sie die 50\%-Äquipotentiallinie ein.
        \item Berechnen Sie den minimalen Außenradius $r_{\mathrm{a}}$ für $r_{\mathrm{i}} = 1;\ 2;\ 3;\ 4;\ 5;\ 10 \ cm$ und
        die zugehörige Dicke des Dielektrikums $d_{\mathrm{i}}$.                        
    \end{enumerate}
}

\Loesung{
	Hier entsteht eine Musterlösung...
}
\subsection{Ladungsträgergeschwindigkeit\label{LadungGeschw}}
\Aufgabe{
    Durch eine Kupferleitung mit dem Querschnitt $A = 1 \ mm^2$ und mit der Länge $L = 10 \ m$ fließt ein Strom $I = 8 \ A$. Ein $mm^3$ Kupfer enthält $8,5\cdot10^{19}$ Atome.
    Es darf davon ausgegangen werden, dass jeweils 1 Elektron pro Atom am Ladungstransport beteiligt ist $(\vartheta = 20 \ ^\circ C)$.
    \newline
    \begin{enumerate}[label=\alph*)]
        \item Bestimmen Sie die Driftgeschwindigkeit der Elektronen in der Kupferleitung.
        \item Welche elektrische Feldstärke $E$ besteht in der Kupferleitung?
        \item Wie hoch ist der Spannungsabfall $U$ in dieser Kupferleitung?
        \item Welchen Widerstand hat die Kupferleitung bei den angegebenen Randbedingungen? Welchen Widerstand nimmt der Draht bei einer Erwärmung auf $\vartheta_{\mathrm{w}} = 180 \ ^\circ C$ an?
        Wie groß ist die prozentuale Widerstandserhöhung?
    \end{enumerate}
}

\Loesung{
	Hier entsteht eine Musterlösung...
}




%\section{Lösungen zu den Übungsaufgaben}
%\renewcommand{\Aufgabe}[1]{}
%\renewcommand{\Loesung}[1]{#1}
%%hier jede Aufgabe als eigene Datei mit dem Input-Befehl einfügen
%\subsection{Elektrisches Homogenfeld 1\label{ElHomFeld}}
\Aufgabe{
    Mit einem Dielektrikum, in dem maximal eine Feldstärke $E = 30 \ \frac{kV}{cm}$ zulässig ist, soll 
	ein Kondensator für den Spannungsbereich von $100 \ kV$ realisiert werden.\\

       \underline{Aufgabenteil 1}:\\
		\phantom{text}\\

        Das Dielektrikum wird zunächst in einem (idealen) \underline{Plattenkondensator} eingesetzt.\\
        \begin{enumerate}[label=\alph*1)]
            \item Skizzieren Sie die Anordnung.
            \item Skizzieren Sie die Feldlinien in der Anordnung.
            \item Wo ist der Ort der maximalen Feldstärke?
            \item Skizzieren Sie den Verlauf des Betrags der elektrischen Feldstärke von einer Kondensatorplatte zur anderen $\left| E(x) \right|$, parallel zu den Kondensatorplatten
            $\left| E_{Diel}(z) \right|$ und in den Kondensatorplatten $\left| E_{Pl}(z) \right|$.
            \item Zeichnen Sie die 50\%-Äquipotentiallinie ein.
            \item Berechnen Sie den minimalen Plattenabstand $d_{Pl,min}$ für die o.a. Randbedingungen.            
        \end{enumerate} 

       \underline{Aufgabenteil} 2:\\
	   \phantom{text}\\

        Nun soll das \underline{Dielektrikum zwischen koaxialen Zylindern} eingesetzt werden.\\
        $\left(E_r(r)=\frac{U}{r\cdot\ln\left(\frac{a}{b}\right)};\ r_a: Außenradius; r_i: Radius \ des \ Innenzylinders\right)$
        \begin{enumerate}[label=\alph*2)]
            \item Skizzieren Sie die Anordnung.
            \item Skizzieren Sie die Feldlinien.
            \item Wo ist der Ort der maximalen Feldstärke?
            \item Skizzieren Sie den Verlauf der elektrischen Feldstärke $E_r(r)$ von $r=r_i$ bis $r=r_a$.
            \item Zeichnen Sie die 50\%-Äquipotentiallinie ein.
            \item Berechnen Sie den minimalen Außenradius $r_a$ für $r_i = 1;\ 2;\ 3;\ 4;\ 5;\ 10 \ cm$ und
            die zugehörige Dicke des Dielektrikums $d_i$.                        
        \end{enumerate}
}

\Loesung{
	Hier entsteht eine Musterlösung...
}
%\subsection{Elektrisches Homogenfeld 2\label{ElHomFeld2}}
\Aufgabe{
	Zwischen den Platten des nebenstehenden Kondensators befinden sich zwei Isolierstoffe mit den Dielektrizitätszahlen $\varepsilon_{\mathrm{r1}}$ und $\varepsilon_{\mathrm{r2}}$. Der Kondensator ist über den
    Schalter $S$ mit einer Batterie verbunden.\\
    \newline
    Gegeben sind die Größen: Plattenteilflächen $A_1 = A_2 = 100 \ cm^2; U_{\mathrm{B}}=100 \ V;d = 1 \ cm$.
    \begin{enumerate}[label=\alph*)]
            \item  Der Schalter ist geschlossen. Es gelte $\varepsilon_{\mathrm{r1}}=\varepsilon_{\mathrm{r2}}=1$. Wie groß sind die Kondensatorladung $Q$, die elektrische Flussdichte $D$, die elektrische Feldstärke $E$
            sowie die Kapazität $C_{\mathrm{AB}}$?
            \item Der Schalter ist geschlossen. Es gelte $\varepsilon_{\mathrm{r1}}=1, \varepsilon_{\mathrm{r2}}=3$. Wie hat sich die Gesamtkapazität allgemein im Vergleich zu Punkt a) verändert? Zu bestimmen
            sind außerdem die Werte $Q_1$ und $Q_2$, $D_1$ und $D_2$ sowie $E_1$ und $E_2$. Welche Spannung $U_{AB}$ stellt sich am Kondensator ein?
            \item  In die Anordnung gemäß Punkt a) wird \underline{nach} dem Öffnen des Schalters $S$ ein anderes Dielektrikum eingeführt, so dass gilt: $\varepsilon_{\mathrm{r1}}=1, \varepsilon_{\mathrm{r2}}=3$.
            Welche Spannung $U_{\mathrm{AB}}$ stellt sich am Kondensator ein?
    \end{enumerate}
}

\Loesung{
	Hier entsteht eine Musterlösung...
}
%....

%Das ist nur Beispielsinhalt, bitte auskommentieren!
%\subsection{Gleichstrommaschine 1\label{AufgGM1}}
\Aufgabe{
    Eine fremderregte Gleichstrommaschine hat die folgenden Angaben auf ihrem Typenschild:
    \begin{itemize}
        \item Nennleistung: $41,8\,\text{kW}$
        \item Nenndrehzahl: $1900\,\frac{1}{\text{min}}$
        \item Ankernennspannung: $440\,\text{V}$
        \item Ankernennstrom: $100\,\text{A}$
        \item Erregung: $240\,\text{V}$
        \item Erregerstrom: $10\,\text{A}$
        \item Leerlaufdrehzahl: $2000\,\frac{1}{\text{min}}$
    \end{itemize}

    Hinweis: Die Nennleistung auf einem Motortypenschild bezeichnet immer die abgegebene mechanische Leistung im Nennpunkt.

    \begin{itemize}
        \item[\bf a)]
        Wie groß ist das Drehmoment der Maschine im Nennpunkt?
        \item[\bf b)]
        Berechnen Sie den Wirkungsgrad der Maschine im Nennpunkt.	
        \item[\bf c)]
        Berechnen Sie das Produkt aus Erregerfluss und der Ankerkonstanten $K\cdot \Phi$
        \item[\bf d)]
        Wie groß ist der Ankerwiderstand $R_A$?
    \end{itemize}
}

\Loesung{
	\begin{itemize}
		
		\item[\bf a)]
		Aus der Nennleistung und der Nenndrehzahl ergibt sich direkt das Nenndrehmoment:
		\begin{align*}
		    P_\text{mech} &= M\cdot \omega\\
		    M &= \frac{P_\text{mech}}{\omega} = \frac{41,8\,\text{kW}}{2\pi\cdot 1900\,\frac{1}{\text{min}} \cdot \frac{1\,\text{min}}{60\,\text{s}}} = 210,08\,\text{Nm}
		\end{align*}
		\item[\bf b)]
		Im Motorbetrieb ist der Wirkungsgrad der Quotient aus mechanischer bezogen auf die elektrische Leistung (siehe Gleichung \ref{GlLeistung}). Hierbei muss sowohl die Ankerleistung, als auch die Erregerleistung berücksichtigt werden.
		\begin{eqa}
			\eta &= \frac{P_\text{mech}}{P_\text{elektr}} = \frac{41,8\,\text{kW}}{440\,\text{V}\cdot 100\,\text{A} + 240\,\text{V}\cdot 10\,\text{A}} = 90,08\,\%\nonumber
		\end{eqa}
		
		\item[\bf c)]
		Nach Gleichung \ref{GlInduzierteSpannungGM} wird die induzierte Spannung berechnet. Im Leerlauf muss die induzierte Spannung der Ankerspannung entsprechen. Gleiches geht aus Gleichung \ref{GlfremderregteGM1} hervor, da im Leerlauf auch das Drehmoment Null sein muss.
		\begin{eqa}
			U_q &= K \cdot \Phi \cdot \omega\tag{\ref{GlInduzierteSpannungGM}}\\
			K\cdot \Phi &= \frac{U_q}{\omega} = \frac{440\,\text{V}}{2\pi\cdot 2000\,\frac{1}{\text{min}} \cdot \frac{1\,\text{min}}{60\,\text{s}}} = 2,1\,\text{Vs}\nonumber
		\end{eqa}
		\item[\bf d)]
		Hier sind zwei Ansätze möglich. Es kann Gleichung \ref{GlfremderregteGM1} auf den Ankerwiderstand umgeformt werden:
		\begin{eqa}
			n &= \frac{U_A}{2\pi K\cdot \Phi} - \frac{R_A\cdot M_i}{2\pi(K\cdot \Phi)^2}\tag{\ref{GlfremderregteGM1}}\\
			R_A &= \left(\frac{U_A}{2\pi K\cdot \Phi} - n\right) \cdot \frac{2\pi(K\cdot \Phi)^2}{M_i}\nonumber\\
			&= \left(\frac{440\,\text{V}}{2\pi\cdot 2,1\,\text{Vs}} - \frac{1900}{60\,\text{s}}\right)\cdot \frac{2\pi\cdot (2,1\,\text{Vs})^2}{210,08\,\text{Nm}} = 220\,\text{m}\Omega\nonumber
		\end{eqa}
		Die andere Berechnungsvariante führt über das Ersatzschaltbild in Abbildung \ref{AbbGMFremderregt}. Im Nennbetrieb können wir die Quellenspannung $U_q$ berechnen. Diese muss kleiner als die Quellenspannung im Leerlauf sein.
		\begin{eqa}
			U_q &= K \cdot \Phi \cdot \omega \tag{\ref{GlInduzierteSpannungGM}}\\
			U_{q,n}&=  2,1\,\text{Vs} \cdot 2\pi\cdot 1900\,\frac{1}{\text{min}} \cdot \frac{1\,\text{min}}{60\,\text{s}} = 418\,\text{V}\nonumber
		\end{eqa}
		Der Spannungsabfall am Widerstand $R_A$ muss durch dem Maschenumlauf die Differenzspannung zwischen Ankerspannung und induzierter Spannung betragen. Da der Ankerstrom im Nennbetrieb bekannt ist, kann der Widerstand über das Ohmsche Gesetz berechnet werden.
		\begin{equation*}
		    R_A = \frac{U_A - U_q}{I_A} = \frac{440\,\text{V} - 418\,\text{V}}{100\,\text{A}} = 220\,\text{m}\Omega
		\end{equation*}
	\end{itemize}
}
\subsection{Elektrisches Homogenfeld 1\label{ElHomFeld}}
\Aufgabe{
    Mit einem Dielektrikum, in dem maximal eine Feldstärke $E = 30 \ \frac{kV}{cm}$ zulässig ist, soll 
	ein Kondensator für den Spannungsbereich von $100 \ kV$ realisiert werden.\\

       \underline{Aufgabenteil 1}:\\
		\phantom{text}\\

        Das Dielektrikum wird zunächst in einem (idealen) \underline{Plattenkondensator} eingesetzt.\\
        \begin{enumerate}[label=\alph*1)]
            \item Skizzieren Sie die Anordnung.
            \item Skizzieren Sie die Feldlinien in der Anordnung.
            \item Wo ist der Ort der maximalen Feldstärke?
            \item Skizzieren Sie den Verlauf des Betrags der elektrischen Feldstärke von einer Kondensatorplatte zur anderen $\left| E(x) \right|$, parallel zu den Kondensatorplatten
            $\left| E_{Diel}(z) \right|$ und in den Kondensatorplatten $\left| E_{Pl}(z) \right|$.
            \item Zeichnen Sie die 50\%-Äquipotentiallinie ein.
            \item Berechnen Sie den minimalen Plattenabstand $d_{Pl,min}$ für die o.a. Randbedingungen.            
        \end{enumerate} 

       \underline{Aufgabenteil} 2:\\
	   \phantom{text}\\

        Nun soll das \underline{Dielektrikum zwischen koaxialen Zylindern} eingesetzt werden.\\
        $\left(E_r(r)=\frac{U}{r\cdot\ln\left(\frac{a}{b}\right)};\ r_a: Außenradius; r_i: Radius \ des \ Innenzylinders\right)$
        \begin{enumerate}[label=\alph*2)]
            \item Skizzieren Sie die Anordnung.
            \item Skizzieren Sie die Feldlinien.
            \item Wo ist der Ort der maximalen Feldstärke?
            \item Skizzieren Sie den Verlauf der elektrischen Feldstärke $E_r(r)$ von $r=r_i$ bis $r=r_a$.
            \item Zeichnen Sie die 50\%-Äquipotentiallinie ein.
            \item Berechnen Sie den minimalen Außenradius $r_a$ für $r_i = 1;\ 2;\ 3;\ 4;\ 5;\ 10 \ cm$ und
            die zugehörige Dicke des Dielektrikums $d_i$.                        
        \end{enumerate}
}

\Loesung{
	Hier entsteht eine Musterlösung...
}
\subsection{Elektrisches Homogenfeld 2\label{ElHomFeld2}}
\Aufgabe{
	Zwischen den Platten des nebenstehenden Kondensators befinden sich zwei Isolierstoffe mit den Dielektrizitätszahlen $\varepsilon_{\mathrm{r1}}$ und $\varepsilon_{\mathrm{r2}}$. Der Kondensator ist über den
    Schalter $S$ mit einer Batterie verbunden.\\
    \newline
    Gegeben sind die Größen: Plattenteilflächen $A_1 = A_2 = 100 \ cm^2; U_{\mathrm{B}}=100 \ V;d = 1 \ cm$.
    \begin{enumerate}[label=\alph*)]
            \item  Der Schalter ist geschlossen. Es gelte $\varepsilon_{\mathrm{r1}}=\varepsilon_{\mathrm{r2}}=1$. Wie groß sind die Kondensatorladung $Q$, die elektrische Flussdichte $D$, die elektrische Feldstärke $E$
            sowie die Kapazität $C_{\mathrm{AB}}$?
            \item Der Schalter ist geschlossen. Es gelte $\varepsilon_{\mathrm{r1}}=1, \varepsilon_{\mathrm{r2}}=3$. Wie hat sich die Gesamtkapazität allgemein im Vergleich zu Punkt a) verändert? Zu bestimmen
            sind außerdem die Werte $Q_1$ und $Q_2$, $D_1$ und $D_2$ sowie $E_1$ und $E_2$. Welche Spannung $U_{AB}$ stellt sich am Kondensator ein?
            \item  In die Anordnung gemäß Punkt a) wird \underline{nach} dem Öffnen des Schalters $S$ ein anderes Dielektrikum eingeführt, so dass gilt: $\varepsilon_{\mathrm{r1}}=1, \varepsilon_{\mathrm{r2}}=3$.
            Welche Spannung $U_{\mathrm{AB}}$ stellt sich am Kondensator ein?
    \end{enumerate}
}

\Loesung{
	Hier entsteht eine Musterlösung...
}
\subsection{Elektrisches Homogenfeld 3\label{ElHomFeld 3}}
\Aufgabe{
    Mit einem Dielektrikum, in dem maximal eine Feldstärke $E = 30 \ \frac{kV}{cm}$ zulässig ist, soll ein Kondensator für den Spannungsbereich von $100 \ kV$ realisiert werden.\\
    \newline
    Das Dielektrikum wird zunächst in einem (idealen) \underline{Plattenkondensator} eingesetzt.\\
    \begin{enumerate}[label=\alph*1)]
        \item Skizzieren Sie die Anordnung.
        \item Skizzieren Sie die Feldlinien in der Anordnung.
        \item Wo ist der Ort der maximalen Feldstärke?
        \item Skizzieren Sie den Verlauf des Betrags der elektrischen Feldstärke von einer Kondensatorplatte zur anderen $\left| E(x) \right|$, parallel zu den Kondensatorplatten
        $\left| E_{\mathrm{Diel}}(z) \right|$ und in den Kondensatorplatten $\left| E_{\mathrm{Pl}}(z) \right|$.
        \item Zeichnen Sie die 50\%-Äquipotentiallinie ein.
        \item Berechnen Sie den minimalen Plattenabstand $d_{\mathrm{Pl,min}}$ für die o.a. Randbedingungen.            
    \end{enumerate}
    Nun soll das \underline{Dielektrikum zwischen koaxialen Zylindern} eingesetzt werden.\\
    \newline
    $\left(E_{\mathrm{r}}(r)=\frac{U}{r\cdot\ln\left(\frac{a}{b}\right)};\ r_{\mathrm{a}}: Außenradius; r_{\mathrm{i}}: Radius \ des \ Innenzylinders\right)$
    \begin{enumerate}[label=\alph*2)]
        \item Skizzieren Sie die Anordnung.
        \item Skizzieren Sie die Feldlinien.
        \item Wo ist der Ort der maximalen Feldstärke?
        \item Skizzieren Sie den Verlauf der elektrischen Feldstärke $E_{\mathrm{r}}(r)$ von $r=r_{\mathrm{i}}$ bis $r=r_{\mathrm{a}}$.
        \item Zeichnen Sie die 50\%-Äquipotentiallinie ein.
        \item Berechnen Sie den minimalen Außenradius $r_{\mathrm{a}}$ für $r_{\mathrm{i}} = 1;\ 2;\ 3;\ 4;\ 5;\ 10 \ cm$ und
        die zugehörige Dicke des Dielektrikums $d_{\mathrm{i}}$.                        
    \end{enumerate}
}

\Loesung{
	Hier entsteht eine Musterlösung...
}
\subsection{Ladungsträgergeschwindigkeit\label{LadungGeschw}}
\Aufgabe{
    Durch eine Kupferleitung mit dem Querschnitt $A = 1 \ mm^2$ und mit der Länge $L = 10 \ m$ fließt ein Strom $I = 8 \ A$. Ein $mm^3$ Kupfer enthält $8,5\cdot10^{19}$ Atome.
    Es darf davon ausgegangen werden, dass jeweils 1 Elektron pro Atom am Ladungstransport beteiligt ist $(\vartheta = 20 \ ^\circ C)$.
    \newline
    \begin{enumerate}[label=\alph*)]
        \item Bestimmen Sie die Driftgeschwindigkeit der Elektronen in der Kupferleitung.
        \item Welche elektrische Feldstärke $E$ besteht in der Kupferleitung?
        \item Wie hoch ist der Spannungsabfall $U$ in dieser Kupferleitung?
        \item Welchen Widerstand hat die Kupferleitung bei den angegebenen Randbedingungen? Welchen Widerstand nimmt der Draht bei einer Erwärmung auf $\vartheta_{\mathrm{w}} = 180 \ ^\circ C$ an?
        Wie groß ist die prozentuale Widerstandserhöhung?
    \end{enumerate}
}

\Loesung{
	Hier entsteht eine Musterlösung...
}




% Literaturverzeichnis
\clearpage
\fancyhead[OL]{\textsc{Literaturverzeichnis}}
\printbibliography

% Index erstellen
\clearpage
\fancyhead[OL]{\textsc{Index}}
\printindex

\end{document}
