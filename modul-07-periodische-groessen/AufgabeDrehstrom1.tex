\subsection{Drehstrom 1}
\Aufgabe{
    Ein Drehstrommotor hat die folgenden Nennwerte:
    \begin{itemize}
        \item Nennspannung $U_N = 400\ V$
        \item Nennstrom $I_N = 10\ A$
    \end{itemize}
 
    \begin{itemize}
        \item [\bf a)] Berechnen Sie die Spannung über jede Wicklung des Motors im Sternbetrieb $U_\Stern$.
        \item [\bf b)] Berechnen Sie die Spannung über jede Wicklung des Motors im Dreieckbetrieb $U_\Dreieck$.
        \item [\bf c)] Berechnen Sie den Strom durch jede Wicklung $I_\Stern$ im Sternbetrieb.
        \item [\bf d)] Berechnen Sie den Strom durch jede Wicklung $I_\Dreieck$ im Dreieckbetrieb.
    \end{itemize}
}


\Loesung{
	\begin{itemize}

	\item[\bf a)] Spannung über jede Wicklung des Motors $U_\Stern$ im Sternbetrieb:
	\begin{eqa}
        U_\Stern = \frac{U_N}{\sqrt{3}} = \frac{400\ V}{\sqrt{3}} \approx 230.94\ V  \nonumber   
    \end{eqa}
	
	\item[\bf b)] Spannung über jede Wicklung des Motors $U_\Dreieck$ im Dreieckbetrieb:
	\begin{eqa}
        U_\Dreieck = U_N = 400\ V  \nonumber
    \end{eqa}

	\item[\bf c)] Strom durch jede Wicklung $I_\Stern$ im Sternbetrieb: 
	\begin{eqa}
        I_\Stern = I_N = 10\ A  \nonumber
    \end{eqa}

	\item[\bf d)] Strom durch jede Wicklung $I_\Dreieck$ im Dreieckbetrieb:
    \begin{eqa}
        I_\Dreieck = \sqrt{3} \cdot I_\Stern = \sqrt{3} \cdot 10\ A \approx 17.32\ A  \nonumber
    \end{eqa}    


\end{itemize}

}