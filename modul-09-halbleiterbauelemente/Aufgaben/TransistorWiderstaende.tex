\subsection{Berechnung der Widerstände einer Transistorschaltung\label{Aufg13}}
% Alt: Aufgabe 13
\Aufgabe{
  Gegeben ist die abgebildete Transistorschaltung mit den folgenden Parametern: $U_0=\mathrm{5\,V}$ , 
  $U_\mathrm{R1}=\mathrm{0,6\,V}$, $U_\mathrm{L}=\mathrm{15\,V}$, $R_\mathrm{1}=\mathrm{300\,\Omega}$, $R_\mathrm{2}=\mathrm{2\,k\Omega}$ und $R_\mathrm{V}=\mathrm{500\,\Omega}$.
  Damit der gezeichnete Siliziumtransistor schaltet, braucht er einen Basisstrom von $50 \, \mathrm{\mu A}$ um 
  durchzuschalten. 
  \begin{itemize}
  \item[a)] Wie groß muss der Widerstand $R_\mathrm{V}$ gewählt werden, damit der Transistor durchschaltet?
  \item[b)] Angenommen der Spannungsabfall des Transistors $U_\mathrm{CE}$ beträgt $7,5 \, \mathrm{V}$, wie groß ist der Strom 
  durch $I_\mathrm{E}$?
  \end{itemize}
  \begin {figure} [H]
     \centering
     \begin{tikzpicture}
   \draw (4,2) node[npn, tr circle] (Q1) {};

   \draw (Q1.C) -- (4,4) 
   to [R,l=$R_\mathrm{L}$] (7,4)
   to [V, name=Ul, v] (7,0);

   \draw (Q1.E) to [short, -*, i, name=Ie] (4,0) 
   -- (7,0);

  \draw (Q1.B) to [R,l_=$R_\mathrm{V}$, i<, name=Ib] (0,2)
  to [R,l=$R_\mathrm{2}$] (0,0)
  to [short] (4,0);

  \draw (0,2) to [R,l_=$R_\mathrm{1}$, *-, v^<, name=Ur1] (0,4)
  -- (-2,4)
  to [V, name=U0, v] (-2,0)
  to [short, -*] (0,0) ;

   \varrmore{U0}{$U_\mathrm{0}$};
   \varrmore{Ur1}{$U_\mathrm{R1}$};
   \varrmore{Ul}{$U_\mathrm{L}$};
   \iarrmore{Ib}{$I_\mathrm{B}$};
   \iarrmore{Ie}{$I_\mathrm{E}$};
\end{tikzpicture}
     \label{fig:FigTransistorWiderstaende}
  \end {figure}
  
  % \speech{Aufgabe 13  

  % Damit der gezeichnete Siliziumtransistor schaltet, braucht er einen Basisstrom von \( 50 \mu A \) um durchzuschalten.  
  
  % a) Wie groß muss der Widerstand \( R_V \) gewählt werden, damit der Transistor durchschaltet?  
  
  % b) Angenommen der Spannungsabfall des Transistors \( U_{CE} \) beträgt \( 7,5V \), wie groß ist der Strom durch \( I_E \)?  
  
  % Beschreibung der Abbildung  
  
  % Die Abbildung zeigt eine **Transistorschaltung** mit einer **Spannungsquelle von 5V** auf der linken Seite, die über einen **300Ω-Widerstand** mit einem **weiteren 300Ω-Widerstand (\( R_V \))** verbunden ist. Parallel zu \( R_V \) ist ein **2kΩ-Widerstand** geschaltet. Diese Schaltung speist den **Basisstrom \( I_B \)** in die Basis eines **npn-Transistors**.  
  
  % Die Basis-Emitter-Spannung \( U_{BE} \) ist mit einem blauen Pfeil markiert. Der **Basisstrom \( I_B \)** ist mit einem roten Pfeil gekennzeichnet.  
  
  % Auf der rechten Seite befindet sich eine **zweite Spannungsquelle mit 15V**, die über einen **500Ω-Kollektorwiderstand** mit dem **Kollektor des Transistors** verbunden ist. Der **Emitterstrom \( I_E \)** ist mit einem roten Pfeil dargestellt.  
  
  % Die Aufgabe besteht darin, \( R_V \) zu berechnen, um den erforderlichen Basisstrom \( I_B \) sicherzustellen, sowie den Emitterstrom \( I_E \) zu bestimmen, wenn der Transistor eine Kollektor-Emitter-Spannung \( U_{CE} = 7,5V \) aufweist.  
  % }  
  
}


\Loesung{
  \begin{itemize}
    \item[a)]
  \end{itemize}
  \[
  \frac{5\,\text{V}}{2\,\text{k}\Omega + 0,3\,\text{k}\Omega} = \frac{U_1}{300\,\Omega}
  \Rightarrow U_\text{1} = 300\,\Omega \cdot \frac{5\,\text{V}}{2\,\text{k}\Omega + 0,3\,\text{k}\Omega} = 0,65\,\text{V}
  \]
  \[
  5\,\text{V} = 0,65\, \text{V} + U_\text{RV} + 0,6\,\text{V}
  \]
  \[
  U_\text{RV} = 5\,\text{V} - 0,65\,\text{V} - 0,6\,\text{V} = 3,75\,\text{V}
  \]
  \[
  R_\text{V} = \frac{3,75\,\text{V}}{50\,\mu \text{A}} = 75\,\text{k}\Omega
  \]

  \begin{itemize}
    \item[b)]
  \end{itemize}
  \[
  I_\text{C} = \frac{15\,\text{V} - 7,5\,\text{V}}{500\,\Omega} = 15\,\text{mA}
  \]
  \[
  I_\text{E} = I_\text{C} + I_\text{B} = 15\,\text{mA} + 50\,\mu \text{A} = 15,05\,\text{mA}
  \]
  \[
  \frac{I_\text{C}}{I_\text{B}} = \beta = 300
  \]
  Siehe: Abschnitt \ref{sec:Bipolartransistor} und \ref{fig:StroemeUndSpannungenBeimBipolartransistor}

%   \speech{
% **Aufgabe 13 – Berechnung des Basis- und Kollektorstroms eines npn-Transistors**

% Gegeben ist ein **npn-Transistor**, der durch einen **Basiswiderstand** gesteuert wird. Die Aufgabe besteht darin, den benötigten Widerstand \( R_V \) sowie die Ströme \( I_C \), \( I_B \) und \( I_E \) zu berechnen.

% ### Teil a) Berechnung des Basis-Vorwiderstands \( R_V \)
% Zunächst wird die Spannung \( U_1 \) bestimmt, die durch den Spannungsteiler mit **2 kΩ und 0,3 kΩ** erzeugt wird. Die allgemeine Formel für einen Spannungsteiler lautet:

% \[
% U_1 = U_0 \cdot \frac{R_2}{R_1 + R_2}
% \]

% Eingesetzte Werte:
% - \( U_0 = 5V \) (Eingangsspannung)
% - \( R_1 = 2 kΩ \)
% - \( R_2 = 0,3 kΩ \)

% \[
% U_1 = 5V \cdot \frac{0,3 kΩ}{2 kΩ + 0,3 kΩ}
% \]

% \[
% U_1 = 5V \cdot \frac{0,3 kΩ}{2,3 kΩ} = 0,65V
% \]

% Nun wird die Spannung am Widerstand \( R_V \) berechnet:

% \[
% U_{RV} = U_0 - U_1 - U_{BE}
% \]

% Gegeben:
% - \( U_0 = 5V \)
% - \( U_1 = 0,65V \)
% - \( U_{BE} = 0,6V \) (Basis-Emitter-Spannung eines npn-Transistors)

% \[
% U_{RV} = 5V - 0,65V - 0,6V = 3,75V
% \]

% Der Basisstrom \( I_B \) ist gegeben mit **50 Mikroampere**:

% \[
% R_V = \frac{U_{RV}}{I_B}
% \]

% \[
% R_V = \frac{3,75V}{50 \mu A} = 75 kΩ
% \]

% **Ergebnis**: Der Basis-Vorwiderstand \( R_V \) muss **75 kΩ** betragen.

% ---

% ### Teil b) Berechnung des Kollektorstroms \( I_C \) und des Emitterstroms \( I_E \)
% Die Spannung über den **Kollektorwiderstand** \( R_C \) beträgt:

% \[
% U_C = 15V - 7,5V
% \]

% \[
% I_C = \frac{U_C}{R_C} = \frac{15V - 7,5V}{500Ω}
% \]

% \[
% I_C = \frac{7,5V}{500Ω} = 15 mA
% \]

% Der Emitterstrom ergibt sich aus der Summe von \( I_C \) und \( I_B \):

% \[
% I_E = I_C + I_B = 15 mA + 50 \mu A = 15,05 mA
% \]

% Die **Stromverstärkung** \( \beta \) wird berechnet als:

% \[
% \beta = \frac{I_C}{I_B} = \frac{15 mA}{50 \mu A} = 300
% \]

% **Ergebnisse**:
% - Der Kollektorstrom beträgt **15 mA**.
% - Der Emitterstrom beträgt **15,05 mA**.
% - Die Stromverstärkung \( \beta \) beträgt **300**.
% }

}



