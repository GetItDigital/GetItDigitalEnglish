
\s{ 
	\section{Einleitung}
	Der Elektromagnetismus bildet die Grundlage vieler Technologien in unserem täglichen Leben.
	Die Magnetresonanztomographie ermöglicht zum Beispiel präzise Diagnosen und Therapien in der Medizin. 
	Die effiziente Erzeugung, Übertragung und Verteilung von elektrischer Energie wird durch Generatoren und Transformatoren gewährleistet. 
	In der Computertechnik sind magnetische Speicher, wie Festplattenlaufwerke, für die Datenverarbeitung und -speicherung unverzichtbar. 
	Die Grenzen der Anwendungsgebiete sind jedoch noch nicht erreicht. Zukünftige technologische Entwicklungen wie Quantencomputer, drahtlose Energieversorgung und magnetische Levitation werden diese weiter verschieben. 
	Ein fundiertes Verständnis magnetischer Größen ist entscheidend für die Weiterentwicklung und Optimierung existierender sowie zukünftiger Technologien.

	Dieses Kapitel bietet eine Einführung in die elektromagnetischen Wirkweisen und die dazugehörigen magnetischen Größen. 
	Zunächst wird das magnetische Feld beschrieben, gefolgt von einer Erklärung relevanter Größen für den magnetischen Kreis wie Durchflutung, magnetischer Fluss und magnetischer Widerstand.
	Anschließend werden die Funktionsweisen der Lorentzkraft, der Induktion und der Induktivität erläutert, gefolgt von der Berechnung des Energieinhalts im magnetischen Feld. Das Kapitel schließt mit einem Exkurs zum Skin-Effekt und Hall-Effekt.

	\begin{Lernziele}{Magnetische Größen}
		Die Studierenden
		\begin{itemize}
			\item kennen die grundlegenden magnetischen Größen im magnetischen Kreis.
			\item verstehen die physikalischen Wirkprinzipien hinter den einzelnen magnetischen Größen.
			\item können die Wechselwirkungen der magnetischen Größen zueinander beschreiben.
			\item können die einzelnen Größen im magnetischen Kreis berechnen.
		\end{itemize}
	\end{Lernziele}
}
\newvideofile{EinleitungMagnetismus}{Magnetismus}

\b{
	\begin{frame}{Lernziele}
		\begin{Lernziele}{Magnetische Größen}
			Die Studierenden
			\begin{itemize}
				\item kennen die grundlegenden magnetischen Größen im magnetischen Kreis.
				\item verstehen physikalischen Wirkprinzipien hinter den einzelnen magnetischen Größen.
				\item können die Wechselwirkungen der magnetischen Größen zueinander beschreiben.
				\item können die einzelnen Größen im magnetischen Kreis berechnen.
			\end{itemize}
		\end{Lernziele}
	\end{frame}
}
\v{
	%Videoframe 
	\begin{frame}{Lernziele}
		\begin{Lernziele}{Einführung in den Magnetismus}
			Du kannst ...
			\begin{itemize}
				\item ... den Aufbau eines Permanentmagnet erklären.
				\item ... das magnetische Feld und seinen Eigenschaften beschreiben.
			\end{itemize}
		\end{Lernziele}
	\speech{LernzieleMagnetismus}{1}{In diesem Video bekommt ihr ein Einblick in die Grundlagen des Magnetismus. Zurerst schauen wir uns den klassischen Aufbau eines Magneten am Beispiel eines ferromagnetischen Materials an. Anschießend schauen wir uns die Eigenschaften eines magnetischen Feld, genauer gesagt, der magnetischen Feldlinien an. 
	Lasst uns mit den Ferromagnetismus starten.}
	\end{frame}
}

\begin{frame} \ftx{Magnetismus}
	\s{ 
		\section{Magnetismus}
		Der Magnetismus ist ein physikalisches Phänomen, das sich in Form von wechselwirkenden Kräften von magnetisierten bzw. magnetisierbaren Festkörpern und bewegten elektrischen Ladungen äußert. Die Kräfte werden mittels eines Magnetfeldes dargestellt. 
		Zu den geläufigsten Formen des Magnetismus werden der Elektromagnetismus und der Magnetismus von Festkörpern gezählt. Der Ferromagnetismus ist die bekannteste und wichtigste Art magnetisierter Festkörper und beschreibt das magnetische Verhalten einiger metallischer Körper, sogenannter ferromagnetischer Werkstoffe. \index{Ferromagnetismus} 
		Wie in Abbildung \ref{Magnet1} ersichtlich, besteht ein ferromagnetisches Material wie Eisen, Cobalt oder Nickel aus vielen kleinen Elementarmagneten\index{Elementarmagnet} in sogenannten Weissschen Bezirken\index{Weisssche Bezirke}, die durch Blochwände\index{Blochwände} voneinander getrennt sind. 
		In einem unmagnetisierten Körper sind diese Weissschen Bezirke willkürlich angeordnet und somit nicht ausgerichtet. Der Körper ist nach außen nicht (oder nur wenig) magnetisiert, da sich die Magnetisierungsrichtungen der einzelnen Bezirke großteils gegenseitig aufheben.
		
		Durch ein starkes externes Magnetfeld (die benötigte Stärke ist temperatur- und materialabhängig) können die ungeordneten Elementarmagnete parallel ausgerichtet werden. Durch die gleiche Ausrichtung der Weissschen Bezirke wird das Material selbst magnetisch.
	}
	
\b{
	\fo{width=\textwidth}{width=\textwidth}{Aufbau_Permanentmagnet_1}{
		\put(24,0){\line(-1,1){2.8}}\put(24,0){Weisssche Bezirke}
		\put(30,3.5){\line(-1,1){4.4}}\put(30.3,3.5){Elementarmagnete}
		\put(36,7){\line(-1,1){3.8}}\put(36.3,7){Blochwände}
	}{{\bf Aufbau eines Permanentmagneten.} Links sind die  Weissschen Bezirke mit den Elementarmagneten willkürlich ausgerichtet, der Körper ist nicht magnetisiert. Der rechte Körper ist magnetisiert, da dort die Elementarmagnete der Weissschen Bezirke gleichgeordnet ausgerichtet sind.\label{Magnet1}}
}	

	\speech{Ferromagnet_1}{1}{Ferromagnetismus tritt typischerweise in eisenhaltigen Materialien auf. In diesen Materialien finden sich viele kleine Elementar Magnete, die sogenannten Weiß schen Bezirke. Diese Bezirke stellen die kleinsten magnetischen Einheiten dar. Innerhalb eines Bezirks sind alle Elementar Magnete gleich ausgerichtet – sie zeigen in dieselbe Richtung. Die einzelnen Bezirke werden durch sogenannte Blochwände voneinander getrennt.
In der linken Abbildung sehen wir ein Stück ferromagnetisches Material, das noch nicht magnetisiert ist. Die Weiß schen Bezirke sind hier in unterschiedliche Richtungen orientiert – manche zeigen nach Norden, andere nach Osten, Süden oder Westen. Durch diese ungeordnete Verteilung heben sich die magnetischen Wirkungen gegenseitig auf – das Material wirkt insgesamt nicht oder nur sehr schwach magnetisch.}
\end{frame}

\begin{frame}  \ftx{Magnetismus}
\fo{width=\textwidth}{width=\textwidth}{Aufbau_Permanentmagnet}{
		\put(24,0){\line(-1,1){2.8}}\put(24,0){Weisssche Bezirke}
		\put(30,3.5){\line(-1,1){4.4}}\put(30.3,3.5){Elementarmagnete}
		\put(36,7){\line(-1,1){3.8}}\put(36.3,7){Blochwände}
	}{{\bf Aufbau eines Permanentmagneten.} Links sind die  Weissschen Bezirke mit den Elementarmagneten willkürlich ausgerichtet, der Körper ist nicht magnetisiert. Der rechte Körper ist magnetisiert, da dort die Elementarmagnete der Weissschen Bezirke gleichgeordnet ausgerichtet sind.\label{Magnet1}}
\speech{Ferromagnet_2}{1}{Auf der rechten Seite sieht die Anordnung anders aus. Hier wurde das gleiche Material einem äußeren Magnetfeld ausgesetzt – zum Beispiel durch die Nähe zu einem starken Permanentmagneten. Dabei richten sich viele der Weiß schen Bezirke parallel zueinander aus. Die Elementar Magnete zeigen nun überwiegend in dieselbe Richtung. Entfernt man das äußere Magnetfeld nun, bleibt diese Ausrichtung bestehen – das Stück Eisen ist nun selbst ein Permanentmagnet.
Wie stellt man nun einen gezielten Permanentmagneten her. Auch hier nutzt man ferromagnetische Materialien – oft aber Kombinationen, die eine stärkere Magnetisierung erlauben als reines Eisen. Das derzeit stärkste bekannte Material für Permanentmagneten ist Neodüm eisen bor.
Im Herstellungsprozess wird das Material zunächst erhitzt. Dadurch können sich die Weiß schen Bezirke besser bewegen. Während dieser Phase wird ein starkes externes Magnetfeld angelegt, sodass sich alle Bezirke ausrichten. Anschließend lässt man das Material im Magnetfeld abkühlen. Sobald es fest ist, bleiben die Bezirke in ihrer ausgerichteten Position – und das Material ist dauerhaft magnetisiert.
Das Ganze lässt sich auch rückgängig machen: Wenn ein Permanentmagnet stark erhitzt wird, verliert er seine Magnetisierung – er wird entmagnetisiert. Je höher dabei die Temperatur ist, desto leichter gelingt die Ummagnetisierung.
Diese Effekte sind vor allem bei sogenannten Permanentmagnetmaschinen wichtig – also Motoren oder Generatoren, die mit solchen Magneten arbeiten. Wenn diese Maschinen sehr heiß werden, besteht die Gefahr, dass die Magnete entmagnetisiert werden – die Maschine zerstört sich dann im schlimmsten Fall selbst. Dieses Thema schauen wir uns im Kapitel elektrische Maschinen noch genauer an.
}
\end{frame}

\subsection{Magnetisches Feld}
\begin{frame} \ftx{Magnetisches Feld}
	\s{
		Durch die geordnete Ausrichtung der Weissschen Bezirke entsteht ein außerhalb des Körpers messbares Magnetfeld. Magnetfelder sind Vektorfelder, die eine Krafteinwirkung auf magnetische Materialien im Raum ausüben. 
		Die Stärke des Magnetfeldes wird über die Größe der magnetischen Feldstärke $ \vec{H}$ beschrieben. Diese vektorielle Größe ordnet dabei jedem Punkt des Raums, auf den das Magnetfeld einwirkt, eine entsprechende Richtung mit einer spezifischen magnetischen Stärke ein. 
		Magnetfelder und die damit verbundene magnetische Feldstärke werden mittels Feldlinien dargestellt. 
		Magnetische Feldlinien besitzen charakteristische Eigenschaften:

		\begin{itemize}
			\item Magnetische Feldlinien sind immer geschlossen (Quellenfrei).
			\item Außerhalb des Magneten verlaufen sie vom Nord- zum Südpol.
			\item Sie treten immer senkrecht aus der Magnetoberfläche aus bzw. ein.
		\end{itemize}
		
		\fo{width=0.8\textwidth}{height=0.65\textheight}{Feldlinien_Permanentmagnet}{
			\put(75,60){\color{magnetfeld}\makebox[0pt]{N}}
			\put(75,7.5){\color{magnetfeld}\makebox[0pt]{S}}
		}{{\bf Feldlinien eines Permanentmagneten.} Links wurden die Feldlinien mit Hilfe von Metallspänen über einem Magnenten visualisert. Die rechte Grafik stellt die Eigenschaften magnetischer Feldlinien dar: Geschlossenheit, Verlauf außerhalb des Magneten von Nord nach Süd, senkrechter Austritt aus dem Magneten. \label{Magnet2}}
		\begin{Merksatz}{Magnetfeld} 
			Magnetfelder sind Vektorfelder, die an jedem Punkt des Raums, auf den sie einwirken, eine für den Punkt spezifische Richtung mit einer spezifischen Stärke besitzen. Die Stärke und Richtung werden mittels der magnetischen Feldstärke $ \vec{H}$ beschrieben.
		\end{Merksatz}
		
	}
	\b{
		\index{Magnetisches Feld}
			\f{width=0.8\textwidth}{height=0.65\textheight}{Feldlinien_Permanentmagnet_1}{}
			}
	  		\speech{magnetisches_feld_1}{1}{Schauen wir uns erstmal das magnetische Feld an. Ihr kennt sicher den Versuch aus der Schule, bei dem man Eisenspäne auf eine Plexiglasplatte streut, unter der ein Magnet liegt. Die Späne ordnen sich entlang der magnetischen Feldlinien an und machen das unsichtbare Feld so sichtbar.
			An der Ausrichtung der Späne erkennt man, wie sich das Magnetfeld im Raum verteilt. Jeder dieser Späne verhält sich dabei wie ein kleiner Elementarmagnet – und richtet sich so aus, dass das Magnetfeld des Stabmagneten erkennbar wird.}
			\end{frame}

\begin{frame} \ftx{Magnetisches Feld}
	\b{
		\index{Magnetisches Feld}
		\onslide<1->{
			\fo{width=0.8\textwidth}{height=0.65\textheight}{Feldlinien_Permanentmagnet}{
				\put(75,60){\color{magnetfeld}\makebox[0pt]{N}}
				\put(75,7.5){\color{magnetfeld}\makebox[0pt]{S}}
			}{{\bf Feldlinien eines Permanentmagneten.} Links wurden die Feldlinien mit Hilfe von Metallspänen über einem Magnenten visualisert. Die rechte Grafik stellt die Eigenschaften magnetischer Feldlinien dar: Geschlossenheit, Verlauf außerhalb des Magneten von Nord nach Süd, senkrechter Austritt aus dem Magneten.}
			}
		\begin{itemize}
			\onslide<2->{
			\item Magnetische Feldlinien sind immer geschlossen (Quellenfrei).
			}
			\onslide<3->{
			\item Außerhalb des Magneten verlaufen sie vom Nord- zum Südpol.
			}
			\onslide<4->{
			\item Sie treten immer senkrecht aus der Magnetoberfläche aus bzw. ein.
			}
		\end{itemize}
	}
	  		\speech{magnetisches_feld_2}{1}{Vergleichen wir den Verlauf der Eisenspäne mit den modelhaften magnetischen Feldlinienverlauf eines Stabmagneten, so werden die typische Linienmuster des Models bestätigt. Es lassen sich hierauf drei zentrale Eigenschaften des Feldlinienverlaufs ableiten.}
			\speech{magnetisches_feld_2}{2}{Zum einen sind magnetische Feldlinienen immer geschlossen. Sie haben weder einen Anfang oder Ende. Oder anders gesagt: Magnetfelder sind quellenfrei. Das ist eine mathematische Eigenschaft des zugrunde liegenden Vektorfeldes.}
			\speech{magnetisches_feld_2}{3}{Zweitens: Außerhalb des Magneten verlaufen die Feldlinien vom Nordpol zum Südpol. Diese Richtung wurde zwar willkürlich festgelegt, ist aber international verbindlich. Innerhalb des Magneten kehren die Feldlinien dann wieder zurück – vom Südpol zum Nordpol – sodass sich der Kreis schließt. Merkt euch am besten nur den Magnetverlauf außerhalb des Magneten, bevor ihr durcheinander kommt.}
			\speech{magnetisches_feld_2}{4}{Und drittens: Magnetische Feldlinien treten immer senkrecht auf Materialgrenzen ein oder aus. Das gilt nicht nur für die Oberfläche eines Magneten, sondern allgemein für alle Übergänge, an denen sich das magnetische Material ändert – also vor allem dort, wo sich die magnetische Permeabilität ändert. Die Permeabilität werden wir später noch genauer besprechen. Nur wenn zwei Materialien exakt die gleiche Permeabilität haben – was in der Praxis selten vorkommt – ist ein anderer Verlauf möglich. Ansonsten gilt: An jeder Materialgrenze verläuft die Feldlinie senkrecht. Diese grundlegenden Eigenschaften des Magnetfeldes helfen uns, das Verhalten von Permanentmagneten zu verstehen. Für uns in der Elektrotechnik sind aber vor allem elektromagnetische Felder relevant – also Magnetfelder, die durch elektrische Ströme entstehen. Genau darum geht es in der nächsten Lerneinheit.}
\end{frame}
