% Kapitel 3 - Ortskurven
%           - Impedanzortskurven
%           - Admittanzortskurven
%           - Ortskurven Grundschaltungen
\section{Ortskurven}
\begin{frame}\ftx{\secname}
\index{Ortskurve}%
\s{%
    Die Darstellung von Ortskurven ist ein Mittel zur Visualisierung der parameterabhängigen Veränderung von komplexen Größen. 
    Zur Analyse des frequenzvariablen Verhaltens von Eintoren bieten sich Impedanz- und Admittanzortskurven an. 
    In diesen kann die Veränderung der Impedanz respektive der Admittanz in Abhängigkeit der (Kreis-)Frequenz dargestellt werden.
}%
    \begin{Lernziele}{Ortskurven}
        Studierende lernen:
        \begin{itemize}
            \item Ortskurven kennen und mögliche Anwendungsbereiche kennen.
            \item Ortskurven prinzipiell zu konstruieren und zu interpretieren.
            \item Impedanz- und Admittanzortskurven von Grundschaltungen kennen.
        \end{itemize}
    \end{Lernziele}
\end{frame}

\subsection{Definition Ortskurve}
\begin{frame}[t]\ftx{\subsecname}
\s{%
    Eine Ortskurve ist die graphische Darstellungen einer komplexen Größe $\underline{z}$ % komplexer Zeiger statt Größe ? 
    in Abhängigkeit eines reellen Parameters $p$ in der komplexen Ebene\cite[Vgl. ]{weissgerber2}:
   \begin{equation}\begin{aligned}
        \label{eq:def:ortskurve}
        \underline{z}(p) 
            &= \Re\{\underline{z}(p)\} + \mathrm{j}\Im\{\underline{z}(p)\} 
                &&\text{mit }\underline{z}\in\mathbb{C};\ p\in\mathbb{R}
    \end{aligned}\end{equation}%
    Ortskurven werden in vielen verschiedenen Bereichen angewandt, um mögliche Zustände eines Systems zu visualisieren.
    Abbildung \ref{fig:ortskurve:stromortskurve:asm} zeigt exemplarisch die Stromortskurve einer Asynchronmaschine in der komplexen Ebene, 
    auch Heylandkreis oder Ossannakreis genannt. [QUELLE:XYZ] Zu Erkennen ist der komplexe Statorstrom $\underline{I}_s$
    in Abhängigkeit des Schlupfes $s$. 
    Die Darstellungsform eignet sich beispielsweise zur Visualisierung verschiedener Betriebszustände, 
    im Fall der Asynchronmaschine sind das der Motor-, Brems- und Generatorbetrieb.
    \fu{\includegraphics{Tikz/pdf/plot_stromortskurve_asynchronmaschine.pdf}}{%
        Beispiel Stromortskurve einer Asynchrommaschine mit Betriebszuständen\label{fig:ortskurve:stromortskurve:asm}%
    }%
    Im Beispiel der Stromortskurve einer Asynchronmaschine lassen sich zudem verschiedene Leistungswerte graphisch ermitteln
    wie in Abbildung \ref{fig:ortskurve:stromortskurve:asm:leistung} exemplarisch dargestellt ist. 
    Die Beispiele dienen der Veranschaulichung der Definition und möglicher Anwendungsbereiche von Ortskurven,
    weshalb von einer Herleitung beider Beispiele abgesehen wird. 
    \fu{\includegraphics{Tikz/pdf/plot_stromortskurve_asynchronmaschine_leistung.pdf}}{%
        Beispiel Stromortskurve einer Asynchrommaschine mit Leistungsgrößen\label{fig:ortskurve:stromortskurve:asm:leistung}%
    }%
    Ortskurven eignen sich gut zur Visualisierung von Systemzuständen, -größen und deren Einfluss aufeinander.
}

\b{%
    \centering
    \textbf{Def. Ortskurve:} Darstellung eines komplexen Zeigers 
    $\underline{z}(p) = \Re\{\underline{z}(p)\} + \mathrm{j}\Im\{\underline{z}(p)\}$\newline
    \hphantom{Def. Ortskurve} in komplexer Ebene in Abhängigkeit eines realen Parameter $p$.
    \begin{minipage}{\textwidth}\centering%
    \begin{minipage}{0.5\textwidth}\centering%
        \only<2-3| handout:2-3>{\resizebox{\textwidth}{!}{\includegraphics{Tikz/pdf/plot_stromortskurve_asynchronmaschine.pdf}}}%
        \only<4-| handout:4->{\resizebox{\textwidth}{!}{\includegraphics{Tikz/pdf/plot_stromortskurve_asynchronmaschine_leistung.pdf}}}%
    \end{minipage}\hspace{0.05\textwidth}%
    \begin{minipage}{0.45\textwidth}%\centering
        \onslide<3-| handout:3->{\textbf{Anwendungsbereiche?}\vspace{1cm}\\}%
        \onslide<4-| handout:4->{\textbf{Visualisierung von:}\vspace{1mm}\\
            \quad - Systemzuständen\\
            \quad - Systemgrößen\\
            \quad - Einfluss von Systemgrößen
        }%
    \end{minipage}%
    \end{minipage}\hfill\vspace{5pt}%

    \only<2-| handout:2->{\textbf{Beispiel:} Stromortskurve einer Asynchrommaschine für $\underline{I}_s(s)$}
}
\end{frame}


\subsection{Zusammenhang Zeigerdiagramm und Ortskurve}
\begin{frame}\ftx{\subsecname}
\index{Zeigerdiagramm}
    \s{
        Ein Zeiger im Zeigerdiagramm stellt eine komplexe Größe stationär in der komplexen Ebene dar.
        Die Darstellungsform dient der Visualisierung von Phasenverschiebungen und Betragsverhältnissen.
        Typische Größen für Wechselstromkreise sind unter anderem Spannung, Strom, Leistung und Impedanz.

        Ortskurven beschreiben die Bahn, die ein Zeiger bei Variation eines Parameters durchläuft.
        Die Ortskurve kann so als Verallgemeinerung eines Zeigers im Zeigerdiagramm verstanden werden. \cite[Vgl.]{weissgerber2}% VGL REF: Weißgerber, Elektrotechnik für Ingenieure, Kapitel 5 Ortskurven       
        Übliche Parameter sind Frequenz und Bauteilgrößen. %Übliche Größen sind Impedanz und Admittanz.

        Die Hervorhebung einzelner Zeiger mit Angabe des variierenden Parameters kann 
        den Verlauf der Ortskurve verdeutlichen und die Interpretation erleichtern.
        Positionsänderungsrate und -richtung geben Aufschluss über die Dynamik des Systems.
    }%
    %\b{% Visualisierung allgemeiner <1>Zeiger, <2>Zeigerschar und <3>Ortskurve
    %    %Todo: Visualisierung Zeiger $\to$ Zeigerschar (Ortskurve) 
    %    %Z.B. Zeiger mit Overlays einblenden, dann Kreis als Kurve (Vgl. Stromortskurve vorher) Ist schöner als nur Text.
    %}
    \b{
        Zeigerdiagramm:\\\quad Zeigt Größe in komplexer Ebene als Zeiger (eines Punktes). 
    }%
    \begin{Merksatz}{Ortskurve}
        Zeigt Größe in komplexer Ebene parameterabhängig als Kurve (Punkteschar).
    \end{Merksatz}
\end{frame}

%%%%%%%%%%%%%%%%%%%%%%%%%%%%%%%%%%%%%%%%%%%%%%%%%%%%%%%%%%%%%%%%%%%%%%%%%%%%%%%%%%%%%%%%%%%%%%%%%%
\s{
\subsection{Impedanzortskurve}
\begin{frame}\ftx{\subsecname}

    In Impedanzortskurven wird die Impedanz eines Systems in Abhängigkeit eines Parameters dargestellt.

    Beispiel \ref{bsp:ortskurve:z:rvar} zeigt eine Impedanzortskurve für eine variable Bauteilgröße.

    Beispiel \ref{bsp:ortskurve:z:fvar} zeigt eine Impedanzortskurve für variable Frequenz.

\begin{bsp}{
    Impedanzortskurve RL-Glied, Widerstand variabel}{ortskurve:z:rvar}% Überschrift, Label
    Konstruktion der Impedanzortskurve einer RL-Serienschaltung für variablen Widerstand $R$.
    Die Schaltung ist in Abbildung \ref{fig:circ:rls:Rvar} abgebildet.

    % alttext: Schaltbild R und L in Serie, Beschriftung R und jwL
    \fu{\begin{tikzpicture}
	\draw (0,0) 
	to[R,l=$R$,o-] (2,0) 
	to[L,l=$\mathrm{j}\omega L$,-o] (4,0);
\end{tikzpicture}%
}{RL-Glied (Serienschaltung)\label{fig:circ:rls:Rvar}}

    Die Impedanz des RL-Gliedes ist gegeben durch:

    \begin{equation*}
        \underline{Z} = R + \mathrm{j}\omega L
    \end{equation*}

    Sei $R$ variabel und beschrieben als ($p$-fache) Vielfache eines Referenzwiderstandes $R_0$ 
    mit $p$ im geschlossenen Intervall $[0,3]$, so gilt:

    \begin{equation*}
        R(p) = p \cdot R_0 \quad\text{mit } p \in [0,3]
    \end{equation*}

    Seien $\omega$ und $L$ konstant, sowie die Reaktanz $\omega L$ gleich einer Referenzreaktanz $X_0$:

    \begin{equation*}
        \omega L = X_0 \quad \text{mit } \omega,\ L = konst.
    \end{equation*}

    Impedanz $\underline{Z}$ in Abhängigkeit von $p$:

    \begin{equation*}
        \underline{Z}(p) = p \cdot R_0 + \mathrm{j}X_0
    \end{equation*}

    \fu{\includegraphics{Tikz/pdf/plot_ortskurve_RLs_Rvar.pdf}}% R=p*R0, p in [0,3]; wL=konst.
        {Impedanzortskurve RL-Glied (Serie)\label{fig:plot:rls:Rvar}}

\end{bsp}

\begin{bsp}{Impedanzortskurve RL-Glied, Frequenz variabel}{ortskurve:z:fvar}% Überschrift, Label

    Konstruktion der Impedanzortskurve der RL-Serienschaltung in Abbildung \ref{fig:circ:RLs:wLvar} 
    für variable Frequenz $f$ und oder variable Induktivität $L$.
    Aufbau wie in Beispiel \ref{bsp:ortskurve:z:rvar}, jedoch mit variabler Frequenz $\omega$.

    \fu{\begin{tikzpicture}
	\draw (0,0) 
	to[R,l=$R$,o-] (2,0) 
	to[L,l=$\mathrm{j}\omega L$,-o] (4,0);
\end{tikzpicture}%
}{RL-Glied (Serienschaltung)\label{fig:circ:RLs:wLvar}}

    Gesamtimpedanz allgemein: 
    \begin{align*}
        \underline{Z}   &= R + \mathrm{j}\omega L \\
                        &= R + \mathrm{j}X
    \end{align*}

    Die Gesamt-Reaktanz $X$ des RL-Gliedes ist proportional zu $\omega$ und proportional zu $L$.
    Das heißt eine Veränderung von $\omega$ hat den selben Einfluss wie eine Veränderung von $L$ 
    auf die Impedanz des RL-Gliedes.
    
    Mit $X$ als Vielfache $p$ einer Referenz-Reaktanz $X_0$ und $R=R_0$ gilt:

    \begin{equation*}
        \underline{Z} = R_0 + p \cdot \mathrm{j}X_0 
    \end{equation*}

    In der komplexen Ebene dargestellt:
    \fu{\captionsetup{labelformat=empty}%
        \includegraphics{Tikz/pdf/plot_ortskurve_RLs_wLvar.pdf}}% wL=p*w0L0, p in [0,3]; R=konst.
        {RL-Glied (Serienschaltung)}

\end{bsp}
\end{frame}
}% end skript only, 
% note: column environment breaks bsp-box -> \s{frame with bsp-box}, \b{frame with columns}
\b{
\subsection{Impedanzortskurve - RL-Glied , $R$ variabel}
\begin{frame}\ftx{\subsecname}
\begin{columns}
    \column{0.5\textwidth}\centering
    \begin{tikzpicture}
	\draw (0,0) 
	to[R,l=$R$,o-] (2,0) 
	to[L,l=$\mathrm{j}\omega L$,-o] (4,0);
\end{tikzpicture}%
\hfill% Schaltbild: R und L in Serie
    %
    \begin{equation*}
        \underline{Z}=R+\mathrm{j}\omega L
    \end{equation*}

    \begin{align*}
        \underline{Z} &= p \cdot R_0 + \mathrm{j}X_0\\
        \text{mit}\ p &\in [0,3]\\
        R_0,\ X_0 &= konst.
    \end{align*}

    \column[c]{0.5\textwidth}
    \includegraphics{Tikz/pdf/plot_ortskurve_RLs_Rvar.pdf}
\end{columns}
\end{frame}

\subsection{Impedanzortskurve - RLs-Glied, $\omega$ variabel}
\begin{frame}\ftx{\subsecname}
\begin{columns}
    \column{0.5\textwidth}\centering
    \begin{tikzpicture}
	\draw (0,0) 
	to[R,l=$R$,o-] (2,0) 
	to[L,l=$\mathrm{j}\omega L$,-o] (4,0);
\end{tikzpicture}%
\hfill% Schaltbild: R und L in Serie
    %
    \begin{equation*}
        \underline{Z}=R+\mathrm{j}\omega L
    \end{equation*}

    \begin{align*}
        \underline{Z} &= R_0 + \mathrm{j}p \cdot \omega_0L_0\\ 
        \text{mit}\ p &\in [0,3]\\
        R_0, L_0, \omega_0 &= konst.
    \end{align*}

    \column{0.5\textwidth}
    \includegraphics{Tikz/pdf/plot_ortskurve_RLs_wLvar.pdf}
\end{columns}
\end{frame}
}% end beamer only
%%%%%%%%%%%%%%%%%%%%%%%%%%%%%%%%%%%%%%%%%%%%%%%%%%%%%%%%%%%%%%%%%%%%%%%%%%%%%%%%%%%%%%%%%%%%%%%%%%

\subsection{Admittanzortskurve - Inversion von Ortskurven}
\begin{frame}\ftx{\subsecname}

\s{
    Analog zur Impedanzkurve bezeichnet eine Admittanzkurve die Ortskurve einer Admittanz. 
    Die Admittanz $\underline{Y}$ entspricht algebraisch dem Kehrwert der Impedanz $\underline{Z}$ und vice versa:

    \begin{equation}
        \underline{Y} = \frac{1}{\underline{Z}} \quad \leftrightarrow \quad \underline{Z} = \frac{1}{\underline{Y}}
    \end{equation}

    Die Umformung von Impedanz zu Admittanz und umgekehrt ist jeweils ein Spezialfall der Möbius-Transformation.

    Für die Beziehung zwischen Admittanz- und Impedanzortskurven gelten daher folgende allgemeinen Eigenschaften von Möbius-Transformation:
    \cite[Vgl. ]{behrends}

    \begin{itemize}
        \item Winkeltreue: $\angle(\underline{Z_1},\underline{Z_2}) = \angle(\underline{Y_1},\underline{Y_2}) \qquad \text{mit } \underline{Y}_i = \underline{Z}_i^{-1},\ i \in [1,2]$ 
        \item Kreisverwandschaft: Kreise und Geraden werden auf Kreise und Geraden abgebildet
    \end{itemize}

    Die Kehrwertbildung entspricht im speziellen der Inversion, einem Elementartyp der Möbius-Transformation. 
    Eine Inversion von Ortskurven führt zu den in Tabelle \ref{tab:moebiusinversion} 
    aufgelisteten geometrischen Umformungen von Kreisen und Geraden. 
    Abbildung \ref{fig:plot:moebiusinversion} zeigt exemplarisch die Inversion eines Gitters durch den Ursprung.
}
    % alttext:
    % Gerade durch Ursprung zu Gerade durch Ursprung
    % Gerade nicht durch Ursprung zu Kreis durch Ursprung
    % Kreis durch Ursprung zu Gerade nicht durch Ursprung
    % Kreis nicht durch Ursprung zu Kreis nicht durch Ursprung
    \definecolor{Gray}{gray}{0.97}
    \begin{table}[H]\centering
        \s{\caption{Typische Formen bei Inversion von Ortskurven}
            \label{tab:moebiusinversion}}
        \begin{tabular}[h]{|r|r|}
            \hline\rowcolor{Gray}
                \textbf{ursprüngliche Ortskurve} &
                \textbf{invertierte Ortskurve} \\
            \hline
                Gerade \phantom{nicht} durch den Ursprung &
                Gerade \phantom{nicht} durch den Ursprung \\
            \hline
                Gerade nicht durch den Ursprung &
                Kreis\quad \phantom{nicht} durch den Ursprung \\
            \hline
                Kreis\quad \phantom{nicht} durch den Ursprung &
                Gerade nicht durch den Ursprung \\
            \hline
                Kreis\quad nicht durch den Ursprung &
                Kreis\quad nicht durch den Ursprung \\
            \hline
        \end{tabular}
    \end{table}
    \fu{\includegraphics{Tikz/pdf/plot_ortskurven_inversion.pdf}}{%
        Beispiel Möbius-Transformation, Inversion von Gitter\protect\footnotemark\label{fig:plot:moebiusinversion}%
    }
    \footnotetext{Angelehnt an MoebiusInversion.svg von Chrislb, CC-BY-SA-2.0-DE, 2005, \url{https://commons.wikimedia.org/wiki/File:MoebiusInversion.svg}}
\end{frame}

%%%%%%%%%%%%%%%%%%%%%%%%%%%%%%%%%%%%%%%%%%%%%%%%%%%%%%%%%%%%%%%%%%%%%%%%%%%%%%%%%%%%%%%%%%%%%%%%%%

\subsection{Ortskurven von Grundschaltungen}
\begin{frame}\ftx{\subsecname}
\s{
    Tabelle \ref{tab:ortskurven:grundschaltungen} zeigt eine Übersicht der Impedanz- und Admittanzortskurven 
    für die vier Grundschaltungen RL- und RC-Glied jeweils in Serien- und Parallelschaltung. 

    Ortskurven schwingungsfähiger LC- und RLC-Glieder unterscheiden sich von den hier dargestellen Ortskurven dadurch, 
    dass der Blindanteil der Impedanz bzw. Admittanz je nach Frequenzbereich sowohl induktiv als auch kapazitiv sein kann,
    je nachdem welcher Anteil überwiegt.
}
% TABELLE
% Note: \resizebox{Breite}{Höhe}{Content}
\begin{table}\centering
    \s{\caption{Ortskurven der Grundschaltungen RL- und RC-Glied}
        \label{tab:ortskurven:grundschaltungen}
    }
    \begin{tabular}[h]{|c|c|c|}
        % Tabellenteil 1
        \hline\s{$\vphantom{\bigg|}$}% Headerrow a bit bigger, oneliner-text vertically centered
        \textbf{Grundschaltung} & \textbf{$\mathbf{\underline{Z}}$-Ortskurve} & \textbf{$\mathbf{\underline{Y}}$-Ortskurve}
        \\\hline% RL-Serie
        % \parbox[...][Höhe][...]{Breite}{Content}
        \parbox[c][4cm][c]{3.75cm}{\centering% Schaltbild: RL-Serie
\begin{tikzpicture}\draw(0,0)to[short,o-](0.25,0)to[R,l=$R$](1.75,0)to[L,l=$L$](3.25,0)to[short,-o](3.5,0);\end{tikzpicture}%
}% Schaltbild
        &
        \parbox[c][4cm][c]{3.75cm}{\centering%
            \resizebox{3.75cm}{!}{\includegraphics{Tikz/pdf/plot_ortskurve_minimal_R_RLZ_R.pdf}}\newline
            \resizebox{3.75cm}{!}{\includegraphics{Tikz/pdf/plot_ortskurve_minimal_R_RLZ_L.pdf}}
        }&
        \parbox[c][4cm][c]{3.75cm}{\centering%
            \resizebox{3.75cm}{!}{\includegraphics{Tikz/pdf/plot_ortskurve_minimal_R_RLY_R.pdf}}\newline
            \resizebox{3.75cm}{!}{\includegraphics{Tikz/pdf/plot_ortskurve_minimal_R_RLY_L.pdf}}
        }\\\hline% RC-Serie
        \parbox[c][4cm][c]{3.75cm}{\centering% Schaltbild: RC-Serie
\begin{tikzpicture}\draw(0,0)to[short,o-](0.25,0)to[R,l=$R$](1.75,0)to[C,l=$C$](3.25,0)to[short,-o](3.5,0);\end{tikzpicture}%
}% Schaltbild
        &
        \parbox[c][4cm][c]{3.75cm}{\centering%
            \resizebox{3.75cm}{!}{\includegraphics{Tikz/pdf/plot_ortskurve_minimal_R_RCZ_R.pdf}}\newline
            \resizebox{3.75cm}{!}{\includegraphics{Tikz/pdf/plot_ortskurve_minimal_R_RCZ_C.pdf}}
        }&
        \parbox[c][4cm][c]{3.75cm}{\centering%
            \resizebox{3.75cm}{!}{\includegraphics{Tikz/pdf/plot_ortskurve_minimal_R_RCY_R.pdf}}\newline
            \resizebox{3.75cm}{!}{\includegraphics{Tikz/pdf/plot_ortskurve_minimal_R_RCY_C.pdf}}
        }
        \s{%Start Skript only, Tabellenteil 2
        \\\hline% RL-Parallel
        \parbox[c][4cm][c]{3.75cm}{\centering% Schaltbild: RL-Parallel
\begin{tikzpicture}\draw(0,0)to[short,o-]
    (0.5,0)--(0.5,+0.75)to[R,l=$R$](2,+0.75)--(2,0)
    (0.5,0)--(0.5,-0.75)to[L,l=$L$](2,-0.75)--(2,0)to[short,-o](2.5,0);
\end{tikzpicture}%}% Schaltbild
        &
        \parbox[c][4cm][c]{3.75cm}{\centering
            \resizebox{3.75cm}{!}{\includegraphics{Tikz/pdf/plot_ortskurve_minimal_P_RLZ_R.pdf}}\newline
            \resizebox{3.75cm}{!}{\includegraphics{Tikz/pdf/plot_ortskurve_minimal_P_RLZ_L.pdf}}
        }&
        \parbox[c][4cm][c]{3.75cm}{\centering
            \resizebox{3.75cm}{!}{\includegraphics{Tikz/pdf/plot_ortskurve_minimal_P_RLY_R.pdf}}\newline
            \resizebox{3.75cm}{!}{\includegraphics{Tikz/pdf/plot_ortskurve_minimal_P_RLY_L.pdf}}
        }
        \\\hline% RC-Parallel
        \parbox[c][4cm][c]{3.75cm}{\centering% RC-Parallel
\begin{tikzpicture}\draw(0,0)to[short,o-]
    (0.5,0)--(0.5,+0.75)to[R,l=$R$](2,+0.75)--(2,0)
    (0.5,0)--(0.5,-0.75)to[C,l=$C$](2,-0.75)--(2,0)to[short,-o](2.5,0);
\end{tikzpicture}%}% Schaltbild
        &
        \parbox[c][4cm][c]{3.75cm}{\centering
            \resizebox{3.75cm}{!}{\includegraphics{Tikz/pdf/plot_ortskurve_minimal_P_RCZ_R.pdf}}\newline
            \resizebox{3.75cm}{!}{\includegraphics{Tikz/pdf/plot_ortskurve_minimal_P_RCZ_C.pdf}}
        }&
        \parbox[c][4cm][c]{3.75cm}{\centering
            \resizebox{3.75cm}{!}{\includegraphics{Tikz/pdf/plot_ortskurve_minimal_P_RCY_R.pdf}}\newline
            \resizebox{3.75cm}{!}{\includegraphics{Tikz/pdf/plot_ortskurve_minimal_P_RCY_C.pdf}}
        }
        }%Ende Skript only, Tabellenteil 2
        \\\hline
    \end{tabular}
\end{table}
\b{\frametitle{\subsecname{} - in Reihe}} 
\end{frame}
\b{%Start beamer only - Tabellenteil 2
\begin{frame}
\frametitle{\subsecname{} - in Parallel}
    \begin{table}\centering
        \begin{tabular}[h]{|c|c|c|}
            \hline
            \textbf{Grundschaltung} & \textbf{$\mathbf{\underline{Z}}$-Ortskurve} & \textbf{$\mathbf{\underline{Y}}$-Ortskurve}
            \\\hline% RL-Parallel
            \parbox[c][4cm][c]{3.75cm}{\centering% Schaltbild: RL-Parallel
\begin{tikzpicture}\draw(0,0)to[short,o-]
    (0.5,0)--(0.5,+0.75)to[R,l=$R$](2,+0.75)--(2,0)
    (0.5,0)--(0.5,-0.75)to[L,l=$L$](2,-0.75)--(2,0)to[short,-o](2.5,0);
\end{tikzpicture}%}% Schaltbild
            &
            \parbox[c][4cm][c]{3.75cm}{\centering% \parbox[...][Höhe][...]{Breite}{Content}
                \resizebox{3.75cm}{!}{\includegraphics{Tikz/pdf/plot_ortskurve_minimal_P_RLZ_R.pdf}}\newline
                \resizebox{3.75cm}{!}{\includegraphics{Tikz/pdf/plot_ortskurve_minimal_P_RLZ_L.pdf}}
            }&
            \parbox[c][4cm][c]{3.75cm}{\centering% \parbox[...][Höhe][...]{Breite}{Content}
                \resizebox{3.75cm}{!}{\includegraphics{Tikz/pdf/plot_ortskurve_minimal_P_RLY_R.pdf}}\newline
                \resizebox{3.75cm}{!}{\includegraphics{Tikz/pdf/plot_ortskurve_minimal_P_RLY_L.pdf}}
            }
            \\\hline% RC-Parallel
            \parbox[c][4cm][c]{3.75cm}{\centering% RC-Parallel
\begin{tikzpicture}\draw(0,0)to[short,o-]
    (0.5,0)--(0.5,+0.75)to[R,l=$R$](2,+0.75)--(2,0)
    (0.5,0)--(0.5,-0.75)to[C,l=$C$](2,-0.75)--(2,0)to[short,-o](2.5,0);
\end{tikzpicture}%}% Schaltbild
            &
            \parbox[c][4cm][c]{3.75cm}{\centering% \parbox[...][Höhe][...]{Breite}{Content}
                \resizebox{3.75cm}{!}{\includegraphics{Tikz/pdf/plot_ortskurve_minimal_P_RCZ_R.pdf}}\newline
                \resizebox{3.75cm}{!}{\includegraphics{Tikz/pdf/plot_ortskurve_minimal_P_RCZ_C.pdf}}
            }&
            \parbox[c][4cm][c]{3.75cm}{\centering% \parbox[...][Höhe][...]{Breite}{Content}
                \resizebox{3.75cm}{!}{\includegraphics{Tikz/pdf/plot_ortskurve_minimal_P_RCY_R.pdf}}\newline
                \resizebox{3.75cm}{!}{\includegraphics{Tikz/pdf/plot_ortskurve_minimal_P_RCY_C.pdf}}
            }
            \\\hline
        \end{tabular}
    \end{table}
\end{frame}
}%Ende beamer only - Tabellenteil 2