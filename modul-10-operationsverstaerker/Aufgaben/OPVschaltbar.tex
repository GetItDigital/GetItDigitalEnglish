\subsection{Verstärkungseinstellung mit schaltbaren Widerständen\label{Aufg13}}
% Alt: Aufgabe 13
\Aufgabe{
    Um die Verstärkung eines OPs einzustellen, können mithilfe von Schaltern verschiedene 
    Widerstände am OP zu- und abgeschaltet werden. Berechnen Sie die untenstehenden OP-Schaltungen 
    in Abhängigkeit der Schalterstellungen S!
    \begin{itemize}
        \item[ \bf a)] Für die Schaltung in Abbildung 1 (4 Schalterstellungen)
    \end{itemize}

    \begin{figure}[H]
        \centering
        \begin{tikzpicture}
\ctikzset{tripoles/en amp/input height=-0.45}
\draw
(0,0) node[op amp, noinv input up, anchor=+] (opamp) {}
(opamp.+) 
(opamp.-) 
(opamp.out) ;

\draw (opamp.out) to[short, -*] (3,-0.5) 
    to[short, -o] (5,-0.5)node[above]{$U_\mathrm{a}$};;
\draw (3,-0.5) 
    to[R,l=$R_\mathrm{f}$, -*] (3,-4) 
    to[R,l=$R_\mathrm{n1}$](3,-6.5) 
    to[R,l=$R_\mathrm{n2}$](3,-9)node[ground]{};
\draw (opamp.+) 
    to[short, -o] (-3,0)node[above] {$U_\mathrm{e}$};
\draw (opamp.-) -- (0,-3) 
    to[short, -*] (0,-4) -- (0,-6.5) 
    to[normal open switch,l=$S2$, -*] (3,-6.5);
\draw (0,-4) 
   to[normal open switch, l=$S1$] (3,-4);

   \draw (-3,-3) to[short, o-] (-2.5,-3) -- (-2.5,-9)node[ground]{};
\end{tikzpicture}

        \label{fig:FigOPVschaltbar1}
    \end{figure}

    \begin{itemize}
        \item[ \bf b)] Für die Schaltung in Abbildung 2 (2 Schalterstellungen)
    \end{itemize}

    \begin{figure}[H]
        \centering
        \begin{tikzpicture}
\draw
(0,0) node[op amp, anchor=+] (opamp) {}
(opamp.+) 
(opamp.-) 
(opamp.out) ;

\draw (opamp.-) 
    to[short, -*] (-1,1);
\draw (-4,1)node[above] {$U_\mathrm{e}$}  to[ R, l=$R_\mathrm{n}$, o-]
    (-1,1);
\draw (-1,1) -- (-1,2.5)
    to[R, l=$R_\mathrm{f1}$, -*] (1,2.5)
    to[R, l=$R_\mathrm{f2}$, -*] (4,2.5)
    to[short, -*] (4,0.5);
\draw (opamp.out) to[short, -o] (5,0.5) node[above]{$U_\mathrm{a}$};
\draw (1,2.5) -- (1,4) 
    to[normal open switch, l=$S1$] (4,4)
    -- (4,2.5);
\draw (opamp.+) -- (0,-2)node[ground]{};

\draw (-4,-1) to[short, o-] (-3.5,-1) -- (-3.5,-2)node[ground]{};
\end{tikzpicture}

        \label{fig:FigOPVschaltbar2}
    \end{figure}

%    \speech{
%        Aufgabe 13: Verstärkungseinstellung mit Schaltern  
%        
%        In dieser Aufgabe werden Operationsverstärkerschaltungen analysiert,  
%        bei denen die Verstärkung durch das Zu- und Abschalten von Widerständen verändert werden kann.  
%        
%        Teilaufgabe a:
%        Analyse der Schaltung in Abbildung 1 mit vier möglichen Schalterstellungen.  
%        Die Schaltung enthält einen Operationsverstärker mit mehreren Widerständen.  
%        Der Widerstand Rf ist fest in der Rückkopplung, während die Widerstände Rn1 und Rn2 über die Schalter S1 und S2  
%        zugeschaltet oder abgeschaltet werden können.  
%        Je nach Stellung der Schalter verändert sich das Widerstandsverhältnis und somit die Verstärkung.  
%        Ziel ist es, die Verstärkungswerte für die vier möglichen Schalterstellungen zu berechnen.  
%        
%        Teilaufgabe b:
%        Analyse der Schaltung in Abbildung 2 mit zwei möglichen Schalterstellungen.  
%        Hier wird die Verstärkung durch den Schalter S1 umgeschaltet.  
%        Dieser wählt zwischen zwei Rückkopplungswiderständen Rf1 und Rf2.  
%        Die Aufgabe besteht darin, die Verstärkung für beide Schalterstellungen zu bestimmen.  
%        
%        Abbildung 13.1 zeigt eine nichtinvertierende Operationsverstärkerschaltung,  
%bei der die Verstärkung durch das Ein- und Ausschalten von Widerständen variiert werden kann.  
%
%Die Schaltung besteht aus einem Operationsverstärker mit folgender Beschaltung:  
%
%- Der Eingang U E wird an den nichtinvertierenden Eingang des Operationsverstärkers angelegt.  
%- Der Ausgang U A ist mit einer Rückkopplung über den Widerstand Rf verbunden.  
%- Am invertierenden Eingang befinden sich zwei Schalter, S1 und S2,  
%  die verschiedene Widerstände Rn1 und Rn2 parallel schalten können.  
%- Diese Widerstände sind mit Masse verbunden,  
%  sodass sich je nach Schalterstellung der Gesamtwiderstand und damit die Verstärkung ändert.  
%
%Die Schalterstellung beeinflusst die effektive Gegenkopplung des Operationsverstärkers.  
%Wenn beide Schalter geöffnet sind, fließt kein zusätzlicher Strom über Rn1 und Rn2.  
%Wenn einer oder beide Schalter geschlossen werden,  
%ändert sich der Widerstand und somit auch die Verstärkung der Schaltung.  
%
%Diese Schaltung ermöglicht vier verschiedene Verstärkungswerte,  
%abhängig von der Kombination der Schalterstellungen. 
%
%Abbildung 13.2 zeigt eine nichtinvertierende Operationsverstärkerschaltung,  
%bei der die Verstärkung durch einen einzigen Schalter S1 umgeschaltet werden kann.  
%
%Die Schaltung besteht aus folgenden Komponenten:  
%
%- Der Eingang U E ist über einen Widerstand Rn mit Masse verbunden  
%  und mit dem nichtinvertierenden Eingang des Operationsverstärkers verbunden.  
%- Der invertierende Eingang ist über zwei Widerstände Rf1 und Rf2 mit dem Ausgang U A verbunden.  
%- Der Schalter S1 ermöglicht es, zwischen zwei verschiedenen Rückkopplungswiderständen Rf1 und Rf2 umzuschalten.  
%
%Je nach Stellung des Schalters S1 wird entweder:  
%
%1. Nur der Widerstand Rf1 in die Rückkopplung einbezogen,  
 %  was zu einer bestimmten Verstärkung führt.  
%2. Der Widerstand Rf2 parallel zu Rf1 geschaltet,  
%   wodurch sich der Gesamtwiderstand verringert und die Verstärkung entsprechend ändert.  
%
%Diese Schaltung bietet zwei verschiedene Verstärkungswerte,  
%die durch einfaches Umschalten des Schalters S1 angepasst werden können.  
%        }
        

}

\Loesung{
    \begin{itemize}
        \item[a)] Nicht-invertierender Verstärker
        
        Die Verstärkung lässt sich durch zwei Schalter (S1 und S2) verändern. Dadurch ergeben sich insgesamt vier mögliche Kombinationen:

        \textbf{1. Fall (S1 geschlossen, S2 offen):}  
        In diesem Zustand bilden die Widerstände $R_\mathrm{n1}$ und $R_\mathrm{n2}$ eine Serienschaltung. Die Gesamtverstärkung ergibt sich somit zu:
        \[
        V = 1 + \frac{R_\mathrm{f}}{R_\mathrm{n1} + R_\mathrm{n2}}
        \]

        \textbf{2. Fall (S1 offen, S2 geschlossen):}  
        In diesem Fall sind der Widerstand $R_\mathrm{f}$ und $R_\mathrm{n1}$ seriell geschaltet. Diese Serienschaltung bildet zusammen mit dem Widerstand $R_\mathrm{n2}$ die Rückkopplungsschleife. Die Verstärkung ist daher:
        \[
        V = 1 + \frac{R_\mathrm{f} + R_\mathrm{n1}}{R_\mathrm{n2}}
        \]

        \textbf{3. Fall (S1 und S2 geschlossen):}  
        Mit beiden Schaltern geschlossen wird der Widerstand $R_\mathrm{n1}$ überbrückt und somit nicht berücksichtigt. Die Verstärkung ergibt sich durch:
        \[
        V = 1 + \frac{R_\mathrm{f}}{R_\mathrm{n2}}
        \]

        \textbf{4. Fall (S1 und S2 offen):}  
        Wenn beide Schalter offen sind, fehlt der Rückkopplungspfad vollständig, womit die Verstärkung theoretisch unendlich groß ist:
        \[
        V = \infty
        \]

        \item[b)] Invertierender Verstärker
        \textbf{1. Zustand (S1 offen):}  
        Sind $R_\mathrm{f1}$ und $R_\mathrm{f2}$ seriell geschaltet, ergibt sich eine größere Rückkopplung und somit eine höhere Verstärkung. Die Verstärkung errechnet sich wie folgt:
        \[
        V = -\frac{R_\mathrm{f1} + R_\mathrm{f2}}{R_\mathrm{n}}
        \]

        \textbf{2. Zustand (S1 geschlossen, S2 offen):}  
        In diesem Zustand wird der Widerstand $R_\mathrm{f2}$ durch den geschlossenen Schalter kurzgeschlossen und nicht berücksichtigt. Die Verstärkung reduziert sich damit auf:
        \[
        V = -\frac{R_\mathrm{f1}}{R_\mathrm{n}}
        \]
    \end{itemize}

 %   \speech{
%        Lösung für Aufgabe 13.
%        
%       Die Aufgabe beschreibt zwei verschiedene Operationsverstärkerschaltungen mit einstellbarer Verstärkung: einen nicht-invertierenden Verstärker und einen invertierenden Verstärker.
%        
%        Teil a: Nicht-invertierender Verstärker
%        
%        Die Schaltung enthält zwei Schalter S1 und S2, die in verschiedenen Kombinationen geschaltet werden können. Je nach Schalterstellung ergibt sich eine unterschiedliche Verstärkung.
%        
%        1. Wenn Schalter S1 geschlossen und Schalter S2 offen ist, ergibt sich die Verstärkung als V gleich 1 plus Klammer auf Rf geteilt durch Klammer auf Rn1 plus Rn2 Klammer zu Klammer zu.
%        
%        2. Wenn Schalter S1 offen und Schalter S2 geschlossen ist, ergibt sich die Verstärkung als V gleich 1 plus Klammer auf Rf geteilt durch Rn1 Klammer zu.
%        
%        3. Wenn Schalter S1 und Schalter S2 geschlossen sind, ergibt sich die Verstärkung als V gleich 1 plus Klammer auf Rf geteilt durch Rn2 Klammer zu.
%        
%        4. Wenn beide Schalter offen sind, ist die Verstärkung theoretisch unendlich, da der Widerstand nicht vorhanden ist.
%        
%        Teil b: Invertierender Verstärker
%        
%        Diese Schaltung besitzt zwei Zustände, die über einen einzelnen Schalter bestimmt werden.
%        
%        1. Wenn Schalter S1 offen ist, ergibt sich die Verstärkung als V gleich minus Klammer auf Rf plus Rf2 Klammer zu geteilt durch Rn.
%        
%        2. Wenn Schalter S2 offen ist, ergibt sich die Verstärkung als V gleich minus Klammer auf Rf geteilt durch Rn.
%        
%        Zusammenfassung:
%        
%        Durch das Ein- und Ausschalten der Schalter kann die Verstärkung der Schaltungen verändert werden, um verschiedene Verstärkungswerte zu erreichen. Die nicht-invertierende Schaltung bietet vier verschiedene Verstärkungen, während die invertierende Schaltung zwei mögliche Verstärkungswerte hat.
%        }
Siehe: Abschnitt \ref{fig: Verschaltung Verstaerker} und Tabelle \ref{tab:Grundschaltungen}
}