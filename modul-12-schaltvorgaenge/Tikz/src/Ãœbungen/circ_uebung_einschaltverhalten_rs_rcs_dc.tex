% u1 über Serienschaltung aus R1 + R2 + C
% u2 über R2 + C
% Uq=konst. über Schalter ab t=0 an Klemmen von u1
% i durch R1
\begin{tikzpicture}
    % Uq über Schalter (ab t=0) an R1 + R2 + C in Reihe
    \draw (0,0) 
    to [vsource] (0,3) % Uq Spannungsquelle
    -- (0.5,3)
    to [switch, l={$t=0$}, -o] (1,3)
    to [R, name=R1, l=$R_1$, i, !vi, *-*] (3.5,3)
    to[R, name=R2, l_=$R_2$, !vi, *-] (3.5,1.5)
    to[C, name=C, l_=$C$, v^, !vi, -*] (3.5,0)
    to [short,*-o] (1,0)
    -- (0,0);

    % u1 Spannungspfeil
    \draw (1,3)
    to [open, name=u1, v^, !vi] (1,0);
    
    % u2 Spannungspfeil und Klemmen
    \draw (3.5,3)
    to [short, -o] (5,3)
    to [open, name=u2, v^, !vi] (5,0)
    to [short, o-] (3.5,0);

    % Uq nur Spannungspfeil
    \draw (-0.5,2.5)
    to [open, name=Uq, v, !vi] (-0.5,0.5); 

    % Spannungs- und Strompfeile, farbig mit label
    \varrmore{Uq}{$U_q$};
    \varrmore{u1}{$u_1$};
    \varrmore{u2}{$u_2$};
    \varrmore{C}{$u_C$};
    \iarrmore{R1}{$i$};
\end{tikzpicture}
