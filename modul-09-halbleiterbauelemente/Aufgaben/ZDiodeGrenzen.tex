\subsection{Ermittlung der Betriebsgrenzen einer Zenerdiode\label{Aufg7}}
% Alt: Aufgabe 7
\Aufgabe{
  Die eingesetzte Zenerdiode besitzt eine Zenerspannung von \( U_\mathrm{Z} = \mathrm{8\,V} \). 
  Die maximale Verlustleistung der Diode beträgt \( P_\mathrm{tot}=\mathrm{8\,W} \). 
  Die Diode arbeitet im stabilisierenden Bereich, wenn der Strom durch sie mindestens \( \mathrm{5\,mA} \) beträgt. \\
  Welchen minimalen und maximalen Widerstandswert darf der Lastwiderstand \( R_\mathrm{L} \) nicht unter- bzw. überschreiten, damit die Diode einerseits im stabilisierenden Bereich arbeitet und andererseits nicht überlastet wird?
  
  \begin {figure} [H]
     \centering
     \begin{tikzpicture}
  \draw(0,2) 
    to[V, name=V1, v] (0,0)
    to[short, -o](5,0) -- (7,0)
    to[R, l_=$R_\mathrm{L}$] (7,2)
    to[short, -o](5,2)
    to[short] (4,2)
    to[R, l_=$R_\mathrm{V}$] (0,2);

  \draw(4,0) 
  to[zDo=$D_\mathrm{Z}$, *-*] (4,2) ;

  \draw (5,2) 
  to[open, name=ua, v^] (5,0);
  
  \varrmore{ua}{$U_\mathrm{A}$};
  \varrmore{V1}{$U_\mathrm{E}$};
\end{tikzpicture}
     \label{fig:FigZDiodeGrenzen}
  \end {figure}

%   \speech{ **Aufgabe 7**  
% Die eingesetzte **Zenerdiode** besitzt eine **Zenerspannung von \( U_Z = 8V \)**.  
% - Die **maximale Verlustleistung** der Diode beträgt **8 W**.  
% - Die Diode arbeitet **im stabilisierenden Bereich**, wenn der **Strom durch die Diode größer als 5 mA** ist.  

% **Fragestellung:**  
% - **Welchen minimalen und maximalen Widerstandswert darf der Lastwiderstand \( R_L \) nicht unter- oder überschreiten**, damit die **Diode im stabilisierenden Bereich arbeitet** und nicht überlastet wird?  
%  **Beschreibung der Abbildung**  
% Die gezeigte Schaltung stellt eine **Zener-Spannungsstabilisierungsschaltung** dar und besteht aus folgenden Komponenten:

% 1. **Eingangsspannungsquelle**  
%    - **Nennwert: 20V** mit einer **Toleranz von ±10%**, also zwischen **18V und 22V**.  
%    - Symbolisiert durch einen **Kreis mit Richtungspfeil**.  

% 2. **Serienwiderstand (10Ω)**  
%    - Platziert im **oberen Zweig der Schaltung**.  
%    - Begrenzung des **Gesamtstroms** in der Schaltung.  

% 3. **Zenerdiode \( D_Z \)**  
%    - In **Sperrrichtung** betrieben.  
%    - Hält die **Spannung über \( R_L \) stabil bei 8V**.  
%    - Beginnt **zu leiten**, wenn die Spannung über **8V steigt** und sorgt für eine **Überspannungsbegrenzung**.  

% 4. **Lastwiderstand \( R_L \)**  
%    - Parallel zur **Zenerdiode** geschaltet.  
%    - Soll mit einer **stabilen Spannung von 8V** betrieben werden.  
%    - Der zulässige **Widerstandsbereich für \( R_L \)** ist zu bestimmen, damit die **Diode weder zu wenig noch zu viel Strom führt**.  

%  **Funktionsweise der Schaltung**  
% - Die **Zenerdiode regelt die Ausgangsspannung auf 8V** unabhängig von **Schwankungen der Eingangsspannung**.  
% - Die **minimale Last \( R_L \) darf nicht zu klein** sein, damit die **Diode noch genug Strom aufnimmt**.  
% - Die **maximale Last \( R_L \) darf nicht zu groß** sein, da sonst die **Zenerdiode zu viel Strom aufnehmen müsste**, was zu einer **Überlastung (8W-Grenze) führen könnte**.  
% - Das **Problem wird durch eine Berechnung der Stromverhältnisse** gelöst.  

% **Zusammenfassung der Aufgabe**  
% - Die Aufgabe erfordert die **Bestimmung des minimalen und maximalen Widerstands \( R_L \)**, damit die **Zenerdiode richtig arbeitet**.  
% - Die **Spannungsschwankungen der Quelle** müssen berücksichtigt werden.  
% - Die Berechnung erfolgt unter der **Bedingung, dass der Strom durch die Zenerdiode größer als 5mA bleibt** und die **maximale Leistung von 8W nicht überschritten wird**.  
% }

}


\Loesung{
  \begin{figure}[H]
    \centering
    \begin{tikzpicture}
    \draw(0,2)
    to[V, name=V1, v] (0,0)
    to[short, -o](5,0) -- (7,0)
    to[R, l_=$R_\mathrm{L}$, v^<, name=RL] (7,2)
    to[short, -o](5,2)
    to[short, i<_, name=IL] (4,2)
    to[R, l_=$R_\mathrm{V}$, i<_, name=RV] (0,2);
    
    \draw(4,0)
    to[zDo=$D_\mathrm{Z}$, *-*, i_<, name=D] (4,2) ;
    
    \draw (5,2)
    to[open, name=ua, v^] (5,0);
    
    \varrmore{ua}{$U_\mathrm{A}$};
    \varrmore{V1}{$U_\mathrm{E}$};
    \iarrmore{RV}{${I}$};
    \iarrmore{D}{$I_\mathrm{D}$};
    \iarrmore{IL}{$I_\mathrm{L}$};
    
\end{tikzpicture}
    \label{fig:LsgZDiodeGrenzen}
  \end{figure}

  Gegeben sind:
\begin{align*}
    U_\mathrm{max} &= \mathrm{22\,V} \\
    U_\mathrm{min} &= \mathrm{18\,V}
\end{align*}

\textbf{Fall 1:} Bei \( U_\mathrm{max} = \mathrm{22\,V} \)
\begin{align*}
    U_\mathrm{RV} &= \mathrm{22\,V} - \mathrm{8\,V} = \mathrm{14\,V} \\
    I &= \frac{\mathrm{14\,V}}{\mathrm{10\,\Omega}} = \mathrm{1{,}4\,A} \\
    I &= I_\mathrm{D} + I_\mathrm{L} \\
    I_\mathrm{D,\ max} &= \frac{\mathrm{8\,W}}{\mathrm{8\,V}} = \mathrm{1\,A} \\
    I_\mathrm{L} &= \mathrm{1{,}4\,A} - \mathrm{1\,A} = \mathrm{0{,}4\,A} \\
    R_\mathrm{L} &= \frac{\mathrm{8\,V}}{\mathrm{0{,}4\,A}} = \mathrm{20\,\Omega} \Rightarrow R_\mathrm{L,\ max} = \mathrm{20\,\Omega}
\end{align*}

\textbf{Fall 2:} Bei \( U_\mathrm{min} = \mathrm{18\,V} \)
\begin{align*}
    U_\mathrm{RV} &= \mathrm{18\,V} - \mathrm{8\,V} = \mathrm{10\,V} \\
    I &= \frac{\mathrm{10\,V}}{\mathrm{10\,\Omega}} = \mathrm{1\,A} \\
    I_\mathrm{D,\ min} &= \mathrm{5\,mA} \\
    I_\mathrm{L} &= \mathrm{1\,A} - \mathrm{5\,mA} = \mathrm{0{,}995\,A} \\
    R_\mathrm{L} &= \frac{\mathrm{8\,V}}{\mathrm{0{,}995\,A}} \approx \mathrm{8\,\Omega} \Rightarrow R_\mathrm{L,\ min} \approx \mathrm{8\,\Omega}
\end{align*}
%   \speech{
% Lösung zu Aufgabe 7:

% Gegeben ist eine Schaltung mit einer Spannungsquelle von 20 Volt plus-minus 10 Prozent, einem Vorwiderstand von 10 Ohm, einer Zenerdiode mit einer Zenerspannung von 8 Volt und einem variablen Lastwiderstand R L. Ziel ist es, die minimalen und maximalen Werte für R L zu bestimmen, sodass die Zenerdiode im stabilisierenden Bereich arbeitet, aber nicht überlastet wird.

% Betrachtung für die maximale Eingangsspannung U max gleich 22 Volt:
% - Die Spannung über dem Vorwiderstand beträgt:
%   U R V gleich 22 Volt minus 8 Volt gleich 14 Volt.
% - Der Gesamtstrom durch den Vorwiderstand ist:
%   I gleich U R V geteilt durch R V.
%   I gleich 14 Volt geteilt durch 10 Ohm gleich 1,4 Ampere.
% - Der Gesamtstrom teilt sich auf in den Strom durch die Zenerdiode I D und den Laststrom I L:
%   I gleich I D plus I L.
% - Die maximale Verlustleistung der Zenerdiode ist gegeben mit 8 Watt. Der maximale Strom durch die Zenerdiode ergibt sich durch:
%   I D max gleich P max geteilt durch U Z.
%   I D max gleich 8 Watt geteilt durch 8 Volt gleich 1 Ampere.
% - Daraus ergibt sich der Laststrom:
%   I L gleich I minus I D max.
%   I L gleich 1,4 Ampere minus 1 Ampere gleich 0,4 Ampere.
% - Die Spannung über dem Lastwiderstand beträgt:
%   U L gleich U Z gleich 8 Volt.
% - Der maximale Lastwiderstand ist dann:
%   R L max gleich U L geteilt durch I L.
%   R L max gleich 8 Volt geteilt durch 0,4 Ampere gleich 20 Ohm.

% Betrachtung für die minimale Eingangsspannung U min gleich 18 Volt:
% - Die Spannung über dem Vorwiderstand beträgt:
%   U R V gleich 18 Volt minus 8 Volt gleich 10 Volt.
% - Der Gesamtstrom durch den Vorwiderstand ist:
%   I gleich U R V geteilt durch R V.
%   I gleich 10 Volt geteilt durch 10 Ohm gleich 1 Ampere.
% - Der Gesamtstrom teilt sich wieder auf in den Strom durch die Zenerdiode I D und den Laststrom I L.
% - Die minimale Stromaufnahme der Zenerdiode, damit sie im stabilisierenden Bereich arbeitet, beträgt 5 Milliampere.
% - Daraus ergibt sich der Laststrom:
%   I L gleich I minus I D min.
%   I L gleich 1 Ampere minus 5 Milliampere gleich 995 Milliampere.
% - Die Spannung über dem Lastwiderstand beträgt:
%   U L gleich 8 Volt.
% - Der minimale Lastwiderstand ist dann:
%   R L min gleich U L geteilt durch I L.
%   R L min gleich 8 Volt geteilt durch 0,995 Ampere gleich 8 Ohm.

% Zusammenfassung:
% - Der Lastwiderstand R L muss zwischen 8 Ohm und 20 Ohm liegen, damit die Zenerdiode im stabilisierenden Bereich bleibt und nicht überlastet wird.
% }
Siehe: Abschnitt \ref{sec:SpezielleDioden}
}



