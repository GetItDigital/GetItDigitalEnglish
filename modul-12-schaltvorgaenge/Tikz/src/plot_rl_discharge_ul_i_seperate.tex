% Aufladevorgang RL Strom und Spannung
% AxisEnv 1: Spannung an der Spule
% AxisEnv 2: Strom an der Spule
\begin{tikzpicture}[x=1cm,y=1cm]
	\begin{axis}[
		name=AxisSpannung,
		%%%%%%% START SAME OPTIONS %%%%%%%
		xmin=-1, xmax=5.5,
		ymin=-1.125, ymax=0.35,
		xlabel={$t$},
		ylabel={$u(t)$},
		axis x line=center,
		axis y line=center,
		ylabel style={anchor=east, at={(ticklabel cs:0.95)}},
		ticks=none,
		mark=none,
		grid=none,
		width=7cm,
		height=3.5cm,
		%%%%%%% END OF SAME OPTIONS %%%%%%%
	]
		\node at (2,-1)[black,anchor=south west,yshift=-1pt]{$-U_q$};			% u_min label
		\addplot[color=black,dashed,	domain=-1:5.4] {-1};          			% u_min Grenze
		\addplot[color=voltage,very thick,	domain=-1:0] {0};           		% t<0
		\addplot[color=voltage,very thick,	domain=0:5.4,samples=61] {-e^(-x)};	% t>0 
		\addplot[color=voltage, very thick] coordinates {(0,-1)(0,0)}; 			% Spannungssprung
		\node at (1,-0.6)[voltage,anchor=west]{$u_L$};							% u_L label
		%\addplot[color=black,dashed, domain=0:1]{x}; % Gradient in t=0, schneidet Asymptote bei t=tau
	\end{axis}
	\begin{axis}[
		at={($(AxisSpannung.south west)+(0,-0.25cm)$)}, % Position 2. Plot
		anchor=north west,
		name=AxisStrom,
		%%%%%%% START SAME OPTIONS %%%%%%%
		xmin=-1, xmax=5.5,
		ymin=-0.125, ymax=1.35,
		xlabel={$t$},
		ylabel={$i(t)$},
		axis x line=center,
		axis y line=center,
		ylabel style={anchor=east, at={(ticklabel cs:0.95)}},
		ticks=none,
		mark=none,
		grid=none,
		width=7cm,
		height=3.5cm,
		%%%%%%% END OF SAME OPTIONS %%%%%%%
		clip=false,
		]
		\node at (2,1)[black,anchor=south west,yshift=-1pt]{$U_q/R$};			% i_max label
		\addplot[color=black,dashed,	domain=-1:5.4] {1};          			% i_max Grenze
        \addplot[color=red,very thick,	domain=0:5.4,samples=61] {e^(-x)}; 		% t>0 
        \addplot[color=red,very thick,	domain=-1:0] {1};  						% t<0
		\node at (1,0.5)[red,anchor=west]{$i$}; 								% i label
	\end{axis}
\end{tikzpicture}