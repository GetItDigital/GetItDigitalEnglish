% RLC Schwingkreis Parallel
\begin{tikzpicture}
    %  invisible boundbox
    \draw[draw=none](-3.75,-0.25) rectangle (+3.75,+3.25); % like RLCs: (7.5 x 2.75) % set draw=black (debug) or draw=none (final)
        
    % draw RLC-Glied
    \draw (0,3.0) to [short, name=I, i>^=$\underline{I}$, o-*] (0,2.5); % U+
    \draw (0,0.5) to [short, *-o] (0,0); % U-

    \draw (0,2.5)
    to [short, *-] (-1.5,2.5)
    to [R, name=R, l=$R$, i>^=$\underline{I}_R$, !i] (-1.5,0.5)
    to [short, -*] (0,0.5);

    \draw (0,2.5)
    to [L, name=L, l=$\mathrm{j}\omega L$, i>^=$\underline{I}_L$, !i, *-*] (0,0.5);

    \draw (0,2.5)
    to [short, *-] (+1.5,2.5)
    to [bipole label style={yshift=+5pt},C, l=$\dfrac{1}{\mathrm{j}\omega C}$, name=C, i>^=$\underline{I}_C$, !i] (+1.5,0.5)
    to [short, -*] (0,0.5); 

    % draw voltage arrows [v^=...] above
    \draw (3,3) to [open, name=U, v^=$\underline{U}$, !v] (3,0);
    \varronly{U};

    % draw arrow [i>^=...] incoming above
    \iarronly{I};
    \iarronly{C};
    \iarronly{L};
    \iarronly{R};

    % Iq Stromquelle
    \draw (0,0) 
    to [short, *-] (-2.75,0)
    to [isource, name=Iq, i=$\underline{I}_q$, !vi] (-2.75,3)
    to [short, -*] (0,3);
    \iarronly{Iq};
\end{tikzpicture}