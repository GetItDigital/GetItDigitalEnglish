\begin{circuitikz}[scale=0.9]
    % Zeichne den offenen Kreis
    \draw[line width=1pt] (0,-0.04) arc[start angle=10,end angle=170,radius=0.42];
    \draw[line width=1pt] (0,-0.19) arc[start angle=-10,end angle=-170,radius=0.42];

    % Zeichne den Pfeil in der Mitte des offenen Kreises
    \draw[-latex, thick, line width=1.1pt] (-0.42,-0.45)    to node[right] {} (-0.42,0.25);

    % Anschlüsse
    \draw (-0.42,0.3) -- (-0.42,1.4) to[short, -o] (1.08,1.4)
    (-0.42,-1.6)to[short, -o] (1.08,-1.6)
    (-0.42,0.8)                to[short, color=current, i_={\textcolor{current}{$i(t)$}}]
    (-0.42,0.9);
    \draw (-0.42,-0.55) -- (-0.42,-1.6);

\end{circuitikz}