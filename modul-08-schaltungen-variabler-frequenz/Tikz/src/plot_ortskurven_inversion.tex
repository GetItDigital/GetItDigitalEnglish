% Inversion eines orthognalen Gitters in der komplexen Ebene
% durch Kehrwertbildung zu Kreisen in der komplexen Ebene vice versa
%
% Geraden bei 
% Im-Teil von -2 (rot gestrichelt), -1 (rot), +1 (blau) und +2 (blau gestrichelt) in horizontal
% Re-Teil von -2 (orange gestrichelt), -1 (orange), +1 (grün) und +2 (grün gestrichelt) in vertikal
% alle mit Re- respektive Im-Teil variabel von -\infty bis +\infty (Pfeilspitze)
%
% Inversion durch Kehrwert -> Kreise 
%
% Kreise mit Radius 1 (nicht gestrichelt) und  Radius 0.5 (gestrichelt)
% rot von (0,0) gegen den Uhrzeigersinn über (0,-1) nach (0,0)
% blau von (0,0) mit dem Uhrzeigersinn über (0,1) nach (0,0)
% orange von (0,0) mit Uhrzeigersinn über (1,0) nach (0,0)
% grün von (0,0) gegen den Uhrzeigersinn über (-1,0) nach (0,0)
\begin{tikzpicture}
\begin{axis}[
	name=plotgeraden,
	%xlabel={$\Re\{\underline{z}\}$},
	%ylabel={$\Im\{\underline{z}\}$},
	axis lines=center,
	xmin=-2.5, xmax=+2.5,
	ymin=-2.5, ymax=+2.5,
	xtick={-2,-1,1,2},xticklabels={,,$1$,},
	ytick={-2,-1,1,2},yticklabels={,,$\mathrm{j}$,},
	width=6cm,
	height=6cm,
]
	% Achsen
	\draw[purple] (axis cs:-2.5,0) -- (axis cs:2.5,0);
	\draw[brown] (axis cs:0,-2.5) -- (axis cs:0,2.5);
	% horizontale Geraden
	\draw[->,blue] (axis cs:-2.5,-1) -- (axis cs:2.5,-1);
	\draw[->,blue,dashed] (axis cs:-2.5,-2) -- (axis cs:2.5,-2);
	\draw[->,red] (axis cs:-2.5,1) -- (axis cs:2.5,1);
	\draw[->,red,dashed] (axis cs:-2.5,2) -- (axis cs:2.5,2);
	% vertikale Geraden
	\draw[->,plotgreen] (axis cs:-1,-2.5) -- (axis cs:-1,2.5);
	\draw[->,plotgreen,dashed] (axis cs:-2,-2.5) -- (axis cs:-2,2.5);
	\draw[->,orange] (axis cs:1,-2.5) -- (axis cs:1,2.5);
	\draw[->,orange,dashed] (axis cs:2,-2.5) -- (axis cs:2,2.5);
\end{axis}
\draw[<->] ($(plotgeraden.east)+(5mm,0)$) -- ($(plotgeraden.east)+(15mm,0)$);
\node[above] at ($(plotgeraden.east)+(10mm,0)$) {$\frac{1}{(\,)}$};
\node[below] at ($(plotgeraden.east)+(10mm,0)$) {\small Invers.};
\begin{axis}[
	at={($(plotgeraden.south east)+(2cm,0)$)},
	name=plotkreise,
	%xlabel={$\Re\{\underline{z}^{-1}\}$},
	%ylabel={$\Im\{\underline{z}^{-1}\}$},
	axis lines=center,
	xmin=-2.5, xmax=+2.5,
	ymin=-2.5, ymax=+2.5,
	xtick={-2,-1,1,2},xticklabels={,,,$1$},
	ytick={-2,-1,1,2},yticklabels={,,,$\mathrm{j}$},
	width=6cm,
	height=6cm,
]
% Achsen
\draw[purple] (axis cs:-2.5,0) -- (axis cs:2.5,0);
\draw[brown] (axis cs:0,-2.5) -- (axis cs:0,2.5);
% Kreise der horizontalen Geraden
\draw[blue] (axis cs:0,1) circle [radius=1];
\draw[blue, dashed] (axis cs:0,0.5) circle [radius=0.5];
\draw[->,blue] (axis cs:0.1,+0.01) -- (axis cs:0,0);
\draw[red] (axis cs:0,-1) circle [radius=1]; 
\draw[red, dashed] (axis cs:0,-0.5) circle [radius=0.5];
\draw[->,red] (axis cs:0.1,-0.01) -- (axis cs:0,0);
% Kreise der vertikalen Geraden
\draw[plotgreen] (axis cs:-1,0) circle [radius=1];
\draw[plotgreen, dashed] (axis cs:-0.5,0) circle [radius=0.5];
\draw[->,plotgreen] (axis cs:-0.01,-0.1) -- (axis cs:0,0);
\draw[orange] (axis cs:1,0) circle [radius=1];
\draw[orange, dashed] (axis cs:0.5,0) circle [radius=0.5];
\draw[->,orange] (axis cs:+0.01,-0.1) -- (axis cs:0,0);
\end{axis}
\end{tikzpicture}