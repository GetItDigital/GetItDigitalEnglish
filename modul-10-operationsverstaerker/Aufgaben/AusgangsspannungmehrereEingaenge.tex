\subsection{Berechnung der Ausgangsspannung mit mehreren Eingangsspannungen\label{Aufg14}}
% Alt: Aufgabe 14
\Aufgabe{
    Gegeben ist die abgebildete Schaltung mit einem idealen Operationsverstärker. Die Betriebsspannung 
beträgt $\pm 15\,\mathrm{V}$.
\begin{itemize}
    \item[a)] Die Widerstände sind alle gleich, es gilt $R_1 = R_2 = R_3 = R_\mathrm{F} = \mathrm{1\,k\Omega}$. 
    Die Spannung $U_\mathrm{E1} = \mathrm{1\,V}$, $U_\mathrm{E2} = \mathrm{2\,V}$, $U_\mathrm{E3} = \mathrm{3\,V}$. 
    Wie hoch ist die Spannung am Ausgang $U_\mathrm{A}$?
    \item[b)] Die Eingangsspannungen bleiben gleich groß, nur die Widerstände betragen nun 
    $R_\mathrm{F} = \mathrm{10\,k\Omega}, R_1=\mathrm{2\,k\Omega}, R_2=\mathrm{5\,k\Omega}, R_3=\mathrm{10\,k\Omega}$. 
    Wie hoch ist die Spannung am Ausgang U$_A$?
    \item[c)] Die Eingangsspannungen bleiben gleich groß, nur die Widerstände betragen nun 
    $R_\mathrm{F} = \mathrm{10\,k\Omega}, R_1 = \mathrm{10\,k\Omega}, R_2 = \mathrm{5\,k\Omega}, R_3 = \mathrm{2\,k\Omega}$. 
    Wie hoch ist die Spannung am Ausgang $U_\mathrm{A}$?
\end{itemize}
\begin {figure} [H]
   \centering
   \begin{tikzpicture}
\ctikzset{tripoles/en amp/input height=0.45}
\draw (0,0)node[en amp, en amp text A](E){}
    (E.out)
    (E.-)
    (E.+);
    \draw (E.out) to[short, -*] (2,0) to[short, -o] (2.5,0);
    \draw (E.-) to[short, -*] (-2,0.5) to[short, -*] (-3,0.5) to[short, -*] (-3,1.5) -- (-3,2.5)
    to[R,l_=$R_\mathrm{1}$] (-5,2.5) to[short, -o] (-7,2.5);
    \draw (-2,0.5) to[short] (-2,2.5) to[R, name=R, l=$R_\mathrm{F}$] (2,2.5) -- (2,0);
    \draw (-3,1.5) to[R, l_=$R_\mathrm{2}$] (-5,1.5) to[short, -o] (-6,1.5);
    \draw (-3,0.5) to[R,l_=$R_\mathrm{3}$, -o] (-5,0.5);

    
    \draw (-7,-2) to[short, o-o] (-6,-2) to[short, -o](-5,-2);
    \draw (E.+) -- (-2,-0.5) to[short, -*] (-2,-2)node[ground]{};
    \draw (2.5,-2) to[short, o-o] (-5,-2);
    
    \draw (2.5,0) to[open, v, name=ua] (2.5,-2);
    \draw (-7,2.5) to[open, v, name=ue1] (-7,-2);
    \draw (-6,1.5) to[open, v, name=ue2] (-6,-2);
    \draw (-5,0.5) to[open, v, name=ue3] (-5,-2);
    
    

    \varrmore{ue1}{$U_\mathrm{E1}$};
    \varrmore{ue2}{$U_\mathrm{E2}$};
    \varrmore{ue3}{$U_\mathrm{E3}$};
    \varrmore{ua}{$U_\mathrm{A}$};
\end{tikzpicture}

   \label{fig:FigAusgangsspannungmehrereEingaenge}
\end {figure}
%\speech{
%    Aufgabe 14: Operationsverstärkerschaltung mit mehreren Eingängen  
%    
%    Gegeben ist eine Schaltung mit einem idealen Operationsverstärker.  
%    Die Betriebsspannung beträgt plus minus 15 Volt.  
%    
%    Die Schaltung besteht aus mehreren Eingangsspannungen U E1, U E2 und U E3,  
%    die über die Widerstände R1, R2 und R3 an den invertierenden Eingang des Operationsverstärkers angeschlossen sind.  
%    Ein Rückkopplungswiderstand R F verbindet den Ausgang mit dem invertierenden Eingang.  
%    Der nichtinvertierende Eingang ist mit Masse verbunden.  
%    
%    Es sind drei Teilaufgaben zu lösen:  
%    
%    a) Berechnung der Ausgangsspannung U A, wenn alle Widerstände gleich sind:  
%       R1 = R2 = R3 = R F = 1 Kiloohm,  
%       und die Eingangsspannungen U E1 = 1 Volt, U E2 = 2 Volt, U E3 = 3 Volt.  
%    
%    b) Berechnung der Ausgangsspannung U A, wenn die Eingangsspannungen gleich bleiben,  
%       aber die Widerstände unterschiedliche Werte haben:  
%       R F = 10 Kiloohm, R1 = 2 Kiloohm, R2 = 5 Kiloohm, R3 = 10 Kiloohm.  
%    
%    c) Berechnung der Ausgangsspannung U A, wenn die Eingangsspannungen gleich bleiben,  
%       aber die Widerstände folgende Werte haben:  
%       R F = 10 Kiloohm, R1 = 10 Kiloohm, R2 = 5 Kiloohm, R3 = 2 Kiloohm.  
%    }
    
}

\Loesung{

Gegeben ist eine Summierschaltung mit einem idealen Operationsverstärker und einer Betriebsspannung von $\pm 15\,\mathrm{V}$.

Die allgemeine Formel für die Ausgangsspannung lautet:
\[
U_\mathrm{A} = -\left( U_\mathrm{E1} \cdot \frac{R_\mathrm{F}}{R_1} + U_\mathrm{E2} \cdot \frac{R_\mathrm{F}}{R_2} + U_\mathrm{E3} \cdot \frac{R_\mathrm{F}}{R_3} \right)
\]


\textbf{a) Alle Widerstände sind gleich groß: $R_1 = R_2 = R_3 = R_F = 10\,\mathrm{k}\Omega$}

Eingangsspannungen:
\[
U_\mathrm{E1} = 1\,\mathrm{V}, \quad U_\mathrm{E2} = 2\,\mathrm{V}, \quad U_\mathrm{E3} = 3\,\mathrm{V}
\]

Berechnung:
\[
U_\mathrm{A} = -\left( 1\,\mathrm{V} \cdot \frac{10\,\mathrm{k}\Omega}{10\,\mathrm{k}\Omega} + 2\,\mathrm{V} \cdot \frac{10\,\mathrm{k}\Omega}{10\,\mathrm{k}\Omega} + 3\,\mathrm{V} \cdot \frac{10\,\mathrm{k}\Omega}{10\,\mathrm{k}\Omega} \right)
\]
\[
U_A = -\left( 1\,\mathrm{V} + 2\,\mathrm{V} + 3\,\mathrm{V} \right) = -6\,\mathrm{V}
\]


\textbf{b) Geänderte Widerstände: $R_\mathrm{F} = 10\,\mathrm{k}\Omega$, $R_1 = 2\,\mathrm{k}\Omega$, $R_2 = 5\,\mathrm{k}\Omega$, $R_3 = 10\,\mathrm{k}\Omega$}

Eingangsspannungen:
\[
U_\mathrm{E1} = 1\,\mathrm{V}, \quad U_\mathrm{E2} = 2\,\mathrm{V}, \quad U_\mathrm{E3} = 3\,\mathrm{V}
\]

Berechnung:
\[
U_\mathrm{A} = -\left( 1\,\mathrm{V} \cdot \frac{10\,\mathrm{k}\Omega}{2\,\mathrm{k}\Omega} + 2\,\mathrm{V} \cdot \frac{10\,\mathrm{k}\Omega}{5\,\mathrm{k}\Omega} + 3\,\mathrm{V} \cdot \frac{10\,\mathrm{k}\Omega}{10\,\mathrm{k}\Omega} \right)
\]
\[
U_\mathrm{A} = -\left( 1\,\mathrm{V} \cdot 5 + 2\,\mathrm{V} \cdot 2 + 3\,\mathrm{V} \cdot 1 \right)
\]
\[
U_\mathrm{A} = -(5\,\mathrm{V} + 4\,\mathrm{V} + 3\,\mathrm{V}) = -12\,\mathrm{V}
\]


\textbf{c) Geänderte Widerstände: $R_\mathrm{F} = 10\,\mathrm{k}\Omega$, $R_1 = 10\,\mathrm{k}\Omega$, $R_2 = 5\,\mathrm{k}\Omega$, $R_3 = 2\,\mathrm{k}\Omega$}

Eingangsspannungen:
\[
U_\mathrm{E1} = 1\,\mathrm{V}, \quad U_\mathrm{E2} = 2\,\mathrm{V}, \quad U_\mathrm{E3} = 3\,\mathrm{V}
\]

Berechnung:
\[
U_\mathrm{A} = -\left( 1\,\mathrm{V} \cdot \frac{10\,\mathrm{k}\Omega}{10\,\mathrm{k}\Omega} + 2\,\mathrm{V} \cdot \frac{10\,\mathrm{k}\Omega}{5\,\mathrm{k}\Omega} + 3\,\mathrm{V} \cdot \frac{10\,\mathrm{k}\Omega}{2\,\mathrm{k}\Omega} \right)
\]
\[
U_\mathrm{A} = -\left( 1\,\mathrm{V} \cdot 1 + 2\,\mathrm{V} \cdot 2 + 3\,\mathrm{V} \cdot 5 \right)
\]
\[
U_\mathrm{A} = -(1\,\mathrm{V} + 4\,\mathrm{V} + 15\,\mathrm{V}) = -20\,\mathrm{V}
\]

Da die Betriebsspannung $\pm 15\,\mathrm{V}$ beträgt, wird der Ausgang auf $-15\,\mathrm{V}$ begrenzt.


\textbf{Zusammenfassung:}
\begin{itemize}
    \item a) $U_\mathrm{A} = -6\,\mathrm{V}$
    \item b) $U_\mathrm{A} = -12\,\mathrm{V}$
    \item c) $U_\mathrm{A} = -15\,\mathrm{V}$ (Begrenzung durch Betriebsspannung)
\end{itemize}

   %         \speech{
    %        Lösung für Aufgabe 14. 
 %           Gegeben ist eine Summierschaltung mit einem idealen Operationsverstärker und einer Betriebsspannung von plus minus 15 Volt. Die allgemeine Formel für die Ausgangsspannung lautet: 
 %           Ua ist gleich minus Klammer auf U e1 mal R f durch R 1 plus U e2 mal R f durch R 2 plus U e3 mal R f durch R 3 Klammer zu. 
 %           Teilaufgabe a. Alle Widerstände sind gleich groß. Es gilt R 1 gleich R 2 gleich R 3 gleich R f gleich 10 Kiloohm. 
 %           Die Eingangsspannungen betragen U e1 gleich 1 Volt, U e2 gleich 2 Volt, U e3 gleich 3 Volt. 
 %           Berechnung. Ua ist gleich minus Klammer auf 1 Volt mal 10 Kiloohm durch 10 Kiloohm plus 2 Volt mal 10 Kiloohm durch 10 Kiloohm plus 3 Volt mal 10 Kiloohm durch 10 Kiloohm Klammer zu. 
 %           Ua ist gleich minus Klammer auf 1 plus 2 plus 3 Klammer zu. Ua ist gleich minus 6 Volt. 
 %           Teilaufgabe b. Geänderte Widerstände. Es gilt R f gleich 10 Kiloohm, R 1 gleich 2 Kiloohm, R 2 gleich 5 Kiloohm, R 3 gleich 10 Kiloohm. 
 %           Die Eingangsspannungen bleiben gleich. U e1 gleich 1 Volt, U e2 gleich 2 Volt, U e3 gleich 3 Volt. 
 %           Berechnung. Ua ist gleich minus Klammer auf 1 Volt mal 10 Kiloohm durch 2 Kiloohm plus 2 Volt mal 10 Kiloohm durch 5 Kiloohm plus 3 Volt mal 10 Kiloohm durch 10 Kiloohm Klammer zu. 
 %           Ua ist gleich minus Klammer auf 5 plus 4 plus 3 Klammer zu. Ua ist gleich minus 12 Volt. 
 %           Teilaufgabe c. Weitere geänderte Widerstände. Es gilt R f gleich 10 Kiloohm, R 1 gleich 10 Kiloohm, R 2 gleich 5 Kiloohm, R 3 gleich 2 Kiloohm. 
 %           Die Eingangsspannungen bleiben gleich. U e1 gleich 1 Volt, U e2 gleich 2 Volt, U e3 gleich 3 Volt. 
 %           Berechnung. Ua ist gleich minus Klammer auf 1 Volt mal 10 Kiloohm durch 10 Kiloohm plus 2 Volt mal 10 Kiloohm durch 5 Kiloohm plus 3 Volt mal 10 Kiloohm durch 2 Kiloohm Klammer zu. 
 %           Ua ist gleich minus Klammer auf 1 plus 4 plus 15 Klammer zu. Ua ist gleich minus 20 Volt. 
 %           Da die Betriebsspannung plus minus 15 Volt beträgt, wird der Ausgang auf minus 15 Volt begrenzt. 
 %           Zusammenfassung. 
 %           a. Ua ist gleich minus 6 Volt. 
 %           b. Ua ist gleich minus 12 Volt. 
 %           c. Ua ist gleich minus 15 Volt aufgrund der Begrenzung durch die Betriebsspannung.
  %          }
  Siehe: Abschnitt \ref{fig: Verschaltung Verstaerker} und Tabelle \ref{tab:Grundschaltungen}
    
}
