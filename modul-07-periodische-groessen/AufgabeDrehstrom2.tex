\subsection{Drehstrom 2}
\Aufgabe{
    Es ist ein symmetrisches Drehstromsystem mit folgenden Spannungen gegeben:
   
    $\underline{U}_1=230~V$, $\underline{U}_2=230~V\cdot \underline{a}^2$, $\underline{U}_3=230~V\cdot \underline{a}$
 
    Die Impedanzen haben alle den Wert $Z=10\Omega $
 
    Für alle Aufgabenteile gilt: Lösen Sie die Aufgaben mit dem geringstmöglichen Aufwand!
 
    \begin{itemize}
        \item [\bf a)] Zeichnen Sie die dreiphasigen und einphasigen ESB in Stern- und Dreieckschaltung und beschrifften Sie die ESB korrekt!
        \item [\bf b)] Berechnen Sie den Betrag der Verbraucherspannungen sowohl für die Stern- als auch die Dreieckschaltung!
        \item [\bf c)] Berechnen Sie die komplexen Verbraucherspannungen in der Dreieckschaltung!
        \item [\bf d)] Zeichnen Sie die Zeigerdiagramme der Verbaucherspannungen für die Stern- und Dreieckschaltung!
        \item [\bf e)] Berechnen Sie die komplexen Lastströme für die Sternschaltung und den Betrag der Lastströme für die Dreieckschaltung!
        \item [\bf f)] Berechnen Sie die Scheinleistung für die Stern- und Dreieckschaltung!
    \end{itemize}
}


\Loesung{
	\begin{itemize}

	\item[\bf a)]
        
        ESBs der Sternschaltung:\\
        \resizebox{0.7\textwidth}{!}{%
        \begin{circuitikz}
            \draw (0, 0) to[short,] (0, 4)
            to[sV, v<=230 V, anchor=west] (2, 4)
            to[short, i=?] (4, 4)
            to[short] (10, 4)
            to[R=10\ohm] (12, 4)
            to[short,] (12, 0);
        
            \draw (0, 2) to[sV, v<=$230 V\cdot \underline{a}^2$, anchor=west, *-] (2, 2)
            to[short, i=?] (4, 2)
            to[short] (10, 2)
            to[R=10\ohm, -*] (12, 2);
        
            \draw (0, 0) to[sV, v<=$230 V\cdot\underline{a}$, anchor=west] (2, 0)
            to[short, i=?] (4, 0)
            to[short] (10, 0)
            to[R=10\ohm] (12, 0);
        
            \draw (-0.5, 2) node {$N_{\mathrm{Q}}$}
            (12.5, 2) node {$N_{\mathrm{V}}$};
        
            \draw [->](5.5, 3.8) -- (5.5, 2.2);
            \draw (5.8, 3) node {?};
            \draw [->](5.5, 1.8) -- (5.5, 0.2);
            \draw (5.8, 1) node {?};
            \draw [<-](7.5, 3.8) -- (7.5, 0.2);
            \draw (7.8, 1) node {?};  
        \end{circuitikz}%
        }
         
        \begin{circuitikz}
            \draw (0, 0) to[sV, v<=230 V] (0, 2)
            to[short, i=?] (5, 2)
            to[R=10\ohm, v=?] (5, 0)
            to[short] (0, 0);
        \end{circuitikz}

        ESBs der Dreieckschaltung:\\
 
        \resizebox{0.7\textwidth}{!}{%
        \begin{circuitikz}
            \draw (0, 0) to[short] (0, 4)
            to[sV, v<=230 V] (2, 4)
            to[short, i=?] (5, 4)
            to[short] (13, 4)
            to[R=10\ohm, i<=?] (13, 0);
        
            \draw (0, 2) to[sV, v<=$230\cdot \underline{a}^2 V$, *-] (2, 2)
            to[short, i=?] (5, 2)
            to[short] (9, 2)
            to[R=10\ohm, label distance = 2.5, i<=?, *-*] (13, 4);
    
            \draw (0, 0) to[sV, v<=$230 V\cdot \underline{a}$] (2, 0)
            to[short, i=?] (5, 0)
            to[short] (13, 0)
            to[R=10\ohm, i_<=?, *-*] (9, 2);
    
            \draw (-0.5, 2) node {$N_Q$};
        
            \draw [->](5, 3.8) -- (5, 2.2);
            \draw (5.3, 3) node {?};
            \draw [->](5, 1.8) -- (5, 0.2);
            \draw (5.3, 1) node {?};
            \draw [<-](7, 3.8) -- (7, 0.2);
            \draw (7.3, 1) node {?};
        \end{circuitikz}%
        }
    
        \begin{circuitikz}
            \draw (0, 0) to[sV, v<=230 V] (0, 2)
            to[short, i=?] (5, 2)
            to[R=$\frac{10\ohm}{3}$, v=?] (5, 0)
            to[short] (0, 0);
        \end{circuitikz}
 
    \item[\bf b)]
        Der Betrag Verbraucherspannungen ist in der Sternschaltung gleich der Quellspannung: 230 V\\
        Der Betrag Verbraucherspannungen ist in der Dreieckschaltung $\sqrt{3}$ mal größer als in der Sternschaltung: $230 V\cdot \sqrt{3}=398,37 V\approx 400 V$
 
    \item[\bf c)]
 
        Die komplexe Verbaucherspannungen in der Dreieckschaltung errechnet sich aus den Differenzen der Sternspannungen.
        
        \begin{eqa}
            \underline{U}_\mathrm{L1L2}&=\underline{U}_\mathrm{L1}-\underline{U}_\mathrm{L2}=230 V-230 V\cdot \underline{a}^2 \notag \\
            &=230 V(1-(-\frac{1}{2}-j\frac{\sqrt{3}}{2}))=\sqrt{3}\cdot 230 V(\frac{3}{2}+j\frac{1}{2}) \notag \\
            &=\sqrt{3}\cdot 230 V\cdot e^{j\cdot 30^\circ} \notag
        \end{eqa}
        \begin{eqa}
            \underline{U}_\mathrm{L2L3}&=\underline{U}_\mathrm{L2}-\underline{U}_\mathrm{L3}=230 V\cdot \underline{a}^2-230 V\cdot \underline{a} \notag \\
            &=230 V(-\frac{1}{2}-j\frac{\sqrt{3}}{2}-(-\frac{1}{2}+j\frac{\sqrt{3}}{2}))=\sqrt{3}\cdot 230 V(-j) \notag \\
            &=\sqrt{3}\cdot 230 V\cdot e^{-j\cdot 90^\circ} \notag
        \end{eqa}
        \begin{eqa}
            \underline{U}_\mathrm{L3L1}&=\underline{U}_\mathrm{L3}-\underline{U}_\mathrm{L1}=230 V\cdot \underline{a}-230 V \notag \\
            &=230 V(-\frac{1}{2}+j\frac{\sqrt{3}}{2}-1)=\sqrt{3}\cdot 230 V(-\frac{\sqrt{3}}{2}+j\frac{1}{2}) \notag \\
            &=\sqrt{3}\cdot 230 V\cdot e^{j\cdot 150^\circ} \nonumber
        \end{eqa}
 
	\item[\bf d)]
 
    Zeigerdiagramme der Verbaucherspannungen für die Stern- und Dreieckschaltung:\\

    \begin{figure}
        \centering
        \resizebox{0.6\textwidth}{!}{%
        \begin{tikzpicture}
            \draw [->, gray] (0, 0) coordinate (A)-- (4,0) coordinate (B);
            \draw [->, gray] (4, 0) -- (6.002, 3.465);
            \draw [->, gray] (0, 0) -- (-2, 3.464) coordinate (C);
            \draw [->, gray] (-2, 3.464) -- (-6.002, 3.465);
            \draw [->, gray] (0, 0) -- (-2, -3.464) coordinate (D);
            \draw [->, gray] (-2, -3.464) -- (0, -6.93);
        
        
            \draw (2.5, 0.3) node[anchor=center] {$\underline{U}_\mathrm{L1}$};
            \draw (-1.5, -1.7) node[anchor=center] {$\underline{U}_\mathrm{L2}$};
            \draw (-0.4, 1.7) node[anchor=center] {$\underline{U}_\mathrm{L3}$};
            \draw (-4, 3.8) node[anchor=center] {-$\underline{U}_\mathrm{L1}$};
            \draw (5.8, 1.5) node[anchor=center] {-$\underline{U}_\mathrm{L2}$};
            \draw (-1.8, -4.6) node[anchor=center] {-$\underline{U}_\mathrm{L3}$};
            \draw (2.7, 2.2) node[anchor=center] {$\underline{U}_\mathrm{L1L2}$};
            \draw (0.7, -2.8) node[anchor=center] {$\underline{U}_\mathrm{L2L3}$};
            \draw (-3.2, 1.5) node[anchor=center] {$\underline{U}_\mathrm{L3L1}$};
        
        
            %\draw pic[draw, angle radius = 13, ->] {angle = B--A--C};
        
    
            \draw [->] (0, 0) -- (6.002, 3.465);
            \draw [->] (0, 0) -- (0, -6.93);
            \draw [->] (0, 0) -- (-6.002, 3.465);
        \end{tikzpicture}%
    }
    \end{figure}
 
    \item[\bf e)]
 
        Der Strom wird über das Ohmsche Gesetz berechnet. Für die Sternschaltung gilt:
        
        \begin{eqa}
            \underline{I}_1&=\frac{\underline{U}_\mathrm{L1}}{\underline{Z}}=\frac{230 V}{10\ohm}=23 A \notag \\
            \underline{I}_2&=\frac{\underline{U}_\mathrm{L2}}{\underline{Z}}=\frac{230 V\cdot \underline{a}^2}{10\ohm}=23 A\cdot\underline{a}^2  \notag \\
            \underline{I}_3&=\frac{\underline{U}_\mathrm{L3}}{\underline{Z}}=\frac{230 V\cdot \underline{a}}{10\ohm}=23 A\cdot\underline{a}   \notag
        \end{eqa}
        
        Bei dem Dreieckstrom wird zwischen dem Strangstrom und dem Dreieckstrom, also dem Strom durch die Verbraucher unterschieden.
        Der Strom in den Verbauchersträngen kann mit den Spannung über die Verbraucher in der Dreieckschaltung errechnet werden:
        
        \begin{eqa}
            I_\Dreieck&=\frac{U\Dreieck}{Z}=\frac{400V}{10\ohm}=40 A \notag
        \end{eqa}
        
        Der Strom im Strang ist wiederum um $\sqrt{3}$ größer, als in den Verbauchersträngen:
        \begin{eqa}
            I_\mathrm{str}=\sqrt{3}\cdot I_\Dreieck=\sqrt{3}\cdot 40 A=69,28 A \notag
        \end{eqa}
        
        Dieser Wert entspricht dem dreifachen Wert des Stroms in einer Sternschaltung
 
 
    \item[\bf f)]
 
        Die allgemein Leistungsgleichung ist folgende:
        
        \begin{eqa}
            S=\sqrt{3}\cdot U_\Dreieck \cdot I_\mathrm{str} \notag
        \end{eqa}
        
        Welcher Strom eingesetzt werden muss, hängt von der jeweiligen Schaltung ab.
        Wie in Aufgabe 5. berechnet ist der Strangstrom in der Dreieckschaltung um den Faktor 3 größer als in der Sternschaltung
        
        Für die Sternschaltung gilt:
        \begin{eqa}
            S=\sqrt{3}\cdot U\cdot I=\sqrt{3}\cdot 400 V \cdot 23 A=15.934 VA \notag
        \end{eqa}
        
        Und für die Dreieckschaltung gilt:
        \begin{eqa}
            S=\sqrt{3}\cdot U\cdot I=\sqrt{3}\cdot 400 V \cdot 3\cdot 23 A=47.804 VA \notag
        \end{eqa}

\end{itemize}

}