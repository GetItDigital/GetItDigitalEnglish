% Phasenverschiebung an Widerstand
\begin{tikzpicture}[x=1cm,y=1cm]
    \draw[draw=none] (-1,-0.75) rectangle (7,2.5); % boundbox
\begin{axis}[
    ylabel={Widerstand $R$},
    %xlabel=$\omega t$,
    xmin=0, xmax=2.5, % in pi
    ymin=-1.5, ymax=1.5,
    xtick={0,0.5,1,1.5,2,2.5},
    xticklabels={$0$, $\frac{\pi}{2}$, $\pi$, $\frac{3}{2}\pi$, $2\pi$, $\omega t$}, % last one is x-axis label
    %ytick={-1,0,1},
    yticklabels={},
    width=8cm,
    height=4cm,
    grid=major,
    domain=0:2.5,
    legend pos=north east,
    %legend style={at={(0.5,-0.1)},anchor=north},
]
    \addplot[gray,forget plot] coordinates {(0,0) (2.5,0)}; % x-axis
    \addplot[thick, samples=100, voltage]{sin(deg(x*pi))};
    \addplot[thick, samples=100, red]{0.75*sin(deg(x*pi))};
    \legend{$\ecv{u_R}$,$\eci{i_R}$}
\end{axis}
\end{tikzpicture}