\subsection{Ermittlung von Basis-Emitter-Spannung und Basisstrom\label{Aufg12}}
Gegeben ist die abgebildete Transistorschaltung mit den folgenden Parametern: $U_0=\mathrm{5\,V}$ ,
$U_L=\mathrm{15\,V}$, $R_\mathrm{V}=\mathrm{100\,k\Omega}$ und $R_\mathrm{V}=\mathrm{500\,\Omega}$.
% Alt: Aufgabe 12
\Aufgabe{
\begin{itemize}
  \item[a)] Wie groß ist die Spannung $U_\mathrm{BE}$ näherungsweise, wenn es sich um einen Siliziumtransistor handelt?
  \item[b)] Wie groß ist der Strom $I_\mathrm{B}$?
\end{itemize} 
\begin{figure}[H]
   \centering
   \begin{tikzpicture}
     \draw (4,2) node[npn, tr circle] (Q1) {};
 
     \draw (Q1.C) -- (4,4) 
     to [R,l=$R_\mathrm{L}$] (7,4)
     to [V, name=Ul, v] (7,0);

     \draw (Q1.E) to [short, -*] (4,0) 
     -- (7,0);

     \draw (Q1.B) to [R,l_=$R_\mathrm{V}$, i<, name=Ib] (0,2)
     to [V, name=U0, v] (0,0)
     to [short] (4,0);

     \draw (Q1.B) ++ (-0.25,0) coordinate(Tb);
     \draw (Q1.E) ++ (0,-0.25) coordinate(Te);
     \draw (Tb) to [open, name=ube, v, voltage=european] (Te);

     \varrmore{ube}{$U_\mathrm{BE}$};
     \varrmore{U0}{$U_\mathrm{0}$};
     \varrmore{Ul}{$U_\mathrm{L}$};
     \iarrmore{Ib}{$I_\mathrm{B}$};
 \end{tikzpicture}
   \label{fig:FigTransistorBasisstrom}
\end{figure}
% \speech{ **Aufgabe 12**  
% **Fragestellung:**  
% a) Wie groß ist die Spannung \( U_{BE} \)?  
% b) Wie groß ist der Strom \( I_B \), wenn es sich um einen Siliziumtransistor handelt?  

%  **Beschreibung der Abbildung**  
% Die gezeigte Schaltung stellt einen **npn-Transistor in einer Basisschaltung** dar, der von zwei Spannungsquellen gespeist wird.

% 1. **Spannungsquellen und Widerstände:**  
%    - **Eine 5V-Spannungsquelle** speist die Basis über einen **100 kΩ-Widerstand**.  
%    - **Eine 15V-Spannungsquelle** versorgt den Kollektor über einen **500Ω-Widerstand**.

% 2. **Transistorkennzeichnung:**  
%    - Der Transistor hat drei Anschlüsse:  
%      - **Basis (B)** – wird über den 100 kΩ-Widerstand gespeist.  
%      - **Kollektor (C)** – verbunden mit dem 500Ω-Widerstand zur 15V-Quelle.  
%      - **Emitter (E)** – mit Masse verbunden.  
%    - Der Transistor wird als **Siliziumtransistor** betrachtet, was bedeutet, dass die typische **Basis-Emitter-Spannung \( U_{BE} \)** etwa **0,7V** beträgt.  

% 3. **Kennzeichnungen in der Schaltung:**  
%    - **Der Basisstrom \( I_B \)** ist als roter Pfeil eingezeichnet, der in die Basis fließt.  
%    - **Die Spannung \( U_{BE} \)** ist als blaue Beschriftung eingezeichnet und befindet sich zwischen Basis und Emitter.  



%  **Zusammenfassung der Aufgabe**  
% - Die Schaltung dient dazu, die **Basis-Emitter-Spannung \( U_{BE} \)** eines Siliziumtransistors zu bestimmen.  
% - Zusätzlich soll der **Basisstrom \( I_B \)** unter Berücksichtigung des Basisvorwiderstands berechnet werden.  
% }

}

\Loesung{
  \begin{itemize}
    \item[a)]
  \end{itemize}
  \begin{equation}
    U_\text{BE} = U_\text{B} - U_\text{E}
  \end{equation} 
  Ein npn-Transistor hat im leitenden Zustand eine typische Basis-Emitter-Spannung von:
  \begin{equation}
    U_\text{BE} \approx 0,6\,\text{V} \text{ bis } 0,7\,\text{V}
  \end{equation}
  Diese Spannung ist eine materialbedingte Eigenschaft des Silizium-Transistors und bleibt nahezu konstant, unabhängig von der Basisspannung \( U_\text{B} \).
  - Obwohl \( U_\text{B} = 5\,\text{V} \) ist, begrenzt der Transistor die Spannung \( U_{BE} \) auf ca. \( 0,6\,\text{V} \), sobald der Transistor leitend ist.  
  - Der überschüssige Spannungsabfall (\( U_B - U_{BE} \)) wird durch andere Ströme und Bauteile der Schaltung ausgeglichen.
  \begin{align*}
    {U_{BE} \approx 0,6\,\text{V}}
  \end{align*}
  \begin{itemize}
    \item[b)] $I_\text{B} = \frac{5\,\text{V} - 0,6\,\text{V}}{100\,\text{k}\Omega} = 44\,\mu\text{A}$
  \end{itemize}
  Siehe: Abschnitt \ref{sec:Bipolartransistor} und \ref{fig:StroemeUndSpannungenBeimBipolartransistor}
%   \speech{
% **Aufgabe 12 – Analyse eines npn-Transistors**

% Gegeben ist ein npn-Transistor. Die Aufgabe besteht darin, die Basis-Emitter-Spannung \( U_{BE} \) sowie den Basisstrom \( I_B \) zu berechnen.

% ### Teil a) Bestimmung der Basis-Emitter-Spannung \( U_{BE} \)
% Die Basis-Emitter-Spannung \( U_{BE} \) eines Silizium-Transistors beträgt im leitenden Zustand typischerweise zwischen **0,6 und 0,7 Volt**.

% Diese Spannung ist eine materialbedingte Eigenschaft von Silizium-Transistoren und bleibt nahezu konstant, unabhängig von der Basisspannung \( U_B \). Das bedeutet:
% - Auch wenn die Basisspannung \( U_B \) **5 Volt** beträgt, begrenzt der Transistor die Spannung \( U_{BE} \) auf ungefähr **0,6 Volt**, sobald er leitend ist.
% - Der überschüssige Spannungsabfall, also **die Differenz zwischen \( U_B \) und \( U_{BE} \)**, wird durch andere Ströme und Bauteile in der Schaltung ausgeglichen.

% Ergebnis:
% - Die Basis-Emitter-Spannung beträgt **ungefähr 0,6 Volt**.

% ### Teil b) Berechnung des Basisstroms \( I_B \)
% Der Basisstrom \( I_B \) ergibt sich aus dem Widerstand in der Basisleitung. Die Formel zur Berechnung lautet:

% \[
% I_B = \frac{U_B - U_{BE}}{R_B}
% \]

% Eingesetzte Werte:
% - Basisspannung \( U_B = 5V \)
% - Basis-Emitter-Spannung \( U_{BE} = 0,6V \)
% - Basiswiderstand \( R_B = 100 k\Omega \)

% Berechnung:
% \[
% I_B = \frac{5V - 0,6V}{100 k\Omega}
% \]

% \[
% I_B = \frac{4,4V}{100 k\Omega} = 44 \mu A
% \]

% Ergebnis:
% - Der Basisstrom beträgt **44 Mikroampere**.

% Fazit:
% - Der npn-Transistor begrenzt die Basis-Emitter-Spannung unabhängig von der angelegten Basisspannung.
% - Der Basisstrom hängt vom Basiswiderstand ab und beträgt hier **44 Mikroampere**.
% }

}
