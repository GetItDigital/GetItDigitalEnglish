\subsection{Berechnung der Ausgangsspannung einer zweistufigen OP-Schaltung\label{Aufg6}}
% Alt: Aufgabe 6
\Aufgabe{
    Berechnen Sie $U_\mathrm{A}$ mit folgenden Werten: \newline
    $R_1 = \mathrm{10\,k\Omega}$, $R_2 = \mathrm{20\,k\Omega}$, $R_3 = \mathrm{30\,k\Omega}$, $R_4 = \mathrm{10\,k\Omega}$, $R_5 = \mathrm{20\,k\Omega}$

    \begin {figure} [H]
   \centering
   \begin{tikzpicture}
\ctikzset{tripoles/en amp/input height=0.45}
\draw (0,0)node[en amp](E){N1}
    (E.out)
    (E.-)
    (E.+);
    \draw (E.out) to[short, -*] (2,0) to[R,l=$R_5$] (4,0) to[short, -*] (4,0.5);
    \draw (E.-) to[short, -*] (-2,0.5) to[short, -*] (-3,0.5) to[short, -*] (-3,1.5) -- (-3,2.5)
    to[R,l_=$R_1$] (-5,2.5) to[short, -o] (-7,2.5);
    \draw (-2,0.5) to[short] (-2,2.5) to[R, name=R, l=$R_4$] (2,2.5) -- (2,0);
    \draw (-3,1.5) to[R, l_=$R_2$] (-5,1.5) to[short, -o] (-6,1.5);
    \draw (-3,0.5) to[R,l_=$R_3$,-o] (-5,0.5);

    \draw (-7,-2) to[short, o-o] (-6,-2) to[short, -o](-5,-2);
    \draw (E.+) -- (-2,-0.5) to[short, -*] (-2,-2);
    \draw (2.5,-2) to[short, -o] (-5,-2);
    \draw (2.5,-2) to[short, -*] (4,-2) to[short, -o](8.5,-2);

        \draw (6,0)node[en amp](E){N2}
    (E.out)
    (E.-)
    (E.+);
    \draw (E.-) -- (4,0.5) -- (4,2.5) to[R,l=$R_6$] (8,2.5) to[short, -*] (8,0);
    \draw (E.out) to[short, -o] (8.5,0);
    \draw (E.+) -- (4,-0.5) -- (4,-2);
    
    \draw (8.5,0) to[open, v, name=ua] (8.5,-2);
    \draw (-7,2.5) to[open, v, name=ue1] (-7,-2);
    \draw (-6,1.5) to[open, v, name=ue2] (-6,-2);
    \draw (-5,0.5) to[open, v, name=ue3] (-5,-2);
    
    \varrmore{ue1}{${U_\mathrm{E1}}$};
    \varrmore{ue2}{${U_\mathrm{E2}}$};
    \varrmore{ue3}{${U_\mathrm{E3}}$};
    \varrmore{ua}{${U_\mathrm{A}}$};
\end{tikzpicture}
   \label{fig:FigOPVzweistufig}
    \end {figure}
 %   \speech{
 %       Aufgabe 6: Operationsverstärkerschaltung mit zwei Stufen  
 %       
 %       Gegeben ist eine Schaltung mit zwei Operationsverstärkern N1 und N2.  
 %       Die Widerstände haben folgende Werte:  
 %       R1 = 10 Kiloohm, R2 = 20 Kiloohm, R3 = 30 Kiloohm,  
 %       R4 = 10 Kiloohm, R5 = 20 Kiloohm.  
 %       
 %       Die Eingangsspannungen U E1, U E2 und U E3 werden über die Widerstände R1, R2 und R3 an den invertierenden Eingang des ersten Operationsverstärkers N1 geführt.  
 %       Der nichtinvertierende Eingang von N1 ist mit Masse verbunden.  
 %       Der Verstärker N1 hat eine Rückkopplung über den Widerstand R4.  
 %       Der Ausgang von N1 ist mit einem weiteren Widerstand R5 verbunden,  
 %       der das Signal an den zweiten Operationsverstärker N2 weiterleitet.  
 %       Auch N2 arbeitet mit einer Rückkopplung über einen Widerstand R6.  
 %       
 %       Gesucht ist die Ausgangsspannung U A in Abhängigkeit der Eingangsspannungen U E1, U E2 und U E3.  
 %       }
        

}

\Loesung{
    Formel:
    \begin{equation}
        U_\mathrm{N1} = - \left( \frac{R_\mathrm{N}}{R_\mathrm{1}} U_\mathrm{E1} + \frac{R_\mathrm{N}}{R_2} U_\mathrm{E2} + \frac{R_\mathrm{N}}{R_\mathrm{3}} U_\mathrm{E3} \right)
    \end{equation}
    \begin{equation}
        U_\mathrm{A} = -\frac{R_\mathrm{6}}{R_\mathrm{5}} U_\mathrm{N1}
    \end{equation}

Berechnung:

Berechnung von $U_\mathrm{N1}$
    \begin{align*}
        U_\mathrm{N1} &= - \left( \frac{10\,\mathrm{k}\Omega}{10\,\mathrm{k}\Omega} U_\mathrm{E1} + \frac{10\,\mathrm{k}\Omega}{20\,\mathrm{k}\Omega} U_\mathrm{E2} + \frac{10\,\mathrm{k}\Omega}{30\,\mathrm{k}\Omega} U_\mathrm{E3} \right) \\
        &= - \left( U_\mathrm{E1} + 0,5 U_\mathrm{E2} + \frac{1}{3} U_\mathrm{E3} \right)
    \end{align*}

Berechnung von $U_\mathrm{A}$
    \begin{align*}
        U_\mathrm{A} &= -2 \cdot U_\mathrm{N1} \\
        &= -2 \left( - \left( U_\mathrm{E1} + 0,5 U_\mathrm{E2} + \frac{1}{3} U_\mathrm{E3} \right) \right) \\
        &= 2 \left( U_\mathrm{E1} + 0,5 U_\mathrm{E2} + \frac{1}{3} U_\mathrm{E3} \right)
    \end{align*}

Die Ausgangsspannung $U_\mathrm{A}$ beträgt also:
    \begin{align*}
        U_\mathrm{A} &= 2 U_\mathrm{E1} + U_\mathrm{E2} + \frac{2}{3} U_\mathrm{E3}
    \end{align*}
    Siehe: Abschnitt \ref{fig: Verschaltung Verstaerker} und Tabelle \ref{tab:Grundschaltungen}

%    \speech{
%        Berechnung der Ausgangsspannung einer zweistufigen Operationsverstärkerschaltung  
%        
%        Die Schaltung besteht aus zwei Operationsverstärkern.  
%        Die Berechnung erfolgt in zwei Schritten:  
%        
%        1. Berechnung der Spannung U N1 am Ausgang des ersten Operationsverstärkers  
%        2. Berechnung der endgültigen Ausgangsspannung U A  
%        
%        Formeln zur Berechnung:  
%        
%        U N1 = - ( R N durch R1 mal U E1 + R N durch R2 mal U E2 + R N durch R3 mal U E3 )  
%        
%        U A = - R6 durch R5 mal U N1  
%        
%        Einsetzen der Werte für die Widerstände:  
%        
%        U N1 = - ( 10 Kiloohm durch 10 Kiloohm mal U E1 + 10 Kiloohm durch 20 Kiloohm mal U E2 + 10 Kiloohm durch 30 Kiloohm mal U E3 )  
%        
%        U N1 = - ( U E1 + 0,5 mal U E2 + 1 Drittel mal U E3 )  
%        
%        Berechnung der Ausgangsspannung:  
%        
%        U A = - 2 mal U N1  
%        
%        U A = 2 mal ( U E1 + 0,5 mal U E2 + 1 Drittel mal U E3 )  
%        
%        Daraus ergibt sich die endgültige Gleichung:  
%        
%        U A = 2 mal U E1 + U E2 + 2 Drittel mal U E3  
%        
%        Die Berechnung zeigt, dass die Verstärkung jeder Eingangsspannung von den Widerstandswerten abhängt.  
%        }
        
}