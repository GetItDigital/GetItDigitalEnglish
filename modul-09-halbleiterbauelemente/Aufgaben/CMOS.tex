\subsection{Erklärung des Funktionsprinzips der CMOS-Technologie\label{Aufg26}}
% Alt: Aufgabe 26
\Aufgabe{
   Was sagt das C in CMOS aus?
   \begin{itemize}
      \item[a)] Unterscheidung von Anreicherungs- und Verarmungstyp
      \item[b)] Die Verwendung von JFETs und MOSFETS in einer Schaltung
      \item[c)] Spiegelt den Verlauf der Transferkennlinie wieder
      \item[d)] Weißt auf die Spannungsempfindlichkeit von MOSFETs hin
      \item[e)] In einer Schaltung werden sowohl p- als auch n-Kanal Transistoren 
      verwendet 
   \end{itemize}
%    \speech{
% Aufgabe 26

% Was sagt das C in CMOS aus?

% Fragen:

% a) Unterscheidung von Anreicherungstyp und Verarmungstyp.

% b) Die Verwendung von JFETs und MOSFETs in einer Schaltung.

% c) Spiegelt den Verlauf der Transferkennlinie wider.

% d) Weißt auf die Spannungsempfindlichkeit von MOSFETs hin.

% e) In einer Schaltung werden sowohl p- als auch n-Kanal Transistoren verwendet.

% }

}



\Loesung{
  \begin{itemize}
   \item [e)] In einer Schaltung werden sowohl p- als auch n-Kanal Transistoren verwendet
   Siehe: Abschnitt \ref{sec:AufbauTransistor}
  \end{itemize}
%   \speech{
% Lösung zu Aufgabe 26:
% Das C in CMOS steht für complementary, was bedeutet, dass in einer Schaltung sowohl p- als auch n-Kanal Transistoren verwendet werden.
% }
}



