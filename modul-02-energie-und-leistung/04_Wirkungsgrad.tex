%\newsection{Kapitel1}

%Hier den Inhalt mit \input einfügen und die folgenden Zeilen entfernen




	\section{Der Wirkungsgrad}
	
\s{
Der Wirkungsgrad beschreibt das Verhältnis von abgegebener Leistung (Nutzleistung) 
zu aufgenommener Leistung (zugeführte Energie) in einem System und ist ein Maß für die Effizienz, 
mit der ein System Energie umwandelt. Am Beispiel eines Elektromotors zeigt Abbildung \ref{fig:Wirkungsgrad} wie sich dessen Leistungsbilanz zusammensetzt.
}
\begin{frame}
   \ftx{Der Wirkungsgrad}
\s{\begin{figure}[H]

	\begin{tikzpicture}
			 \node at (0,0) {\includesvg[width=1\textwidth]{Wirkungsgrad_Copy}};
			 \node at (-3.5,0.6) {\shortstack[l]{zugeführte\\Leistung $P_\mathrm{{zu}}$}};
			 \node at (-0.35,1.5) {Verluste $P_\mathrm{v}$};  
			 \node at (3.65,0.6) {\shortstack[l]{abgegebene\\Leistung $P_\mathrm{{ab}}$}};
	\end{tikzpicture}
	\caption{\textbf{Wirkungsgrad eines Elektromotors.} Zeigt das Verhältnis der abgegebenen
	 Leistung zur zugeführten Leistung. Der Wirkungsgrad beschreibt die Effizienz des Motors bei der Umwandlung von Energie.}
	\label{fig:Wirkungsgrad}
	\end{figure}}
\b{
	Der Wirkungsgrad beschreibt das Verhältnis von abgegebener zu zugeführter Leistung
    \begin{figure}[H]
	\begin{tikzpicture}
			 \node at (0,0) {\includesvg[width=1\textwidth]{Wirkungsgrad_Copy}};
			 \node at (-3.5,0.6) {\shortstack[l]{zugeführte\\Leistung $P_\mathrm{{zu}}$}};
			 \node at (-0.35,1.5) {Verluste $P_\mathrm{v}$};  
			 \node at (3.65,0.6) {\shortstack[l]{abgegebene\\Leistung $P_\mathrm{{ab}}$}};
	\end{tikzpicture}
	\end{figure}

}
\s{
	Der Elektromotor wird am Eingang mit elektrischer Energie versorgt, welche er in einer Eingangsleistung $P_\mathrm{zu}$ aufnimmt.
	Die gewünschte Leistung am Ausgang $P_\mathrm{ab}$ ist in diesem Fall rotatorischer Natur. Die durch den Motor verursachte Abwärme, aber auch die verursachten Geräusche oder Reibung, ergeben zusammen die Verlustleistung $P_\mathrm{v}$. 
}


\subsection{Berechnung des Wirkungsgrades}

\s{
Der Wirkungsgrad wird typischerweise als Prozentsatz ausgedrückt. Ein Wirkungsgrad von 100 \%
bedeutet, dass die gesamte zugeführte Energie effektiv in Nutzenergie umgewandelt wird. 
Dies ist jedoch in der Praxis aufgrund von Verlusten, meist in Form von Wärme, Reibung oder Schall, nicht erreichbar.
Mathematisch wird der Wirkungsgrad $\eta$ (eta) wie folgt definiert:

\begin{equation}
	\eta = \frac{P_\mathrm{{ab}}}{P_\mathrm{{zu}}} \cdot 100 \%
\end{equation}
}

	
\b{
\begin{equation*}
	\eta = \frac{P_\mathrm{{ab}}}{P_\mathrm{{zu}}} \cdot 100 \%
\end{equation*}
}
\end{frame}

\begin{frame}	
\ftx{Der Wirkungsgrad}
\s{
	Im realen Fall wird der Wirkungsgrad nie 100 \% betragen, da eine Energiewandlung grundsätzlich mit Verlusten behaftet ist.
	Um die Verluste möglichst gering zu halten ist es wichtig sich der Verlustmechanismen bewusst zu werden und diese nach Möglichkeit zu reduzieren.
	Der nachfolgende Abschnitt erläutert das Prinzip der Verlustmechanismen anhand eines Leuchtmittels.
}

\s{


\subsection{Verlustmechanismen eines Leuchtmittels}

Ein Paradebeispiel für die Relevanz des Wirkungsgrades ist die klassische Glühlampe. Sie erzeugt Licht, indem
Strom durch einen Draht fließt, der sich dabei so stark erhitzt, dass er zu leuchten beginnt.
Dabei werden weniger als 5 \% der eingesetzten Energie in Licht umgewandelt. Mehr als 95 \% der Energie werden in
Wärme umgewandelt. Aufgrund der schlechten Effizienz wurden Glühlampen seit 2009 in der EU schrittweise verboten. 
Abbildung \ref{fig:Energieumwandlung Glühlampe und Energiesparlampe} zeigt links eine klassische Glühlampe und rechts eine Energiesparlampe.

\begin{figure}[H]
	\hspace{-0.8cm}
	\begin{tikzpicture}
		\node at (0,0) {\includesvg[width=1\textwidth]{Wirkungsgrad_Birne}};
				 \node at (-3.2,1.3) {Licht};
				 \node at (0.8,1.3) {Licht};
				 \node at (0.35,1.8) {Wärme};  
				 \node at (3.4,1.4) {Wärme};

	\end{tikzpicture}
	\caption{\textbf{Energieumwandlung in Glühlampe und Energiesparlampe.} Zeigt das eine Glühlampe den Großteil der aufgenommenen
	 elektrischen Energie in Wärme und nur einen kleinen Teil in Licht umwandelt, wandelt die Energiesparlampe einen deutlich höheren Anteil der elektrischen Energie in Licht um.}
	\label{fig:Energieumwandlung Glühlampe und Energiesparlampe}
\end{figure}
}

	\b{\begin{figure}[H]
		\hspace{-0.8cm}
		\begin{tikzpicture}
			\node at (0,0) {\includesvg[width=1\textwidth]{Wirkungsgrad_Birne}};
					 \node at (-3.2,1.3) {\small Licht};
					 \node at (0.8,1.3) {\small Licht};
					 \node at (0.35,1.8) {\small Wärme};  
					 \node at (3.4,1.4) {\small Wärme};
	
		\end{tikzpicture}
	\end{figure}
}
\end{frame}
\s{
Ein hoher Wirkungsgrad ist in vielen Anwendungen erstrebenswert, da dies bedeutet, dass weniger
Energie in i.d.R. nicht gewünschte Energieformen gewandelt wird, sondern dem System als Nutzenergie zur Verfügung steht. 
In der Praxis sind jedoch viele Faktoren wie Materialbeschaffenheit,
Bauweise und Betriebsbedingungen entscheidend für die Maximierung des Wirkungsgrades.
In der Umwelt- und Energietechnik spielt der Wirkungsgrad eine entscheidende Rolle, da er direkt
mit dem Energieverbrauch und den Umweltauswirkungen zusammenhängt. Effizientere Systeme
können dazu beitragen, den Energiebedarf zu reduzieren und Treibhausgasemissionen zu verringern.
}
% Mehr auf Energiesparlampe eingehen: (wieso ist die besser?)
% technische Umsetzung? Warum ist die Besser? (weniger Abwärme)
% Bei Gelichem Licht, weniger Wärme, dadurch weniger Verluste, Dadurch besserer Wirkungsgrad.
% Wie werden die ESL technisch realisiert und wie bekommt man es hin weniger wärme zu produzieren..





% \begin{figure}[h]
% 	\centering
% 		\begin{tikzpicture}
% 			% Include the SVG image
% 			\node[anchor=south west,inner sep=0] (image) at (5,0) {\includesvg[width=0.28\textwidth]{Glühbirne}};
	
% 			% Adjust positions manually
% 			\node at (-1.5,7.5) (label1) {Glaskolben};
% 			\node at (-1.45,6.0) (label2) {Glühwendel};
% 			\node at (-1.5,4.5) (label3) {Gasfüllung};
% 			\node at (-1.3,3.0) (label4) {Sockelkontakt};
% 			\node at (-1.5,1.5) (label5) {Fußkontakt};
	
% 			% Arrows pointing to the corresponding parts
% 			\draw[->,thick] (label1.east) -- (6.3,7.5);
% 			\draw[->,thick] (label2.east) -- ++(2.5cm,0) |- (6,5.7);
% 			\draw[->,thick] (label3.east) -- ++(2.5cm,0) |- (6,4.0);
% 			\draw[->,thick] (label4.east) -- ++(4cm,0) |- (6,1.5);
% 			\draw[->,thick] (label5.east) -- ++(2.5cm,0) |- (6.3,0.1);
% 		\end{tikzpicture}
% 	\caption{Glühbirne}
% 	\label{fig:Glühbirne}
% 	\end{figure}

%%%%%%%%%%%%%%%% Schaltbilder einer Glühbirne %%%%%%%%%%%%%%%%%

\begin{frame}
	\ftx{Der Wirkungsgrad}
\begin{figure}[H]
	\centering
\hspace{1cm} % Um es in der Mitte zu haben
	\begin{tikzpicture}
  \draw (0,3) to[V, i, name=I] (0,0)
		(0,3) -- (1.5,3)
		(1.5,3) to[switch, l={Schalter}] (2.5,3)
		(2.5,3) -- (4,3)
        (4,3) to[lamp, l={Glühlampe}] (4,0) 
        (4,0) -- (0,0); 

		% Draw current arrow
		\iarrmore{I}{$I$};

		% Draw voltage arrows
		\draw[-latex, thick, draw=voltage] (-0.6,2)  to node[midway,left,  color=voltage] {$U_\mathrm{q}$} (-0.6,1);
		
	\end{tikzpicture}
	\s{\caption{Beispiel Schaltung Wirkungsgrad}}
	\label{fig:Nennleistung}
\end{figure}
\end{frame}

% Beschriftung der Komponenten.

\begin{frame}
	\ftx{Der Wirkungsgrad}
\begin{figure}[H]
	\begin{tikzpicture}
		\centering
		         \node at (0,0) {\includesvg[width=1\textwidth]{Schaltkreis}};
				 \node at (-5.3,-0.2) {\shortstack[l]{Chemischer\\Energiespeicher ($E_\mathrm{el}$)}};
				 \node at (0.1,2.4) {Schalter};  
				 \node at (5.1,-0.1) {Verbraucher ($W_\mathrm{el}$)};

	\end{tikzpicture}

    \s{\caption{\textbf{Einfache Schaltung.} zeigt eine einfache elektrische Schaltung, die aus einer Batterie, einem Schalter und einer Glühlampe besteht. Durch das Schließen
	 des Schalters wird der Stromkreis geschlossen, wodurch die Glühlampe leuchtet, indem elektrische Energie in Licht und Wärme umgewandelt wird.}} 
    \label{fig:Schaltkreis}
\end{figure}
\end{frame}

\s{
%%%%%%%%%%%%%%%%%%%%%%%%%%%%%
%%% Beispiel Wirkungsgrad %%%
\begin{frame}
	\ftx{Der Wirkungsgrad}
\begin{bsp}{Berechnungs des Wirkungsgrades}{}

%Hier kommt ein Beispiel zur Berechnung des Wirkungsgrades rein.

Eine Wärmekraftmaschine entnimmt einer Wärmequelle eine Energie von \( Q_\text{zu} = \SI{5000}{\joule} \) und verrichtet dabei eine mechanische Arbeit von \( W = \SI{1200}{\joule} \).

		\vspace{0.3cm}
		\textbf{Gesucht:} Wirkungsgrad \( \eta \)

		\vspace{0.2cm}
		\textbf{Formel:}
		\[
			\eta = \frac{\text{Nutzarbeit}}{\text{Zugeführte Wärmeenergie}} \cdot 100\% = \frac{W}{Q_\text{zu}} \cdot 100\%
		\]

		\vspace{0.2cm}
		\textbf{Einsetzen:}
		\[
			\eta = \frac{1200\,\text{J}}{5000\,\text{J}} \cdot 100\% = 24\%
		\]

		\vspace{0.2cm}
		Der Wirkungsgrad beträgt \( \eta = \SI{24}{\percent} \).

\end{bsp}

\end{frame}
%%% Ende Beispiel Wirkungsgrad %%%
%%%%%%%%%%%%%%%%%%%%%%%%%%%%%%%%%%
}