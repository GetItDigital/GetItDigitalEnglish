%%%%%%%%%%%%%%%%%%%%%%%%%%%%%%%%%%%%%%%%%%%%%%%%%%%%%%%%%%%%%%%%%%%%%%%%%%%%%%%%%%%
%%%%%%%%%%%%%%%%%%%%%%%%%%%%%%%%%%%%%%%%%%%%%%%%%%%%%%%%%%%%%%%%%%%%%%%%%%%%%%%%%%%
%                               START KAPITEL 2 
%%%%%%%%%%%%%%%%%%%%%%%%%%%%%%%%%%%%%%%%%%%%%%%%%%%%%%%%%%%%%%%%%%%%%%%%%%%%%%%%%%%
%%%%%%%%%%%%%%%%%%%%%%%%%%%%%%%%%%%%%%%%%%%%%%%%%%%%%%%%%%%%%%%%%%%%%%%%%%%%%%%%%%%

\section[Grundlagen]{Grundlagen der Berechnung von Schaltvorgängen}% Ev. nur "Grundlagen" ...
\label{sec:grundlagen}
\begin{frame}\ftx{\secname}
\s{
    % Einführung
    Grundlage der Berechnung von Schaltvorgängen sind Differentialgleichungen (DGL).
    %Betrachtet werden lineare, zeitinvariante (LZI) Schaltungen mit Energiespeichern (Kapazitäten, Induktivitäten).

    % Problemstellung
    Die Aufstellung der Kirchhoffschen Regeln bei Schaltungen mit Energiespeichern führt zu DGL 
    von Spannungen an beziehungsweise Strömen in den jeweiligen Komponenten.
    Das bedeutet, dass die jeweiligen Momentanwerte von Spannungen und Strömen nicht ausreichen, 
    um den Zustand des Systems zu beschreiben.

    % Lösung
    Die Zeitverläufe von Spannungen und Strömen während Schaltvorgängen lassen sich durch Lösung 
    ihrer jeweiligen DGL bestimmen. 
}
\begin{Lernziele}{Grundlagen}
    \s{Studierende lernen:}%
    \begin{itemize}\b{\setlength\itemsep{0.7em}}
        \item{grundlegende Begriffe kennen und verstehen (Definitionen)}% Wissen
        \item{Berechnungsmethoden im Zeit- und Bildbereich (Laplace) kennen} % Wissen, Anwendung
        \item{DGLen für Schaltungsgrößen von LZI-Schaltungen aufzustellen} % Anwendung
        \item{DGL-Eigenschaften aus Schaltungseigenschaften abzuleiten} % Verstehen, Analyse
        \item{Zeitverläufe bei Schaltvorgängen zu bestimmen (DGLen zu lösen)}% Anwendung
    \end{itemize}
\end{Lernziele}

\end{frame}

%%%%%%%%%%%%%%%%%%%%%%%%%%%%%%%%%%%%%%%%%%%%%%%%%%%%%%%%%%%%%%%%%%%%%%%%%%%%%%%%%%%%%%%%%%%%%%%%%%

\subsection{Definitionen}
\label{sec:grundlagen:definitionen}
\begin{frame}\ftx{\subsecname}
\s{
    Tabelle \ref{tab:definitions} zeigt eine Übersicht wichtiger Begriffe mit Kurzform der hier verwendeten Definitionen.
    Bestimmte Begriffe werden in den folgenden Abschnitten genauer erläutert und diskutiert. 
    
    Speziell die hier verwendeten Definitionen für Transienten, Ausgleichs-, Einschwing- und Schaltvorgänge
    werden in Abschnitt \ref{sec:grundlagen:definitionen:diskussion} in Abgrenzung zur Definition in anderen Quellen 
    erläutert und die Wahl der hier getroffenen Definition begründet.
    
    Die Tabelle dient als Referenz für die Verwendung der Begriffe in diesem Modul und kann nach Art eines Glossars verwendet werden.
}
\begin{table}[h]
    \b{\renewcommand{\thefootnote}{\fnsymbol{footnote}}} % https://www.sascha-frank.com/footnote.html
    \resizeboxb{0.95\textwidth}{!}{%
    \s{\caption{Gruppierte Begriffe mit Definitionen in Kurzform}}
        \label{tab:definitions}
        \index{Schalten}\index{Schalter}%
        \index{Eigenschaft>harmonisch}\index{Eigenschaft>periodisch}\index{Eigenschaft>zeitinvariant}%
        \index{Eigenschaft>homogen}\index{Eigenschaft>linear}\index{Eigenschaft>Ordnung}%
        \index{Größe>Abklingkonstante}\index{Größe>Dämpfungskonstante}\index{Größe>Eigenfrequenz}%
        \index{Größe>Resonanzfrequenz}\index{Größe>Zeitkonstante}%
        \index{Vorgang>Ausgleichsvorgang}\index{Vorgang>Einschwingvorgang}\index{Vorgang>Schaltvorgang}\index{Vorgang>Transiente}%
        \centering
        \begin{tabular}{@{}crl@{}} % remove inter/border column space with @{}: https://tex.stackexchange.com/questions/490181/actual-split-cell-in-tabular
            \toprule
            \textbf{\rlap{G.}}& \textbf{Begriffe} & \textbf{Definitionen} \\
            \midrule
            \multirow{ 2}{*}{\rotatebox[origin=c]{90}{\textbf{allg.}}} 
            & Schalten & \glqq durch Betätigen eines Schalters in einen (Betriebs)zustand versetzen\grqq \s{\cite{duden}}\b{[\footnote{Duden online}]} \\
            & Schalter & Komponente, hier zum öffnen oder schließen einer elektr. Verbindung \\
            % Etwas schalten durch betätigen eines Schalters; ein Schalter kann nur betätigt werden, nicht geschalten. ?
            &&\\[-4pt]%\midrule
            \multirow{ 8}{*}{\rotatebox[origin=c]{90}{\textbf{Eigenschaften}}} % 6 rows, but multirow 7 for space (bigger rows inbetween)
            \s{& harmonisch$^1$        & sinusförmig, beschreibt Schwingung der Form: $y(t) = \hat{X} \cdot \sin(\omega t + \varphi)$ \\}
            & periodisch$^1$        & \glqq in gleichen Abständen, regelmäßig [auftretend, wiederkehrend]\grqq \s{\cite{duden}}\b{[\footnotemark[\value{footnote}]]}\\% Hier: in der Regel auf Zeitabstände bezogen \\
            & zeitinvariant$^1$     & invariant/unveränderlich über die Zeit\\
            & gewöhnlich$^2$  & Ableitung(en) nach einer unabhängigen Variable (z.B. Zeit $t$), Vgl. partiell\\
            & homogen$^2$     & Form gewöhnlich: $\sum a_i(t) \dt[i]\, f_i(y(t)) \overset{!}{=} 0$ \quad mit Störfunktion $b=0$ $\vphantom{\Big|}$\\
            & linear$^2$      & Form gewöhnlich: $\sum a_i(t) \dt[i]\, y(t) = b(t)$\qquad d.h. mit $\dt[i]\, y(t)$ linear $\vphantom{\Big|}$\\
            & Ordnung$^2$     & \makebox[7mm][l]{$n$} $[1]$ d.h. höchstens Ableitung der Ordnung $n$ in DGL \\
            \s{%
            &&\\[-4pt]%\midrule
            \multirow{ 5}{*}{\rotatebox[origin=c]{90}{\textbf{Größen}}} 
            & Abklingkonst.     & \makebox[7mm][l]{$\delta$} $[\mathrm{Hz}]$ Maß für die Dämpfung linear gedämpfter Schwingungssysteme\\ %\todo{Bezug DGL} 
            & Dämpfungskonst.   & \makebox[7mm][l]{$\delta$} $[\mathrm{Hz}]$ synonym zu Abklingkonstante\\ %\todo{Bezug DGL} 
            & Eigenfreq.        & \makebox[7mm][l]{$f_{0/d}$} $[\mathrm{Hz}]$ Freq. mit der ein System (un/)gedämpft selbst schwingen kann \\
            & Resonanzfreq.     & \makebox[7mm][l]{$f_0$}    $[\mathrm{Hz}]$ Freq. bei der ein System in Resonanz tritt, gleich unged. Eigenfreq. \\
            & Zeitkonstante     & \makebox[7mm][l]{$\tau$}   $[\mathrm{s}]$ Maß für Steilheit von Exponentialverläufen der Form $\mathrm{e}^{-\sfrac{t}{\tau}}$ \\ %\todo{Bezug DGL} 
            }%
            &&\\[-4pt]%\midrule
            \multirow{ 4}{*}{\rotatebox[origin=c]{90}{\textbf{Vorgänge}}} 
            & Ausgleichsvorg.   & Vorgang in einem System, dass einen stationären Zustand anstrebt \\
            \s{& Ausschwingvorg.& Ausgleichsvorgang, schwingend, exergon (spontan) \\}
            & Einschwingvorg.   & Ausgleichsvorgang, schwingend, endergon (nicht spontan) \\
            % ALTERNATIV: Ausgleichsvorgang energie unabhängig (Prozess beim Zustreben eines stationären Zustandes)
            % ALTERNATIV: Einschwingvorgang, nur wenn dieser schwingt ()
            & Schaltvorg.       & nicht stationärer Zustand nach Schalten\\% Anm.: Schließt instabile Zustände mit ein \\
            & Transiente        & Ausgleichsvorgang (synonym), Vgl. transient (adj.) \\
            %Dt. (hin)über(/durch)gehen\\% Ggt. persistent (adj.) von Lat. \textit{persistere} --verharren-- \\
            % ALTERNATIV: eher das von ... bis ... betonen? Wegen "Übergang" als freie Übersetzung. Mh...
            \s{%
            &&\\[-4pt]%\midrule
            \multirow{ 4}{*}{\rotatebox[origin=c]{90}{\textbf{Zustände}}} 
            & eingeschwungen    & stationär nach Einschwingvorgang \\
            & persistent        & stationär von Lat. \textit{persistere}, Dt. verharren, Ggt. von transient \\
            & stationär         & zeitlich konstante/periodische Größen (z.B. bei DC/AC) \cite[S. 362]{hagmann} \\
            & transient         & übergangsweise von Lat. \textit{transire}, Dt. (hin)übergehen, Ggt. von persistent \\
            }\\[-10pt]%
            \bottomrule%
            \\[-10pt]%
            \rlap{$^1$\footnotesize{allgemein} \qquad $^2$\footnotesize{bezogen auf DGL}}&&
        \end{tabular}%line break
        }% end resizebox
    \b{\renewcommand{\thefootnote}{\arabic{footnote}}} % https://www.sascha-frank.com/footnote.html
    \end{table}
\end{frame}

%------------------------------------------------------------------------------------------------%

\s{% Skript only
\subsubsection{Erläuterungen}
\label{sec:grundlagen:definitionen:erlaeuterungen}
\begin{frame}\ftx{\subsubsecname}
    %\todo{engl. Übersetzung ergänzen bei Ausgleich-, Einschwingvorgang und Transiente: engl. transient}

    \underline{ZUSTÄNDE}:

    \textbf{ausgeglichen}\index{Zustand>ausgeglichen}:
        Beschreibt einen stationären Zustand nach einem Ausgleichsvorgang.

    \textbf{eingeschwungen}\index{Zustand>eingeschwungen}:
        Beschreibt einen statiönären Zustand direkt nach einem Einschwingvorgang.
        Der Begriff wird in der Regel auch für stationäre Zustände nach Ausgleichsvorgängen verwendet, 
        bei denen keine Schwingung auftritt, und zudem unabhängig von der Energieflussrichtung.

    \textbf{stationär}\index{Zustand>stationär}: Zustandsbeschreibung 
        für zeitlich konstante Größen (z.B. bei Gleichspannungsversorgung) 
        oder zeitlich periodische Größen (z.B. bei Wechselspannungspannungs) \cite[S. 362]{hagmann}
        % Note, musss kein Gleichgewichtszustand sein. z.B. durch externe Energiezufuhr konstant gehalten (Regelung) https://link.springer.com/chapter/10.1007/978-3-642-88485-6_5

    \underline{GRÖSSEN}:

    Die Größen \textbf{Abklingkonstante}, \textbf{Dämpfungskonstante}, \textbf{Eigenfrequenz}, \textbf{Resonanzfrequenz} 
    und \textbf{Zeitkonstante} sind wichtige Kenngrößen für die Beschreibung des Schwingverhaltens von Systemen.
    Sie sind in Kapitel \ref{sec:schaltvorgaengezeitbereich:rlc} am Beispiel eines RLC-Reihenschwingkreises genauer erläutert.
    
    \underline{FÄLLE}: 
    
    Die Fallunterscheidung von \textbf{aperiodischem Fall}, \textbf{aperiodischem Grenzfall}, \textbf{periodischem Fall}
    bei linearen, zeitinvarainten, schwingungsfähigen Systemen ist in Kapitel \ref{sec:schaltvorgaengezeitbereich:rlc:fallunterscheidung}
    am Beispiel eines RLC-Reihenschwingkreises erläutert.

\end{frame}
}% Ende Skript only

%------------------------------------------------------------------------------------------------%

\s{% Skript only
\subsubsection{Diskussion und Abgrenzung}
\label{sec:grundlagen:definitionen:diskussion}
\begin{frame}\ftx{\subsubsecname}
    Dieser Abschnitt ist für das Verständnis der Modulinhalte nicht zwingend erforderlich und als Ergänzung zu verstehen.
    Da einige der in Tabelle \ref{tab:definitions} aufgeführten Begriffe in der Literatur unterschiedlich verwendet werden, ist die Wahl 
    der Definitionen hier begründet und abgegrenzt.

	Verlinkung deutsch- und englischsprachiger Wikipediaartikel [Stand 15.11.2024]:
	\begin{itemize}
        \item dt. Ausgleichsvorgang $\leftrightarrow$    engl. transient state                  % dt. Wiki.-Art.: stat. $\to$ Veränd.$\to$ stat.
        \item dt. Einschwingvorgang $\to$     engl. transient response $\to$ dt. Transiente     % dt. Wiki.-Art: Veränderung $\to$ stationär
        \item dt. Transiente $\leftrightarrow$ engl. transient response
    \end{itemize}

    \underline{VORGÄNGE}:

    \textbf{Ausgleichsvorgang}\index{Vorgang>Ausgleichsvorgang}, m.
        \begin{itemize}
            \item Weißgerber: Vorgang von stationärer Zustand nach Eingriff bis stationärer Zustand
            \item Hagmann: Vorgang ab Veränderung bis stationär %(entspricht engl. Wikipedia für transient)
            \item hier, angelehnt an Hagemann: Vorgang wenn bestimmter stationärer Zustand angestrebt wird
        \end{itemize}
        Wie Hagmann, nur dass quasi \glqq unterbrochene\grqq\ Ausgleichsvorgänge ebenfalls als Ausgleichsvorgänge bezeichnet werden.
        Diese Definition schließt Anwendungsfälle in der Leistungselektronik wie bei Hoch- und Tiefsetzstellern 
        unter der Annahme von realen Bauteilen ($R>0$) mit ein.  

    \textbf{Schaltvorgang}\index{Vorgang>Schaltvorgang}, m.

        \begin{itemize}
            \item Weißgerber: Ausgleichsvorgang nach Schalten\cite[S. 1]{weissgerber3}
                % Heißt: stationär -> schalten -> schaltvorgang, nicht stationär -> stationär
            \item Hagmann: Synonym Ausgleichsvorgang, Unmittelbar nach Schaltungsänderung[sic!], geht über in stationären Zustand\cite[S. 362]{hagmann}
                % Heißt: schalten -> schaltvorgang, nicht stationär -> bis?
            \item hier: Ausgleichsvorgang ausgelöst durch Schalten (angelehnt an Hagmann)
        \end{itemize}

        In diesem Modul bezieht sich der Begriff Schaltvorgang nur den Vorgang nach(!) dem Schalten, welche durch dieses ausgelöst werden.
        Mit der hier verwendeten Definition lassen sich vom Schalten unabhängige Ausgleichsvorgänge ausschließen. 

    % Anm. zu a: Weißgerber 3, S. 1: "Häufigste Ursache von Ausgleichsvorgängen sind die sogenannten Schaltvorgänge, das sind Ausgleichsvorgänge nach dem Schließen oder Öffnen eines Schalters im Netzwerk."
    % ... Ursache von Ausgleichsvorgängen sind Schaltvorgänge ... also Ausgleichsvorgänge ... Ouroboroesk?

    \textbf{Ausschwingvorgang}\index{Vorgang>Ausschwingvorgang}, m.

        Bezeichnet hier exergone Ausgleichsvorgänge, das heißt Ausgleichsvorgänge, die sich bei positiver Netto-\underline{Energieabfuhr} vollziehen. 
        Entsprechend Wortbedeutung im wörtlichen Sinn, ist ein Ausschwingvorgang das Gegenstück zum Einschwingvorgang.

    \textbf{Einschwingvorgang}\index{Vorgang>Einschwingvorgang}, m.

        Bezeichnet hier im engeren Sinn endergone Ausgleichsvorgänge, das heißt Ausgleichsvorgänge, die sich bei positiver Netto-\underline{Energiezufuhr} vollziehen. 
        Wird üblicherweise wie hier im weiteren Sinn synonym für alle Ausgleichsvorgänge verwendet.
        Bezieht sich hier wie üblich sowohl auf Abläufe, die schwingend als auch nicht schwingend verlaufen. [Vgl. der Worte schwingen, Schwung]
        Die Definition ist gewählt, um eine feinere Klassifizierung möglicher Ausgleichsvorgänge zu erleichtern.
    
    \textbf{Transiente}\index{Vorgang>Transiente}, w. % Vollständigkitshalber

        Hier synonym zu Ausgleichsvorgang betrachtet, frei übersetzbar mit \glqq Übergang\grqq aus dem Lateinischen.
        % Der Begriff bezeichnet beispielsweise störungsausgelöste zeitlich begrenzte nicht periodisch auftretende Vorgänge in der Regelungstechnik. % QUELLE, Netztechnik? (Zdrallek?)
        Vergleich transient (adj.) von Lat. \textit{transire} Dt. (hin-)über(/durch)-gehen, 
	    Gegenteil von persistent (adj.) von Lat. \textit{persistere} Dt. verharren.

\end{frame}
}% Ende Skript only

%%%%%%%%%%%%%%%%%%%%%%%%%%%%%%%%%%%%%%%%%%%%%%%%%%%%%%%%%%%%%%%%%%%%%%%%%%%%%%%%%%%%%%%%%%%%%%%%%%

\subsection{Berechnungsmethoden für Schaltvorgänge}
\label{sec:grundlagen:berechnungsmethoden}
\begin{frame}\ftx{\subsecname\ für Schaltvorgänge}
\s{%
    % Methoden
    Zur Berechnung von Schaltvorgängen stehen prinzipiell zwei Methoden zur Verfügung. 
    Beide Methoden basieren auf der Lösung der Differentialgleichung (DGL) einer Schaltung.
    
    Die Lösung der DGL erfolgt:
    \begin{itemize}
        \item \underline{im Zeitbereich} (durch Zerlegung in flüchtigen und eingeschwungenen Zustand) \textbf{oder}
        \item \underline{im Bildbereich} (mit Laplace-Transformation).
    \end{itemize}

    % Flussdiagramm
    Abbildung \ref{fig:plot:flussdiagrammbildbereich} stellt den Ablauf zur Berechnung im Zeit- und im Bildbereich zum Vergleich als Flussdiagramm gegenüber.

    \fu{\includegraphics{Tikz/pdf/plot_flussdiagramm_dgl_berechnungsmethoden.pdf}}{Flussdiagramm, Berechnungsmethoden für Schaltvorgänge\label{fig:plot:flussdiagrammbildbereich}}

    Die Benennung und Darstellung beider Methoden sind angelehnt an \cite[S. 51]{weissgerber3}, 
    finden sich jedoch in ähnlicher Form auch in anderen Lehrbüchern zur Elektrotechnik, Regelungstechnik und Systemtheorie. 

    % Kurze Erklärung
    Die Lösung der DGL im Zeitbereich ist die allgemein anwendbare Methode und wird in diesem Modul ausführlich behandelt.
    Der Rechenaufwand kann je nach Komplexität der DGL stark variieren.
    Die Methode basiert auf der Zerlegung in einen flüchtigen und einen eingeschwungenen Zustand.

    Die Lösung der DGL im Bildbereich mithilfe der Laplace-Transformation ist eine spezielle Methode, 
    die bei linearen, zeitinvarianten Systemen mit Energiespeichern angewendet werden kann. 
    Sie bietet sich bei verschwindenden Anfangsbedingungen (Vgl. Kapitel \ref{sec:grundlagen:zeitbereich}) an.
    In einigen Fällen kann die Methoden den Rechenaufwand reduzieren, da die transformierte DGL algebraisch gelöst werden kann.
    Die Methode wird in diesem Modul nicht weiter behandelt.
    
    Die Idee in den Bildbereich zu wechseln findet sich wieder in Abschnitt \ref{sec:grundlagen:dgl} bei der Aufstellung einer DGL 
    und in Abschnitt \ref{sec:grundlagen:zeitbereich:partikulaereloesung} bei der Anwendung komplexer Wechselstromrechnung (Vgl. Modul 7) zur Bestimmung eingeschwungener Zustände.
}%
\b{% Folie: Berechnungsmethoden für Schaltvorgänge
    % Bild: Flussdiagramm, Berechnungsmethoden für Schaltvorgänge
    Schaltvorgänge können im Zeit- und Bildbereich berechnet werden.
    \begin{center}
        \resizebox{0.75\textwidth}{!}{\includegraphics{Tikz/pdf/plot_flussdiagramm_dgl_berechnungsmethoden.pdf}}
    \end{center}
    Beide Methoden basieren auf dem Lösen von Differentialgleichungen.
}%
\end{frame}

%%%%%%%%%%%%%%%%%%%%%%%%%%%%%%%%%%%%%%%%%%%%%%%%%%%%%%%%%%%%%%%%%%%%%%%%%%%%%%%%%%%%%%%%%%%%%%%%%%

\subsection{Differentialgleichungen}
\label{sec:grundlagen:dgl}
\begin{frame}\ftx{\subsecname}
\s{% Anfang Skript only
    Zu Beginn der Berechnung von Schaltvorgängen steht die Aufstellung der Differentialgleichungen (DGL).
    Die DGL für Spannungen und Ströme basieren basieren prinzipiell auf den Kirchhoffschen Regeln: 
    \begin{align}
        \text{Knotenregel:}     & \qquad \sum_{j=1}^{n} \eci{i}_j = 0 \qquad \text{mit $n$ Stromzweigen im Knoten}\label{eq:grundlagen:kirchhoff:knotenregel} \\
        \text{Maschenregel:}    & \qquad \sum_{k=1}^{m} \ecv{u}_k = 0 \qquad \text{mit $m$ Teilspannungen in Masche}\label{eq:grundlagen:kirchhoff:maschenregel}
    \end{align}

    und den folgenden Strom-Spannungs-Beziehungen (Komponenten-Gl.) für $R$, $L$ und $C$.:
    \begin{align}
        &\text{Widerstand:}  & \ecv{u_R}(t) &= R \cdot \eci{i_R}(t) \vphantom{\dt} &&\label{eq:grundlagen:bauteilgleichung:widerstand}\\
        &\text{Induktivität:}& \ecv{u_L}(t) &= L \cdot \dt\, \eci{i_L}(t)          &&\label{eq:grundlagen:bauteilgleichung:induktivitaet}\\
        &\text{Kapazität:}   & \eci{i_C}(t) &= C \cdot \dt\, \ecv{u_C}(t)          &&\label{eq:grundlagen:bauteilgleichung:kapazitaet}
    \end{align}
    In welcher Reihenfolge die Gleichungen am geschicktesten aufgestellt, umgeformt und ineinander eingesetzt werden ist 
    abhängig von der jeweiligen Schaltungstopologie und der Zielgröße. 

    Ähnlich wie beim Lösen von DGLen 
    (Vgl. Kap. \ref{sec:grundlagen:berechnungsmethoden} und Abb. \ref{fig:plot:flussdiagrammbildbereich} zu Berechnungsmethoden)
    kann auch beim Aufstellen der DGLen sowohl im Zeitbereich als auch im Bildbereich vorgegangen werden. 
    Die Methode im Zeitbereich ist die in diesem Modul ausführlicher behandelte Methode. 
    Abbildung \ref{fig:plot:dglaufstellen} zeigt schematisch beide Vorgehensweisen im Vergleich. 

    % Bild: Aufstellungsmethoden für DGL (Zeitbereich, Bildbereich)
    \fu{\includegraphics{Tikz/pdf/plot_flussdiagramm_dgl_aufstellen.pdf}}{Flussdiagramm, Aufstellungsmethoden für Differentialgleichungen\label{fig:plot:dglaufstellen}}

    % Kirchhoff + Bauteilgleichungen
    Bei einfachen Schaltungen mit einer einzelnen Masche (reine Reihenschaltung) oder zwei einzelnen Knoten (reine Parallelschaltung)
    lassen sich die Differentialgleichungen direkt aus den Kirchhoffschen Regeln und den Bauteilgleichungen aufstellen (Zeitbereich).
    Gegebenenfalls muss die aufgestellte Gleichung noch differenziert werden, um die benötigten Ableitungen zu erhalten
    (beziehungsweise um Integrale zu entfernen). 
    Bei komplexeren Netzwerken ist der Rechenaufwand deutlich größer. 

    Eine Möglichkeit den Rechenaufwand zu minimieren, ist mit Methoden der komplexen Wechselstromrechnung zu arbeiten (Bildbereich). 
    Der Ansatz funktioniert ebenso für DC-Schaltungen, da nicht wirklich mit Frequenzen gerechnet wird. 
    Sinn und Zweck ist die Aufstellung der DGL durch algebraische Umformung der komplexen Kirchhoffschen Regeln.
    Hierbei werden die Schaltungselemente durch Impedanzen respektive Admittanzen ersetzt und die zu untersuchende Größe
    beispielsweise durch einen komplexen Spannungsteilers bestimmt. 
    Die Umformung zur DGL geschieht durch Rücktransformation der komplexen Größen in den Zeitbereich.
    $\mathrm{j}\omega$ wird dabei durch $\frac{\mathrm{d}}{\mathrm{d} t}$ ersetzt. 
    
    Der Vorteil des Ansatzes im Bildbereich ist, dass komplexe Spannungs- und Stromteiler anwendbar sind. 
    Dadurch bietet sich der Ansatz vor allem bei gemischten Reihen- und Parallelschaltungen an.  

    Ein ähnlicher Ansatz wäre das Maschenstromverfahren in Matrixform mit komplexen Größen. [Vgl. Modul 7] 
    Dieses wird an dieser Stelle jedoch nicht weiter betrachtet.

    % Beispiele
}% Ende Skript only
\b{% Folie: Schaltungselemente + Kirchhoff + Komponentengleichungen
    \includegraphics{Tikz/pdf/slide_aktive_passive_bauele2.pdf}
}%
\end{frame}

\begin{frame}\ftx{\subsecname}% 
\b{% Folie: Aufstellungsmethoden für DGL (Zeitbereich, Bildbereich)
    Mögliche Vorgehensweisen zum Aufstellen von Differentialgleichungen:\\[5pt]
    \resizebox{\textwidth}{!}{\includegraphics{Tikz/pdf/plot_flussdiagramm_dgl_aufstellen.pdf}}
}
\end{frame}

%------------------------------------------------------------------------------------------------%
\begin{frame}[t]\ftx{\subsecname aufstellen, Beispiel}
    \b{% Beamer only Beispiel, Start
    % Obere Hälfte
    \begin{minipage}{\textwidth}\centering\vspace{5pt}%
        \includegraphics{Tikz/pdf/circ_switchable_rc_charge_dc.pdf}
    \end{minipage}

    % Untere Hälfte
    \begin{minipage}{\textwidth}%
        % Slide 1
        \only<beamer:1-2| handout:1>{%
        \vspace{1em}\underline{\textbf{Zeitbereich:}}\qquad Gesucht sei $\ecv{u_C(t\geq0)}$.%
        \vspace{3pt}
        \begin{equation*}\begin{aligned}
        \begin{split}
            &t\geq0&\textbf{Kirchhoff:}&&&&&\textbf{Komponenten:}\\
                &\text{Knoten:}& \eci{i_R} = \eci{i_C} &= \eci{i} &&&& \eci{i_C} = C \cdot \dt\, \ecv{u_C}\\
                &\text{Masche:}& \ecv{u_R} + \ecv{u_C} &= \ecv{u} &&&& \ecv{u_R} = R \cdot \eci{i_R}\\[+5pt]
            \hline\\[-10pt]%
            \onslide<2->{% start overlay
                && R \cdot \eci{i} + \ecv{u_C} &= \ecv{U_q} \\
                &\textbf{DGL: }& RC \cdot \dt \ecv{u_C} + \ecv{u_C} &= \ecv{U_q} &&&&\quad\text{1. Ord.}
            }% end overlay
        \end{split}\end{aligned}
        \end{equation*}
        }%
        % Slide 2
        \only<beamer:3| handout:2>{%
        \vspace{1em}\underline{\textbf{Bildbereich:}}\qquad Gesucht sei $\ecv{u_C(t\geq0)}$.%%
        \begin{align*}
            \ecv{\underline{U}_C} &= \ecv{\underline{U}_q} \cdot \frac{ \frac{1}{\mathrm{j}\omega C} }{R + \frac{1}{\mathrm{j}\omega C} } \vphantom{\bigg|}\\
            \left(R + \frac{1}{\mathrm{j}\omega C}\right) \cdot \ecv{\underline{U}_C} &= \ecv{\underline{U}_q} \cdot \frac{1}{\mathrm{j}\omega C} \vphantom{\bigg|}\\
            \left(\mathrm{j}\omega \cdot RC + 1\right) \cdot \ecv{\underline{U}_C} &= \ecv{\underline{U}_q} \vphantom{\bigg|}\\
            \textbf{DGL}:\quad RC\cdot \dt\, \ecv{u_C} + \ecv{u_C} &= \ecv{U_q} \vphantom{\bigg|}
        \end{align*}
        }%
    \end{minipage}
    }% Beamer only Beispiel, Ende
    \end{frame}

%------------------------------------------------------------------------------------------------%

\subsubsection{Eigenschaften und allgemeine Form}
\label{sec:grundlagen:dgl:eigenschaften}
\begin{frame}\ftx{\subsubsecname}
\s{%
    % Komponenten Eingrenzung
    In allen hier untersuchten Schaltungen kommen ausschließlich folgende Komponenten zum Einsatz:
    Kapazitäten, Induktivitäten, ohmsche Widerstände, ideale Spannungsquellen, ideale Stromquellen, 
    ideale elektrische Leiter und ideale Schalter. 
    In den Schaltungen kommen keine weiteren Bauteile wie z.B. Transistoren, Dioden oder Operationsverstärker vor.
    % Hier: Bild mit Liste von Schaltsymbolen? Wie in Folien nur horizontal

    % Andere Einflüsse ausschließen
    Zur Berechnung von Schaltvorgängen wird in diesem Modul stets von idealen Bauteilen in isothermen, homogenen, isotropen Medien ausgegangen. 
    Weitere Einflüsse wie Temperaturabhängigkeiten, nichtlineare Effekte oder Ähnliches werden nicht betrachtet.

    % LZI-Systeme (Zeitinvarianz abschnittsweise), Komponenten
    Die aufgezählten Komponenten sind alle \textbf{linear}\index{Eigenschaft>Linearität} und 
    \textbf{zeitinvariant}\index{Eigenschaft>Zeitinvarianz}. 
    Die gleichen Eigenschaften gelten auch für Schaltungen, die sich nur diesen Komponenten zusammen setzen.
    Das heißt, es handelt sich um lineare, zeitinvariante Systeme (LZI-Systeme)\index{lineare, zeitinvariante Systeme (LZI-Systeme)}. 
    Die Zeitinvarianz für Schalter gilt dabei eingeschränkt (abschnittsweise) für alle Zeiträume außerhalb von Schaltzeitpunkten 
    und uneingeschränkt für alle anderen Komponenten.

    % DGL-Eigenschaften:
    Aus den Eigenschaften der Schaltungen lassen sich folgende für Eigenschaften der DGLs ableiten:%
    \begin{itemize}
        \item \makebox[4cm][l]{\textbf{gewöhnlich}}           \makebox[2cm][l]{}                (nur abhängig von einer Variable, hier der Zeit)
        \item \makebox[4cm][l]{\textbf{linear}}               \makebox[2cm][l]{}                (Linearität der Schaltungselemente)
        \item \makebox[4cm][l]{\textbf{konst. Koeffizienten}} \makebox[2cm][l]{$a_i=konst.$}    (Zeitinvarianz der Schaltungselemente)
        \item \makebox[4cm][l]{\textbf{inhomogen}}            \makebox[2cm][l]{$b(t)\neq 0$}    (bei Anregung durch Spannungs- oder Stromquellen)
        \item \makebox[4cm][l]{\textbf{homogen}}              \makebox[2cm][l]{$b(t)= 0$}       (ohne Anregung durch Spannungs- oder Stromquellen)
    \end{itemize} 

    Für eine betrachtete Schaltungsgröße $y$ in Abhängigkeit der Zeit $t$ ergibt sich daraus 
    die folgende Form für gewöhnliche, inhomogene, lineare DGLs mit konstanten Koeffizienten:

    % DGL allgemein
    \begin{align}\label{eq:grundlagen:dgl:inhomogen}
        \sum_{i=0}^{n} a_i \cdot \frac{\d^i y(t)}{\d t^i} = b(t)\\[2pt]\nonumber
        \text{mit} \quad a_i, b(t) \in \mathbb{R}; \quad i, n \in \mathbb{N}
    \end{align}

    Mit der \textbf{Ordnung}\index{Differentialgleichung>Ordnung} $n$, 
    der \textbf{Störfunktion}\index{Differentialgleichung>Störfunktion} $b(t)$ und den \textbf{konstanten Koeffizienten} $a_i$. 

}%
\b{%
    \centering%
    \includegraphics{Tikz/pdf/slide_aktive_passive_bauele.pdf}
}%
\end{frame}

%------------------------------------------------------------------------------------------------%

\subsubsection{Ordnung von Differentialgleichungen}
\label{sec:grundlagen:dgl:ordnung}
\begin{frame}\ftx{\subsubsecname}
\s{%
    Der Aufbau einer Schaltung bestimmt die Form der Differentialgleichungen (DGLs) zur Beschreibung ihrer Systemgrößen (Ströme, Spannungen).
    Die Ordnung der DGL entspricht direkt der Anzahl an Energiespeichern, welche nicht zu einem Energiespeicher zusammengefasst werden können.\cite[S. 4]{weissgerber3}
    
    % Beispiele
    Abbildung \ref{fig:plot:beispielordnungdgl} zeigt exemplarisch mehrere Schaltungen und die Ordnung der zugehörigen DGL.
}%
    \begin{figure}[H]\centering
        \begin{subfigure}{0.24\textwidth}\centering
            \resizebox{\textwidth}{!}{\includegraphics{Tikz/pdf/circ_example_ordnung_01.pdf}}
            \s{\caption*{1 Energiespeicher\newline $\rightarrow$ DGL 1. Ord.}}%
            \b{\caption*{1 En.sp. $\rightarrow$ 1. Ord.}}%
        \end{subfigure}
        \begin{subfigure}{0.24\textwidth}\centering
            \resizebox{\textwidth}{!}{\includegraphics{Tikz/pdf/circ_example_ordnung_02.pdf}}
            \s{\caption*{2 Energiespeicher\newline $\rightarrow$ DGL 2. Ord.}}%
            \b{\caption*{2 En.sp. $\rightarrow$ 2. Ord.}}%
        \end{subfigure}
        \begin{subfigure}{0.24\textwidth}\centering
            \resizebox{\textwidth}{!}{\includegraphics{Tikz/pdf/circ_example_ordnung_03.pdf}}
            \s{\caption*{2 Energiespeicher\newline $\rightarrow$ DGL 2. Ord.}}%
            \b{\caption*{2 En.sp. $\rightarrow$ 2. Ord.}}%
        \end{subfigure}
        \begin{subfigure}{0.24\textwidth}\centering
            \resizebox{\textwidth}{!}{\includegraphics{Tikz/pdf/circ_example_ordnung_04.pdf}}
            \s{\caption*{2 Energiespeicher\newline $\rightarrow$ DGL 2. Ord.}}%
            \b{\caption*{2 En.sp. $\rightarrow$ 2. Ord.}}%
        \end{subfigure}
        \s{\caption{Beispiel Schaltungen und Ordnung der Differentialgleichung}}
        \label{fig:plot:beispielordnungdgl}
    \end{figure}
\s{%
    Die Ordnung der DGL bestimmt die Anzahl der Konstanten in der allgemeinen Lösung der DGL.
}%
\b{%
    \renewcommand{\thefootnote}{\fnsymbol{footnote}}% https://www.sascha-frank.com/footnote.html
    \centering
    Anzahl Energiespeicher\footnotemark \ $=$\  Ordnung der DGL
    \footnotetext{nicht zusammenfassbar}
    \renewcommand{\thefootnote}{\arabic{footnote}}
}
\end{frame}

%%%%%%%%%%%%%%%%%%%%%%%%%%%%%%%%%%%%%%%%%%%%%%%%%%%%%%%%%%%%%%%%%%%%%%%%%%%%%%%%%%%%%%%%%%%%%%%%%%

\subsection{Berechnungsverfahren im Zeitbereich}
\label{sec:grundlagen:zeitbereich}
\begin{frame}\ftx{\subsecname}
\s{
    Die Berechnung von Schaltvorgängen im Zeitbereich basiert auf der Lösung der Differentialgleichungen (DGLs) 
    für die Schaltungsgrößen. Allgemein exestieren für DGLs unendlich viele Lösungen. 
    
    % Anfangsbedingungen
    Wichtig für die Eingrenzung der allgemeinen Lösung ist die Kenntnis der \textbf{Anfangsbedingungen}\index{Anfangsbedingungen} (AB).
    Die AB sind Werte von Schaltungsgrößen zu einem bestimmten Zeitpunkt $t_0$ während des Beobachtungszeitraums (Schaltvorgang).
    Dabei handelt es sich beiselsweise um Spannungen oder Ströme einzelner Bauteile oder um deren Ableitungen.

    % Problem und Lösung
    Für DGL $n$-ter Ordnung sind $n$ AB notwendig, um eine eindeutige Lösung zu erhalten.
    Sind die AB nicht explizit gegeben, können diese gegebenenfalls durch die Schaltungstopologie und 
    den Zustand unmittelbar vor dem Schalten hergeleitet werden. 
    Dabei ist zu unterscheiden, welche Schaltungsgrößen springen können und welche stetig sind.
        
    % Lösungsmethoden
    Da es sich bei Induktivitäten und Kapazitäten um Energiespeicher handelt,
    ist deren Verhalten zu einem Zeitpunkt $t$ abhängig von der bis dahin zu-/abgeführten Energie.

    In kurzer Gegenüberstellung gilt mit $E = \int p(t) \,\d t = \int u(t) \cdot i(t) \,\d t$ und den Gl. 
    \ref{eq:grundlagen:bauteilgleichung:induktivitaet} und \ref{eq:grundlagen:bauteilgleichung:kapazitaet}:
    \begin{align}
        \textbf{Kapazität:}     && E_{el}  &= \frac{1}{2} C \cdot \ecv{u}^2  &&\Rightarrow\qquad \ecv{u} = stetig && \text{(Speicher el. Energie)} \\[+4pt]
        \textbf{Induktivität:}  && E_{mag} &= \frac{1}{2} L \cdot \eci{i}^2  &&\Rightarrow\qquad \eci{i} = stetig && \text{(Speicher mag. Energie)}
    \end{align}
    Da Energie sich nur mit endlicher Geschwindigkeit bewegen kann, gilt \textbf{Stetigkeit} für $u_C$ und $i_L$:
    \begin{align}
        \ecv{u}_{C,t-} &= \ecv{u}_{C,t+} & 
        \eci{i}_{L,t-} &= \eci{i}_{L,t+} \\
        \text{mit}\quad 
        \lim_{\pm\infty \rightarrow t} \ecv{u}_C(t) &= \ecv{u}_{C,t\pm} & 
        \lim_{\pm\infty \rightarrow t} \eci{i}_L(t) &= \eci{i}_{L,t\pm}
    \end{align}
    Das bedeutet, dass die Spannung einer Kapazität ebenso wie der Strom einer Induktivität 
    unmittelbar vor dem Schalten gleich groß ist wie zu Beginn des folgenden Schaltvorganges.
    % Herleitung theoretisch direkt mit Komponentengleichung möglich. Sprung einer Größe heißt d/dt nicht definiert.
}% New Equations for Beamer, otherwise spacing is wrong (empty line above \align environment) with \s{}\begin{align}
\b{
    \underline{Jede DGL hat unendlich viele Lösungen.}
    \vspace{0.25cm}
    \pause

    Eine eindeutige Lösung für DGL $n$-ter Ordnung braucht $n$ \textbf{Anfangsbedingungen}.
    \vspace{0.10cm}
    \pause

    \textbf{Def. Anfangsbed.:} Wert beliebiger Schaltungsgröße $y(t)$ während Schaltvorgang.
    \vspace{0.25cm}%
    \pause%
    \\[-10pt]\noindent\rule{\textwidth}{0.4pt} % horizontale Linie \rule[height above baseline]{width}{thickness}
    
    Mit:
    \begin{align*}
        \textbf{Kapazität:}     && E_{el}  &= \frac{1}{2} C \cdot \ecv{u}^2  &&\Longrightarrow\qquad \ecv{u} = stetig && \text{(Speicher el. Energie)} \\[+4pt]
        \textbf{Induktivität:}  && E_{mag} &= \frac{1}{2} L \cdot \eci{i}^2  &&\Longrightarrow\qquad \eci{i} = stetig && \text{(Speicher mag. Energie)}
    \end{align*}%
    \pause%
    gilt unmittelbar vor und nach einem Schaltzeitpunkt:
    \begin{align*}
        \ecv{u_{C,t-}} &= \ecv{u_{C,t+}} & 
        \eci{i_{L,t-}} &= \eci{i_{L,t+}} \\
        \text{mit}\quad 
        \lim_{\pm\infty \rightarrow t} \ecv{u_C(t)} &= \ecv{u_{C,t\pm}} & 
        \lim_{\pm\infty \rightarrow t} \eci{i_L(t)} &= \eci{i_{L,t\pm}}
    \end{align*}
}
\end{frame}
%------------------------------------------------------------------------------------------------%

\subsubsection{Zerlegung von Ausgleichsvorgängen und Lösung der inhomogenen DGL}
\label{sec:grundlagen:zeitbereich:zerlegung}
\begin{frame}\ftx{\subsubsecname}
\begin{columns}
\column[t]{0.5\textwidth}
\b{%
    \makebox[5cm][r]{Zerlegung der \textbf{Systemgröße}} $y(t)$ in:\\
    \makebox[5cm][r]{\textbf{eingeschwungener Zustand}} $\ecb{y_{\mathrm{e}}}(t)$\\
    \makebox[5cm][r]{und \textbf{flüchtiger Zustand}} $\ecr{y_{\mathrm{f}}}(t)$.
    \vspace{0.5cm}%
    \pause%
}
\s{% Zerlegung: flüchtig, eingeschwungen
    Der sich ergebende Zeitverlauf einer Größe $y(t)$ kann während eines Ausgleichsvorgangs zerlegt werden
    in einen \textbf{flüchtigen Zustand}\index{Zustand>flüchtig} $y_f$ und 
    in einen \textbf{eingeschwungenen Zustand}\index{Zustand>eingeschwungen} $y_e$. 
    \begin{equation}\label{eq:grundlagen:zerlegung}
        y(t) := y_{\mathrm{f}}(t) + y_{\mathrm{e}}(t)
    \end{equation}
    % Def. Flüchtiger Zustand und Eingeschwungener Zustand
    Der eingeschwungene Zustand $y_{\mathrm{e}}$ beschreibt den stationäre Zustand, den die Größe $y$ anstrebt.
    Der flüchtige Zustand $y_{\mathrm{f}}$, auch freier Zustand genannt, repräsentiert den abklingenden Teil von $y$:
    \begin{align}
        \lim_{t \to \infty} y_{\mathrm{e}}(t) &=  \lim_{t \to \infty} y(t) & 
            y_{\mathrm{e}} &= \text{stationär}  \label{eq:grundlagen:eingeschwungen}\\
        \lim_{t \to \infty} y_{\mathrm{f}}(t) &=  0  & 
            y_{\mathrm{f}} &= y - y_{\mathrm{e}} \label{eq:grundlagen:fluechtig} 
    \end{align}
    % Bildbeschreibung
    Abbildung \ref{plot:example:fluechtig_eingeschwungen} zeigt exemplarisch einen Ausgleichsvorgang mit Zerlegung einer Größe $y(t)$ 
    in einen flüchtigen und eingeschwungenen Zustand. Die Darstellung entspricht dem Spannungsverlauf einer Kapazität 
    beim Laden mit Gleichspannung über einen ohmschen Widerstand wie in Beispiel \ref{bsp:rc:chargedc}.

    % Bild
    \begin{figure}[H]\centering
        \includegraphics{Tikz/pdf/plot_example_fluechtig_eingeschwungen.pdf}
        \caption{Ausgleichsvorgang zerlegt in eingeschwungenen und flüchtigen Zustand}
        \label{plot:example:fluechtig_eingeschwungen}
    \end{figure}%
}%
\b{%
    \includegraphics{Tikz/pdf/plot_example_fluechtig_eingeschwungen_def.pdf}%
    \pause%
}%
\s{%
    % Herleitung DGL eingeschwungen, flüchtig
    Da die Größe $y(t)$ nach Gleichung \ref{eq:grundlagen:eingeschwungen} für $t \to \infty$ in $y_e(t)$ übergeht, 
    muss $y_e(t)$ ebenfalls die DGL aus Gleichung \ref{eq:grundlagen:dgl:inhomogen} inklusive Störterm $b$ erfüllen:
    % DGL eingeschwungen
    \begin{align}\label{eq:grundlagen:dgl:eingeschwungen}
        \sum_{i=0}^{n} a_i \cdot \frac{\d^i(t)}{\d t^i}\,y_{\mathrm{e}} &= b(t)%\\\nonumber
        %\text{mit} \quad a_i, b(t) \in \mathbb{R}; \quad i, n &\in \mathbb{N}
    \end{align}
    Ziehen wir die DGL für $y_e$ aus Gl. \ref{eq:grundlagen:dgl:eingeschwungen} 
    von der allgemeinen inhomogenen DGL aus Gl. \ref{eq:grundlagen:dgl:inhomogen} ab, erhalten wir mit 
    der Zerlegung nach Gl. \ref{eq:grundlagen:zerlegung} die homogene DGL für den flüchtigen Zustand:
    % DGL flüchtig
    \begin{align}\label{eq:grundlagen:dgl:fluechtig}
        \sum_{i=0}^{n} a_i \cdot \frac{\d^i(t)}{\d t^i}\,y_{\mathrm{f}} = 0
    \end{align}
    % Erklärung - Vergleich Lösungen der inhomogenen DGL
    Mathematisch entspricht die Zerlegung des Zeitverlaufs $y(t)$ in einen flüchtigen und eingeschwungenen Zustand 
    der Zerlegung der \textbf{Lösung der inhomogenen DGL}\index{Differentialgleichung>Lösung der inhomogenen} $y$
    in die \textbf{Lösung der homogenen DGL}\index{Differentialgleichung>Lösung der homogenen} $y_{\mathrm{h}}$ (ohne Störterm) und 
    in eine \textbf{partikuläre Lösung}\index{Differentialgleichung>partikulare Lösung} $y_{\mathrm{p}}$ der inhomogenen DGL mit: 
    \begin{align}
        \text{homogene Lösung}\qquad y_{\mathrm{h}} &= y_{\mathrm{f}} \qquad \text{flüchtiger Zustand}\\
        \text{partikuläre Lösung}\qquad y_{\mathrm{p}} &= y_{\mathrm{e}} \qquad \text{eingeschw. Zustand}
    \end{align}
    %Und den zugehörigen DGL: 
    %\begin{align*}
    %    \sum_{i=0}^{n} a_i \cdot \dt[i]\, y_{\mathrlap{\mathrm{h}}\hphantom{h}}(t) &= 0 \\
    %    \sum_{i=0}^{n} a_i \cdot \dt[i]\, y_{\mathrlap{\mathrm{p}}\hphantom{h}}(t) &= b(t) \\    
    %\end{align*}
    Die Zerlegung von Ausgleichsvorgängen in einen eingeschwungenen und einen flüchtigen Zustand 
    findet sich in ähnlicher Form in den Lehrbüchern von Albach\cite{albach}, Hagmann\cite{hagmann} und Weißgerber\cite{weissgerber3}.  
}%
\column[t]{0.5\textwidth}
\b{%
    \makebox[5cm][r]{Entspricht der Zerlegung von} $y$ in:\\
    \makebox[5cm][r]{eine \textbf{partikuläre Lösung}} $\ecb{y_{\mathrm{p}}}$\\
    \makebox[5cm][r]{und die \textbf{allg. homogene Lösung}} $\ecr{y_{\mathrm{h}}}$

    \begin{equation*}
    % y - y_e = y_f mit y_e = y_p und y_f = y_h
        y(t) - 
        \underbrace{
            \ecb{y_{\mathrm{e}}}(t)
        }_{\displaystyle \ecb{y_{\mathrm{p}}}(t)}
        = \underbrace{
            \ecr{y_{\mathrlap{\mathrm{f}}\hphantom{h}}}(t)
        }_{\displaystyle \ecr{y_{\mathrm{h}}}(t)} 
    \end{equation*}%
    \pause%
    \qquad mit
    \begin{equation*}\begin{aligned}%
    % inhom. DGL - DGL eingeschwungen = hom. DGL (flüchtig)
    \begin{split}
        \sum a_i\cdot \dt[i]\ y_{\hphantom{e}} &= b(t) \\
    \mathllap{\huge\mathbf - \hspace{5pt}} 
        \ec[blue]{\sum a_i \cdot \dt[i]\ y_{\mathrlap{\mathrm{e}}\hphantom{e}}} &\ec[blue]{{}= b(t)}\\[+5pt]
    \hline\\[-15pt]
    \mathllap{\huge\mathbf = \hspace{5pt}} 
        \ec[red]{\sum a_i \cdot \dt[i]\ y_{\mathrlap{\mathrm{f}}\hphantom{e}}} &\ec[red]{{}= 0}
    \end{split}\end{aligned}
    \end{equation*}
}%
\end{columns}
\end{frame}

%------------------------------------------------------------------------------------------------%

\subsubsection{Berechnung der homogenen Lösung}
\label{sec:grundlagen:dgl:homogeneloesung}
\begin{frame}\ftx{\subsubsecname}
\b{%
    \begin{align*}
        &\textbf{Homogene DGL:}& 
            \sum_{i=0}^{n} a_i \cdot \dt[i]\, y_{\mathrm{h}} &= 0 &
            &\text{(gew., lin., konst. Koeff.)}\vphantom{\bigg|}\\[4pt]
        &\text{Ansatz:}&
            y_{\mathrm{h}}(t) &= K \cdot \mathrm{e}^{\lambda t} &
            &\text{(Exponentialfkt.)}\vphantom{\bigg|}\\
        &\text{Char. Polynom:}&
            \sum_{i=0}^{n} a_i \cdot \lambda^i &= 0 \quad\Rightarrow\quad \lambda_i = ... &
            &\text{(Eigenwerte: $\lambda$)} \\[4pt]
        &\text{1. Lösungsansatz:}&
            y_{\mathrm{h}}(t) &= \sum_{i=1}^{n} K_i \cdot \mathrm{e}^{\lambda_i t} &
            &\text{ für $\lambda_i \neq \lambda_{i-1} \ \forall \ i$} \\
        &\text{2. Lösungsansatz:}&
            y_{\mathrm{h}}(t)&= \sum_{j=1}^{m} K_j \cdot t^{j-1}  \cdot \mathrm{e}^{\lambda t} &
            &\text{ für $\lambda_j = \lambda \ \forall \ j$}\\[6pt]
        &\textbf{allg. hom. Lösung:}&
            y_{\mathrm{h}}(t)&= \ ... \ = y_{\mathrm{f}} &
            &\text{(Linearkombination)}
    \end{align*}
}%
\s{%
    % Ausgangsposition
    Für die Lösung von gewöhnlicher, linearer, homogener DGLen mit konstanten Koeffizienten der Form:
    
    \begin{equation}\label{eq:grundlagen:dgl:homogen}
        \sum_{i=0}^{n} a_i \cdot \dt[i]\, y_{\mathrm{h}} = 0
    \end{equation}

    wird in der Regel der Ansatz einer Exponentialfunktion verwendet:

    % Ansatz homogene Lösung
    \begin{equation}\label{eq:grundlagen:dglexponentialansatz}
        y_{\mathrm{h}}(t) = K \cdot \mathrm{e}^{\lambda t}
    \end{equation}

    %Eine mögliche Herleitung für den Exponentialansatz erfolgt für DGL erster Ordnung durch Trennung der Variablen.
    %Durch Integration der getrennten Variablen mit definierten Integrationsgrenzen 
    %(untere Grenzen: $t=0$ und $y(0)=y_0$; obere Grenze: $t=t_1$ und $y(t_1)=y_1$) ergibt sich \cite[S. 364]{hagmann}:
    % Ref: Hagmann, Grundlagen Elektrotechnik, AULA Verlag, Aufl. 18, 2020 --- S. 364
    % -> DGL 1. Ord. -> Trennung der Variablen -> Exponentialansatz
    %\begin{align*}
    %    \dt y_{\mathrm{h}} &= \lambda \cdot y_{\mathrm{h}} \\
    %    \frac{\d y_{\mathrm{h}}}{y_{\mathrm{h}}} &= \lambda \,\d t \\
    %    \int_{y_0}^{y_1} \frac{1}{y_{\mathrm{h}}} \,\d y_{\mathrm{h}} &= \lambda \int_{0}^{t_1} \d t \\
    %    \ln{\left(\frac{y_0}{y_1}\right)} &= \lambda \cdot t_1 \\
    %    \frac{y_0}{y_1} &=  \mathrm{e}^{\lambda t_1}\\
    %\end{align*}

    % Eigenwerte lambda
    Dabei ist $K$ eine Konstante und $\lambda$ die Lösung des charakteristischen Polynoms \index{Differentialgleichung>Charakteristische Polynom}
    und wird auch als \textbf{Eigenwert}\index{Differentialgleichung>Eigenwert} bezeichnet. 
    Durch Einsetzen des Exponential-Ansatzes in \ref{eq:grundlagen:dgl:homogen}, Differenzieren und Ausklammern:
    % Charakteristische Polynom
    \begin{align}
        %\sum_{i=0}^{n} a_i \cdot \dt[i]\, y_{\mathrm{h}}(t) &= 0 \nonumber\\
        \sum_{i=0}^{n} a_i \cdot \dt[i]\, K \cdot \mathrm{e}^{\lambda t} &= 0 \nonumber\\[2pt]
        \left(\sum_{i=0}^{n} a_i \cdot \lambda^i \right) K \cdot \mathrm{e}^{\lambda t} &= 0 \nonumber\\[4pt]
    \intertext{kann das \textbf{charakteristische Polynom}\index{Differentialgleichung>Charakteristische Polynom} aufgestellt werden:}
        \sum_{i=0}^{n} a_i \cdot \lambda^i &= 0
    \end{align}

    % Höhere Ordnung, Anzahl Eigenwerte, komplexe Werte
    Die maximale Anzahl an Eigenwerten ist gegeben durch die Ordnung einer DGL. 
    Da die Koeffizienten $a_i$ reell sind, müssen die Eigenwerte $\lambda_i$ ebenfalls reell sein, oder paarweise komplex konjugiert.

    % Konvergenz, Bedingung
    \textbf{Bedingung:} Um die homogene Differentialgleichung in Gleichung \ref{eq:grundlagen:dgl:homogen} 
    zu erfüllen, muss der Realteil aller Eigenwerte negativ sein $\Re\{\underline{\lambda}_i\}<0$. 
    Nur dann konvergieren die Exponentialterme aus dem Ansatz in Gleichung \ref{eq:grundlagen:dglexponentialansatz}
    für $t \to \infty$.

    % Homogene Lösung
    Die \textbf{allgemeine homogene Lösung} $y_{\mathrm{h}}$ einer DGL $n$-ter Ordnung ist die Linearkombination
    von $n$ linear unabhängigen Lösungen der homogenen DGL.

    Die Form der Lösung hängt davon ab, ob es sich um unterschiedliche Eigenwerte, gleiche Eigenwerte (Mehrfache Nullstellen des charakteristischen Polynoms) 
    oder paarweise komplex konjugierte Eigenwerte handelt.

    Sind alle Eigenwerte $\lambda_i$ verschieden, ergibt sich die homogene Lösung zu:
    \begin{equation}\label{eq:grundlagen:homogeneloesung:lambdaungleich}
        y_{\mathrm{h}}(t) = \sum_{i=1}^{n} K_i \cdot \mathrm{e}^{\lambda_i t} \qquad \text{für} \quad \lambda_i \neq \lambda_j \ \forall \ i,j
    \end{equation}
    Bei paarweise komplex konjugierten Eigenwerten kann diese Lösung umformuliert werden. [Herleitung in Abschnitt \ref{sec:schaltvorgaengezeitbereich:exkurse:herleitung}.]
    Für $\lambda_1 = \lambda_2^* = a + \mathrm{j} b \in \mathbb{C}$ ergibt sich für die homogene Lösung:
    \begin{equation}\label{eq:grundlagen:homogeneloesung:lambdakomplex}
        \begin{aligned}
        y_{\mathrm{h}}(t) &= K_1 \cdot \mathrm{e}^{a \cdot t} \cdot \cos(b \cdot t) + K_2 \cdot \mathrm{e}^{a \cdot t} \cdot \sin(b \cdot t) \\
        &= C_0 \cdot \mathrm{e}^{a \cdot t} \cdot \cos(b \cdot t + \varphi_0)
        \end{aligned}
    \end{equation}
    Sind alle Eigenwerte $\lambda_i$ gleich ergibt sich die homogene Lösung zu:
    \begin{equation}\label{eq:grundlagen:homogeneloesung:lambdagleich}
        y_{\mathrm{h}}(t) = \sum_{i=1}^{n} K_i \cdot t^{i-1}  \cdot \mathrm{e}^{\lambda t} \qquad \text{für} \quad \lambda = \lambda_i \ \forall \ i
    \end{equation}
    % Gemischte Nullstellen
    Bei gemischten Nullstellen ergibt sich die Lösung aus einer Kombination der obigen Lösungen wie in
    Beispiel \ref{bsp:grundlagen:linearunabhaengige_loesungen_homo_dgl} gezeigt ist.

    \begin{bsp}{Linear unabhängige Lösungen einer DGL}{grundlagen:linearunabhaengige_loesungen_homo_dgl}
        Dieses Beispiel ist nahezu unverändert übernommen aus \cite[S. 241]{albach}. 
        Gesucht ist die allgemeine Lösung der folgenden homogenen DGL 5. Ordnung:
        \begin{equation*}
            %\dt[5] y(t) + 7 \dt[4] y(t) + 26 \dt[3] y(t) + 62 \dt[2] y(t) + 8 \dt y(t) + 75 y(t) = 0
            \dt[\ecr{5}] y(t) + \ecb{7} \dt[\ecr{4}] y(t) + \ecb{26} \dt[\ecr{3}] y(t) + \ecb{62} \dt[\ecr{2}] y(t) + \ecb{8} \dt y(t) + \ecb{75} y(t) = 0
        \end{equation*}
        Das charakteristische Polynom:
        \begin{equation*}
            %\lambda^5 + 7 \cdot \lambda^4 + 26 \cdot \lambda^3 + 62 \cdot \lambda^2 + 85 \cdot \lambda + 75 = 0
            \lambda^{\ecr{5}} + \ecb{7} \cdot \lambda^{\ecr{4}} + \ecb{26} \cdot \lambda^{\ecr{3}} + \ecb{62} \cdot \lambda^{\ecr{2}} + \ecb{85} \cdot \lambda + \ecb{75} = 0
        \end{equation*}
        besitzt eine einfache und zwei doppelte, konjugierte, komplexe Nullstellen:
        \begin{equation*}
            \lambda_1 = -3, \quad \lambda_2 = \lambda_3 = -1+\mathrm{2j}, \quad \lambda_4 = \lambda_5 = -1-\mathrm{2j}
        \end{equation*}
        Mit den Ansätzen aus Gl. \ref{eq:grundlagen:homogeneloesung:lambdaungleich}, \ref{eq:grundlagen:homogeneloesung:lambdakomplex} und
        \ref{eq:grundlagen:homogeneloesung:lambdagleich} ergibt sich die allgemeine Lösung zu:
        \begin{equation*}
            y_{\mathrm{h}}(t) = K_1 \cdot \mathrm{e}^{-3 t} + (K_2 + K_3 \cdot t) \cdot \mathrm{e}^{-t} \sin(2t) + (K_4 + K_5 \cdot t) \cdot \mathrm{e}^{-t} \cos(2t)
        \end{equation*}
    \end{bsp}
}% Ende Skript only
\end{frame}

\begin{frame}[t]\ftx{Beispiel: Linear unabhängige Lösungen einer DGL}% top-alignment wegen Sprung durch \pause
\b{\vspace{5pt}% % Beispiel aus bsp-box gekürzt
    Gesucht ist die allgemeine Lösung der folgenden homogenen DGL 5. Ordnung:\footnotemark
    \begin{equation*}
        % \dt[5] y(t) + 7 \dt[4] y(t) + 26 \dt[3] y(t) + 62 \dt[2] y(t) + 8 \dt y(t) + 75 y(t) = 0
        \only<1,2| handout:0>{%
            \dt[5] y(t) + 7 \dt[4] y(t) + 26 \dt[3] y(t) + 62 \dt[2] y(t) + 8 \dt y(t) + 75 y(t) = 0
        }%
        \only<3-| handout:1>{
            \dt[\ecr{5}] y(t) + \ecb{7} \dt[\ecr{4}] y(t) + \ecb{26} \dt[\ecr{3}] y(t) + \ecb{62} \dt[\ecr{2}] y(t) + \ecb{8} \dt y(t) + \ecb{75} y(t) = 0
        }%
    \end{equation*}

    \pause%

    Das charakteristische Polynom:
    \begin{equation*}
        %\lambda^5 + 7 \cdot \lambda^4 + 26 \cdot \lambda^3 + 62 \cdot \lambda^2 + 85 \cdot \lambda + 75 = 0
        \only<2| handout:0>{%
            \lambda^5 + 7 \cdot \lambda^4 + 26 \cdot \lambda^3 + 62 \cdot \lambda^2 + 85 \cdot \lambda + 75 = 0
        }%
        \only<3-| handout:1>{%
            \lambda^{\ecr{5}} + \ecb{7} \cdot \lambda^{\ecr{4}} + \ecb{26} \cdot \lambda^{\ecr{3}} + \ecb{62} \cdot \lambda^{\ecr{2}} + \ecb{85} \cdot \lambda + \ecb{75} = 0
        }%
    \end{equation*}

    \pause\pause%

    besitzt eine einfache und zwei doppelte, konjugierte, komplexe Nullstellen:
    \begin{equation*}
        %\lambda_1 = -3, \quad \lambda_2 = \lambda_3 = -1+\mathrm{2j}, \quad \lambda_4 = \lambda_5 = -1-\mathrm{2j}
        \only<4| handout:0>{%
            \lambda_1 = -3, \quad \lambda_2 = \lambda_3 = -1+\mathrm{2j}, \quad \lambda_4 = \lambda_5 = -1-\mathrm{2j}
        }%
        \only<5-| handout:1>{%
            \lambda_1 = \ec[teal]{-3}, \quad \lambda_2 = \lambda_3 = \ec[purple]{-1}\ec[violet]{{}+{}2}\mathrm{j}, \quad \lambda_4 = \lambda_5 = \ec[purple]{-1}\ec[violet]{{}-{}2}\mathrm{j}
        }%
    \end{equation*}
    
    \pause\pause%

    Die allgemeine Lösung lautet:
    \begin{equation*}
        %y_{\mathrm{h}}(t) = K_1 \cdot \mathrm{e}^{-3 t} + (K_2 + K_3 \cdot t) \cdot \mathrm{e}^{-t} \sin(2t) + (K_4 + K_5 \cdot t) \cdot \mathrm{e}^{-t} \cos(2t)
        y_{\mathrm{h}}(t) = K_1 \cdot \mathrm{e}^{\ec[teal]{-3} t} + (K_2 + K_3 \cdot t) \cdot \mathrm{e}^{\ec[purple]{-}t} \sin(\ec[violet]{2}t) + (K_4 + K_5 \cdot t) \cdot \mathrm{e}^{\ec[purple]{-}t} \cos(\ec[violet]{2}t)
    \end{equation*}
    \footnotetext{leicht verändert aus: Manfred Albach, {\em Elektrotechnik 2}, Pearson Deutschland GmbH 3. Auflage, 2020.}
}%
\end{frame}
    

%------------------------------------------------------------------------------------------------%

\subsubsection{Berechnung der Partikulären Lösung}
\label{sec:grundlagen:zeitbereich:partikulaereloesung}
\begin{frame}\ftx{\subsubsecname}
\s{%
    Für die Berechnung einer partikulären Lösung $y_{\mathrm{p}}$ einer inhomogenen DGL 
    können verschiedene Methoden gewählt werden. 
    Eine Möglichkeit ist einen Lösungsansatz ähnlich der Anregung zu wählen. 
    
    % Lösungsansatz
    Speziell für Schaltvorgänge gilt, dass $y_{\mathrm{p}}$ dem eingeschwungenen Zustand entspricht
    wie in Abs. \ref{sec:grundlagen:zeitbereich:zerlegung} hergeleitet wurde. Daraus folgt für lineare, zeitinvariante Schaltelementen:

    \begin{itemize}
        \item Bei DC-Anregung (Gleichgrößen) ist die partikuläre Lösung ebenfalls eine Gleichgröße
        \item Bei AC-Anregung (Wechselgrößen) ist die partikuläre Lösung ebenfalls eine Wechselgröße
    \end{itemize}

    Die Bestimmung kann durch Methoden der statischen Netzwerkberechnung erfolgen. 
    Hierfür reichen die Kirchhoffschen Regeln, die Gleichungen für die Bauelemente $R$, $L$ und $C$
    und den daraus resultierenden Strom- und Spannungsteilerregeln. 
    Für Wechselgrößen werden die komplexen Wechselstrom-Äquivalente der aufgezählten Methoden verwendet.

    \textbf{Transformation von DGLen sinusförmiger Größen in den komplexen Zahlenraum:}

    Da bei AC-Anregung die DGL \ref{eq:grundlagen:dgl:eingeschwungen} für den eingeschwungenen Zustand für Wechselgrößen gilt, 
    kann die DGL in den komplexen Zahlenraum transformiert werden. Das Vorgehen entspricht der Transformation 
    der differentiellen Strom- und Spannungsbeziehungen für $R$, $L$ und $C$ zu deren komplexen Strom- und Spannungsbeziehungen
    anhand deren jeweiligen Impedanz $\underline{Z}$ respektive Admittanz $\underline{Y}$. 

    Eine sinusförmige Größe $y(t)$ mit Amplitude $\hat{Y}$ und einem Nullphasenwinkel $\varphi$ lässt sich mit komplexem Amplitudenzeiger (Phasor) $\underline{\hat{Y}}$ 
    wie folgt beschreiben [Vgl. Modul 7]:
    \begin{align*}
        y(t) &= \hat{Y} \cdot \cos(\omega t + \varphi) = \Re\left\{ \underline{\hat{Y}} \cdot \mathrm{e}^{\mathrm{j} \omega t} \right\} &
            &\Longleftrightarrow& 
            \underline{\hat{Y}} &= \hat{Y} \cdot \mathrm{e}^{\mathrm{j} \varphi} 
            \vphantom{\bigg|}\\
    % for preview comment out \intertext{}
    \intertext{Ableitungen nach der Zeit $\dt$ werden im komplexen Zahlenraum durch $\mathrm{j} \omega$ ersetzt:}
        \dt y(t) &= \hat{Y} \cdot \omega \cdot \cos(\omega t + \varphi + \frac{\pi}{2}) &
            &\Longleftrightarrow& 
            \mathrm{j} \omega \cdot  \underline{\hat{Y}} &= \hat{Y} \cdot \omega \cdot \mathrm{e}^{\mathrm{j} \left(\varphi + \frac{\pi}{2}\right)}
    \end{align*}
    Der Faktor $\mathrm{j}\omega$ entspricht rücktransformiert dem Vorfaktor $\omega$ und einer Phasenverschiebung von $\sfrac{\pi}{2}$. 
    Dadurch gilt allgemein bei DGLen sinusförmiger Größen (Index $ac$) der Form in Gleichung \ref{eq:grundlagen:dgl:inhomogen}: 
    \begin{equation}
        \sum_{i=0}^n a_i \cdot \left(\dt\right)^i y_{ac} = b_{ac}
        \qquad \Longleftrightarrow \qquad 
        \sum_{i=0}^n a_i \cdot \left(\mathrm{j}\omega\right)^i \cdot \underline{\hat{Y}} = \underline{\hat{B}}
        \label{eq:grundlagen:dgl:komplex}
    \end{equation}
    Alternativ kann auch der Imaginärteil komplexer Amplitudenzeiger zur Darstellung von sinusförmiger Wechselgrößen verwendet werden. 
    [Vergleich Modul 7 - Periodische Größen]

}%
    %Todo: Beispiel RC mit DC Anregung und mit AC Anregung gegenübergestellt.
\b{%
    Der \textbf{eingeschwungene Zustand} $y_{\mathrm{e}}$ entspricht der \textbf{partikulären Lösung} $y_{\mathrm{p}}$.
    \vspace{2mm}

    In LZI-Systemen gilt eingeschwungen:
    \begin{align*}
        \text{DC-Anregung:} && y_{\mathrm{p}}(t) &= \text{konst.} &&\text{(Gleichgröße)}\\
        \text{AC-Anregung:} && y_{\mathrm{p}}(t) &= \hat{Y} \cdot \sin(\omega t + \varphi) &&\text{(Wechselgröße)}
    \end{align*}

    Dadurch sind die bekannten \textbf{Methoden der Netzwerkberechnung anwendbar}.
    \vspace{2mm}

    \begin{itemize}
        \item[] DC: Kapazitäten wie offene Schaltungen und Induktivitäten wie Kurzschlüsse
        \item[] AC: Komplexe Wechselstromrechnung (Impedanzen, Admittanzen)
    \end{itemize}
}
\end{frame}
%------------------------------------------------------------------------------------------------%

\subsubsection{Zusammenfassung}
\label{sec:grundlagen:zeitbereich:zusammenfassung}
\begin{frame}\ftx{\subsubsecname}
\s{
    Das Berechnen von Ausgleichsvorgängen im Zeitbereich besteht zum einen im Aufstellen einer passenden DGL
    und zum anderen auf dem Lösen der aufgestellten DGL. Merkbox \ref{merk:berechnungsverfahren:zeitbereich} fasst die
    die Berechnung von Ausgleichsvorgängen im Zeitbereich zu fünf Schritten zusammen. 
    Die Schritte sind stark angelehnt die Vorgehensweise von Weißgerber \cite[S. 6]{weissgerber3}.
}
    \begin{Merksatz}{Verfahren für DGL}
        \label{merk:berechnungsverfahren:zeitbereich}%\qquad\qquad Bsp. inhomogen, linear, 2. Ord.
            \begin{itemize}\b{\setlength\itemsep{0.7em}}
            \item[1.] \makebox[4.5cm][l]{Differentialgleich. aufstellen}    (lin., gew.)    \hfill $\sum a_i\cdot \frac{\d^i y}{\d t^i} = b $
            %\item[2.] \makebox[4.5cm][l]{Flüchtiger Zustand}                (homo. Lsg.)    \hfill EW:\quad $\sum a_i \cdot \lambda^i = 0$
            %\item[] \hfill z.B. \quad $y_{\mathrm{f}} = \sum K_i \, \mathrm{e}^{\lambda_i t}$
            \item[2.] \makebox[4.5cm][l]{Flüchtiger Zustand}                (homo. Lsg.)    \hfill $y_{\mathrm{f}} \overset{*}{=} \sum K_i^{'} \cdot \mathrm{e}^{\lambda_i t}$
            \item[3.] \makebox[4.5cm][l]{Eingeschwung. Zustand}             (part. Lsg.)    \hfill $y_{\mathrm{e}} = y(t \to \infty)$
            \item[4.] \makebox[4.5cm][l]{Überlagern}                        (allg. Lsg.)    \hfill $y = y_{\mathrm{f}} + y_{\mathrm{e}}$
            \item[5.] \makebox[4.5cm][l]{Konstante(n) bestimmen}            (Anfangsbed.)   \hfill $i_L, u_C = stetig$
        \end{itemize}
        \vspace{5pt}
        \tiny%
        \begin{equation*}\,^{*}\text{mit}\qquad
            K_i^{'} = \left\{ 
            \begin{aligned}
                &K_i \hphantom{{}\cdot t^{k}}
                    \qquad\text{für}\quad\mathrlap{
                         \lambda_i \neq \lambda_j \qquad \forall\quad i,j
                    }\hspace{5cm}\text{(einfache NS)}\\
                &K_i \cdot t^{k}
                    \qquad\text{für}\quad\mathrlap{
                        \lambda_{i} = \lambda_{i-k} \quad \text{mit} \quad k \in [0,\ 1,\ \dots,\ m{-}1]
                    }\hspace{5cm}\text{($m$-fache NS, $k$-ter gleicher EW)}% k+1-ter EW bzw. ... k-ter gleicher EW
            \end{aligned} \right.
        \end{equation*}
        \normalsize%
    \end{Merksatz}
\s{
    Die Reihenfolge der Schritte 2. und 3. ist prinzipiell beliebig. 
    In der Mathematik ist die gezeigte Reihenfolge üblicher mit der Bestimmung der allgemeinen homogenen Lösung 
    vor der Lösung der partikulären Lösung(en). 
    Aus elektrotechnischer Sicht kann es sich anbieten zunächst die den eingeschwungenen Zustand (partikuläre Lösung) 
    zu bestimmen. 
}
\end{frame}


%%%%%%%%%%%%%%%%%%%%%%%%%%%%%%%%%%%%%%%%%%%%%%%%%%%%%%%%%%%%%%%%%%%%%%%%%%%%%%%%%%%%%%%%%%%%%%%%%%
\s{\b{% Auskommentiert
\subsection{Berechnungsverfahren im Bildbereich}
\label{sec:grundlagen:bildbereich}
\begin{frame}\ftx{\subsecname}
 Laplace...
\end{frame}
}}% Ende Auskommentiert
%------------------------------------------------------------------------------------------------%