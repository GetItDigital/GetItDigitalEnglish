\subsection{Einsatz einer Zenerdiode zur Spannungsstabilisierung\label{Aufg6}}
% Alt: Aufgabe 6
\Aufgabe{
  Mit Hilfe einer Zenerdiode soll ein Lastwiderstand \( R_\mathrm{L} = \mathrm{160\,\Omega} \) mit einer stabilisierten Spannung von \( U_\mathrm{A} = \mathrm{8\,V} \) versorgt werden. \\
  Dem Datenblatt der Zenderdiode ist zu entnehmen, dass bei einer Zenerspannung von \( \mathrm{-8\,V} \) ein Zenerstrom von \( \mathrm{-0{,}45\,A} \) fließt.
  
  \begin {figure} [H]
     \centering
     \begin{tikzpicture}
  \draw(0,2) 
    to[V, name=V1, v] (0,0)
    to[short, -o](5,0) -- (7,0)
    to[R, l_=$R_\mathrm{L}$] (7,2)
    to[short, -o](5,2)
    to[short] (4,2)
    to[R, l_=$R_\mathrm{V}$] (0,2);

  \draw(4,0) 
  to[zDo=$D_\mathrm{Z}$, *-*] (4,2) ;

  \draw (5,2) 
  to[open, name=ua, v^] (5,0);
  
  \varrmore{ua}{$U_\mathrm{A}$};
  \varrmore{V1}{$U_\mathrm{E}$};
\end{tikzpicture}
     \label{fig:FigZDiodeStabilisierung}
  \end {figure}

  \begin{itemize}
      \item[a)] Berechnen Sie den Vorwiderstand \( R_\mathrm{V} \), sodass sich bei einer Eingangsspannung von \( U_\mathrm{E} = \mathrm{10\,V} \) eine Ausgangsspannung von \( U_\mathrm{A} = \mathrm{8\,V} \) einstellt.
      \item[b)] Wie groß sind die Leistungen, die an \( R_\mathrm{V} \), \( R_\mathrm{L} \) und an der Diode \( D_\mathrm{Z} \) umgesetzt werden?
  \end{itemize}


%   \speech{ **Aufgabe 6**  
% Mit Hilfe einer **Zenerdiode** soll ein **Lastwiderstand \( R_L = 160Ω \)** mit einer **stabilisierten Spannung \( U_A = 8V \)** versorgt werden.  
% Dem **Datenblatt der Z-Diode** ist zu entnehmen, dass für eine **Zenerspannung von \(-8V\)** ein **Zenerstrom von \(-0,45A\)** fließt.  

% **Fragestellungen:**  
% a) **Berechnen Sie den Vorwiderstand \( R_V \)** so, dass sich bei einer **Eingangsspannung von \( U_E = 10V \)** eine **Ausgangsspannung von \( U_A = 8V \)** einstellt.  
% b) **Wie groß sind die Leistungen, die an \( R_V \), \( R_L \) und \( D_Z \) umgesetzt werden?**  



%  **Beschreibung der Abbildung**  
% Die gezeigte Schaltung ist eine **Spannungsstabilisierungsschaltung mit einer Zenerdiode** und besteht aus folgenden Komponenten:

% 1. **Eingangsspannungsquelle \( U_E \)**  
%    - Symbolisiert durch eine **Gleichspannungsquelle mit \( U_E = 10V \)**.  
%    - Diese **versorgt die gesamte Schaltung** mit Energie.  

% 2. **Vorwiderstand \( R_V \)**  
%    - Platziert **in Serie mit der Zenerdiode**.  
%    - Seine Aufgabe ist die **Begrenzung des Stroms**, um die Zenerdiode nicht zu überlasten.  

% 3. **Zenerdiode \( D_Z \)**  
%    - Schaltet sich bei **Erreichen der Zenerspannung von \(-8V\)** in den **Leitungsmodus**.  
%    - Hält die **Spannung an \( R_L \) konstant auf \( U_A = 8V \)**.  

% 4. **Lastwiderstand \( R_L \) mit \( 160Ω \)**  
%    - Parallel zur **Zenerdiode geschaltet**.  
%    - Die **Ausgangsspannung \( U_A = 8V \)** fällt über \( R_L \) ab.  
%    - Es fließt ein **Laststrom**, der sich aus **\( I_L = \frac{U_A}{R_L} \)** berechnet.  



%  **Funktionsweise der Schaltung**  
% - Wenn die **Eingangsspannung \( U_E = 10V \)** beträgt, sorgt die **Zenerdiode dafür**, dass die **Spannung an \( R_L \) stabil bei 8V bleibt**.  
% - Der **Vorwiderstand \( R_V \)** muss so gewählt werden, dass der **Gesamtstrom aufgeteilt** wird zwischen der **Zenerdiode** und dem **Lastwiderstand**.  
% - Zur **Berechnung der Leistungen** müssen die **verlorene Leistung am Vorwiderstand**, die **Verlustleistung der Zenerdiode** und die **nutzbare Leistung am Lastwiderstand** bestimmt werden.  



%  **Zusammenfassung der Aufgabe**  
% - Die Aufgabe erfordert die **Berechnung des Vorwiderstandes \( R_V \)** zur **Spannungsstabilisierung mit einer Zenerdiode**.  
% - Weiterhin soll bestimmt werden, wie sich die **elektrische Leistung** auf die **verschiedenen Schaltungselemente** verteilt.  
% - Die **Zenerdiode fungiert als Spannungsregler**, indem sie bei überschüssiger Spannung einen **Teil des Stroms aufnimmt**.  
% }

}


\Loesung{
  \begin {figure} [H]
  \centering
  \begin{tikzpicture}
  \draw(0,2) 
    to[V, name=V1, v] (0,0)
    to[short, -o](5,0) -- (8.5,0)
    to[R, l_=$R_\mathrm{L}$, v^<, name=RL] (8.5,2)
    to[short, -o](5,2)
    to[short] (4,2)
    to[R, l_=$R_\mathrm{V}$, i<_, name=RV] (0,2);

  \draw(4,0) 
  to[zDo=$D_\mathrm{D}$, *-*, i_<, name=D] (4,2) ;

  \draw (5,2) 
  to[open, name=ua, v^] (5,0);
  
  \varrmore{ua}{$U_\mathrm{A}$};
  \varrmore{V1}{$U_\mathrm{E}$};
  \varrmore{RL}{$U_\mathrm{RL}$};
  \iarrmore{RV}{${I}$};
  \iarrmore{D}{$I_\mathrm{Z}$};
  
  \draw[->,shift={(2,1)}] (150:0.5) arc (150:-150:0.5) node at(0,0){$M_\mathrm{1}$};  
  \draw[->,shift={(6.5,1)}] (150:0.5) arc (150:-150:0.5) node at(0,0){$M_\mathrm{2}$};
\end{tikzpicture}

  \label{fig:LsgZDiodeStabilisierung}
\end {figure}
\begin{itemize}
\item[a)]
\begin{align*}
  I_\mathrm{RL} &= \frac{\mathrm{8\,V}}{\mathrm{160\,\Omega}} = \mathrm{0{,}05\,A} \\
  I &= \mathrm{0{,}05\,A} + \mathrm{0{,}45\,A} = \mathrm{0{,}5\,A} \\
  U_\mathrm{RV} &= \mathrm{10\,V} - \mathrm{8\,V} = \mathrm{2\,V} \\
  R_\mathrm{V} &= \frac{\mathrm{2\,V}}{\mathrm{0,5\,A}} = \mathrm{4\,\Omega}
\end{align*}

\item[b)]
\begin{align*}
  P_\mathrm{RV} &= \mathrm{2\,V} \cdot \mathrm{0{,}5\,A} = \mathrm{1\,W} \\
  P_\mathrm{RL} &= \mathrm{8\,V} \cdot \mathrm{0{,}05\,A} = \mathrm{0{,}4\,W} \\
  P_\mathrm{DZ} &= \mathrm{8\,V} \cdot \mathrm{0{,}45\,A} = \mathrm{3{,}6\,W}
\end{align*}
\end{itemize}
Siehe: Abschnitt \ref{sec:SpezielleDioden}

% \speech{
% Lösung zu Aufgabe 6:

% Die Aufgabe behandelt eine Schaltung mit einer Zenerdiode, einem Lastwiderstand von 160 Ohm und einer stabilisierten Ausgangsspannung von 8 Volt. Es soll der Vorwiderstand \( R_V \) berechnet werden, sodass sich eine stabile Spannung von 8 Volt einstellt, sowie die Leistungen, die an den Widerständen und der Diode umgesetzt werden.

% ### Teil a: Berechnung des Vorwiderstands \( R_V \)

% Die gegebene Eingangsspannung beträgt 10 Volt, die Zenerspannung der Diode beträgt -8 Volt, und der Strom durch die Zenerdiode ist mit -0,45 Ampere gegeben.

% Zunächst wird der Strom durch den Lastwiderstand \( R_L \) berechnet:
% \[
% I_{R_L} = \frac{U_A}{R_L} = \frac{8V}{160\Omega} = 0,05A
% \]
% Der Gesamtstrom in der Schaltung setzt sich zusammen aus dem Strom durch den Lastwiderstand und dem Strom durch die Zenerdiode:
% \[
% I = I_{R_L} + I_D = 0,05A + 0,45A = 0,5A
% \]
% Die Spannung am Vorwiderstand \( R_V \) ergibt sich als Differenz zwischen Eingangsspannung und Ausgangsspannung:
% \[
% U_{R_V} = U_E - U_A = 10V - 8V = 2V
% \]
% Damit kann der Vorwiderstand berechnet werden:
% \[
% R_V = \frac{U_{R_V}}{I} = \frac{2V}{0,5A} = 4\Omega
% \]

% ### Teil b: Berechnung der Leistungen

% Die Leistung, die am Vorwiderstand umgesetzt wird, ergibt sich aus:
% \[
% P_{R_V} = U_{R_V} \cdot I = 2V \cdot 0,5A = 1W
% \]
% Die Leistung, die am Lastwiderstand \( R_L \) umgesetzt wird:
% \[
% P_{R_L} = U_A \cdot I_{R_L} = 8V \cdot 0,05A = 0,4W
% \]
% Die Verlustleistung der Zenerdiode berechnet sich aus:
% \[
% P_{D_Z} = U_A \cdot I_D = 8V \cdot 0,45A = 3,6W
% \]

% ### Beschreibung der Abbildung:

% Die Schaltung in der Abbildung zeigt eine Spannungsquelle mit 10 Volt, die über den Vorwiderstand \( R_V \) mit einer Zenerdiode in Parallelschaltung mit dem Lastwiderstand \( R_L \) verbunden ist. Der Strom durch den Vorwiderstand wird als Pfeil in die Zenerdiode eingezeichnet, wo sich der Strom in zwei Teilströme aufteilt: ein Teil fließt durch den Lastwiderstand, ein anderer durch die Zenerdiode. Die Spannungen sind ebenfalls eingezeichnet: Die Versorgungsspannung mit 10 Volt, die stabilisierte Spannung von 8 Volt über der Zenerdiode und dem Lastwiderstand sowie die Spannung über dem Vorwiderstand mit 2 Volt.

% Die Berechnungen in der Abbildung sind klar strukturiert und zeigen Schritt für Schritt, wie die Widerstandswerte und Leistungen bestimmt werden. Die Ergebnisse bestätigen die geforderte Ausgangsspannung von 8 Volt und zeigen die Leistungsverteilung in der Schaltung.
% }

}



