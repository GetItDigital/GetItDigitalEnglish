\subsection{Arbeit und Energie\label{Arbeit und Energie}}
\Aufgabe{
	
Ein Elektromotor hebt eine Masse von m=10 kg um eine Höhe von 5m.
    \begin{itemize}
        \item[\bf a)]
        Berechnen Sie die verrdichtete Arbeit.
        \item[\bf b)]
        Wieviel Energie wird benötigt, wenn der Wirkungsgrad des Motors $\eta = 80\%$ beträgt?
    \end{itemize}
}

\subsection{Höhenenergie im Stausee\label{Höhenenergie im Stausee}}
\Aufgabe{

Ein Stausee enthält \SI{2\,000\,000}{\cubic\meter} Wasser auf einer Höhe von \SI{100}{\meter}.\\
Berechne die gespeicherte potenzielle Energie.\\
\( \rho = 1000\,\mathrm{kg/m^3}, \quad g = 9{,}81\,\mathrm{m/s^2} \)

}

\subsection{Bewegungsenergie eines Autos\label{Bewegungsenergie eines Autos}}
\Aufgabe{

Ein Auto mit der Masse 1200\,kg fährt mit 72\,km/h.\\
Berechne die kinetische Energie.

}

\subsection{Elektrische Energie eines Geräts\label{Energie und eines Geräts}}
\Aufgabe{

Ein Wasserkocher mit 2000\,W läuft 3\,Minuten.

\begin{itemize}
        \item[\bf a)]
        Berechne, wie viel elektrische Energie er verbraucht hat.
        \item[\bf b)]
        Wie viel Energie ist nötig, um 5\,Liter Wasser von 20\,°C auf 100\,°C zu erwärmen?
    \end{itemize}
}

\subsection{Energie und Leistung\label{Energie und Leistung}}
\Aufgabe{

Ein Heizgerät hat eine Leistungsaufnahme von P = 2kw und wird für t = 3h betrieben.
    \begin{itemize}
        \item [\bf a)] Wie viel Energie wird verbraucht?
        \item [\bf b)] Wenn der Strompreis bei 0,30 €/kWh liegt, wie hoch sind die kosten?
    \end{itemize}
}

\subsection{Wirkungsgrad\label{Wirkungsgrad}}
\Aufgabe{

Ein Generator erzeugt eine elektrische Leistung von $P_{\text{el}}$ = 1kW,in dem er eine mechanische Leistung von $P_{\text{mech}}$ = 1,5 kW aufnimmt
   \begin{itemize}
     \item [\bf a)]Berechnen sie den Wirkungsgrad
     \item [\bf b)]Wie groß muss $P_{\text{mech}}$ sein, um $P_{\text{el}}$ = 2kw zu erzeugen?
   \end{itemize}
}   