% Schaltsymbol eines Zweitors (Vierpols)
\begin{tikzpicture}%
%   invisible boundbox
    \draw[draw=none](-2,-1.25) rectangle (+2,+1.25);% gleiche wie bei circ_eintor.tex
%   Zweitor (Vierpol)
    \node[draw=none] {Zweitor};%
    \draw(0,0) node[dipchip, hide numbers, num pins=6, no topmark, external pins width=0](C){};% Mittelpunkt des Vierpols
    \draw(C.bpin 1) to[short,-o, name=i11, i_<, !i] ++(-0.5,0) coordinate(In1);%
    \draw(C.bpin 3) to[short,-o, name=i12, i_>, !i] ++(-0.5,0) coordinate(In2);%
    \draw(C.bpin 6) to[short,-o, name=i21, i>,  !i] ++(0.5,0) coordinate(Out1);%
    \draw(C.bpin 4) to[short,-o, name=i22, i<,  !i] ++(0.5,0) coordinate(Out2);%
    \draw(In1) to[open, name=u1, v=$\underline{U}_1$, !v] (In2);%
    \draw(Out1) to[open, name=u2, v^=$\underline{U}_2$, !v] (Out2);%

    %draw arrows
    \varronly{u1};
    \varronly{u2};
    %\iarrmore{i11}{$I_{11}$};
    %\iarrmore{i12}{$I_{12}$};
    %\iarrmore{i21}{$I_{21}$};
    %\iarrmore{i22}{$I_{22}$};
    
\end{tikzpicture}%
