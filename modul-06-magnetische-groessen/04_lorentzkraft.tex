\section{Lorentzkraft\label{KapLorentzkraft}}

\begin{frame} \ftx{Lorentzkraft} \index{Lorentzkraft}
	\s{Befindet sich in einem Magnetfeld ein Leiter, der mit Strom durchflossen wird, wirkt auf ihn eine Kraft, die senkrecht zum Leiter und dem Magnetfeld steht. Diese Kraft wird Lorentzkraft genannt. Verläuft der Leiter senkrecht zu den magnetischen Feldlinien, ist die Kraft am größten.}

	\fo{width=0.45\textwidth}{height=0.8\textheight}{Lorentzkraft}{
		\put(9.2, 88){\color{current}$I$}
		\put(50, 90){\color{current}$I$}
		\put(15, 34){\color{gray}$F$}
		\put(78, 72){\color{magnetfeld}$N$}
		\put(78, 33){\color{magnetfeld}$S$}
	}{{\bf Versuchsaufbau zur Lorentzkraft.} Ablenkung eines freischwingenden stromdurchflossenen Leiters im Feld eines Permanentmagneten, wodurch sich der Leiter in die Richtung der Kraft bewegt. Die Richtung kann mit der Drei-Finger-Regel der rechten Hand bestimmt werden. Der Daumen orientiert sich dabei an der Richtung des Stromflusses, der Zeigefinger an der Richtung der magnetischen Feldlinien. Der Mittelfinger zeigt dann in die Richtung, in die die Kraft wirkt. \label{Lorentz1}}

\end{frame}

\begin{frame} \ftx{Lorentzkraft}
	\s{ % Abbildungen für Skript
		Schrittweise Erklärung zum in Abbildung \ref{Lorentz1} gezeigten Versuchsaufbau mit einem stromdurchflossenen Leiter im Feld eines Permanentmagneten:

		\begin{minipage}{0.25\textwidth}
			\foo{width=0.9\textwidth}{}{Lorentzkraft_Feldverlaeufe1}{
				\put(85, 78){\color{magnetfeld}N}
				\put(85, 15){\color{magnetfeld}S}
				\put(40, -8){\color{gray}+}
			}{\label{Lorentz2}}
		\end{minipage}
		\begin{minipage}{0.74\textwidth}
			\begin{itemize}
				\item Ein Leiter befindet sich zwischen den Polen eines Hufeisenmagneten.
				\item Der Leiter ist hier nicht stromdurchflossen, sodass das Feld des Permanentmagneten einzeln betrachtet werden kann.
				\item Die Magnetfeldlinien verlaufen außerhalb des Permanentmagneten vom Nord- zum Südpol und schließen sich innerhalb des Magneten.
			\end{itemize}
		\end{minipage}
		\begin{minipage}{0.25\textwidth}
			\foo{width=0.9\textwidth}{}{Lorentzkraft_Feldverlaeufe2}{
				\put(40, -8){\color{gray}$\Downarrow$}
			}{}
		\end{minipage}
		\begin{minipage}{0.74\textwidth}
			\begin{itemize}
				\item Nun wird der Leiter stromdurchflossen betrachtet, ohne das Feld des Permanentmagneten zu berücksichtigen.
				\item Um den stromdurchflossenen Leiter entsteht ein magnetisches Feld in Abhängikeit der Flussrichtung des Stroms (Rechte-Hand-Regel).
			\end{itemize}
		\end{minipage}
		\begin{minipage}{0.25\textwidth}
			\foo{width=0.9\textwidth}{}{Lorentzkraft_Feldverlaeufe3}{
				\put(85, 78){\color{magnetfeld}N}
				\put(85, 15){\color{magnetfeld}S}
				\put(5, 55){\color{gray}$F$}
			}{}
		\end{minipage}
		\begin{minipage}{0.74\textwidth}
			\begin{itemize}
				\item Das Feld des Permanentmagneten und das magnetische Feld des stromdurchflossenen Leiters werden nun zusammen betrachtet.
				\item Beide Felder überlagern sich, sodass auf der Rechten Seite eine Feldverstärkung und links eine Feldschwächung auftritt.
				\item Das System versucht diese Feldverzerrung auszugleichen, wodurch eine Kraft auf den Leiter wirkt, die senkrecht zum Leiter und dem Magnetfeld steht (Drei-Finger-Regel).
				\item Der Leiter bewegt sich in die Richtung, in die die Kraft wirkt.
			\end{itemize}
		\end{minipage}
		\begin{minipage}{0.25\textwidth}
			\foo{width=0.9\textwidth}{}{Lorentzkraft_Feldverlaeufe4}{
				\put(85, 78){\color{magnetfeld}N}
				\put(85, 15){\color{magnetfeld}S}
				\put(54.6, 55){\color{gray}$F$}
			}{}
		\end{minipage}
		\begin{minipage}{0.74\textwidth}
			\begin{itemize}
				\item Hier wird die Flussrichtung des stromdurchflossenen Leiters umgekehrt.
				\item Die Lorentzkraft wirkt entsprechend in die umgekehrte Richtung (Drei-Finger-Regel).
				\item  Der Leiter bewegt sich in die Richtung, in die die Kraft wirkt.
			\end{itemize}
		\end{minipage}
	}
	\b{ % Abbildungen für Folien
		\begin{columns}
			\column[c]{0.25\textwidth}
			\foo{}{width=0.8\textwidth}{Lorentzkraft_Feldverlaeufe1}{
				\put(85, 78){\color{magnetfeld}N}
				\put(85, 15){\color{magnetfeld}S}
			}{}
			\pause
			\column[c]{0.25\textwidth}
			\foo{}{width=0.8\textwidth}{Lorentzkraft_Feldverlaeufe2}{
				\put(-22, 47){\color{gray}+}
			}{}
			\pause
			\column[c]{0.25\textwidth}
			\foo{}{width=0.8\textwidth}{Lorentzkraft_Feldverlaeufe3}{
				\put(85, 78){\color{magnetfeld}N}
				\put(85, 15){\color{magnetfeld}S}
				\put(5, 55){\color{gray}$F$}
				\put(-22, 47){\color{gray}$\Rightarrow$}
			}{}
			\pause
			\column[c]{0.25\textwidth}
			\foo{}{width=0.8\textwidth}{Lorentzkraft_Feldverlaeufe4}{
				\put(85, 78){\color{magnetfeld}N}
				\put(85, 15){\color{magnetfeld}S}
				\put(54.6, 55){\color{gray}$F$}
			}{}
		\end{columns}
	}
\end{frame}

\begin{frame} \ftx{Lorentzkraft} \index{Lorentzkraft}
	\begin{minipage}{0.6\textwidth}
	\s{Zur Berechnung der Lorentzkraft kann die vektorielle oder die skalare Darstellung genutzt werden. Bei der vektoriellen Berechnung (Gleichung \ref{GlLorentzkraftvek}) bewirkt eine bewegte Einheitsladung $q$ mit der Geschwindigkeit $\vec{v}$ im Magnetfeld $\vec{B}$ die Kraft $\vec{F}$:}	
	\b{Auf eine mit der Geschwindigkeit $\vec{v}$ bewegte Einheitsladung $q$ im Magnetfeld $\vec{B}$ wirkt eine Kraft:}
		\begin{eqa}
			\vec{F} = q \cdot (\vec{v} \cdot \vec{B}) \label{GlLorentzkraftvek}
		\end{eqa}
	\end{minipage}
	\hfill%
	\begin{minipage}{0.35\textwidth}
		\foo{width=\textwidth}{width=\textwidth}{Lorentzkraft_Berechnung}{
			\put(12, 31){\color{gray}$\vec{F}$}
			\put(78, 72){\color{magnetfeld}$N$}
			\put(78, 33){\color{magnetfeld}$S$}
			\put(18, 22){\color{magnetfeld}$\vec{B}$}
			\put(74, 18){\color{current}$\vec{v}$}
		}{} 
	\end{minipage}

\end{frame}

\begin{frame} \ftx{Lorentzkraft}
	\s{Bei der skalaren Berechnung entfallen die Vektoren, stattdessen wird der Winkel sin ${\alpha}$ im Bezug auf die Geschwindigkeit $v$ und das Magnetfeld $B$ in die Rechnung einbezogen. Dabei gilt, dass die Lorentzkraft sowohl zu der Flussdichte, als auch auf der Geschwindigkeit der Ladung senkrecht steht. Der Betrag der Kraft ergibt sich bei geraden Leitern und konstantem Magnetfeld zu:}
	\b{
		\begin{eqa}
			F &= q \cdot v \cdot B  \cdot \sin {\alpha}\\ 
			F &=I\cdot \ell\cdot N \cdot B\cdot \sin{\alpha}
		\end{eqa}
	}
	\s{
		\begin{eqa}
			F &= q \cdot v \cdot B  \cdot \sin {\alpha}\\ 
			F &=I\cdot \ell\cdot N \cdot B\cdot \sin{\alpha}\label{GlLorentzkraft}
		\end{eqa}

	}

	\s{Die Gleichung \ref{GlLorentzkraft} illustriert die skalare Berechung in einem stromdurchflossenen Leiter. $\ell$ ist die wirksame Leiterlänge mit der Anzahl von $N$ parallelen Leitern, die mit dem Strom $I$ durchflossen werden. 
	$\alpha$ ist der Winkel zwischen dem Leiter und der Richtung der magnetischen Flussdichte $\vec{B}$. Stehen Leiter und Magnetfeld senkrecht zueinander, entfällt $\sin{\alpha}$.}
\end{frame}
\s{\begin{Merksatz}{Lorentzkraft}
	Die Lorentzkraft beschreibt die Kraft, die auf eine bewegte Ladung in einem Magnetfeld wirkt. Fließt die Ladung durch einen Leiter, so wirkt die Lorentzkraft senkrecht sowohl zum Stromfluss als auch zur magnetischen Flussdichte.
\end{Merksatz}}
\begin{frame} \ftx{Lorentzkraft}
	\s{ % Abbildungen für Skript
		Zur Veranschaulichung des Prinzips eines Elektromotors wird eine stromdurchflossene Leiterschleife, die beweglich auf einem drehbar gelagerten Rotor befestigt ist, im Feld eines Permanentmagneten betrachtet:

		\begin{minipage}{0.25\textwidth}
			\foo{width=0.9\textwidth}{}{Lorentzkraft_Feldverlaeufe5}{
				\put(85, 78){\color{magnetfeld}N}
				\put(85, 15){\color{magnetfeld}S}
				\put(40, -8){\color{gray}+}
			}{\label{Lorentz4}}
		\end{minipage}
		\begin{minipage}{0.74\textwidth}
			\begin{itemize}
				\item Eine Leiterschleife befindet sich auf einem drehbaren Element, dem Rotor, im Magnetfeld eines Hufeisenmagneten.
				\item Der Leiter ist hier noch nicht stromdurchflossen, sodass das Feld des Permanentmagneten einzeln betrachtet werden kann.
				\item Die Magnetfeldlinien verlaufen außerhalb des Permanentmagneten vom Nord- zum Südpol.
			\end{itemize}
		\end{minipage}
		\begin{minipage}{0.25\textwidth}
			\foo{width=0.9\textwidth}{}{Lorentzkraft_Feldverlaeufe6}{
				\put(5, 46.5){\color{magnetfeld}N}
				\put(55, 46.5){\color{magnetfeld}S}
				\put(40, -8){\color{gray}$\Downarrow$}
			}{}
		\end{minipage}
		\begin{minipage}{0.74\textwidth}
			\begin{itemize}
				\item Nun wird die Leiterschleife stromdurchflossen betrachtet, ohne das Feld des Permanentmagneten zu berücksichtigen.
				\item Um die stromdurchflossene Leiterschleife entsteht ein magnetisches Feld in Abhängikeit der Flussrichtung des Stroms (Rechte-Hand-Regel).
			\end{itemize}
		\end{minipage}
		\begin{minipage}{0.25\textwidth}
			\foo{width=0.9\textwidth}{}{Lorentzkraft_Feldverlaeufe7}{
				\put(85, 78){\color{magnetfeld}N}
				\put(85, 15){\color{magnetfeld}S}
				\put(8, 52){\color{gray}$F$}
				\put(54, 42){\color{gray}$F$}
			}{}
		\end{minipage}
		\begin{minipage}{0.74\textwidth}
			\begin{itemize}
				\item Das Feld des Permanentmagneten und das magnetische Feld der stromdurchflossenen Leiterschleife werden nun zusammen betrachtet.
				\item Die Lorentzkraft, die je Leiter senkrecht zum Leiter und dem Magnetfeld steht, wirkt auf die Leiterschleife. 
				\item Die Leiterschleife beginnt sich aufgrund der Lorentzkraft zu drehen.%
			\end{itemize}%
		\end{minipage}
		\begin{minipage}{0.25\textwidth}
			\foo{width=0.9\textwidth}{}{Lorentzkraft_Feldverlaeufe8}{
				\put(85, 78){\color{magnetfeld}N}
				\put(85, 15){\color{magnetfeld}S}
				\put(8, 40){\color{gray}$F$}
				\put(54, 56){\color{gray}$F$}
			}{}
		\end{minipage}
		\begin{minipage}{0.74\textwidth}
			\begin{itemize}
				\item Hier wird die Flussrichtung des stromdurchflossen Leiters umgekehrt.
				\item Entsprechend kehrt sich auch die Polung der stromdurchflossenen Leiterschleife und damit die Drehrichtung der Leiterschleife um (Drei-Finger-Regel).
			\end{itemize}
		\end{minipage}
	}
	\b{ % Abbildungen für Folien
		\begin{columns}
			\column[c]{0.25\textwidth}
			\foo{}{width=0.8\textwidth}{Lorentzkraft_Feldverlaeufe5}{
				\put(85, 78){\color{magnetfeld}N}
				\put(85, 15){\color{magnetfeld}S}
			}{}
			\pause
			\column[c]{0.25\textwidth}
			\foo{}{width=0.8\textwidth}{Lorentzkraft_Feldverlaeufe6}{
				\put(4, 46){\color{magnetfeld}N}
				\put(55, 46){\color{magnetfeld}S}
				\put(-22, 47){\color{gray}+}
			}{}
			\pause
			\column[c]{0.25\textwidth}
			\foo{}{width=0.8\textwidth}{Lorentzkraft_Feldverlaeufe7}{
				\put(85, 78){\color{magnetfeld}N}
				\put(85, 15){\color{magnetfeld}S}
				\put(6.5, 52){\color{gray}$F$}
				\put(53, 42){\color{gray}$F$}
				\put(-22, 47){\color{gray}$\Rightarrow$}
			}{}
			\pause
			\column[c]{0.25\textwidth}
			\foo{}{width=0.8\textwidth}{Lorentzkraft_Feldverlaeufe8}{
				\put(85, 78){\color{magnetfeld}N}
				\put(85, 15){\color{magnetfeld}S}
				\put(6.5, 40){\color{gray}$F$}
				\put(53, 56){\color{gray}$F$}
			}{}
		\end{columns}
	}
\end{frame}

\begin{frame} \ftx{Beispiel: Lorentzkraft}
	\begin{bsp}{Lorentzkraft}{}
		%\todo{Quelle: \cite{Tkotz}}
		Ein Gleichstrommotor hat im Luftspalt eine magnetische Flussdichte von $B=0,8\,\mathrm{T}$. Unter den Polen befinden sich insgesamt $N=400$ Wicklungen, die mit einem Strom von $I=10\,\mathrm{A}$ durchflossen werden. Die wirksame Leiterlänge ist $\ell=150\,\mathrm{mm}$.

		Berechnen Sie die Kraft $F$ am Umfang des Ankers.\pause
		\begin{align*}
			F & =B\cdot I\cdot \ell\cdot N                                                                                                               \\
			  & =0,8\,\tfrac{\mathrm{Vs}}{\mathrm{m}^2}\cdot 10\,\mathrm{A}\cdot0,15\,\mathrm{m}\cdot 400                                                        \\
			  & =480\,\frac{\mathrm{kg}\cdot\mathrm{m}^2\cdot\mathrm{s}\cdot\mathrm{A}\cdot\mathrm{m}}{\mathrm{s}^3\cdot\mathrm{A}\cdot\mathrm{m}^2}     = 480\,\mathrm{N}
		\end{align*}

	\end{bsp}
\end{frame}

\begin{frame} \ftx{Lorentzkraft}
	\s{Zwei parallele Leiter, die von Strom durchflossen werden, üben ebenfalls eine Lorentzkraft aufeinander aus, da jeder stromdurchflossene Leiter um sich herum ein Magnetfeld aufbaut (siehe Abbildung \ref{Lorentz5}). Ist die Stromrichtung in beiden Leitern gleich, entsteht eine Anziehungskraft, bei umgekehrter Stromrichtung wirkt die Kraft abstoßend.}
	\b{
		\begin{minipage}{0.48\textwidth}
			Kraft zwischen zwei geraden, parallelen, dünnen und unendlich langen Leitern:
			\begin{eq}
				F_{12} = \frac{\ell\cdot \mu_0\cdot I_1\cdot I_2}{2\cdot \pi\cdot r}
			\end{eq}
		\end{minipage}
		\hfill%
		\begin{minipage}{0.48\textwidth}
			\foo{}{width=\textwidth}{Lorentzkraft_parallel}{
			\put(13.5, 94){\color{current}$I$}
			\put(13.5, 32){\color{current}$I$}
			\put(5, 66){\color{gray}$F$}
			\put(31, 66){\color{gray}$F$}

			\put(57.5, 90.5){\color{current}$I$}
			\put(83.5, 94){\color{current}$I$}
			\put(51, 66){\color{gray}$F$}
			\put(89, 66){\color{gray}$F$}
		}{}
		\end{minipage}
	}
	\s{
		\fo{width=0.65\textwidth}{}{Lorentzkraft_parallel}{
			\put(14.5, 94){\color{current}$I$}
			\put(14.5, 32){\color{current}$I$}
			\put(5, 66){\color{gray}$F$}
			\put(32, 66){\color{gray}$F$}

			\put(58.5, 91){\color{current}$I$}
			\put(83.5, 94){\color{current}$I$}
			\put(52, 66){\color{gray}$F$}
			\put(90, 66){\color{gray}$F$}
		}{{\bf Lorentzkraft zwischen zwei parallelen Leitern.} Ist die Flussrichtung des Stroms gleich, ziehen sie sich an. Bei gegensätzlicher Flussrichtung stoßen sich die Leiter ab. \label{Lorentz5}}


		Für den theoretischen Spezialfall zweier geraden, parallelen, dünnen und unendlich langen Drähte gilt das Verhältnis:
		
		\begin{eq}
			F_{1,2} = \frac{\ell\cdot \mu_0\cdot I_1\cdot I_2}{2\cdot \pi\cdot r}
		\end{eq}

		Da in der Praxis diese speziellen Voraussetzungen nicht zutreffen, dient die Formel lediglich als  Näherung an reale Fälle, um die Krafteinwirkung der Lorentzkraft abschätzen zu können.
	}
\end{frame}