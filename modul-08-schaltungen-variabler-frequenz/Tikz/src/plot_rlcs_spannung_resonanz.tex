% RLC-Serienschwingkreis - Spannungskurven |U_R|, |U_L|, |U_C| von w/w0
\def\Uq{10}%    10 V
\def\R{10}%     10 Ohm
\def\L{10e-3}%  10 mH
\def\C{10e-6}%  10 uF
\def\Xk{31.623}% Xk = sqrt(L/C) = 31.623 Ohm
\def\Q{3.1623}% Güte Q = Xk/R
% w0 = 1/sqrt(LC) = 3162.278 Hz
\begin{tikzpicture}[x=1cm,y=1cm]
    \draw[draw=none] (-1.05,-1) rectangle (10.75,6.75); % boundbox (not matching width and height, don't know why) % set draw=black (debug) or draw=none (final)
    \begin{axis}[
        xlabel={$\omega/\omega_0$},
        ylabel={$|U|\ [\mathrm{V}]$},
        ylabel style={yshift=-0.2cm},
        xmin=0, xmax=2,
        ymin=0, ymax=40,
        domain=0.01:2,% division by zero if x=0
        width=12cm,
        height=8cm,
        samples=100,
        grid=both,
        xtick={0,0.25,0.5,0.75,1,1.25,1.5,1.75,2},
    ]
    % Spannungsteiler: |U_bauteil| = Uq * |Z_bauteil| / |Z| 
    % R = +Xk / Q
    % XL= +Xk *(w/w0) 
    % XC= -Xk *(w0/w)
    % |Z|= Xk * sqrt( (1/Q)^2 + (w/w0 - w0/w)^2 )
    \addplot[thick, color=blue,]        {\Uq};\label{leg_Uq}
    \addplot[thick, color=black,]       {\Uq * (1 /\Q) / (sqrt((1/\Q)^2 + (x-1/x)^2))}; \label{leg_UR}
    \addplot[thick, color=orange,domain=0:0.999]             {\Uq*(1/x-x)/ (sqrt((1/\Q)^2 + (x-1/x)^2))}; \label{leg_UX}
    \addplot[thick, color=orange,domain=1.001:2,forget plot] {\Uq*(x-1/x)/ (sqrt((1/\Q)^2 + (x-1/x)^2))};%better resolution as with abs()
    \addplot[thick, color=plotgreen,  dashed,]{\Uq * (1 * x) / (sqrt((1/\Q)^2 + (x-1/x)^2))}; \label{leg_UL}
    \addplot[thick, color=red,dashed,]{\Uq * (1 / x) / (sqrt((1/\Q)^2 + (x-1/x)^2))}; \label{leg_UC}
    \addlegendentry{$|\underline{U}_{q}|$}
    \addlegendentry{$|\underline{U}_{R}|$}
    \addlegendentry{$|\underline{U}_{X}|$}
    \addlegendentry{$|\underline{U}_{L}|$}
    \addlegendentry{$|\underline{U}_{C}|$}
    \end{axis}
\end{tikzpicture}
