\subsection{Ausgangsspannung einer zweistufigen OP-Schaltung bestimmen\label{Aufg15}}
% Alt: Aufgabe 15
\Aufgabe{
    Gegeben ist die folgende Schaltung. Berechnen Sie die Ausgangsspannung \( U_\mathrm{a} \). \\
Die Widerstandswerte lauten: \\
$
R_\mathrm{1} = \mathrm{10\,k\Omega},\\
R_\mathrm{2} = \mathrm{30\,k\Omega},\\
R_\mathrm{3} = \mathrm{30\,k\Omega},\\
R_\mathrm{4} = \mathrm{10\,k\Omega}
$
    \begin {figure} [H]
   \centering
   \begin{tikzpicture}
\ctikzset{tripoles/en amp/input height=0.45}
\draw (0,0)node[en amp](E){N1}
    (E.out)
    (E.-)
    (E.+);
    \draw (E.out) to[short, -*] (2,0) to[R,l=$R_3$] (4,0) to[short, -*] (4,0.5);
    \draw (E.-) to[short, -*] (-2,0.5) --(-3,0.5);
    \draw (-2,0.5) to[short] (-2,2.5) to[R, name=R, l=$R_2$] (2,2.5) -- (2,0);

    \draw (-6,0.5) to[R, l=$R_1$, o-] (-3,0.5);
    \draw (-5, -0.5) to[R, l=$R_1$, o-] (-2,-0.5);

    \draw (-6,-2) to[short, o-o](-5,-2);
    \draw (E.+) to[short, -*] (-2,-0.5) to[R, l=$R_2$, -*] (-2,-2)node[below, ground]{};
    \draw (2.5,-2) to[short, -o] (-5,-2);
    \draw (2.5,-2) to[short, -*] (4,-2)node[below, ground]{} to[short, -o](8.5,-2);

        \draw (6,0)node[en amp](E){N2}
    (E.out)
    (E.-)
    (E.+);
    \draw (E.-) -- (4,0.5) -- (4,2.5) to[R,l=$R_4$] (8,2.5) to[short, -*] (8,0);
    \draw (E.out) to[short, -o] (8.5,0);
    \draw (E.+) -- (4,-0.5) -- (4,-2);
    
    \draw (8.5,0) to[open, v, name=ua] (8.5,-2);
    \draw (-6,0.5) to[open, v, name=ue1] (-6,-2);
    \draw (-5,-0.5) to[open, v, name=ue2] (-5,-2);
    
    \varrmore{ue1}{$U_\mathrm{E1}$};
    \varrmore{ue2}{$U_\mathrm{E2}$};
    \varrmore{ua}{$U_\mathrm{a}$};
\end{tikzpicture}
   \label{fig:FigZweistufigerOPV}
    \end {figure}
 %   \speech{
 %       Aufgabe 15: Zweistufige Operationsverstärkerschaltung  
 %       
 %       Gegeben ist eine Schaltung mit zwei Operationsverstärkern N1 und N2.  
 %       Die Widerstände haben folgende Werte:  
 %       R1 = 10 Kiloohm, R2 = 30 Kiloohm, R3 = 30 Kiloohm, R4 = 10 Kiloohm.  
 %       
 %       Die Eingangsspannungen U E1 und U E2 werden über zwei Widerstände R1 an den invertierenden Eingang des ersten Operationsverstärkers N1 geführt.  
 %       Der nichtinvertierende Eingang von N1 ist mit Masse verbunden.  
 %       Die Rückkopplung erfolgt über den Widerstand R2.  
 %       
 %       Der Ausgang des ersten Operationsverstärkers ist über den Widerstand R3  
 %       mit dem invertierenden Eingang des zweiten Operationsverstärkers N2 verbunden.  
 %       Der nichtinvertierende Eingang von N2 ist ebenfalls mit Masse verbunden.  
 %       Die Rückkopplung erfolgt über den Widerstand R4.  
 %       
%        Gesucht ist die Ausgangsspannung U A in Abhängigkeit der Eingangsspannungen U E1 und U E2.  
%        }
        
}

\Loesung{
    Die Ausgangsspannung nach dem ersten Operationsverstärker (\( U_\mathrm{a1} \)) ergibt sich zu:
    \begin{align*}
        U_\mathrm{a1} &= (U_\mathrm{E1} - U_\mathrm{E2}) \cdot \frac{R_\mathrm{2}}{R_\mathrm{1}} \\
        U_\mathrm{a1} &= \frac{\mathrm{30\,k\Omega}}{\mathrm{10\,k\Omega}} \cdot (U_\mathrm{E1} - U_\mathrm{E2})
    \end{align*}
    Daraus ergibt sich für die gesamte Ausgangsspannung (\( U_\mathrm{a} \)):
    \begin{align*}
        U_\mathrm{a} &= - \frac{R_\mathrm{4}}{R_\mathrm{3}} \cdot U_\mathrm{a1} \\
        U_\mathrm{a} &= - \frac{\mathrm{10\,k\Omega}}{\mathrm{30\,k\Omega}} \cdot \frac{\mathrm{30\,k\Omega}}{\mathrm{10\,k\Omega}} \cdot (U_\mathrm{E1} - U_\mathrm{E2}) \\
        U_\mathrm{a} &= U_\mathrm{E2} - U_\mathrm{E1}
    \end{align*}


%    \speech{
%        Lösung für Aufgabe 15.
%        
%        Gegeben ist eine zweistufige Operationsverstärkerschaltung. Die Ausgangsspannung des ersten Operationsverstärkers, U A1, wird mit folgender Formel berechnet:
%        
%        U A1 ist gleich Klammer auf U E1 minus U E2 Klammer zu mal R 2 durch R 1.
%        
%        Einsetzen der Werte. U A1 ist gleich 30 durch 10 mal Klammer auf U E1 minus U E2 Klammer zu.
%        
       % Daraus ergibt sich für die gesamte Ausgangsspannung U A:
       % 
        %U A ist gleich minus R 4 durch R 3 mal U A1.
       % 
        %Einsetzen der Werte. U A ist gleich minus 10 durch 30 mal 30 durch 10 mal Klammer auf U E1 minus U E2 Klammer zu. 

%        Nach Kürzen bleibt. U A ist gleich U E1 plus U E2.
%        
%        Die Ausgangsspannung ist somit die Summe der beiden Eingangsspannungen U E1 und U E2.
%        }
Siehe: Abschnitt \ref{fig: Verschaltung Verstaerker} und Tabelle \ref{tab:Grundschaltungen}
}