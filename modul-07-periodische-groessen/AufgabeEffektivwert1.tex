\subsection{Effektivwert 1}
\Aufgabe{

    Gegeben sei eine sinusförmige Spannung
    
    \begin{eq}
        u(t) = 150\ V \cdot \sin(100\pi t)  \nonumber
    \end{eq}

    wobei $t$ in Sekunden und die Spannung $u(t)$ in Volt angegeben sind.\\

    Berechnen Sie für diese Spannung:

    \begin{itemize}
        \item[\bf a)] Den \textbf{arithmetischen Mittelwert} $\bar{u}$ der Spannung über eine Periode.
        \item[\bf b)] Den \textbf{Effektivwert} $U_{Eff}$ der Spannung.
    \end{itemize}
}

\Loesung{
	\begin{itemize}

	\item[\bf a)]
        Der arithmetische Mittelwert einer Funktion u(t) über eine Periode T ist gegeben durch:
        \begin{eq}
            \bar{u} = \frac{1}{T} \int_0^T u(t) dt  \nonumber
        \end{eq}
        
        Da $u(t)$ eine Sinusfunktion darstellt und die Sinuswelle über eine Periode symmetrisch ist, heben 
        sich die positiven und negativen Teile der Welle gegenseitig auf. Daher ist der arithmetische 
        Mittelwert für jede Sinuswelle immer null:
        
        \begin{eq}
            \bar{u} = 0 \nonumber
        \end{eq}

    \item[\bf b)]
        Der Effektivwert einer Sinusspannung $u(t)$ wird durch die folgende Formel berechnet:
        \begin{eq}
            U_{Eff} = \frac{\hat{U}}{\sqrt{2}}   \nonumber
        \end{eq}
        
        In diesem Fall ist $\hat{U} = 150\ V$, die Amplitude der Spannung.
        
        \begin{eq}
            U_{Eff} = \frac{150\ V}{\sqrt{2}} \approx \frac{150\ V}{1.414} \approx 106.1\ V  \nonumber
        \end{eq}
	
\end{itemize}

}