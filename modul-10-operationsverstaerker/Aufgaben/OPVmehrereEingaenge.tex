\subsection{Ausgangsspannung eines Operationsverstärkers mit mehreren Eingängen berechnen\label{Aufg5}}
% Alt: Aufgabe 5
\Aufgabe{
    Gegeben ist die abgebildete Schaltung mit einem idealen Operationsverstärker. Die Betriebsspannung beträgt 
    $\mathrm{\pm15\,V}$.
    \begin{itemize}
    \item[a)] Die Widerstände sind alle gleich, es gilt $R_1 = R_2 = R_3 = R_\mathrm{F} = 10\,\mathrm{k\Omega}$. Die Spannung $U_\mathrm{E1} = \mathrm{1\,V}$, $U_\mathrm{E2} = \mathrm{2\,V}$, 
    $U_\mathrm{E3} = \mathrm{3\,V}$. Wie hoch ist die Spannung am Ausgang $U_\mathrm{A}$?
    \item[b)] Die Eingangsspannungen bleiben gleich groß, nur die Widerstände betragen nun $R_\mathrm{F} = \mathrm{10\,k\Omega}$, $R_\mathrm{1} = \mathrm{2\,k\Omega}$,
     $R_\mathrm{2} = \mathrm{5\,k\Omega}$, $R_\mathrm{3} = \mathrm{10\,k\Omega}$. Wie hoch ist die Spannung am Ausgang $U_\mathrm{A}$?
    \item[c)] Die Eingangsspannungen bleiben gleich groß, nur die Widerstände betragen nun $R_\mathrm{F} = \mathrm{10\,k\Omega}$, $R_\mathrm{1} = \mathrm{10\,k\Omega}$, 
    $R_\mathrm{2} = \mathrm{5\,k\Omega}$, $R_\mathrm{3} = \mathrm{2\,k\Omega}$. Wie hoch ist die Spannung am Ausgang $U_\mathrm{A}$?
    \end{itemize}
    \begin {figure} [H]
   \centering
   \begin{tikzpicture}
\ctikzset{tripoles/en amp/input height=0.45}
\draw (0,0)node[en amp, en amp text A](E){}
    (E.out)
    (E.-)
    (E.+);
    \draw (E.out) to[short, -*] (2,0) to[short, -o] (2.5,0);
    \draw (E.-) to[short, -*] (-2,0.5) to[short, -*] (-3,0.5) to[short, -*] (-3,1.5) -- (-3,2.5)
    to[R,l_=$R_\mathrm{1}$] (-5,2.5) to[short, -o] (-7,2.5);
    \draw (-2,0.5) to[short] (-2,2.5) to[R, name=R, l=$R_\mathrm{F}$] (2,2.5) -- (2,0);
    \draw (-3,1.5) to[R, l_=$R_\mathrm{2}$] (-5,1.5) to[short, -o] (-6,1.5);
    \draw (-3,0.5) to[R,l_=$R_\mathrm{3}$, -o] (-5,0.5);

    
    \draw (-7,-2) to[short, o-o] (-6,-2) to[short, -o](-5,-2);
    \draw (E.+) -- (-2,-0.5) to[short, -*] (-2,-2)node[ground]{};
    \draw (2.5,-2) to[short, o-o] (-5,-2);
    
    \draw (2.5,0) to[open, v, name=ua] (2.5,-2);
    \draw (-7,2.5) to[open, v, name=ue1] (-7,-2);
    \draw (-6,1.5) to[open, v, name=ue2] (-6,-2);
    \draw (-5,0.5) to[open, v, name=ue3] (-5,-2);
    
    

    \varrmore{ue1}{$U_\mathrm{E1}$};
    \varrmore{ue2}{$U_\mathrm{E2}$};
    \varrmore{ue3}{$U_\mathrm{E3}$};
    \varrmore{ua}{$U_\mathrm{A}$};
\end{tikzpicture}

   \label{fig:FigOPVmehrereEingaenge}
    \end {figure}

%    \speech{
%        Aufgabe 5: Operationsverstärkerschaltung mit mehreren Eingängen  
%        
%        Gegeben ist eine Schaltung mit einem idealen Operationsverstärker.  
%        Die Betriebsspannung beträgt plus minus 15 Volt.  
%        
%        Die Schaltung besteht aus mehreren Eingangsspannungen U E1, U E2 und U E3,  
%        die über die Widerstände R1, R2 und R3 an den invertierenden Eingang des Operationsverstärkers angeschlossen sind.  
%        Ein Rückkopplungswiderstand R F verbindet den Ausgang mit dem invertierenden Eingang.  
%        Der nichtinvertierende Eingang ist mit Masse verbunden.  
%        
%        Es sind drei Teilaufgaben zu lösen:  
%        
%        a) Berechnung der Ausgangsspannung U A, wenn alle Widerstände gleich sind:  
%           R1 = R2 = R3 = R F = 10 Kiloohm,  
%           und die Eingangsspannungen U E1 = 1 Volt, U E2 = 2 Volt, U E3 = 3 Volt.  
%        
%        b) Berechnung der Ausgangsspannung U A, wenn die Eingangsspannungen gleich bleiben,  
%           aber die Widerstände unterschiedliche Werte haben:  
%           R F = 10 Kiloohm, R1 = 2 Kiloohm, R2 = 5 Kiloohm, R3 = 10 Kiloohm.  
%        
%        c) Berechnung der Ausgangsspannung U A, wenn die Eingangsspannungen gleich bleiben,  
%           aber die Widerstände folgende Werte haben:  
%           R F = 10 Kiloohm, R1 = 10 Kiloohm, R2 = 5 Kiloohm, R3 = 2 Kiloohm.  
%        }
%        
}

\Loesung{
	Formel:
    \begin{eqa}
        U_\mathrm{A} = - (U_\mathrm{E1} \cdot \frac{R_\mathrm{F}}{R_1} + U_\mathrm{{E2}} \cdot \frac{R_\mathrm{F}}{R_2} + U_\mathrm{E3} \cdot \frac{R_\mathrm{F}}{R_3})
    \end{eqa}
    \begin{itemize}
        \item [ \bf a)] 
        \begin{align*}
            U_\mathrm{A} =\mathrm{- (1\,V \cdot \frac{10\,k\Omega}{10\,k\Omega} + 2\,V \cdot \frac{10\,k\Omega}{10\,k\Omega} + 3\,V \cdot \frac{10\,k\Omega}{10\,k\Omega}) = -6\,V}
        \end{align*}
        \item [ \bf b)] 
        \begin{align*}
            U_\mathrm{A} =\mathrm{ - (1\,V \cdot \frac{10\,k\Omega}{2\,k\Omega} + 2\,V \cdot \frac{10\,k\Omega}{5\,k\Omega} + 3\,V \cdot \frac{10\,k\Omega}{10\,k\Omega}) = -12\,V}
        \end{align*}
        \item [ \bf c)] 
        \begin{align*}
            U_\mathrm{A} =\mathrm{ - (1\,V \cdot \frac{10\,k\Omega}{10\,k\Omega} + 2\,V \cdot \frac{10\,k\Omega}{5\,k\Omega} + 3\,V \cdot \frac{10\,k\Omega}{2\,k\Omega}) = -20\,V}
        \end{align*}
        Anmerkung: $-20\,\mathrm{V}$ nicht möglich, da die Betriebsspannung $\mathrm{\pm 15\,V}$ beträgt. Dies führt dazu, dass der niedrigste Wert $\mathrm{-15\,V}$ sein kann.
    \end{itemize}
    Siehe: Abschnitt \ref{fig: Verschaltung Verstaerker} und Tabelle \ref{tab:Grundschaltungen}

%    \speech{
%        Berechnung der Ausgangsspannung einer Operationsverstärkerschaltung mit mehreren Eingängen  
%        
%        Die gegebene Schaltung ist eine invertierende Verstärkerschaltung mit drei Eingangsspannungen  
%        und einem Rückkopplungswiderstand.  
%        
%        Die allgemeine Formel zur Berechnung der Ausgangsspannung lautet:  
%        
%        U A = - ( U E1 mal R F durch R1 + U E2 mal R F durch R2 + U E3 mal R F durch R3 )  
%        
%        Drei verschiedene Fälle werden berechnet:  
%        
%        a) Alle Widerstände sind gleich, es gilt:  
%        R1 = R2 = R3 = R F = 10 Kiloohm.  
%        Die Eingangsspannungen sind:  
%        U E1 = 1 Volt, U E2 = 2 Volt, U E3 = 3 Volt.  
%        Einsetzen der Werte ergibt:  
%        U A = - ( 1 Volt mal 10 Kiloohm durch 10 Kiloohm + 2 Volt mal 10 Kiloohm durch 10 Kiloohm + 3 Volt mal 10 Kiloohm durch 10 Kiloohm )  
%        = - 6 Volt.  
%        
%        b) Unterschiedliche Widerstände:  
%        R F = 10 Kiloohm, R1 = 2 Kiloohm, R2 = 5 Kiloohm, R3 = 10 Kiloohm.  
%        Einsetzen der Werte ergibt:  
%        U A = - ( 1 Volt mal 10 Kiloohm durch 2 Kiloohm + 2 Volt mal 10 Kiloohm durch 5 Kiloohm + 3 Volt mal 10 Kiloohm durch 10 Kiloohm )  
%        = - 12 Volt.  
%        
%        c) Weitere Variation der Widerstände:  
%        R F = 10 Kiloohm, R1 = 10 Kiloohm, R2 = 5 Kiloohm, R3 = 2 Kiloohm.  
%        Einsetzen der Werte ergibt:  
%        U A = - ( 1 Volt mal 10 Kiloohm durch 10 Kiloohm + 2 Volt mal 10 Kiloohm durch 5 Kiloohm + 3 Volt mal 10 Kiloohm durch 2 Kiloohm )  
%        = - 20 Volt.  
%        
%        Anmerkung: Da die Betriebsspannung der Schaltung plus minus 15 Volt beträgt,  
%        kann die tatsächliche Ausgangsspannung den Wert von -20 Volt nicht erreichen.  
%        Sie wird stattdessen auf -15 Volt begrenzt.  
%        }
        
}