\newvideofile{Ladung}{Die elektrische Ladung}
\section{Die elektrische Ladung}

\begin{frame}
	\ftx{Lernziele: Die Elektrische Ladung}

	\begin{Lernziele}{Die Elektrische Ladung}
        Die Studierenden können
        \begin{itemize}
			\item die Eigenschaften elektrischer Ladungen sowie im Zusammenhang stehende physikalische Phänomene beschreiben
			\item elektrische Felder beschreiben und für einfache Ladungsanordnungen berechnen
			\item mit dem Coulomb´schen Gesetz Kräfte auf Ladungen berechnen
	%		\item die elektrische Spannung aus anderen Feldgrößen errechnen 
	%		\item Bestimmung des Zusammenhanges zwischen Ladung, elektrischer Flussdichte, Feldstärke und Permittivität im 
	%		Homogenfeld
        \end{itemize}
    \end{Lernziele}

	\speech{Lernziele_ladung}{1}{In diesem Video wird die elektrische Ladung eingeführt. Ziel ist es, dass ihr im Anschluss an dieses Video die Eigenschaften 
	elektrischer Ladungen sowie die von ihr ausgehenden Phänomene beschreiben könnt. Dazu gehören unter anderem die elektrischen Felder sowie Kräfte, welche mit Hilfe des Coulombschen Gesetzes berechnet werden können.} 
\end{frame}



\subsection{Der Atomaufbau}

\s{Viele physikalische Prozesse sind nur durch klassische Phönomene aus der Mechanik nicht zu erklären. 
Bereits etwa 600 v. Chr. entdeckte der griechische Philosoph Thales von Milet, dass Bernstein, wenn er an einem
 Fell gerieben wird, leichte Objekte wie Federn anzieht. In weiteren Experimenten konnte bedeutend 
 später nachgewiesen werden, dass es neben der Gravitationskraft eine weitere, sogenannte
  \textbf{elektrische Kraft} gibt. Diese kann sowohl anziehend als auch abstoßend wirken. Als Ursache 
  dieser Kraftwirkungen werden \textbf{elektrische Ladungen} postuliert und als (historisch willkürlich) 
  \textbf{positiv} und \textbf{negativ} festgelegt. Gleichnamige Ladungen stoßen sich ab, ungleichnamige
   Ladungen hingegen ziehen sich an.
	
}


\begin{frame}
	\ftx{Atomaufbau - Das Bohr´sche Atommodell}
	\s{
	Eine einfache, den beobachteten Effekten zu Grunde liegende Modellvorstellung entwickelte Nils Bohr 1913.
	Laut diesem in Abbildung \ref{fig:bohr} dargestellten Atommodell besteht jedes Atom aus einen Atomkern und einer Elektronenhülle.
	Der Atomkern wiederum besteht aus dicht gepackten Protonen (positive Ladungsträger) und Neutronen (elektrisch neutral).
    Die Hülle besteht aus Elektronen (negative Ladungsträger), die den Atomkern auf konzentrischen Bahnen 
    mit unterschiedlichen Radien umkreisen. 
	}
   
        
		%\b{
        %Der Atomkern besteht aus $\mathbf{Protonen}$ und $\mathbf{Neutronen}$.\\
		%\phantom{Hallo Welt!}\\
		%Dieser wird von  $\mathbf{Elektronen}$ umkreist.\\
		%\phantom{Hallo Welt!}\\
		%Elektronen sind Träger der kleinsten positiven Ladungseinheit \textbf{-\textit{e}}.
		%Protonen tragen die Ladungseinheit \textbf{+\textit{e}}.\\
		%\phantom{Hallo Welt!}\\
		%Diese kleinste Ladungseinheit ist die $\mathbf{Elementarladung}$:
		
		%\begin{equation*}
		%\hspace*{-0.5cm}  \mathbf{e = 1,6 \cdot 10^{-19} \, \mathrm{C}} \, \, \, \rm{(Naturkonstante!)}
		%\end{equation*}

		%Sie ist für positive und negative Ladungen betragsgleich.
		%}

		


       
        
	 	
	 
	\begin{figure}[h!]
		\centering
		\b{\includesvg[width=0.8\textwidth]{atommodell_bohr}}
		\s{\includesvg[width=0.8\textwidth]{atommodell_bohr}}
		\s{\caption{Illustration des Bohr'schen Atommodells}}
		\label{fig:bohr}
	\end{figure}


	\speech{bohr}{1}{ Wir widmen uns nun den Grundlagen der Elektrotechnik und beginnen damit, zu verstehen, warum die klassische Mechanik allein oft nicht ausreicht,
	 um alle hier auftretenden physikalischen Phänomene zu beschreiben.
	Bereits in der Antike, etwa 600 vor Christus, machte der griechische Philosoph Thales von Milet eine interessante Beobachtung. Wenn er Bernstein an einem Fell rieb,
	 zog dieser leichte Objekte, wie zum Beispiel Federn, an. Dies war einer der frühen Hinweise darauf, dass es neben der uns bekannten Gravitationskraft
	  eine weitere grundlegende Kraft gibt.
	Spätere Experimente bestätigten die Existenz dieser sogenannten elektrischen Kraft. 
	Eine zentrale Eigenschaft dieser Kraft ist, dass sie sowohl anziehend als auch
	 abstoßend wirken kann, ganz im Gegensatz zur Gravitation, die stets anziehend ist.
	Als Ursache für diese Kraftwirkungen postulieren wir elektrische Ladungen. Diese wurden historisch, wenn auch willkürlich, als positiv und negativ definiert.
	 Ein grundlegendes Gesetz lautet: Gleichnamige Ladungen stoßen sich ab, während ungleichnamige Ladungen sich anziehen.	
	Um die Herkunft und das Verhalten dieser Ladungen zu verstehen, betrachten wir eine Modellvorstellung vom Aufbau der Materie. Ein wichtiges, grundlegendes 
	Atommodell wurde 1913 von Nils Bohr entwickelt. 
	Laut diesem Modell besteht jedes Atom aus einem Atomkern und einer Elektronenhülle. Der Atomkern im Zentrum des Atoms setzt sich aus Protonen zusammen, 
	welche positive Ladungsträger sind, und Neutronen, die elektrisch neutral sind.
	Die Hülle des Atoms wird von Elektronen gebildet. Elektronen sind negative Ladungsträger und umkreisen den Atomkern auf bestimmten Bahnen.
	Das Wechselspiel zwischen den positiven Ladungen im Kern und den negativen Ladungen der Elektronen in der Hülle ist entscheidend für die elektrischen 
	Eigenschaften von Materialien und bildet die Basis für viele Konzepte der Elektrotechnik, die wir in den folgenden Lektionen behandeln werden.
	Damit haben wir die grundlegenden Bausteine und Kräfte kennengelernt, die für das Verständnis elektrischer Phänomene unerlässlich sind.}
\end{frame}

\begin{frame}
\ftx{Elektronen und Ionen}

\s{	Die \textbf{Ladung} $Q$ eines Protons und eines Elektrons ist betragsmäßig gleich und wird als 
		Elementarladung
		$e$ bezeichnet. Ihr experimentell 
		bestimmter Wert beträgt $e = 1,6 \cdot 10^{-19} \, \mathrm{C}$. Das Proton ist positiv und 
		das Elektron negativ geladen. Ein Atom hat im Regelfall
		die identische Anzahl Protonen wie Elektronen und ist daher von außen gesehen elektrisch
		 neutral.

		Das Coulomb ist die Einheit der elektrischen Ladung und wird wie folgt beschrieben:



		\begin{equation*}
		\mathit{[Q]} = \rm{ 1 \, Coulomb = 1 \, C = 1 \, As}
		\end{equation*}
		
		\newpage
		
		}


    \begin{columns}

		\column[c]{0.5\textwidth}
		\b{

		Kleinste Ladungseinheit ist die Elementarladung: \\

		\phantom{.}\\



		 $e = 1,6 \cdot 10^{-19} \, \mathrm{C}$ (Naturkonstante!)\\

		 \phantom{.}\\
		\begin{Merksatz}{}
			$Q = \pm n \cdot e$
		\end{Merksatz}
		\hspace{30pt}mit n = 1, 2, 3,...\\
		\phantom{test}\\
		$\mathit{[Q]} = \rm{ 1 \, Coulomb = 1 \, C = 1 \, As}$
		}

       
        \column[c]{0.5\textwidth}

		\begin{figure}[h!]
			\centering
			\b{\includesvg[width=\textwidth]{ionisierung}}
			\s{\includesvg[width=0.6\textwidth]{ionisierung}}
			\s{\caption{Trennen eines Atoms in ein positiv geladenes Ion und ein freies Elektron durch Energiezufuhr}}
			\label{fig:ionisierung}
		\end{figure}


		\b{Freie Elektronen als kleinste, leichteste und beweglichste Ladungsträger bedeutsam!}
			

			\s{Durch Zufuhr von Energie oder durch Wechselwirkung mit anderen Teilchen (beispielsweise durch
			Reibung eines Plastikstabes an einem Stück Stoff, Reibung von Kleidung an einer
			 Kunststoffrutsche, UV-Strahlung) kann sich die Elektronenanzahl
		   eines Atoms verändern. Bekommt ein Atom ein oder mehrere zusätzliche Elektronen, liegt also ein
			\textbf{Elektronenüberschuss} vor, wird es negativ 
		   geladen. Verliert es Elektronen, also bei einem \textbf{Elektronenmangel}, wird es positiv 
		   geladen. In beiden Fällen wird es als Ion bezeichnet.
		   Da jeweils nur ganze Elektronen dem Ion hinzugefügt oder abgezogen werden können, ist die 
		   entsprechende Ladung $Q$ eines Ions (und damit auch jedes anderen Objektes) immer ein Vielfaches
			der Elementarladung $e$.
	   
		   \begin{Merksatz}{}
			   $Q = \pm \, n \cdot e$
		   \end{Merksatz}
	   
		   Die Masse der sich im Kern befindenden Protonen und Neutronen ist um viele Größenordnungen
			höher als die der Elektronen. Folglich sind die aus dem Atomverband befreiten
			Elektronen als kleinste, leichteste und beweglichste Ladungsträger für die Elektrotechnik von 
			großer Bedeutung.

			
		   }
		   \speech{elektronen}{1}{Wie wir bereits gesehen haben, sind Elektronen und Protonen die Ladungsträger im Atom. , Ihre Ladung ist betragsmäßig 
			identisch und wird als Elementarladung e bezeichnet. Experimentell wurde dieser Wert auf circa eins komma sechs mal zehn hoch minus neunzehn Coulomb bestimmt.
			Dabei ist die Ladung des Protons positiv und die des Elektrons negativ. Ein Atom ist im Normalzustand elektrisch neutral,
			 da es in der Regel die gleiche Anzahl positiver Protonen und negativer Elektronen besitzt.
			Die Einheit der elektrischen Ladung ist das Coulomb, abgekürzt mit einem großen C. Ein Coulomb entspricht einem Ampere mal einer Sekunde. 	
			Die Anzahl der Elektronen in einem Atom kann sich jedoch verändern. Dies geschieht beispielsweise durch Zufuhr von Energie,
			 wie dargestellt auf der rechten Seite der Folie, wo Energie zur Trennung führt, oder auch durch Wechselwirkungen mit anderen Teilchen, wie Reibung oder UV-Strahlung.
			Wenn ein Atom Elektronen aufnimmt, also einen Überschuss an negativen Ladungsträgern hat, wird es negativ geladen. 
			Verliert ein Atom Elektronen, hat es einen Mangel an negativer Ladung, und wird positiv geladen. In beiden Fällen sprechen wir von einem Ion.
			Da bei diesen Prozessen immer ganze Elektronen hinzugefügt oder abgezogen werden, ist die resultierende Ladung eines Ions, und generell die Ladung jedes Objektes,
			 immer ein ganzzahliges Vielfaches der Elementarladung e. Dies wird durch die Formel Q gleich Plusminus N mal e ausgedrückt, wobei n für eine ganze Zahl steht.
			Die Abbildung auf der rechten Seite illustriert diesen Prozess: Energiezufuhr führt zur Abtrennung eines Elektrons, wodurch ein freies Elektron und ein positiv 
			geladenes Ion entstehen.
			Diese aus dem Atomverband gelösten Elektronen sind für die Elektrotechnik von herausragender Bedeutung. Aufgrund ihrer im Vergleich zu Protonen und Neutronen 
			sehr viel geringeren Masse sind sie die kleinsten, leichtesten und vor allem beweglichsten Ladungsträger. 
			Sie sind der Hauptgrund für die elektrische Leitfähigkeit vieler Materialien.}
    \end{columns}
\end{frame}










\begin{frame}
	\ftb{Ladungsdichten}

	\s{Nicht nur die Anzahl, sondern auch die Verteilung der elektrischen Ladungsträger 
	spielt in der Elektrotechnik eine wichtige Rolle. So ist es beispielsweise für 
	die Gestaltung und Funktionsweise von elektrischen Bauelemente (wie Widerständen, Kondensatoren 
	oder Halbleiterbauelementen) entscheidend zu wissen, wie sich die Ladungen entlang von Linien, 
	auf Oberflächen oder innerhalb von Volumina verteilen. Dabei wird in diesem Modul von einer gleichmäßigen
	Ladungsträgerverteilung ausgegangen. 

	Die \textbf{Punktladung} dient als idealisiertes Modell einer Ladungsverteilung auf
	 einen Punkt ohne räumliche Ausdehnung. Ist die reale räumliche Verteilung an der 
	 entsprechenden Stelle nicht weiter relevant, werden Ladungen meist als Punktladungen betrachtet.

	 Die \textbf{Linienladungsdichte} $\lambda$ (lambda), angegeben in C/m, beschreibt die Verteilung der elektrischen
	  Ladung entlang einer Linie, wie sie beispielsweise in einem sehr dünnen Draht vorkommt,
	  der nur in einer Dimension ausgedehnt ist (siehe Abbildung \ref{fig:linienladung}).  


	  \begin{figure}[h!]
		\centering

		\s{\includesvg[width=0.3\textwidth]{linienladung}}
		\s{\caption{Linienladungsdichte: Eine Ladung $Q$ gleichmäßig verteilt auf einem eindimensionalen
		 Pfad mit der Länge d$l$}}
		\label{fig:linienladung}
	\end{figure}





	Da die Ladungsträger sowie ihre atomaren Strukturen im Vergleich zu den in Bauteilen
	 verwendeten Strukturen vernachlässigbar klein sind, kann statt der diskreten eine kontinuierliche
	Verteilung angenommen werden. Folglich lässt sich die Linienladungsdichte als Ableitung der Ladung
	 pro Linienlänge beschreiben. 

	\begin{equation*}
		\lambda = \lim \limits_{l \to 0} \frac{\Delta Q}{\Delta l} = \frac{\mathrm{d}Q}{\mathrm{d}l}
	\end{equation*}

	Die Ladung $Q$ ist analog dazu das Integral der Linienladungsdichte $\lambda$ über die Länge $l$.
	
	\begin{equation*}
		Q = \int\limits_l \lambda \, \mathrm{d}l
	\end{equation*}


	Die \textbf{Flächenladungsdichte} $\sigma$ (sigma) in $\mathrm{C/m^2}$ beschreibt die Verteilung der Ladung $Q$,
	 die pro Flächeneinheit $A$ verteilt ist (siehe Abbildung \ref{fig:flächenladung}). 
	Sie wird beispielsweise bei der Berechnung von Kondensatoren, welche zu einem späteren Zeitpunkt eingeführt werden, benötigt.

	\begin{figure}[h!]
		\centering
	
		\s{\includesvg[width=0.4\textwidth]{flächenladung}}
		\s{\caption{Flächenladungsdichte: Eine Ladung $Q$ gleichmäßig verteilt auf einer zweidimensionalen Fläche mit der Größe d$A$.}}
		\label{fig:flächenladung}
	\end{figure}


	

	Vergleichbar zur Linienladungsdichte kann auch $\sigma$ über eine Grenzwertbetrachtung und damit über eine Ableitung beschrieben werden: 


	\begin{equation*}
		\sigma = \lim \limits_{A \to 0} \frac{\Delta Q}{\Delta A} = \frac{\mathrm{d}Q}{\mathrm{d}A}
	\end{equation*}

	\begin{equation*}
		Q = \iint\limits_A \sigma \, \mathrm{d}A
	\end{equation*}

	Die \textbf{Raumladungsdichte} $\rho$ (rho) in $\mathrm{C/m^3}$ gibt die Anzahl der freien Ladungsträger im betrachteten Volumen an. 
	Hierbei kann es sich beispielsweise um Elektronen in einem Leiter oder geladene Ionen in einem Gasgemisch handeln.
	

	\begin{figure}[h!]
		\centering
	
		\s{\includesvg[width=0.4\textwidth]{raumladung}}
		\s{\caption{Raumladungsdichte $\rho$: Eine Ladung $Q$ gleichmäßig verteilt in einem dreidimensionalen
		Volumen mit der Größe d$V$.}}
		\label{fig:raumladung}
	\end{figure}
	
	

	\begin{equation*}
		\rho = \lim \limits_{V \to \infty} \frac{\Delta Q}{\Delta V} = \frac{\mathrm{d}Q}{\mathrm{d}V}
	\end{equation*}
	\begin{equation*}
		Q = \iiint\limits_V \rho \, \mathrm{d}V
	\end{equation*}	

	
	}

	\b{
	\begin{columns}
		\column[c]{0.3\textwidth}
		
				\textbf{Linienladung}
				\vspace{20pt}
				\begin{figure}[h!]
					\centering
				
				{\includesvg[width=0.7\textwidth]{linienladung}}
					
				\end{figure}	
			
			\vspace{10pt}
				 
				\begin{equation*}
				\lambda = \lim \limits_{l \to 0} \frac{\Delta Q}{\Delta l} = \frac{\mathrm{d}Q}{\mathrm{d}l}
			\end{equation*}
			
			\begin{equation*}
				Q = \int\limits_l \lambda \, \mathrm{d}l
			\end{equation*}
			

		\column[c]{0.02\textwidth}
			\rule{.1mm}{0.7\textheight}


		\column[c]{0.30\textwidth}
		\vspace{2pt}
	
	%\vspace{5pt}
	\visible<2->{
			\textbf{Flächenladung}

			\vspace{12pt}
			
			\begin{figure}[h!]
				\centering
			
			{\includesvg[width=0.7\textwidth]{flächenladung}}
				
			\end{figure}	
			
			\vspace{5pt}
			
			\begin{equation*}		
				\sigma = \lim \limits_{A \to 0} \frac{\Delta Q}{\Delta A} = \frac{\mathrm{d}Q}{\mathrm{d}A}
			\end{equation*}

			\begin{equation*}
				Q = \iint\limits_A \sigma \, \mathrm{d}A
			\end{equation*}
	}
				
		\column[c]{0.02\textwidth}
		\rule{.1mm}{0.7\textheight}

		 \column[c]{0.3\textwidth}
		 \visible<3->{
	
		 \textbf{Raumladung}
		
		 \begin{figure}[h!]
			\centering
		
			{\includesvg[width=0.7\textwidth]{raumladung}}
			
			
		\end{figure}		
		 \begin{equation*}
			\rho = \lim \limits_{V \to 0} \frac{\Delta Q}{\Delta V} = \frac{\mathrm{d}Q}{\mathrm{d}V}
		\end{equation*}
		\begin{equation*}
			Q = \iiint\limits_V \rho \, \mathrm{d}V
		\end{equation*}	
		 }	
		\end{columns}

		\speech{ladungsdichten}{1}{Nicht immer ist nur die absolute Anzahl an Ladungsträgern von Bedeutung. Auch die Dichte oder die räumliche Verteilung 
		elektrischer Ladungsträger spielt oft eine entscheidende Rolle, beispielsweise für die Funktionsweise vieler elektrischer Bauelemente. ,
		Eine idealisierte Modellvorstellung für eine Ladungsverteilung ohne räumliche Ausdehnung ist die Punktladung. Wenn die genaue räumliche Ausdehnung der Ladung an
		einer bestimmten Stelle für die Betrachtung nicht relevant ist, können wir sie als Punktladung behandeln, was viele Rechnungen vereinfacht.
		Reale Ladungsverteilungen erstrecken sich jedoch oft entlang von Linien, auf Flächen oder in Volumina. Da die Elementarladungen und ihre atomaren Strukturen im Vergleich zu den von uns betrachteten Bauteilen 
		meist sehr klein sind, können wir oft eine kontinuierliche Verteilung annehmen.
		Beginnen wir mit der Linienladungsdichte, bezeichnet mit dem griechischen Buchstaben Lambda. 
		Sie beschreibt die Verteilung der elektrischen Ladung entlang einer Linie, wie sie
		 beispielsweise bei einem sehr dünnen Draht vorliegen kann. Vorstellen kann man sich dies vergleichbar zu Perlen, die auf einer Schnur aufgefädelt sind.
		Die Einheit der Linienladungsdichte ist Coulomb pro Meter. Mathematisch definieren wir Lambda als den Grenzwert des Verhältnisses von Ladungsänderung  Delta Q zu Längenänderung 
		Delta L, wenn Delta L gegen null geht. Das führt zur Ableitung Lambda gleich Delta Q nach Delta L.
		Umgekehrt erhalten wir die Gesamtladung Q entlang einer Linie, indem wir die Linienladungsdichte Lambda über die Länge L integrieren. Q gleich Integral Lambda D L.}
		
		\speech{ladungsdichten}{2}{Für Ladungen, die sich auf einer Oberfläche verteilen, nutzen wir die Flächenladungsdichte, symbolisiert durch das Sigma. Ihre Einheit ist Coulomb pro
		 Quadratmeter. Diese Größe ist beispielsweise bei der Berechnung von Kondensatoren von Bedeutung. Ähnlich wie bei der Linienladungsdichte definieren wir Sigma als den Grenzwert von 
		 Delta Kuh durch Delta A, wenn die Fläche Delta A gegen null schrumpft.
		Die Gesamtladung Kuh auf einer Fläche A ergibt sich entsprechend durch das Flächenintegral der Flächenladungsdichte Sigma über die Fläche A.}

		\speech{ladungsdichten}{3}{Schließlich gibt es die Raumladungsdichte, dargestellt durch den Buchstaben Rho. Sie beschreibt die Verteilung der Ladung innerhalb eines Volumens.
		Ihre Einheit ist Coulomb pro Kubikmeter. Hierbei kann es sich um freie Elektronen in einem Leiter oder auch geladene Ionen in einem Gas handeln. 
		Die Raumladungsdichte Rho ist definiert als der Grenzwert von Delta Kuh durch Delta V, wenn das Volumen Delta V gegen null geht.
		Die Gesamtladung Kuh in einem Volumen V erhalten wir analog zu den vorherigen Beispiel durch das Volumenintegral der Raumladungsdichte Rho über das Volumen V. ,}

	}
	
    
\end{frame}


\newpage



\begin{frame}
	\ftb{Elektrische Leiter - Metalle}




		\s{Diese Aufteilung der Ladungsträger ist beispielsweise in Metallen gut zu sehen. 
		Viele Metalle weisen einen besonderen Atomverband auf. Die Atome ordnen 
		sich bei ihnen in einer regelmäßigen \textbf{Gitterstruktur} an. Dadurch sind die 
		äußersten Elektronen eines jeden Atoms innerhalb des Metallkörpers 
		nahezu frei beweglich. Diese freien Elektronen verleihen Metallen ihre 
		elektrisch leitende Eigenschaft. In Abbildung \ref{fig:metalle} wird diese Struktur dargestellt.
		Häufig werden Metalle in der Elektrotechnik deshalb kurz als (elektrische) 
		Leiter bezeichnet. Im Allgemeinen sind Leiter elektrisch neutral, da in jedem Atom die Summe aller
		Elektronen der Summe aller Protonen entspricht.
		


			\begin{figure}[h!]
				\centering
			
				\s{\includesvg[width=0.4\textwidth]{metalle}}
				\s{\caption{Ausschnitt aus einem Metallblech. Eingezeichnet sind die positiv geladenen Atomkerne sowie deren frei bewegliche äußeren Elektronen.}}
				\label{fig:metalle}
			\end{figure}
		
	
 
		Durch das Hinzuführen von Elektronen entsteht ein Elektronenüberschuss, wodurch der Leiter negativ geladen wird.
		Werden Elektronen entzogen entsteht ein Elektronenmangel und der Leiter wird positiv geladen (Abbildung \ref{fig:leiter}).


		\begin{figure}[h!]
			\centering
	
			\s{\includesvg[width=0.7\textwidth]{leiter_ladung}}
			\s{\caption{Leiter, bei denen an der Oberfläche ein Elektronenüberschuss (links) beziehungsweise ein Elektronenmangel (rechts) herrscht.}}
			\label{fig:leiter}
		\end{figure}
	


		
		}
		\begin{columns}
			\column[c]{0.6\textwidth}
		\b{
		\vspace{-15pt}	
		
		\hspace{-2pt}Atomanordnung in \textbf{Gitterstruktur},\\
		äußere Elektronen nahezu frei beweglich\\
		$\rightarrow$ verleiht elektrisch leitende Eigenschaften\\
		\phantom{test}\\

		Summe Elektronen $\widehat{=}$ Summe Protonen\\
		$\rightarrow$ elektrisch neutral\\
		\phantom{test}\\
		\phantom{test}\\

		\onslide<2->{Hinzuführen von Elektronen\\
		$\rightarrow$ negativ geladen
		}\\
		\phantom{test}\\
		\phantom{test}\\
		\onslide<3->{Entziehen von Elektronen\\
		$\rightarrow$ positiv geladen
		
		}


		

            
        \column[c]{0.3\textwidth}

		\begin{figure}[h!]
			\centering
		
			{\includesvg[width=\textwidth]{Metalle2}}
			\label{fig:metallefolien}
		\end{figure}




			\vspace*{0.8cm}
			\onslide<2->{
				
			\begin{figure}[h!]
				\centering
	
				{\includesvg[width=0.7\textwidth]{e_ueberschuss}}
				\label{fig:le_ueber}
			\end{figure}
		}
			\onslide<3->{
				\begin{figure}[h!]
					\centering
				
					{\includesvg[width=0.7\textwidth]{e_mangel}}
					\label{fig:le_mangel}
				\end{figure}
				}
		}
    \end{columns}

	\speech{metalle}{1}{Diese Aufteilung der Ladungsträger ist beispielsweise in Metallen und anderen elektrischen Leitern gut zu sehen. Die Fähigkeit, elektrischen Strom zu leiten,
	 hängt eng mit der Verteilung der Ladungsträger in diesen Materialien zusammen. Viele Metalle weisen eine besondere Atomstruktur auf: Die Atome sind in einer regelmäßigen 
	 Gitterstruktur angeordnet.
Das Entscheidende bei Metallen ist, dass die äußersten Elektronen eines jeden Atoms innerhalb dieses Metallkörpers nur schwach gebunden und dadurch 
nahezu frei beweglich sind. Diese Struktur ist hier stilisiert in der obersten Abbildung auf der Folie dargestellt. Diese frei beweglichen Elektronen
 sind der Grund dafür, dass Metalle elektrisch leitende Eigenschaften besitzen. 
Im elektrisch neutralen Zustand eines Leiters ist die Gesamtzahl der Elektronen, einschließlich der freien, gleich der Gesamtzahl der Protonen in den Atomkernen.}
\speech{metalle}{3}{Ein Leiter kann jedoch elektrisch geladen werden. Wie die unteren beiden Darstellungen auf der Folie zeigen, geschieht dies durch das Hinzufügen oder das
 Entziehen von Elektronen. Wenn wir einem Leiter Elektronen hinzufügen, entsteht ein Elektronenüberschuss, und der Leiter wird negativ geladen. Entziehen wir Elektronen,
  entsteht ein Elektronenmangel, und der Leiter wird positiv geladen.
  Wichtig ist hierbei, dass in Metallen die positiven Atomkerne in ihrer Gitterstruktur fixiert sind und sich nicht frei bewegen können. Der Ladungstransport, der für die elektrische 
  Leitung verantwortlich ist, erfolgt primär durch die Bewegung dieser freien Elektronen.}
\end{frame}










\begin{frame}
	\ftb{Das Coulombsche Gesetz}

		\s{
		Geladene Teilchen beeinflussen sich gegenseitig, wobei sich Ladungen gleichen Vorzeichens
		abstoßen und unterschiedlichen Vorzeichens anziehen.
	    Bereits um 1785 konnte Charles Augustin de Coulomb experimentell nachweisen, dass sich der 
		Betrag der Kraft proportional zu jeder der kugelsymmetrischen Ladungen $Q_1$ und $Q_2$, aber antiproportional zum 
		Quadrat des Abstandes $r$ der beiden Ladungen voneinander verhält.

		
		Die proportionale Wechselwirkung zwischen den Ladungen $Q_1$ sowie $Q_2$ und dem Abstandsquadrat
		$r^2$ führt mit der Proportionalitätskonstante $\frac{1}{4 \pi \varepsilon_0}$ zur Kraft 
		zwischen den Ladungen. Dieser proportionale Zusammenhang ist in Abbildung \ref{fig:coulomg} dargestellt.



		\begin{figure}[h!]
			\centering
			\s{\includesvg[width=0.7\textwidth]{kraft}}
			\s{\caption{Anordnung von zwei Punktladungen 
			$Q_1$ und $Q_2$ sowie die auf sie wirkenden Kräfte.}}
			\label{fig:coulomg}
		\end{figure}

		
			Durch das Einführen eines Proportionalitätstermes kann aus dem zuvor bestimmten proportionalen
			Zusammenhang die resultierende Kraft $F$ direkt errechnet werden. Die resultierende Gleichung \ref{eq:coulomb}
			 wird Coulombsches Gesetz genannt.
			 
			 \begin{equation}
				F \sim \frac{Q_1 Q_2}{r^2} \rightarrow F = \frac{1}{4 \pi \varepsilon_0} \frac{Q_1 Q_2}{r^2} 
				\label{eq:coulomb}
			\end{equation}
			 
			 
			Der Faktor $\varepsilon_0$ wird als \textbf{elektrische Feldkonstante} (auch Dielektrizitätskonstante des Vakuums) bezeichnet,
			während die 4 $\pi$ ihren Ursprung in geometrischen Betrachutngen der Anordnung haben.

			Der Wert der elektrischen Feldkonstante beträgt etwa:

			\begin{equation*}
				\varepsilon_0 = 8,854 \cdot 10^{-12} \rm{As/Vm}
			\end{equation*}  

			Gibt es mehr als zwei Ladungen, kann die resultierende Kraft auf jede der einzelnen Ladungen durch die
			Addition aller Einzelkräfte errechnet werden, die sich aus jeder Zweierkombination von Ladungen ergibt, an 
			der die Zielladung beteiligt ist. Dieser Effekt wird als Superpositionsprinzip bezeichnet, welches auch im
			 Modul 5 \textit{Erweiterte Gleichstromnetzwerke} zur Berechnung von sich ergebenden Spannungen verwendet wird.

			In der Praxis wird dieser Effekt unter anderem beim elektrostatischen Aufspannen von Papier auf Plottern, 
			bei Laserdruckern zur Übertragung des Tonerpulvers auf das Papier, bei Touchscreens zur
			Bestimmung des Berührungspunktes oder bei Plasmabildschirmen zur kontrollierten 
			Gasentladung von Ionen zur Lichterzeugung genutzt.

		
		}


		\b{
		\hspace{-9pt}Ladungen üben Kräfte aufeinander aus.\\
		\phantom{text}\\
		\begin{columns}
			\column[c]{0.5\textwidth}
			Gleichnamige Ladung $\rightarrow$ abstoßend 
			\column[c]{0.5\textwidth}
			Ungleichnamige Ladung $\rightarrow$ anziehend
		\end{columns}
 
		\begin{figure}[h!]
			\centering
			\only<1>{{\includesvg[width=0.7\textwidth]{kraft2}}}
			\only<2>{{\includesvg[width=0.7\textwidth]{kraft}}}
			\only<3>{{\includesvg[width=0.7\textwidth]{kraft}}}
			\only<4>{{\includesvg[width=0.7\textwidth]{kraft}}}
			\label{fig:coulomb}
		\end{figure}

		\pause
		\pause
		
	



		\begin{equation*}
			F \sim \frac{Q_1 Q_2}{r^2} \rightarrow F = \frac{1}{4 \pi \varepsilon_0} \frac{Q_1 Q_2}{r^2} 
		\end{equation*}

		}

		\pause
		\b{

		$\varepsilon_0$: \textbf{elektrische Feldkonstante (Dielektrizitätskonstante)} 
		\begin{equation*}
			\varepsilon_0 = 8,854 \cdot 10^{-12} \rm{As/Vm}
		\end{equation*}  

		}

	\speech{coulomb}{3}{Wie wir bereits wissen, üben geladene Teilchen Kräfte aufeinander aus. Ladungen gleichen Vorzeichens stoßen sich ab, während Ladungen unterschiedlichen 
	Vorzeichens sich anziehen.
	Quantifiziert wurde dieser Zusammenhang maßgeblich von Charles Augustin de Coulomb, der bereits um 1785 experimentell nachwies, wie sich die Stärke dieser Kraft verhält. Er 
	fand heraus, dass der Betrag der Kraft proportional zum Produkt der beiden Ladungen Q 1 und Q 2 ist und gleichzeitig antiproportional zum Quadrat des Abstandes R zwischen ihnen.
Um aus diesem proportionalen Zusammenhang eine direkte Berechnung der Kraft F zu ermöglichen, wird eine Proportionalitätskonstante eingeführt. Diese führt uns zur mathematischen 
Formulierung des Coulombschen Gesetzes: F ist gleich Eins geteilt durch Vier Pi Epsilon Null mal Q Eins Q Zwei durch R Quadrat.}
\speech{coulomb}{4}{In dieser Formel ist der Faktor Epsilon Null die elektrische Feldkonstante, oft auch Dielektrizitätskonstante des Vakuums genannt. 
Sie ist eine Naturkonstante und hat einen Wert von etwa Acht Komma Acht Fünf mal Zehn hoch minus Zwölf Ampere sekunden pro Voltmeter
Die 4 pi in der Formel rühren von geometrischen Betrachtungen der räumlichen Ausbreitung her.
Wichtig zu verstehen ist auch das Superpositionsprinzip von Kräften. Wenn nicht nur zwei, sondern mehrere Ladungen vorhanden sind, dann ist die resultierende Kraft auf eine einzelne Ladung die
 Vektorsumme aller Kräfte, die von jeder anderen Ladung einzeln auf sie ausgeübt werden. Dieses Prinzip werden wir später auch noch an anderer Stelle, zum Beispiel bei der Berechnung
  von Spannungen, wiederfinden.
Die Effekte, die das Coulombsche Gesetz beschreibt, finden vielfältige Anwendung in der Praxis. Beispielsweise in vielen elektrostatische. Prozesse, wie sie beim Aufspannen von Papier 
in Plottern, bei der Tonerübertragung in Druckern, bei der Funktionsweise von Touchscreens zur Positionsbestimmung oder auch bei der Lichtemission in Plasmabildschirmen genutzt werden.}		
\end{frame}





\newpage


