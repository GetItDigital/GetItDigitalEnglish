%Lernziele
\begin{frame}
    \b{ \frametitle{Bauelemente}
        \begin{Lernziele}{Halbleiterbauelemente}
            \title{Lernziele: Halbleiter}
            Die Studierenden können
            \begin{itemize}
                \item Bauteile für verschiedene Anwendungen auswählen.
                \item den Aufbau verschiedener Halbleiterbauelemente beschreiben.
                \item Arbeitspunkte bestimmen und fehlende Bauteilwerte berechnen. 
                \item Anwendungen einzelner Bauelemente nennen.
            \end{itemize}
        \end{Lernziele}  
    }
\end{frame}


\begin{frame}
    \b{\frametitle{Diode}
    \begin{figure}[H]
        \begin{minipage}[c]{0.3\textwidth}
            \centering
            \begin{circuitikz}[]
                \draw (0,0) to[D, v>, name=UD] (0,-2);
                \varrmore{UD} {$U_\mathrm{D}$};
            \end{circuitikz}
        \end{minipage}
        \begin{minipage}[c]{0.3\textwidth}
            \centering
            \includesvg[width= 0.15\textwidth]{Bilder/kap2/SiliziumDiode/SiliziumDiode2.svg}  
        \end{minipage}
        \begin{minipage}[c]{0.25\textwidth}
            \centering
            \includesvg[width= .8\textwidth]{Bilder/kap2/SiliziumDiode/SiliziumDiode3.svg}
        \end{minipage}
        %\label{fig:SiliziumDiode}
        %\caption{\textbf{Silizium Diode.} V.l.n.r. Schaltzeichen, Bauform und Querschnitt einer Silizium Diode.} 
    \end{figure}
    }
\end{frame}

\begin{frame}
\b{\frametitle{Diodenkennlinie}
    \begin{columns}
        \column[c]{0.3\textwidth}
            \resizebox{1\totalheight}{!}{\includesvg[width=1.6\textwidth]{Bilder/kap2/FolienDiodenkennlinieUrsprungpA.svg}}
                  
            \column[c]{0.6\textwidth}
            $I_\mathrm{D} = I_\mathrm{S} \cdot (e^{\frac{U}{U_\mathrm{T}}} - 1) = I_\mathrm{S}  e^{\frac{U_\mathrm{D}}{U_\mathrm{T}}} - I_\mathrm{S}$
                \begin{itemize}
                    \item $I_\mathrm{D}$: Diodennenstrom
                    \item $I_\mathrm{S}$: Sperrstrom
                    \item $U_\mathrm{D}$: Flussspannung
                    \item $U_\mathrm{T}$: Temperaturspannung
                    \item $U_\mathrm{T} = \frac{k \cdot T}{e}$, bei Raumtemperatur $U_\mathrm{T} \approx 25\,\mathrm{mV}$
                \end{itemize}
            \end{columns}}
\end{frame}

\begin{frame}
\b{\frametitle{Temperaturverhalten von Dioden}
    \begin{columns}
        \column[c]{0.3\textwidth}
            \resizebox{1\totalheight}{!}{\includesvg[width=1.6\textwidth]{Bilder/kap2/FolienTemperaturverhaltenDioden.svg}}
                  
            \column[c]{0.6\textwidth}
            $I_\mathrm{D} = I_\mathrm{S} \cdot (e^{\frac{U}{U_\mathrm{T}}} - 1) = I_\mathrm{S}  e^{\frac{U_\mathrm{D}}{U_\mathrm{T}}} - I_\mathrm{S}$ mit $U_\mathrm{T} = \frac{k \cdot T}{e}$
                \begin{itemize}
                    \item Boltzmannkonstante $k = 1.380649 \cdot 10^{-23}\,\mathrm{J/K}$
                    \item Jede 5 °C Temperaturerhöhung verdoppelt den Diodenstrom
                    \item Die Flussspannung $U_\mathrm{D}$ sinkt um ca. 2 mV pro °C bei konstantem Diodenstrom
                    \item $T_2 > T_1 > T_0 $
                \end{itemize}
            \end{columns}}
        \end{frame}


\begin{frame}
\b{\frametitle{Belastungsgrenzen einer Diode}
    \begin{columns}
        \column[c]{0.3\textwidth}
            \resizebox{1\totalheight}{!}{\includesvg[width=1.6\textwidth]{Bilder/kap2/DiodeBelastungsgrenzen.svg}}
                  
            \column[c]{0.6\textwidth}
        
                \begin{itemize}
                    \item Eine Diode verträgt in Sperrrichtung mehr Spannung als in Flussrichtung.
                    \item Bei hohen Spannungen in Sperrrichtung kann es zu einem Durchbruch kommen. 
                    \item Die Diode in Sperrrichtung wird in dem violett markierten Bereich irreversibel zerstört.
                \end{itemize}
            \end{columns}}
        \end{frame}

\begin{frame}
    \b{\frametitle{Diode -- Anwendung/ Grundschaltungen}
        \begin{figure}[H]
            \centering
            \includesvg[width= 0.7\textwidth]{Bilder/kap2/Diodenkennlinie.svg}
            %\caption{\textbf{Diodenkennlinie.} Relevante Bereiche v.l.n.r. Durchbruchbereich, Sperrbereich und Durchlassbereich}  
            %\label{fig:Diodenkennlinie}
        \end{figure}
    }
\end{frame}

\begin{frame}
    \b{\frametitle{Diode -- Elektrisches Verhalten}
        \begin{figure}[H]
            \begin{minipage}[c]{0.5\textwidth}
            \centering
            \includesvg[width= \textwidth]{Bilder/kap2/DiodenkennlinieUndErsatzschaltbild/DiodenkennlinieUndErsatzschaltbild1.svg}
            \end{minipage}
            \begin{minipage}[c]{0.3\textwidth}
            \centering
                \scalebox{0.9}{\begin{circuitikz}    
    \draw (0,0) 
        to[short, o-] (1,0)
        to[D = $D$] (2.5,0) 
        to[V, v>, name=U1] (4,0)
        to[R=$r_\mathrm{D}$] (5.5,0)
        to[short ,i, name=out, -o] (6.5 ,0);
    
    \iarrmore {out} {$i_\mathrm{D}$};
    
    \draw (0,-1) to[open,v>, name=Uv] (6.5,-1);
    \varrmore{Uv}{$U_\mathrm{D}$};
    \varrmore{U1}{$U_\mathrm{S}$};
    
\end{circuitikz}}
            \end{minipage}
            %\label{fig:DiodenkennlinieUndErsatzschaltbild}
            %\caption{\textbf{Diodenkennlinie und Ersatzschaltbild.} Links: Diodenkennlinie mit Näherung für den differenziellen
            %Widerstand. Rechts: Dioden-Ersatzschaltbild mit der idealen Diode, Spannungsquelle und differenzieller Widerstand.} 
        \end{figure}

    }
\end{frame}

\begin{frame}
    \b{\frametitle{Diode -- Elektrisches Verhalten}
        \begin{figure}[H]
            \begin{minipage}[c]{0.5\textwidth}
            \centering
            \includesvg[width= \textwidth]{Bilder/kap2/GrafischeBestimmungArbeitspunkt/GrafischeBestimmungArbeitspunkt1.svg}
            \end{minipage}
            \begin{minipage}[c]{0.3\textwidth}
            \centering
                \begin{circuitikz}
    \draw (0,0) to[R=$R$] (4,0)
        to[D, v>, i>, name=D] (4,-3)
        to[short] (0,-3)
        to[V, v<, name=U0] (0,0);
    
    \varrmore {D}{$U_\mathrm{D}$};
    \varrmore {U0}{$U_\mathrm{0}$};
    \iarrmore {D}{$I_\mathrm{D}$};
\end{circuitikz}
            \end{minipage}
            %\label{fig:GrafischeBestimmungArbeitspunkt}
            %\caption{\textbf{ Grafische Bestimmung Arbeitspunkt.} Links: Diodenkennlinie (schwarz), Widerstandsgerade (blau) und
            %Asymptote der maximalen Verlustleitung (rot) der Diode. Rechts: Schaltung der Diode mit Vorwiderstand.} 
        \end{figure}
    
    }
\end{frame}

\begin{frame}
    \b{\frametitle{Diode -- Elektrisches Verhalten}
        \textbf{Merke:}
        \begin{itemize}
            \item Die Kennlinie der Diode ist stark nichtlinear.
            \item Das Verhalten der Kennlinie kann in drei Bereichen, den Durchbruch-, Sperr- und Durchlassbereich, beschrieben werden.
            \item Mittels idealer Diode, Spannungsquelle und Widerstand kann die reale Diode in verschiedenen Bereichen angenähert werden.
        \end{itemize}
        }
\end{frame}

\begin{frame}
    \b{ \frametitle{Diode -- Aufbau}
        \begin{figure}[H]
            \begin{minipage}[L]{0.3\textwidth}
            \raggedright
            \includesvg[scale=0.6]{Bilder/kap2/SchichtaufbauVonDioden/SchichtaufbauVonDioden1.svg}
            \end{minipage}
            \begin{minipage}[c]{0.3\textwidth}
            \raggedleft
            \includesvg[scale=0.6]{Bilder/kap2/SchichtaufbauVonDioden/SchichtaufbauVonDioden2.svg}  
            \end{minipage}
        
            %\label{fig:SchichtaufbauVonDioden}
            %\caption{\textbf{Schichtaufbau von Dioden.} Aufbau und Vergleich einer klassischen Si-Diode (links) 
            %und Schottky-Diode(rechts). Siliziumdioxid ($\mathrm{SiO_2}$) ist ein natürliches Oxid von Silizium und dient zur Isolierung. Aluminium (Al) ist ein typisches Metall für elektrische Kontakte.} 
        \end{figure}
    }   
\end{frame}

\begin{frame}
\b{\frametitle{Kennlinien einer Zener-Diode}
    \begin{columns}
        \column[c]{0.3\textwidth}
            \resizebox{1\totalheight}{!}{\includesvg[width=1.8\textwidth]{Bilder/kap2/FolienZDiodeKennlinien.svg}}
            \column[c]{0.6\textwidth}
        
                \begin{itemize}
                    \item Zener-Effekt bei einer Durchbruchspannung von $<5\,V$ (Elektronen tunneln in Sperrrichtung)
                    \item Avalanche-Effekt bei einer Durchbruchspannung von $>5\,V$ (Lawinendurchbrucheffekt)
                    \item $U_\mathrm{BR}$ im Bereich von $1,8\,V$ bis $300\,V$ möglich.
                \end{itemize}
            \end{columns}
            }
\end{frame}

\begin{frame}
    \b{\frametitle{Diode -- Aufbau}
        \textbf{Merke:}
        \begin{itemize}
            \item Schottky-Dioden weisen keinen pn-Übergang auf, sondern einen Schottky-Kontakt.
            \item Bei geeigneter Materialkombination bilden sich zwischen Metall und Halbleiter eine RLZ aus. 
        \end{itemize}
        }
\end{frame}

\begin{frame}
    \b{
        \frametitle{Diode -- spezielle Dioden}
        \begin{table}[H]
            \centering 
            \begin{tabular}{ |p{2.1cm}|p{1.4cm}|p{3.2cm}|p{2.8cm}|p{3cm}| }
                \hline
                \textbf{Bezeichnung} & \textbf{Symbol} & \textbf{Kennlinie} & \textbf{Eigenschaften} & \textbf{Anwendung} \\
                \hline
                Gleich\-richter\-diode  
                &
                \begin{minipage}[t][2.6cm][c]{1.4cm}
                    \centering \begin{circuitikz} \draw (0,1) to[D] (0,-1); \end{circuitikz}
                \end{minipage}
                & 
                \begin{minipage}[t][2.6cm][c]{3.3cm}
                    \centering \includesvg[width= 0.9\textwidth]{kap2/spezielleDiodenÜbersichtVerschiedenerDioden/KennlinieGleichrichterdiode.svg}
                \end{minipage}
                & 
                hoher Durchlassstrom &
                 große Sperrspannung,\newline Gleichrichtung\\
                \hline
                Schalt\-diode 
                &
                \begin{minipage}[t][2.6cm][c]{1.4cm}
                    \centering \begin{circuitikz} \draw (0,1) to[D] (0,-1); \end{circuitikz}
                \end{minipage}
                & 
                \begin{minipage}[t][2.6cm][c]{3.3cm}
                    \centering \includesvg[width= 0.9\textwidth]{kap2/spezielleDiodenÜbersichtVerschiedenerDioden/KennlinieSchaltdiode.svg}
                \end{minipage}
                & 
                kleiner Durchbruchwiderstand &
                hoher Sperrwiderstand,\newline kleine Umschaltzeiten\\
                \hline
            \end{tabular}
        \end{table}
    }

    %Gesprochener Text:
    %%
\end{frame}

\begin{frame}
    \b{
        \frametitle{Diode -- spezielle Dioden}
        \begin{table}[H]
            \centering 
            \begin{tabular}{ |p{2.1cm}|p{1.4cm}|p{3.2cm}|p{2.8cm}|p{3cm}| }
                \hline
                \textbf{Bezeichnung} & \textbf{Symbol} & \textbf{Kennlinie} & \textbf{Eigenschaften} & \textbf{Anwendung} \\
                \hline
                Schottky\-diode
                &
                \begin{minipage}[t][2.6cm][c]{1.4cm}
                    \centering \begin{circuitikz} \draw (0,1) to[sD] (0,-1); \end{circuitikz}
                \end{minipage}
                & 
                \begin{minipage}[t][2.6cm][c]{3.3cm}
                    \centering \includesvg[width= 0.9\textwidth]{kap2/spezielleDiodenÜbersichtVerschiedenerDioden/KennlinieSchottkydiode.svg}
                \end{minipage}
                & 
                 kleine Durchlassspannung &
                 kleine Sperrspannung,\newline HF-Gleichrichter,\newline Freilaufdiode,\newline Schaltnetzteile\\
                \hline
                Z-Diode
                &
                \begin{minipage}[t][2.6cm][c]{1.4cm}
                    \centering \begin{circuitikz} \draw (0,1) to[zD] (0,-1); \end{circuitikz}
                \end{minipage}
                & 
                \begin{minipage}[t][2.6cm][c]{3.3cm}
                    \centering \includesvg[width= 0.9\textwidth]{kap2/spezielleDiodenÜbersichtVerschiedenerDioden/KennlinieZ-Diode.svg}
                \end{minipage}
                & 
                definierte Durchbruchspannung &
                Stabilisierung von Spannungen,\newline Begrenzung\\
                \hline
            \end{tabular}
        \end{table}
    }

    %Gesprochener Text:
    %%
\end{frame}

\begin{frame}
    \b{
        \frametitle{Diode -- spezielle Dioden}
        \begin{table}[H]
            \centering 
            \begin{tabular}{ |p{2.1cm}|p{1.4cm}|p{3.2cm}|p{2.8cm}|p{3cm}| }
                \hline
                \textbf{Bezeichnung} & \textbf{Symbol} & \textbf{Kennlinie} & \textbf{Eigenschaften} & \textbf{Anwendung} \\
                \hline
                Tunnel\-diode
                &
                \begin{minipage}[t][2.6cm][c]{1.4cm}
                    \centering \begin{circuitikz} \draw (0,1) to[tD] (0,-1); \end{circuitikz}
                \end{minipage}
                & 
                \begin{minipage}[t][2.6cm][c]{3.3cm}
                    \centering \includesvg[width= 0.9\textwidth]{kap2/spezielleDiodenÜbersichtVerschiedenerDioden/KennlinieTunneldiode.svg}
                \end{minipage}
                & 
                negativer differentieller Widerstand &
                Entdämpfung von Schwingkreisen,\newline HF-Oszillator\\
                \hline
                Diac
                &
                \begin{minipage}[t][2.6cm][c]{1.4cm}
                    \centering \begin{circuitikz} \draw (0,1) to[biD] (0,-1); \end{circuitikz}
                \end{minipage}
                & 
                \begin{minipage}[t][2.6cm][c]{3.3cm}
                    \centering \includesvg[width= 0.9\textwidth]{kap2/spezielleDiodenÜbersichtVerschiedenerDioden/KennlinieDiac-Diode.svg}
                \end{minipage}
                & 
                gesteuerter Durchbruch & 
                Entdämpfung,\newline Triggerdiode\\
                \hline
        \end{tabular}
    \end{table}
    }

    %Gesprochener Text:
    %%
\end{frame}

\begin{frame}
    \b{
        \frametitle{Diode -- spezielle Dioden}
        \begin{table}[H]
            \centering 
            \begin{tabular}{ |p{2.1cm}|p{1.4cm}|p{3.2cm}|p{2.8cm}|p{3cm}| }
                \hline
                \textbf{Bezeichnung} & \textbf{Symbol} & \textbf{Kennlinie} & \textbf{Eigenschaften} & \textbf{Anwendung} \\
                \hline
                Photo\-diode
                &
                \begin{minipage}[t][2.6cm][c]{1.4cm}
                    \centering \documentclass{standalone}
\usepackage{circuitikz}

\begin{document}

\begin{circuitikz}[font=\LARGE, european]

\draw (0,0) to[pDo] (2,0);

\end{circuitikz}

\end{document}

                \end{minipage}
                & 
                \begin{minipage}[t][2.6cm][c]{3.3cm}
                    \centering \includesvg[width= 0.9\textwidth]{kap2/spezielleDiodenÜbersichtVerschiedenerDioden/KennliniePhotodiode.svg}
                \end{minipage}
                & 
                Strom ändert sich proportional zur Lichtleistung &
                Photoempfänger,\newline Messtechnik,\newline Solarzellen\\
                \hline
                LED,\newline Laserdiode
                &
                \begin{minipage}[t][2.6cm][c]{1.4cm}
                    \centering \begin{circuitikz} \draw (0,1) to[leD] (0,-1); \end{circuitikz}
                \end{minipage}
                & 
                \begin{minipage}[t][2.6cm][c]{3.3cm}
                    \centering \includesvg[width= 0.9\textwidth]{kap2/spezielleDiodenÜbersichtVerschiedenerDioden/KennlinieLED.svg}
                \end{minipage}
                & 
                Durchlassstrom erzeugt optische Strahlung &
                Beleuchtung,\newline Strahlungsquelle\\
                \hline
            \end{tabular}
            \caption{\textbf{Übersicht typischer Dioden.}}
        \end{table}
    }

    %Gesprochener Text:
    %%
\end{frame}

\begin{frame}
    \b{ \frametitle{Diode bei Wechselspannung}
\begin{itemize}
    \item In Sperrrichtung fällt die gesamte Spannung über die Diode ab. 
    \item Positive Halbwelle der Wechselspannung wird durchgelassen. Die Flussspannung der Diode wird abgezogen.
    \item Negative Halbwelle der Wechselspannung wird blockiert.
    \item Die Diode wirkt als Gleichrichter.
\end{itemize}
        \begin{figure}[H]
            \begin{minipage}[c]{0.3\textwidth}
            \centering
                \begin{tikzpicture}
    \draw(0,0)
        to[sV, v<, name=UE] (0,2)
        to[D, l=$D_1$] (3,2)
        to[R, v, name=R, l=$R$, -] (3,0)
        to[short,-] (0,0);
    \varrmore{UE}{$u_\mathrm{E}$};
    \varrmore{R}{$u_\mathrm{R}$};
\end{tikzpicture}
            \end{minipage}
            \begin{minipage}[c]{0.6\textwidth}
            \centering
            \includesvg[width=\textwidth]{Bilder/kap2/Einweggleichrichter/EinweggleichrichterTeil2.svg}  
             \end{minipage}
            %\label{fig:Einweggleichrichter}
            %\caption{\textbf{Einweggleichrichter.} Links: Schaltkreis des Einweggleichrichters. 
            %Rechts: Spannungsverlauf der Eingangsspannung und Lastspannung des Einweggleichrichters.} 
        \end{figure}
    }   
\end{frame}

\begin{frame}
    \b{ \frametitle{Einweggleichrichter}
    \begin{itemize}
        \item Eine Gleichspannung kann nicht mit einer Diode allein erzeugt werden. 
        \item Signal muss durch einen Kondensator geglättet werden. 
    \end{itemize}
        \begin{figure}[H]
            \begin{tikzpicture}

\draw (0,0) -- (3,0) to [D, -*] (3,-2) ; 
\draw (3,-4) to[R, l= $R_\mathrm{Last}$,*-] (3,-2);
\draw (3,-4) -- (4,-4) to [C] (4,-2) -- (3,-2);

\draw (0,0) to [V,name=U0] (0,-4) -- (3,-4);

\varrmore {U0}{$U_0$};

\end{tikzpicture}

            %\caption{\textbf{Schaltkreis eines Einweggleichrichters.} }  
            %\label{fig:EinweggleichrichterSchaltung}
        \end{figure}
    }
\end{frame}

\begin{frame}
    \b{ \frametitle{Spannungsverlauf eines Einweggleichrichters}
        \begin{figure}[H]

            \centering
                 \includesvg[width=\textwidth]{Bilder/kap2/FolienSpannungsverlaufGleichrichter.svg}

            %\label{fig:Einweggleichrichter}
            %\caption{\textbf{Einweggleichrichter.} Spannungsverlauf eines Gleichrichters.  
        \end{figure}
    }   

\end{frame}

\begin{frame}
    \b{ \frametitle{ Stromverlauf eines Einweggleichrichters}
    
    \begin{figure}
        \centering
        \includesvg[width=\textwidth]{Bilder/kap2/FolienStromverlaufGleichrichter.svg}
        %\caption{\textbf{Stromverlauf eines Einweggleichrichters.} }  
        %\label{fig:EinweggleichrichterStromverlauf}
    \end{figure}
    }
\end{frame}


\begin{frame}
    \b{ \frametitle{Diode -- Anwendungen/ Grundschaltungen}
        \begin{figure}[H]
            \centering
            \begin{tikzpicture}[]
    \draw (-0.5,0)
        to[sV, v<, name=UE] (-0.5,3)
        to[short, -*] (3,3)
        to[D, -*, l=$D_2$] (4.5,1.5)
        to[short, i, -] (4.5,2.5)
        to[short, -] (6,2.5)
        to[R, v, name=R, l=$R$] (6,-0.5)
        to[short, -] (1.5,-0.5)
        to[short, -*] (1.5,1.5)
        to[D, -*, l=$D_3$] (3,0)
        to[short, -] (-0.5,0);
    \draw (3,0)
        to[D, -, l=$D_4$] (4.5,1.5);
    \draw (1.5,1.5)
        to[D, -, l=$D_1$] (3,3);

    \varrmore{UE}{$u_\mathrm{E}$};
    \varrmore{R}{$u_\mathrm{R}$};
\end{tikzpicture}
            %\label{fig:BrueckengleichrichterSchaltung}
            %\caption{\textbf{Brückengleichrichter Schaltung.} Schaltkreis eines Brückengleichrichters mit angelegter Last $R$.} 
        \end{figure}
    }   
\end{frame}

\begin{frame}
    \b{ \frametitle{Diode -- Anwendungen/ Grundschaltungen}
        \begin{figure}[H]
            \centering
            \begin{tikzpicture}[]
    \draw (-0.5,0)
        to[sV, v<, name=UE] (-0.5,3)
        to[short, -*] (3,3)
        to[D, -*, l=$D_2$] (4.5,1.5)
        to[short, i, -] (4.5,2.5)
        to[short, -] (7.5,2.5)
        to[R, v, name=R, l=$R$] (7.5,-0.5)
        to[short, -] (1.5,-0.5)
        to[short, -*] (1.5,1.5)
        to[D, -*, l=$D_3$] (3,0)
        to[short, -] (-0.5,0);
    \draw (3,0)
        to[D, -, l=$D_4$] (4.5,1.5);
    \draw (1.5,1.5)
        to[D, -, l=$D_1$] (3,3);

    \draw (5.75,2.5) to[C,*-*] (5.75,-0.5);

    \varrmore{UE}{$u_\mathrm{E}$};
    \varrmore{R}{$u_\mathrm{R}$};
\end{tikzpicture}

            %\label{fig:Brueckengleichrichter mit Glättungskondensator}
            %\caption{\textbf{Brückengleichrichterschaltung mit Glättungskondensator}} 
        \end{figure}
    }   
\end{frame}

\begin{frame}
    \b{ \frametitle{Diode -- Anwendungen/ Grundschaltungen}
        \begin{figure}[H]
            \begin{minipage}[c]{0.48\textwidth}
                \raggedright
                \begin{tikzpicture}[/tikz/circuitikz/bipoles/length=1cm, scale=.8]
    \small
    \draw (-0.5,0)[blue]
        to[sV, v<, name=UE] (-0.5,3)
        to[short, i, -*, name=I1] (3,3)
        to[D, -*, l=$D_2$, fill=blue] (4.5,1.5)
        to[short, i, -] (4.5,2.5)
        to[short, i, -, name=I2] (6,2.5)
        to[R, v, name=R, l=$R$] (6,-0.5)
        to[short, i, -, name=I3] (1.5,-0.5)
        to[short, -*] (1.5,1.5)
        to[D, -*, l=$D_3$, fill=blue] (3,0)
        to[short, i, -, name=I4] (-0.5,0);
    \draw (3,0)
        to[D, -, l=$D_4$] (4.5,1.5);
    \draw (1.5,1.5)
        to[D, -, l=$D_1$] (3,3);

    \varrmore[black]{UE}{$u_\mathrm{E}$};
    \varrmore[blue]{R}{$u_\mathrm{R}$};
    \iarrmore[blue]{I1}{};
    \iarrmore[blue]{I2}{};
    \iarrmore[blue]{I3}{};
    \iarrmore[blue]{I4}{};
\end{tikzpicture}
            \end{minipage}
            \begin{minipage}[c]{0.48\textwidth}
                \raggedleft
                \begin{tikzpicture}[/tikz/circuitikz/bipoles/length=1cm, scale=.8]
    \small
    \draw (-0.5,0)[red]
        to[short, i^, -*, name=I4] (3,0)
        to[D, -*, l=$D_4$, fill=red] (4.5,1.5)
        to[short, i, -] (4.5,2.5)
        to[short, i, -, name=I2] (6,2.5)
        to[R, v, name=R, l=$R$] (6,-0.5)
        to[short, i, -, name=I3] (1.5,-0.5)
        to[short, -*] (1.5,1.5)
        to[D, -*, l=$D_1$, fill=red] (3,3)
        to[short, i, -, name=I1] (-0.5,3)
        to[sV, v<, name=UE] (-0.5,0);
    \draw (3,3)
        to[D, -, l=$D_2$] (4.5,1.5);
    \draw (1.5,1.5)
        to[D, -, l=$D_3$] (3,0);

    \varrmore[black]{UE}{$u_\mathrm{E}$};
    \varrmore[red]{R}{$u_\mathrm{R}$};
    \iarrmore[red]{I1}{};
    \iarrmore[red]{I2}{};
    \iarrmore[red]{I3}{};
    \iarrmore[red]{I4}{};
\end{tikzpicture} 
            \end{minipage}
        
            %\label{fig:BrueckengleichrichterBeschaltet}
            %\caption{\textbf{Strompfade im Brückengleichrichter.} Links: Strompfad bei der positiven Halbwelle der 
            %Eingangsspannung. Rechts: Strompfad bei der negativen Halbwelle der Eingangsspannung.} 
        \end{figure}
    }   
\end{frame}

\begin{frame}
    \b{ \frametitle{Diode -- Anwendungen/ Grundschaltungen}
        \begin{figure}[H]
            \centering
            \includesvg[width= 0.7\textwidth]{Bilder/kap2/BrueckengleichrichterSignalverlaeufe.svg}
            %\caption{\textbf{Brückengleichrichter Signalverläufe.} Verlauf der Eingangsspannung und Lastspannung des 
            %Brückengleichrichters }  
            %\label{fig:BrueckengleichrichterSignalverlaeufe}
        \end{figure}
    }
\end{frame}

\begin{frame}
    \b{ \frametitle{Delon-Schaltung}
    \begin{itemize}
        \item Beinhaltet eine Gleichrichterschaltung aus zwei Dioden und zwei Kondensatoren.
        \item Positive Halbwelle lädt den Kondensator $C_1$ über die Diode $D_1$ auf.
        \item Negative Halbwelle lädt den Kondensator $C_2$ über die Diode $D_2$ auf.
        \item Dadurch, dass der Ausgang parallel zu den in Reihe geschalteten Kondensatoren liegt, wird die Spannung verdoppelt. 
    \end{itemize}
    \begin{figure}[H]
        \centering
        \scalebox{0.8}{\begin{tikzpicture}
    
\draw (0,0) to[short,-*] (2,0) to [D,l=$D_1$, -*] (4,0) -- (7,0) to[R,v>,l=$R_\mathrm{Last}$,name=Ulast] (7,-4) -- (4,-4);


\draw (0,0) to [vsourcesin,name=U0] (0,-2) to[short, -*](4,-2) to[C,l=$C_1$] (4,0);
\draw (4,-4) to[C,l=$C_2$,*-](4,-2);
\draw (4,-4) to[D,l=$D_2$] (2,-4) -- (2,0);

\varrmore {U0}{$U_0$};
\varrmore{Ulast}{$U_\mathrm{Last}$};

\end{tikzpicture}}
        %\caption{\textbf{Delon-Schaltung.}  
        %\label{fig:Delonschaltung}
    \end{figure}
    }
\end{frame}

\begin{frame}
    \b{\frametitle{Diode -- Anwendungen/ Grundschaltungen}
        \textbf{Merke:}
        \begin{itemize}
            \item Dioden werden in Gleichrichter eingesetzt um Wechselspannungen in Gleichspannungen umzuformen.
            \item In Einweggleichrichter wird nur eine Polarität der Halbwellen durchgelassen.
            \item Brückengleichter nutzen beide Polaritäten der Eingangsspannung.
        \end{itemize}
        }
\end{frame}

\begin{frame}
    \b{ \frametitle{Bipolartransistor}
    \begin{figure}[H]
        \centering
        \scalebox{0.8}{
\begin{circuitikz}[scale=.5]

\draw [ fill={rgb,255:red,0; green,38; blue,191}, opacity=0.27 ] (2.5,13.5) rectangle (5,9.75) node[pos=.5, opacity=1] {\textcolor{black}{$\mathrm{n}$}};
\draw [ fill={rgb,255:red,161; green,0; blue,0}, opacity=0.40 ] (5,13.5) rectangle (7.5,9.75) node[pos=.5, opacity=1] {\textcolor{black}{$\mathrm{p}$}};
\draw [ fill={rgb,255:red,0; green,38; blue,191}, opacity=0.27 ] (7.5,13.5) rectangle (10,9.75) node[pos=.5, opacity=1] {\textcolor{black}{$\mathrm{n}$}};
\draw [ fill={rgb,255:red,161; green,0; blue,0}, opacity=0.40 ] (15,13.5) rectangle (17.5,9.75) node[pos=.5, opacity=1] {\textcolor{black}{$\mathrm{p}$}};
\draw [ fill={rgb,255:red,0; green,38; blue,191}, opacity=0.27 ] (17.5,13.5) rectangle (20,9.75) node[pos=.5, opacity=1] {\textcolor{black}{$\mathrm{n}$}};
\draw [fill={rgb,255:red,161; green,0; blue,0}, opacity=0.40] (20,13.5) rectangle (22.5,9.75) node[pos=.5, opacity=1] {\textcolor{black}{$\mathrm{p}$}};

    \draw (6.25,13.5) to[short, -o] (6.25,14.75) node[above] {$\mathrm{B}$} ;
    \draw (2.5,11.5) to[short, -o] (1.25,11.5)node[above] {$\mathrm{E}$} ;
    \draw (10,11.5) to[short, -o] (11.25,11.5) node[above] {$\mathrm{C}$};
    \draw (15,11.5) to[short, -o] (13.75,11.5) node[above] {$\mathrm{E}$};
    \draw (18.75,13.5) to[short, -o] (18.75,14.75)node[above] {$\mathrm{B}$} ;
    \draw (22.5,11.5) to[short, -o] (23.75,11.5) node[above] {$\mathrm{C}$};
    \draw (6.25,6) to[short, *-o] (6.25,7.25)node[above] {$\mathrm{B}$} ;
    \draw (18.75,6) to[short, *-o] (18.75,7.25) node[above] {$\mathrm{B}$};


    \draw (6.25,6) to[D] (3.75,6);
    \draw (6.25,6) to[D] (8.75,6);
    \draw (16.25,6) to[D] (18.75,6);
    \draw (21.25,6) to[D] (18.75,6);
    \draw (3.75,6) to[short, -o] (1.25,6) node[above] {$\mathrm{E}$} ;
    \draw (8.75,6) to[short, -o] (11.25,6) node[above] {$\mathrm{C}$} ;
    \draw (16.25,6) to[short, -o] (13.75,6) node[above] {$\mathrm{C}$} ;
    \draw (21.25,6) to[short, -o] (23.75,6) node[above] {$\mathrm{C}$} ;
    
    %\draw (6.25,0) to[Tnpn] (6.25,3.5)node[above] {$\mathrm{C}$}; 
    %\draw (5.4,1.75) to[short] (5.25,1.75)node[left] {$\mathrm{B}$};
    %\draw (18.75,0) to[Tpnp] (18.75,3.5)node[above] {$\mathrm{E}$}; 
    %\draw (17.9,1.75) to[short] (17.75,1.75) node[left] {$\mathrm{B}$};
    %\draw (6.25, 0) to[short] (6.25,-0.5)node[below] {$\mathrm{E}$};
    %\draw (18.75,0) to[short] (18.75, -0.5) node[below] {$\textbf{C}$};

    \draw (6.25, 0) node[npn] (Q1){}
        (Q1.base) to[short, -o] ++ (-0.5, 0) node[left] {$\mathrm{B}$}
        (Q1.collector) to[short, -o] ++ (0,1) node[above] {$\mathrm{C}$}
        (Q1.emitter) to[short, -o] ++ (0, -1) node[below] {$\mathrm{E}$};
    
    \draw (18.75, 0) node[pnp, yscale=-1, ] (Q2){}
        (Q2.base) to[short, -o] ++ (-0.5, 0) node[left] {$\mathrm{B}$}
        (Q2.collector) to[short, -o] ++ (0,1) node[above] {$\mathrm{C}$}
        (Q2.emitter) to[short, -o] ++ (0, -1) node[below] {$\mathrm{E}$};
\end{circuitikz}
}
        %\caption{\textbf{Übersicht Bipolartransistoren.} Vereinfachte Darstellung des Querschnittes, Logik und Schaltzeichen von npn- und pnp-Transistoren.}  
        %\label{fig:UebersichtBipolartransistoren}
    \end{figure}
    }
\end{frame}

\begin{frame}
    \b{ \frametitle{Bipolartransistor -- Elektrisches Verhalten}
        \begin{figure}[H]
            \centering
            \includesvg[width= 0.5 \textwidth]{Bilder/kap3/VerhaltenUnbeschalteterBipolartransistor.svg}
            %\caption{\textbf{Verhalten unbeschalteter Bipolartransistor.} Querschnitt eines unbeschalteten npn-Transistors und zugehörige Bändermodell}  
            %\label{fig:VerhalenUnbeschalteterBipolartransistor}
        \end{figure}
    }
\end{frame}

\begin{frame}
    \b{ \frametitle{Bipolartransistor -- Elektrisches Verhalten}
        \begin{figure}[H]
            \centering
            \includesvg[width= 0.7 \textwidth]{Bilder/kap3/BipolartransistorMitKollektor-Emitter-Spannung.svg}
            %\caption{\textbf{Bipolartransistor mit Kollektor-Emitter-Spannung.} Querschnitt eines npn-Transistors und zugehörige Bändermodell bei einer positiven Spannung zwischen Kollektor und Emitter.}  
            %\label{fig:BipolartransistorMitKollektor-Emitter-Spannung}
        \end{figure}
    }
\end{frame}

\begin{frame}
    \b{ \frametitle{Bipolartransistor -- Elektrisches Verhalten}
        \begin{figure}[H]
            \centering
            \includesvg[width= 0.7\textwidth]{Bilder/kap3/BeschalteterBipolartransistor.svg}
            %\caption{\textbf{Beschalteter Bipolartransistor} Querschnitt eines npn-Transistors und zugehörige Bändermodell bei einer positiven Spannung zwischen Kollektor und Emitter sowie Basis und Emitter}  
            %\label{fig:BeschalteterBipolartransistor}
        \end{figure}
        }
\end{frame}

\begin{frame}
    \b{\frametitle{Bipolartransistor -- Elektrisches Verhalten}
        \textbf{Merke:}
        \begin{itemize}
            \item Bipolartransistoren weisen zwei pn-Übergänge auf.
            \item Es gibt drei Anschlüsse, den Kollektor (C), die Basis (B) und den Emitter (E).
        \end{itemize}
        }
\end{frame}

\begin{frame}
    \b{ \frametitle{Bipolartransistor -- Elektrisches Verhalten}    
        \begin{figure}[H]
            \begin{minipage}[c]{0.48\textwidth}
                \raggedright
                \scalebox{0.9}{\begin{tikzpicture} [scale=.9, font=\small]
    \draw (2,0) node[npn] (Q1) {};

    \draw(0,0)
        to[short, i>, name=ib] (Q1.B);

    \draw (Q1.C)
        to [short, i<, name=ic] (2,1)
        to [short] (2,2)
        to [short] (4,2)
        to [short] (4,0)
        to [V,v>, name=uce](4,-2)
        to [short, -*] (2,-2)
        to [short] (0,-2)
        to [V, v<, name=ube] (0,0);

    \draw (2,-2) 
        to[short] (2,-1)
        to[short, i>, name=ie] (Q1.E);

    \varrmore {uce}{$U_\mathrm{CE} > 0$};
    \varrmore {ube}{$U_\mathrm{BE} > 0$};
    \iarrmore {ib}{$I_\mathrm{B} > 0$};
    \iarrmore {ic}{$I_\mathrm{C} > 0$};
    \iarrmore {ie}{$I_\mathrm{E} < 0$};

\end{tikzpicture}}
            \end{minipage}
            \begin{minipage}[c]{0.48\textwidth}
                \raggedleft
                \scalebox{0.9}{\input{Tikz/StroemeUndSpannungenBeimBipolartransistor_2.tex}}
            \end{minipage}
            %\caption{\textbf{Ströme und Spannungen beim Bipolartransistor.} Relevante Ströme und Spannungen bei npn und pnp 
            %Transistor.}  
            %\label{fig:StroemeUndSpannungenBeimBipolartransistor}
        \end{figure}
    }
\end{frame}

\begin{frame}
    \b{ \frametitle{Transistorspannungen und -ströme}
    \begin{columns}
        \column[c]{0.3\textwidth}
            \resizebox{0.8\textwidth}{!}{\begin{tikzpicture}

\draw (0,0) node[npn] (Q1) {};
\draw (-3,0)node[left]{B}  to[short,i,name=IB, o-] (Q1.B);
\draw (0,3)node[above]{C}  to[short,i,name=IC, o-] (Q1.C);
\draw (Q1.E) to[short,i,name=IE, -o] (0,-3)node[below]{E} ;
\draw (0,3) to[open,v>,name=UCB] (-3,0);
\draw (-3,0) to[open,v>,name=UBE] (0,-3);
\draw (2.5,3) to[open,v>,name=UCE] (2.5,-3);
\draw[thick] (-0.2,0) circle (1cm);

\iarrmore{IE}{$I_\mathrm{E}$};
\iarrmore{IB}{$I_\mathrm{B}$};
\iarrmore{IC}{$I_\mathrm{C}$};
\varrmore{UCB}{$U_\mathrm{CB}$};
\varrmore{UBE}{$U_\mathrm{BE}$};
\varrmore{UCE}{$U_\mathrm{CE}$};

\end{tikzpicture}}     
            \column[c]{0.6\textwidth}
        
                \begin{itemize}
                    \item Werte im durchgeschalteten Zustand:
                    \item $U_\mathrm{BE} \approx 0.6\,\mathrm{V}$ (Basis-Emitter-Spannung)
                    \item $U_\mathrm{CB} \approx 2 ... 300\,\mathrm{V}$ (Kollektor-Basis-Spannung)
                    \item $I_\mathrm{B} = 1 \, \%; I_\mathrm{C} = 99 \, \%; I_\mathrm{E} = 100 \, \%$
                    \item Im Transistor gelten Knoten- und Maschenregel:
                    \item $U_\mathrm{CE} = U_\mathrm{CB} + U_\mathrm{BE}$ = 0
                    \item $I_\mathrm{B} + I_\mathrm{C} - I_\mathrm{E} = 0$
                    \item Transistor verstärkt Strom und Spannung:
                    \item $ B = \frac{I_\mathrm{C}}{I_\mathrm{B}}$ 
                    \item $ D = \frac{\Delta U_\mathrm{BE}}{\Delta U_\mathrm{CE}}$
                \end{itemize}
            \end{columns}}
        \end{frame}

\begin{frame}
    \b{ \frametitle{Transistorschaltungen -- AC-Verstärker}
        \begin{figure}[H]
                \begin{tikzpicture} [scale= 0.45, font=\scriptsize, /tikz/circuitikz/bipoles/length=.8cm] 
    \draw (0,0) 
        node[above left] {B} to[sV, v, name=ue] (0,-4)
        to[short, -*] (4,-4)
        to[short] (5,-4) node[below]{E}
        to[short, -*] (6,-4) node[ground] {}
        to[short, -o] (9,-4);
    
    \draw (0,0) to[short, i>, name=ie] (1,0)
        to[R=$r_\mathrm{BE}$] (3,0)
        to[short, -*] (4,0)
        to[short, -*] (6,0)
        to[short, i<, name=ia] (9,0)
        to[short, -o] (9,0);
    
    \draw (6,2.5) node[left] {C} to[I, i>^, name=i1] (6,0);
    \draw (6,2.5) to[short] (8.5,2.5) node [ground]{};
    
    \draw (4,0) to[R=$R_\mathrm{E}$] (4,-4);
    
    \draw (9, 0) to[open,v>, name=ua] (9,-4);
    
    \varrmore{ua}{$u_\mathrm{a}$};
    \iarrmore{ie}{$i_\mathrm{e}$};
    \iarrmore{ia}{$i_\mathrm{a}$};
    \iarrmore{i1}{$\beta \cdot i_\mathrm{b}$};
    \varrmore{ue}{$u_\mathrm{e}$}; 
    
\end{tikzpicture} 
                \caption{\textbf{}}
        \end{figure}
    }
\end{frame}
\begin{frame}
    \b{ \frametitle{Transistorschaltungen -- AC-Verstärker}
        \begin{figure}[H]
            \begin{circuitikz}

\draw (2,0) node[npn](Q1){};

\draw (2,0) node[npn, tr circle] (Q1) {}
    (Q1.base) node[anchor=south east] {B}
    (Q1.collector) node[anchor=east] {C}
    (Q1.emitter) node[anchor=east] {E};

\draw (-0.5,0) to[R,l=$R_\mathrm{1}$] (-0.5,4)to[short,-*](2,4)to[short,-o](4,4)node[right]{$U_\mathrm{S}$};
\draw (-2,0) to[short,i>, name=ie, o-] (-1.5,0) to[C,-*] (-0.5,0) to[short, i>,name=ib] (Q1.B);

\draw (-2,-2)node[ground]{} to[short,-o](-2,-2);
\draw (-2,0) to[open,v,name=Ue](-2,-2);

\draw (4,-2) node[ground]{} to[short,-o](4,-2);
\draw (4,2) to[open,v,name=Ua](4,-2);


\draw (2,4) to[R,l=$R_\mathrm{C}$,-*] (2,2);
\draw (4,2) to[short,i>,name=ia, o-](3.5,2) to[C] (2.5,2) -- (2,2);
\draw (2,2) to[short,i>,name=ic] (Q1.C);

\draw (1,-0.75) to[open, v>, name=ube] (1.25,-1); % Dieser Pfeil sollte eigentlich nicht in der normalen Spannungsfarbe angezeigt werden, da im Bild daneben mit den Spannungsverläufen U_BE rot eingezeichnet ist... 

\draw (3.25,1) to[open,v>, name=uce] (3.25,-1);

\draw (Q1.E) to[short] (2,-2) node[ground]{};


\varrmore{uce}{$U_\mathrm{CE}$};
\varrmore{ube}{$U_\mathrm{BE}$};
\varrmore{Ue}{$U_\mathrm{e}$};
\varrmore{Ua}{$U_\mathrm{a}$};

\iarrmore {ia}{$i_\mathrm{a}$};
\iarrmore{ic}{$i_\mathrm{c}$};
\iarrmore{ib}{$i_\mathrm{b}$};
\iarrmore{ie}{$i_\mathrm{e}$};

\end{circuitikz}
        \end{figure}
    }
\end{frame}


\begin{frame}
    \b{ \frametitle{Bipolartransistor -- Elektrisches Verhalten}
        \begin{figure}[H]
            \centering
            \includesvg[width= 0.7\textwidth]{Bilder/kap3/EingangskennlinieEinesNpnTransistors.svg}
            %\caption{\textbf{Eingangskennlinie eines npn Transistors.} }  
            %\label{fig:EingangskennlinieEinesNpnTransistors}
        \end{figure}
    }
\end{frame}

\begin{frame}
    \b{\frametitle{Bipolartransistor -- Elektrisches Verhalten}
        \textbf{Merke:}
        \begin{itemize}
            \item Die Eingangskennlinie $I_\mathrm{B}=f(U_\mathrm{BE})$ gleicht der einer Diode.
            \item Eine Änderung von $U_\mathrm{CE}$ führt zu einer Verschiebung der Kennlinie.
        \end{itemize}
        }
\end{frame}

\begin{frame}
    \b{ \frametitle{Bipolartransistor -- Elektrisches Verhalten}
        \begin{figure}[H]
            \centering
            \includesvg[width= \textwidth]{Bilder/kap3/AusgangskennlinienfeldEinesNpnTransistors.svg}
            %\caption{\textbf{Ausgangskennlinienfeld eines npn Transistors.}Kennzeichnung von relevanten Bereichen und Größen sowie 
            %grafische Ermittlung der Early-Spannung}  
            %\label{fig:AusgangskennlinienfeldEinesNpnTransistors}
        \end{figure}}
\end{frame}

\begin{frame}
    \b{ \frametitle{Ausgangskennlinie}
    \begin{itemize}
    \item Zusammenhang zwischen Kollektor-Emitter-Spannung $U_\mathrm{CE}$ und Kollektorstrom $I_\mathrm{C}$ bei konstantem Basisstrom $I_\mathrm{B}$.
    \item Differentieller Widerstand $r_\mathrm{CE} = \frac{\Delta U_\mathrm{CE}}{\Delta I_\mathrm{C}}$.
\end{itemize}
\begin{figure}[H]
            \begin{minipage}[b]{0.48\textwidth}
                \raggedleft
                \scalebox{0.9}{\begin{circuitikz}
\draw (0,0) node[npn] (Q1) {};
\draw (2,1.5) to[short, o-](0,1.5) to[short, i>, name=IC] (Q1.C);

\draw (-2,0) to[short, i>, name= IB, o-] (Q1.B);
\draw (Q1.E) to[short, -*] (0,-2)node[ground]{};
\draw (-2,-2) to[short, o-o] (2,-2);

\draw (-2,0) to[open,v>,name=UBE](-2,-2);
\draw (2,1.5) to[open,v,name=UCE](2,-2);

\varrmore{UBE}{$U_\mathrm{BE}$};
\varrmore{UCE}{$U_\mathrm{CE}$};
\iarrmore{IB}{$I_\mathrm{B} = konst$};
\iarrmore{IC}{$I_\mathrm{C}$};
\end{circuitikz}}
            \end{minipage}
            \begin{minipage}[b]{0.48\textwidth}
                \raggedright
                \scalebox{0.9}{\includesvg[width= \textwidth]{Bilder/kap2/FolienAusgangskennlinie.svg}}
            \end{minipage}
            %\caption{\textbf{Ersatzschaltbild eines Bipolartransistors.} Links: vollständiges Kleinsignalersatzschaltbild mit allen 
            %relevanten Komponenten und Größen, rechts: Vereinfachung durch Vernachlässigung der differentiellen Stromverstärkung.}  
            %\label{fig:ErsatzschaltbildEinesBipolartransistors}
        \end{figure}
    }
\end{frame}

\begin{frame}
    \b{\frametitle{Bipolartransistor -- Elektrisches Verhalten}
        \textbf{Merke:}
        \begin{itemize}
            \item Die Ausgangskennlinie stellt $I_\mathrm{C}=f(U_\mathrm{CE})$ dar.
            \item Im aktiven Bereich flacht der Verlauf ab, der Einfluss von $U_\mathrm{CE}$ auf $I_\mathrm{C}$ sinkt.
            \item $I_\mathrm{B}$ beeinflusst die Höhe der Kennlinie.
        \end{itemize}
        }
\end{frame}

\begin{frame}
    \b{ \frametitle{Stromsteuerungskennlinie}
\begin{itemize}
    \item Zusammenhang zwischen Basisstrom $I_\mathrm{B}$ und Kollektorstrom $I_\mathrm{C}$ bei konstanter Kollektor-Emitter-Spannung $U_\mathrm{CE}$.
    \item Keine Gerade: Zuerst linearer Anstieg, dann nimmt die Steigung mit wachsendem Basisstrom zu.
    \end{itemize}
\begin{figure}
    \centering
\begin{minipage}[b]{0.48\textwidth}
                \raggedleft
                                 \scalebox{0.9}{\begin{circuitikz}
\draw (0,0) node[npn] (Q1) {};
\draw (2,1.5) to[short, o-](0,1.5) to[short, i>, name=IC] (Q1.C);

\draw (-2,0) to[short, i>, name= IB, o-] (Q1.B);
\draw (Q1.E) to[short, -*] (0,-2)node[ground]{};
\draw (-2,-2) to[short, o-o] (2,-2);

\draw (-2,0) to[open,v>,name=UBE](-2,-2);
\draw (2,1.5) to[open,v,name=UCE](2,-2);

\varrmore{UBE}{$U_\mathrm{BE}$};
\varrmore{UCE}{$U_\mathrm{CE}$};
\iarrmore{IB}{$I_\mathrm{B} = konst$};
\iarrmore{IC}{$I_\mathrm{C}$};
\end{circuitikz}}
            \end{minipage}
            \begin{minipage}[b]{0.48\textwidth}
                \raggedright

                 \scalebox{1}{\includesvg[width= \textwidth]{Bilder/kap2/FolienStromsteuerungskennlinie.svg}}
            \end{minipage}
\end{figure}
    }
\end{frame}

\begin{frame}
    \b{
\frametitle{Stromsteuerungskennlinie}
\begin{itemize}
    \item Kurvenschar bei variabler Kollektor-Emitter-Spannung $U_\mathrm{CE}$.
    \end{itemize}
\begin{figure}
    \centering
    \scalebox{1}{\includesvg[width= \textwidth]{Bilder/kap2/FolienStromsteuerungskennlinieKurvenschar.svg}}
\end{figure}
    }
\end{frame}

\begin{frame}
    \b{ \frametitle{Rückwirkungskennlinie}
\begin{figure}
    \centering
    \includesvg[width= \textwidth]{Bilder/kap2/FolienRueckwirkungskennlinie.svg}
    %\caption{\textbf{Rückwirkungskennlinie}
    %\label{fig:Rückwirkungskennlinie}
\end{figure}
    }
\end{frame}

\begin{frame}
    \b{ \frametitle{Bipolartransistor -- Elektrisches Verhalten}
        \begin{figure}[H]
            \centering
            \includesvg[scale=.5, pretex=\small]{Bilder/kap3/Vierquadranten-KennlinienfeldEinesNpnTransistors.svg}
            %\caption{\textbf{Vierquadranten-Kennlinienfeld eines npn Transistors }}  
            %\label{fig:Vierquadranten-KennlinienfeldEinesNpnTransistors}
        \end{figure}
    }
\end{frame}

\begin{frame}
    \b{ \frametitle{Bipolartransistor -- Elektrisches Verhalten}
        \begin{figure}[H]
            \begin{minipage}[b]{0.48\textwidth}
                \raggedright
                \scalebox{0.9}{\begin{tikzpicture} [
    /tikz/circuitikz/bipoles/length=1cm,
    scale=.7]
    
    \small
    \draw(0,0)
        to[short, o-] (3,0)
        to[V, v<, name=V1, *-] (3,3)
        to[R, i<, name=RBE , l =$r_\mathrm{BE}$, -o] (0,3)
        node[anchor=east] {B}
        to[open,v>, name=UBE, white] (0, 0);
        
    \draw(8,3)   
        node[anchor=west] {C}
        to[short, i, name=IC, o-] (6,3)
        to[short, -] (4.5,3)
        to[I, name=I1, -*] (4.5,0)
        node[anchor=north] {E}
        to[short, -o] (8,0)
        to[open,v<, name=UCE] (8, 3);

    \draw(6,3)
       to[R, name=RCE , l =$r_\mathrm{CE}$, *-*] (6,0);

    \draw(4.5,0)
        to[short] (3,0);

    \iarrmore {RBE}{$i_\mathrm{B}$};
    \iarrmore {IC}{$i_\mathrm{C}$};
    \iarrmore {I1}{$\beta\cdot i_\mathrm{B}$};
    \varrmore{UBE}{$u_\mathrm{BE}$};
    \varrmore{UCE}{$u_\mathrm{CE}$};
    \varrmore{V1}{$D\cdot u_\mathrm{CE}$};

\end{tikzpicture}}
            \end{minipage}
            \begin{minipage}[b]{0.48\textwidth}
                \raggedleft
                \scalebox{0.9}{\begin{tikzpicture} [
    /tikz/circuitikz/bipoles/length=1cm,
    scale=.7]
    
    \small
    \draw(1,0)
        to[short, o-*] (3,0)
        to[R, name=RBE , l =$r_\mathrm{BE}$, -] (3,3)
        to[short, i<, name=IB, -o] (1,3)
        node[anchor=east] {B}
        to[open,v>, name=UBE, white] (1, 0);
        
    \draw(8,3)   
        node[anchor=west] {C}
        to[short, i, name=IC, o-] (6,3)
        to[short, -] (4.5,3)
        to[I, name=I1, -*] (4.5,0)
        node[anchor=north] {E}
        to[short, -o] (8,0)
        to[open,v<, name=UCE] (8, 3);

    \draw(6,3)
       to[R, name=RCE , l =$r_\mathrm{CE}$, *-*] (6,0);

    \draw(4.5,0)
        to[short] (3,0);
       

    \iarrmore {IB}{$i_\mathrm{B}$};
    \iarrmore {IC}{$i_\mathrm{C}$};
    \iarrmore {I1}{$\beta\cdot i_\mathrm{B}$};
    \varrmore{UBE}{$u_\mathrm{BE}$};
    \varrmore{UCE}{$u_\mathrm{CE}$};

\end{tikzpicture}}
            \end{minipage}
            %\caption{\textbf{Ersatzschaltbild eines Bipolartransistors.} Links: vollständiges Kleinsignalersatzschaltbild mit allen 
            %relevanten Komponenten und Größen, rechts: Vereinfachung durch Vernachlässigung der differentiellen Stromverstärkung.}  
            %\label{fig:ErsatzschaltbildEinesBipolartransistors}
        \end{figure}
    }
\end{frame}

\begin{frame}
    \b{\frametitle{Bipolartransistor -- Elektrisches Verhalten}
        \textbf{Merke:}
        \begin{itemize}
            \item Das Vierquadrantenkennlinienfeld zeigt die Strom-Spannungs-Kennlinien am Ein- und Ausgang sowie die Kopplungen dazwischen. 
            \item Mittels der differentiellen Größen kann das Ersatzschaltbild aufgestellt werden. 
        \end{itemize}
        }
\end{frame}

\begin{frame}
    \b{ \frametitle{Bipolartransistor -- Beispiel 2.1}
            Die Versorgungsspannung $U_\mathrm{V}$ beträgt $20\,\mathrm{V}$. Über den Widerstand $R_C$, 
            der einen ohmschen Verbraucher repräsentiert, soll die halbe Versorgungsspannung anliegen und ein Strom von $10\,\mathrm{mA}$ fließen.
            
            %Abbildung 34
            \begin{figure}[H]
                \centering
            \begin{minipage}[b]{0.48\textwidth}
                \raggedright
                \scalebox{0.5}{ \includesvg{Bilder/kap3/ArbeitenMitTransistorkennlinie.svg}}
            \end{minipage}
            \begin{minipage}[b]{0.48\textwidth}
                \raggedleft
                \scalebox{0.8}{\input{Tikz/BeispieleinerTransistorschaltungzurBestimmungvonrelevantenParametern.tex}}
            \end{minipage}
                            %\caption{\textbf{Beispiel einer Transistorschaltung zur Bestimmung von relevanten Parametern} }  
                %\label{fig:BeispieleinerTransistorschaltungZurBestimmungvonRelevantenParametern}
                
            \end{figure}
    }
\end{frame}

\begin{frame}
    \b{ \frametitle{Bipolartransistor -- Beispiel 2.2}
      \begin{figure}[H]
            \centering
            \includesvg[scale=.5, pretex=\small]{Bilder/kap3/Vierquadranten-KennlinienfeldMitArbeitspunkt.svg}
            %\caption{\textbf{Vierquadranten-Kennlinienfeld mit Arbeitspunkt A.} }  
            %\label{fig:Vierquadranten-KennlinienfeldMitArbeitspunkt}
        \end{figure}
    }
\end{frame}

\begin{frame}
    \b{ \frametitle{Bipolartransistor -- Beispiel 2.2}
        In den jeweiligen Quadranten ergeben sich die folgenden Arbeitspunkte und Kenngrößen.
        \begin{flalign*}
            &\text{Ausgang (bei: $I_\mathrm{B}={20}\,{\mathrm{\mu A}}$):}&&\\
            &&U_\mathrm{CE} &= {10}\,\mathrm{V}\\
            &&I_\mathrm{C} &= {10}\,\mathrm{mA}  \\
            &&R_\mathrm{C} &= \frac{U_\mathrm{RC}}{I_\mathrm{RC}} = \frac{U_\mathrm{V}/2}{I_\mathrm{C}} = \frac{{10}\,\mathrm{V}}{{10}\,\mathrm{mA}} = {1000}\,{\Omega}\\
            &\text{Stromsteuerung:}&&\\
            &&I_\mathrm{C} &= {10}\,\mathrm{mA} \\
            &&I_\mathrm{B} &= {25}\,\mathrm{\mu A} \\
            &&B&=\frac{I_\mathrm{C}}{I_\mathrm{B}}=\frac{{10}\,\mathrm{mA}}{{25}\,\mathrm{\mu A}}=400 \\
        \end{flalign*}
    }
\end{frame}

\begin{frame}
    \b{ \frametitle{Bipolartransistor -- Beispiel 2.2}
        In den jeweiligen Quadranten ergeben sich die folgenden Arbeitspunkte und Kenngrößen.
        \begin{flalign*}
            &\text{Eingang:}&&\\
            &&I_\mathrm{B} &= {25}\,\mathrm{\mu A} \\
            &&U_\mathrm{BE} &= {0,72}\,\mathrm{V} \\
            &&R_\mathrm{V} &= \frac{U_\mathrm{RV}}{I_\mathrm{RV}} = \frac{U_\mathrm{V} - U_\mathrm{BE}}{I_\mathrm{B}} = \mathrm{\frac{{20}\,{V} - {0,72}\,{V}}{{25}\,{\mu A}}} ={771,2}\,\mathrm{k \Omega}\\
            &\text{Rückwirkung:}&&\\
            &&U_\mathrm{BE} &= {0,72}\,\mathrm{V} \\
            &&U_\mathrm{CE} &= {10}\,\mathrm{V} \\
            &&D &= \frac{U_\mathrm{BE}}{U_\mathrm{CE}} = \mathrm{\frac{0,72\,V}{10\,V}} = {0,072}
        \end{flalign*}
    }
\end{frame}

\begin{frame}
    \b{\frametitle{Bipolartransistor -- Elektrisches Verhalten}
        \begin{figure}[H]
            \centering
            \includesvg[scale=.5, pretex=\small]{Bilder/kap3/Vierquadranten-KennlinienfeldBeiACEingangssignal.svg}
            %\caption{\textbf{Vierquadranten-Kennlinienfeld bei AC Eingangssignal} }  
            %\label{fig:Vierquadranten-KennlinienfeldBeiACEingangssignal}
        \end{figure}
    }
\end{frame}


\begin{frame}
     \b{ \frametitle{Bipolartransistor -- Aufbau}
         \begin{figure}[H]
            \centering
            \includesvg[width= 0.7\textwidth]{Bilder/kap3/Aufbaunpn-Bipolartransistor.svg}
            %\caption{\textbf{Aufbau npn-Bipolartransistor.} Links: idealer schematischer Aufbau eines npn-Bipolartransistors, angelehnt an dessen Schaltbild. Rechts: realer Aufbau eines planaren npn-Transistors.}  
            %\label{fig:AufbauNpn-Bipolartransistor}
        \end{figure}
        
     }
\end{frame}

\begin{frame}
     \b{ \frametitle{Bipolartransistor -- Anwendung/ Grundschaltungen} 
         \begin{figure}[H]
            \begin{minipage}[c]{0.3\textwidth}
                \raggedright
                    \scalebox{0.9}{\input{Tikz/Grundschaltungen1.tex}}
                \end{minipage}
                \begin{minipage}[c]{0.3\textwidth}
                \centering
                    \scalebox{0.9}{\begin{tikzpicture}[scale= 0.65, font=\small, /tikz/circuitikz/bipoles/length=1cm] 
    \draw (2,0) node[npn, tr circle, yscale=-1] (Q1) {}
        (Q1.base) node[anchor=south east] {B}
        (Q1.collector) node[anchor=east] {C}
        (Q1.emitter) node[anchor=east] {E};
    
    \draw(0,0)
        to[short, o-] (1,0)
        to (Q1.B);
    
    \draw (Q1.E)
        to [R] (2,4)
        to (4,4)
        to [short, -o](4,0);
    
    \draw [short, o-] (4, -0.5) to (4,-1.5)
        to [short, -*] (4, -2)
        to [short, -*] (2, -2)
        to [short, -o] (0, -2)
        to [short, -o] (5,-2);
    
    \draw (2,-2) to (Q1.C);
        
    \draw (2,1) to[short, *-] (2,1)
        to [short, -o] (5,1);
    
    \draw (0,0) to[open,v>, name=Ue] (0,-2);
    \varrmore{Ue}{$U_\mathrm{e}$};
    
    \draw (5,-2) to[open,v<, name=Ua] (5,1);
    \varrmore{Ua}{$U_\mathrm{a}$};
    
    \draw (3.75,0.25) to[open,v<, name=Uv] (3.75,-0.75);
    \varrmore{Uv}{$U_\mathrm{V}$};
    
\end{tikzpicture}}
                \end{minipage}
                \begin{minipage}[c]{0.3\textwidth}
                \raggedleft
                    \scalebox{0.9}{\begin{tikzpicture}[scale= 0.65, font=\small, /tikz/circuitikz/bipoles/length=1cm] 
    \draw (0,0) node [npn, tr circle, rotate=90, yscale=-1]  (Q1) {} 
        (Q1.base) node[anchor=north east ] {B}    
        (Q1.collector) node[anchor=south east ] {C} 
        (Q1.emitter) node[anchor=south ] {E};  

    \draw(-2,0)
        to[short, o-] (-0.5,0);

    \draw (Q1.C)
        to (1,0)
        to (1,0.75)
        to [R] (1,4)
        to (3,4)
        to [short, -o](3,0);


    \draw [short, o-] (3, -0.5) to (3,-1.5)
        to [short, -*] (3, -2)
        to [short, -*] (0, -2)
        to [short, -o] (-2, -2)
        to [short, -o] (4,-2);

    \draw (0,-2) to (Q1.B);
    
    \draw (1,1) to[short, *-] (1,1)
        to [short, -o] (4,1);

    \draw (-2,0) to[open,v>, name=Ue] (-2,-2);
    \varrmore{Ue}{$U_\mathrm{e}$};

    \draw (4,-2) to[open,v<, name=Ua] (4,1);
    \varrmore{Ua}{$U_\mathrm{a}$};

    \draw (2.75,0.25) to[open,v>, name=Uv] (2.75,-0.75);
    \varrmore{Uv}{$U_\mathrm{V}$};

\end{tikzpicture}}
                \end{minipage}
            %\caption{\textbf{Grundschaltungen npn-Bipolartransistor.} V.l.n.r Emitter-, Kollektor- und Basisschaltung.}  
            %\label{fig:GrundschaltunenBipolartransistor}
        \end{figure}}

\end{frame}


\begin{frame}
    \b{ \frametitle{Bipolartransistor -- Anwendung/ Grundschaltungen}
        \begin{figure}[H]
            \begin{minipage}[b]{0.3\textwidth}
                \raggedright
                \scalebox{0.9}{\begin{tikzpicture} [scale= 0.45, font=\scriptsize, /tikz/circuitikz/bipoles/length=.8cm] 
    \draw (0,0)
        to[sV, name=ue] (0,-4)
        to[short, -*] (4,-4)
        to[R=$r_{BE}$] (4,0)
        node[above] {B};
    
    \draw (0,0) to[short ,i, name=ie] (4,0);
    
    \draw (4,-4) to[short, -*] node[below left] {E} (6,-4)
        to[short, -*] (8,-4)
        to[short, -o] (10,-4);
    
    \draw (8,-4) 
        to[R=$R_{C}$] (8,0)
        to[short,i,name=ia, *-o](10,0);
    
    \draw (8,0) to [short] (6,0) node[above] {C}
        to[I, name=I1] (6,-4);
    
    \varrmore{ue}{$u_{e}$}; 
    \iarrmore{I1}{$\beta \cdot i_{B}$};
    \iarrmore{ie}{$i_e$};
    \iarrmore{ia}{$i_a$};
    
    \draw (10, 0) to[open,v>, name=ua] (10,-4);
    \varrmore{ua}{$u_\mathrm{a}$};
    
    \draw (6,-4) node[ground] {};
    
\end{tikzpicture} }
            \end{minipage}
            \begin{minipage}[b]{0.3\textwidth}
                \centering
                \scalebox{0.9}{\begin{tikzpicture} [scale= 0.45, font=\scriptsize, /tikz/circuitikz/bipoles/length=.8cm] 
    \draw (0,0) 
        node[above left] {B} to[sV, v, name=ue] (0,-4)
        to[short, -*] (4,-4)
        to[short] (5,-4) node[below]{E}
        to[short, -*] (6,-4) node[ground] {}
        to[short, -o] (9,-4);
    
    \draw (0,0) to[short, i>, name=ie] (1,0)
        to[R=$r_\mathrm{BE}$] (3,0)
        to[short, -*] (4,0)
        to[short, -*] (6,0)
        to[short, i<, name=ia] (9,0)
        to[short, -o] (9,0);
    
    \draw (6,2.5) node[left] {C} to[I, i>^, name=i1] (6,0);
    \draw (6,2.5) to[short] (8.5,2.5) node [ground]{};
    
    \draw (4,0) to[R=$R_\mathrm{E}$] (4,-4);
    
    \draw (9, 0) to[open,v>, name=ua] (9,-4);
    
    \varrmore{ua}{$u_\mathrm{a}$};
    \iarrmore{ie}{$i_\mathrm{e}$};
    \iarrmore{ia}{$i_\mathrm{a}$};
    \iarrmore{i1}{$\beta \cdot i_\mathrm{b}$};
    \varrmore{ue}{$u_\mathrm{e}$}; 
    
\end{tikzpicture} }
            \end{minipage}
            \begin{minipage}[b]{0.3\textwidth}
                \raggedleft
                \scalebox{0.9}{\begin{tikzpicture}[scale= 0.45, font=\scriptsize, /tikz/circuitikz/bipoles/length=.8cm] 
    \draw (0,0) 
        to[sV, name=ue] (0,-4)
        to[short, -*] (4,-4) node[below] {B}
        to [short] (4,-3)
        to[R,v> ,i>, name=R1, l=$r_\mathrm{BE}$] (4,-1)
        to [short, i>, name =ib, -*] (4,0)
        node[above] {E};
    
    \draw (0,0) to[short ,i, name=ie] (4,0);
    
    \draw (4,-4) to[short, -*] (6,-4)
        to[short, -*] (8,-4)
        to[short, -o] (10,-4);
    
    \draw (6,-4) node[ground] {};
    
    \draw (8,-4) to[R=$R_\mathrm{C}$, v>,  -*] (8,0) node[above] {C}
        to[I, i>^, name=I1] (4,0);
    
    \draw (10,0) to[short, i>, name=ia, o-] (8,0);
    
    \draw (10, 0) to[open,v>, name=ua] (10,-4);
    
    \varrmore{ua}{$u_\mathrm{a}$};
    \iarrmore{ie}{$i_\mathrm{e}$};
    \iarrmore{ib}{$i_\mathrm{b}$};
    \iarrmore{ia}{$i_\mathrm{a}$};
    \iarrmore{I1}{$\beta \cdot i_\mathrm{B}$};
    \varrmore{ue}{$u_\mathrm{e}$};
    
\end{tikzpicture} }
                \end{minipage}
            %\label{fig:ErsatzschaltbilderEmitter-Kollektor-UndBasisschaltung}
            %\caption{\textbf{Ersatzschaltbilder der Grundschaltungen.} V.l.n.r Emitter-, Kollektor- und Basisschaltung von npn-Transistoren.} 
        \end{figure}}
\end{frame}

\begin{frame}
    \b{ \frametitle{Bipolartransistor -- Anwendung/ Grundschaltungen}
        \begin{table}[H]
            \centering
            \begin{tabular}{|c|c|c|c|}
                \hline
                $r_\mathrm{e}$ & mittel z.B. 1$\mathrm{k \Omega}$ & klein z.B. 50$\mathrm{\Omega}$ & groß z.B. 100$\mathrm{k \Omega}$ \\
                \hline
                $r_\mathrm{a}$ & mittel z.B. 10$\mathrm{k \Omega}$ & groß z.B. 100$\mathrm{k \Omega}$ & lein z.B. 50$\mathrm{\Omega}$ \\
                \hline
                $v_\mathrm{i}$ & groß z.B. 100 & $<$1 z.B. 0.9 &  groß z.B. 100 \\
                \hline
                $v_\mathrm{u}$ & groß z.B. 100 &  groß z.B. 100 & $<$1 z.B. 0.99 \\
                \hline
                $v_\mathrm{p}$ & sehr groß z.B. 1$\mathrm{k \Omega}$ &  groß z.B. 100 & groß z.B. 100 \\
                \hline
                $\varphi_\mathrm{u}$ & gegenphasig 180° & gleichphasig 0° & gleichphasig 0° \\
                \hline
            \end{tabular}
            \caption{Elektrische Eigenschaften von Grundschaltungen}
            %\label{tab:EigenschaftenGrundschaltungen}
        \end{table}
    }
\end{frame}


\begin{frame}
     \b{ \frametitle{Bipolartransistor -- Anwendung/ Grundschaltungen}
         \begin{figure}[H]
            \begin{minipage}[h]{0.48\textwidth}
            \centering
                \begin{tikzpicture} [scale=0.9, /tikz/circuitikz/bipoles/length=1.1cm]
    \draw (2,0) node[npn](Q1){};
    
    \draw (2,0) node[npn, tr circle] (Q1) {}
        (Q1.base) node[anchor=south east] {B}
        (Q1.collector) node[anchor=east] {C}
        (Q1.emitter) node[anchor=east] {E};
    
    \draw (Q1.B) to[R=$R_1$, -o] (-1.5,0);
    
    \draw (Q1.C) to [lamp] (2,2.5)
        node[vcc](VCC){$U_\mathrm{V}$};
    
    \draw (0.5,-0.25) to[open, v>, name=ube] (1.5,-1.25);

    \draw (2.5,-1) to[open,v<, name=uce] (2.5,1);
    
    \draw (Q1.E) to[short] (2,-1) node[ground]{};
    
    \varrmore[blue]{uce}{$U_\mathrm{CE}$};
    \varrmore[red]{ube}{$U_\mathrm{BE}$};    
\end{tikzpicture}
            \end{minipage}
            \begin{minipage}[h]{0.48\textwidth}
                \centering
                \includesvg[width= \textwidth]{Bilder/kap3/TransistorAlsSchalter/TransistorAlsSchalter2.svg}  
            \end{minipage}
            %\label{fig:TransistorAlsSchalter}
            %\caption{\textbf{Transistor als Schalter.} Links: Schaltbild mit Lampe als Verbraucher in der Kollektor-Emitter-Strecke. Rechts: 
            %Spannungsverläufe mit $U_\mathrm{CE}$ als Folge von $U_\mathrm{BE}$.} 
        \end{figure}
     }
\end{frame}

\begin{frame}
    \b{\frametitle{Bipolartransistor -- Anwendung/ Grundschaltungen}
        \textbf{Merke:} \\
        Im Vergleich zu einem elektromechanischen Schalter in Form eines Relais weisen Transistoren einen 
        deutlich kleineren Bauraum und geringeren Preis auf. Durch den kontaktlosen Aufbau tritt zudem kein 
        Kontaktprellen auf, was zu einer höheren Lebensdauer führt.
        }
\end{frame}

\begin{frame}
    \b{ \frametitle{Bipolartransistor -- Anwendung/ Grundschaltungen}
        \begin{figure}[H]
            \centering
            \includesvg[width= 0.7\textwidth]{Bilder/kap3/DarlingtonTransistor.svg}
            %\caption{\textbf{Darlington Transistor.} Schaltung und Schaltsymbol eines npn-Darlington-Transistor}  
            %\label{fig:DarlingtonTransistor}
        \end{figure}}
\end{frame}

\begin{frame}
    \b{ \frametitle{Feldeffekttransistor}
    \begin{figure}[H]
        \centering
        \scalebox{0.9}{\begin{tikzpicture}[]
    \tikzset{edge from parent/.style={draw,edge from parent path={(\tikzparentnode.south)-- +(0,-14pt)-| (\tikzchildnode)}},
    every tree node/.style={align=center,anchor=north}}
    
    \node {Transistortypen}
      [sibling distance=7cm]
      child {node {bipolar Transistoren}
        [sibling distance=2cm, level distance=2.5cm]
        child {node {\shortstack{NPN\\\begin{circuitikz}\draw (0,0) node[npn] {};\end{circuitikz}}}}
        child {node {\shortstack{PNP\\\begin{circuitikz}\draw (0,0) node[pnp, yscale=-1] {};\end{circuitikz}}}}
      }
      child {node {Feldeffekttransistoren}
        [sibling distance=6cm]
        child {node {\shortstack{Sperrschicht FET\\JFET}}
            [sibling distance=2cm, level distance=2.5cm]
            child {node {\shortstack{N-Kanal\\\begin{circuitikz}\draw (0,0) node[njfet] {};\end{circuitikz}}}}
            child {node {\shortstack{P-Kanal\\\begin{circuitikz}\draw (0,0) node[pjfet, yscale=-1] {};\end{circuitikz}}}}
        }
        child {node {MOSFET}
        [sibling distance=4cm]
            child {node {\shortstack{Anreicherung\\selbstsperrend}}
                [sibling distance=2cm, level distance=2.5cm]
                child {node {\shortstack{N-Kanal\\\begin{circuitikz}\draw (0,0) node[nigfete] {};\end{circuitikz}}}}
                child {node {\shortstack{P-Kanal\\\begin{circuitikz}\draw (0,0) node[pigfete, yscale=-1] {};\end{circuitikz}}}}
            }
            child {node {\shortstack{Verarmung\\selbstleitend}}
                [sibling distance=2cm, level distance=2.5cm]
                child {node {\shortstack{N-Kanal\\\begin{circuitikz}\draw (0,0) node[nigfetd] {};\end{circuitikz}}}}
                child {node {\shortstack{P-Kanal\\\begin{circuitikz}\draw (0,0) node[pigfetd, yscale=-1] {};\end{circuitikz}}}}
            }
        }
      };
    \end{tikzpicture}}
        %\caption{\textbf{Transistortypen.} Übersicht der gängigsten Transistortypen, sowohl bipolar Transistoren als auch unipolare 
        %(Feldeffekttransistoren). Bei FETs wird zwischen JFETs und MOSFETs (selbstsperrend und selbstleitend) unterschieden bzw. 
        %bei jedem Typ wird zusätzlich zwischen n- und p"~Kanal unterschieden.}  
        %\label{fig:Transistortypen}
    \end{figure}
    }
\end{frame}

\begin{frame}
    \b{ \frametitle{Feldeffekttransistor -- Elektrisches Verhalten}
        \begin{figure}[H]
            \centering
            \includesvg[width= 0.7\textwidth]{Bilder/kap4/Sperrbereichn-KanalMOSFET.svg}
            %\caption{\textbf{Sperrbereich n-Kanal MOSFET.} Querschnitt eines n-Kanal MOSFETs (selbstsperrend) ohne angelegte 
            %Spannungen}  
            %\label{fig:SperrbereichN-KanalMOSFET}
        \end{figure}

    }
\end{frame}

\begin{frame}
    \b{ \frametitle{Feldeffekttransistor -- Elektrisches Verhalten}
        \begin{figure}[H]
            \centering
            \includesvg[width= 0.7\textwidth]{Bilder/kap4/FolienMOSFETUDS.svg}
            %\caption{\textbf{Schrittweise Beschaltung}  
            %\label{fig:SchrittweiseBeschaltung1}
        \end{figure}
    }
\end{frame}

\begin{frame}
    \b{ \frametitle{Feldeffekttransistor -- Elektrisches Verhalten}
        \begin{figure}[H]
            \centering
            \includesvg[width= \textwidth]{Bilder/kap4/FolienMOSFETUGB.svg}
            %\caption{\textbf{Schrittweise Beschaltung}  
            %\label{fig:SchrittweiseBeschaltung1}
        \end{figure}
    }
\end{frame}

\begin{frame}
    \b{ \frametitle{Feldeffekttransistor -- Elektrisches Verhalten}
        \begin{figure}[H]
            \centering
            \includesvg[width= \textwidth]{Bilder/kap4/FolienMOSFETSteuerspannung.svg}
            %\caption{\textbf{Schrittweise Beschaltung}  
            %\label{fig:SchrittweiseBeschaltung1}
        \end{figure}
    }
\end{frame}

\begin{frame}
    \b{ \frametitle{Feldeffekttransistor -- Elektrisches Verhalten}
        \begin{figure}[H]
            \centering
            \includesvg[width= \textwidth]{Bilder/kap4/ohmscherBereichn-KanalMOSFET.svg}
            %\caption{\textbf{ohmscher Bereich n-Kanal MOSFET.} Querschnitt eines n-Kanal MOSFETs (selbstsperrend) ohne angelegte 
            %Spannungen}  
            %\label{fig:ohmscherBereichn-KanalMOSFET}
        \end{figure}
    }
\end{frame}

\begin{frame}
    \b{ \frametitle{Feldeffekttransistor -- Elektrisches Verhalten}
        \begin{figure}[H]
            \centering
            \includesvg[width=\textwidth]{Bilder/kap4/Ausgangskennlinienfeldn-KanalMOSFET.svg}
            %\caption{\textbf{Ausgangskennlinienfeld n-Kanal MOSFET.} Ausgangskennlinienfeld eines n-Kanal MOSFETs 
            %(selbstsperrend) mit dem linearen- und Sättigungsbereich. Drainstrom als Folge der Drain-Source-Spannung bei 
            %verschiedenen Gate-Source-Spannungen.}  
            %\label{fig:Ausgangskennlinienfeldn-KanalMOSFET}
        \end{figure}
    }
\end{frame}

\begin{frame}
    \b{ \frametitle{Feldeffekttransistor -- Elektrisches Verhalten}
        \begin{figure}[H]
            \centering
            \includesvg[width= 0.7\textwidth]{Bilder/kap4/SaettigungsbereichN-KanalMOSFET.svg}
            %\caption{\textbf{Sättigungsbereich n-Kanal MOSFET.} Querschnitt eines n-Kanal MOSFETs (selbstsperrend) mit angelegter 
            %Spannung $U_{GS}$ und $U_{DS}$}  
            %\label{fig:SaettigungsbereichN-KanalMOSFET}
        \end{figure}
    }
\end{frame}

\begin{frame}
    \b{\frametitle{Feldeffekttransistor -- Elektrisches Verhalten}
        \textbf{Merke:}
        \begin{itemize}
            \item MOSFETs besitzen drei Anschlüsse, Gate(G), Source(S) und Drain (D).
            \item Das elektrische Verhalten wird leistungslos gesteuert.
            \item Durch $U_\mathrm{GS}$ erfolgt eine Inversion der Ladungsträger unterhalb des Gates.
        \end{itemize}
        }
\end{frame}

\begin{frame}
    \b{ \frametitle{Feldeffekttransistor -- Elektrisches Verhalten}
        \begin{table}[H]
            \begin{figure}[H]
                \begin{minipage}[c]{0.1\textwidth}
                    \centering
                    Type
                \end{minipage}
                \begin{minipage}[c]{0.2\textwidth}
                    \centering
                    Symbol
                \end{minipage}
                \begin{minipage}[c]{0.33\textwidth}
                    \centering
                    Eingangskennlinie
                \end{minipage}
                \begin{minipage}[c]{0.33\textwidth}
                    \centering
                    Ausgangskennlinienfeld
                \end{minipage}
            \end{figure}
            \begin{figure}[H]
                \begin{minipage}[c]{0.1\textwidth}
                \centering
                n-JFET
                \end{minipage}
                \begin{minipage}[c]{0.2\textwidth}
                    \centering
                    \begin{circuitikz} 
    \draw(0,0) node[njfet] (njfet) {}
    (njfet.G) node[anchor=east] {G}
    (njfet.D) node[anchor=south] {D}
    (njfet.S) node[anchor=north] {S};
\end{circuitikz}
                    \end{minipage}
                \begin{minipage}[c]{0.33\textwidth}
                \centering
                \includesvg[width=0.8\textwidth, pretex=\small]{Bilder/kap4/UebersichtFeldeffekttransistoren/EingangskennlinieN-Jfet.svg}  
                \end{minipage}
                \begin{minipage}[c]{0.33\textwidth}
                \centering
                \includesvg[width=0.8\textwidth, pretex=\small]{Bilder/kap4/UebersichtFeldeffekttransistoren/AusgangskennlinieN-Jfet.svg}
                \end{minipage}
            \end{figure}
            \begin{figure}[H]
                \begin{minipage}[c]{0.1\textwidth}
                \centering
                p-JFET
                \end{minipage}
                \begin{minipage}[c]{0.2\textwidth}
                \centering
                \begin{circuitikz} 
    \draw(0,0) node[pjfet, yscale=-1] (pjfet) {}
    (pjfet.G) node[anchor=east] {G}
    (pjfet.D) node[anchor=south] {D}
    (pjfet.S) node[anchor=north] {S};
\end{circuitikz}
                \end{minipage}
                \begin{minipage}[c]{0.33\textwidth}
                \centering
                \includesvg[width=0.8\textwidth, pretex=\small]{Bilder/kap4/UebersichtFeldeffekttransistoren/EingangskennlinieP-Jfet.svg}  
                \end{minipage}
                \begin{minipage}[c]{0.33\textwidth}
                \centering
                \includesvg[width=0.8\textwidth, pretex=\small]{Bilder/kap4/UebersichtFeldeffekttransistoren/AusgangskennlinieP-Jfet.svg}
                \end{minipage} 
            \end{figure}
        \end{table}
    }
\end{frame}

\begin{frame}
    \b{ \frametitle{Feldeffekttransistor -- Elektrisches Verhalten}
        \begin{table}[H]
            \begin{figure}[H]
                \begin{minipage}[c]{0.1\textwidth}
                \centering
                n-Mosfet, 
                selbstsperrend
                \end{minipage}
                \begin{minipage}[c]{0.2\textwidth}
                \centering
                \begin{circuitikz} 
    \draw(0,0) node[nigfete] (nigfete) {}
    (nigfete.G) node[anchor=east] {G}
    (nigfete.D) node[anchor=south] {D}
    (nigfete.S) node[anchor=north] {S};
\end{circuitikz}
                \end{minipage}
                \begin{minipage}[c]{0.33\textwidth}
                \centering
                \includesvg[width=0.8\textwidth, pretex=\small]{Bilder/kap4/UebersichtFeldeffekttransistoren/EingangskennlinieN-MosfetSelbstsperrend.svg}  
                \end{minipage}
                \begin{minipage}[c]{0.33\textwidth}
                \centering
                \includesvg[width=0.8\textwidth, pretex=\small]{Bilder/kap4/UebersichtFeldeffekttransistoren/AusgangskennlinieN-MostfetSelbstsperrend.svg}
                \end{minipage}
            \end{figure}
            \begin{figure}[H]
                \begin{minipage}[c]{0.1\textwidth}
                \centering
                p-Mosfet, 
                selbstsperrend
                \end{minipage}
                \begin{minipage}[c]{0.2\textwidth}
                \centering
                \begin{circuitikz} 
    \draw(0,0) node[pigfete, yscale=-1] (pigfete) {}
    (pigfete.G) node[anchor=east] {G}
    (pigfete.D) node[anchor=south] {D}
    (pigfete.S) node[anchor=north] {S};
\end{circuitikz}
                \end{minipage}
                \begin{minipage}[c]{0.33\textwidth}
                \centering
                \includesvg[width=0.8\textwidth, pretex=\small]{Bilder/kap4/UebersichtFeldeffekttransistoren/EingangskennlinieP-MosfetSelbstsperrend.svg}  
                \end{minipage}
                \begin{minipage}[c]{0.33\textwidth}
                \centering
                \includesvg[width=0.8\textwidth, pretex=\small]{Bilder/kap4/UebersichtFeldeffekttransistoren/AusgangskennlinieP-MosfetSelbstsperrend.svg}
                \end{minipage}
            \end{figure}
    \end{table}
    }
\end{frame}


\begin{frame}
    \b{ \frametitle{Feldeffekttransistor -- Elektrisches Verhalten}
        \begin{table}[H]
            \begin{figure}[H]
                \begin{minipage}[c]{0.1\textwidth}
                \centering
                n-Mosfet, \newline
                selbstleitend
                \end{minipage}
                \begin{minipage}[c]{0.2\textwidth}
                \centering
                \begin{circuitikz} 
    \draw(0,0) node[nigfetd] (nigfetd) {}
    (nigfetd.G) node[anchor=east] {G}
    (nigfetd.D) node[anchor=south] {D}
    (nigfetd.S) node[anchor=north] {S};
\end{circuitikz}
                \end{minipage}
                \begin{minipage}[c]{0.33\textwidth}
                \centering
                \includesvg[width=0.8\textwidth, pretex=\small]{Bilder/kap4/UebersichtFeldeffekttransistoren/EingangskennlinieN-MosfetSelbstleitend.svg}  
                \end{minipage}
                \begin{minipage}[c]{0.33\textwidth}
                \centering
                \includesvg[width=0.8\textwidth, pretex=\small]{Bilder/kap4/UebersichtFeldeffekttransistoren/AusgangskennlinieN-MosfetSelbstleitend.svg}
                \end{minipage}
            \end{figure}
            \begin{figure}[H]
                \begin{minipage}[c]{0.1\textwidth}
                \centering
                p-Mosfet, \newline
                selbstleitend
                \end{minipage}
                \begin{minipage}[c]{0.2\textwidth}
                \centering
                \begin{circuitikz} 
    \draw(0,0) node[pigfetd, yscale=-1] (pigfetd) {}
    (pigfetd.G) node[anchor=east] {G}
    (pigfetd.D) node[anchor=south] {D}
    (pigfetd.S) node[anchor=north] {S};
\end{circuitikz}
                \end{minipage}
                \begin{minipage}[c]{0.33\textwidth}
                \centering
                \includesvg[width=0.8\textwidth, pretex=\small]{Bilder/kap4/UebersichtFeldeffekttransistoren/EingangskennlinieP-MosfetSelbstleitend.svg}  
                \end{minipage}
                \begin{minipage}[c]{0.33\textwidth}
                \centering
                \includesvg[width=0.8\textwidth, pretex=\small]{Bilder/kap4/UebersichtFeldeffekttransistoren/AusgangskennlinieP-MosfetSelbstleitend.svg}
                \end{minipage}
            \end{figure}
    \end{table}
    }
\end{frame}

\begin{frame}
    \b{\frametitle{Feldeffekttransistor -- Elektrisches Verhalten}
        \textbf{Merke:}
        \begin{itemize}
            \item Die Dotierung des Kanals gibt die notwendigen Vorzeichen der Ströme und Spannungen am Transistor vor.
            \item Die Eingangskennlinien können durch die Lage der Schwellenspannung $U_\mathrm{th}$ unterschieden werden.
        \end{itemize}
        }
\end{frame}

\begin{frame}
    \b{ \frametitle{Feldeffekttransistor -- Aufbau}
        \begin{figure}[H]
            \centering
            \includesvg[width= \textwidth]{Bilder/kap4/QuerschnittCMOSStruktur.svg}
            %\caption{\textbf{Querschnitt CMOS Struktur.} Integration eines PMOS und NMOS in gemeinsamen p-dotierten Silizium 
            %Substrat.}  
            %\label{fig:QuerschnittCMOSStruktur}
        \end{figure}
    }
\end{frame}

\begin{frame}
    \b{ \frametitle{Feldeffekttransistor -- Aufbau}
        \begin{figure}[H]
            \centering
            \includesvg[width= \textwidth]{Bilder/kap4/QuerschnittN-KanalJFET.svg}
            %\caption{\textbf{Querschnitt n-Kanal JFET.} Allgemeiner Aufbau eines n-Kanal JFETs mit dotierten Gebieten und 
            %notwendigen Elektroden.}  
            %\label{fig:QuerschnittN-KanalJFET}
        \end{figure}
    }
\end{frame}

\begin{frame}
    \b{\frametitle{Feldeffekttransistor -- Aufbau}
        \textbf{Merke:}
        \begin{itemize}
            \item CMOS-Technik kombiniert komplementäre MOSFETs in einem Bauteil.
            \item JFETs weisen den einfachsten Aufbau von FETs auf.
        \end{itemize}
        }
\end{frame}

\begin{frame}
    \b{ \frametitle{Feldeffekttransistor -- Anwendungen/ Grundschaltungen}
        \begin{figure}[H]
            \centering
            \begin{minipage}[c]{.48\textwidth}
                \centering
                \begin{tikzpicture}
    \draw (0.5,-1.25)                                 
        to [V,v<,name=ugs, *-] (0.5,0.5) 
        to [short] (1,0.5)
        node (mosfet1) [nigfete, anchor=G, tr circle] {};
    
    \draw (mosfet1.D)
        to [short, -*] (2,2)
        to [R, name=R, l=$R$, v<] ++(-3.5,0)
        to [V,v>,name=uv](-1.5,-1.25)
        to [short, -o](3,-1.25)
        (2,2) to[short, -o] (3,2);    

    \draw (mosfet1.S)
        to [short, -*] (2,-1.25);

    \draw (3,-1.25) to[open,v<, name=uds] (3,2);

    \varrmore{uds}{$U_\mathrm{DS}$};
    \varrmore{ugs}{$U_\mathrm{GS}$};
    \varrmore{uv}{$U_\mathrm{V}$};

    \node at (mosfet1.G)[anchor=south, xshift=-4pt]{$\mathrm{G}$};
    \node at (mosfet1.D)[anchor=east, yshift=+1pt]{$\mathrm{D}$};
    \node at (mosfet1.S)[anchor=east, yshift=-1pt]{$\mathrm{S}$};
\end{tikzpicture}
            \end{minipage}
            \begin{minipage}[c]{.48\textwidth}
                \centering
                \begin{circuitikz}
    \draw (2.5,0) to[short, o-*] (1.5,0)
        to[R=$R$] (-1,0)
        to [V, v>, name=uv] (-1,-3.25)
        to [short, -*] (1.5,-3.25)
        to[short, -o] (2.5,-3.25);

    \draw (1.5,0) to[R] (1.5,-3.25);  
    \draw [->, >=Stealth] (2,-1.75) -- (1,-1.25); %Hab diese Zeichnung nicht anders hinbekommen

    \draw (2.5,-3.25) to[open,v<, name=uds] (2.5,0);

    \varrmore{uds}{$U_\mathrm{DS}$};
    \varrmore{uv}{$U_\mathrm{V}$};
\end{circuitikz}

            \end{minipage}
        
            %\caption{\textbf{MOSFET als steuerbarer Widerstand.} Links: Spannungsteiler mit MOSFET als steuerbares Element. Rechts: 
            %Spannungsteiler mit Repräsentation des MOSFETs als steuerbaren Widerstand. }  
            %\label{fig:MOSFETAlsSteuerbarerWiderstand}
        \end{figure}
    }
\end{frame}

\begin{frame}
    \b{\frametitle{Feldeffekttransistor -- Anwendungen/ Grundschaltungen}
        \textbf{Merke:}
        Der ohmsche Widerstand eines MOSFETs kann durch $U_\mathrm{GS}$ gesteuert werden, 
        der Durchgangswiderstand ist allerdings gering und der Wertebereich sehr schmal.  
        }
\end{frame}

\begin{frame}
    \b{ \frametitle{Feldeffekttransistor -- Anwendungen/ Grundschaltungen}
        \begin{figure}[H]
            \centering
            \begin{minipage}[c]{.48\textwidth}
                \centering
                \begin{tikzpicture} 
    \draw (0,-0.5) node[nigfete, tr circle] (Q1) {}
        (Q1.gate) node[anchor=south east] {G}
        (Q1.source) node[anchor=east] {S}
        (Q1.drain) node[anchor=east] {D};
    
    \draw (Q1.D) 
        to [short] (0,1)
        to[R=$R_\mathrm{L}$] (0,3)
        node[vcc](VCC){$U_\mathrm{V}$};
    
    \draw (0,1) to[short, *-o] (2,1);
    
    \draw (Q1.G) 
        to[R=$R_\mathrm{in}$, -o] ++(-3,0);
    
    \draw (Q1.S) 
        to[short, -*] (0,-2) node[ground]{}
        to[short, -o] (2,-2);
    
    \draw (2,1) to[open,v>, name=ua] (2,-2);
    \varrmore{ua}{$U_\mathrm{a}$};
    
\end{tikzpicture}
            \end{minipage}
            \begin{minipage}[c]{.48\textwidth}
                \centering
                \begin{tikzpicture} 

    \draw (0,3) node[vcc](VCC){$U_\mathrm{V}$}
        to[R, l_=$R_\mathrm{L}$] (0,1)
        to[normal open switch, -*, l_={}](0,-2) node[ground]{}
        to[short, -o] (2,-2);
    \draw (0,1) to[short, *-o] (2,1);
    \draw (2,1) to[open,v>, name=ua] (2,-2);
    
    \varrmore{ua}{$U_\mathrm{a}$};
    
\end{tikzpicture}
            \end{minipage}
        
            %\caption{\textbf{MOSFET als Schalter.} Links: Schaltung mit MOSFET als steuerbaren Schalter. 
            %Rechts: Ersatzschaltbild mit der Repräsentation des MOSFETs als Schalter.}  
            %\label{fig:MOSFETAlsSchalter}
        \end{figure}
    }
\end{frame}

\begin{frame}
    \b{\frametitle{Feldeffekttransistor -- Anwendungen/ Grundschaltungen}
        \textbf{Merke:}
        Der große Vorteil von MOSFETs in
        diesem Szenario ist ihre hohe Schaltgeschwindigkeit, der niedrige Widerstand im leitenden Zustand
        sowie der hohe Eingangswiderstand am Gate.
        }
\end{frame}

\begin{frame}
    \b{ \frametitle{Feldeffekttransistor -- Anwendungen/ Grundschaltungen}
        \begin{figure}[H]
            \begin{minipage}[h]{0.48\textwidth}
                \centering
                \begin{tikzpicture}
    \draw (-1,1)  to [short, o-*] (2,1) to[short, -o] (4,1) node[right]{$5\,V$};
    \draw (-1,-4)  to [short, o-*] (1.98,-4) to[short, -o] (4,-4) node[right]{$0\,V$};
    
    % Input node
    \draw (-1,-1.5) node[left] {$E$} 
        to[short, o-*] (0,-1.5)
        to[short] (0,0);

    \draw (0,-1.5) 
        to[short] (0,-3);

    % PMOS Transistor
    \draw (0,0) -- (1,0) node (pmos) [pigfete, anchor=G, tr circle] {}
        (pmos.gate) node[anchor=north east] {G}
        (pmos.source) node[anchor=east] {S}
        (pmos.drain) node[anchor=east] {D};
    \draw (pmos.S) -- ++(0,0.5);
    
    % NMOS Transistor
    \draw (0,-3) -- (1,-3) node (nmos) [nigfete, anchor=G, tr circle] {}
        (nmos.gate) node[anchor=south east] {G}
        (nmos.source) node[anchor=east] {S}
        (nmos.drain) node[anchor=east] {D};
    \draw (nmos.S) -- ++(0,-0.5) node[ground] {};
    
    % Connecting PMOS and NMOS drains to output
    \draw (pmos.D) -- (nmos.D);
    
    \draw (1.98,-1.5) to[short, *-o] (4,-1.5)node[right]{$Q$};
    \draw (-1,-1.5) to[open, v, name=UGS1] (-1,-4); 
    \varrmore {UGS1}{$U_\mathrm{GS1}$};
    \draw (-1,1) to[open, v<, name=UGS2]  (-1,-1.5); 
    \varrmore {UGS2}{$U_\mathrm{GS2}$};
    
\end{tikzpicture}
            \end{minipage}
            \begin{minipage}[c]{0.48\textwidth}
                \centering
                \includesvg[width=0.9\textwidth, pretex=\small]{Bilder/kap4/MosfetAlsInverter/MosfetAlsInverter3.svg}  
            \end{minipage}
            %\caption{\textbf{MOSFET als Inverter.} Links: Schaltkreis eines CMOS Inverter. Rechts: Spannungskennlinie mit 
            %reales Ausgangssignal (blau) und ideales Ausgangssignal (grün).} 
            %\label{fig:MOSFETAlsInverter}
        \end{figure}
    }
\end{frame}









