\subsection{Ausgangsspannung eines Komparators ermitteln\label{Aufg9}}
% Alt: Aufgabe 9
\Aufgabe{
    Geben ist der dargestellte Komparator. Die Spannung $U_\mathrm{1}= \mathrm{8,25\,V}$ sei konstant. Wie groß ist die 
    Ausgangsspannung für $U_2 = \mathrm{0\,V; 4\,V; 8\,V; 8,2\,V; 8,249\,V; 8,251\,V; 8,3\,V; 10\,V}$?
   \begin{figure}[H]
    \centering
    % Linkes Bild
    \begin{subfigure}[b]{0.45\textwidth}
        \centering
        \begin{tikzpicture}
\ctikzset{tripoles/en amp/input height=-0.45}
            \draw (0,0)node[en amp](E){};
 (E.+)
 (E.-)
 (E.out) 
 (E.down)
 (E.up)
 -- (E.up);

\draw (E.up) to[short, -o] (0,2)node[above] {$15 \, \mathrm{V}$};
\draw (E.down) to[short, -o] (0,-2)node[below] {$-15 \, \mathrm{V}$};

\draw (E.+) to[short, -o] (-4,0.5);
\draw (E.-) to[short, -o] (-2.5,-0.5);
\draw (E.out) to[short, -o] (2.5,0);
\draw (0.65,-1.117) -- (0.65,-1.617) -- (1.5,-1.617) to[short, -*] (1.5,-3) node[ground]{};
\draw (-4,-3) to[short, o-o] (2.5,-3);

\draw (-4,0.5) to[open, v, name=u1] (-4,-3);
\draw (-2.5,0.5) to[open,v,name=ud] (-2.5,-0.5);
\draw (-2.5,-0.5) to[open,v,name=u2] (-2.5,-3);
\draw (2.5,0) to[open,v,name=ua] (2.5,-3);


\varrmore{u1}{$U_\mathrm{1}$};
\varrmore{u2}{$U_\mathrm{2}$};
\varrmore{ud}{$U_\mathrm{D}$};
\varrmore{ua}{$U_\mathrm{A}$};
\end{tikzpicture}
    \end{subfigure}
    % Rechtes Bild
    %\begin{subfigure}[b]{0.45\textwidth}
        %\centering
        %\includesvg[width=0.8\textwidth]{Bilder/Aufgaben/FigAusgangsspannungKomparator.svg}
    %\end{subfigure}
    \label{fig:FigAusgangsspannungKomparator}
\end{figure}
%\speech{
%    Aufgabe 9: Komparatorschaltung  
%    
%    Gegeben ist eine Komparatorschaltung mit einem idealen Operationsverstärker.  
%    Die Versorgungsspannungen betragen plus 15 Volt und minus 15 Volt.  
%    Der Komparator vergleicht zwei Eingangsspannungen, U1 und U2.  
%    
%    Die Spannung U1 ist konstant mit einem Wert von 8,25 Volt.  
%    Die Spannung U2 variiert zwischen verschiedenen Werten.  
%    Gesucht ist die Ausgangsspannung U A für verschiedene Werte von U2.  
%    
%    Die Schaltung besteht aus einem Operationsverstärker,  
%    dessen nichtinvertierender Eingang mit der konstanten Spannung U1 verbunden ist.  
%    Die variable Eingangsspannung U2 wird an den invertierenden Eingang angelegt.  
%    Der Ausgang U A gibt entweder die obere Versorgungsspannung von plus 15 Volt  
%    oder die untere Versorgungsspannung von minus 15 Volt aus, abhängig von der Vergleichsspannung.  
%    
%    Rechts neben der Schaltung ist ein Diagramm abgebildet.  
%    Es zeigt die Ausgangsspannung U A in Abhängigkeit von der Differenzspannung U D.  
%    Die Kennlinie weist einen abrupten Übergang in der Nähe von U D gleich null auf.  
%    Dies bedeutet, dass der Komparator bei einem bestimmten Schwellenwert der Eingangsspannung schaltet.  
%    
%    Die Aufgabe besteht darin, zu bestimmen, wie sich U A verhält,  
%    wenn U2 unterschiedliche Werte zwischen null und zehn Volt annimmt.  
%    }
    

}

\Loesung{
	Ausnutzung der Maschenregel besagt: 
    \begin{align*}
        -U_1 + U_2 + U_D &= 0 \\
        U_\mathrm{D} &= U_1 - U_2
    \end{align*}
    \begin{itemize}
		\item  $U_2 = 0\,V$  
		\begin{align*}
            U_\mathrm{D} &= \mathrm{ 8,25\,V - 0\,V = 8,25\,V}\\
            U_\mathrm{A} &= \mathrm{13,5\,V} 
		\end{align*}
		\item  $U_2 = 4\,V$ 
		\begin{align*}
            U_\mathrm{D} &= \mathrm{ 8,25\,V - 4\,V = 4,25\,V}\\
            U_\mathrm{A} &= \mathrm{13,5\,V} 
		\end{align*}
		\item $U_2 = 8\,V$  
		\begin{align*}
            U_\mathrm{D} &= \mathrm{ 8,25\,V - 8\,V = 0,25\,V}\\
            U_\mathrm{A} &= \mathrm{13,5\,V} 
		\end{align*} 
        \item $U_2 = 8,2\,V$  
		\begin{align*}
            U_\mathrm{D} &= \mathrm{ 8,25\,V - 8,2\,V = 0,05\,V}\\
            U_\mathrm{A} &= \mathrm{8\,V} 
		\end{align*}
        \item $U_2 = 8,249\,V$  
		\begin{align*}
            U_\mathrm{D} &= \mathrm{ 8,25\,V - 8,249\,V = 1\,mV}\\
            U_\mathrm{A} &= \mathrm{4\,V} 
		\end{align*}
        \item $U_2 = 8,251\,V$ 
		\begin{align*}
            U_\mathrm{D} &= \mathrm{ 8,25\,V - 8,251\,V = -1\,mV}\\
            U_\mathrm{A} &= \mathrm{-4\,V} 
		\end{align*}
        \item $U_2 = 8,3\,V$  
		\begin{align*}
            U_\mathrm{D} &= \mathrm{ 8,25\,V - 8,3\,V = -0,05\,V}\\
            U_\mathrm{A} &= \mathrm{-8\,V} 
		\end{align*}
        \item $U_2 = 10\,V$  
		\begin{align*}
            U_\mathrm{D} &= \mathrm{ 8,25\,V - 10\,V = -1,75\,V}\\
            U_\mathrm{A} &= \mathrm{-13,5\,V} 
		\end{align*}
    \end{itemize}
    Siehe: Abschnitt \ref{fig: Verschaltung Verstaerker} und Tabelle \ref{tab:Grundschaltungen}
 %   \speech{
 %       Analyse eines Komparators anhand der Maschenregel  
 %       
 %       Ein Komparator vergleicht zwei Eingangsspannungen U1 und U2 und erzeugt eine Ausgangsspannung U A entsprechend der Differenzspannung U D.  
 %       
 %       Die Maschenregel ergibt die Gleichung:  
 %       U D = U1 - U2  
 %       
 %       Für verschiedene Werte von U2 werden die Differenzspannung U D und die Ausgangsspannung U A berechnet:  
 %       
 %       1. Falls U2 gleich 0 Volt:  
 %          U D = 8,25 Volt - 0 Volt = 8,25 Volt  
 %          U A = 13,5 Volt  
 %       
 %       2. Falls U2 gleich 4 Volt:  
 %          U D = 8,25 Volt - 4 Volt = 4,25 Volt  
 %          U A = 13,5 Volt  
 %       
 %       3. Falls U2 gleich 8 Volt:  
 %          U D = 8,25 Volt - 8 Volt = 0,25 Volt  
 %          U A = 13,5 Volt  
 %       
 %       4. Falls U2 gleich 8,2 Volt:  
 %          U D = 8,25 Volt - 8,2 Volt = 0,05 Volt  
 %          U A = 8 Volt  
 %       
 %       5. Falls U2 gleich 8,249 Volt:  
 %          U D = 8,25 Volt - 8,249 Volt = 1 Millivolt  
 %          U A = 4 Volt  
 %       
 %       6. Falls U2 gleich 8,251 Volt:  
 %          U D = 8,25 Volt - 8,251 Volt = -1 Millivolt  
 %          U A = -4 Volt  
 %       
 %       7. Falls U2 gleich 8,3 Volt:  
 %          U D = 8,25 Volt - 8,3 Volt = -0,05 Volt  
 %          U A = -8 Volt  
 %       
 %       8. Falls U2 gleich 10 Volt:  
 %          U D = 8,25 Volt - 10 Volt = -1,75 Volt  
 %          U A = -13,5 Volt  
 %       
 %       Der Komparator zeigt ein typisches Verhalten:  
 %       Solange U2 kleiner als 8,25 Volt ist, bleibt U A positiv.  
 %       Überschreitet U2 den Wert von 8,25 Volt, kippt die Ausgangsspannung auf negative Werte.  
%        }
        
}