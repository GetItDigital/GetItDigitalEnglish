%hier jede Aufgabe als eigene Datei mit dem Input-Befehl einfügen
%\subsection{Gleichstrommaschine 1\label{AufgGM1}}
\Aufgabe{
    Eine fremderregte Gleichstrommaschine hat die folgenden Angaben auf ihrem Typenschild:
    \begin{itemize}
        \item Nennleistung: $41,8\,\text{kW}$
        \item Nenndrehzahl: $1900\,\frac{1}{\text{min}}$
        \item Ankernennspannung: $440\,\text{V}$
        \item Ankernennstrom: $100\,\text{A}$
        \item Erregung: $240\,\text{V}$
        \item Erregerstrom: $10\,\text{A}$
        \item Leerlaufdrehzahl: $2000\,\frac{1}{\text{min}}$
    \end{itemize}

    Hinweis: Die Nennleistung auf einem Motortypenschild bezeichnet immer die abgegebene mechanische Leistung im Nennpunkt.

    \begin{itemize}
        \item[a)]
        Wie groß ist das Drehmoment der Maschine im Nennpunkt?
        \item[b)]
        Berechnen Sie den Wirkungsgrad der Maschine im Nennpunkt.
        \item[c)]
        Berechnen Sie das Produkt aus Erregerfluss und der Ankerkonstanten $K\cdot \Phi$
        \item[d)]
        Wie groß ist der Ankerwiderstand $R_A$?
    \end{itemize}
}

\Loesung{
	\begin{itemize}

		\item[a)]
		Aus der Nennleistung und der Nenndrehzahl ergibt sich direkt das Nenndrehmoment:
		\begin{align*}
		    P_\text{mech} &= M\cdot \omega\\
		    M &= \frac{P_\text{mech}}{\omega} = \frac{41,8\,\text{kW}}{2\pi\cdot 1900\,\frac{1}{\text{min}} \cdot \frac{1\,\text{min}}{60\,\text{s}}} = 210,08\,\text{Nm}
		\end{align*}
		\item[b)]
		Im Motorbetrieb ist der Wirkungsgrad der Quotient aus mechanischer bezogen auf die elektrische Leistung (siehe Gleichung \ref{GlLeistung}). Hierbei muss sowohl die Ankerleistung, als auch die Erregerleistung berücksichtigt werden.
		\begin{eqa}
			\eta &= \frac{P_\text{mech}}{P_\text{elektr}} = \frac{41,8\,\text{kW}}{440\,\text{V}\cdot 100\,\text{A} + 240\,\text{V}\cdot 10\,\text{A}} = 90,08\,\%\nonumber
		\end{eqa}

		\item[c)]
		Nach Gleichung \ref{GlInduzierteSpannungGM} wird die induzierte Spannung berechnet. Im Leerlauf muss die induzierte Spannung der Ankerspannung entsprechen. Gleiches geht aus Gleichung \ref{GlfremderregteGM1} hervor, da im Leerlauf auch das Drehmoment Null sein muss.
		\begin{eqa}
			U_q &= K \cdot \Phi \cdot \omega\tag{\ref{GlInduzierteSpannungGM}}\\
			K\cdot \Phi &= \frac{U_q}{\omega} = \frac{440\,\text{V}}{2\pi\cdot 2000\,\frac{1}{\text{min}} \cdot \frac{1\,\text{min}}{60\,\text{s}}} = 2,1\,\text{Vs}\nonumber
		\end{eqa}
		\item[d)]
		Hier sind zwei Ansätze möglich. Es kann Gleichung \ref{GlfremderregteGM1} auf den Ankerwiderstand umgeformt werden:
		\begin{eqa}
			n &= \frac{U_A}{2\pi K\cdot \Phi} - \frac{R_A\cdot M_i}{2\pi(K\cdot \Phi)^2}\tag{\ref{GlfremderregteGM1}}\\
			R_A &= \left(\frac{U_A}{2\pi K\cdot \Phi} - n\right) \cdot \frac{2\pi(K\cdot \Phi)^2}{M_i}\nonumber\\
			&= \left(\frac{440\,\text{V}}{2\pi\cdot 2,1\,\text{Vs}} - \frac{1900}{60\,\text{s}}\right)\cdot \frac{2\pi\cdot (2,1\,\text{Vs})^2}{210,08\,\text{Nm}} = 220\,\text{m}\Omega\nonumber
		\end{eqa}
		Die andere Berechnungsvariante führt über das Ersatzschaltbild in Abbildung \ref{AbbGMFremderregt}. Im Nennbetrieb können wir die Quellenspannung $U_q$ berechnen. Diese muss kleiner als die Quellenspannung im Leerlauf sein.
		\begin{eqa}
			U_q &= K \cdot \Phi \cdot \omega \tag{\ref{GlInduzierteSpannungGM}}\\
			U_{q,n}&=  2,1\,\text{Vs} \cdot 2\pi\cdot 1900\,\frac{1}{\text{min}} \cdot \frac{1\,\text{min}}{60\,\text{s}} = 418\,\text{V}\nonumber
		\end{eqa}
		Der Spannungsabfall am Widerstand $R_A$ muss durch dem Maschenumlauf die Differenzspannung zwischen Ankerspannung und induzierter Spannung betragen. Da der Ankerstrom im Nennbetrieb bekannt ist, kann der Widerstand über das Ohmsche Gesetz berechnet werden.
		\begin{equation*}
		    R_A = \frac{U_A - U_q}{I_A} = \frac{440\,\text{V} - 418\,\text{V}}{100\,\text{A}} = 220\,\text{m}\Omega
		\end{equation*}
	\end{itemize}
}
%\input{Aufgabe2}
%....
%Das ist nur Beispielsinhalt, bitte auskommentieren!
%\subsection{Gleichstrommaschine 1\label{AufgGM1}}
\Aufgabe{
    Eine fremderregte Gleichstrommaschine hat die folgenden Angaben auf ihrem Typenschild:
    \begin{itemize}
        \item Nennleistung: $41,8\,\text{kW}$
        \item Nenndrehzahl: $1900\,\frac{1}{\text{min}}$
        \item Ankernennspannung: $440\,\text{V}$
        \item Ankernennstrom: $100\,\text{A}$
        \item Erregung: $240\,\text{V}$
        \item Erregerstrom: $10\,\text{A}$
        \item Leerlaufdrehzahl: $2000\,\frac{1}{\text{min}}$
    \end{itemize}

    Hinweis: Die Nennleistung auf einem Motortypenschild bezeichnet immer die abgegebene mechanische Leistung im Nennpunkt.

    \begin{itemize}
        \item[\bf a)]
        Wie groß ist das Drehmoment der Maschine im Nennpunkt?
        \item[\bf b)]
        Berechnen Sie den Wirkungsgrad der Maschine im Nennpunkt.	
        \item[\bf c)]
        Berechnen Sie das Produkt aus Erregerfluss und der Ankerkonstanten $K\cdot \Phi$
        \item[\bf d)]
        Wie groß ist der Ankerwiderstand $R_A$?
    \end{itemize}
}

\Loesung{
	\begin{itemize}
		
		\item[\bf a)]
		Aus der Nennleistung und der Nenndrehzahl ergibt sich direkt das Nenndrehmoment:
		\begin{align*}
		    P_\text{mech} &= M\cdot \omega\\
		    M &= \frac{P_\text{mech}}{\omega} = \frac{41,8\,\text{kW}}{2\pi\cdot 1900\,\frac{1}{\text{min}} \cdot \frac{1\,\text{min}}{60\,\text{s}}} = 210,08\,\text{Nm}
		\end{align*}
		\item[\bf b)]
		Im Motorbetrieb ist der Wirkungsgrad der Quotient aus mechanischer bezogen auf die elektrische Leistung (siehe Gleichung \ref{GlLeistung}). Hierbei muss sowohl die Ankerleistung, als auch die Erregerleistung berücksichtigt werden.
		\begin{eqa}
			\eta &= \frac{P_\text{mech}}{P_\text{elektr}} = \frac{41,8\,\text{kW}}{440\,\text{V}\cdot 100\,\text{A} + 240\,\text{V}\cdot 10\,\text{A}} = 90,08\,\%\nonumber
		\end{eqa}
		
		\item[\bf c)]
		Nach Gleichung \ref{GlInduzierteSpannungGM} wird die induzierte Spannung berechnet. Im Leerlauf muss die induzierte Spannung der Ankerspannung entsprechen. Gleiches geht aus Gleichung \ref{GlfremderregteGM1} hervor, da im Leerlauf auch das Drehmoment Null sein muss.
		\begin{eqa}
			U_q &= K \cdot \Phi \cdot \omega\tag{\ref{GlInduzierteSpannungGM}}\\
			K\cdot \Phi &= \frac{U_q}{\omega} = \frac{440\,\text{V}}{2\pi\cdot 2000\,\frac{1}{\text{min}} \cdot \frac{1\,\text{min}}{60\,\text{s}}} = 2,1\,\text{Vs}\nonumber
		\end{eqa}
		\item[\bf d)]
		Hier sind zwei Ansätze möglich. Es kann Gleichung \ref{GlfremderregteGM1} auf den Ankerwiderstand umgeformt werden:
		\begin{eqa}
			n &= \frac{U_A}{2\pi K\cdot \Phi} - \frac{R_A\cdot M_i}{2\pi(K\cdot \Phi)^2}\tag{\ref{GlfremderregteGM1}}\\
			R_A &= \left(\frac{U_A}{2\pi K\cdot \Phi} - n\right) \cdot \frac{2\pi(K\cdot \Phi)^2}{M_i}\nonumber\\
			&= \left(\frac{440\,\text{V}}{2\pi\cdot 2,1\,\text{Vs}} - \frac{1900}{60\,\text{s}}\right)\cdot \frac{2\pi\cdot (2,1\,\text{Vs})^2}{210,08\,\text{Nm}} = 220\,\text{m}\Omega\nonumber
		\end{eqa}
		Die andere Berechnungsvariante führt über das Ersatzschaltbild in Abbildung \ref{AbbGMFremderregt}. Im Nennbetrieb können wir die Quellenspannung $U_q$ berechnen. Diese muss kleiner als die Quellenspannung im Leerlauf sein.
		\begin{eqa}
			U_q &= K \cdot \Phi \cdot \omega \tag{\ref{GlInduzierteSpannungGM}}\\
			U_{q,n}&=  2,1\,\text{Vs} \cdot 2\pi\cdot 1900\,\frac{1}{\text{min}} \cdot \frac{1\,\text{min}}{60\,\text{s}} = 418\,\text{V}\nonumber
		\end{eqa}
		Der Spannungsabfall am Widerstand $R_A$ muss durch dem Maschenumlauf die Differenzspannung zwischen Ankerspannung und induzierter Spannung betragen. Da der Ankerstrom im Nennbetrieb bekannt ist, kann der Widerstand über das Ohmsche Gesetz berechnet werden.
		\begin{equation*}
		    R_A = \frac{U_A - U_q}{I_A} = \frac{440\,\text{V} - 418\,\text{V}}{100\,\text{A}} = 220\,\text{m}\Omega
		\end{equation*}
	\end{itemize}
}

\subsection{Materialabhängigkeit des Widerstandes R\label{Materialabhängigkeit des Widerstandes}}
\Aufgabe{
	 Ein Draht aus Kupfer hat folgende Eigenschaften:
\begin{itemize}
  \item Länge: \( l = \SI{5}{\meter} \)
  \item Querschnittsfläche: \( A = \SI{1}{\milli\meter\squared} \)
  \item spezifischer Widerstand von Kupfer: \( \rho = \SI{1.68e-8}{\ohm\meter} \)
\end{itemize}
Berechne \( R \) des Drahtes.
}

\subsection{Elektrischer Widerstand \label{Elektrischer Widerstand}}

\Aufgabe{

Eine kugelförmige Hohlstruktur besteht aus zwei ideal leitfähigen Schalen – einer inneren mit Radius \( a \) und einer äußeren mit Radius \( b \). 

Der Zwischenraum ist vollständig mit einem homogenen, leitfähigen Material gefüllt, das die elektrische Leitfähigkeit \( \kappa \) (in \si{\siemens\per\meter}) besitzt.

Zwischen den leitenden Schalen wird durch eine Spannungsquelle eine Spannung
\[
U = \varphi_1 - \varphi_2
\]
angelegt. 

Der Strom kann sich aufgrund der vollständigen Füllung und der idealen Kontakte ungehindert ausbreiten.

\medskip

Berechnen Sie den elektrischen Widerstand \( R \) zwischen den beiden kugelförmigen Schalen in Abhängigkeit von \( a \), \( b \) und \( \kappa \).

}

\subsection{Elektrischer Widerstand eines beschichteten Drahts\label{Elektrischer Widerstand eines beschichteten Drahts}}

\Aufgabe{

Ein runder Aluminiumdraht besitzt einen Radius von \( a = \SI{0.5}{\milli\meter} \).  
Er ist mit einer dünnen Goldschicht von \( b = \SI{0.1}{\milli\meter} \) Dicke überzogen.  

Für die spezifische elektrische Leitfähigkeit und den Temperaturkoeffizienten der beiden Metalle gelten folgende Werte:

\begin{itemize}
  \item \textbf{Aluminium:} 
  \[
  \kappa_\text{Alu} = \SI{35}{\meter \per (\ohm \milli\meter\squared)}, \quad \alpha_\text{Alu} = 3{,}5 \cdot 10^{-3}\,\si{\per\kelvin}
  \]
  
  \item \textbf{Gold:} 
  \[
  \kappa_\text{Gold} = \SI{44}{\meter \per (\ohm \milli\meter\squared)}, \quad \alpha_\text{Gold} = 3{,}6 \cdot 10^{-3}\,\si{\per\kelvin}
  \]
\end{itemize}

\textbf{Gegeben:}  
Drahtlänge: \( l = \SI{1.5}{\meter} \)  
Temperatur: \( T = \SI{25}{\celsius} \) (Ausgangstemperatur)

\begin{itemize}
  \item[a)] Berechnen Sie den elektrischen Gleichstromwiderstand des Aluminiumkerns und der Goldummantelung bei \( T = \SI{25}{\celsius} \).
  \item[b)] Wie verändert sich der jeweilige Gleichstromwiderstand, wenn die Temperatur auf \( T = \SI{90}{\celsius} \) ansteigt?
\end{itemize}


}



\subsection{Reale Stromquellen\label{Reale Stromquelle}}
\Aufgabe{

Eine reale Stromquelle hat eine Leerlaufspannung \( U_0 = 9\,\si{\volt} \) und einen Innenwiderstand \( R_\text{i} = 1\,\si{\ohm} \).

a) wie groß ist die Klemmenspanung, wenn ein Lastwiderstand von  \( R_l = 4\, \si{\ohm} \) angeschlossen wird ?.
b) wie groß ist der Strom durch die Last?

}

\subsection{Spannungsquelle\label{Spannungsquelle}}
\Aufgabe{

Gegeben ist ein Netzwerk aus einer Spannungsquelle mit \( U = \SI{10}{\volt} \),  
einem Reihenwiderstand \( R_1 = \SI{2}{\ohm} \)  
und einem parallelen Widerstand \( R_2 = \SI{4}{\ohm} \).  

Berechne Ströme und Spannungen im Netzwerk.

}


\subsection{Kapazität und Kondensator\label{Kapazität und Kondensator}}

\Aufgabe{

\textbf{Teil 1: Einzelner Kondensator}

Gegeben:
\[
C = \SI{10}{\pico\farad}, \quad U = \SI{5}{\volt}
\]

\begin{itemize}
    \item[a)] Berechne die Ladung \( Q \)
    \item[b)] Berechne die im Kondensator gespeicherte Energie
\end{itemize}

\vspace{1em}

\textbf{Teil 2: Zwei Kondensatoren in Reihe}

Zwei Kondensatoren \( C_1 \) und \( C_2 \) sind in Reihe geschaltet.

\[
C_1 = \SI{4}{\micro\farad}, \quad C_2 = \SI{6}{\micro\farad}, \quad U_\text{ges} = \SI{12}{\volt}
\]

\begin{itemize}
    \item[c)] Berechne die Gesamtkapazität \( C_\text{ges} \)
    \item[d)] Wie groß ist die Spannung über \( C_1 \)?
\end{itemize}
}

\subsection{Induktivität und Spule\label{Induktivität und Spüle}}

\Aufgabe{

\textbf{Teil 1: Energie in der Spule}

Gegeben ist eine Spule mit der Induktivität
\[
L = \SI{2}{\henry}
\]
und einem Strom
\[
I = \SI{3}{\ampere}.
\]

\begin{itemize}
    \item[a)] Berechne die im Magnetfeld der Spule gespeicherte Energie.
    \item[b)] Wie ändert sich die Energie, wenn der Strom auf \( I = \SI{0.5}{\ampere} \) reduziert wird?
\end{itemize}

\vspace{1em}

\textbf{Teil 2: Induktivität einer zylindrischen Spule}

Eine zylindrische Spule hat
\[
N = 500 \text{ Wicklungen}, \quad l = \SI{20}{\centi\meter}, \quad A = \SI{5}{\centi\meter\squared}
\]
Das Innere der Spule ist luftgefüllt.

Berechne die Induktivität \( L \) der Spule.

}

\subsection{Induktivität, Magnetische Flussdichte und Magnetfeldenergie\label{Induktivität, Magnetische Flussdichte und Magnetfeldenergie}}

\Aufgabe{

Ein luftgefüllter, langer zylindrischer Spulenleiter besitzt folgende Eigenschaften:

\begin{itemize}
  \item Anzahl der Windungen: \( N = 500 \)
  \item Länge der Spule: \( l = \SI{0.5}{\meter} \)
  \item Querschnittsfläche: \( A = 4 \cdot 10^{-4} \, \si{\meter\squared} \)
  \item Stromstärke: \( I = \SI{2.5}{\ampere} \)
\end{itemize}

\textbf{Gegeben:}  
Magnetische Feldkonstante: \( \mu_0 = 4\pi \cdot 10^{-7} \, \si{\henry\per\meter} \)

\begin{itemize}
  \item[a)] Berechnen Sie die magnetische Flussdichte \( \vec{B} \) im Inneren der Spule.
  \item[b)] Bestimmen Sie die Induktivität \( L \) der Spule.
  \item[c)] Berechnen Sie die im Magnetfeld gespeicherte Energie \( E_m \).
\end{itemize}

}




