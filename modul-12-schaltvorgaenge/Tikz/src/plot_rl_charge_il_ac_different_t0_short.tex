% Schaltvorgang: Induktivität wird über Widerstand an Wechselspannung angeschlossen
%
% AxisEnv 1: Quellspannung u_q(t) = Û_q * sin(omega*t + phi_q)
% AxisEnv 2: Strom durch Induktivität i_L(t) = i_{L,h}(t) + i_{L,p}(t) für t_0 = ... (Fall A)
% AxisEnv 3: Strom durch Induktivität i_L(t) = i_{L,h}(t) + i_{L,p}(t) für t_0 = ... (Fall B)

% Allgemein Variablen
\def\ampuq{1} 		% Amptlitude Û_q
\def\phiuq{0} 		% Phase phi_q in Grad -> u_q(t) = Û_q * sin(omega*t + phi_q)
\def\omegauq{1} 	% Kreisfrequenz omega in Hz -> für t=x=1s -> omega*x = 1 = 360°
\def\tnullA{0.375}	% Fall A: Schaltzeitpunkt t_0 
\def\tnullB{0.5} 	% Fall B: Schaltzeitpunkt t_0 

% ACHTUNG: Einstellbare Variablen, theoretisch von einander abhängig, ohne x-Achsen-Beschriftung aber egal (omega beliebig und t bzw. t_0 auch)
\def\ttau{0.5} 	% Zeitkonstante tau = L/R
\def\phiqL{45} 	% Phase \phi_{qL} = arctan(omega*L/R) in Grad, Phasenverschiebung zwischen u_q und i_L [Zwischen 0 und 90° !]
\begin{tikzpicture}[x=1cm,y=1cm]
	\begin{axis}[
		name=AxisUq,
		%%%%%%% START SAME OPTIONS %%%%%%% 
		xmin=-0.5, xmax=2.5, % x = \omega*t [Grad] / 360° = t / T
		ymin=-1.125, ymax=1.375, % \delta y = 2.5 auf 3.5cm
		xlabel={$t$},
		ylabel={$u_q(t)$},
		axis x line=center,
		axis y line=center,
		ylabel style={anchor=east, at={(ticklabel cs:0.95)}},
		ticks=none,
		mark=none,
		grid=none,
		width=7cm,
		height=3.5cm,
		%%%%%%% END OF SAME OPTIONS %%%%%%%
	]
		% Hilfslinien 
		\addplot[color=black,dashed,domain=-0.5:2.45] {+\ampuq};          			% +Û_q
		\addplot[color=black,dashed,domain=-0.5:2.45] {-\ampuq};          			% -Û_q
		% Plots
		\addplot[color=voltage,very thick,samples=161,domain=-0.5:2.45] {\ampuq*cos(\omegauq*x*360 + \phiuq)};% u_q(t)
		% Koordinaten
		\coordinate (X1_1) at (\tnullA,1.5);
		\coordinate (X2_1) at (\tnullB,1.5);
	\end{axis}
	\begin{axis}[
		at={($(AxisUq.south west)+(0,-0.5cm)$)}, % Position 2. Plot
		anchor=north west,
		name=AxisUc1,
		%%%%%%% START SAME OPTIONS %%%%%%%
		xmin=-0.5, xmax=2.5, % x = \omega*t [Grad] / 360° = t / T
		ymin=-1.125, ymax=1.375, % \delta y = 2.5 auf 3.5cm
		xlabel={$t$},
		ylabel={$i(t)$},
		axis x line=center,
		axis y line=center,
		ylabel style={anchor=east, at={(ticklabel cs:0.95)}},
		ticks=none,
		mark=none,
		grid=none,
		width=7cm,
		height=3.5cm,
		%%%%%%% END OF SAME OPTIONS %%%%%%%
		clip=false,
	]
		% Hilfslinien 
		\addplot[color=black,dashed,domain=-0.5:2.45] {+\ampuq};          			% +Û_q
		\addplot[color=black,dashed,domain=-0.5:2.45] {-\ampuq};          			% -Û_q
		\addplot[color=black, thick, dashed, ] coordinates {(\tnullA, {-\ampuq*cos(\omegauq*\tnullA*360 + \phiuq + \phiqL)}) (\tnullA,0)}; % Grenze t_0
		% Plots
		\addplot[color=black, thick,samples=161,domain=-0.5:2.45] {\ampuq*cos(\omegauq*x*360 + \phiuq + \phiqL)};% u_{C,e}(t) eingeschwungen
		\addplot[color=black, thick, dashed, samples=161, domain=\tnullA:2.45] {-\ampuq*exp(-(x-\tnullA)/\ttau)*cos(\omegauq*\tnullA*360 + \phiuq + \phiqL)};% u_{C,f}(t) flüchtig
		\addplot[color=red, very thick, samples=161, domain=\tnullA:2.45] {% Überlagerung
			\ampuq*cos(\omegauq*x*360 + \phiuq + \phiqL)								% eingeschwungen
			-\ampuq*exp(-(x-\tnullA)/\ttau)*cos(\omegauq*\tnullA*360 + \phiuq + \phiqL)} % flüchtig
		;
		% Koordinaten
		\coordinate (X1_2) at (\tnullA,-1.5);
		\coordinate (X2_2) at (\tnullB,-1.5);
	\end{axis}
		%Schaltzeitpunkte
		\draw[color=gray, densely dotted] (X1_1) -- (X1_2);
		%\draw[color=gray, densely dotted] (X2_1) -- (X2_2);
		\node[yshift=-0.3cm] at (X1_2) {$t_0$};
		%\node[yshift=-0.3cm] at (X2_2) {$t_1$};
\end{tikzpicture}
