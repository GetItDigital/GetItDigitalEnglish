\subsection{Maschenstromverfahren 1}
\Aufgabe{
    Gegeben ist das Netzwerk einer realen Stromgquelle $I_\mathrm{q} = 6\ A$ mit einem Leitwert
    vo $G_\mathrm{i} = \frac{2}{3}\ S$.

    \begin{figure}[H]
        \resizebox{0.3\textwidth}{!}{%  
        \begin{circuitikz}                                       
            \draw (0,0) [I, name=Iq] to (0,3);
            \draw (0,3) to [short, -*] (2,3)
                to [R=$G_\mathrm{i}$, -*] (2,0)
                to (0,0);
            \draw (2,0) to [short, -o] (4,0);
            \draw (2,3) to [short, -o] (4,3);
            \iarrmore{Iq}{$I_\mathrm{q}$};
        \end{circuitikz}   
        }\centering                                                           
    \end{figure}

    Wandeln Sie die angegebene reale Stromquelle in eine reale Spannungsquelle um.
    Befolgen Sie hierzu die folgenden Schritte:
        
    \begin{itemize}
            \item[\bf a)]
                Bestimmen Sie den Widerstandswert des Leitwertes. 
            \item[\bf b)]
                Wandeln Sie die reale Stromquelle in eine reale Spannungsquelle um. 
            \item[\bf c)]
                Berechnen Sie die Spannung der umgewandelten Spannungsquelle.
    \end{itemize}
    }

\Loesung{
	\begin{itemize}
        \item[\bf a)]
            Widerstandswert:    \\

            \begin{eq}
                R_\mathrm{i} = \frac{1}{G_\mathrm{i}} = \frac{1}{\frac{2}{3}\ S} = 1,5\ \Omega   \nonumber
            \end{eq}

		\item[\bf b)]
		    Spannungsquelle:    \\

            \begin{figure}[H]
                \resizebox{0.3\textwidth}{!}{%  
                \begin{circuitikz}
                    \draw (8,0) [V^<, name=Uq] to (8,3);
                    \draw (8,3) to [R, l=$R_\mathrm{i}$, v, name=Ri, -o] (12,3);
                    \draw (8,0) to [short, -o] (12,0);
                    
                    \varrmore{Uq}{$U_\mathrm{q}$};
                    \varrmore{Ri}{$U_\mathrm{Ri}$};
                    \end{circuitikz}   
                }\centering                                                           
            \end{figure}

        \item[\bf c)]
            Spannungswert \\

            \begin{eq}
                U_\mathrm{q} = R_\mathrm{i} \cdot I_\mathrm{q} = 1,5\ \mathrm{\Omega} \cdot 6\ \mathrm{A} = 9\ \mathrm{V}   \nonumber
            \end{eq}

	\end{itemize}
}