\subsection{Arbeit und Energie\label{Arbeit und Energie}}
\Loesung{
	\begin{itemize}	
		\item[\bf a)]
		\begin{equation*}
		    W = m\cdot g \cdot h 
		\end{equation*}
		\begin{equation*}
		    W = 10 kg \cdot 9,81 \frac{m}{s^2}\cdot 5 m
		\end{equation*}
		\begin{equation*}
			W = 490,5J
		\end{equation*}
		\item[\bf b)]
		\begin{equation*}
		    Wirkungsgrad = \frac{genutzte Energie}{aufgewendete Energie} = \eta
		\end{equation*}
		\begin{equation*}
		    E_{zu}=\frac{W}{\eta}=\frac{490,5J}{80\%}=613,125J
		\end{equation*}
	\end{itemize}
	}

\subsection{Höhenenergie im Stausee\label{Höhenenergie im Stausee}}
\Loesung{
    \begin{itemize}
        \item[\bf a)]
        \begin{equation*}
            E_{\text{pot}} = m \cdot g \cdot h
        \end{equation*}
        \begin{equation*}
            m = \rho \cdot V = 1000\,\mathrm{kg/m^3} \times 2\,000\,000\,\mathrm{m^3} = 2\,000\,000\,000\,\mathrm{kg}
        \end{equation*}
        \begin{equation*}
            E_{\text{pot}} = 2\,000\,000\,000\,\mathrm{kg} \times 9{,}81\,\mathrm{m/s^2} \times 100\,\mathrm{m}
        \end{equation*}
        \begin{equation*}
            E_{\text{pot}} = 1{,}962 \times 10^{12}\,\mathrm{J}
        \end{equation*}
    \end{itemize}
}

\subsection{Bewegungsenergie eines Autos\label{Bewegungsenergie eines Autos}}
\Loesung{
    \begin{itemize}
        \item[\bf a)]
        \begin{equation*}
            E_{\text{kin}} = \frac{1}{2} m v^2
        \end{equation*}
        \begin{equation*}
            m = 1200\,\mathrm{kg}
        \end{equation*}
        \begin{equation*}
            v = 72\,\mathrm{km/h} = 72 \times \frac{1000\,\mathrm{m}}{3600\,\mathrm{s}} = 20\,\mathrm{m/s}
        \end{equation*}
        \begin{equation*}
            E_{\text{kin}} = \frac{1}{2} \times 1200\,\mathrm{kg} \times (20\,\mathrm{m/s})^2
        \end{equation*}
        \begin{equation*}
            E_{\text{kin}} = 240\,000\,\mathrm{J}
        \end{equation*}
    \end{itemize}
}


\subsection{Elektrische Energie eines Geräts\label{Energie und eines Geräts}}
\Loesung{
    \begin{itemize}
        \item[\bf a)]
        \begin{equation*}
            P = 2000\,\mathrm{W}
        \end{equation*}
        \begin{equation*}
            t = 3\,\mathrm{min} = 180\,\mathrm{s}
        \end{equation*}
        \begin{equation*}
            E = P \cdot t = 2000\,\mathrm{W} \times 180\,\mathrm{s} = 360\,000\,\mathrm{J}
        \end{equation*}
        
        \item[\bf b)]
        \begin{equation*}
            m = 5\,\mathrm{kg} \quad (\text{Dichte Wasser} \approx 1\,\mathrm{kg/L})
        \end{equation*}
        \begin{equation*}
            c = 4186\,\mathrm{\frac{J}{kg \cdot K}} \quad (\text{spezifische Wärmekapazität von Wasser})
        \end{equation*}
        \begin{equation*}
            \Delta T = 100^\circ\mathrm{C} - 20^\circ\mathrm{C} = 80\,\mathrm{K}
        \end{equation*}
        \begin{equation*}
            Q = m \cdot c \cdot \Delta T = 5\,\mathrm{kg} \times 4186\,\mathrm{\frac{J}{kg \cdot K}} \times 80\,\mathrm{K} = 1\,674\,400\,\mathrm{J}
        \end{equation*}
    \end{itemize}
}


\subsection{Energie und Leistung\label{Energie und Leistung}}
\Loesung{
	\begin{itemize}
        \item[\bf a)]
        \begin{equation*}
        E = P\cdot t
		\end{equation*}   
		\begin{equation*}
		E = 2\,\text{kW} \cdot 3\,\text{h} = 6\,\text{kWh}
		\end{equation*}
		\item[\bf b)]
		\begin{equation*}
		C = E\cdot P
		\end{equation*}
		\begin{equation*}
		C = 6\,\mathrm{kWh} \cdot 0{,}30\,\frac{€}{\mathrm{kWh}} = 1{,}80\,\text{€}
		\end{equation*}
    \end{itemize}
	}
\subsection{Wirkungsgrad\label{Wirkungsgrad}}
	\Loesung{
	\begin{itemize}
		\item[\bf a)]
		\begin{equation*}
		\eta = \frac{P_{\text{el}}}{P_{\text{mech}}} = \frac{1\,\text{kW}}{1{,}5\,\text{kW}} \approx 0{,}6667 \Rightarrow \eta \approx \SI{66.67}{\percent}
		\end{equation*}   
		\item[\bf b)]
		\begin{equation*}
		P_{\text{mech}} = \frac{2\,\text{kW}}{0{,}6667} \approx 3\,\text{kW}
		\end{equation*}		
		\end{itemize}
	}