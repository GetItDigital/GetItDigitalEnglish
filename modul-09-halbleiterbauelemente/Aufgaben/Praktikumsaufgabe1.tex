\subsection{Praktikumsaufgabe 1\label{Aufg29}}
\Aufgabe{
    \begin{figure}[H]
    \centering
    \begin{tikzpicture}
    \draw (2,4) to[R,l=$R_1$, -*] (2,2) to[R,l=$\mathrm{R_2}$,-*] (2,0);
    \draw (2,2) -- (4,2) to[R,l=$\mathrm{R_L}$] (4,0) -- (2,0);
    \draw (2,0) to[short,-*] (0,0)node[ground]{};
    \draw (2,4) -- (0,4) to[sinusoidal voltage source,name=V] (0,0);
    
    \varrmore{V}{$\mathrm{V}$};
\end{tikzpicture}
   
    \label{fig:Praktikumsaufgabe1}
    \end{figure}

    Berechnen Sie den Widerstand $\mathrm{R_1}$, sodass an $\mathrm{R_2}$ eine Spannung von 1\,V anliegt. Gegeben ist eine Eingangsspannung von $\mathrm{U_E}$ und ein Widerstand $\mathrm{R_2}$.
    
    \begin{itemize}
        \item[a)] Leiten Sie die allgemeine Formel für den Spannungsteiler her.
        \item[b)] Berechnen Sie $\mathrm{R_1}$ für $\mathrm{U_E} = 5\,\text{V}$ und $\mathrm{R_2} = 2\,\text{k}\Omega$.
    \end{itemize}
}


\Loesung{

}