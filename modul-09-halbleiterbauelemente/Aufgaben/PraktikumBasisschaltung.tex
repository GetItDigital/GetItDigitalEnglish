\subsection{Differentieller Widerstand einer Diode\label{Aufg34}}
% Alt: Praktikumsaufgabe 7
\Aufgabe{
    \begin{figure}[H]
    \centering
    \begin{tikzpicture}
    \draw (-1,2) to[sinusoidal voltage source,name=V2] (-1,0) -- (0,0);
    \draw  (-1,2) -- (2,2);
    \draw (2,0) to[potentiometer, l=$P_\mathrm{1}$, *-] (2,2);
    \draw (0,0) to[short,-*](1,0)node[ground]{} -- (3.25,0);
    
    \draw (2,1)--(3,1)to[R,l=$R_\mathrm{4}$](5,1) to[C,l=$C_\mathrm{1}$, -*] (5,-1);
    \draw (5,-1) to[D,l=$D_\mathrm{1}$] (7,-1) -- (7,-3) to[short,-*] (2,-3)node[ground]{}--(0,-3);
    
    \draw (0,-1) to[R,l=R1] (5,-1);
    \draw (0,-1) to[V,name=V1](0,-3);
    
    \draw (3,1) to[open,v,name=du](3,0);
    
    \varrmore{du}{$\mathrm{\Delta U}$};
    \varrmore{V1}{$V_\mathrm{1}$};
    \varrmore{V2}{$V_\mathrm{2}$};
\end{tikzpicture}
   
    \label{fig:FigPraktikumBasisschaltung}
    \end{figure}
    
  Analysieren Sie das Grundverhalten eines Bipolartransistors in einer Basisschaltung.
    
    Gegeben:
    \begin{itemize}
        \item NPN-Transistor mit Stromverstärkungsfaktor $\beta = 100$
        \item Basisstrom $I_\mathrm{B} = 20\,\mathrm{\mu A}$
    \end{itemize}
    
    \begin{itemize}
        \item[a)] Berechnen Sie den Kollektorstrom $I_\mathrm{C}$.
        \item[b)] Bestimmen Sie die Kollektor-Emitter-Spannung $U_{CE}$, wenn der Kollektorwiderstand $R_\mathrm{C} = 1\,\mathrm{k\Omega}$ und die Versorgungsspannung $U_\mathrm{CC} = 12\,\mathrm{V}$ beträgt.
    \end{itemize}
}



\Loesung{
    \begin{itemize}
        \item[a)] Berechnung des Kollektorstroms $I_\mathrm{C}$  
        Der Kollektorstrom eines Bipolartransistors ergibt sich aus der Beziehung:
        $$ I_\mathrm{C} = \beta \cdot I_\mathrm{B} $$
        mit den gegebenen Werten:
        $$ I_\mathrm{C} = 100 \cdot 20\,\mathrm{\mu A} $$
        $$ I_\mathrm{C} = 2\,\mathrm{mA} $$
    
        \item[b)] Bestimmung der Kollektor-Emitter-Spannung $U_\mathrm{CE}$  
        Die Kollektor-Emitter-Spannung ergibt sich aus der Gleichung:
        $$ U_\mathrm{CE} = U_\mathrm{CC} - I_\mathrm{C} \cdot R_\mathrm{C} $$
        Einsetzen der Werte:
        $$ U_\mathrm{CE} = 12\,\mathrm{V} - (2\,\mathrm{mA} \cdot 1\,\mathrm{k\Omega}) $$
        $$ U_\mathrm{CE} = 12\,\mathrm{V} - 2\,\mathrm{V} $$
        $$ U_\mathrm{CE} = 10\,\mathrm{V} $$
    \end{itemize}
    Siehe: Abschnitt \ref{fig:DiodenkennlinieUndErsatzschaltbild}
}