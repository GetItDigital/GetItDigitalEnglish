%Superposition
\newpage
%Einleitung
\begin{frame}
	\fta{Superpositionsprinzip}

	\s{ 
        Nach der grundsätzlichen Analyse von Maschen und Knoten in elektrischen Netzwerken
        werden folgend zunehmend kompliziertere elektrische Netzwerke
        betrachtet. Hierzu wird das Überlagerungsverfahren nach Helmholtz verwendet. 
        Das Überlagerungsverfahren wird auch als Superpositionsprinzip bezeichnet. Beim Superpositionsprinzip
        werden nacheinander alle Quellen einzeln ausgewertet und die Ergebnisse der einzelnen 
        Berechnungen überlagert. 
    }


    \begin{Lernziele}{Superpositionsprinzip}
        Die Studierenden 
        \begin{itemize}
            \item verstehen die Bedingungen der Systemtheorie für die Analyse von Gleichstromnetzwerken. 
            \item können den Überlagerungssatz (das Superpositionsprinzip) auf elektrische Netzwerke anwenden. 
        \end{itemize}                   
    \end{Lernziele}

\end{frame}
 
%Exkurs Systemtheorie
\begin{frame}
    \ftb{Exkurs Systemtheorie}

    \s{
        Wird in einem Modell lediglich das Einganssignal und das Ausgangssignal betrachtet, wird von
        einer Blackbox gesprochen. Hier ist keinerlei Information darüber gegeben, was zwischen diesen 
        beiden Signalen passiert. Die Abbildung \ref{BildBlackBox} beschreibt so ein Modell. Hier 
        wird eine Black Box betrachtet, welche als Eingansgröße $F_\mathrm{E}$ und als Ausgangsgröße $F_\mathrm{A}$
        aufweist. In der Blackbox wird das Eingangssignal in das Ausgangssignal transformiert. 
        Das Bildnis insgesamt wird als System bezeichnet. 
    }

    \b{
    \begin{itemize}    
		\item Betrachtung einer Blackbox als System mit Eingangsgrößen und Ausgangsgrößen.
    \end{itemize}
    }

    \fu{
        \begin{tikzpicture}
    \draw (2,0) rectangle (4,2);
    \draw[->] (0,1) -- (2,1);
    \draw[->] (4,1) -- (6,1);
    \draw(-0.5,1) node {$F_\mathrm{E}$};
    \draw(6.5,1) node {$F_\mathrm{A}$};
    \draw(3,1) node {$T$};
\end{tikzpicture}
        }{{\bf Blackbox in der Systemtheorie.} Eine Blackbox mit linksseitiger Eingangsgröße und rechtsseitiger Ausgangsgrößen.
        \label{BildBlackBox}}

    \b{
    \begin{itemize}    
		\item Transformation des Eingangssignals $F_\mathrm{E}$ in ein Ausgangssignal $F_\mathrm{A}$.
    \end{itemize}
    }

    \s{
        Ausgehend von der vorgestellten Blackbox lässt sich das Ausgangssignal $F_\mathrm{A}$ in der Abhängigkeit 
        des Eingangssignals $F_\mathrm{E}$ bestimmen. Diese Transformation lässt sich durch die Gleichung 
        \ref{GleichungFunktionSystem} beschreiben.
    }

    \begin{eq}
        F_\mathrm{A}=T(F_\mathrm{E})      \label{GleichungFunktionSystem}
    \end{eq}
\end{frame}

\begin{frame}
    \ftx{Black Box}

    \s{
        Ein elektrisches Netzwerk kann auch als solch ein System betrachtet werden. Das in der Abbildung
        \ref{BildElektrischesSystem} dargestellte elektrische Netzwerk verfügt über eine Eingangsspannung 
        $U_\mathrm{E}$ als Eingangsgröße und eine Ausgangsspannung $U_\mathrm{A}$ als Ausgangsgröße. Die Blackbox, welche 
        das zu transformierende System abbildet, setzt sich aus den beiden Widerständen $R_1$ und $R_2$ 
        zusammen.  
    }

    \b{
        \begin{itemize}    
            \item Beschreibung einer Black Box als elektrischen Netzwerk mit einer Eingangsspannung
            und einer Ausgangsspannung.
        \end{itemize}
    }

    \fu{
        \begin{circuitikz}
    \draw(0,4) to [short,o-] (2,4)
    (2,4) to [R=$R_1$,-*] (2,2)
    to [R=$R_2$,-*] (2,0)
    to[short,-o] (4,0)
    (2,2) to[short,-o] (4,2)
    (0,0) to[short,o-] (2,0);
    \draw[->] (0,3.5) -- (0,0.5);
    \draw[->] (4,1.5) -- (4,0.5);
    \draw(-0.5,2) node {$U_\mathrm{E}$};
    \draw(4.5,1) node {$U_\mathrm{A}$};
    \draw[red](1,-0.2) rectangle (3,4.2);
    \draw[red](3.5,3) node {$T$};
\end{circuitikz}
        
        }{{\bf Spannungsteiler als Blackbox.} Betrachtung eines elektrischen Netzwerkes als Blackbox mit Eingangsspannung und Ausgangsspannung.
        \label{BildElektrischesSystem}}

    \s{
        Der Zusammenhang der Abhänigkeit des Ausgangssignals eines Systems vom Eingangssignal aus Gleichung \ref{GleichungFunktionSystem} gilt weiterhin. Das Ausgangssignal 
        lässt sich als Transformation des Eingangssignales wie in der Gleichung \ref{GleichungNetzwerk} 
        beschreiben. Das Ausgangssignal ist gleich der Spannung, welche über dem Widerstand $R_2$ abfällt. 
        Diese Spannung lässt sich aus dem Produkt der Eingangsspannung und dem Widerstandsverhältnis berechnen. 
        Dieser Zusammenhang wurde auch bereits beim Spannungsteiler im Kapitel 4 behandelt.    
    }
    
    \b{
        \begin{itemize}    
            \item Transformation der Eingangsspannung $U_\mathrm{E}$ in eine Ausgangsspannung $U_\mathrm{A}$.
        \end{itemize}
    }

    \begin{eq}
        U_\mathrm{A}=T(U_\mathrm{E})=U_\mathrm{E}\cdot\frac{R_2}{R_1+R_2}  \label{GleichungNetzwerk}
    \end{eq}
\end{frame}

\begin{frame}
    \ftx{LTI Systeme}

    \s{
        \textbf{Kausalität}\\

        Beruht eine Ausgangsgröße ausschließlich aus der Transformation einer Eingangsgröße, so ergibt sich
        ein direktes Verhältnis zwischen Ursache und Wirkung. Dies bedingt auch, dass die Ausgangsgröße vor 
        einer Anregung keine sich ändernde Systemantwort liefert. Ein System, in welchem diese Bedingungen vorherrschen, 
        wird als kauseles System bezeichnet (vgl. Abbildung \ref{BildKausalesSystem}). 
        So zeigt beispielsweise eine Gleichspannungsanregung eines Systems eine zeitlich indifferente
        Gleichstromantwort. Ein System, welches sich anders verhält, wird als nicht kausales Systeme bezeichnet. 
        Wenn für $t < t_0$ der Wert des Eingangssignals Null ist, muss der Wert des Ausgangssignals 
        für denselben Zeitraum ebenfalls Null sein, damit es die Bedingungen eins kausalen Systems
        erfüllt. 
    }

    \b{
        \begin{itemize}    
            \item Kausale Systeme zeigen keine Sytemantwort vor der Anregung
            \item Zeitinvariante Systeme reagieren immer gleich auf ein Eingangssignal, unabhängig vom Zeitpunkt
            \item Lineares System $\rightarrow$ endliche Ausgangsamplitude bei endlicher Eingangsamplitude
            \item Zeitinvariante und lineare System werden als LTI-Systeme kategorisiert
        \end{itemize}
    }

    \fu{
        \resizebox{.9\textwidth}{!}{%
        \begin{minipage}{0.2\textwidth}
            \begin{circuitikz}
    \draw[very thin,color=gray] (-0.2,-0.2) grid (2.2,1.2);
    \draw[->] (-0.2,0) -- (2.2,0) node[right] {$t$};
    \draw(0.2,0) node[below] {$t_0$};
    \draw[->] (0,-0.2) -- (0,1.2) node[above] {$x(t)$};							
    \draw[very thick,red](0.2,0) -- (0.2,0.9)
    (0.2,0.9) -- (2,0.9); 
    \draw[dotted,very thick,red](-0.2,0) -- (-0.2,0.9)
    (-0.2,0.9) -- (1.6,0.9); 
\end{circuitikz}	
		\end{minipage}
        \hspace{0.1cm}
        \begin{minipage}{0.5\textwidth}
            \begin{tikzpicture}
    \draw (2,0) rectangle (4,2);
    \draw[double,->] (0,1) -- (2,1);
    \draw[double,->] (4,1) -- (6,1);
    \draw(-0.5,1) node {$x(t)$};
    \draw(6.5,1) node {$y(t)$};
    \draw(3,1) node {$T$};
\end{tikzpicture}	
		\end{minipage}
        \hspace{0.1cm}
        \begin{minipage}{0.2\textwidth}
            \begin{circuitikz}
    \draw[very thin,color=gray] (-0.2,-0.2) grid (2.2,1.2);
    \draw[->] (-0.2,0) -- (2.2,0) node[right] {$t$};
    \draw(0.2,0) node[below] {$t_0$};
    \draw[->] (0,-0.2) -- (0,1.2) node[above] {$y(t)$};							
    \draw[very thick,green](0.2,0.0) -- (0.6,0.6)
    (0.6,0.6) -- (2,0.6); 
    \draw[dotted, very thick,green](-0.2,0.0) -- (0.1,0.6)
    (0.1,0.6) -- (2,0.6);
\end{circuitikz}	
		\end{minipage}
        }
	}{{\bf Transformation eines kausalen Systems.} Aus $x(t)$ transformiertes System zu $y(t)$ für ein kausales System. \label{BildKausalesSystem}}

    \s{
        \textbf{Zeitinvarianz}\\
        
        Reagiert das Ausgangssignal eines Systems zeitlich immer gleich auf ein Eingangssignal, so wird 
        von einem zeitinvarianten Systemen gesprochen. Verschiebt sich der Zeitpunkt des Eingangssignals beispielsweise 
        um $t_0$, so muss weiterhin bei zeitinvarianten Systemen das Ausgangssignal immer gleich, ausgehend vom Eingangssignal,
        reagieren.

        Die Stabilität eines Systems beschreibt ebenfalls einen Zusammenhang zwischen dem Eingangssignal
        und dem Ausgangssignal. Stellt sich bei einem Eingangssignal mit endlicher Amplitude ein nicht über
        alle Grenzen wachsendes Ausgangssignal ein, ist das System stabil. Wächst das Ausgangssignal nach 
        der Inaktivierung des Eingangssignales weiter, ist das Systen instabil. 

        \begin{eq}
            T(\alpha \cdot u_\mathrm{e})=\alpha \cdot T(u_\mathrm{e})    \label{GleichungLinearität}
        \end{eq}

        \textbf{Linearität}\\

        Ist die Reaktion des Ausgangssignals zu der Anregung des Eingangssinals proportional, wird das
        System als \textbf{lineares} System beschrieben (Gleichung \ref{GleichungLinearität}). Wirken zwei überlagerte 
        Eingangssignale in ein System, so können sie in diesem Fall separat betrachtet und aufsummiert werden. Auf diese Weise 
        können auch die Transformationen der Eingangssignale, wie in der Gleichung \ref{GleichungÜberlagerteSignale},
        getrennt betrachtet werden. 

        \begin{eq}
            T(u_1+u_2)=T(u_1)+T(u_2)    \label{GleichungÜberlagerteSignale}
        \end{eq}

        Systeme, die sowohl zeitinvariant, als auch linear sind, werden als LTI-Systeme (Linear Time Invariant) 
        bezeichnet. Die Systeme in diesem betrachteten Modul werden in Näherung als LTI-Systeme betrachtet. Bauelemente,
        welche Nichtlinearitäten aufweisen, werden beispielsweise im Kapitel über periodische Größen behandelt. 
    }   

\end{frame}

\begin{frame}
    \ftx{LTI-Systeme}

    \begin{Merksatz}{LTI-Systeme}
        LTI-Systeme stellen lineare und zeitinvariante Systeme dar. Gleichstromnetzwerke müssen als LTI-Systeme betrachtet werden, damit
        der Überlagerungssatz auf sie angewendet werden kann. 
    \end{Merksatz}

\end{frame}



%Überlagerungssatz
\begin{frame}
    \ftb{Überlagerungssatz}

    \s{
        Sind die elektrischen Netzwerke als betrachtete Systeme auf ihre vorhandene Linearität geprüft, 
        kann der Überlagerungssatz angewendet werden. Hier können verschiedene Quellen von einander getrennt 
        betrachtet werden. Auf diese Weise werden beim Überlagerungssatz alle Quellen
        ausgeschaltet und reihenweise die einzelnen Quellen eingeschaltet. Beim Ausschalten von Quellen 
        werden Stromquellen und Spannungsquellen verschieden umgewandelt. Beim Ausschalten einer idealen
        Spannungsquelle entsteht nach Abbildung \ref{BildSpannungErsetzen} am Ort der Spannungsquelle ein Kurzschluss. 
    }

    \b{
        Bedinungen und Ausführung des Überlagerungssatzes: 
        \begin{itemize}
            \item Vorausgesetzte Linearität
            \item Ausschalten der Quellen
            \item Ideale Spannungsquelle wird durch einen Kurzschluss ersetzt
            \item Ideale Stromquelle wird durch offene Klemmen, einen Leerlauf, ersetzt
        \end{itemize}
    }

    \fu{
        \resizebox{.6\textwidth}{!}{%
        \hspace{0.1cm}
        \begin{minipage}{0.3\textwidth}
            \begin{circuitikz}
    \draw (0,0) to [V=$U_\mathrm{q}$] (0,3);
    \draw (-1.5,1.5) node[left, align=center] {Ideale\\Spannungs-\\quelle};
\end{circuitikz}	
		\end{minipage}
        \hspace{0.1cm}
        \begin{minipage}{0.2\textwidth}
            \begin{tikzpicture}
    \draw[->] (0,0) -- (2,0);
\end{tikzpicture}		
		\end{minipage}
        \hspace{0.1cm}
        \begin{minipage}{0.3\textwidth}
            \begin{circuitikz}
    \draw (0,0) to [short,-*] (0,1);
    \draw (0,1) to [short] (0,2);
    \draw (0,3) to [short,-*] (0,2);
    \draw (1.5,1.5) node[right, align=center] {Kurz-\\schluss};
\end{circuitikz}		
		\end{minipage}
        \hspace{0.1cm}
        }
	}{{\bf Umwandlung einer Spannungsquelle.} Wird eine Spannungsquelle deaktiviert, verbleibt ein Kurzschluss. Mit diesem Kurzschluss erfolgt die weitere 
    Netzwerkberechnung.  \label{BildSpannungErsetzen}}

    \s{
        Eine Stromquelle hinterlässt beim Ausschalten lediglich offene Klemmen (Abbildung \ref{BildStromErsetzen}). 
        Dieser Leerlauf verhindert die weitere Betrachtung dieses Pfades im Netzwerk. 
    }

    \fu{
        \resizebox{.6\textwidth}{!}{%
        \hspace{0.1cm}
        \begin{minipage}{0.3\textwidth}
            \begin{circuitikz}
    \draw (0,0) to [I=$I_\mathrm{q}$] (0,3);
    \draw (-1.5,1.5) node[left, align=center] {Ideale\\Stromquelle};
\end{circuitikz}	
		\end{minipage}
        \hspace{0.1cm}
        \begin{minipage}{0.2\textwidth}
            \begin{tikzpicture}
    \draw[->] (0,0) -- (2,0);
\end{tikzpicture}		
		\end{minipage}
        \hspace{0.1cm}
        \begin{minipage}{0.3\textwidth}
            \begin{circuitikz}
    \draw (0,0) to [short,-*] (0,1);
    \draw (0,3) to [short,-*] (0,2);
    \draw (1.5,1.5) node[right, align=center] {Leerlauf\\(offene\\Klemmen)};
\end{circuitikz}		
		\end{minipage}
        \hspace{0.1cm}
        }
	}{{\bf Umwandlung einer Stromquelle.} Die Deaktivierung einer Stromquelle hinterlässt offene Klemmen.   \label{BildStromErsetzen}}

\end{frame}

\begin{frame}
    \ftx{Umwandlung von Quellen}

    \begin{Merksatz}{Umwandlung von Quellen}
        Beim Ausschalten von Spannungsquellen und Stromquellen hinterlassen deaktivierte Spannungsquellen einen Kurzschluss und 
        deaktivierte Stromquellen einen Leerlauf.
    \end{Merksatz}

\end{frame}

\begin{frame}
    \ftx{Zusammenfassung Überlagerungssatz}

    \s{
        Sind bis auf eine Quelle alle anderen Quellen ausgeschaltet,
        wird das elektrische Netzwerk für die übriggebliebene Quelle analysiert. Das wird dann aufeinanderfolgend 
        mit jeder Quelle durchgeführt. Am Ende werden die Einzelwirkungen als Summe betrachtet. 
        Beispielsweise lässt sich der Strom $I_\mathrm{R}$ durch einen Widerstand R, welcher von zwei Quellen $Q_1$ und $Q_2$ 
        versorgt wird, durch 
        die Summe der Teilströme der beiden Quellen erklären (vgl. Gleichung \ref{GleichungZweiQuellen}). 
    }

    \b{
        Annahme: 
        \begin{itemize}
            \item Der Widerstand $R$ wird durch die beiden Quellen $Q_1$ und $Q_2$ versorgt
            \item Separate Berechnung des Stromes durch $R$ 
            \item Gesamtstrom durch $R$ ergibt sich aus der Summe der Ströme aus den beiden Quellen:
        \end{itemize}
    }

    \begin{eq}
        I_\mathrm{R}=f(Q_1,Q_2)    \label{GleichungZweiQuellen}
    \end{eq}


\end{frame}

\begin{frame}
    \ftx{Beispiel}

    \s{
        Zur Verdeutlichung des Überlagerungssatzes wird in der Abbildung \ref{BildÜberlagerungssatz} ein elektrisches Netzwerk mit zwei 
        Spannungsquellen und drei Widerständen abgebildet. Um dieses Netzwerk mit mehr als einer Quelle zu berechnen, 
        wird das Netzwerk für beide Spannungsquellen einzeln betrachtet. 
    }

    \b{
        Berechnung eines Beispielnetzwerks:  

    }

    \fu{
        \resizebox{.6\textwidth}{!}{%
		\begin{circuitikz}
    \draw (0,3) to [V_, name=U1] (0,0)
    (0,3) to [R=$R_1$,-*] (3,3)
    to [R=$R_3$] (6,3)
    to [V, name=U2] (6,0)
    to [short, -*] (3,0)
    to [short] (0,0)
    (3,3) to [R=$R_2$] (3,0);
    \varrmore{U1}{$U_{1}$};
    \varrmore{U2}{$U_{2}$};
\end{circuitikz}
		}%
    }{{\bf Elektrisches Netzwerk mit zwei Spannungsquellen zur Erklärung des Überlagerungssatzes.} Die Spannungsquellen sollen
    nacheinander für das elektrische Netzwerk analysiert werden. \label{BildÜberlagerungssatz}}

    \s{
        Das vorgestellte elektrische Netzwerk wird in der Abbildung \ref{BildÜberlagerungssatzQuellen} noch einmal jeweils für die 
        Berechnung der beiden Spannungsquellen seperat angezeigt. Für die Netzwerkbetrachtung mit der Spannungsquelle
        $U_1$ wird die Spannungsquelle $U_2$ kurzgeschlossen. Die beiden Widerstände $R_2$ und $R_3$ liegen nun parallel
        zueinander. Die Spannung von $U_1$ verteilt sich nun über $R_1$ und die Parallelschaltung $R_{23}$. Äquivalent dazu wird
        bei der Betrachtung des Netzwerkes für die Spannungsquelle $U_2$ die Spannungsquelle $U_1$ kurzgeschlossen. Nun bilden 
        die beiden Widerstände $R_1$ und $R_2$ eine Parallelschaltung. Die Spannung der Spannungsquelle $U_2$ verteilt sich über 
        die Parallelschaltung $R_{12}$ und den Widerstand $R_3$. 
    }

\end{frame}  



\begin{frame}
    \ftx{Beispiel Lösung}

    \fu{
        \resizebox{0.8\textwidth}{!}{%
		\begin{circuitikz}
    \draw (0,3) to [V_, name=U1] (0,0)
    (0,3) to [R=$R_1$,-*] (3,3)
    to [R=$R_3$] (6,3)
    to (6,0) 
    to [short, -*] (3,0)
    to [short] (0,0)
    (3,3) to [R=$R_2$] (3,0)
    (-1,3.5) node {$U_1)$};
    \varrmore{U1}{$U_{1}$};

    \draw (8,0) to (8,3) 
    to [R=$R_1$,-*] (11,3)
    to [R=$R_3$] (14,3)
    to [V, name=U2] (14,0)
    to [short, -*] (11,0)
    to [short] (8,0)
    (11,3) to [R=$R_2$] (11,0)
    (7,3.5) node {$U_2)$};
    \varrmore{U2}{$U_{2}$};
\end{circuitikz}
		}%
    }{{\bf Elektrisches Netzwerk mit zwei Spannungsquellen zur Erklärung des Überlagerungssatzes.} Links wird das Netzwerk für die Analyse
    der Spannungsquelle $U_1$ und rechts für die Spannungsquelle $U_2$ gezeigt.  \label{BildÜberlagerungssatzQuellen}}

    \s{
        In der Gleichung \ref{GleichungU1} und der Gleichung \ref{GleichungU2} werden die Spannungen für die beiden Spannungsquellen
        $U_1$ und $U_2$ am Widerstand $R_3$ separat bestimmt. In der Gleichung \ref{GleichungUR3} werden die beiden Spannungen über den 
        Widerstand $R_3$ dann nach dem Überlagerungssatz aufsummiert.
    }
    
    

    \begin{eq}
        U_\mathrm{R3}(U_1) = U_1 \cdot \frac{R_\mathrm{23}}{R_1+R_\mathrm{23}} \label{GleichungU1} 
    \end{eq}
    \begin{eq}
        U_\mathrm{R3}(U_2) = U_2 \cdot \frac{R_\mathrm{12}}{R_3+R_\mathrm{12}} \label{GleichungU2} 
    \end{eq}
    \begin{eq}
        U_\mathrm{R3} = U_\mathrm{R3}(U_1) + U_\mathrm{R3}(U_2) \label{GleichungUR3}
    \end{eq}

\end{frame}


\begin{frame}
    \ftx{Überlagerungssatz}

    \begin{Merksatz}{Überlagerungssatz}
        Lineare und zeitlich invariante elektrische Netzwerke mit mehr als einer Quelle können als Summe der Teilanalysen von jeder 
        einzelnen Quelle bestimmt werden. 
    \end{Merksatz}

\end{frame}

\newpage
\begin{frame}
    \ftx{Beispiel Superpositionsverfahren}
    \begin{bsp}{Superpositionsverfahren}{}

    \s{
        Gegeben ist das elektrische Netzwerk nach Abbildung \ref{BildBeispielaufgabeSuperposition}. Die folgenden 
        Aufgaben sollen bearbeitet werden: 
    }

        \begin{enumerate}
            \only<1>{	
                \item[]
                Anwendung des Superpositionsverfahrens (Überlagerungssatz): 
                \begin{enumerate}
                    \item[a)] Einzeichnen der Ströme und Spannungen
                    \item[b)] Netzwerkberechnung $I_3$ für $U_\mathrm{g}$
                    \item[c)] Netzwerkberechnung $I_3$ für $I_\mathrm{g}$
                    \item[d)] Wie groß ist der Strom $I_3$
                \end{enumerate}

                
                \fu{
                    \begin{circuitikz}
    \draw
    (0,2) to [V, name=Ug] (0,0)
    (0,2) to [R=$R_\mathrm{i}$] (0,4)
    (0,4) to [short,-*] (4,4)
    (12,4) to [short,-*] (8,4)
    (4,4) to [R=$R_1$] (4,0)
    (4,4) to [R=$R_3$, i, name=R3] (8,4)
    (8,4) to [R=$R_2$] (8,0)
    (12,0) to [I, name=Ig] (12,4)
    (12,0) to [short,-*] (8,0)
    to [short,-*] (4,0)
    to [short] (0,0); 
    
    \varrmore{Ug}{$U_\mathrm{g}$};
    \iarrmore{Ig}{$I_\mathrm{g}$};
    \iarrmore{R3}{$I_\mathrm{3}$};
\end{circuitikz}
                }{{\bf Beispiel.} Netzwerkanalyse mit Kirchhoffschen Regeln. \label{BildBeispielaufgabeSuperposition}}
                    }
			\only<2>{			
				\item[a)] Angabe der Ströme und Spannungen:
                \begin{figure}[H]
                    \resizebox{0.65\textwidth}{!}{%   
                        \begin{circuitikz}
    \draw
    (0,2) to [V, name=Ug] (0,0)
    (0,2) to [R=$R_\mathrm{i}$, i, v, name=Ri] (0,4)
    (0,4) to [short,-*] (4,4)
    (12,4) to [short,-*] (8,4)
    (4,4) to [R=$R_1$, i, v, name=R1] (4,0)
    (4,4) to [R=$R_3$, i, v, name=R3] (8,4)
    (8,4) to [R=$R_2$, i, v, name=R2] (8,0)
    (12,0) to [I, name=Ig] (12,4)
    (12,0) to [short,-*] (8,0)
    to [short,-*] (4,0)
    to [short] (0,0); 
    \varrmore{Ug}{$U_\mathrm{g}$};
    \iarrmore{Ig}{$I_\mathrm{g}$};
    \varrmore{Ri}{$U_\mathrm{Ri}$};
    \iarrmore{Ri}{$I_\mathrm{U}$};
    \varrmore{R1}{$U_{1}$};
    \iarrmore{R1}{$I_{1}$};
    \varrmore{R2}{$U_{2}$};
    \iarrmore{R2}{$I_{2}$};
    \varrmore{R3}{$U_{3}$};
    \iarrmore{R3}{$I_{3}$};
\end{circuitikz}
                    }\centering                                                           
                \end{figure}
                    }    
			\only<3>{
				\item[b)] Netzwerkberechnung $I_3(U_\mathrm{g})$ für $U_\mathrm{g}$:
				\onslide<3>{
                    \begin{figure}[H]
                        \resizebox{0.65\textwidth}{!}{%  
                        \begin{circuitikz}
    \draw
    (0,2) to [V, name=Ug] (0,0)
    (0,2) to [R=$R_\mathrm{i}$, i, v, name=Ri] (0,4)
    (0,4) to [short,-*] (4,4)
    (12,4) to [short,-*] (8,4)
    (4,4) to [R=$R_1$, i, v, name=R1] (4,0)
    (4,4) to [R=$R_3$, i, v, name=R3] (8,4)
    (8,4) to [R=$R_2$, i, v, name=R2] (8,0)
    (12,0) to [short,-*] (8,0)
    to [short,-*] (4,0)
    to [short] (0,0); 
    \varrmore{Ug}{$U_\mathrm{g}$};
    \varrmore{Ri}{$U_\mathrm{Ri}$};
    \iarrmore{Ri}{$I_\mathrm{U}$};
    \varrmore{R1}{$U_{1}$};
    \iarrmore{R1}{$I_{1}$};
    \varrmore{R2}{$U_{2}$};
    \iarrmore{R2}{$I_{2}$};
    \varrmore{R3}{$U_{3}$};
    \iarrmore{R3}{$I_{3}$};
    \draw (4,4) node[above] {$K_1$};
\end{circuitikz}
                        }\centering                                                           
                    \end{figure}
					\begin{eqa}
						K_1: I_\mathrm{ges}&=I_1+I_3 \nonumber\\                        
                        I_\mathrm{ges}&=\frac{U_\mathrm{g}}{R_\mathrm{ges}}=\frac{U_\mathrm{g}}{R_\mathrm{i}+\frac{R_1\cdot(R_2+R_3)}{R_1+R_2+R_3}} \nonumber\\
                        \frac{I_3}{I_\mathrm{ges}}&=\frac{R_1}{R_1+R_2+R_3} \nonumber\\
                        I_3(U_\mathrm{g})&=I_\mathrm{ges}\cdot\frac{R_1}{R_1+R_2+R_3}=\frac{U_\mathrm{g}}{R_\mathrm{i}+\frac{R_1\cdot(R_2+R_3)}{R_1+R_2+R_3}}\cdot\frac{R_1}{R_1+R_2+R_3}\nonumber
					\end{eqa}
				}
			}
			\only<4>{
				\item[c)] Netzwerkberechnung $I_3(I_\mathrm{g})$ für $I_\mathrm{g}$:
				\onslide<4>{
                    \begin{figure}[H]
                        \resizebox{0.65\textwidth}{!}{%  
                        \begin{circuitikz}
    \draw
    (0,2) to [short] (0,0)
    (0,2) to [R=$R_\mathrm{i}$, i, v, name=Ri] (0,4)
    (0,4) to [short,-*] (4,4)
    (12,4) to [short,-*] (8,4)
    (4,4) to [R=$R_1$, i, v, name=R1] (4,0)
    (4,4) to [R=$R_3$, i, v, name=R3] (8,4)
    (8,4) to [R=$R_2$, i, v, name=R2] (8,0)
    (12,0) to [I, name=Ig] (12,4)
    (12,0) to [short,-*] (8,0)
    to [short,-*] (4,0)
    to [short] (0,0); 
    \iarrmore{Ig}{$I_\mathrm{g}$};
    \varrmore{Ri}{$U_\mathrm{Ri}$};
    \iarrmore{Ri}{$I_\mathrm{U}$};
    \varrmore{R1}{$U_{1}$};
    \iarrmore{R1}{$I_{1}$};
    \varrmore{R2}{$U_{2}$};
    \iarrmore{R2}{$I_{2}$};
    \varrmore{R3}{$U_{3}$};
    \iarrmore{R3}{$I_{3}$};
    \draw (4,4) node[above] {$K_1$};
    \draw (8,4) node[above] {$K_2$};
\end{circuitikz}
                        }\centering                                                           
                    \end{figure}
					\begin{eqa}
						K_2: I_\mathrm{ges}&=I_\mathrm{g}=I_2-I_3 \nonumber\\                        
                        \frac{-I_3}{I_\mathrm{ges}}&=\frac{R_2}{(R_1||R_\mathrm{i})+R_2+R_3} \nonumber\\
                        I_3(I_\mathrm{g})&=-I_\mathrm{g}\cdot\frac{R_2}{(R_1||R_\mathrm{i})+R_2+R_3}\nonumber
				    \end{eqa}
				}
			}
            \only<5>{
				\item[d)] Wie groß ist der Strom $I_3(U_\mathrm{g},I_\mathrm{g})$? \\ 

				\onslide<5>{
                    Superposition: \\
					\begin{eqa}
						I_3(U_\mathrm{g},I_\mathrm{g}) &= I_3(U_\mathrm{g})+I_3(I_\mathrm{g}) \nonumber\\
                        I_3(U_\mathrm{g},I_\mathrm{g}) &= \frac{U_\mathrm{g}}{R_\mathrm{i}+\frac{R_1\cdot(R_2+R_3)}{R_1+R_2+R_3}}\cdot\frac{R_1}{R_1+R_2+R_3}+(-I_\mathrm{g}\cdot\frac{R_2}{(R_1||R_\mathrm{i})+R_2+R_3}) \nonumber
				    \end{eqa}
				}
			}
		\end{enumerate}

    \end{bsp}

\end{frame}