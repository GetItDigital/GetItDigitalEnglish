\subsection{Knotenpotentialverfahren 1}
\Aufgabe{
    Gegeben ist das Netzwerk einer realen Spannungsquelle $U_\mathrm{q} = 3,6\ V$ mit einem Innenwiderstand
    vo $R_\mathrm{i} = 0,2\ \Omega$.

    \begin{figure}[H]
        \resizebox{0.3\textwidth}{!}{%  
        \begin{circuitikz}
            \draw (8,0) [V^<, name=Uq] to (8,3);
            \draw (8,3) to [R, l=$R_\mathrm{i}$, v, name=Ri, -o] (12,3);
            \draw (8,0) to [short, -o] (12,0);
            
            \varrmore{Uq}{$U_\mathrm{q}$};
            \varrmore{Ri}{$U_\mathrm{Ri}$};
            \end{circuitikz}   
        }\centering                                                           
    \end{figure}

    Wandeln Sie die angegebene reale Spannungsquelle in eine reale Stromquelle um.
    Befolgen Sie hierzu die folgenden Schritte:
        
    \begin{itemize}
            \item[\bf a)]
                Bestimmen Sie den Leitwert des Widerstandwertes. 
            \item[\bf b)]
                Wandeln Sie die reale Spannungsquelle in eine reale Stromquelle um. 
            \item[\bf c)]
                Berechnen Sie den Stromwert der umgewandelten Stromquelle.
        \end{itemize}
    }

\Loesung{
	\begin{itemize}
        \item[\bf a)]
            Leitwert:    \\
            \begin{eq}
                G_\mathrm{i} = \frac{1}{R_\mathrm{i}} = \frac{1}{0,2\ \Omega} = 5\ \mathrm{S}   \nonumber
            \end{eq}

		\item[\bf b)]
		    Stromquelle:    \\
            \begin{figure}[H]
                \resizebox{0.3\textwidth}{!}{%  
                \begin{circuitikz}                                       
                    \draw (0,0) [I, name=Iq] to (0,3);
                    \draw (0,3) to [short, -*] (2,3)
                        to [R=$R_\mathrm{i}$, -*] (2,0)
                        to (0,0);
                    \draw (2,0) to [short, -o] (4,0);
                    \draw (2,3) to [short, -o] (4,3);
                    \iarrmore{Iq}{$I_\mathrm{q}$};
                \end{circuitikz}   
                }\centering                                                           
            \end{figure}

        \item[\bf c)]
            Stromwert: \\
            \begin{eq}
                I_\mathrm{q} = U_\mathrm{q} \cdot G_\mathrm{i} = 3,6\ \mathrm{V} \cdot 5\ \mathrm{S} = 72\ \mathrm{A}     \nonumber
            \end{eq}

	\end{itemize}
}