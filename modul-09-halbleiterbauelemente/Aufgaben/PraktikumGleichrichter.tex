\subsection{Untersuchung eines einfachen Gleichrichters\label{Aufg31}}
% Alt: Praktikumsaufgabe 3
\Aufgabe{

    Analysieren Sie den folgenden Gleichrichter. Zusätzlich sind die folgenden Parameter gegeben:
    
    $\hat{u}_\mathrm{E} = 5\,\mathrm{V}$,
    $R_\mathrm{L} = 1\,\mathrm{k\Omega}$
    und eine Diodendurchlassspannung von $U_\mathrm{f} = 0,7\,\mathrm{V}$.
    
    \begin{figure}[H]
        \centering
        \begin{tikzpicture}
    \draw (4,0)
    to[short, -*] (0,0) node[ground]{}
    to[sV,name=UE, v<] (0,2)
    to[D, l=$D_\mathrm{1}$] (4,2) 
    to[R, l=$R_\mathrm{L}$] (4,0);

    \draw (4,2)
    to[short, *-o] (5.5,2)
    to[open,v^,name=UA] (5.5,0)
    to[short, o-*] (4,0);

    \varrmore{UE}{$u_\mathrm{E}(t)$};
    \varrmore{UA}{$U_\mathrm{A}$};
\end{tikzpicture}
        \label{fig:FigPraktikumGleichrichter}
    \end{figure}
    
    \begin{itemize}
        \item[a)] Berechnen Sie die Ausgangsspannung $U_\mathrm{A}$.
        \item[b)] Bestimmen Sie den durch den Lastwiderstand $R_\mathrm{L}$ fließenden Strom $I_\mathrm{L}$.
    \end{itemize}
}


\Loesung{
    \begin{itemize}
        \item[a)] Berechnung der Ausgangsspannung $U_\mathrm{A}$ \\[0.2cm]
              Die gegebene Schaltung besteht aus einer Diode in Vorwärtsrichtung und einem Lastwiderstand.  
              Die Ausgangsspannung $U_\mathrm{A}$ entspricht der Eingangsspannung abzüglich der Diodendurchlassspannung:
              \[
                  U_\mathrm{A} = U_\mathrm{E} - U_\mathrm{f}
              \]
              Einsetzen der Werte:
              \[
                  U_\mathrm{A} = 5\,\mathrm{V} - 0{,}7\,\mathrm{V} = 4{,}3\,\mathrm{V}
              \]
              
        \item[b)] Bestimmung des Laststroms $I_\mathrm{L}$ \\[0.2cm]
              Der durch den Lastwiderstand $R_\mathrm{L}$ fließende Strom berechnet sich nach dem Ohmschen Gesetz:
              \[
                  I_\mathrm{L} = \frac{U_\mathrm{A}}{R_\mathrm{L}}
              \]
              Einsetzen der Werte:
              \[
                  I_\mathrm{L} = \frac{4{,}3\,\mathrm{V}}{1\,\mathrm{k\Omega}} = 4{,}3\,\mathrm{mA}
              \]
    \end{itemize}
    Siehe: Abschnitt \ref{sec:Einweggleichrichter}
}
