% Addition zweier Schwingungen gleicher Frequenz
% Zeitverlauf- und Zeigerdiagramm
\def\amplitudeA{0.75}% amplitudeA*sin(wt+phiA)
\def\phiA{0}% in degrees
\def\amplitudeB{1}% amplitudeB*sin(wt+phiB)
\def\phiB{+90}% in degrees
\def\anglewatched{0}% clockwise (set to 0), counter clockwise (set to 360) [with phiA=0, phiB=-90]
\begin{tikzpicture}[x=1cm,y=1cm]
    \begin{axis}[%
        name=Zeitverlaufdiagramm,
        axis lines=center,
        xlabel=$\omega_d t$,
        ylabel=$\ecv{u(t)}$,
        xlabel style={xshift=0.75cm, yshift=-0.25cm},% set after(!) axis lines=center
        xmin=-0.375, xmax=2.125,
        ymin=-1.75, ymax=1.75,
        width=10cm, height=6cm,
        xtick distance=0.5, minor x tick num=1, xticklabels={,$0$,$\pi$,$2\pi$,$3\pi$,$4\pi$}, xticklabel style={yshift=-1.75cm},
        ytick distance=1.0, minor y tick num=1, yticklabels={\empty},
        grid=both, 
        grid style={line width=.1pt, draw=gray!10}, major grid style={line width=.2pt,draw=gray!50},
        domain=-0.375:2.125,
        samples=100,
        clip=false,
    ]
        % draw plots
        \addplot[thick,blue, domain=-0.375:2.075]    {\amplitudeA*cos(x*360+\phiA)}; % Schwingung A
        \addplot[thick,red, domain=-0.375:2.075 ]    {\amplitudeB*cos(x*360+\phiB)}; % Schwingung B
        \addplot[thick,GETgreen, domain=-0.375:2.075]  {\amplitudeA*cos(x*360+\phiA) + \amplitudeB*cos(x*360+\phiB)}; % Schwingung C = A + B
        
        % draw vertical line as marker (watched angle für Imaginärteil(!) zum Vergleichen)
        %\addplot[mark=*,GETgreen] coordinates {({(\anglewatched-90)/360},{ \amplitudeA*cos(\anglewatched-90+\phiA) + \amplitudeB*cos(\anglewatched-90+\phiB) })}; % green anglewatch marker, on height of Im{z0}

        % Phasenverschiebung
        \draw[|->|,red] ({0+\phiA/360},{+0.075}) -- ({0-\phiB/360},{+0.075}) node[midway, above]{};
        \draw[|->|,GETgreen] ({0+\phiA/360},{-0.075}) -- ({0-atan( (\amplitudeA*sin(\phiA)+\amplitudeB*sin(\phiB)) / (\amplitudeA*cos(\phiA)+\amplitudeB*cos(\phiB)) )/360},{-0.075}) node[midway, below]{$\ \psi$};
        \draw[dotted, red] ({0-\phiB/360},{+0.075}) -- ({0-\phiB/360},{\amplitudeB});
        \draw[dotted, GETgreen] ({0-atan( (\amplitudeA*sin(\phiA)+\amplitudeB*sin(\phiB)) / (\amplitudeA*cos(\phiA)+\amplitudeB*cos(\phiB)) )/360},{-0.075}) 
            -- ({0-atan( (\amplitudeA*sin(\phiA)+\amplitudeB*sin(\phiB)) / (\amplitudeA*cos(\phiA)+\amplitudeB*cos(\phiB)) )/360},{sqrt( (\amplitudeA*cos(\phiA)+\amplitudeB*cos(\phiB))^2+(\amplitudeA*sin(\phiA)+\amplitudeB*sin(\phiB))^2)});

        % Beschriftung
        \draw[thick, blue] ({1-\phiA/360+0.05},{\amplitudeA}) node[above]{${z_1}$};
        \draw[thick, red ] ({1-\phiB/360-0.05},{\amplitudeB}) node[above]{${z_2}$};
        \draw[thick, GETgreen] ({1-atan( (\amplitudeA*sin(\phiA)+\amplitudeB*sin(\phiB)) / (\amplitudeA*cos(\phiA)+\amplitudeB*cos(\phiB)) )/360}, {sqrt( (\amplitudeA*cos(\phiA)+\amplitudeB*cos(\phiB))^2+(\amplitudeA*sin(\phiA)+\amplitudeB*sin(\phiB))^2)}) node[above]{$z_0$}; 

        % Amplitudenlinien 
        \draw[dotted, blue](0,{-\amplitudeA}) -- (3,{-\amplitudeA});
        \draw[dotted, red] (0,{-\amplitudeB}) -- (3,{-\amplitudeB});
        \draw[dotted, GETgreen] (0, -{sqrt( (\amplitudeA*cos(\phiA)+\amplitudeB*cos(\phiB))^2+(\amplitudeA*sin(\phiA)+\amplitudeB*sin(\phiB))^2)}) -- (3, -{sqrt( (\amplitudeA*cos(\phiA)+\amplitudeB*cos(\phiB))^2+(\amplitudeA*sin(\phiA)+\amplitudeB*sin(\phiB))^2)});

    \end{axis}
    \begin{axis}[
        at={($(Zeitverlaufdiagramm.south east)+(0.75cm,0cm)$)},
        name=Zeigerdiagramm,
        axis lines=center,
        %x dir=reverse, % x-Achse umdrehen ... dann Koordinaten alle ebenfalls ändern und alles 90° drehen für quasi Re-Achse nach oben, Im-Achse nach links
        xmin=-1.75,xmax=+1.75,
        ymin=-1.75,ymax=+1.75,
        width=6cm, height=6cm,
        xtick distance=1, xticklabels={\empty}, minor x tick num=1,
        ytick distance=1, yticklabels={\empty}, minor y tick num=1,
        grid=none,% none, minor, major, both
        grid style={line width=.1pt, draw=gray!10},major grid style={line width=.2pt,draw=gray!50},
        clip=false,
    ]
        % Kreise
        \draw[fill=none, thick, blue]  (0,0) circle [radius=\amplitudeA];
        \draw[fill=none, thick, red]   (0,0) circle [radius=\amplitudeB];
        \draw[fill=none, thick, GETgreen] (0,0) circle [radius={sqrt( (\amplitudeA*cos(\phiA)+\amplitudeB*cos(\phiB))^2+(\amplitudeA*sin(\phiA)+\amplitudeB*sin(\phiB))^2)}];
        
        % Zeiger
        \draw[->,thick,blue]    (0,0) -- 
                                ({\amplitudeA*cos(\phiA+\anglewatched)},{\amplitudeA*sin(\phiA+\anglewatched)});
        \draw[->,thick,red]     (0,0) -- 
                                ({\amplitudeB*cos(\phiB+\anglewatched)},{\amplitudeB*sin(\phiB+\anglewatched)});
        \draw[->,thick,red,opacity=0.5]% parallel, start from tip of blue arrow to form a triangle     
                                ({\amplitudeA*cos(\phiA+\anglewatched)},{\amplitudeA*sin(\phiA+\anglewatched)}) -- 
                                ({\amplitudeA*cos(\phiA+\anglewatched)+\amplitudeB*cos(\phiB+\anglewatched)},
                                 {\amplitudeB*sin(\phiB+\anglewatched)+\amplitudeA*sin(\phiA+\anglewatched)});
        \draw[->,thick, blue,opacity=0.5]% parallel, start from tip of red arrow to form a triangle
                                ({\amplitudeB*cos(\phiB+\anglewatched)},{\amplitudeB*sin(\phiB+\anglewatched)}) -- 
                                ({\amplitudeA*cos(\phiA+\anglewatched)+\amplitudeB*cos(\phiB+\anglewatched)},
                                 {\amplitudeB*sin(\phiB+\anglewatched)+\amplitudeA*sin(\phiA+\anglewatched)});
        \draw[->,thick,GETgreen](0,0) -- 
                                ({\amplitudeA*cos(\phiA+\anglewatched)+\amplitudeB*cos(\phiB+\anglewatched)},
                                {\amplitudeB*sin(\phiB+\anglewatched)+\amplitudeA*sin(\phiA+\anglewatched)});

        % Winkel
        \draw[->,black] (30:{1.1*sqrt( (\amplitudeA*cos(\phiA)+\amplitudeB*cos(\phiB))^2+(\amplitudeA*sin(\phiA)+\amplitudeB*sin(\phiB))^2)}) % (start angle:radius), radius=1.1*amplitudeC
            arc [start angle=30, end angle=60, radius={1.1*sqrt( (\amplitudeA*cos(\phiA)+\amplitudeB*cos(\phiB))^2+(\amplitudeA*sin(\phiA)+\amplitudeB*sin(\phiB))^2)}] % Orientation (counter clockwise)
            node[midway, above, yshift=-4pt, sloped] {$\omega t$} % working
        ;
        \draw[->,red]       ({\anglewatched}:0.40) arc [start angle={\anglewatched}, end angle={\anglewatched+\phiB}, radius=0.40]% from 0 degree to phiB
            %node[midway, above, yshift=-4pt, sloped] {$\varphi_2$}% wrong radius ... ?
        ;
        \draw[->,GETgreen] ({\anglewatched}:0.55) arc [start angle={\anglewatched}, end angle={(\anglewatched + atan( (\amplitudeA*sin(\phiA)+\amplitudeB*sin(\phiB)) / (\amplitudeA*cos(\phiA)+\amplitudeB*cos(\phiB)) ) )}, radius=0.55];

        %\draw[->,blue]    ({\anglewatched}:0.25) arc [start angle={\anglewatched}, end angle={(\anglewatched + \phiA)}, radius=0.25]; % für \phiA = 0 unpraktisch! (nur Spitze, kein Bogen)

        % Beschriftung
        \draw (-1.5,1) node[align=left]{%
            $\ec[GETgreen]{\underline{Z}_0}$\\
            $\ec[blue]{\underline{Z}_1}$\\
            $\ec[red]{\underline{Z}_2}$%
        };
        
    \end{axis}
\end{tikzpicture}
