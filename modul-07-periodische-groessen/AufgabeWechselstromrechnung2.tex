\subsection{Komplexe Wechselstromrechnung 2}
\Aufgabe{
    
    Eine Impedanz weist bei der Frequenz $f = 1\ kHz$ den Wert $\underline{Z} = (300+\mathrm{j}400)\ \Omega$ auf. 

    \begin{itemize}
        \item[\bf a)] Bestimmen Sie den Scheinwiderstand (Betrag) der Impedanz.
        \item[\bf b)] Ermitteln Sie die Admittanz $\underline{Y} = \frac{1}{\underline{Z}}$ sowie den Scheinleitwert. 
        \item[\bf c)] Geben Sie zur Realisierung der Impedanz $\underline{Z}$ eine Schaltung aus zwei in Reihe geschalteten 
            Bauelementen an und dimensionieren Sie diese Schaltung.
        \item[\bf d)] Geben Sie zur Realisierung der Impedanz $\underline{Z}$ eine Schaltung aus zwei parallel geschalteten 
            Bauelementen an und dimensionieren Sie diese Schaltung.  
    \end{itemize}

    Außerdem wird an die Impedanz $\underline{Z} = (300+\mathrm{j}400)\ \Omega$ eine Spannung 
    $u(t)=\hat{U} \cdot \cos(2\pi f t + \varphi_U)$ mit $\hat{U}=5\ V$, $\varphi_U=\pi/4$ und $f=1\ kHz$ gelegt. 

    \begin{itemize}
        \item[\bf e)] Ermitteln Sie die komplexe Amplitude $\underline{\hat{U}}$ der anliegenden Spannung.
        \item[\bf f)] Berechnen Sie die komplexe Amplitude $\underline{\hat{I}}$ des Stromes durch die Impedanz. 
        \item[\bf g)] Geben Sie den zeitlichen Verlauf des Stromes $i(t)$ an.
        \item[\bf h)] Stellen Sie die Impedanz in Polarform dar und geben Sie den Winkel $\varphi = arg\{\underline{Z}\}$
            an. Vergleichen Sie den Winkel der Impedanz mit der Phasenverschiebung zwischen Spannung und Strom. Was stellen
            Sie fest? 
    \end{itemize}
}


\Loesung{
	\begin{itemize}

	\item[\bf a)]
        \begin{eqa}
            |\underline{Z}| &= \sqrt{Re^2+Im^2} \nonumber \\  
             &= \sqrt{300^2+400^2}\ \Omega \nonumber \\ 
             &= 500\ \Omega     \nonumber
        \end{eqa}
	
	\item[\bf b)]
    \begin{eqa}
        \underline{Y} &= \frac{1}{\underline{Z}} = \frac{1}{(300+\mathrm{j}400)\ \Omega} \nonumber \\  
         &= \frac{1}{(300+\mathrm{j}400)\ \Omega} \cdot \frac{(300-\mathrm{j}400)\ \cancel{\Omega}}{(300-\mathrm{j}400)\ \cancel{\Omega}} \nonumber \\ 
         &= \frac{1}{300^2+400^2} \cdot (300-\mathrm{j}400)\ S     \nonumber   \\
         &= (0.0012-\mathrm{j}0.0016)\ S     \nonumber     \\
        |\underline{Y}| &= \frac{1}{|\underline{Z}|} \nonumber      \\
         &= \frac{1}{500}\ S     \nonumber
    \end{eqa}
	
    \item[\bf c)] 
        Reihenschaltung: \\
        Realteil $\rightarrow$ Ohmscher Wiederstand \\
        Imaginärteil $\rightarrow$ Spule oder Kondensator? \\
        Reaktanz Spule: $X_L=2\pi f \cdot L > 0$       \\
        Reaktanz Kondensator: $X_C=-\frac{1}{2\pi f\cdot C}  < 0$   \\

        \begin{figure}[H]
            \centering
            \resizebox{0.35\textwidth}{!}{
            \begin{circuitikz}
                \draw(0,0) to [R=${R}$, o-] (2,0);
                \draw(2,0) to [L=${L}$, -o] (4,0);
            \end{circuitikz}
            } 
        \end{figure} 

        \begin{eqa}
            R &= 300\ \Omega     \nonumber   \\
            L &= \frac{X_L}{2\pi \cdot f} \nonumber  \\
             &= \frac{400\ \Omega}{2\pi \cdot 1\ kHz}  \nonumber   \\
             &= 63.66\ mH  \nonumber   \\
            \underline{Z} &= R+\mathrm{j}X_L = R+\mathrm{j}\ 2\pi \cdot f \cdot L        \nonumber
        \end{eqa}
    
	\item[\bf d)]
        Parallelschaltung: \\
        $\rightarrow$ Addition von Leitwerten:\\

        \begin{eq}
            \underline{Y} = G + \mathrm{j}B = (0.0012-\mathrm{j}0.0016)\ S    \qquad  (\text{aus b)})      \nonumber
        \end{eq}

        $B<0 \rightarrow$   Spule, induktiver Blindleitwert \\
        $B>0 \rightarrow$   Kondensator, kapazitiver Blindleitwert  \\

        \begin{figure}[H]
            \centering
            \resizebox{0.5\textwidth}{!}{
            \begin{circuitikz}
                \draw(0,0) to [short,o-*] (2,0);
                \draw(2,-1) to [short] (2,1);
                \draw(2,1) to [R=${R}$] (4,1);
                \draw(2,-1) to [L=${L}$] (4,-1);
                \draw(4,-1) to [short] (4,1);
                \draw(4,0) to [short,*-o] (6,0);
            \end{circuitikz}
            } 
        \end{figure} 

        \begin{eqa}
            R &= \frac{1}{G}    \nonumber   \\
             &= \frac{1}{0.0012\ S}    \nonumber   \\
             &= 833\ \Omega     \nonumber   \\
            B &= -\frac{1}{2\pi \cdot f \cdot L} \nonumber  \\
            L &= \frac{1}{2\pi \cdot f \cdot B}  \nonumber   \\
             &= 99.47\ mH  \nonumber   
        \end{eqa}
	
    \item[\bf e)] 
        \begin{eqa}
            \underline{U} &= \hat{U} \cdot \mathrm{e}^{\mathrm{j} \varphi_\mathrm{U}}  = 5\ V \cdot \mathrm{e}^{\mathrm{j} \pi/4} = 5\ V \cdot \mathrm{e}^{\mathrm{j} 45^\circ}    \nonumber   \\
            &= 5\ V \cdot (\cos(45^\circ)+\mathrm{j}\sin(45^\circ))    \nonumber   \\
            &= 5\ V \cdot \frac{\sqrt{2}}{2} (1+\mathrm{j}) \nonumber   \\
            &\approx (3,53 + \mathrm{j} 3,53) \ V    \nonumber
        \end{eqa}
    
	\item[\bf f)]
	
    \item[\bf g)] 
    
    \item[\bf h)] 

\end{itemize}

}