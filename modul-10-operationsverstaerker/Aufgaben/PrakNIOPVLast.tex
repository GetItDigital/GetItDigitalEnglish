\subsection{Analyse eines nicht-invertierenden Verstärkers mit Last\label{Aufg25}}
% Alt: Praktikumsaufgabe 11
\Aufgabe{
    Analysieren Sie die Verstärkung eines nicht-invertierenden Operationsverstärkers und berechnen Sie die Ausgangsspannung $U_\mathrm{A}$.
    Gegeben:
    \begin{itemize}
        \item Versorgungsspannung: $\pm 15\,\mathrm{V}$
        \item Eingangsspannung: $U_\mathrm{E} = 1\,\mathrm{V}$
        \item Rückkopplungswiderstand: $R_\mathrm{R} = 20\,\mathrm{k\Omega}$
        \item Eingangswiderstand: $R_\mathrm{E} = 10\,\mathrm{k\Omega}$
        \item Lastwiderstand: $R_\mathrm{Last} = 10\,\mathrm{k\Omega}$
    \end{itemize}
    
    \begin{itemize}
        \item[a)] Berechnen Sie die Ausgangsspannung $U_\mathrm{A}$ mit den gegebenen Werten.
        \item[b)] Wie verändert sich $U_\mathrm{A}$, wenn $R_\mathrm{R}$ auf $30\,\mathrm{k\Omega}$ erhöht wird?
        \item[c)] Welchen Einfluss hat der Lastwiderstand $R_\mathrm{Last}$ auf das Ausgangssignal?
    \end{itemize}
    \begin {figure} [H]
    \centering
    \begin{tikzpicture}
    \draw
    (0,0) node[op amp, anchor=+] (opamp) {}
    (opamp.+) 
    (opamp.-) 
    (opamp.out) ;
    \draw
        (opamp.up) -- ++(0,0.5) node[right] {$+$}; % Text leicht rechts
        \draw
        (opamp.down) -- ++(0,-0.5) node[right] {$-$}; % Text leicht rechts
    
    \draw (opamp.out) to[short, -*] (4,0.5)to[R,l=$R_\mathrm{Last}$] (4,-1.5) to[short,-*] (-1,-1.5)node[ground]{};
    
    \draw (-1,-1.5) -- (-3.5,-1.5);
    
\draw (opamp.-) to[short,-*] (0,3) -- (1,3) to[R,l=$R_\mathrm{R}$](4,3)--(4,0.5);
    \draw (-2,2)node[ground]{} -- (-2,3) to[R,l=$R_\mathrm{E1}$] (0,3);
    
    \draw (-3.8,0) to[R,l=$R_\mathrm{3}$] (opamp.+);
    
    \draw (opamp.up) -- (1.1,4) to[short, -*] (-4,4) -- (-6,4) to[V,name=V1] (-6,2)node[ground]{};
    
    \draw (opamp.down) -- (1.1,-3) to[short,-*](-4,-3) -- (-6,-3) to[V,name=V2] (-6,-5)node[ground]{};
    \draw (-4,-3) to[R,l=$R_\mathrm{2}$] (-4,-1) to[potentiometer, l=$P_\mathrm{1}$] (-4,1) to [R,l=$R_\mathrm{1}$](-4,4);
    
    \draw (-3.1,0) to[open,v,name=ue](-3.1,-1.5);
    \draw (3,0.5) to[open,v,name=ua](3,-1.5);
    
    \varrmore{ua}{$U_\mathrm{A}$};
    \varrmore{ue}{$U_\mathrm{E}$};
    \varrmore{V1}{$U_\mathrm{1}$};
    \varrmore{V2}{$U_\mathrm{2}$};
\end{tikzpicture}
   
    \label{fig:FigPrakNIOPVLast}
    \end {figure}
}

\Loesung{
    \textbf{a) Berechnung der Ausgangsspannung $\mathrm{U_A}$} \\
    Gegeben sind:
    \[
    U_\mathrm{E} = 1\,\mathrm{V}, \quad R_\mathrm{R} = 20\,\mathrm{k\Omega}, \quad R_\mathrm{E} = 10\,\mathrm{k\Omega}
    \]
    Einsetzen in die Verstärkungsformel:
    \[
    V = 1 + \frac{20\,\mathrm{k\Omega}}{10\,\mathrm{k\Omega}} = 1 + 2 = 3
    \]
    Die Ausgangsspannung ergibt sich zu:
    \[
    U_\mathrm{A} = V \cdot U_\mathrm{E} = 3 \cdot 1\,\mathrm{V} = 3\,\mathrm{V}
    \]

    \textbf{b) Veränderung von $\mathrm{R_R}$} \\[0.2cm]
    Wenn $R_\mathrm{R}$ auf $30\,\mathrm{k\Omega}$ erhöht wird:
    \[
    V = 1 + \frac{30\,\mathrm{k\Omega}}{10\,\mathrm{k\Omega}} = 1 + 3 = 4
    \]
    \[
    U_\mathrm{A} = 4 \cdot 1\,\mathrm{V} = 4\,\mathrm{V}
    \]

    \textbf{c) Einfluss des Lastwiderstands} \\[0.2cm]
    Der Lastwiderstand $R_\mathrm{Last}$ beeinflusst die Schaltung nur minimal, solange der Operationsverstärker im linearen Bereich arbeitet.  
    Falls der Lastwiderstand jedoch zu klein gewählt wird, kann der Operationsverstärker an seine Ausgangsstromgrenze kommen, was zu einer Spannungsbegrenzung führt. \\
    
    Siehe: Abschnitt \ref{fig: Verschaltung Verstaerker} und Tabelle \ref{tab:Grundschaltungen}
}