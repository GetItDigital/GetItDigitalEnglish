%Scheinleistung, Wirkleistung und Blindleistung
\begin{frame}
    \fta{Scheinleistung, Wirkleistung und Blindleistung}

    \s{
        In der Gleichstrombetrachtung elektrischer Netzwerke ergab sich lediglich eine Wirkleistung an den Komponenten. 
        In der Wechselstrombetrachtung elektrischer Netzwerke werden durch die resultierenden Blindströme auch Blindleistungen
        erzeugt. Die sich daraus ergebende Scheinleistung teilt sich auf die Wirkleistung und die Blindleistung auf. 
        Die folgenden Grundlagen und Leistungsarten, welche durch eine Wechselspannung hervorgerufen werden, werden folgend besprochen:
    }

    \begin{Lernziele}{Scheinleistung, Wirkleistung und Blindleistung}
        Die Studierenden
        \begin{itemize}
            \item kennen den Unterschied zwischen der Augenblicksleistung, der Wirkleistung und der Blindleistung.
            \item können die Scheinleistung bestimmen und kategorisieren.
        \end{itemize}
    \end{Lernziele}

\end{frame}

%Grundlagen


\begin{frame}
    \ftb{Grundlagen: Additionstheorem und Arithmetischer Mittelwert}

    \s{
        Für die Berechnung des Effektivwertes wurde bereits ein Additionstheorem vorgestellt. Ein weiteres Additionstheorem 
        wird in der Gleichung \ref{GleichungAdd1} vorgestellt. Hier wird aus dem Produkt zweier Sinusfunktionen mit unterschiedlichen
        Argumenten ein Zusammenhang mit Summen und Differenzen in Cosinusfunktionen. Hierdurch wird das Produkt eliminiert und eine 
        Summe zweier Cosinusfunktion erstellt. Das Integrieren der Summe lässt sich durch das Aufteilen der Summe in zwei separate
        Integrationvorgänge vereinfachen. 
    }

    \b{
        Additionstheorem zur Umwandlung eines Produkts aus zwei Sinusfunktionen und verschiedenen Argumenten:
    }

    \begin{eq}
        \sin(x) \cdot \sin(y) = \frac{1}{2} (\cos(x-y) + cos(x+y))          \label{GleichungAdd1}
    \end{eq}

    \s{
        Mit dem arithmetische Mittelwert einer Funktion soll ein konstanter Wert errechnet werden, welcher nicht mehr zeitabhängig 
        ist. In der Elektrotechnik soll dieser Wert in bestimmten Fällen, ähnlich wie der Effektivwert bei sinusförmigen Strom- und 
        Spannungsverläufen, zur besseren Veranschaulichung herangezogen werden. Berechnet wird der arithmetische Mittelwert über 
        folgende Integralformel:
    }

    \b{
        Die Grundlage der Leistungsberechnung von Wechselgrößen ist der arithmetische Mittelwert:
        \begin{itemize}
            \item Es soll ein konstanter Wert errechnet werden, welcher nicht mehr zeitabhängig ist
            \item Dient der besseren Veranschaulichung, ähnlich wie der Effektivwert
        \end{itemize}
    }

    \begin{eq}
        \overline{u(t)}=\frac{1}{T}\int_{t}^{t+T} u(t)dt \label{ArythMittel}
    \end{eq}

    \s{
        In der Formel steht $T$ für die Periodenlänge. Hier ist $t$ dabei ein beliebig wählbarer Zeitpunkt, ab welchem die Betrachtung 
        starten soll (meistens $t=0$). Vereinfacht ausgedrückt steht im Integral die Fläche zwischen der Funktionskurve und der Abszisse
        und im Nenner des Bruches die Periodenlänge. Als Beispiel soll eine sogenannte Sägezahnspannung herangezogen werden, welche in 
        der Abbildung \ref{BildLei1} dargestellt wird. 
    }

    \b{
        Mit:
        \begin{itemize}
            \item T=Periodenlänge
            \item t=belibig wählbarer Anfangszeitpunkt (meistens t=0)
        \end{itemize} 
    }

\end{frame}
    

\begin{frame}
    \ftx{Arithmetischer Mittelwert: Beispiel}

    \b{
        Beispiel der Mittelwertbildung anhand einer Sägezahnspannung:
        \begin{itemize}
            \item Zuerst wird eine Funktion über eine Periodenlänge aufgestellt
            \begin{eq}
                u(t)=\frac{10~V}{T}\cdot t
            \end{eq}
            \item<2-> Arithmetischer Mittelwert $\overline{u(t)}$:
        \end{itemize}
    }

    \fu{
        \resizebox{0.7\textwidth}{!}{
            \begin{tikzpicture}
    \draw[->] (0, 0) -- (0, 5);
    \draw[->] (0, 0) -- (10, 0);
    \node at (-0.5, 5.2) {$u(t)$};
    \node at (10.125, -0.25) {$t$};
    \foreach \x in {0, 2, 4, 6, 8}
        {
            \draw (\x,0) -- (\x,-0.2);
        }
    \foreach \x in {0, 1, 2, 3, 4}
        {
            \draw (0, \x) -- (-0.2, \x);
        }

    \draw [line width=1.5pt] (0, 0) -- (4, 4);
    \draw [line width=1.5pt] (4, 4) -- (4, 0);
    \draw (4, 0) -- (8, 4);
    \draw (8, 4) -- (8, 0);
    \draw (8, 0) -- (10, 2);

    \node at (4, -0.5) {$T$};
    \node at (8, -0.5) {$2T$};
    \node at (-0.5, 2) {$5V$};
    \node at (-0.5, 4) {$10V$};
    \pause

    % Mittelwert
    \draw[dashed, red] (0,2) -- (10,2);
    \draw[pattern=north east lines, pattern color=red!20, dashed] (0,0)--(0,2)--(2,2);
    \draw[pattern=north east lines, pattern color=red!20, dashed] (2,2)--(4,4)--(4,2);
    \node[red] at (5, 2.5) {$\overline{u(t)}$};
\end{tikzpicture}
        }
    }{{\bf Verlauf einer Sägezahnspannung.} Über die Sägezahnspannung wird die Ermittlung des rot gekennzeichneten 
    Mittelwertes erläutert. \label{BildLei1}}    

    \s{
        Soll der arythmetische Mittelwert errechnet werden, muss zuerst eine Funktion über eine Periode aufgestellt werden.
        In diesem Fall kann eine Periode folgendermaßen ausgedrückt werden:

        \begin{eq}
            \overline{u(t)}=\frac{10~V}{T}\cdot t
        \end{eq}

        Diesen Zusammenhang eingefügt in die Gleichung \ref{ArythMittel} und für die Integrationsgrenzen $t=0$ bis $T$ ergibt sich die 
        Gleichung:

        \begin{eq}            
            \overline{u(t)}=\frac{1}{T}\int_{0}^{T} \frac{10~V}{T}\cdot t~dt
        \end{eq}

        Wird das Integral aufgelöst erhält man den arythmetischen Mittelwert für die Sägezahnspannung:

        \begin{eq}            
            \overline{u(t)}=\frac{1}{T}\left[\frac{10~V}{T}\cdot\frac{t^2}{2}\right]_0^T=\frac{1}{2}\cdot 10~V
        \end{eq}

        Somit beträgt der arythmetische Mittelwert die Hälfte der maximalen Spannung. Liegt eine ideale Sinusschwingung vor, ist der 
        arythmetische Mittelwert immer null. Relevant wird der arythmetische Mittelwert, wenn die Leistung von sinusförmigen Strömen 
        und Spannungen errechnet werden soll.
    }

\end{frame}

\begin{frame}
    \ftx{Arithmetischer Mittelwert: Beispiel}
        
    \b{
        \begin{itemize}
            \item Diese Funktion wird in die Gleichung für den arithmetischen Mittelwert eingesetzt
        \end{itemize}

        \begin{eq}            
            \overline{u(t)}=\frac{1}{T}\int_{0}^{T} \frac{10~V}{T}\cdot t~dt
        \end{eq}
        
        \begin{itemize}
            \item Nach auflösen des Integrals erhält man den arithmetischen Mittelwert der Sägezahnspannung
        \end{itemize}

        \begin{eq}            
            \overline{u(t)}=\frac{1}{T}\left[\frac{10~V}{T}\cdot\frac{t^2}{2}\right]_0^T
            =\frac{1}{2}\cdot 10~V
        \end{eq}

        \begin{itemize}
            \item Der arithmetische Mittelwert bei reinen Wechselgrößen ist immer null
        \end{itemize}
    }

\end{frame}


\begin{frame}
    \ftb{Wirkleistung und Blindleistung}

    \s{
        Wirkleistung bei Gleichstrom:
        In elektrischen Gleichstromnetzwerken wird die Leistung nach der bekannten Gleichung $P=U\cdot I$ bestimmt.
        Bei bekanntem elektrischen Widerstand kann die Gleichung dahingehend umgestellt werden, dass nach dem ohmschen
        Gesetz entweder die Spannung oder der Stromeliminiert werden.  
    }

    \b{
        Wirkleistung bei Gleichstrom:
        \begin{itemize}
            \item Leistungsberechnung
        \end{itemize}
    }

    \begin{eq}
        P = U \cdot I = I^2 \cdot R = \frac{U^2}{R}     \label{GleichungLei1}
    \end{eq}
    
\end{frame}



\begin{frame}
    \ftx{Wirkleistung Widerstand}

    \s{
        In einem System mit Wechselgrößen ist die Leistung, wie auch die Größen Strom und Spannung, zeitabhängig.
        In der Gleichung \ref{GleichungLei2} wird dies dadurch ausgedrückt, dass die Formelzeichen klein geschrieben sind
        und abhängig von der Zeit $t$ sind:
    }

    \b{
        \begin{itemize}
            \item Zeitabhängige Wechselgrößen:
        \end{itemize}
    }

    \begin{eq}
        P = U \cdot I \quad \Leftrightarrow \quad p(t)=u(t)\cdot i(t)       \label{GleichungLei2}
    \end{eq}

    \s{
        Weil die Größen der Spannung und des Stromes zeitabhängig sind, ist auch diese Funktion der Leistung zeitabhängig.
        Aufgrund der Zeitabhängigkeit wird auch von der \textbf{Augenblicksleistung} für einen Zeitpunkt t gesprochen. 
        Die Maßeinheit der elektrischen Augenblicksleistung ist weiterhin das Watt.
    }   

    \b{
        \begin{itemize}
            \item Zeitabhängige Spannung und Ströme:
        \end{itemize}
    }

    \begin{eq}
        u(t)=\hat{U}\cdot \sin(\omega t + \varphi_\mathrm{U}) \qquad i(t)=\hat{I}\cdot \sin(\omega t + \varphi_\mathrm{I})      \label{GleichungLei3}
    \end{eq}

    \s{
        Werden die Augenblickswerte der Spannung und des Stromes in die Gleichung \ref{GleichungLei2} der Augenblicksleistung 
        eingesetzt, so ergibt sich in der folgenden Gleichung \ref{GleichungLei4} die Augenblicksleistung. Hier lässt sich die 
        Gleichung in einen Vorfaktor, einen konstanten Anteil und einen zeitabhängigen Anteil aufteilen. Der Vorfaktor besteht
        aus der Hälfte des Produktes der Amplitudenwerte von Spannung und Strom. Der konstante Anteil besteht aus dem ersten 
        Cosinussummanden, hier sind lediglich die Phasenwinkel von Spannung und Strom ausschlaggebend. Der zeitabhängige Anteil 
        wird durch den zweiten Cosinus dargestellt. Hier werden neben den Phasenwinkeln auch die Kreisfrequenz und der Zeitpunkt
        wichtig.  
    }

    \b{
        \begin{itemize}
            \item \textbf{Augenblicksleistung}:
        \end{itemize}
    }

    \begin{eqa}
        \text{allgemein:} \quad p(t) &= u(t) \cdot i(t)  \\
        \text{sinusförmig:} \quad p(t) &= \frac{\hat{U}\cdot\hat{I}}{2}(\underbrace{\cos(\varphi_\mathrm{U}-\varphi_\mathrm{I})}_\text{konstanter Anteil} + \underbrace{\cos(2\omega t +\varphi_\mathrm{U}+\varphi_\mathrm{I})}_\text{zeitabhängiger Anteil})    \label{GleichungLei4}
    \end{eqa}
    
\end{frame}


\begin{frame}
    \ftx{Augenblicksleistung}

    \begin{Merksatz}{Augenblicksleistung}
        Die Augenblicksleistung ist das Produkt aus den Aufgeblickswerten der Spannung und des Stromes. 

        \begin{eq}
            p(t)=u(t)\cdot i(t)      \nonumber
        \end{eq}

    \end{Merksatz}

\end{frame}

\begin{frame}
    \ftx{Mittlere Leistung}

    \s{
        Wird die gesamte Leistung eines Systems über einen idealen ohmschen Widerstand aufgenommen, bedeutet dies, dass die 
        Spannung in der Strom keine Phasenverschiebung aufweisen. Der Konstante Anteil der Augenblicksleistung ist positiv. 
        Die Augenblicksleistung pulsiert mit doppelter Frequenz im Vergleich zur Frequenz von Spannung und Strom. Dabei hat 
        die Leistung stets ein positives Vorzeichen. 

        Häufig ist es nicht erheblich, wie der genaue zeitliche Verlauf der Augenblicksleistung ist. Vielmehr ist bei Anwendungen 
        lediglich die im Mittel umgesetzte Leistung, beispielsweise Wärmeanwenungen, von Interesse. Die Mittlere Leistung der 
        zeitveränderlichen Augenblicksleistung ist nach Gleichung \ref{ArythMittel} definiert. 
        Für rein sinusförmige Spannungen und Ströme lässt sich eine mittlere Leistung nach Gleichung \ref{GleichungLei5}
        bestimmen. Die Angaben der Spannung $U$ und des Stromes $I$ bilden die Effektivwerte. Der Winkel $\varphi$ ergibt 
        sich aus der Phasenverschiebung zwischen der Spannung und dem Strom. Der gesamte Ausdruck  ${\bf\cos(\varphi)}$ wird auch
        als {\bf Leistungsfaktor} bezeichnet. 
    }

    \b{
        Mittlere Leistung:
        \begin{itemize}
            \item<1-> Effektivwert von Spannung $U$ und Strom $I$
            \item<2-> Phasenverschiebung zwischen Spannung und Strom
            \item<2-> \textbf{Leistungsfaktor}: $\cos(\varphi)$
        \end{itemize}
    }

    \begin{eq}
        \overline{p}_\mathrm{Sin} = P = U \cdot I \cdot \cos(\varphi)     \label{GleichungLei5}
    \end{eq}

    \s{
        Ohne die Phasenverschiebung zwischen der Spannung und dem Strom an einem ideale Widerstand ist der Term des 
        Leistungsfaktors Vernachlässigbar. So lässt sich die mittlere Leistung in der Wechselstromtechnik bei sinusförmigen
        Spannungs- und Stromverläufen analog zur Gleichstromtechnik auf zwei zeitunabhängige Größen zurückführen. Der 
        Effektivwert der Spannung und des Stromes werden für rein sinusförmige Größen zur Ermittlung der mittelren Leistung
        herangezogen.

        \begin{eq}
            \overline{p}_\mathrm{Ohm} = \frac{\hat{U}}{\sqrt{2}} \cdot \frac{\hat{I}}{\sqrt{2}} = \frac{\hat{U}\cdot\hat{I}}{2} = U \cdot I     \label{GleichungLei6}
        \end{eq} 
    }

    \b{
        \begin{itemize}
            \item<3-> Mittlere Lesitung an einem idealen Wirderstand ohne Phasenverschiebung zwischen Spannung und Strom:
                \begin{eq}
                    \overline{p}_\mathrm{Ohm} = \frac{\hat{U}}{\sqrt{2}} \cdot \frac{\hat{I}}{\sqrt{2}} = \frac{\hat{U}\cdot\hat{I}}{2} = U \cdot I   \nonumber
                \end{eq}
        \end{itemize}
    }


\end{frame}


\begin{frame}
    \ftx{Leistung Induktivität}
    
    \s{
        Ähnlich verhält es sich mit der Augenblicksleistung an Spulen und Kondensatoren. An einer Induktivität eilt der 
        Strom der Spannung um 90° nach ($+\pi/2$). Der Verlauf der Spannung und des Stromes an einer Induktivität wird in der Abbildung 
        \ref{BildInduktiveLast} dargestellt. Dazu wird in grün die Augenblicksleistung an der Induktivität dargestellt. Der 
        Konstante Anteil der Augenblicksleistung an einer Induktivität beträgt Null. An dem Verlauf lässt sich das Pulsieren 
        der Augenblicksspannung mit der doppelten Frequenz der Spannung bzw. des Stromes feststellen. Die positiven und 
        negativen Halbwellen der Augenblicksleistung heben sich gegenseitig im Mittel auf.  Das heißt, dass die aufgenommene 
        Leistung wieder vollständig abgegeben wird und so von der Induktivität keine Leistung aufgenommen wird. 
    }

    \b{
        Augenblicksleistung an einer Induktivität:
        \begin{itemize}
            \item<1-> Spannung und Strom um $+\pi/2$ phasenverschoben
            \item<1-> Gleichanteil gleich null
            \item<2-> Augenblicksleistung pulsiert mit doppelter Frequenz
            \item<2-> Die aufgenommene Leistung innerhalb einer Periode wird vollständig wieder abgegeben
        \end{itemize}
    }

    \fu{
        \resizebox{0.8\textwidth}{!}{
            \begin{circuitikz}[domain=0:8,samples=100]
    %Gitter
    \draw[very thin,color=gray] (-0.1,-2.1) grid (8.2,2.1);
    \draw[->] (-0.2,0) -- (8.5,0) node[right] {$t$};
    \draw[->] (0,-2.1) -- (0,2.2) node[above] {};
    \draw (0,-2.1) node[below] {$0$};
    \draw (2,-2.1) node[below] {$\pi$};
    \draw (4,-2.1) node[below] {$2\pi$};
    \draw (6,-2.1) node[below] {$3\pi$};
    \draw (8,-2.1) node[below] {$4\pi$};
    \draw[color=voltage] (-0.5,1.5) node[] {$u(t)$};
    \draw[color=red] (-0.5,-1) node[] {$i(t)$};
    \draw[color=voltage, smooth, thick] plot[id=Sinus11] function{1.5 * sin(pi/2 * x)} node[right] {};
    \draw[color=red, smooth, thick] plot[id=Sinus12] function{1 * sin((pi/2) * x - (pi/2))} node[right] {};
    \pause

    \draw[color=green] (-0.5,-0.5) node[] {$p(t)$};
    \draw[color=green, smooth, thick] plot[id=Sinus13] function{(1.5 * sin(pi/2 * x)) * (1 * sin((pi/2) * x - (pi/2)))} node[right] {};
\end{circuitikz}
        }
    }{{\bf Verlauf des Stromes, der Spannung und der zeitabhängigen Leistung an einer induktive Last.} Die Augenblicksleistung weist die 
    doppelte Frequenz gegenüber der Spannung bzw. des Stromes auf. Die positiven und negativen Halbwellen heben sich gegenseitig
    auf, die aufgenommene Leistung wird vollständig wieder abgegeben. \label{BildInduktiveLast}}

\end{frame}

\begin{frame}
    \ftx{Leistung Kapazität}

    \s{
        Beim Kondensator verhält es sich dem Prinzip nach äquivalent zur Induktivität. Hier eilt der Strom der Spannung um 90° ($-\pi/2$). 
        Der konstante Anteil der Augenblicksleistung beträgt Null und die Leistung pulsiert mit doppelter Frequenz um die Nulllinie. 
        Wie auch bei der Induktivität wird von einer Kapazität die aufgenommene Leistung wieder vollständig abgegeben, sodass im 
        Mittel keine Leistung aufgenommen wird. Das Verhalten der zeitabhängigen Leistung an einer kapazitiven Last wird in der Abbildung
        \ref{BildKapazitiveLast} vorgestellt. 
    }

    \b{
        Augenblicksleistung an einer Kapazität:
        \begin{itemize}
            \item<1-> Spannung und Strom um $-\pi/2$ phasenverschoben
            \item<1-> Konstanter Anteil der Augenblicksleistung gleich Null 
            \item<2-> Augenblicksleistung pulsiert mit doppelter Frequenz
            \item<2-> Aufgenommene Leistung wird vollständig wieder abgegeben
        \end{itemize}
    }

    \fu{
        \resizebox{0.8\textwidth}{!}{
            \begin{circuitikz}[domain=0:8,samples=100]
    %Gitter
    \draw[very thin,color=gray] (-0.1,-2.1) grid (8.2,2.1);
    \draw[->] (-0.2,0) -- (8.5,0) node[right] {$t$};
    \draw[->] (0,-2.1) -- (0,2.2) node[above] {};
    \draw (0,-2.1) node[below] {$0$};
    \draw (2,-2.1) node[below] {$\pi$};
    \draw (4,-2.1) node[below] {$2\pi$};
    \draw (6,-2.1) node[below] {$3\pi$};
    \draw (8,-2.1) node[below] {$4\pi$};
    \draw[color=voltage] (-0.5,1.5) node[] {$u(t)$};
    \draw[color=red] (-0.5,-1) node[] {$i(t)$};
    \draw[color=voltage, smooth, thick] plot[id=Sinus14] function{1.5 * sin(pi/2 * x)} node[right] {};
    \draw[color=red, smooth, thick] plot[id=Sinus15] function{1 * sin((pi/2) * x + (pi/2))} node[right] {};
    \pause

    \draw[color=green] (-0.5,-0.5) node[] {$p(t)$};
    \draw[color=green, smooth, thick] plot[id=Sinus16] function{(1.5 * sin(pi/2 * x)) * (1 * sin((pi/2) * x + (pi/2)))} node[right] {};
\end{circuitikz}	
        }
    }{{\bf Verlauf des Stromes, der Spannung und der zeitabhängigen Leistung an einer kapazitiven Last.} Die Augenblicksleistung weist die 
    doppelte Frequenz gegenüber der Spannung bzw. des Stromes auf. Die positiven und negativen Halbwellen heben sich gegenseitig
    auf, die aufgenommene Leistung wird vollständig wieder abgegeben. \label{BildKapazitiveLast}}
\end{frame}




\begin{frame}
    \ftx{Blindleistung}

    \s{
        Ist die Augenblicksleistung größer Null, wird Leistung aufgenommen. Entgegengesetzt wird Leistung abgegeben, wenn die 
        Augenblicksleistung kleiner Null ist. Durch die nicht vorhandene Phasenverschiebung zwischen dem Strom und der Spannung
        am ohmschen Widerstand ist die mittlere Leistung stets größer null. Der ohmsche Widerstand nimmt also stets Leistung auf. 
        Die Leistung wird hier als Wirkleistung $P$ bezeichnet, welche für Anwendungen genutzt werden kann. Sie wird anhand der
        bereits vorgestellten Gleichung \ref{GleichungLei5} für sinusförmige Spannungen und Ströme berechnet. 
        Die beschriebenen Leistungen erklären Wirkleistungsanteile und Blindleistungsanteile. Die Blindleistung beschreibt dabei 
        denjenigen Anteil der Leistung, welcher für die Erzeugung von elektrischen und magnetischen Feldern notwenig ist. Ist die 
        elektrische Impedanz rein induktiv oder kapazitiv, liegt eine Phasenverschiebung von $\pm 90^\circ$ vor. Der rein induktive
        oder rein kapazitive Zweipol nimmt Leistung auf, welche anschließend wieder vollständig abgegeben wird. Das dadurch 
        resultierende Mittel der Leistung beträgt immer null. Diese Leistung wird als Blindleistung $Q$ Bezeichnet. Es wird dabei
        zwischen induktiver und kapazitiver Blindleistung unterschieden. Für die induktive Blindleistung ergibt sich eine positive
        Phasenverschiebung für die Leistung, was genau dem Verlauf des Sinusfunktion entspricht. Die kapazitive Blindleistung wird 
        entgegengesetzt verschoben, sodass sich ebenfall eine Sinusfunktion, jedoch mit negativem Vorzeichen definieren lässt. 
        In den Gleichungen \ref{GleichungLei7} und \ref{GleichungLei8} werden die Gleichungen für die Berechnungen der induktiven
        und kapazitiven Blindleistung vorgestellt. Die Maßeinheit für die induktive und kapazitive Blindleistung ist das 
        var (Volt-Ampere-reaktiv).

        \begin{eq}
            Q_\mathrm{Ind} = U \cdot I \cos(\varphi + 90^\circ) = U \cdot I \cdot \sin(\varphi)        \label{GleichungLei7}        
        \end{eq}

        \begin{eq}
            Q_\mathrm{Kap} = U \cdot I \cos(\varphi - 90^\circ) = - U \cdot I \cdot \sin(\varphi)        \label{GleichungLei8}
        \end{eq}
    }

    \b{
        Die Blindleistung beschreibt denjenigen Anteil der Leistung, welcher zur Erzeugung von elektrischen und magnetischen Feldern 
        benötigt wird.
        \begin{itemize}
            \item Periodischer Anteil der Augenblicksleistung mit Gleichanteil gleich Null
        \end{itemize}

            \begin{minipage}[t]{0.49\textwidth}
                \begin{itemize}
                    \item<2-> Induktive Blindleistung: \\
                    \begin{eq}
                        Q_\mathrm{Ind} = U \cdot I \cos(\varphi + 90^\circ) = U \cdot I \cdot \sin(\varphi)       \nonumber   
                    \end{eq}
                \end{itemize}
            \end{minipage}
            \begin{minipage}[t]{0.49\textwidth}
                \begin{itemize}
                    \item<3-> Kapazitive Blindleistung: \\
                    \begin{eq}
                        Q_\mathrm{Kap} = U \cdot I \cos(\varphi - 90^\circ) = - U \cdot I \cdot \sin(\varphi)       \nonumber
                    \end{eq}
                \end{itemize}
            \end{minipage}
    }

    \only<4->{
    \begin{eq}
        [Q] = 1\ Volt-Ampere-reaktiv = 1\ \mathrm{var} \nonumber
    \end{eq}
    }

    \s{    
        Bei idealen kapazitiven oder induktiven Widerständen wird keine Energie in Form von Wärme umgesetzt.
        Es ändert sich lediglich das magnetische oder elektrische Feld, so wird also Energie gespeichert oder 
        abgegeben.
    }

\end{frame}



\begin{frame}
    \ftx{Wirkleistung und Blindleistung}

    \begin{Merksatz}{Wirkleistung und Blindleistung}
        In der komplexen Ebene wird die Leistung durch die Wirkleistung und die Blindleistung erklärt. Die Wirkleistung
        beschreibt den Realteil und die Blindleistung den Imaginärteil der komplexen Scheinleistung. Die Wirkleistung 
        und die Blindleistung werden wie folgt definiert:

        \begin{eq}
            P = U \cdot I \cdot \cos(\varphi)   \qquad  Q = U \cdot I \cdot \sin(\varphi)   \nonumber
        \end{eq}

    \end{Merksatz}

\end{frame}






%Scheinleistung
\begin{frame}
    \ftb{Scheinleistung}

    \s{
        Bisher wurden die Grundlagen der Leistungsbetrachtung von idealen ohmschen und induktiven/kapazitiven Impedanzen 
        betrachtet. Jedoch treten in den meisten Netzwerken sowohl ohmsche als auch induktive oder kapazitive Impedanzen 
        auf. In solchen Netzen wird dann unterschieden zwischen Scheinleistung, Wirkleistung und Blindleistung. Die 
        Scheinleistung steht dabei für die Gesamtleistung des Systems. Um die Scheinleistung zu berechnen, wird mit dem 
        Maximalwert gerechnet, also wenn keine Phasenverschiebung vorliegen würde. Damit ist dieser Wert gleich der 
        Leistung eines idealen ohmschen Widerstands:
    }

    \b{
        \begin{itemize}
            \item Scheinleistung an einem idealen Widerstand:
        \end{itemize}
    }

    \begin{eq}
        S=U\cdot I = P_\mathrm{R} \label{s}
    \end{eq}

    \s{
        Wird die Scheinleistung in der komplexen Ebene betrachtet, kommen die Festzeiger zum Einsatz. Wird die kompexe Spannung 
        mit dem komplexen Strom multipliziert, würden nach den Regeln des kompexen rechnens die Phasenwinkel addiert werden.
        Um eine korrekte Leistungsberechnung durchzuführen, muss jedoch eine Differenz zwischen den Phasenwinkeln vorliegen.
        Um diesem Problem abhilfe zu schaffen, wird mit dem konjugierten komplex Wert des Stroms gerechnet:

        \begin{eqa}
            \underline{S}&=\underline{U}\cdot \underline{I}^* \nonumber \\
            mit~~~ \underline{U}&=U\cdot e^{j\varphi_\mathrm{u}} \nonumber \\
            und~~~ \underline{I}^*&=I\cdot e^{-j\varphi_\mathrm{i}} \\
            folgt~~~ \underline{S}&=U\cdot e^{j\varphi_\mathrm{u}} \cdot I\cdot e^{-j\varphi_\mathrm{i}}\nonumber  \\
            &=U\cdot I\cdot e^{j(\varphi_\mathrm{u}-\varphi_\mathrm{i})} \\
            &=U\cdot I\cdot e^{j(\varphi)} \nonumber 
        \end{eqa}
    }
     
    \b{
            \begin{minipage}[t]{0.5\textwidth}
                \begin{itemize}
                    \item Komplexe Scheinleistung: \\
                    \begin{eqa}
                        \underline{S}&=\underline{U}\cdot \underline{I}^* \nonumber \\
                        mit~~~ \underline{U}&=U\cdot e^{j\varphi_\mathrm{u}} \nonumber \\
                        und~~~ \underline{I}^*&=I\cdot e^{-j\varphi_\mathrm{i}} \\
                        folgt~~~ \underline{S}&=U\cdot e^{j\varphi_\mathrm{u}} \cdot I\cdot e^{-j\varphi_\mathrm{i}}\nonumber  \\
                        &=U\cdot I\cdot e^{j(\varphi_\mathrm{u}-\varphi_\mathrm{i})} \\
                        &=U\cdot I\cdot e^{j(\varphi)} \nonumber 
                    \end{eqa}
                \end{itemize}
            \end{minipage}
            \begin{minipage}[t]{0.49\textwidth}
                \begin{itemize}
                    \item Zeigerdiagramm: \\
                    \fu{
                        \resizebox{0.6\textwidth}{!}{
                        \begin{tikzpicture}
                            \draw[->] (0, -0.3) -- (0, 3);
                            \draw[->] (-0.3, 0) -- (3, 0);
                            \node at (0, 3.2) {$\Im(\underline{S})$};
                            \node at (3.125, -0.25) {$\Re(\underline{S})$};
                            \draw (0, 0) coordinate (a);
                            \draw (2.12, 0) coordinate (b);
                            \draw (2.12, 2.12) coordinate (c);
                            \draw[->, line width=2pt] [red] (0, 0) -- (2.12, 2.12);
                            \draw pic[draw, angle radius = 40, ->] {angle = b--a--c};
                            \node at (1, 1.5) {$S$};
                            \node at (0.9, 0.5) {$\varphi$};
                            \node at (0, 4.5) {$ $};
                            \draw[line width=1pt, dashed] (2.12, -0.2) -- (2.12, 2.12);
                            \node at (2.12, -0.5) {$P$};
                            \draw[line width=1pt, dashed] (-0.2, 2.12) -- (2.12, 2.12);
                            \node at (-0.7, 2.12) {$Q$};
                        \end{tikzpicture}
                        }
                    }{} 
                \end{itemize}   
            \end{minipage}
    }
    
\end{frame}

\begin{frame}
    \ftx{Leistungsarten}

    \s{
        Über den komplexen Wert der Spannung und den konjugiert komplexen Wert des Stromes lässt sich separat die Wirkleistung
        und die Blindleistung oder direkt die komplexe Scheinleistung bestimmen:
    }

    \b{
        \begin{itemize}
            \item Von der Spannung und dem Strom zur komplexen Scheinleistung:
        \end{itemize}
    }

    \begin{eq}
        U \cdot I \cdot e^{j\varphi}=U\cdot I\cdot(\cos\varphi+j~\sin\varphi)=P+jQ=\underline{S}
    \end{eq}
    
    \s{
        Außerdem lässt sich über den Leistungsfaktor auch direkt ein Zusammenhang zwischen der Wirkleistung bzw. der Blindleistung 
        und der Scheinleistung nach Gleichung \ref{GleichungSch1} festlegen. 
    }

    \b{
        \begin{itemize}
            \item Umrechnung Wirkleistung/Blindleistung und Scheinleistung über Leistungsfaktor:
        \end{itemize}
    }

    \begin{eq}
        P = \cos(\varphi) \cdot S    \qquad \qquad      Q=S\cdot \sin(\varphi)          \label{GleichungSch1}
    \end{eq}

\end{frame}


\begin{frame}
    \ftx{Komplexe Scheinleistung}

    \begin{Merksatz}{Komplexe Scheinleistung}
        Die komplexe Scheinleistung setzt sich aus der Wirkleistung und der Blindleistung zusammen. Sie kann auf 
        verschiedene Weisen bestimmt werden: 

        \begin{eq}
            \underline{S} = \underline{U} \cdot \underline{I}^* = U \cdot I\cdot e^{j(\varphi_\mathrm{u}-\varphi_\mathrm{i})} = P + jQ \nonumber 
        \end{eq}

    \end{Merksatz}
\end{frame}



        
\begin{frame}
    \ftx{Betriebsdiagramm}

    \s{
        Die Größen Wirkleistung und Blindleistung sind in der komplexen Zahlenebene zu finden. Dabei ist die Wirkleistung ein rein 
        reeler Wert und die Blindleistung stellt einen rein imaginären Wert dar. Die Scheinleistung, welche auch eine komplexe Größe 
        ist, kann als Betrag des Weges von Wirk- und Blindleistung dargestellt werden (vgl. Abbildung \ref{BildBetriebsdiagramm}). 
        Die hier dargestellte Scheinleistung $S$ mit zugehörigen Phasenwinkel $\varphi$ ist im $I.$ Quadranten verortet. Dies wird 
        durch einen positiven Wirkleistungsanteil und einen positiven Blindleistungsanteil verursacht. Die positive Wirkleistung
        deutet an, dass von etwas elektrische Leistung aufgenommen wird, also eine elektrische Last darstellt. Deswegen wird hier
        vom Lastbetrieb gesprochen. Die positive Blindleistung spricht für die vorherschenden Effekte einer Spule und wirkt somit 
        induktiv. Die Scheinleistung im $I.$ Quadranten wird im induktiven Lastbetrieb betrieben. Wirkt eine negative Wirkleistung 
        in die Scheinleistung, wird Wirkleistung abgegeben, wie bei einem Generator, welcher Leistung abgibt. Bei einer positiven 
        Blindleistung liegt hier ein induktiver Generatorbetrieb vor ($II.$ Quadrant). Ist die Blindleistung negativ, überwiegen 
        kapazitive Effekte. Bei negativer Wirkleistung wird diese Betriebsform kapazitiver Generatorbetrieb genannt ($III.$ Quadrant).
        Als letztes wird im $IV.$ Quadranten ein kapazitiver Lastbetrieb beschrieben. Hier wirkt eine positive Wirkleistung und eine 
        negative Blindleistung in die Scheinleistung ein. 
    }

    \b{
        Betriebsdiagramm für die komplexe Scheinleistung eines komplexen Zweipols:
        \begin{itemize}
            \item<2-> Positive Wirkleistung $\rightarrow$ Lastbetrieb
            \item<3-> Positive Blindleistung $\rightarrow$ Induktiv
            \item<4-> Scheinleistung im I. Quadranten
            \item<5-> ~ im II. Quadranten: Induktiver Generatorbetrieb
            \item<5-> ~ im III. Quadranten: Kapazitiver Generatorbetrieb 
            \item<5-> ~ im IV. Quadranten: Kapazitiver Lastbetrieb
        \end{itemize}
    }

    \fu{
        \resizebox{0.3\textwidth}{!}{
        \begin{tikzpicture}
    \draw[->] (0, -4) -- (0, 4);
    \draw[->] (-4, 0) -- (4, 0);
    \node at (0, 4.2) {$\Im(\underline{S})$};
    \node at (4.125, -0.25) {$\Re(\underline{S})$};
    \draw (0, 0) coordinate (a);
    \draw (2.12, 0) coordinate (b);
    \draw (2.12, 2.12) coordinate (c);
    \pause

    \draw[->, line width=2pt] [blue] (0, 0) -- (2.12, 0);
    \node at (1.2, -0.25) {$P$};
    \node at (2, -4) {$Lastbetrieb$};
    \pause

    \draw[->, line width=2pt] [yellow] (2.12, 0) -- (2.12, 2.12);
    \node at (2.3, 1.1) {$Q$};
    \node at (4, 2) {Induktiv};
    \pause

    \draw (0,0) circle (3);
    \draw[->, line width=2pt] [red] (0, 0) -- (2.12, 2.12);
    \draw pic[draw, angle radius = 40, ->] {angle = b--a--c};
    \node at (1, 1.5) {$S$};
    \node at (0.9, 0.5) {$\varphi$};
    \node at (3,3) {$I.$};
    \pause

    \node at (-2, -4) {$Generatorbetrieb$};
    \node at (4, -2) {Kapazitiv};
    \node at (-4, -2) {{\"U}bererregt};
    \node at (-4, 2) {Untererregt};
    \node at (-3,3) {$II.$};
    \node at (-3,-3) {$III.$};
    \node at (3,-3) {$IV.$};
\end{tikzpicture}
        }
    }{{\bf Zeigerdiagramm der Leistung.} Zusammensetzung von Blind-, Wirk- und Scheinleistung in Abhängigkeit der Phasenverschiebung 
    zwischen Strom und Spannung. Hier eine induktive Last im positiven Imaginärteil. \label{BildBetriebsdiagramm}}

\end{frame}


\begin{frame}
    \ftx{Blindleistung im Generatorbetrieb}

    \s{
        Die Blindleistung wird unter anderem benötigt, um Magnetfelder in elektrischen Maschinen zu erzeugen. Der 
        Blindleistungsbedarf entsteht durch Komponenten im System, welche entweder induktiv oder kapazitiv wirken. 
        Wenn der Strom zur Erzeugung des Magnetfeldes (Erregerstrom) höher ist, als es für die nominale Leistung des 
        Generators erforderlich wäre, ist ein Generator übererregt. In diesem Fall ist der Generator in der Lage, 
        Blindleistung in das Netz abzugeben. Die hierdurch erhöhte Blindleistung im Netz führt zu einer Spannungsanhebung. 
        Steigt der Energiebedarf im Netz, kann diese Gegebenheit dafür genutzt werden, einem Absinken der Netzspannung 
        entgegenzuwirken. 
        Andersherum arbeitet ein Generator mit einem geringeren Erregerstrom als benötigt im untererregten Zustand. Der 
        Generator nimmt hier Blindleistung aus dem Netz auf, um das erforderliche Magnetfeld aufrecht zu halten. Der 
        Bezug von Blindleistung führt dann wiederum zu einer Spannungsabsenkung im Netz. Bei einer starken untererregung 
        eines Generators kann es zum Ausfall der elektrischen Maschine kommen. 
    }

    \b{
        Blindleistung im Generatorbetrieb:
        \begin{itemize}
            \item Erregerstrom des Generators $>$ Bedarf des magnetischen Feldes $\rightarrow$ Übererregt   \\
                Blindleistung wird an das Netz abgegeben $\rightarrow$ Spannungsanhebung    \\
                Anwendung: Spannungshaltung
            \item Erregerstrom des Generators $<$ Bedarf des magnetischen Feldes $\rightarrow$ Untererregt  \\
                Blindleistung muss aus dem Netz bezogen werden $\rightarrow$ Spannungsabsenkung \\
                Eventueller Ausfall der elektrischen Maschinen
        \end{itemize}
    }

\end{frame}


\begin{frame}
    \ftx{Zusammenfassung}

    \s{
        Die Einheiten der drei Leistungsarten sind theoretisch alle vergleichbar. Um zwischen den Leistungsarten zu unterscheiden, 
        wurden für die Scheinleistung und die Blindleistung neue 
        Einheiten gewählt, welche physikalisch jedoch dasselbe aussagen. Für die Scheinleistung wurde die Einheit Volt-Ampere [VA] 
        festgelegt und für die Blindleistung Volt-Ampere-reaktiv [var]. Die Wirkleistung wird in der bekannten Einheit Watt [W] 
        angegeben.

        %  \begin{center}
        \begin{table}[H]
            \centering
            \caption{{\bf Zusammenfassung der Leistungsarten.} Wirkleistung, Blindleistung und Scheinleistung welche durch 
            Wechselgrößen hervorgerufen werden, aufgelistet mit den Formelzeichen und Einheiten.}
            \label{TabelleLeistungsarten}
                \begin{tabular}{  c  c  c  c }
                    \hline
                    \multicolumn{2}{c}{\bf Leistungsart} & \multicolumn{2}{c}{\bf Einheit} \\
                    \hline
                    Wirkleistung & P & W & (Watt) \\
                    Blindleistung & Q & var & (Volt-Ampere-reaktiv) \\
                    Scheinleistung & S  & VA & (Volt-Ampere) \\
                    \hline
                \end{tabular}
        \end{table}
    % \end{center}
    }

    \b{
        Leistungsarten an Wechselgrößen:\\
        \begin{itemize}
            \item Wirkleistung $P = U \cdot I \cdot \cos(\varphi)$: ohmscher Anteil
            \item Blindleistung $Q = U \cdot I \cdot \sin(\varphi)$: induktiver/kapazitiver Anteil
            \item Scheinleistung $\underline{S} = \underline{U} \cdot \underline{I}^* = P + jQ$: Gesamtleistung des Systems
        \end{itemize}


        \begin{table}[H]
            \centering
                \begin{tabular}{  c  c  c  c }
                    \hline
                    \multicolumn{2}{c}{\bf Leistungsart} & \multicolumn{2}{c}{\bf Einheit} \\
                    \hline
                    Wirkleistung & P & W & (Watt) \\
                    Blindleistung & Q & var & (Volt-Ampere-reaktiv) \\
                    Scheinleistung & S  & VA & (Volt-Ampere) \\
                    \hline
                \end{tabular}
        \end{table}
    }
\end{frame}


\begin{frame}
    \ftb{Leistung und elektrische Energie}

    \s{
        Neben der elektrischen Leistung ist auch die elektrische Energie eine genutzte Größe, um beispielsweise Verbräuche 
        von Haushalten zu erfassen. Hier wird die elektrische Leistung pro Zeiteinheit gemessen. Nach der Definition aus der 
        Thermodynamik ist die Energie die über einen bestimmten Zeitraum aufgebrachte Leistung. Es wird also gemäß Gleichung
        \ref{GleichungEne1} die elektrische Leistung während einer Zeit zwischen $t_1$ und $t_2$ aufintegriert um die elektrische
        Energei zu bestimmen. 
    }

    \b{
        \begin{itemize}
            \item Elektrische Energie:
        \end{itemize}
    }

    \begin{eq}
        E_\mathrm{el} = \int_{t_1}^{t_2} p_\mathrm{el}(t)\ \d t        \label{GleichungEne1}
    \end{eq}


    \s{
        Vergleichbar mit der Wirkleistung und der Blindleistung lässt sich auch die Energie in eine Wirkarbeit und eine 
        Blindarbeit unterteilen. Die Blindarbeit beschreibt dabei denjenigen Anteil der elektrischen Energie, welcher nicht
        in Nutzenergie, bzw. Wirkarbeit. umgewandelt wird. 
    }

    \b{
        \begin{itemize}
            \item Aufteilung in Wirkarbeit und Blindarbeit:
        \end{itemize}
    }

    \begin{eq}
        W_\mathrm{N} = \int_{t_1}^{t_2} p(t) \ \d t    \qquad \qquad    W_\mathrm{Q} = \int_{t_1}^{t_2} Q \ \d t   \label{GleichungEne2}
    \end{eq}

    \s{
        Im stationären Zustand werden lediglich die Effektivwerte von Spannung und Strom und somit auch von der Scheinleistung,
        Wirkleistung und Blindleistung betrachtet und sind somit zeitlich konstant. 
    }

    \b{
        \begin{itemize}
            \item Berechnung mit Effektivwerten:
        \end{itemize}
    }

    \begin{eq}
        W_\mathrm{N} = P \cdot (t_2 - t_1)    \qquad \qquad    W_\mathrm{Q} = Q \cdot (t_2 - t_1)   \label{GleichungEne3}
    \end{eq}

\end{frame}


\begin{frame}
    \ftx{Beispielaufgabe}
	\begin{bsp}{Leistungsberechnung}{Leistungsberechnung:}

    \s{
        An einer Wechselstromlast werden die folgenden Werte für die Wechselspannung und den Wechselstrom gemessen:
    }



    \begin{enumerate}
        \only<1>{	
            \item[]

            Gegeben sind:   \\
            \begin{eq}
                \hat{U} = 12\ V      \qquad        \hat{I} = 2A\ \nonumber   \qquad         \varphi = 60\degree \ \nonumber
            \end{eq}

            Die folgenden Aufgaben sollen bearbeitet werden:

            \begin{itemize}
                \item[\bf a)] Berechnung von Wirkleistung und Blindleistung.
                \item[\bf b)] Berechnung der Scheinleistung mit dem konjugierten komplexen Strom.  
            \end{itemize}
        }
        \only<2>{	
            \item[\bf a)]
            Die Wirkleistung beträgt:
            \begin{eqa}
                P &= U \cdot I \cdot \cos(\varphi) = \frac{\hat{U}}{\sqrt{2}} \cdot \frac{\hat{I}}{\sqrt{2}} \cdot \cos(\varphi)
                = \frac{12\ V}{\sqrt{2}} \cdot \frac{2\ A}{\sqrt{2}} \cdot \cos(60^o) \nonumber \\
                P &= 6\ W    \nonumber
            \end{eqa}
            Die Blindleistung beträgt:
            \begin{eqa}
                Q &= U \cdot I \cdot \sin(\varphi) = \frac{\hat{U}}{\sqrt{2}} \cdot \frac{\hat{I}}{\sqrt{2}} \cdot \sin(\varphi)
                = \frac{12\ V}{\sqrt{2}} \cdot \frac{2\ A}{\sqrt{2}} \cdot \sin(60^o) \nonumber \\
                Q &= 10,392\ var    \nonumber
            \end{eqa}
        } 
        \only<3>{	
            \item[\bf b)]
            Die Scheinleistung beträgt:
            \begin{eqa}
                \underline{S} &= \underline{U} \cdot \underline{I}^* = 
                \frac{\underline{\hat{U}}}{\sqrt{2}} \cdot \frac{\underline{\hat{I}}^*}{\sqrt{2}} =
                \frac{12\ V}{\sqrt{2}} \cdot e^{j(2\pi\cdot20\ Hz)} \cdot \frac{2\ A}{\sqrt{2}} \cdot e^{j(2\pi\cdot20\ Hz+\frac{\pi}{3})} \nonumber \\
                \underline{S} &= 12\ VA \cdot e^{j\frac{\pi}{3}} = 6\ W + j10,392\ var    \nonumber
            \end{eqa}
        } 
    \end{enumerate}
    \end{bsp}
\end{frame}

\newpage