%%%%%%%%%%%%%%%%%%%%%%%%%%%%%%%%%%%%%%%%%%%%%%%%%%%%%%%%%%%%%%%%%%%%%%%%%%%%%%%%%%%
%%%%%%%%%%%%%%%%%%%%%%%%%%%%%%%%%%%%%%%%%%%%%%%%%%%%%%%%%%%%%%%%%%%%%%%%%%%%%%%%%%%
%                               START KAPITEL 1
%%%%%%%%%%%%%%%%%%%%%%%%%%%%%%%%%%%%%%%%%%%%%%%%%%%%%%%%%%%%%%%%%%%%%%%%%%%%%%%%%%%
%%%%%%%%%%%%%%%%%%%%%%%%%%%%%%%%%%%%%%%%%%%%%%%%%%%%%%%%%%%%%%%%%%%%%%%%%%%%%%%%%%%
\newpage

\section{Frequenzgang, Amplitudengang, Phasengang}
\label{sec:frequenzgang}
\begin{frame}\ftx{\secname}
\s{% Einleitung
    In diesem Kapitel beschäftigen wir uns mit dem Frequenzgang, Amplitudengang und Phasengang von Zweitoren. 
    Diese dienen zur Beschreibung des frequenzvariablen Verhaltens elektrischer Netze in Form von Zweitoren wie in Abbildung \ref{fig:zweitor} gezeigt. 
    Am Beispiel einfacher passiver Filterschaltungen werden wir die Begriffe Frequenz-, Amplituden- und Phasengang einführen und deren Bedeutung und Anwendung erläutern. 

    \fu{\includegraphics{Tikz/pdf/circ_zweitor.pdf}
    }{Schaltsymbol eines Zweitors (Vierpols) mit Eingang (links) und Ausgang (rechts)\label{fig:zweitor}} % in Folien im folgenden Kapitel (Frequenzgang, Amplitudengang, Phasengang)
}%\s Ende
\begin{Lernziele}{Frequenzgang, Amplitudengang, Phasengang}
    Studierende lernen:
    \begin{itemize}
        \item Frequenzgänge anhand der Schaltungstopologie algebreaisch zu bestimmen
        \item die Funktionsweise einfacher Hoch- und Tiefpassfiltern zu verstehen
        \item das Grenzverhalten von Vierpolen anhand ihrer Frequenzgänge zu analysieren
    \end{itemize}
\end{Lernziele}
\end{frame}

\begin{frame}\ftx{\secname}
\b{%
    \begin{minipage}[t][3.5cm][t]{\textwidth}% upper half
        \textbf{Analyse frequenzvariablen Verhaltens elektrischer Netze \only<3->{(Zweitore)}}
        \begin{figure}
            \centering
            \only<beamer:1-2| handout:0>{\includegraphics{Tikz/pdf/circ_zweitor_quelle_last.pdf}}%
            \only<beamer:3-| handout>{\includegraphics{Tikz/pdf/circ_zweitor_longpins.pdf}}%
        \end{figure}%
    \end{minipage}
    \hspace{0.02\linewidth}% for vertical alignment % ref: https://tex.stackexchange.com/questions/378548/vertical-alignment-of-side-by-side-minipages
    \begin{minipage}[c][4cm][c]{0.48\textwidth}% left col
        \only<2>{$\mathrlap{\textbf{Wie sieht das Übertragungsverhalten aus?}}$}%
        \onslide<3->{\textbf{Linear und zeitinvariant (LZI):}}
        \begin{itemize}
            \item<4-> Sinusform am Eingang $u_1(\omega t)$
            \item<5-> $\rightarrow$ Sinusform am Ausgang $u_2(\omega t)$
            \item<5-> $\rightarrow$ Frequenz gleich $\omega_2 = \omega_1$
            \item<6-> $\rightarrow$ Phasenverschiebung $\Delta\varphi = ?$
            \item<6-> $\rightarrow$ Amplitudenverhältnis $\frac{\hat{U}_2}{\hat{U}_1} = ?$
        \end{itemize}
    \end{minipage}
    \hspace{0.02\linewidth}% for vertical alignment
    \begin{minipage}[c][4cm][c]{0.48\textwidth}% right col
        % first plot variant used for beamer and handout, next plot variants only for beamer
        \only<beamer:4|  handout:0>{\includegraphics{Tikz/pdf/plot_sin_comp_amplitude_phase01.pdf}}% If broken slide animations, delete TexAux files and recompile
        \only<beamer:5-6|  handout>{\includegraphics{Tikz/pdf/plot_sin_comp_amplitude_phase02.pdf}}% If broken slide animations, delete TexAux files and recompile
        \only<beamer:7|  handout:0>{\includegraphics{Tikz/pdf/plot_sin_comp_amplitude_phase03.pdf}}% If too long, comment out some of the \only<n>{...} lines
        \only<beamer:8|  handout:0>{\includegraphics{Tikz/pdf/plot_sin_comp_amplitude_phase04.pdf}}% 
        \only<beamer:9| handout:0>{\includegraphics{Tikz/pdf/plot_sin_comp_amplitude_phase05.pdf}}% 
        \only<beamer:10| handout:0>{\includegraphics{Tikz/pdf/plot_sin_comp_amplitude_phase06.pdf}}% 
        \only<beamer:11| handout:0>{\includegraphics{Tikz/pdf/plot_sin_comp_amplitude_phase07.pdf}}% 
        % Currently linear change in amplitude and phase. Could also use "real" change, e.g. Tiefpass behaviour  
    \end{minipage}%
}%
\end{frame}




%%%%%%%%%%%%%%%%%%%%%%%%%%%%%%%%%%%%%%%%%%%%%%%%%%%%%%%%%%%%%%%%%%%%%%%%%%%%%%%%%%%

\subsection[Definitionen]{Definition Frequenzgang, Amplitudengang und Phasengang}
\begin{frame}\ftx{\subsecname}
\begin{columns}[t]
    \column{0.45\textwidth}% left col
        \b{%
            \textbf{Zweitor (Vierpol):}
            \fu{\includegraphics{Tikz/pdf/circ_zweitor.pdf}}{}% im Skript im vorigen Kapitel (Wiederholung)
            \pause%
            \textbf{Linear, zeitinvariant (LZI)}: 
            \begin{equation*}\begin{aligned}
                u_1(t)& = \hat{U}_1 \cdot \sin(\omega t)\\
                \xrightarrow{LZI}\  u_2(t)& = \hat{U}_2 \cdot \sin(\omega t + \varphi)
            \end{aligned}\end{equation*}%  
            \begin{itemize}
                \item[] Sinusform bleibt erhalten
                \item[] Frequenz gleich $\omega_2 = \omega_1$
            \end{itemize}%
        }%
    \column{0.55\textwidth}% right col
        \b{%
        \pause%
            \textbf{Frequenzgang:}%
            \begin{align*}
                \underline{F}(\mathrm{j}\omega) &= \frac{\underline{U}_2}{\underline{U}_1} = A(\omega)\cdot\mathrm e^{\mathrm j\varphi(\omega)}\\[-12pt]% onslide breaks line spacing, manual adjustment
                \onslide<4->{\intertext{\textbf{Amplitudengang:}}%
                A(\omega) &= |\underline{F}(\mathrm{j}\omega)| = \frac{|\underline{U}_2|}{|\underline{U}_1|}\\
                &= \sqrt{\left(\Re\{\underline{F}(\mathrm{j}\omega)\}\right)^2 + \left(\Im\{\underline{F}(\mathrm{j}\omega)\}\right)^2}\\[-12pt]% onslide breaks line spacing, manual adjustment
                }% end onslide
                \onslide<5->{\intertext{\textbf{Phasengang:}}%
                \varphi(\omega) &= \angle \underline{F}(\mathrm{j}\omega) = \angle \underline{U}_2 - \angle \underline{U}_1\\
                &= \arctan\left(\frac{\Im\{\underline{F}(\mathrm{j}\omega)\}}{\Re\{\underline{F}(\mathrm{j}\omega)\}}\right)%
                }% end onslide
            \end{align*}
        }%
        \s{%
            % Definition Frequenzgang
            \index{Frequenzgang}%
            \index{lineares zeitinvariantes System}%
            Der \textbf{Frequenzgang} $\underline{F}(\mathrm{j}\omega)$, auch komplexer Amplitudengang genannt, 
            beschreibt das Verhältnis von Ausgangs- zu Eingangssignal eines linearen zeitinvarainten Systems (LZI-System) bei sinusförmiger Anregung.
            Er bietet eine Möglichkeit frequenzabhängiges Verhalten von Zweitoren zu untersuchen, 
            ist ein Spezialfall der Laplace-Übertragungsfunktion aus der Systemtheorie und 
            wird wie folgt definiert:

            \begin{equation}
                \underline{F}(\mathrm{j}\omega) = \frac{\underline{U}_2}{\underline{U}_1}% F(jw)=U2/U1
                \label{eq:def:F}
            \end{equation}

            % Amplituden und Phasenänderung
            \index{Amplitudengang}%
            \index{Phasengang}%
            Aufgrund der Linearität von LZI-Systemen bleiben Frequenz und Sinusform am Ausgang erhalten. 
            Amplitude und Phase können sich in Abhängigkeit der Frequenz vom Eingang zum Ausgang ändern. 
            Erkennbar ist dies insbesondere in der Schreibweise in Polarkoordinaten. 

            \begin{equation}
                \underline{F}(\mathrm{j}\omega) = A(\omega)\cdot\mathrm e^{\mathrm j\varphi(\omega)}% F(jw)=A*exp(jphi)
                \label{eq:def:Fpolar}
            \end{equation}
            Der \textbf{Amplitudengang} $A(\omega)$ beschreibt die relative Amplitudenänderung 
            zwischen Aus- und Eingangsspannung und entspricht dem Betrag des Frequenzganges.
            Der \textbf{Phasengang} $\varphi(\omega)$ hingegen, beschreibt die absolute Phasenänderung
            zwischen Aus- und Eingangsspannung und entspricht der Phase des Frequenzganges.
            Über $\underline{F}(\mathrm{j}\omega)$ können wir beide wie folgt definieren: 

            \begin{align}
                \label{eq:def:A}
                A(\omega)       & = |\underline{F}(\mathrm{j}\omega)|&
                                & = \sqrt{\left(\Re\{\underline{F}(\mathrm{j}\omega)\}\right)^2 + \left(\Im\{\underline{F}(\mathrm{j}\omega)\}\right)^2}&
                                & = \frac{|\underline{U}_2|}{|\underline{U}_1|}\\
                \label{eq:def:phi}
                \varphi(\omega) & = \angle \underline{F}(\mathrm{j}\omega)&
                                & = \arctan\left(\frac{\Im\{\underline{F}(\mathrm{j}\omega)\}}{\Re\{\underline{F}(\mathrm{j}\omega)\}}\right)&
                                & = \angle \underline{U}_2 - \angle \underline{U}_1
            \end{align}
        }
% Hinweis: Spannungsbezug Regel, Strombezug nicht betrachtet 
%Der Frequenzgang bezieht sich in der Regel auf Spannungen, nicht auf Ströme, da in der Regel Spannungs- 
%statt Stromquellen verwendet werden. In diesem Skript bezieht sich der Frequenzgang daher immer auf Spannungen.
\end{columns}
\end{frame}

%%%%%%%%%%%%%%%%%%%%%%%%%%%%%%%%%%%%%%%%%%%%%%%%%%%%%%%%%%%%%%%%%%%%%%%%%%%%%%%%%%%

\subsection{Frequenzgang am Beispiel eines einfachen Tiefpass-Filters}
\label{sec:frequenzgang:tiefpass:herleitung}
\begin{frame}[t]\ftx{\subsecname}
\begin{columns}[T]%
\column{0.45\textwidth}% left col
    \s{\index{Filter>Tiefpass}%
        Das Zweitor in Abbildung \ref{circ:filter:rctief} stellt einen einfachen Tiefpass-Filter (kurz Tiefpass) dar.
        Der Name Tiefpass leitet sich von dem Verhalten her, dass Signale tiefer Frequenzen nahezu unverändert passieren,
        Signale hoher Frequenzen jedoch stark gedämpft (gefiltert) werden.
        Die Funktionsweise und das Verhalten des Tiefpasses lassen sich gut mittels einer Frequenzganganalyse veranschaulichen.

        \fu{\includegraphics{Tikz/pdf/circ_tiefpass_rc_annotated.pdf}}{Schaltbild, RC-Tiefpass 1. Ordnung\label{circ:filter:rctief}}%

        Der gezeigte Tiefpass besteht aus einer Serienschaltung aus $R$ und $C$ mit Eingangsspannung $\underline{U}_1$
        über beiden Elementen und Ausgangsspannung $\underline{U}_2$ über $C$.
        Da die Einzelelemente $R$ und $C$ lineare, zeitinvariante (LZI-)Bauteile sind, 
        ist auch der daraus zusammen gesetzte Tiefpass ein LZI-System.
        Das bedeutet, dass bei Anregung der Schaltung mit einer sinusförmigen Eingangsspannung $u_1(t)$ (links),
        die Ausgangsspannung $u_2(t)$ (rechts) ebenfalls sinusförmig ist und die gleiche Frequenz besitzt wie die Eingangsspannung.
    }%
    \b{%
    {\centering
    \vspace{-7.5mm}%
    \onslide<1->{\includegraphics{Tikz/pdf/circ_tiefpass_rc_annotated.pdf}}%
    }\vspace{3mm}%
        
    \textbf{RC-Tiefpass 1. Ordnung}
    \begin{equation*}\begin{aligned}
            u_1(t)& = \hat{U}_1 \cdot \sin(\omega t)\\\xrightarrow{LZI}\ 
            u_2(t)& = \hat{U}_2 \cdot \sin(\omega t + \varphi)
        \end{aligned}\end{equation*}%
    \pause%
    }
\column{0.55\textwidth}% right col
    \s{%
        \begin{equation}\begin{aligned}
            u_1(t)& = \hat{U}_1 \cdot \sin(\omega t)\\\xrightarrow{LZI}\ 
            u_2(t)& = \hat{U}_2 \cdot \sin(\omega t + \varphi)
        \end{aligned}\end{equation}%
        Die LZI-Eigenschaft ist Voraussetzung zur Bestimmung des Frequenzganges und die Anwendung der komplexen Wechselstromrechnung.
        Zusätzliche Voraussetzung ist die Annahme einer sinusförmigen Anregung im eingeschwungenen Zustand, von der wir zwecks Analyse ausgehen.
        Statt der Ein- und Ausgangsspannung $u_1(t)$ und $u_2(t)$ im Zeitbereich betrachten wir die entsprechende
        komplexe Ein- und Ausgangsspannung $\underline{U}_1$ und $\underline{U}_2$ im Frequenzbereich wie im Schaltbild dargestellt.

        Der Frequenzgang $\underline{F}(\mathrm{j}\omega)$ entspricht nach Gleichung \ref{eq:def:F} 
        dem komplexen Spannungsverhältnis von Ausgangs- zu Eingangsspannung $\underline{U}_2/\underline{U}_1$.
        Gemäß komplexer Spannungsteilerregel entspricht das Spannungsverhältnis dem Verhältnis der Impedanz am Ausgang (Impedanz der Kapazität) und 
        der Impedanz am Eingang (Summe der Impedanz von Kapazität und Widerstand), kurz $\frac{\underline{Z}_C}{R + \underline{Z}_C}$.

        Gleichung \ref{eq:tiefpass:frequenzgang} zeigt den ermittelten Frequenzgang des Tiefpasses.
        Zur Vereinfachung wurden Nenner und Zähler rationalisiert, der Nenner sortiert und der Term
        (optional) in kartesische Koordinaten umgeformt. (Vergleich kommentierte Nebenrechnung \ref{eq:tiefpass:frequenzgang:nebenrechnung}):
    }%
    \b{%
        \textbf{Frequenzgang:}%
        \begin{align*}% lesser space above than \begin{equation}\begin{align} ... as in skript version below
            \underline{F}(\mathrm{j}\omega)   & = \frac{\underline{U}_2}{\underline{U}_1} 
                                    = \frac{\frac{1}{\mathrm{j}\omega C}}{R + \frac{1}{\mathrm{j}\omega C}}
                                    = \frac{1}{\mathrm{j}\omega CR + 1} \\
                                    & = \frac{1}{1 + \mathrm{j}\omega CR}
                                    = \frac{1 - \mathrm{j}\omega CR}{1 + (\omega CR)^2}
        \end{align*}%
        \pause%
    }%
    \s{%
        \begin{equation}
            \begin{aligned}
                \underline{F}(\mathrm{j}\omega)   & = \frac{\underline{U}_2}{\underline{U}_1}
                                        = \frac{\frac{1}{\mathrm{j}\omega C}}{R + \frac{1}{\mathrm{j}\omega C}}
                                        = \frac{1}{\mathrm{j}\omega CR + 1} \\
                                        & = \frac{1}{1 + \mathrm{j}\omega CR}
                                        = \frac{1 - \mathrm{j}\omega CR}{1 + (\omega CR)^2}
            \end{aligned}\label{eq:tiefpass:frequenzgang}
        \end{equation}
    }%

    \s{
        Damit haben wir den $\underline{F}(\mathrm{j}\omega)$ in Abhängigkeit von $\omega$ 
        und den Bauteilgrößen $C$ und $R$ ermittelt. 
        Mittels Betragsbildung lässt sich daraus gemäß \eqref{eq:def:A} der Amplitudengang 
        und durch Phasenberechnung gemäß \eqref{eq:def:phi} der Phasengang bestimmen.
        
        Zur Betragsbildung bietet sich die Form $\frac{1}{1+\mathrm{j}\omega CR}$ an.
        Da der Betrag im Zähler direkt ablesbar $1$ beträgt, muss nur der Betrag im Nenner 
        bestimmen werden wie sie in \eqref{eq:tiefpass:ampli} gezeigt ist.
        Alternativ kann auch die kartesische Form aus \eqref{eq:tiefpass:frequenzgang} in 
        \eqref{eq:def:A} eingesetzt werden. 

        Für den Phasengang $\varphi(\omega)$ bietet sich die kartesische Schreibweise in 
        der Form $\frac{1-\mathrm{j}\omega CR}{1+(\omega CR)^2}$ an.
        In dieser lassen sich der Real- und Imaginärteil leicht ablesen. 
        Der Nenner hat dabei als rein reelle Größe keinen Einfluss auf die Phasenlage 
        wie in \eqref{eq:tiefpass:phase} gezeigt ist.
    }%
    
    \b{\textbf{Amplituden- und Phasengang:}
        \begin{align*}
            A(\omega)       &= |\underline{F}(\mathrm{j}\omega)| = \frac{|\underline{U}_2|}{|\underline{U}_1|}
                            &= \frac{1}{\sqrt{1 + (\omega CR)^2}} \\%
            \varphi(\omega) &= \arctan\left(\frac{\Im\{\underline{F}\}}{\Re\{\underline{F}\}}\right)
                            &= \arctan\left(-\omega CR\right)        
        \end{align*}
    }
    \s{% Ohne Überschrift, Rechnung etwas ausführlicher
        \begin{align}
            A(\omega)       &= |\underline{F}(\mathrm{j}\omega)| = \left|\frac{1}{1+\mathrm{j}\omega CR}\right|
                                &&= \frac{\left|\qquad\!1\!\qquad\right|}{\left|1+\mathrm{j}\omega CR\right|} % Extra step
                            &= \frac{1}{\sqrt{1 + (\omega CR)^2}} \label{eq:tiefpass:ampli}\\
            \varphi(\omega) &= \angle \underline{F}(\mathrm{j}\omega) = \angle\left(\frac{1 - \mathrm{j}\omega CR}{\redcancel{1+(\omega CR)^2}}\right)
                                &&= \arctan\left( \dfrac{ \dfrac{-\omega CR}{\redcancel{1+(\omega CR)^2}} }{ \dfrac{1}{\redcancel{1+(\omega CR)^2}} } \right) % Extra step
                            &= \arctan\left(-\omega CR\right) \label{eq:tiefpass:phase}
        \end{align}

        Alternativ kann der Phasengang auch aus der nicht kartesischen Form des Frequenzganges bestimmt werden. 
        Der Phasengang ergibt sich aus der Phasenlage des Zählers subtrahiert mit der Phasenlage des Nenners: 
        
        \begin{align}
            \varphi(\omega) 
            &= \angle \left(\frac{1}{1+\mathrm{j}\omega CR}\right) 
            = \underbrace{\arctan\left(\frac{0}{\redcancel{1}}\right)}_{\varphi_{Zaehler}}-\underbrace{\arctan\left(\frac{\omega CR}{\redcancel{1}}\right)}_{\varphi_{Nenner}}
            &= \arctan\left(-\omega CR\right) \label{eq:tiefpass:phase:alternative}
        \end{align}

        Die Herangehensweise bietet sich vor allem an, wenn sich Betrag und/oder Phase des Zählers direkt ablesen lassen. 
        Dadurch kann ein Rechenschritt zum Umrechnen in kartesische Koordinaten gespaart werden.

        Im folgenden Exkurs wird die seperate Betrachtung von Zähler und Nenner von Frequenzgängen näher erläutert:

        \textbf{Exkurs Berechnung komplexer Zahlen in (kartesischen) Descartes- und Polar-Koordinaten}

        Der Frequenzgang nimmt in der Regel die Form eines komplexen gebrochen rationalen Bruches an. 
        Das bedeutet, dass sowohl Nenner als auch Zähler sich als komplexe Polynome darstellen lassen:
        
        \begin{equation}
            \label{eq:def:F:komplexgebrochenrationalerbruch}
            \underline{F}(\mathrm{j}\omega) 
                = \frac{\underline{F}_Z(\mathrm{j}\omega)}{\underline{F}_N(\mathrm{j}\omega)}
                = \frac{a_n \omega^n + \dots + a_1 \omega + a_0}{b_m \omega^m + \dots + b_1 \omega + b_0} \qquad \text{mit} 
                \qquad a_i, b_j \in \mathbb{C} 
                \qquad n,m,\omega \in \mathbb{R} 
                %\qquad i \in \left[0,...,n\right] 
                %\qquad j \in \left[0,...,m\right]
        \end{equation}
        
        Liegt $\underline{F}(\mathrm{j}\omega)$ nun als komplexer Bruch vor mit Zähler und Nenner in kartesischen Koordinaten, 
        so können wir $A(\omega)$ und $\varphi(\omega)$ durch seperate Betragsbildung und Phasenberechnung von Zähler und Nenner bestimmen.
        Hierfür werden die Formeln aus \eqref{eq:def:A} und \eqref{eq:def:phi} für Zähler (Index $Z$) und Nenner (Index $N$) angewandt. 
        Da Amplituden- und der Phasengang wie in \eqref{eq:def:Fpolar} gezeigt die Polar-Koordinaten (Amplitude und Phase) des Frequenzganges 
        darstellen ergibt sich folgendes Berechnungsschema:

        \begin{align}
            \label{eq:def:F:bruch}
            \underline{F}(\mathrm{j}\omega) 
                            & = \frac{\underline{F}_Z(\mathrm{j}\omega)}{\underline{F}_N(\mathrm{j}\omega)}&
                            & = \frac{|\underline{F}_Z| \cdot \mathrm{e}^{\mathrm{j}\angle{\underline{F}_Z}}}{|\underline{F}_N| \cdot \mathrm{e}^{\mathrm{j}\angle{\underline{F}_N}}}&
                            & = \frac{A_Z}{A_N} \cdot \mathrm{e}^{\mathrm{j} (\varphi_Z - \varphi_N)}
                            \vphantom{\Bigg|}\\
            \label{eq:def:A:bruch}
            A(\omega)       & = |\underline{F}(\mathrm{j}\omega)|&
                            & = \frac{|\underline{F}_Z|}{|\underline{F}_N|}&
                            & = \frac{A_Z}{A_N}&
                            \vphantom{\Bigg|}\\
            \label{eq:def:phi:bruch}
            \varphi(\omega) & = \angle \underline{F}(\mathrm{j}\omega)&
                            & = \angle{\underline{F}_Z} - \angle{\underline{F}_N}&
                            & = \varphi_Z - \varphi_N
        \end{align}

        Im Allgemeinen bietet sich für Multiplikationen und Divisionen komplexer Zahlen die Berechnung in Polarkoordinaten (mit Betrag und Argument) an.
        Eine Multiplikation/Division führt bei komplexen Zahlen zur Multiplikation/Division der Beträge und Addition/Subtraktion der Argumente.
        Bei Additionen und Subtraktionen eignet sich die kartesische Form mit Real- und Imaginärteil besser für Berechnungen, 
        da diese direkt addiert/subtrahiert werden können.
    }
\end{columns}
    \CMT{% COMMENT ANFANG
    \begin{align}
        \underline{z}   
        &= \Re\{\underline{z}\} + \mathrm{j} \Im\{\underline{z}\}  &
            &= x + \mathrm{j}y &
            &\text{mit}\quad x,y \in \mathbb{R} \\
        &= |\underline{z}|\cdot \mathrm{e}^{\mathrm{j}\angle{\underline{z}}} &
            &= a \cdot \mathrm{e}^{\mathrm{j} \beta} &
            &\text{mit}\quad a,\beta \in \mathbb{R}\\
    \end{align}
    }% COMMENT ENDE
\end{frame}

%%%%%%%%%%%%%%%%%%%%%%%%%%%%%%%%%%%%%%%%%%%%%%%%%%%%%%%%%%%%%%%%%%%%%%%%%%%%%%%%%%%

\subsection{Grenzverhalten am Beispiel eines einfachen Tiefpass-Filters}
\label{sec:frequenzgang:tiefpass:grenzverhalten}
\begin{frame}[t]\ftx{\subsecname}
\b{
    \begin{columns}[T]
        \column{0.45\textwidth}% column links 1/2
            {\centering%
            \vspace{-7.5mm}%
            \onslide<1->{\includegraphics{Tikz/pdf/circ_tiefpass_rc_annotated.pdf}}%
            \vspace{2mm}%

            \onslide<6->{\resizebox{\textwidth}{!}{\includegraphics{Tikz/pdf/plot_tiefpass_mini_ampli_lin_samesize.pdf}}}

            \onslide<6->{\resizebox{\textwidth}{!}{\includegraphics{Tikz/pdf/plot_tiefpass_mini_phase_lin_samesize.pdf}}}
            }%
        \column{0.55\textwidth}% column rechts 2/2
            \b{\textbf{Grenzverhalten:} für $f \rightarrow 0$ und $f \rightarrow \infty$\\[-12pt]%
            }%
            % Rechnung
            \begin{align*}
                \onslide<1->{\underline{F}(\mathrm{j}\omega)&=A(\omega)\cdot\mathrm e^{\mathrm j\varphi(\omega)} = \frac{1}{1 + \mathrm{j}\omega CR} &&}\\[+8pt] % extra space
                \onslide<2->{A(\omega)      &=\frac{1}{\sqrt{1 + (\omega CR)^2}}                                    &&\\}%
                \onslide<3->{%
                                            &\lim_{\omega \rightarrow 0}A(\omega) = \frac{1}{1}                     &&=1\\
                                            &\lim_{\omega \rightarrow \infty}A(\omega) = \frac{1}{\sqrt{\infty}}    &&=0\\[+8pt] % extra space
                }%
                \onslide<4->{\varphi(\omega)&=\arctan\left(-\omega CR\right)                                        &&\\}%
                \onslide<5->{%
                                            &\lim_{\omega \rightarrow 0}         \varphi(\omega) = \arctan(0)       &&=0\degree\\
                                            &\lim_{\omega \rightarrow \infty}    \varphi(\omega) = \arctan(-\infty) &&=-90\degree
                }%   
            \end{align*}
    \end{columns}
}
\s{
    In diesem Kapitel wird das Grenzverhalten des Tiefpass-Filters 1. Ordnung aus dem vorigen Kapitel (s. Abbildung \ref{circ:filter:rctief}) für sehr hohe und sehr niedrige Frequenzen untersucht.
    Dazu werden die Grenzwerte des Amplitudengangs $A(\omega)$ und des Phasengangs $\varphi(\omega)$ bestimmt, sowie die Ergebnisse beschrieben, interpretiert und grafisch dargestellt.
    
    \textbf{Grenzwerte:}

    Für den Amplitudengang aus Gleichung \ref{eq:tiefpass:ampli} ergeben sich
    für $f \rightarrow 0$ und $f \rightarrow \infty$ die folgenden Grenzwerte:

    \begin{align}
        &&A(\omega) &=\frac{1}{\sqrt{1 + (\omega CR)^2}}    &                                           &                           &   &    &&\nonumber\\
        &&          &                                       &   \lim_{\omega \rightarrow 0}A(\omega)    &=\frac{1}{1}               &   &=1  &&\label{eq:tiefpass:ampli:lim}\\
        &&          &                                       &   \lim_{\omega \rightarrow \infty}A(\omega)&=\frac{1}{\sqrt{\infty}}  &   &=0  &&\nonumber
    \end{align}
    %\\\intertext{%
    %    Für den Phasengang aus Gleichung \ref{eq:tiefpass:phase} ergeben sich analog dazu die Grenzwerte:%
    %}%
    Für den Phasengang aus Gleichung \ref{eq:tiefpass:phase} ergeben sich analog dazu die Grenzwerte:
    \begin{align}
        &&\varphi(\omega)   &=\arctan\left(-\omega CR\right)&                                                   &                   &   &           &&\nonumber\\
        &&                  &                               &   \lim_{\omega \rightarrow 0}\varphi(\omega)      &=\arctan(0)        &   &=0\degree  &&\label{eq:tiefpass:phase:lim}\\
        &&                  &                               &   \lim_{\omega \rightarrow \infty}\varphi(\omega) &=\arctan(-\infty)  &   &=-90\degree&&\nonumber
    \end{align}

    An Gleichung \ref{eq:tiefpass:ampli:lim} ist zu erkennen, dass der Tiefpass bei sehr niedrigen Frequenzen kaum dämpft ($A(\omega) \rightarrow 1$), 
    bei sehr hohen Frequenzen hingegen stark dämpft ($A(\omega) \rightarrow 0$). 
    
    An Gleichung \ref{eq:tiefpass:phase:lim} ist zu erkennen, dass der Tiefpass bei sehr niedrigen Frequenzen (bei kaum Dämpfung) kaum Phasenverschiebung aufweist ($\varphi(\omega) \rightarrow 0\degree$),
    bei sehr hohen Frequenzen (starke Dämpfung) hingegen eine Phasenverschiebung bis $-90\degree$ aufweist. 

    \textbf{Erklärung des Grenzverhaltens anhand des Schaltbildes:}

    Das Grenzverhalten bezüglich Amplitudenänderung und Phasenverschiebung lässt sich gut anhand des 
    Schaltbildes \ref{circ:filter:rctief} erklären. % TODO: Schaltbilder für Grenzfälle mit Spannungspfeilen und Impedanzen einfügen
    
    Im Grenzfall einer Gleichspannung ($f=0$) am Eingang sperrt die Kapazität ($X_C \rightarrow \infty$). 
    Ihr Verhalten entspricht hier zwei offenen Klemmen, wodurch die gesamte Eingangsspannung am Ausgang $U_2$ über der Kapazität $C$ anliegt. 
    Die relative Amplitudenänderung beträgt daher $1$ (ungedämpft) und die Phasenverschiebung $0\degree$ (in Phase). 

    Bei sehr hohen Frequenzen ($f\rightarrow \infty$) verhält sich die Kapazität wie ein Kurzschluss ($X_C \rightarrow 0$).
    Damit geht die Ausgangsspannung gegen null (starke Dämpfung, $A\rightarrow 0$) und die Eingangsspannung liegt annäherend vollständig über $R$ an.
    Eingangsspannung und Strom liegen dadurch näherungsweise in Phase. Die Phasenverschiebung von Ausgangsspannug über der Kapazität
    zu Eingangsspannung beträgt dadurch näherungsweise $-90\degree$.
    
    \textbf{Visualisierung:}

    % Hinweis zur Visualisierung der Funktionen
    Um eine Vorstellung des Verhaltens von Amplituden- und Phasengang zu bekommen, können wir die Funktionen graphisch darstellen. 

    Der Amplitudengang $A(\omega)$ und der Phasengang $\varphi(\omega)$ sind in 
    Abbildung \ref{plot:tiefpass:ampli:mini:lin} und \ref{plot:tiefpass:phase:mini:lin} dargestellt.
    Beide Graphen sind mit linearen Skalen dargestellt, wobei die Skalierungen beider x-Achsen für $\omega \left[\mathrm{Hz}\right]$ identisch sind. 
    Eine Veränderung des Faktors $CR$ führt in linearer Darstellung lediglich zu einer Stauchung oder Streckung der Graphen in x-Richtung, 
    weshalb auf eine explizite Angabe der Werte verzichtet wurde. 

    \begin{figure}[h]\centering
        \begin{subfigure}{0.45\textwidth}\centering
            \includegraphics{Tikz/pdf/plot_tiefpass_mini_ampli_lin_samesize.pdf}
            \caption{Amplitudengang $A(\omega)$, lineare Achsen}
            \label{plot:tiefpass:ampli:mini:lin}
        \end{subfigure}
        \begin{subfigure}{0.45\textwidth}\centering
        \includegraphics{Tikz/pdf/plot_tiefpass_mini_phase_lin_samesize.pdf}
            \caption{Phasengang $\varphi(\omega)$, lineare Achsen}
            \label{plot:tiefpass:phase:mini:lin}
        \end{subfigure}
        \caption{Lineardarstellung des Amplitudengang und Phasengang eines Tiefpass 1. Ordnung}
%        \caption{Amplitudengang $A(\omega)$ und Phasengang $\varphi(\omega)$ eines Tiefpass 1. Ordnung, lineare Achsen}
        \label{plot:tiefpass:mini:lin}
    \end{figure}

    % Hinweis: Logarithmische Skala für Frequenzgang
    In Kapitel \ref{sec:frequenzganglog} wird die Darstellung des Frequenzganges mit logarithmischer Skala eingeführt.
    Auch die Definition der \textbf{Grenzfrequenz} $f_g$ (respektive der Grenzkreisfrequenz $\omega_g$) zur Unterteilung 
    des Frequenzbereichs in einen \textbf{Durchlassbereich} und einen \textbf{Sperrbereich} wird dort erläutert.

    \textbf{Exkurs zur Herleitung der Graphenform:}
    % Erklärung der Graphenform für A(omega) anhand von Näherungen (Potenzfunktion)

    Die Graphenform des Amplitudengangs $A(\omega)$  in linearer Darstellung lässt sich daran erklären, dass dieser sich 
    für sehr niedrige Frequenzen ($f \rightarrow 0$) der waagrechten Gerade $A=1$ annähert und 
    für sehr hohe Frequenzen ($f \rightarrow \infty$) der Potenzfunktion $\frac{1}{\omega CR}$ annähert.
    Abbildung \ref{plot:tiefpass:ampli:lin:approx} zeigt diese Näherungen für $A(x)=\frac{1}{\sqrt{1+x^2}}$ mit $x=\omega CR$.

    \fu{\includegraphics{Tikz/pdf/plot_tiefpass_ampli_lin_approx.pdf}}{
        Amplitudengang Tiefpass 1. Ordnung, lineare Achsen, Näherungen mit $x=\omega CR$ \label{plot:tiefpass:ampli:lin:approx}
    }

    % Erklärung der Graphenform für phi(omega) anhand von Tangensfunktion
    Die Form des Phasenganges lässt sich leicht aus der Form einer (Arcus-)Tangensfunktion ableiten.
    Abbildung \ref{plot:tiefpass:phase:lin:approx} zeigt die Tangensfunktion und dessen Umkehrfunktion Arcustangens, sowie deren 
    jeweiligen Spiegelungen an der y-Achse in linearer Darstellung.
    
    \fu{\includegraphics{Tikz/pdf/plot_tan_atan_mirror.pdf}}{
        Phasengang Tiefpass 1. Ordnung, lineare Achsen, Vergleich Tangens/Arcustangens \label{plot:tiefpass:phase:lin:approx}
    }

    Die Umkehrung der jeweiligen Funktion ist durch Spiegelung des Graphen an der Winkelhalbierenden 
    (schwarz gestrichelte Gerade) und Tausch der x- und y-Achse erreichbar,
    vorrausgesetzt die jeweilige Funktion ist stetig monoton wachsend oder stetig monoton fallend. 
    Der Tangens ist daher zur Bildung des Arcustangens auf den Winkelbereich $-90\degree$ bis $+90\degree$ beschränkt 
    bei einem Wertebereich von $-\infty$ bis $+\infty$.
    Aufgrund der nicht darstellbaren Polstellen ist der dargestellte Wertebereich auf $-6$ bis $+6$ beschränkt.

    Der Phasengang des Tiefpass 1. Ordnung entspricht dem abgebildeten $\arctan(-x)$ für $x=\omega CR >0$. 
    Da es physikalisch keine negativen Frequenzen gibt, gilt $\omega>0 \rightarrow x>0$, wodurch der Wertebereich des Phasengangs 
    nur den halben Wertebereich des Arcustangens abdeckt, also von $-90\degree$ bis $0\degree$.
}%
\end{frame}

%%%%%%%%%%%%%%%%%%%%%%%%%%%%%%%%%%%%%%%%%%%%%%%%%%%%%%%%%%%%%%%%%%%%%%%%%%%%%%%%%%%

\subsection{Frequenzgang am Beispiel eines einfachen Hochpass-Filters}
\label{sec:frequenzgang:hochpass:herleitung}
\begin{frame}\ftx{\subsecname}
    \begin{columns}
        \column[T]{0.45\textwidth}% left col
            \s{\index{Filter>Hochpass}%
                Ein Hochpass-Filter (kurz Hochpass) funktioniert ähnlich einem Tiefpass-Filter. 
                Dem Namen entsprechend passieren beim Hochpass jedoch Signale hoher Frequenzen nahezu ungedämpft (ungefiltert), 
                während Signale niedriger Frequenzen stark gedämpft (gefiltert) werden.  

                % Aufbau (Schaltbild)
                Abbildung \ref{circ:filter:rchoch} zeigt exemplarisch einen RC-Hochpass erster Ordnung, 
                der im Folgenden näher untersucht wird.
                Der Hochpass besteht aus einer einfach Serienschaltung von Widerstand $R$ und Kapazität $C$. 
                Die Eingangsspannung $\underline{U}_1$ liegt über $R$ und $C$ in Reihe an, 
                während die Ausgangsspannung $\underline{U}_2$ über $R$ abgegriffen wird.
                Der Aufbau ist identisch mit dem RC-Tiefpass erster Ordnung, 
                nur dass die Ausgangsspannung über $R$ statt über $C$ anliegt.
   
                \fu{\includegraphics{Tikz/pdf/circ_hochpass_rc_annotated.pdf}}{Schaltbild, RC-Hochpass 1. Ordnung\label{circ:filter:rchoch}}%

                Die Herleitung des Frequenzganges erfolgt analog zum Vorgehen beim Tiefpass 
                in Kapitel \ref{sec:frequenzgang:tiefpass:herleitung}.
                Da es sich beim Hochpass um ein LZI-System handelt gilt: 
                \begin{equation}
                \begin{aligned}
                                        u_1(t)& = \hat{U}_1 \cdot \sin(\omega t)\\
                    \xrightarrow{LZI}   u_2(t)& = \hat{U}_2 \cdot \sin(\omega t + \varphi)
                \end{aligned}%
                \end{equation}
            }%
            \b{%
                {\centering
                \vspace{-7.5mm}%
                \includegraphics{Tikz/pdf/circ_hochpass_rc_annotated.pdf}%
                }\vspace{3mm}%

                \textbf{RC-Hochpass 1. Ordnung}

                \begin{align*}% less space above than \begin{equation}\begin{align} ... as in skript version above
                                        u_1(t)& = \hat{U}_1 \cdot \sin(\omega t)\\
                    \xrightarrow{LZI}\  u_2(t)& = \hat{U}_2 \cdot \sin(\omega t + \varphi)
                \end{align*}%
            }%

        \column[t]{0.55\textwidth}% right col
            \s{%
                Das heißt, bei sinusförmiger Eingangsspannung $u_1(t)$, ist auch die Ausgangsspannung $u_2(t)$ 
                sinusförmig mit gleicher Frequenz wie die Eingangsspannung.

                Der Frequenzgang lässt sich dadurch gemäß Gleichung \ref{eq:def:F} bestimmen,
                indem die komplexe Wechselspannungen $\underline{U}_2$ am Ausgang und $\underline{U}_1$ am Eingang 
                ins Verhältnis gesetzt werden. Dadurch erhalten wir:

                \begin{equation}
                    \begin{aligned}
                        \underline{F}(\mathrm{j}\omega)   & = \frac{\underline{U}_2}{\underline{U}_1} 
                                                = \frac{R}{R + \frac{1}{\mathrm{j}\omega C}} 
                                                = \frac{1}{1 + \frac{1}{\mathrm{j}\omega CR}} \\
                                                & = \frac{1}{1 - \mathrm{j}\ \frac{1}{\omega CR}} 
                                                = \frac{1 + \mathrm{j}\ \frac{1}{\omega CR}}{1 + \frac{1}{(\omega CR)^2}}
                    \end{aligned}\label{eq:hochpass:frequenzgang}
                \end{equation}
            }%
            \b{%
                \textbf{Frequenzgang:}%
                \begin{align*}% lesser space above than \begin{equation}\begin{align} ... as in skript version below
                    \underline{F}(\mathrm{j}\omega)   & = \frac{\underline{U}_2}{\underline{U}_1} 
                                            = \frac{R}{R + \frac{1}{\mathrm{j}\omega C}} 
                                            = \frac{1}{1 + \frac{1}{\mathrm{j}\omega CR}} \\
                                            & = \frac{1}{1 - \mathrm{j}\ \frac{1}{\omega CR}} 
                                            = \frac{1 + \mathrm{j}\ \frac{1}{\omega CR}}{1 + \frac{1}{(\omega CR)^2}}
                \end{align*}
            }%
            \s{%
                Für den Amplitudengang $A(\omega)$ und den Phasengang $\varphi(\omega)$ ergeben sich die folgenden Gleichungen:
            
                \begin{align}
                    A(\omega)       &= |\underline{F}(\mathrm{j}\omega)| = \left|\frac{1}{1-\mathrm{j}\frac{1}{\omega CR}}\right|
                                    &= \frac{1}{\sqrt{1 + \frac{1}{(\omega CR)^2}}} \label{eq:hochpass:ampli}\\
                    \varphi(\omega) &= \angle \underline{F}(\mathrm{j}\omega) = \angle\left(\frac{1 + \mathrm{j}\frac{1}{\omega CR}}{1 + \frac{1}{(\omega CR)^2}}\right)
                                    &= \arctan\left(\frac{1}{\omega CR}\right)\label{eq:hochpass:phase}
                \end{align}
            }%
            \b{%
                \textbf{Amplituden- und Phasengang:}
                \begin{align*} % Leicht gekürzt gegenüber Skriptversion
                    A(\omega)       &= |\underline{F}(\mathrm{j}\omega)| = \frac{|\underline{U}_2|}{|\underline{U}_1|}
                                    &= \frac{1}{\sqrt{1 + \frac{1}{(\omega CR)^2}}} \\
                    \varphi(\omega) &= \arctan\left(\frac{\Im\{\underline{F}\}}{\Re\{\underline{F}\}}\right)
                                    &= \arctan\left(\frac{1}{\omega CR}\right)\\
                \end{align*}
            }%
    \end{columns}
\end{frame}

%%%%%%%%%%%%%%%%%%%%%%%%%%%%%%%%%%%%%%%%%%%%%%%%%%%%%%%%%%%%%%%%%%%%%%%%%%%%%%%%%%%

\subsection{Grenzverhalten am Beispiel eines einfachen Hochpass-Filters}
\label{sec:frequenzgang:hochpass:grenzverhalten}
\begin{frame}\ftx{\subsecname}
% beamer links: Schaltbild, lineare Plots A(\omega), \varphi(\omega)
% beamer rechts: Rechnung Grenzverhalten von A und \varphi für f=0 und f->\inf
\b{%
    \begin{columns}[T]
        \column{0.45\textwidth}% column links 1/2
            {\centering\vspace{-4mm}%
            \includegraphics{Tikz/pdf/circ_hochpass_rc_annotated.pdf}%
            \vspace{2mm}

            \resizebox{\textwidth}{!}{\includegraphics{Tikz/pdf/plot_hochpass_mini_ampli_lin_samesize.pdf}}

            \resizebox{\textwidth}{!}{\includegraphics{Tikz/pdf/plot_hochpass_mini_phase_lin_samesize.pdf}}
            }
        \column{0.55\textwidth}% column rechts 2/2
            \textbf{Grenzverhalten:} für $f \rightarrow 0$ und $f \rightarrow \infty$
            % Rechnung
            \begin{align*}
                \underline{F}(\mathrm{j}\omega)&=A(\omega)\cdot\mathrm e^{\mathrm j\varphi(\omega)} = \frac{1}{1 - \mathrm{j}\frac{1}{\omega CR}} &&\\
                &&&\\[-12pt]% little empty line for extra space
                    A(\omega)                   &=\frac{1}{\sqrt{1 + \left(\frac{1}{\omega CR}\right)^2}}               &&\\
                                                &\lim_{\omega \rightarrow 0}A(\omega) = \frac{1}{\infty}                &&=0\\
                                                &\lim_{\omega \rightarrow \infty}A(\omega) = \frac{1}{1}                &&=1\\
                                                &&&\\[-12pt]% little empty line for extra space
                    \varphi(\omega)&=\arctan\left(\frac{1}{\omega CR}\right)                                            &&\\
                                                &\lim_{\omega \rightarrow 0}         \varphi(\omega) = \arctan(+\infty) &&=+90\degree\\
                                                &\lim_{\omega \rightarrow \infty}    \varphi(\omega) = \arctan(0)       &&=0\degree
            \end{align*}
    \end{columns}
}%
\s{%
    Das Grenzverhalten des Hochpass-Filters 1. Ordnung aus Kapitel \ref{sec:frequenzgang:hochpass:herleitung} 
    wird in diesem Kapitel für sehr hohe und sehr niedrige Frequenzen untersucht.
    Die Grenzwertbestimmung des Amplitudengangs $A(\omega)$ und des Phasengangs $\varphi(\omega)$ erfolgt analog 
    zu der des Tiefpass-Filter wie in Kapitel \ref{sec:frequenzgang:tiefpass:grenzverhalten} beschrieben. 
    
    \textbf{Grenzwerte:}

    Für den Amplitudengang aus Gleichung \ref{eq:hochpass:ampli} ergeben sich die folgenden Grenzwerte 
    für $f \rightarrow 0$ und $f \rightarrow \infty$:
    
    \begin{align}
        &&A(\omega) &=\frac{1}{\sqrt{1 + \left(\frac{1}{\omega CR}\right)^2}}&  &                   &&          &&\nonumber\\
        &&          &&  \lim_{\omega \rightarrow \infty}A(\omega)               &=\frac{1}{\infty}  &&=0        &&\\
        &&          &&  \lim_{\omega \rightarrow 0}A(\omega)                    &=\frac{1}{\sqrt{1}}&&=1        &&\nonumber
    \end{align}
    %\\\intertext%
    \begin{align}
        &&\varphi(\omega)   &=\arctan\left(\frac{1}{\omega CR}\right)&          &                   &&          &&\nonumber\\
        &&          &&  \lim_{\omega \rightarrow 0}\varphi(\omega)              &=\arctan(\infty)   &&=+90\degree&&\\
        &&          &&  \lim_{\omega \rightarrow \infty}\varphi(\omega)          &=\arctan(0)        &&=0\degree &&\nonumber
    \end{align}

    Das heißt, bei sehr niedrigen Frequenzen dämpft der Hochpass stark ($A(\omega) \rightarrow 0$) mit Phasenverschiebung bis $+90\degree$
    und bei sehr hohen Frequenzen kaum ($A(\omega) \rightarrow 1$) mit Phasenverschiebung bis $0\degree$.

    Anhand des Schaltbildes \ref{circ:filter:rchoch} lässt sich dieses Grenzverhalten gut erklären.
    Bei Gleichspannung ($f=0$) sperrt die Kapazität, wodurch die gesamte Eingangsspannung über dieser abfällt.
     
    Bei sehr kleinen Frequenzen ($f \to 0$) liegt die Eingangsspannung näherungsweise vollständig über der Kapazität an. 
    Der Strom eilt der Eingangsspannung bis $+90\degree$ voraus, wodurch sich eine positive Phasenverschiebung 
    bis $+90\degree$ zur Ausgangsspannung über dem Widerstand ergibt.

    Bei sehr hohen Frequenzen ($f \rightarrow \infty$) verhält sich die Kapazität wie ein Kurzschluss, 
    wodurch die Eingangsspannung näherungsweise vollständig über dem Widerstand am Ausgang anliegt.
    Dadurch sind Ausgangs- und Eingangsspannung näherungsweise gleich in Amplitude und in Phase.

    \textbf{Visualisierung:}

    Abbildung \ref{plot:hochpass:mini:lin} zeigt den Amplitudengang $A(\omega)$ und den Phasengang $\varphi(\omega)$ des Hochpass 1. Ordnung in linearer Darstellung
    mit gleicher Skalierung und gleichem Wertebereich für die x-Achsen.

    \begin{figure}[h]\centering
        \begin{subfigure}{0.45\textwidth}\centering
            \includegraphics{Tikz/pdf/plot_hochpass_mini_ampli_lin_samesize.pdf}
            \caption{Amplitudengang $A(\omega)$, lineare Achsen}
            \label{plot:hochpass:ampli:mini:lin}
        \end{subfigure}
        \begin{subfigure}{0.45\textwidth}\centering
        \includegraphics{Tikz/pdf/plot_hochpass_mini_phase_lin_samesize.pdf}
            \caption{Phasengang $\varphi(\omega)$, lineare Achsen}
            \label{plot:hochpass:phase:mini:lin}
        \end{subfigure}
        \caption{Lineardarstellung des Amplitudengang und Phasengang eines Hochpass 1. Ordnung}
%        \caption{Amplitudengang $A(\omega)$ und Phasengang $\varphi(\omega)$ eines Hochpass 1. Ordnung, lineare Achsen}
        \label{plot:hochpass:mini:lin}
    \end{figure}

    Eine logarithmische Darstellung des Frequenzganges wird in Kapitel \ref{sec:frequenzganglog:vergleichtiefpasshochpass} eingeführt.
}%
\end{frame}

            
%%%%%%%%%%%%%%%%%%%%%%%%%%%%%%%%%%%%%%%%%%%%%%%%%%%%%%%%%%%%%%%%%%%%%%%%%%%%%%%%%%%

\subsection{Vergleich passive Filter 1. Ordnung mit $L$ und $R$}
\label{sec:frequenzgang:rlfilter}
\begin{frame}\ftx{\subsecname}
\s{%
    Tiefpass und Hochpass lassen sich auch mit einer Induktivität $L$ und einem Widerstand $R$ realisieren. 
    In Tabelle \ref{tab:filter:rl:vergleich:circeq} sind die Schaltbilder für RL-Filter 1. Ordnung 
    und ihre RC-Pendants gegenübergestellt. Zudem ist zu jedem Schaltbild die Formel für den Frequenzgang aufgeführt.
}%
% BRAILLE TABELLE als Kommentarblock
% Titel: Vergleich Schaltbild und Frequenzgang für RL- und RC-Filter 1. Ordnung
% 3 Zeilen, 6 Spalten mit $w=\omega$:
\b{\s{%
\begin{tabular}{|c|cc|cc|c|}%
\hline
. % Header
    & Tiefpass
    & Frequenzgang
    & Hochpass
    & Frequenzgang
    & Grenzfrequenz \\
\hline
RL % Zeile 1
    & TP U1 über RL mit U2 über R
    & TP $F(jw)=1/(1+j*wL/R)$
    & HP U1 über RL mit U2 über L
    & HP $F(jw)=1/(1-j*R/(wL))$
    & $w_g = R/L$ \\
\hline
RC % Zeile 2
    & TP U1 über RC mit U2 über C
    & TP $F(jw)=1/(1+j*wRC)$
    & HP U1 über RC mit U2 über R
    & HP $F(jw)=1/(1-j/(wRC))$
    & $w_g = 1/(RC)$ \\
\hline
\end{tabular}%
}}

% NICHT BRAILLE TABELLE zum TeXen
\begin{table}[h]\centering
    \s{\caption{Vergleich Schaltbild und Frequenzgang für RL- und RC-Filter 1. Ordnung}}%
    \label{tab:filter:rl:vergleich:circeq}
    \resizeboxsb{1}{1}{% 90% textwidth skript, 100% textwidth beamer
        \begin{tabular}{|c|c l|c l|c|}
            \hline$\vphantom{\Big|}$ % empty cell, vphantom padding for rowheight
                &   \textbf{Tiefpass}
                &   $\quad\underline{F}(\mathrm{j}\omega)$
                &   \textbf{Hochpass}
                &   $\quad\underline{F}(\mathrm{j}\omega)$
                &   $\omega_g$
            \\\hline\parbox[c][2cm][c]{0.3cm}{\rotatebox[origin=c]{90}{\textbf{RL}}}
                &   \parbox[c][2cm][c]{3.0cm}{\resizebox{\linewidth}{!}{\includegraphics{Tikz/pdf/circ_tiefpass_rl_annotated.pdf}}} % resizebox breaks tikz without parbox around
                &   \parbox[c][2cm][c]{1.7cm}{$\dfrac{1}{1+\mathrm{j} \omega \frac{L}{R}}$}
                &   \parbox[c][2cm][c]{3.0cm}{\resizebox{\linewidth}{!}{\includegraphics{Tikz/pdf/circ_hochpass_rl_annotated.pdf}}}
                &   \parbox[c][2cm][c]{1.7cm}{$\dfrac{1}{1-\mathrm{j}\frac{R}{\omega L}}$}
                &   \parbox[c][2cm][c]{0.5cm}{$\dfrac{R}{L}$}
            \\\hline\parbox[c][2cm][c]{0.3cm}{\rotatebox[origin=c]{90}{\textbf{RC}}}
                &   \parbox[c][2cm][c]{3.0cm}{\resizebox{\linewidth}{!}{\includegraphics{Tikz/pdf/circ_tiefpass_rc_annotated.pdf}}}
                &   \parbox[c][2cm][c]{1.7cm}{$\dfrac{1}{1+\mathrm{j}\omega CR}$}
                &   \parbox[c][2cm][c]{3.0cm}{\resizebox{\linewidth}{!}{\includegraphics{Tikz/pdf/circ_hochpass_rc_annotated.pdf}}}
                &   \parbox[c][2cm][c]{1.7cm}{$\dfrac{1}{1-\mathrm{j}\frac{1}{\omega CR}}$}
                &   \parbox[c][2cm][c]{0.5cm}{\centering $\dfrac{1}{CR}$}
            \\\hline
        \end{tabular}
    }%
\end{table}

\s{%
    Im Vergleich der Schaltbilder ist zu erkennen, dass die RL- und RC-Varianten sich lediglich in der Anordnung von $L$ und $R$
    beziehungsweise $R$ und $C$ unterscheiden. 
    Die Frequenzgänge der RL- und RC-Varianten unterscheiden sich nur im Faktor $\dfrac{L}{R}$ und $CR$ vor $\omega$ voneinander.
    Wie in Kapitel \ref{sec:frequenzganglog:grenzfrequenz} noch näher erläutert wird, entspricht dieser Faktor dem Kehrwert der Grenzkreisfrequenz 
    $\omega_g$, welche vollständigkeitshalber mit in der Tabelle aufgeführt ist.
}%

\end{frame}

%-------------------------------------------------------------------------------------------------%
% Aufgabenstellung + Schaltbild in einem Frame:

\subsection{Beispiel Filterschaltung: Aufgabe}
\begin{frame}\ftx{\subsecname}
\begin{bsp}{Filterschaltung: Aufgabe}{}% label as second madatory argument is printed in Folien, can't escape with \b{...} -> no label for this bsp
\noindent%
\begin{minipage}{\textwidth}
    \begin{minipage}[c]{0.44\textwidth}
        \fu{\includegraphics{Tikz/pdf/circ_aufgabe_filter.pdf}}{Filterschaltung, passiv}% Schaltbild
        \b{\vfill}%
        \begin{align*}
        u_1(t) &= \hat{U}_1\cdot \sin(\omega t)             & \underline{U}_1 &= \hat{U}_1\\
        u_2(t) &= \hat{U}_2\cdot \sin(\omega t + \varphi)   & \underline{U}_2 &= \hat{U}_2 \cdot \mathrm{e}^{\mathrm{j}\varphi}
        \end{align*}
        \b{\vfill}
    \end{minipage}\hfill
    \begin{minipage}[c]{0.52\textwidth}
        \textbf{Aufgabe:}\\\b{\vfill}
        a) Leiten Sie $\underline{F}(\mathrm{j}\omega) = \frac{\underline{U}_2}{\underline{U}_1}$ allgemein her.\\
        \b{\vfill}
        b) Zeichnen Sie Betrag und Phase von $\underline{F}(\mathrm{j}\omega)$ für:
        \begin{itemize}
            \item[] $R_1 = 900\ \Omega$
            \item[] $R_2 = 100\ \Omega$
            \item[] $C = 1,25\ \mu \mathrm{F}$
        \end{itemize}
        \b{\vfill}
        c) Bei welcher Frequenz $f$ ist $\hat{U}_2 = \frac{1}{2}\hat{U}_1$? \\
        Welche Phasenverschiebung besitzt $u_2$ nun gegenüber $u_1$?
    \end{minipage}	
\end{minipage}
\end{bsp}%
\end{frame}%
%
%-------------------------------------------------------------------------------------------------%
%
\begin{frame}[allowframebreaks]\ftx{\subsecname - Lösung}% EVIL: allowed (auto/manual) framebreaks
\begin{bsp}{Filterschaltung: Lösung}{}% label as second madatory argument is printed in Folien, can't escape with \b{...} -> no label for this bsp
\textbf{Lösung a) Frequenzgang allgemein}
    \begin{align*}
        \underline{F}(\mathrm{j}\omega) &= \frac{R_1 || \underline{Z}_C}{R_2 + ( R_1 || \underline{Z}_C)} 
        = \left.\frac{\frac{\frac{R_1}{\mathrm{j}\omega C}}{R_1 + \frac{1}{\mathrm{j}\omega C}}}{
            R_2 + \frac{\frac{R_1}{\mathrm{j}\omega C}}{R_1 + \frac{1}{\mathrm{j}\omega C}}}\quad \right| 
            \cdot \frac{R_1 + \frac{1}{\mathrm{j}\omega C}}{R_1 + \frac{1}{\mathrm{j}\omega C}}
        \\[+2pt]
        &= \frac{\frac{R_1}{\mathrm{j}\omega C}}{R_1\cdot R_2 + \frac{R_2}{\mathrm{j}\omega C}+\frac{R_1}{\mathrm{j}\omega C}}
	    = \highlight{hfid_polar}{\frac{R_1}{R_1+R_2+\mathrm{j}\omega CR_1R_2}}[\hfofflowerb][\hfoffupperb][2]
            &&\hat{=}{} \frac{a+jb}{c+jd} 
            & a,b,c,d &\in \mathbb{R}
        \\[+6pt]
        &= \highlight{hfid_kart}{\frac{R_1(R_1+R_2)-\mathrm{j}\omega CR_1^2R_2}{(R_1+R_2)^2+(\omega CR_1R_2)^2}}[\hfofflowerb][\hfoffupperb][3]
        \quad\text{(kartesisch)}
            &&\hat{=}{} \frac{a'+jb'}{c'}
            & a',b',c' &\in \mathbb{R}
    \end{align*}\vspace{-15pt}%
%--8<---------------cut here---------------start------------->8---%
\b{\framebreak}% MANUAL EVIL PAGEBREAK ...(auto breaks here anyway)
%--8<---------------cut here---------------end--------------->8---%
\b{% F(jw) allg. aus Frame vorher
\begin{equation*}
    \underline{F}(\mathrm{j}\omega) 
    = 
    \underbrace{
        \frac{R_1}{R_1+R_2+\mathrm{j}\omega CR_1R_2}
    }_{\mathrel{
        \hat{=}{} \frac{a+jb}{c+jd}
    }}%
    = 
    \underbrace{%
        \frac{R_1(R_1+R_2)-\mathrm{j}\omega CR_1^2R_2}{(R_1+R_2)^2+(\omega CR_1R_2)^2}
    }_{\mathrel{
        \hat{=}{} \frac{a'+jb'}{c'}
    }}%
\end{equation*}%
}%
    \begin{align*}
        &\text{Betrag:}&
            |A(\omega)| &= \frac{R_1}{\sqrt{(R_1+R_2)^2+(\omega CR_1R_2)^2}}&
            &\text{mit}&
            |\underline{F}(\mathrm{j}\omega)| &= \frac{\sqrt{a^2+b^2}}{\sqrt{c^2+d^2}}\\
        &\text{Phase:}&
            \varphi(\omega) &= \arctan\left(\frac{-\omega C R_1^2R_2}{R_1(R_1+R_2)}\right)&
            &\text{mit}&
            \varphi(\omega) &= \arctan\left(\frac{b'}{a'}\right)\\[+2pt]
        &&
            &= \arctan\left(\frac{-\omega C R_1R_2}{R_1+R_2}\right)\\[-4pt]
        &\text{Altern.:}&
            \varphi(\omega ) &= \arctan(0) - \arctan\left(\frac{\omega CR_1R_2}{R_1+R_2}\right)&
            &\text{mit}&
            %\varphi(\omega) &= \arctan\left(\frac{b}{a}\right) - \arctan\left(\frac{d}{c}\right)% too big
            %\varphi(\omega) &= \varphi_{Z} - \varphi_{N}
            %\varphi(\omega) &= \underbrace{\varphi_{Z}}_{\mathclap{\text{\tiny{Zähler}}}} - \underbrace{\varphi_{N}}_{\mathclap{\text{\tiny{Nenner}}}}
            \varphi(\omega) &= \overbrace{\varphi_{Z}}^{\mathclap{\scriptscriptstyle{\arctan(\frac{b}{a})}\ \ }} - \overbrace{\varphi_{N}}^{\mathclap{\ -\ \scriptscriptstyle{\arctan(\frac{d}{c})}}}
    \end{align*}

%--8<---------------cut here---------------start------------->8---%
\b{\framebreak}% EVIL PAGEBREAK ... didn't want to copy whole box
% Empty line after this, otherwise frame might break later on
%--8<---------------cut here---------------end--------------->8---%

\textbf{Lösung b) Grenzverhalten und Skizze}

    \noindent%
    \begin{minipage}{\textwidth}
        \begin{minipage}[c][6cm][c]{0.48\textwidth}

            $A(\omega) = \frac{R_1}{\sqrt{(R_1+R_2)^2+(\omega CR_1R_2)^2}}$ \\

            \qquad $\ \ \ \ \lim \limits_{\omega \to 0} A(\omega) = \frac{R_1}{R_1+R_2} = 0,9$\\
            
            \qquad $\ \ \ \ \lim \limits_{\omega \to \infty} A(\omega) = \frac{R_1}{\infty} = 0$\\
            
            $\varphi(\omega) = \arctan(-\frac{\omega CR_1R_2}{R_1+R_2})$\\

            \qquad $\lim \limits_{\omega \to 0} \varphi(\omega) = \arctan(0) = 0$\\

            \qquad $\lim \limits_{\omega \to \infty} \varphi(\omega) = \arctan(-\infty) = -90\degree$\\
        \end{minipage}\hfill
        \begin{minipage}[c][6cm][c]{0.48\textwidth}
            \resizeboxsb{0.75}{0.75}{% 90% textwidth skript, 100% textwidth beamer
            \includegraphics{Tikz/pdf/plot_beispiel_filter_amplitudengang_lin.pdf}%
            }% end resizeboxsb
            \hfill
            \resizeboxsb{0.75}{0.75}{% 90% textwidth skript, 100% textwidth beamer
            \includegraphics{Tikz/pdf/plot_beispiel_filter_phasengang_lin.pdf}%
            }% end resizeboxsb
        \end{minipage}
    \end{minipage}

%--8<---------------cut here---------------start------------->8---
\b{\framebreak}% EVIL PAGEBREAK ... didn't want to copy whole box
% Empty line after this, otherwise frame might break later on
%--8<---------------cut here---------------end--------------->8---

\textbf{Lösung c) Bestimmten Arbeitspunkt ermitteln}\vspace{+5pt}

	Ges.: Frequenz $f_{1}$ bei der $\hat{U}_2 = \frac{1}{2}\hat{U}_1$ und Phasenverschiebung $u_2$ zu $u_1$ bei $f_1$:\b{\hfill}\vspace{-10pt}%
    \begin{align*}%
        A(\omega_1) &= \frac{R_1}{\sqrt{(R_1+R_2)^2+(\omega CR_1R_2)^2}} \overset{!}{=} \frac{1}{2} & 
        &\text{mit}&
        A(\omega) = \frac{U_2}{U_1} = \frac{\hat{U}_2}{\hat{U}_1} \overset{!}{=} \left.\frac{1}{2}\right|_{f=f_1}
        \\[+2pt]
        &\Rightarrow \frac{(R_1+R_2)^2 + (\omega_1 CR_1R_2)^2}{R_1^2} = 4 &
        &\Rightarrow &
        (R_1+R_2)^2 + (\omega_1 C R_1R_{\mathrm{C}})^2 = 4R_1^2
        \\
        &\Rightarrow (\omega_1 CR_1R_2)^2 = 4R_1^2 - (R_1+R_2)^2 &
        &\Rightarrow&
        \omega_1 = \pm \frac{\sqrt{4R_1^2 - (R_1^2 + 2R_1R_2 + R_2^2)}}{CR_1R_2}
        \\[+2pt]
        \omega_1 &= 1,3304\cdot 10^4\ \frac{1}{\mathrm{s}},\ f_1 = 2,117\ \mathrm{kHz} &
        &\text{mit}&
        \omega_1 < 0 \text{ unphysikalisch},\ f=\frac{\omega}{2\pi}
        \\
        \varphi(\omega_1) &= \arctan(-1,4967) = -56,25\degree &
        &\text{mit}&
        \varphi(\omega) = \arctan\left(\frac{\Im\{\underline{F}\}}{\Re\{\underline{F}\}}\right)
    \end{align*}
	% Reference for +- symbol with paranthesis around minus, +- mit runden Klammern um das Minus:
	% https://tex.stackexchange.com/questions/17550/plus-minus-symbol-with-parenthesis-around-the-minus-sign
\end{bsp}% splitted bsp-box with evil framebreaks to stretch content over multiple frames
\end{frame}