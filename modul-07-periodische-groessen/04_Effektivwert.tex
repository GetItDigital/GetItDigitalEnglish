%Effektivwert

\begin{frame}
    \fta{Effektivwert} \label{Kap.Effektivwert}

    \s{
        Durch die periodischen Spannungsverläufe weist jeder Zeitschritt einen anderen Spannungswert auf. 
        Hierbei kann direkt lediglich der Spitzenwert angegeben werden, mit welchem jedoch nicht immer 
        gearbeitet werden kann. Um eine nützliche Vergleichbarkeit von Spannungswerten zu gewährleisten 
        wird der Effektivwert herangezogen. Der Effektivwert beschreibt beispielsweise die identische 
        Leistung, welche an einem Widerstand anfallen würde, würde dieser an einer Gleichstromquelle angeschlossen
        sein.   
        Zur weiteren Erklärung des Effektivwertes und der dafür erforderlichen Grundlagen werden in diesem 
        Kapitel die folgenden Inhalte besprochen:
    }
    
    \begin{Lernziele}{Effektivwert}
        Die Studierenden
        \begin{itemize}
            \item verstehen die Funktion des Effektivwertes.
            \item kennen die Einflüsse von Amplitude und Kurvenform auf den Effektivwert.
            \item können den Effektivwert einer Wechselspannung bzw. eines Wechselstromes berechnen. 
        \end{itemize}
    \end{Lernziele}

\end{frame}

%Grundlagen
\begin{frame}
    \ftb{Grundlagen: Quadratischer Mittelwert und Additionstheorem}

    \s{
        Neben dem arithmetischen Mittelwert, welcher in der Wechselstromrechnung zur Bestimmung des Mittelwerts
        genutzt wird, dient der quadratische Mittelwert der Bestimmung des Effektivwerts. Der quadratische Mittelwert 
        (RMS - Root Mean Square) wird herangezogen, um die Vergleichbarkeit von Wechselspannungsgrößen zu ermöglichen.
        Hierbei wird eine Funktion f(t) zunächst quadriert. Darauffolgend wird der Mittelwert der quadrierten Funktion 
        berechnet. Im letzten Schritt folgt das Radizieren des berechneten Mittelwertes. Der mathematische Zusammenhang 
        des RMS wird in der Gleichung \ref{GleichungEff1}. 
    }

    \b{
        Der Effektivwert ermöglicht die Vergleichbarkeit von Wechselspannungen durch den quadratischen Mittelwert (RMS). 
        Der RMS wird über die Wurzel des Mittelwertes und die Quadratur einer Funktion ermittelt: 
    }

    \begin{eq}
        RMS=\sqrt{\frac{1}{T}\int_{0}^T f(t)^2 dt}          \label{GleichungEff1}
    \end{eq}

    \s{
        Bei der Bildung des Quadratischen Mittelwerts wird auch die Sinusfunktion quadriert. Als Hilfestellung 
        wird ein Additionstheorem vorgestellt, mit welchem diese Berechnung vereinfacht werden soll. Additionstheoreme
        beziehen sich auf die Zusammenhänge der Trigonometzrie um eine Funktionen umzuformen. Die Umformung der 
        quadratischen Sinusfunktion anhand des Additionstheorems wird in der Gleichung \ref{GleichungEff2} dargestellt. 
        Hier wird aus der quadratischen Sinusfunktion eine Funktion, welche lediglich von einer einfachen Cosinusfunktion 
        abhängig ist. 
    }

    \b{
        Bei der Berechnung des Effektivwertes ist das folgende Additionstheorem hilfreich:
    }

    \begin{eq}
        \sin^2(x)=\frac{1}{2}(1-\cos(2x))         \label{GleichungEff2}
    \end{eq}
\end{frame}

%Amplitude und Kurvenform
\begin{frame}
    \ftb{Amplitude}

    \s{
        Die Ampltude einer Funktion beschreibt den maximalen Ausschalg während einer Periode. Bei einer Wechselspannung 
        gibt der Amplitudenwert den höchsten momentanen Spannungswert der Spannung an. 
        In der Abbildung \ref{BildAmplituden} werden Spannungssignale mit einer Frequenz von $50\,Hz$ und somit mit einer identischen Periodendauer 
        von $T=20\,ms$ dargestellt. Sie unterscheiden sich dabei beim Amplitudenwerten. Die Scheitelwerte der Amplituden der 
        dargestellten Spannungen betrgen $\hat{U}_1=1\,V$, $\hat{U}_2=2\,V$ und $\hat{U}_3=4\,V$.
        Die Amplitude ist eine Art Faktor für die Berechnung des Effektivwertes. So hängt die Höhe des Effektivwertes direkt
        von dem Amplitudenwert ab.
    }

    \b{
        Zusammenhang zwischen Amplitudenwert und Effektivwert:
        \begin{itemize}
            \item Die Amplitude beschreibt den Scheitelwert $\hat{U}$ der Sinusspannung
            \item Verschiedene Amplituden bei unterschiedlichen Sinusspannungen
            \item Der Effektivwert ist direkt proportional zum Amplitudenwert $U_\mathrm{Eff} \sim \hat{U}$
        \end{itemize}
    }

    \fu{
        %vergrößern auf Folienbreite
		\resizebox{0.5\textwidth}{!}{
			\begin{circuitikz}[domain=0:8,samples=100]
    \draw[very thin,color=gray] (-0.1,-2.1) grid (8.1,2.1);
    \draw[->] (-0.2,0) -- (8.2,0) node[right] {$t$};
    \draw[->] (0,-2.1) -- (0,2.2) node [above,color=blue] {$u(t)$};
    \draw (0,-2.1) node[below] {$0$};
    \draw (2,-2.1) node[below] {$5$};
    \draw (4,-2.1) node[below] {$10$};
    \draw (6,-2.1) node[below] {$15$};
    \draw (8,-2.1) node[below] {$20$};
    \draw (-0.1,0) node[left] {$0V$};
    \draw (-0.1,2) node[left] {$4V$};
    \draw (-0.1,-2) node[left] {$-4V$};
    \draw (2,0.75) node[color=cyan] {$u_1$};
    \draw (2,1.25) node[color=blue] {$u_2$};
    \draw (2,2.25) node[color=teal] {$u_3$};
    \draw[color=cyan,smooth] plot[id=Sinus1] function{0.5 * sin((pi/4) * x)};
    \draw[color=blue,smooth] plot[id=Sinus2] function{1 * sin((pi/4) * x)};
    \draw[color=teal,smooth] plot[id=Sinus3] function{2 * sin((pi/4) * x)};

    %ohne diesen Befehl macht das \pause in der Tikz Umgebung die Footline kaputt
    \onslide<1->
\end{circuitikz}
		}
	}{{\bf Sinusspannungen mit unterschiedlichen Amplitudenwerten.} Gewichtung von den Amplitudenwerten $\hat{U}_1=1\,V$, 
    $\hat{U}_2=2\,V$ und $\hat{U}_3=4\,V$. \label{BildAmplituden}}

\end{frame}

\begin{frame}
    \ftb{Kurvenform} 

    \s{
        Neben der Amplitude eines Signals ist die Kurvenform des Signals ausschlaggebend für den Effektivwert. 
        Bei der Berechung des Effektivwerts wird der Scheitelfaktor $C$ (engl. für crest factor) bestimmt, welcher mit dem Ampltudenwert
        verrechnet wird. In der Gleichung \ref{GleichungKur1} wird die Beziehung zwischen dem Amplitudenwert und den 
        Effektivwert über den Scheitelfaktor angegeben.  
    }

    \b{
        Beziehung zwischen dem Scheitelwert und dem Effektivwert über den Scheitelfaktor:
    }

    \begin{eq}
        \hat{U} = C \cdot U_\mathrm{Eff}        \label{GleichungKur1}
    \end{eq}
       
    \s{
        Dieser Scheitelfaktor beträgt für Sinussignale $C_S=\sqrt{2}$, für Dreiecksignale 
        $C_D=\sqrt{3}$ und für Rechtecksignale $C_R=1$. Somit beträgt der Effektivwert einer Sinusspannung 
        etwa $70,7$ Prozent des Scheitelwerts der Amplitude. Neben dem Scheitelfaktor als Zusammenhang zwischen
        Scheitelwert und Effektivwert findet noch der Formfaktor $F$ Anwendung. Der Formfaktor wird als Quotient
        aus dem Effektivwert und dem Gleichrichtwert definiert. 

        \begin{table}[H]
            \centering%
            \caption{{\bf Kurvenformen und Scheitelfaktoren.} Die Scheitelfaktoren für sinusförmige, Dreieck- und Rechtecksignale.}%
            \label{Tabelle}%
        \begin{tabular}{ c c c c }
            \hline &&&\\[-6pt]
            \textbf{Kurvenform:}
                & Sinus
                & Dreieck
                & Rechteck \\[+4pt]
            \hline &&&\\[-6pt]
            \textbf{Scheitelfaktor (C):}
                & $ \sqrt{2} $ 
                & $ \sqrt{3} $     
                & $ 1 $ \\[+4pt]
            \hline
        \end{tabular}
        \end{table}
    }

    \b{
        Formfaktoren von verschiedenen Kurvenformen:
        
        \begin{table}[H]
            \centering%\label{Tabelle}%
        \begin{tabular}{ c c c c }
            \hline &&&\\[-6pt]
            \textbf{Kurvenform:}
                & Sinus
                & Dreieck
                & Rechteck \\[+4pt]
            \hline &&&\\[-6pt]
            \textbf{Scheitelfaktor (C):}
                & $ \sqrt{2} $ 
                & $ \sqrt{3} $     
                & $ 1 $ \\[+4pt]
            \hline
        \end{tabular}
        \end{table}
    }


\end{frame}


\begin{frame}
    \ftx{Einflüsse auf den Effektivwert}

    \begin{Merksatz}{Einflüsse auf den Effektivwert}
        Der Effektivwert ist linear proportional zum Scheitelwert. Der Umrechnungfaktor vom Effektivwert zum Scheitelwert 
        heißt Scheitelfaktor und hängt von der Kurvenform des Singals ab. 
    \end{Merksatz}
\end{frame}


\begin{frame}
    \ftb{Effektivwert}

    \s{
        Der Effektivwert, beispielsweise einer Sinusspannung, soll für die Leistungsaufnahme als ein Momentanwert verwendet 
        werden, welcher die gleiche Leistungsaufnahme bewirkt, die dem Leistungswert einer Gleichspannung unter vergleichbaren 
        Gegebenheiten entsprechen würde. 
        Der
        Effektivwert ist auch beim Ohm'schen Gesetz und bei den Kirchhoffschen Gesetzen anwendbar und somit
        für alle grundlegenden Gesetze geeignet. 

        Anhand eines Wechselstromsignales $i(t)$ soll die Berechnung des Effektivwertes erläutert werden. Wird die Definition 
        des Stromsignales in die Gleichung \ref{GleichungEff3} eingesetzt, so ergibt sich:
    }

    \begin{eq}
        f(t) = i(t) = \hat{I} \cdot \sin(\omega t) \quad \rightarrow \quad I_\mathrm{Eff}=\sqrt{\frac{1}{T}\int_{0}^T {\hat{I}}^2 \cdot \sin^2(\omega t) dt}    \label{GleichungEff3}
    \end{eq}

    \s{
        Die Quadratur des Scheitelwerts ist nicht zeitabhängig und kann vor das Integralzeichen gezogen werden. Außerdem kann 
        der Faktor aus dem Additionstheorem aus dem Integral gezogen werden, sodass die nochfolgende Gleichung \ref{GleichungEff4}
        entsteht:
    }

    \begin{eq}
        I_\mathrm{Eff}=\sqrt{\frac{\hat{I}^2}{2T}\int_{0}^T {1-\cos(2\omega t) dt}}      \label{GleichungEff4}
    \end{eq}

    \s{
        Nach der Differenzregel können Differenzen in einem Integranden gesondert bestimmt werden. Auf diese Weise werden
        in der nachfolgenden Gleichung \ref{GleichungEff5} die Integrale für die beiden Operanden der Differenz seperat 
        bestimmt. 
    }

    \begin{eq}
        \text{mit} \quad \int_0^T 1 dt = T \quad \text{und} \quad \int_0^T \cos(2 \omega t) dt = 0        \label{GleichungEff5}
    \end{eq}

    \s{
        Durch das Auflösen des Integrals entsteht:
    }

    \begin{eq}
        I_\mathrm{Eff} = \sqrt{\frac{\hat{I}^2}{2T}\cdot (T - 0)}          \label{GleichungEff6}
    \end{eq}

    \s{
        Hier lässt sich die Periodendauer $T$ aus dem Zähler und dem Nenner kürzen, sodass folgender Zusammenhang entsteht:
    }

    \begin{eq}
        I_\mathrm{Eff} = \sqrt{\frac{\hat{I}^2}{2}}              \label{GleichungEff7}
    \end{eq}

    \s{
        Nun wird noch aus dem Zähler und dem Nenner getrennt die Wurzel gezogen. Der selbe Zusammenhang lässt sich so auch 
        für die Spannung definieren:

    \begin{eq}
        I_\mathrm{Eff} = \frac{\hat{I}}{\sqrt{2}}   \qquad \text{und}   \qquad      U_\mathrm{Eff} = \frac{\hat{U}}{\sqrt{2}}        \label{GleichungEff8}
    \end{eq}
    }

\end{frame}


\begin{frame}
    \ftx{Effektivwert einer Sinusschwingung}

    \begin{Merksatz}{Effektivwert einer Sinusschwingung}
        Für rein sinusförmige Spannungen und Ströme gilt der Scheitelfaktor $\sqrt{2}$:
        \begin{eq}
            I_\mathrm{Eff} = \frac{\hat{I}}{\sqrt{2}}   \qquad \text{und}   \qquad      U_\mathrm{Eff} = \frac{\hat{U}}{\sqrt{2}} \nonumber
        \end{eq}
    \end{Merksatz}
\end{frame}


\begin{frame}
    \ftx{Komplexer Festzeiger des Effektivwertes}

    \s{
        %Bei der linearen Netzwerkanalyse wird das Netz im quasistationären Zustand betrachtet. Im quasistationären
        %Zustand können repräsentative Berechnungen zu jedem Zeitpunkt durchgeführt werden. Die relative Zuordnung 
        %der Ströme und Spannungen bleiben erhalten. Daher kann die Nullphase eliminiert werden.
        
        In der Abbildung \ref{BildEff1} wird eine Sinusspnnung mit einem Amplitudenwert von $4\,V$ dargestellt. 
        Dazu wird der zu der Sinusspannung zugehörige Effektivwert bei $4\,V/\sqrt{2}$ eingezeichnet. 
        Vergleichbar verhält es sich mit den Spannungen im Haushalt. So würde bei einer Spannungsmessung im Haushalt
        ein Effektivwert von $230\,V$ angezeigt werden. Der Amplitudenwert der eigentlich anliegenden Wechselspannung
        beträgt dabei allerdings $230\,V\cdot\sqrt{2}$ also $325\,V$. 
    }

    \b{
        Effektivwert einer Wechselspannung:
        \begin{itemize}
            \item Formverlauf einer Sinusspannung mit einem Amplitudenwert von $4\,V$ 
            \item<2-> Zugehöriger Effektivwert bei $4\,V/\sqrt{2}$
        \end{itemize}
    }

    \fu{
		\resizebox{0.5\textwidth}{!}{
			\begin{circuitikz}[domain=0:8,samples=100]
    \draw[very thin,color=gray] (-0.1,-2.1) grid (8.1,2.1);
    \draw[->] (-0.2,0) -- (8.2,0) node[right] {$t$};
    \draw[->] (0,-2.1) -- (0,2.2) node [above,color=voltage] {$u(t)$};
    \draw (0,-2.1) node[below] {$0$};
    \draw (2,-2.1) node[below] {$5$};
    \draw (4,-2.1) node[below] {$10$};
    \draw (6,-2.1) node[below] {$15$};
    \draw (8,-2.1) node[below] {$20$};
    \draw (-0.1,0) node[left] {$0V$};
    \draw (-0.1,2) node[left] {$4V$};
    \draw (-0.1,-2) node[left] {$-4V$};
    \draw (2,2.25) node[color=voltage] {$U_3$};
    \draw[color=voltage,smooth] plot[id=Sinus3] function{2 * sin((pi/4) * x)};
    \pause

    % Effektivwert
    \draw[dashed, red] (0,1.414) -- (8,1.414);
    \node[right] at (8,1.414) {$U_\mathrm{Eff}(U_3)$};

\end{circuitikz}	
		}
	}{{\bf Formverlauf einer Sinusspannung.} Sinusspannung mit einem Amplitudenwert von $4\,V$ und zugehörigem Effektivwert von $4\,V/\sqrt{2}$. \label{BildEff1}}

\end{frame}



\begin{frame}
    \ftx{Beispielaufgabe}
	\begin{bsp}{Effektivwert}{Effektivwert:}

    \s{
        Wird die Spannung beispielsweise an einer Haushaltssteckdose gemessen, so wird der Effektivwert von 230V angegeben. 
    }

    \begin{enumerate}
        \only<1>{	
            \item[]
            
            Die folgenden Aufgaben sollen zum besseren Verständnis des Effektivwertes bearbeitet werde: 
            \begin{itemize}
                \item[a)] Berechnung des Amplitudenwertes der Spannung an einer Haushaltssteckdose.
                \item[b)] Berechneng des Effektivwertes davon ausgehend, dass es sich um ein Rechtsecksignal handeln würde.
            \end{itemize}
        }

        \only<2>{	
            \item[a)]
            Der Amplitudenwert an einer Steckdose mit einem Effektivwert von 230V beträgt: 
            \begin{eqa}
                \hat{U} &= I_\mathrm{Eff} \cdot \sqrt{2} = 230V \cdot \sqrt{2}      \nonumber   \\
                \hat{U} &= 325,269V         \nonumber
            \end{eqa}  
        }

        \only<3>{	
            \item[b)]
            Der Effektivwert einer Rechteckspannung mit einem Scheitelwert von 325V beträgt: 
            \begin{eqa}
                U_\mathrm{Rechteck} &= \frac{\hat{U}}{\sqrt{3}} = \frac{325V}{\sqrt{3}}      \nonumber   \\
                U_\mathrm{Rechteck} &= 187,639V         \nonumber
            \end{eqa}  
        }

    \end{enumerate}
    \end{bsp}
\end{frame}
\newpage