\begin{tikzpicture}[scale=1, transform shape]
    \begin{axis}[
    width=4.5cm, % Breite des Graphen
    height=3.5cm, % Höhe des Graphen
    xmin=0, xmax=10,
    ymin=0, ymax=120,
    axis lines=left,
    axis on top=true,
    domain=0:10,
    xtick=\empty,
    ytick=\empty,
    ylabel style={rotate=270, anchor=east, yshift=1cm}, % Position der Beschriftung oben links, horizontal
     ylabel={\it V}, % Hier die gewünschte Beschriftung einfügen
    xlabel={f[Hz]},
    clip mode=individual % Verhindert das Abschneiden von Elementen
    ]
    \path[draw=none] (axis cs:-4, 0) rectangle (axis cs:16.6,120);
    
    
    \coordinate (xaxis) at (axis description cs:15.5,0); % Verwendet die relative Positionierung für x-Achse
    
    \addplot+[mark=none, thick, blue] coordinates {(0,80) (10,80)};
    % Adding the left-side label
    \node[anchor=east] at (axis cs:0,80) {$1 + \frac{R_2}{R_1}$};
    \end{axis}
    \begin{axis}[
    width=4.5cm, % Breite des Graphen
    height=3.5cm, % Höhe des Graphen
    xmin=0, xmax=10,
    ymin=-180, ymax=180,
    axis y line*=right,
    axis x line=none,
    ytick=\empty,
    ylabel={$\varphi$},
    ylabel style={rotate=270, anchor=west, yshift=0.8cm},
    clip mode=individual, % Verhindert das Abschneiden von Elementen
    after end axis/.code={
        \draw[->] (axis cs:10,180) -- (axis cs:10,190);
    }
    ]
    \addplot+[mark=none, thick, red] coordinates {(0,0) (10,0)};
    \node[anchor=west] at (axis cs:10,0) {$0^\circ$};
    \end{axis}
    \end{tikzpicture}