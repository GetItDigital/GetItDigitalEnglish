\subsection{Transistor-Verbundverstärker\label{Aufg27}}
% Alt: Klausuraufgabe 2
\Aufgabe{
   Zwei npn-Bipolartransistoren bilden zusammen einen rauscharmen Verstärker. 
  Die gesamte Spannungsverstärkung $A_\mathrm{V}$ hat hier den Betrag von etwa 1400. 
  Praktische Anwendung ist die rauscharme Detektion kleiner Signale im Bereich um und unter $1\, \mathrm{\mu V}$.
  Die Versorgungsspannung beträgt $10\,\mathrm{V}$. Am Kollektor $\mathrm{T_2}$ misst man hier $4\,\mathrm{V}$ gegenüber Masse (ist also nicht $U_\mathrm{CE}$!).
  In den Arbeitspunkten von $\mathrm{T_1}$ und $\mathrm{T_2}$ kann für die Basis-Emitter-Spannungen $U_\mathrm{BE}$ jeweils $600\,\mathrm{mV}$ angenommen werden.
  Die Kollektorströme $\mathrm{I_{C}}$ sind sehr groß gegenüber den Basisströmen $I_\mathrm{B}$ anzunehmen. Daher gilt hier ausreichend genau $I_\mathrm{C} = I_\mathrm{E}$.
$
R_1 = 10\, \mathrm{k\Omega} \\ 
R_2 = 10\, \mathrm{k\Omega} \\
R_3 = 2\, \mathrm{k\Omega} \\
R_4 = 300\, \mathrm{k\Omega} \\
R_5 = 10\, \mathrm{M\Omega} \\
V_1 = 10\, \mathrm{V} \\
V_2 = AC\, 0,1\, \mathrm{mV} \\
C_1 = 1\, \mathrm{\mu F} \\
C_2 = 1\, \mathrm{\mu F} \\
$
\begin {figure} [H]
   \centering
   \begin{tikzpicture}
    \draw (0,0)
    node[npn] (Q1) {$T_1$};

    \draw (Q1.collector)
    to[short, -*] (0,1)
    to [R, l_=$R_\mathrm{1}$,] (0,4);

    \draw (Q1.base)
    to[short, -*] (-1,0)
    to[C, l_=$C_\mathrm{1}$](-4,0)
    to[sV, name=UE, v_] (-4,-4)
    node[ground]{};

    \draw (Q1.emitter)
    -- (0,-1)
    node[ground]{};

    \draw (-1,0)
    -- ++ (0,-2)
    to[R,l=$R_\mathrm{4}$, -*] (3,-2);

    \draw (3,0)
    node[npn] (Q2) {$T_2$};

    \draw (Q2.emitter)
    to[short](3,-2)
    to[R,l=$R_\mathrm{3}$] (3,-4)
    node[ground]{};

    \draw (0,1)
    -- (1,1)
    -- (2,0)
    to[short] (Q2.base);

    \draw (Q2.collector)
    to[short,-*](3,1)
    to[R,l_=$R_\mathrm{2}$](3,4)
    -- (6,4)
    to[V,name=U1, v] (6,2)
    node[ground]{};

    \draw (0,4)
    to[short, -*] (3,4);

    \draw (3,1)
    to[C,l=$C_\mathrm{2}$](6,1)
    to[R,l=$R_\mathrm{5}$](6,-1)
    node[ground]{};

    \varrmore{U1}{$U_\mathrm{1}$};
    \varrmore{UE}{$u_\mathrm{E}(t)$};
\end{tikzpicture}
   \label{fig:FigKlausurVerbundverstaerker}
\end {figure}
\begin{itemize}
   \item[a)]
   Welcher Arbeitspunkt-Gleichstrom fließt durch $R_2$ (hier gleich $I_\mathrm{C,T2}$)
   \item[b)]
  Für hier gilt: $R_4 >> R_3$, welcher Strom fließt also etwa durch $R_3$?
   \item[c)]
   Welche Spannung fällt über $R_3$ ab? 
   \item[d)]
   Wie groß ist die Gleichspannung ander Basis von $T_2$, hier identisch mit $U_\mathrm{CE,T1}$?
   \item[e)]
   Wie groß ist der Strom durch $R_1$ (hier gleich $I_\mathrm{C,T1}$)?
   \item[f)]
   Erläutern Sie in Stichworten die Arbeitspunktstabiliserung der gesamten Schaltung aufgrund der Gegenkopplung über $R_4$.
   \item[g)]
   $U_\mathrm{T} = 26\, \mathrm{mV}$. Wie groß ist die Steilheit $S$ von $T_\mathrm{1}$?
   \item[h)]
   Wie groß ist die Spannungsverstärkung $A_\mathrm{V} = \frac{\Delta U_\mathrm{CE}}{\Delta U_\mathrm{BE}}$, also die erste Transistorstufe?
   \item[i)]
   Für die zweite Transistorstufe nun anders zu rechnen, nämlich mit $R_3$ und $R_2$: 
   Wie ist die Spannungsverstärkung $A_\mathrm{V}$ von $T_\mathrm{2}$, hier definiert als $\frac{\Delta U_\mathrm{C}}{\Delta U_\mathrm{B}}$?
   \item[j)]
   Wie ist dann die gesamte Spannungsverstärkung der beiden Stufen $T_\mathrm{1}$ und $T_\mathrm{2}$ zusammen?
   \item[k)]
   Zeichnen Sie ien Kleinsignal-Ersatzschaltbild der gesamten Schaltung. Ersetzen Sie darin die beiden Transistoren im Arbeitspunkt jeweils durch eine steuerbare Stromquelle S und einen Eingangswiderstand $r_\mathrm{BE}$.
\end{itemize}
% \speech{
% **Aufgabe 27: Rauscharm verstärkende Transistorschaltung mit zwei npn-Transistoren**

% Diese Aufgabe behandelt eine rauscharm verstärkende Schaltung mit zwei npn-Bipolartransistoren. Die gesamte Spannungsverstärkung \( A_V \) beträgt etwa 1400. Eine praktische Anwendung dieser Schaltung ist die rauscharme Detektion kleiner Signale im Mikrovolt-Bereich. Die Versorgungsspannung beträgt 10 Volt.

% Am Kollektor des zweiten Transistors \( T_2 \) misst man 4 Volt gegen Masse. Der Wert von \( U_{CE} \) ist hier also nicht direkt relevant. In den Arbeitspunkten der beiden Transistoren \( T_1 \) und \( T_2 \) kann die Basis-Emitter-Spannung \( U_{BE} \) jeweils mit 600 Millivolt angenommen werden. Die Kollektorströme \( I_C \) sind sehr groß im Vergleich zu den Basisströmen \( I_B \), daher gilt näherungsweise \( I_C = I_E \).

% Die gegebenen Widerstands- und Kapazitätswerte sind:
% - \( R_1 = 10 \text{k}\Omega \),
% - \( R_2 = 10 \text{k}\Omega \),
% - \( R_3 = 2 \text{k}\Omega \),
% - \( R_4 = 300 \text{k}\Omega \),
% - \( R_5 = 10 \text{M}\Omega \),
% - \( V_1 = 10V \),
% - \( V_2 = AC 0.1mV \),
% - \( C_1 = 1\mu F \),
% - \( C_2 = 1\mu F \).

%  **Beschreibung der Schaltung:**
% Die Schaltung zeigt eine Verstärkerschaltung mit zwei npn-Transistoren \( T_1 \) und \( T_2 \). 

% - **Erster Transistor \( T_1 \)**:
%   - Sein Kollektor ist über einen Widerstand \( R_3 \) mit der Versorgungsspannung \( V_1 \) verbunden.
%   - Die Basis ist über \( R_1 \) mit der Versorgungsspannung und über \( R_2 \) mit Masse verbunden.
%   - Der Emitter ist direkt mit Masse verbunden.
%   - Ein Wechselspannungssignal wird über \( V_2 \) an die Basis von \( T_1 \) angelegt.

% - **Zweiter Transistor \( T_2 \)**:
%   - Sein Kollektor ist über \( R_4 \) mit der Versorgungsspannung verbunden.
%   - Sein Emitter ist mit dem Kollektor von \( T_1 \) verbunden, wodurch eine Stromkopplung entsteht.
%   - Zusätzlich gibt es eine kapazitive Kopplung durch die Kondensatoren \( C_1 \) und \( C_2 \), welche das Wechselspannungssignal weitergeben.

% Der verstärkte Ausgang \( u_A \) wird am Kollektor des zweiten Transistors abgegriffen.

%  **Fragen zur Aufgabe:**
% **(a)** Welcher Arbeitspunkt-Gleichstrom fließt durch \( R_2 \)? (Hier gilt \( I_C = I_{C,T2} \))

% **(b)** Es gilt \( R_4 \gg R_3 \). Welcher Strom fließt also etwa durch \( R_3 \)?

% **(c)** Welche Spannung fällt über \( R_3 \) ab?

% **(d)** Wie groß ist die Gleichspannung an der Basis von \( T_2 \)? Diese ist identisch mit \( U_{CE,T1} \).

% **(e)** Wie groß ist der Strom durch \( R_1 \)? (Hier gleich \( I_C,T1 \))

% **(f)** Erklären Sie in Stichpunkten die Arbeitspunktstabilisierung der gesamten Schaltung aufgrund der Gegenkopplung über \( R_4 \).

% **(g)** Der Temperaturkoeffizient \( U_T \) beträgt 26 mV. Wie groß ist die Steilheit \( S \) von \( T_1 \)?

% **(h)** Wie groß ist die Spannungsverstärkung \( A_V = \frac{\Delta U_{C,E}}{\Delta U_{B,E}} \), also die erste Transistorstufe?

% **(i)** Für die zweite Transistorstufe muss die Berechnung anders durchgeführt werden, nämlich mit \( R_3 \) und \( R_2 \). Wie ist die Spannungsverstärkung \( A_V \) von \( T_2 \) definiert als \( \frac{\Delta U_A}{\Delta U_C} \)?

% **(j)** Wie ist dann die gesamte Spannungsverstärkung der beiden Stufen \( T_1 \) und \( T_2 \) zusammen?

% **(k)** Zeichnen Sie ein Kleinsignal-Ersatzschaltbild der gesamten Schaltung. Ersetzen Sie darin die beiden Transistoren im Arbeitspunkt jeweils durch eine steuerbare Stromquelle \( S \) und einen Eingangswiderstand \( r_{BE} \).

% }



}



\Loesung{
   \begin{itemize}
      \item[a)]
    \end{itemize}
    \begin{align*}
      I_\mathrm{C} &= \frac{6\,\mathrm{V}}{10\,\mathrm{k\Omega}} = 0,6\,\mathrm{mA}
    \end{align*}
    \begin{itemize}
      \item[b)]
    \end{itemize}
    \begin{align*}
      I_\mathrm{C,T2} &= 0,6\,\mathrm{mA}
    \end{align*}
    \begin{itemize}
      \item[c)]
    \end{itemize}
    \begin{align*}
      U_{\mathrm{R3}} &= 0,6\,\mathrm{mA} \cdot 2\,\mathrm{k\Omega} = 1,2\, \mathrm{V}
    \end{align*}
    \begin{itemize}
      \item[d)]
    \end{itemize}
    \begin{align*}
      1,2\, \mathrm{V} + 0,6\,\mathrm{V} = 1,8\,\mathrm{V}
    \end{align*}
    \begin{itemize}
      \item[e)]
    \end{itemize}
    \begin{align*}
      I_\mathrm{C,T1} &= \frac{8,2\,\mathrm{V}}{10\,\mathrm{k\Omega}} = 0,82\,\mathrm{mA}
    \end{align*}
    \begin{itemize}
      \item[f)]
    \end{itemize}
    \begin{itemize}
      \item Erhöhung von $I_\mathrm{C}$ führt zu einem Spannungsanstieg über $R_4$
      \item dadurch sinkt die Basisspannung von $T_\mathrm{1}$, was $I_\mathrm{C}$ wieder reduziert
      \item negative Rückkopplung stabilisiert Arbeitspunkt
    \end{itemize}
    \begin{itemize}
      \item[g)]
    \end{itemize}
    \begin{align*}
      S &= \frac{I_\mathrm{C,T1}}{U_\mathrm{T}} = \frac{0,82\,\mathrm{mA}}{26\,\mathrm{mA}} = 31,54\, \frac{\mathrm{mA}}{\mathrm{V}}
    \end{align*}
    \begin{itemize}
      \item[h)]
    \end{itemize}
    \begin{align*}
      A_\mathrm{V} &= S \cdot (-R_\mathrm{C}) = -315
    \end{align*}
    \begin{itemize}
      \item[i)]
    \end{itemize}
    \begin{align*}
      A_\mathrm{V} &= -\frac{R_\mathrm{C}}{R_\mathrm{E}} = \frac{10\,\mathrm{k\Omega}}{2\,\mathrm{k\Omega}} = -5
    \end{align*}
    \begin{itemize}
      \item[j)]
    \end{itemize}
    \begin{align*}
      -5 \cdot -315 = 1575 \approx 1600
    \end{align*}
    \begin{itemize}
      \item[k)]
    \end{itemize}
    \begin {figure} [H]
        \centering
        \begin{tikzpicture}
    \draw (0,0) to[short, o-*] (1,0) to[short, -o] (3,0)node[above]{B};
    \draw (5,0)node[above]{C} to[short, o-*] (7,0) to[short,-o] (8,0)node[above]{B};
    \draw (10,0)node[above]{C} to[short, o-*] (12,0) to[short, -*] (13,0) to[short, -o] (15,0);
    \draw (0,-2) to[short, o-*] (1,-2) -- (1.5,-2)node[ground]{};
    \draw (1.5,-2) to[short, -] (3,-2) to[short, -*] (4,-2)node[below]{E} to[short, -*] (7,-2) to[short, -*] (9,-2)node[below]{E} to[short, -] (10.5,-2) to[R=$R_3$, -*] (12,-2) to[short, -*] (13,-2) to[short, -o] (15,-2);


    \draw (15,0) to[open,v>, name=ua] (15,-2);
    \draw (0,0) to[open,v>, name=ue] (0,-2);

    \draw (1,0) to[R=$R_4$] (1,-2);
    \draw (3,0) to[R=$r_\mathrm{BE 1}$] (3,-2);
    \draw (5,0) to[I, l=$S_1$] (5,-2);
    \draw (7,0) to[R=$R_1$] (7,-2);
    \draw (8,0) to[R=$r_\mathrm{BE2}$] (8,-2);
    \draw (10,0) to[I, l=$S_2$] (10,-2);
    \draw (12,0) to[R=$R_2$] (12,-2);
    \draw (13,0) to[R=$R_5$] (13,-2);


    \varrmore{ue}{$u_\mathrm{E}$};
    \varrmore{ua}{$u_\mathrm{A}$};

\end{tikzpicture}
        \label{fig:LsgKlausurVerbundverstaerker}
    \end {figure}
    Siehe: Abschnitt \ref{sec:DarlingtonTransistor}
    
    % \speech{ Lösung zu Aufgabe 27

    % a) Der Arbeitspunkt-Gleichstrom durch den Widerstand R2 entspricht dem Kollektorstrom I_C von T2. Dieser ergibt sich aus der Spannung von 6 Volt, geteilt durch den Widerstandswert von 10 Kiloohm. Das ergibt I_C gleich 0,6 Milliampere.
    
    % b) Da der Widerstand R4 deutlich größer als R3 ist, fließt durch R3 annähernd der gleiche Strom wie durch R2, also ebenfalls 0,6 Milliampere.
    
    % c) Die Spannung über R3 berechnet sich als Produkt aus Strom und Widerstand. Das ergibt 0,6 Milliampere mal 2 Kiloohm gleich 1,2 Volt.
    
    % d) Die Gleichspannung an der Basis von T2 entspricht der Summe aus U_R3 und der Basis-Emitter-Spannung U_BE. Das ergibt 1,2 Volt plus 0,6 Volt gleich 1,8 Volt.
    
    % e) Der Strom durch R1 entspricht dem Kollektorstrom von T1. Dieser wird berechnet als 8,2 Volt geteilt durch 10 Kiloohm, was 0,82 Milliampere ergibt.
    
    % f) Die Arbeitspunktstabilisierung erfolgt durch die negative Rückkopplung über R4. Eine Erhöhung von I_C führt zu einem Spannungsanstieg über R4, wodurch die Basisspannung von T1 sinkt und I_C wieder reduziert wird. Dadurch wird der Arbeitspunkt stabilisiert.
    
    % g) Die Steilheit S von T1 berechnet sich als I_C von T1 geteilt durch U_T. Das ergibt 0,82 Milliampere geteilt durch 26 Milliampere, was ungefähr 31,54 Milliampere pro Volt ergibt.
    
    % h) Die Spannungsverstärkung A_V der ersten Transistorstufe berechnet sich als das Produkt von Steilheit S und dem negativen Widerstand R_C. Das ergibt minus 315.
    
    % i) Die Spannungsverstärkung A_V der zweiten Stufe ergibt sich aus dem Verhältnis von R_C zu R_E, also minus 10 Kiloohm geteilt durch 2 Kiloohm, was minus 5 ergibt.
    
    % j) Die gesamte Spannungsverstärkung der beiden Stufen berechnet sich als das Produkt aus den einzelnen Verstärkungen. Minus 5 mal minus 315 ergibt 1575, was auf 1600 gerundet wird.
    
    %  k) Das Ersatzschaltbild dieser Schaltung stellt die beiden Transistorstufen in ihrer Ersatzform dar, wobei die Transistoren durch steuerbare Stromquellen und Eingangswiderstände ersetzt werden.

      % Struktur des Ersatzschaltbildes:
      % Eingangsspannung U_E liegt links und wird über den Widerstand R4 an die Basis des ersten Transistors geführt.
      % Erster Transistor (T1):
      % Der Transistor wird im Arbeitspunkt durch eine steuerbare Stromquelle S1 ersetzt.
      % Parallel zur Basis-Emitter-Strecke von T1 befindet sich der Widerstand r_BE1, der den Eingangswiderstand der Transistorstufe darstellt.
      % Die Kollektorschaltung umfasst den Widerstand R1 und die Verbindung zur Basis des zweiten Transistors.
      % Zweiter Transistor (T2):
      % Auch dieser Transistor wird durch eine steuerbare Stromquelle S2 ersetzt.
      % Parallel zur Basis-Emitter-Strecke von T2 befindet sich der Widerstand r_BE2 als Eingangswiderstand der zweiten Stufe.
      % Die Kollektorschaltung enthält R2 und R3, die zur Verstärkung beitragen.
      % Ausgangsspannung U_A:
      % Der Ausgang wird am Kollektor von T2 abgegriffen.
      % Der Widerstand R5 ist mit dem Ausgang verbunden und dient zur weiteren Signalverarbeitung.
      % Funktionsweise:
      % Die erste Transistorstufe verstärkt die Eingangsspannung U_E und gibt eine verstärkte Spannung an die Basis der zweiten Stufe weiter.
      % Die zweite Transistorstufe sorgt für eine weitere Verstärkung des Signals, wobei das Verhältnis von R_C zu R_E entscheidend für die Verstärkung ist.
      % Durch die Gegenkopplung über R4 wird der Arbeitspunkt stabilisiert und die Linearität verbessert.
      % Die Ersatzschaltung mit steuerbaren Stromquellen zeigt, dass die Transistoren als spannungsgesteuerte Stromquellen arbeiten.
      % Fazit:
      % Dieses Kleinsignal-Ersatzschaltbild ermöglicht eine vereinfachte Analyse der Verstärkereigenschaften, indem die nichtlinearen Transistoreigenschaften durch lineare Ersatzbauelemente wie Widerstände und Stromquellen ersetzt werden. 
      % Dadurch können Verstärkungsfaktoren, Eingangs- und Ausgangswiderstände sowie die Stabilität der Schaltung leichter berechnet werden.

     % }
    
    

    
}
