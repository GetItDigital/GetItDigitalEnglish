\subsection{Untersuchung eines Aufwärtswandlers zur Spannungsanhebung\label{Aufg28}}
% Alt: Praktikumsaufgabe 14
\Aufgabe{
    Mit einem Aufwärtswandler können kleine Gleichspannungen (z. B. aus einer Batterie) in eine höhere Gleichspannung umgesetzt werden.  
    Untersuchen Sie die Funktionsweise eines Aufwärtswandlers, indem Sie die Schaltung analysieren.
    
    Gegeben:
    \begin{itemize}
        \item Eingangsspannung: $U_\mathrm{E} = 1\,\mathrm{V}$
        \item Induktivität: $L = 900\,\mathrm{\mu H}$
        \item Diode: Ideale Diode (keine Spannungsabfälle berücksichtigt)
        \item Kondensator: $C = 470\,\mathrm{\mu F}$
        \item Schalter: Wird periodisch geöffnet und geschlossen
    \end{itemize}
    
    \begin{itemize}
        \item[a)] Erklären Sie die Funktionsweise des Aufwärtswandlers in den zwei Phasen:  
            \begin{itemize}
                \item \textit{Phase 1: Schalter geschlossen}
                \item \textit{Phase 2: Schalter geöffnet}
            \end{itemize}
        \item[b)] Leiten Sie die Formel für die Ausgangsspannung $U_\mathrm{A}$ in Abhängigkeit des Tastverhältnisses $D$ her:
            \[
            U_\mathrm{A} = \frac{U_\mathrm{E}}{1 - D}
            \]
        \item[c)] Berechnen Sie $U_\mathrm{A}$ für verschiedene Tastverhältnisse $D$:
            \begin{itemize}
                \item $D = 0{,}3$
                \item $D = 0{,}5$
                \item $D = 0{,}7$
            \end{itemize}
        \item[d)] Was passiert, wenn $D$ gegen $1$ geht? Welche praktischen Begrenzungen gibt es?
    \end{itemize}
    \begin{figure} [H]
    \centering
    \begin{tikzpicture}
    \draw (0,3) to[V,name=V1, -*] (0,0)node[ground]{} to[short,-*] (2,0) to[normal open switch, l=$S$, -*] (2,3);
    \draw (0,3) to[L,l=$L_\mathrm{1}$] (2,3) to[D,l=$D_\mathrm{1}$] (4,3) to[C,l=$C_\mathrm{1}$] (4,0) -- (2,0);
    
    
    \varrmore{V1}{$U_\mathrm{0}$};
\end{tikzpicture}
   
    \label{fig:FigPrakAufwaertswandler}
    \end{figure}
}

\Loesung{
    \textbf{a) Erklärung der Funktionsweise des Aufwärtswandlers} \\[0.2cm]
    Der Aufwärtswandler arbeitet in zwei Phasen:

    \begin{itemize}
        \item \textbf{Phase 1 (Schalter geschlossen):}  
        Die Induktivität $L$ wird mit Strom geladen und speichert Energie in ihrem Magnetfeld.  
        Die Diode sperrt, sodass kein Strom zum Ausgang fließt.
        
        \item \textbf{Phase 2 (Schalter geöffnet):}  
        Der gespeicherte magnetische Fluss der Induktivität bricht zusammen, wodurch die Induktivität eine Spannung induziert.  
        Diese Spannung addiert sich zur Eingangsspannung und lädt den Kondensator $C$ auf eine höhere Spannung auf.  
        Der Ausgang erhält nun eine höhere Spannung als die Eingangsspannung.
    \end{itemize}

    \textbf{b) Herleitung der Ausgangsspannungsformel} \\[0.2cm]
    Im stationären Zustand gilt das Verhältnis zwischen Ein- und Ausgangsspannung:
    \[
    U_\mathrm{A} = \frac{U_\mathrm{E}}{1 - D}
    \]
    wobei $D$ das Tastverhältnis ist (Verhältnis der Zeit, in der der Schalter geschlossen ist).

    \textbf{c) Berechnung der Ausgangsspannung für verschiedene $D$} \\[0.2cm]
    Gegeben ist $U_\mathrm{E} = 1\,\mathrm{V}$.

    \begin{itemize}
        \item Für $D = 0{,}3$:
        \[
        U_\mathrm{A} = \frac{1\,\mathrm{V}}{1 - 0{,}3} = \frac{1\,\mathrm{V}}{0{,}7} = 1{,}43\,\mathrm{V}
        \]
        \item Für $D = 0{,}5$:
        \[
        U_\mathrm{A} = \frac{1\,\mathrm{V}}{1 - 0{,}5} = \frac{1\,\mathrm{V}}{0{,}5} = 2\,\mathrm{V}
        \]
        \item Für $D = 0{,}7$:
        \[
        U_\mathrm{A} = \frac{1\,\mathrm{V}}{1 - 0{,}7} = \frac{1\,\mathrm{V}}{0{,}3} = 3{,}33\,\mathrm{V}
        \]
    \end{itemize}

    \textbf{d) Grenzen für $D \to 1$} \\[0.2cm]
    Theoretisch würde $U_\mathrm{A} \to \infty$, wenn $D$ gegen $1$ geht.  
    In der Praxis gibt es jedoch Begrenzungen durch:
    \begin{itemize}
        \item Die Sättigung der Induktivität
        \item Verluste durch Widerstände, Diode und Schalter
        \item Begrenzte Schaltfrequenzen
    \end{itemize}
    Typische Werte für $D$ liegen in der Regel unter 0,8, um einen stabilen Betrieb zu gewährleisten.
}
