%hier jede Aufgabe als eigene Datei mit dem Input-Befehl einfügen
%\subsection{Gleichstrommaschine 1\label{AufgGM1}}
\Aufgabe{
    Eine fremderregte Gleichstrommaschine hat die folgenden Angaben auf ihrem Typenschild:
    \begin{itemize}
        \item Nennleistung: $41,8\,\text{kW}$
        \item Nenndrehzahl: $1900\,\frac{1}{\text{min}}$
        \item Ankernennspannung: $440\,\text{V}$
        \item Ankernennstrom: $100\,\text{A}$
        \item Erregung: $240\,\text{V}$
        \item Erregerstrom: $10\,\text{A}$
        \item Leerlaufdrehzahl: $2000\,\frac{1}{\text{min}}$
    \end{itemize}

    Hinweis: Die Nennleistung auf einem Motortypenschild bezeichnet immer die abgegebene mechanische Leistung im Nennpunkt.

    \begin{itemize}
        \item[a)]
        Wie groß ist das Drehmoment der Maschine im Nennpunkt?
        \item[b)]
        Berechnen Sie den Wirkungsgrad der Maschine im Nennpunkt.
        \item[c)]
        Berechnen Sie das Produkt aus Erregerfluss und der Ankerkonstanten $K\cdot \Phi$
        \item[d)]
        Wie groß ist der Ankerwiderstand $R_A$?
    \end{itemize}
}

\Loesung{
	\begin{itemize}

		\item[a)]
		Aus der Nennleistung und der Nenndrehzahl ergibt sich direkt das Nenndrehmoment:
		\begin{align*}
		    P_\text{mech} &= M\cdot \omega\\
		    M &= \frac{P_\text{mech}}{\omega} = \frac{41,8\,\text{kW}}{2\pi\cdot 1900\,\frac{1}{\text{min}} \cdot \frac{1\,\text{min}}{60\,\text{s}}} = 210,08\,\text{Nm}
		\end{align*}
		\item[b)]
		Im Motorbetrieb ist der Wirkungsgrad der Quotient aus mechanischer bezogen auf die elektrische Leistung (siehe Gleichung \ref{GlLeistung}). Hierbei muss sowohl die Ankerleistung, als auch die Erregerleistung berücksichtigt werden.
		\begin{eqa}
			\eta &= \frac{P_\text{mech}}{P_\text{elektr}} = \frac{41,8\,\text{kW}}{440\,\text{V}\cdot 100\,\text{A} + 240\,\text{V}\cdot 10\,\text{A}} = 90,08\,\%\nonumber
		\end{eqa}

		\item[c)]
		Nach Gleichung \ref{GlInduzierteSpannungGM} wird die induzierte Spannung berechnet. Im Leerlauf muss die induzierte Spannung der Ankerspannung entsprechen. Gleiches geht aus Gleichung \ref{GlfremderregteGM1} hervor, da im Leerlauf auch das Drehmoment Null sein muss.
		\begin{eqa}
			U_q &= K \cdot \Phi \cdot \omega\tag{\ref{GlInduzierteSpannungGM}}\\
			K\cdot \Phi &= \frac{U_q}{\omega} = \frac{440\,\text{V}}{2\pi\cdot 2000\,\frac{1}{\text{min}} \cdot \frac{1\,\text{min}}{60\,\text{s}}} = 2,1\,\text{Vs}\nonumber
		\end{eqa}
		\item[d)]
		Hier sind zwei Ansätze möglich. Es kann Gleichung \ref{GlfremderregteGM1} auf den Ankerwiderstand umgeformt werden:
		\begin{eqa}
			n &= \frac{U_A}{2\pi K\cdot \Phi} - \frac{R_A\cdot M_i}{2\pi(K\cdot \Phi)^2}\tag{\ref{GlfremderregteGM1}}\\
			R_A &= \left(\frac{U_A}{2\pi K\cdot \Phi} - n\right) \cdot \frac{2\pi(K\cdot \Phi)^2}{M_i}\nonumber\\
			&= \left(\frac{440\,\text{V}}{2\pi\cdot 2,1\,\text{Vs}} - \frac{1900}{60\,\text{s}}\right)\cdot \frac{2\pi\cdot (2,1\,\text{Vs})^2}{210,08\,\text{Nm}} = 220\,\text{m}\Omega\nonumber
		\end{eqa}
		Die andere Berechnungsvariante führt über das Ersatzschaltbild in Abbildung \ref{AbbGMFremderregt}. Im Nennbetrieb können wir die Quellenspannung $U_q$ berechnen. Diese muss kleiner als die Quellenspannung im Leerlauf sein.
		\begin{eqa}
			U_q &= K \cdot \Phi \cdot \omega \tag{\ref{GlInduzierteSpannungGM}}\\
			U_{q,n}&=  2,1\,\text{Vs} \cdot 2\pi\cdot 1900\,\frac{1}{\text{min}} \cdot \frac{1\,\text{min}}{60\,\text{s}} = 418\,\text{V}\nonumber
		\end{eqa}
		Der Spannungsabfall am Widerstand $R_A$ muss durch dem Maschenumlauf die Differenzspannung zwischen Ankerspannung und induzierter Spannung betragen. Da der Ankerstrom im Nennbetrieb bekannt ist, kann der Widerstand über das Ohmsche Gesetz berechnet werden.
		\begin{equation*}
		    R_A = \frac{U_A - U_q}{I_A} = \frac{440\,\text{V} - 418\,\text{V}}{100\,\text{A}} = 220\,\text{m}\Omega
		\end{equation*}
	\end{itemize}
}
%\input{Aufgabe2}
%....

%%%%%%%%%%%%%%%%%%%%%%%%%%%%%%%%%%%%%%%%%%%%%%%%%%%%%%%%%%%%%%%%%%%%%%%%%%%%%%%%%%%

\subsection{Einschaltverhalten eines RC-Übertragungsgliedes}
\Aufgabe{
Gegeben ist das rechts dargestellte Übertragungsglied (Zweitor). 
Untersucht wird das Einschwingverhalten der Spannung $u_2$ am Ausgang 
bei Anschließen einer idealen Gleichspannungsquelle $U_q$.\\
\begin{minipage}[b]{0.5\textwidth}% links
    Die Schaltung ist vor dem Schaltzeitpunkt $t=0$ im stationären Zustand mit $u_2(t<0) = 0$.
    \begin{itemize}
        \item[a)] Stellen Sie die DGL für $u_C$ für $t \geq 0$ auf.
        \item[b)] Bestimmen Sie die Funktion $u_C(t)$ für $t \geq 0$ inklusive aller Konstanten. 
        \item[c)] Bestimmen Sie die Funktion $u_2$ für $t \geq 0$ aus der Lösung b) für $u_C(t)$. Vereinfachen Sie soweit wie möglich.
    \end{itemize}
\end{minipage}%
\begin{minipage}[b]{0.5\textwidth}% rechts
    \begin{figure}[H]\centering\vspace{0pt}% empty spacing für vertical alignment for tikzpicture
        \includegraphics{Tikz/pdf/Übungen/circ_uebung_einschaltverhalten_rs_rcs_dc.pdf}
    \end{figure}
\end{minipage}
\begin{itemize}
    \item[d)] Skizzieren Sie qualitativ die Zeitverläufe der Spannungen $u_2$, $u_{R_2}$ und $u_C$ 
    vom stationären Zustand für $t<0$ bis die dargestellten Größen eingeschwungen sind ($t=5\tau$).  
    Kennzeichnen Sie Zeitkonstanten, Anfangs- und Endwerte der Verläufe. 
\end{itemize}
}
\Loesung{
\begin{itemize}
\item[a)] DGL für $u_C(t \geq 0)$:
    \begin{align*}
        R_1 \cdot \eci{i} + R_2 \cdot \eci{i} + \ecv{u_C} &= \ecv{U_q} \\
        \underbrace{(R_1 + R_2) \cdot C}_{\ =\tau} \cdot \dt\, \ecv{u_C} + \ecv{u_C} &= \ecv{U_q}
    \end{align*}
\item[b)] Zeitverlauf $u_C$ für $t\geq0$:
    \begin{align*}
        \ecv{u_{C,\mathrm{f}}}=\ecv{u_{C,\mathrm{h}}} &= K \cdot e^{-\frac{t}{\tau}}  &&\text{mit } \tau = (R_1 + R_2) \cdot C  \\
        \ecv{u_{C,\mathrm{e}}}=\ecv{u_{C,\mathrm{p}}} &= \ecv{U_q} \\
        \ecv{u_C} = \ecv{u_{C,\mathrm{f}}} + \ecv{u_{C,\mathrm{e}}} &= K \cdot \mathrm{e}^{-\frac{t}{\tau}} + \ecv{U_q} \\
        \ecv{u_C}(0) &= K \cdot \redcancel{e^{0}} + \ecv{U_q} \overset{!}{=} 0 &&\Longrightarrow K = - \ecv{U_q} \\
        \ecv{u_C}(t) &= \left(1 - \mathrm{e}^{-\frac{t}{\tau}}\right) \cdot \ecv{U_q}
    \end{align*}
\item[c)] Spannung $u_2(t)$ aus $u_C(t)$ für $t \geq 0$:
\begin{align*}
    \ecv{u_2} &= \eci{i} \cdot R_2 + \ecv{u_C} \\
    &= C \cdot R_2 \cdot \dt\, \ecv{u_C} + \ecv{u_C} \\
    &= C \cdot R_2 \cdot \dt\, \left(1 - \mathrm{e}^{-\frac{t}{\tau}}\right) \cdot \ecv{U_q} + \left(1 - \mathrm{e}^{-\frac{t}{\tau}}\right) \cdot \ecv{U_q} \\
    &= \frac{C \cdot R_2}{\tau} \cdot \mathrm{e}^{-\frac{t}{\tau}} \cdot \ecv{U_q} + \left(1 - \mathrm{e}^{-\frac{t}{\tau}}\right) \cdot \ecv{U_q} \\
    &= \frac{R_2}{R_1 + R_2} \cdot \mathrm{e}^{-\frac{t}{\tau}}  \cdot \ecv{U_q} + \left(1 - \mathrm{e}^{-\frac{t}{\tau}}\right) \cdot \ecv{U_q} \\
    &= \left( 1 + \frac{R_2}{R_1+R_2} \cdot \mathrm{e}^{-\frac{t}{\tau}} - \mathrm{e}^{-\frac{t}{\tau}} \right) \cdot \ecv{U_q} \\
    &= \left( 1 - \frac{R_1}{R_1+R_2} \cdot \mathrm{e}^{-\frac{t}{\tau}} \right) \cdot \ecv{U_q}
\end{align*}
\end{itemize}
\begin{minipage}{0.5\textwidth}
\begin{itemize}
    \item[d)] Skizze für $\ecv{u_C}$, $\ecv{u_2}$ und $\ecv{u_{R_2}}$.
    \begin{align*}
        \ecv{u_{R_2}}(0^-) &= \ecv{u_2}(0^-) = 0 \\
        \ecv{u_{R_2}}(0^+) &= \ecv{u_2}(0^+) = \frac{R_2}{R_1+R_2} \cdot \ecv{U_q}%\\
        %\ecv{u_{R_2}}(t>0) &= \frac{R_2}{R_1+R_2}\cdot\mathrm{e}^{-\frac{t}{\tau}}\cdot \ecv{U_q} 
    \end{align*}
\end{itemize}
\end{minipage}%
\begin{minipage}{0.5\textwidth}
\begin{figure}[H]
    \includegraphics{Tikz/pdf/Übungen/plot_uebung_einschaltverhalten_rs_rcs_dc.pdf}
\end{figure}
\end{minipage}
\newpage % ACHTUNG <- PAGEBREAK (LAYOUT)
}


%%%%%%%%%%%%%%%%%%%%%%%%%%%%%%%%%%%%%%%%%%%%%%%%%%%%%%%%%%%%%%%%%%%%%%%%%%%%%%%%%%%

\subsection{Einschaltverhalten (DC) eines verlustbehafteten Kondensators\label{AufgEinschaltverhalten1}}
\Aufgabe{
    % Übernommen von Prof. Exknowski, FH SWF Hagen, 2024, ET3 (Übung 6, Aufgabe 2), leicht verändert
    Das Einschaltverhalten der dargestellten Schaltung soll untersucht werden.
    \begin{figure}[H]\centering
        \includegraphics{Tikz/pdf/Übungen/circ_uebung_einschaltverhalten_rc_schaltung.pdf}% U_q an R1 in Reihe mit (R2 parallel zu C) ab t=0
    \end{figure}

    Zum Zeitpunkt $t = 0$ wird der Schalter umgelegt, so dass die Gleichspannung $U_q$ an der Schaltung anliegt. Die Kapazität ist vor dem Schaltzeitpunkt vollständig entladen.
    \begin{itemize}
        \item[a)] Bestimmen Sie die Gleichung für die Spannung an der Kapazität $u_C(t)$ für $t \geq 0$. 
        \item[b)] Zeichnen Sie den zeitlichen Verlauf der Spannungen $u_C(t)$ und $u_{R1}(t)$ sowie des Stromes $i_C(t)$. Kennzeichnen Sie charakteristische Punkte.
        \item[c)] Nach vollständigem Aufladen der Kapazität $C$ auf $U_{0}$ wird der Schalter wieder in Ausgangsposition gebracht. 
        Wie lange braucht die Kapazität $C$, um auf $10\,\%$ der Spannung $U_{0}$ zu entladen? 
        \item[d)] Die Kapazität wird wie in c) entladen, allerdings mit offenem Schalter (Leerlauf). 
        Wie lange dauert der Entladevorgang (von $U_0$ bis $10\,\%\cdot U_0$)? 
        Vergleichen Sie die Dauer mit der aus c). 
    \end{itemize}
\newpage % ACHTUNG <- PAGEBREAK (LAYOUT)
}
\Loesung{
    % Übernommen von Prof. Exknowski, FH SWF Hagen, 2024, ET3 (Übung 6, Aufgabe 2), leicht verändert
 a) ESB ($t \geq 0$): $U_q$ an Schaltung: $R_1$ in Serie mit Parallelschaltung aus $R_2$ und $C$.

 \underline{1. DGL aufstellen für $u_C$ ($t \geq 0$)}
 \begin{align*}
    u_{R1} + u_C &= U_q &  u_{R_1} &= R_1 \cdot i_{R1}\\
    R_1 \cdot i_{R1} + u_C &= U_q &  i_{R1} &= i_{R2} + i_C\\
    R_1 \cdot (i_C + i_{R2}) + u_C &= U_q &  i_C &= C \cdot \dt\, u_C \qquad i_{R2} = \frac{u_C}{R_2}\\
    C \cdot R_1 \cdot \dt\, u_C + \frac{R_1}{R_2} \cdot u_C + u_C &= U_q &&\Big| \cdot R_2\\
    C \cdot R_1 \cdot R_2 \cdot  \dt\, u_C + \left(R_1 + R_2 \right) \cdot u_C &= U_q \cdot R_2 &&\Big| : (R_1+R_2)\\
    \underbrace{C \cdot \frac{R_1 \cdot R_2}{R_1+R_2}}_{\tau} \cdot \dt\, u_C + u_C &= U_q \cdot \frac{R_2}{R_1+R_2}
 \end{align*}

 \underline{2. Homogene Lösung und 3. Partikuläre Lösung ($t \to \infty$)}
 \begin{align*}
    u_{C,h} &= K \cdot \mathrm{e}^{\lambda t} = K \cdot \mathrm{e}^{-\frac{t}{\tau}} & \tau &= C \cdot \frac{R_1 \cdot R_2}{R_1+R_2} \\[2pt]
    u_{C,p} &= U_q \cdot \frac{R_2}{R_1+R_2} & &\text{$C$ entspricht Leerlauf}
\end{align*}

\underline{4. Überlagerung und 5. Konstante $K$ bestimmen}
\begin{align*}
    u_C(t) = u_{C,h} + u_{C,p} &= K \cdot \mathrm{e}^{-\frac{t}{\tau}} + U_q \cdot \frac{R_2}{R_1+R_2} \\
    u_C(0) &= K \cdot \redcancel{\mathrm{e}^{0}} + U_q \cdot \frac{R_2}{R_1+R_2} \overset{!}{=} 0 &
    \Rightarrow K &= -U_q \cdot \frac{R_2}{R_1+R_2} \\
    u_C(t) &= U_q \cdot \frac{R_2}{R_1+R_2} \cdot \left( 1 - \mathrm{e}^{-\frac{t}{\tau}} \right)
\end{align*}

\begin{minipage}[t]{0.48\textwidth}
    b) % Berechnungen für Skizzen
    % Aufladevorgang RC Spannung mit Tau
    % AxisEnv 1: Spannung am Kondensator
    Skizze $u_C(t)$, $u_{R_1}$ und $i_C$ mit $u_C(t)$ aus a):
    \begin{align*}
        u_{R1}(t) &= U_q - u_C(t) \\
            &= U_q - U_q \cdot \frac{R_2}{R_1+R_2} \cdot \left( 1 - \mathrm{e}^{-\frac{t}{\tau}} \right) \\
            &= U_q \cdot \frac{R_1}{R_1+R_2} + U_q \cdot \frac{R_2}{R_1+R_2} \cdot \mathrm{e}^{-\frac{t}{\tau}} \\[4pt]
        i_C(t) &= C \cdot \dt\, u_C(t) \\
            &= C \cdot U_q \cdot \frac{R_2}{R_1+R_2} \cdot \frac{1}{\tau} \cdot \mathrm{e}^{-\frac{t}{\tau}} \\
            &= U_q \cdot \frac{\colorcancel{C}{purple} \cdot \redcancel{R_2}}{\colorcancel{R_1+R_2}{violet}} \cdot \frac{\colorcancel{R_1+R_2}{violet}}{\colorcancel{C}{purple} \cdot R_1 \cdot \redcancel{R_2}}  \cdot \mathrm{e}^{-\frac{t}{\tau}}\\
            &= \frac{U_q}{R_1} \cdot \mathrm{e}^{-\frac{t}{\tau}}
    \end{align*}
\end{minipage}%
\begin{minipage}[t]{0.48\textwidth}\centering
    Zeitverläufe:\\
    \includegraphics{Tikz/pdf/Übungen/plot_uebung_einschaltverhalten_rc_schaltung.pdf}% Skizzen zu Aufgabenteil b) uC, uR1, iC
\end{minipage}
\begin{itemize}
\item[c)] Entladen von $C$ über Parallelschaltung aus $R_1$ und $R_2$ für $t'\geq0$ bis $U_0 \cdot 10\,\%$ \\
Entlade-Zeitkonstante: $\tau_1= \tau = C \cdot \frac{R_1 \cdot R_2}{R_1+R_2}$:
\begin{align*}
    u_C(t') = U_0 \cdot \mathrm{e}^{-\frac{t'}{\tau}} &\overset{!}{=} U_0 \cdot 10\,\% \\
    \mathrm{e}^{-\frac{t'}{\tau}} &= \frac{1}{10} \\
    -\frac{t'}{\tau} &= \ln\left(\frac{1}{10}\right) = -\ln\left(10\right) \\
    t' &= \ln\left(10\right) \cdot \tau \approx 2,3025 \cdot \tau
\end{align*}
\item[d)] Entladen von $C$ über $R_2$, wenn Schalter öffnet (Leerlauf) für $t'\geq0$ bis $U_0 \cdot 10\,\%$.\\
Lösung wie bei Aufgabe c) nur mit Entlade-Zeitkonstante $\tau_2 = R_2 \cdot C$. \\
Die Kapazität entlädt langsamer als in Aufgabe c) ($\tau_2 > \tau$), da $R_2 > R_1||R_2$.
\end{itemize}
}

%%%%%%%%%%%%%%%%%%%%%%%%%%%%%%%%%%%%%%%%%%%%%%%%%%%%%%%%%%%%%%%%%%%%%%%%%%%%%%%%%%%

% Übung 7, Prof. Exnowski
\subsection{Schaltverhalten einer RL-Serienschaltung}\label{Einschaltverhalten RL-Serie}
\Aufgabe{
    % Übernommen von Prof. Exknowski, FH SWF Hagen, 2024, ET3 (Übung 7, Aufgabe 1), leicht verändert
    Das Schaltverhalten, der in der linken Abbildung dargestellten Schaltung, soll untersucht werden.\\
    Der Verlauf der Spannung $u_1(t)$ ist in der rechten Abbildung dargestellt.

\begin{minipage}{\textwidth}\centering
    \begin{minipage}{0.25\textwidth}\centering
    \includegraphics{Tikz/pdf/Übungen/circ_uebung_7.2.pdf}% Vgl. ET3 Judith Ackers, Übung 7.2
    \end{minipage}
    \quad\quad
    \begin{minipage}{0.25\textwidth}\centering
    \includegraphics{Tikz/pdf/Übungen/plot_uebung_7.2.pdf}% Vgl. ET3 Judith Ackers, Übung 7.2
    \end{minipage}
\end{minipage}
    
    Vor dem Zeitpunkt $t=0$ war die Spannung $u_1 = 0$, so dass zum Zeitpunkt $t=0$ keine Energie in der Spule gespeichert ist. Weiterhin gilt $t_1 \gg 5\tau$.

    \begin{itemize}
        \item[a)] Es soll der Aufladevorgang ($0<t<t_1$) betrachtet werden. Es gilt $\tau = 1\,ms$ und $R=100\,\Omega$. Wie groß ist die Induktivität $L$?
        \item[b)] Es soll weiterhin der Aufladevorgang ($0<t<t_1$) betrachtet werden. Gegeben ist nun die Spannung $U_0=1\,kV$. Wie groß ist der Strom durch die Spule $i_L$ zum Zeitpunkt $t=3\tau$?
        \item[c)] Es wird weiterhin der Aufladevorgang betrachtet. Nun gilt jedoch: $R = 5\,\Omega$, $L=100\,mH$ und $U_0=2,8\,kV$.
        \begin{itemize}
        \item Nach welcher Zeit $t_0$ ist der Einschwingvorgang abgeschlossen?
        \item Wie groß sind der Spulenstrom $i_L$ und die Spulenspannung $u_L$ nach $t=50\,ms$?
        \end{itemize}
        \item[d)] Nun soll der Entladevorgang ($t>t_1$) betrachtet werden. Es gelten die Werte aus dem vorangehenden Aufgabenteil. Wie groß sind Spulenstrom $i_L$ und Spulenspannung $u_L$ nach $t=t_1 +50\,ms$?
        \item[e)] Zeichnen Sie die zeitlichen Verläufe der Spulenspannung $u_L$, des Stromes $i_L$, sowie der Spannung $u_R$. Kennzeichnen Sie die Zeitkonstante $\tau$, sowie weitere charakteristische Punkte. Nehmen Sie an $t_1 = 120\,ms$.
        \end{itemize}
}

\Loesung{
    % Übernommen von Prof. Exknowski, FH SWF Hagen, 2024, ET3 (Übung 7, Aufgabe 1), leicht verändert
    \begin{align*}
        \text{DGL (allgemein):}&& \frac{L}{R}\cdot\dt\, i &= \frac{U_1}{R_1} \quad\text{(Einschalten)}\\
        \text{Lösungsansatz:}&& i &= \frac{U_0}{R}\cdot(1-e^{-\frac{t}{\tau}})\quad\text{mit}\quad\tau=\frac{L}{R}
    \end{align*}
        \\
        
    a) Induktivität berechnen.\\[-2em]
        \begin{center}
            $\tau=\frac{L}{R}$\qquad$L=\tau\cdot R=1\,ms\cdot 100\,\Omega=0,1\,H$
        \end{center}
    b) Aufladevorgang, Spulenstrom $i_L$ zum Zeitpunkt $t=3\tau$.\\[-1em]
        \begin{align*}
            i(t=3\cdot\tau) &= \frac{U_0}{R}(1-e^{-\frac{t}{\tau}})\\
            &=\frac{1\,kV}{100\,\Omega}\cdot(1-e^{-3}) = 10\,A\cdot0,95\\
            &=9,5\,A
        \end{align*}    
    c) Zeit bis zum Abschluss des Einschwingvorgangs, Spulen -Strom und -Spannung nach $50\,ms$.\\[-1em]
        \begin{align*}
        \tau&=\frac{L}{R}=\frac{0,1\,H}{5\,\Omega}=0,02\,s=20\,ms\\\\
        i(t=50\,ms)&=\frac{U_0}{R}\cdot(1-e^{-\frac{t}{\tau}})\\
        &=\frac{2,8\,kV}{5\,\Omega}\cdot(1-e^{-\frac{50\,ms}{20\,ms}})\\
        &=0,56\,kA\cdot(0,918)=0,514\,kA\\\\
        u_L&=L\cdot\dt i_L = \frac{L}{R}\cdot U_0 \cdot (-\frac{1}{\tau})\cdot(-e^{-\frac{t}{\tau}})\\
        &= U_0\cdot e^{-\frac{t}{\tau}}\\\\
        u_L(t=50\,ms)&=2,8\,kV\cdot e^{-\frac{50\,ms}{20\,ms}}=229,8\,V\\
        t_0&=5\cdot\tau=5\cdot20\,ms=100\,ms
        \end{align*}
    d) Entladevorgang, Spulen -Strom und -Spannung nach $t = t_1 + 50\,ms$.\\[-1em]
        \begin{align*}
            \text{DGL (Entladevorgang):}&& \frac{L}{R}\cdot\dt\, i+i &=0\\
            \text{Lösungsansatz:}&& i &= \frac{U_0}{R}\cdot e^{-\frac{t}{\tau}}\quad\text{mit}\quad\tau=\frac{L}{R}\\
        \end{align*}
        \begin{align*}
            u_L&=L\cdot\dt i = \frac{L}{R}\cdot u_0\cdot (-\frac{1}{\tau})\cdot e^{-\frac{t}{\tau}}\\
            &=-U_0\cdot e^{-\frac{t}{\tau}}\\
            i(50\,ms)&=\frac{2,8\,kV}{5\Omega}\cdot e^{-\frac{50\,ms}{20\,ms}}=46\,A\\
            u_L(50\,ms)&=-2,8\,kV\cdot e^{-\frac{50\,ms}{20\,ms}}=-230\,V
        \end{align*}
    e) Zeitliche Verläufe von $u_L, i_L, u_R$.\\[-2em]
    \begin{center}
        \includegraphics{Tikz/pdf/Übungen/plot_uebung_7.1_a.pdf}
    \end{center}
}

\subsection{Einschaltverhalten einer RL-Serienschaltung}\label{Einschaltverhalten_RL-Serie}
% Übernommen von Prof. Exknowski, FH SWF Hagen, 2024, ET3 (Übung 7, Aufgabe 2), leicht verändert
\Aufgabe{

Das Einschaltverhalten einer Serienschaltung aus ohmschem Widerstand $R$ und Induktivität\\
 $L=100\,mH$ soll untersucht werden. Zum Zeitpunkt $t=0$ wird an die Serienschaltung eine Spannung $u=U_0 = 35\,V$ angelegt. Für $t\leq0$ gilt $u=0$. Die Spulenspannung zum Zeitpunkt $t_1=3\,ms$ ist bekannt und beträgt $u_L(t_1)= 26\,V$.\\[2ex]
Bestimmen Sie die Zeitkonstante $\tau$, den Widerstand $R$, sowie die Spannung $u_R(t=5\,ms)$.
}

\Loesung{%

    \begin{align*}
        \text{DGL:}&&\quad\frac{L}{R}\cdot\dt i+i&=\frac{U_0}{R}\\
        \text{Lösung:}&&i&=\frac{U_0}{R}\cdot(1-e^{-\frac{t}{\tau}})\quad\text{mit}\quad \tau = \frac{L}{R}
    \end{align*}
    \begin{align*}
        u_L &= L\cdot\dt i =U_0\cdot e^{-\frac{t}{\tau}}\\
        u_R &= u-u_L = U_0\cdot(1-e^{-\frac{t}{\tau}})\\
        i(t=3\,ms)&= \frac{U_0}{R}\cdot(1-e^{-\frac{t}{\tau}})\\
        u_L(t=3\,ms)&= U_0\cdot e^{-\frac{t}{\tau}} = 26\,V\\
        \frac{u_L(t=3\,ms)}{U_0}&=e^{-\frac{t}{\tau}}\\
        -\frac{t}{\tau}&=\ln\left(\frac{u_L(t=3ms)}{U_0}\right)\\
        \tau&=-\frac{t}{\ln\left(\frac{u_L(t=3ms)}{U_0}\right)} = -\frac{3ms}{\ln\left(\frac{26\,V}{35\,V}\right)} = 10,1\,ms\\
        R &= \frac{L}{\tau} = \frac{100\,mH}{10,1\,ms} = 9,9\,\Omega\\
        u_R(t=5\,ms)&=U_0\cdot(1-e^\frac{-5\,ms}{10,1\,ms})\\
        &=35\,V\cdot 0,39 = 13,65\,V
    \end{align*}

}

%%%%%%%%%%%%%%%%%%%%%%%%%%%%%%%%%%%%%%%%%%%%%%%%%%%%%%%%%%%%%%%%%%%%%%%%%%%%%%%%%%%

\subsection{ET3 Ü8 Schaltverhalten eines RC-Netzwerks}\label{AufgSchaltverhaltenRCNetzwerk}
\Aufgabe{
    % Übernommen von Prof. Exknowski, FH SWF Hagen, 2024, ET3 (Übung 8), leicht verändert
    \begin{minipage}{0.6\textwidth}
        Der Schalter in der rechts dargestellten Schaltung wird zum Zeitpunkt $t=0$ geschlossen. 
        \\
        
        \begin{tabular}{lll}
            & $R_1 = 1{,}2\,\text k \Omega$	& $R_2 = 2\,\text k \Omega$	\\
            & $C_1 = 1\,\mu \text F$ & $R_3 = 500\,\Omega$	\\
            & $U_q = 250\,\text{V}$	& \\
        \end{tabular}
        \\
        
        Bestimmen Sie den zeitlichen Verlauf des Stromes $i_3$ durch den Widerstand $R_3$ und zeichnen Sie diesen als Funktion von der Zeit (Liniendiagramm) für $-\tau<t<5\tau$. \\
        
        Das Netzwerk sei zum Zeitpunkt $-\tau$ in einem stationären Zustand. Dabei ist $\tau$ die Zeitkonstante des Netzwerks.
    \end{minipage}
    \hspace{0.3cm}
    \begin{minipage}{0.38\textwidth}
        \centering
        \includegraphics{Tikz/pdf/Übungen/circ_uebung_8.pdf}
    \end{minipage}
}
\Loesung{
DGL für $\ecv{u_C}$ aufstellen und darüber mit $\eci{i_3}=\frac{\ecv{u_C}}{R_3}$ den Strom $\eci{i_3}$ bestimmen.\\ 
$R_1$ und $R_2$ nach dem Schaltzeitpunkt ($t>0$) als Parallelschaltung $R_{12}$ zusammenfassen.

\textbf{1. DGL} für $\ecv{u_C}$ ($t > 0$) mit $\eci{i_3}=\frac{\ecv{u_C}}{R_3}$: 
\begin{align*}
    \ecv{U_q} &= \ecv{u_{12}} + \ecv{u_C} &&\text{mit}\quad \ecv{u_{12}} \mathrlap{\text{ über Parallelschaltung aus $R_1$ und $R_2$}}\\
    &= R_{12} \cdot \eci{i_{12}} + \ecv{u_C} &&\text{mit}\quad R_{12}=R_1||R_2,& \eci{i_{12}}&=\eci{i_C}+\eci{i_3}\\
    &= R_{12} \cdot (\eci{i_C} + \eci{i_3}) + \ecv{u_C} &&\text{mit}\quad \eci{i_{C}}=C\cdot\dt\,\ecv{u_C},& \eci{i_3}&=\frac{\ecv{u_C}}{{R_3}}\\
    &= R_{12} \cdot (C \cdot \dt\,\ecv{u_C} + \frac{\ecv{u_C}}{{R_3}}) + \ecv{u_C}\\
    &= C \cdot R_{12} \cdot \dt\,\ecv{u_C} + \left( \frac{R_{12}}{{R_3}} + 1 \right) \cdot \ecv{u_C}& &\bigg|\ \cdot R_3\\
    \ecv{U_q} \cdot R_3
    &= C \cdot R_{12} \cdot R_3 \cdot \dt\,\ecv{u_C} + \left(R_{12}+R_3\right) \cdot \ecv{u_C}& &\bigg| :\left(R_{12}+R_3\right)\\
    \ecv{U_q} \cdot \frac{R_3}{R_{12}+R_3}
    &= \underbrace{C \cdot \frac{R_{12} \cdot R_3}{R_{12} + R_3}}_{\tau} \cdot \dt\,\ecv{u_C} + \ecv{u_C}& &\Longrightarrow\quad \tau=C \cdot \frac{R_{12} \cdot R_3}{R_{12} + R_3} 
\end{align*}
\textbf{3. Homogene Lösung} (flüchtig) und \textbf{2. Partikuläre Lösung} ($t \to \infty$, eingeschwungen):
\begin{align*}
    \ecv{u_{C,\mathrm{h}}} &= K \cdot \mathrm{e}^{-\frac{t}{\tau}}& &\text{mit}\quad \tau=C \cdot \frac{R_{12} \cdot R_3}{R_{12} + R_3} \\
    \ecv{u_{C,\mathrm{p}}} &= \ecv{U_q} \cdot \frac{R_3}{R_{12}+R_3}& &\text{$C_1$ entspricht Leerlauf}
\end{align*}
\textbf{4. Überlagerung} und \textbf{5. Konstante(n) bestimmen}:
\begin{align*}
    \ecv{u_C} &= \ecv{u_{C,\mathrm{h}}} + \ecv{u_{C,\mathrm{p}}}\\
    \ecv{u_C}(t=0) &= K \cdot \redcancel{\mathrm{e}^{0}} + \ecv{U_q} \cdot \frac{R_3}{R_{12}+R_3} \overset{!}{=} \ecv{U_q}& &\text{(Anfangsbedingung)}\\
    \Rightarrow K &= \ecv{U_q} \cdot \left( 1 - \frac{R_3}{R_{12}+R_3} \right)\\
    &= \ecv{U_q} \cdot \frac{R_{12}}{R_{12}+R_3}\\[2pt]
    \ecv{u_C} &= \ecv{U_q} \cdot \left( 1 - \frac{R_3}{R_{12}+R_3} \right) \cdot \mathrm{e}^{-\frac{t}{\tau}} + \ecv{U_q} \cdot \frac{R_3}{R_{12}+R_3}& &\bigg| :R_3\\[2pt]
    \eci{i_3} &= \ecv{U_q} \cdot \left( \frac{1}{R_3} - \frac{1}{R_{12}+R_3} \right) \cdot \mathrm{e}^{-\frac{t}{\tau}} + \ecv{U_q} \cdot \frac{1}{R_{12}+R_3}
\end{align*}
Zeitkonstante $\tau$ sowie Anfangs- und Endwerte des Stromes $i_3$ für Skizze:\\
\begin{minipage}{0.5\textwidth}% Grenzwerte
\begin{align*}
    R_{12} &= R_1||R_2 = \frac{1,2\,\mathrm{k}\Omega \cdot 2\,\mathrm{k}\Omega}{1,2\,\mathrm{k}\Omega + 2\,\mathrm{k}\Omega} = 750\,\Omega\\[2pt]
    \tau &= C_1 \cdot \frac{R_{12} \cdot R_3}{R_{12} + R_3} \\
    &= 1\,\mu \mathrm{F} \cdot \frac{750\,\Omega \cdot 500\,\Omega}{750\,\Omega + 500\,\Omega} = 0,3\,\mathrm{ms} \\[2pt]
    \eci{i_3}(t=0_{-}) &= \frac{\ecv{U_q}}{R_{2}+R_3} = \frac{250\,\mathrm{V}}{2\,\mathrm{k}\Omega + 500\,\Omega} = 0,1\,\mathrm{A}\\[2pt]
    \eci{i_3}(t=0_{+}) &= \frac{\ecv{U_q}}{R_3} = \frac{250\,\mathrm{V}}{500\,\Omega} = 0,5\,\mathrm{A}\\[2pt]
    \eci{i_3}(t\to\infty) &= \frac{\ecv{U_q}}{R_{12}+R_3} = \frac{250\,\mathrm{V}}{750\,\Omega + 500\,\Omega} = 0,2\,\mathrm{A}
\end{align*}
\end{minipage}% Plot
\begin{minipage}{0.5\textwidth}\centering%
\includegraphics{Tikz/pdf/Übungen/plot_uebung_8.pdf}
\end{minipage}
}

%%%%%%%%%%%%%%%%%%%%%%%%%%%%%%%%%%%%%%%%%%%%%%%%%%%%%%%%%%%%%%%%%%%%%%%%%%%%%%%%%%%

\subsection{Schaltvorgang bei Wechselspannung}\label{AufgWechselspannungSchalten}
\Aufgabe{
    Zum Zeitpunkt $t=t_0$ wird in der gezeigten Schaltung die Wechselspannung $u_1(t)$ an 
    die Gesamtschaltung aus $R_1$ in Reihe zur Parallelschaltung aus $R_2$ und $C$ angelegt. 
    Für $t<t_0$ ist die Schaltung im stationären Zustand.
    Die Quellenspannung ist gegeben durch $u_1(t) = \hat{U}_1 \cdot \sin(\omega t)$.
    \begin{figure}[H]\centering
        \includegraphics{Tikz/pdf/Übungen/circ_uebung_einschaltverhalten_rs_rlp.pdf}
    \end{figure}
    \begin{itemize}
    \item[a)] Welcher Strom $i_{L,e}(t)$ stellt sich im eingeschwungenen Zustand bei der Induktivität ein? \\
    Bestimmen Sie die Amplitude und den Nullphasenwinkel des Stromes.
    \item[b)] Stellen Sie die DGL von $i_L(t)$ für $t \geq t_0$ auf.
    Welche Zeitkonstante ergibt sich für den flüchtigen Zustand?  
    Hinweis (DGL):\qquad $\tau \cdot \frac{\mathrm{d}}{\mathrm{d}t}\, i_L + i_L = \text{störterm}$
    \item[c)] Wann muss geschalten werden, damit der flüchtige Zustand von $i_L(t)$ minimal wird? 
        Bei welchen möglichen Schaltzeitpunkten wird dieser maximal? 
        Geben Sie die Antwort mithilfe der Phasenverschiebung $\varphi$ zwischen $u_1(t)$ und $i_L(t)$ an.
    \item[d)] Sei zum Schaltzeitpunkt $t_0$ der eingeschwungene Zustand von $i_{L,e}(t)$ 
    bei seinem negativen Scheitelwert. Die Zeitkonstante $\tau$ beträgt eine Periodendauer $T=\frac{2\pi}{\omega}$.
    Bestimmen Sie das Strommaximum während des Einschwingvorgangs.
    \end{itemize}
}
\Loesung{
    \begin{itemize}
    \item[a)] Eingeschwungener Zustand ($t \to \infty$) mit komplexer Wechselstromrechnung:
    \begin{align*}
        \eci{\underline{I}_{L,e}} &= \frac{\ecv{\underline{U}_L}}{\underline{Z}_L}
        = \frac{\ecv{\underline{U}_1}}{\underline{Z}_L} \cdot \frac{R_2 || \underline{Z}_L}{R_1 + R_2 || \underline{Z}_L} 
        \qquad\text{mit}\quad
        \ecv{\underline{U}_1} = \ecv{\hat{U}_1} \cdot \mathrm{e}^{\mathrm{j}0} = \ecv{\hat{U}_1}\\
        &= \frac{\ecv{\hat{U}_1}}{\underline{Z}_L} \cdot \frac{\frac{R_2\cdot\underline{Z}_L}{R_2+\underline{Z}_L}}{R_2 + \frac{R_1\cdot\underline{Z}_L}{R_2+\underline{Z}_L}}
        \qquad \bigg| \cdot\frac{R_2+\underline{Z}_L}{R_2+\underline{Z}_L}\\
        &= \frac{\ecv{\hat{U}_1}}{\redcancel{\underline{Z}_L}} \cdot \frac{R_2\cdot\redcancel{\underline{Z}_L}}{R_1\cdot(R_2+\underline{Z}_L)+R_2\cdot\underline{Z}_L}\\
        &= \ecv{\hat{U}_1} \cdot \frac{R_2}{R_1 \cdot R_2 + \mathrm{j}\omega L \cdot(R_1+R_2)}  \\
        \varphi &= 0 + \redcancel{\arctan\left(\frac{0}{R_2}\right)} - \arctan\left(\frac{\omega L \cdot(R_1+R_2)}{R_1 \cdot R_2}\right) \\
        &= - \arctan\left(\frac{\omega L \cdot(R_1+R_2)}{R_1 \cdot R_2}\right),\qquad -\frac{\pi}{2} \leq \varphi \leq 0 \\
        \eci{\hat{I}_{L,e}} &= \ecv{\hat{U}_1} \cdot \frac{R_2}{\sqrt{\left(R_1\cdot R_2\right)^2 + \left(\omega L\cdot (R_1+R_2)\right)^2}}\\
        \eci{i_{L,e}}(t) &= \eci{\hat{I}_{L,e}} \cdot \sin(\omega t + \varphi) \qquad \text{Strom eilt nach }\ecv{u_1}\text{ mit }\varphi<0 
    \end{align*}
    \item[b)] DGL für $i_L(t)$ ($t \geq t_0$):
    \begin{align*}
        \ecv{u_1} &= \ecv{u_{R_1}} + \ecv{u_L} \\
        &= R_1 \cdot (\eci{i_2} + \eci{i_L}) + L \cdot \dt\, \eci{i_L}\\
        &= R_1 \cdot \left( \frac{\ecv{u_L}}{R_2} + \eci{i_L} \right) + L \cdot \dt\, \eci{i_L}\\
        &= \frac{R_1}{R_2} \cdot L \cdot \dt\, \eci{i_L} + R_1 \cdot \eci{i_L} + L \cdot \dt\, \eci{i_L}  \\
        &= L \cdot \left( \frac{R_1}{R_2} + 1 \right) \cdot \dt\, \eci{i_L} + R_1 \cdot \eci{i_L} \qquad \bigg| \cdot \frac{1}{R_1}\\
        \frac{\ecv{u_1}}{R_1} &= \underbrace{L \cdot \left( \frac{1}{R_1} + \frac{1}{R_2} \right)}_{\ =\tau} \cdot \dt\, \eci{i_L} + \eci{i_L}\\ 
        \tau &= L \cdot \left( \frac{1}{R_1} + \frac{1}{R_2} \right) = \frac{L}{R_1||R_2}
    \end{align*}
    \item[c)] Schaltzeitpunkte für minimalen und maximalen flüchtigen Zustand ($t_{min},\ t_{max}$). \\ 
    Kein Einschwingen, wenn 
    $\eci{i_{L,f}}=0 \Leftrightarrow \eci{i_L}=\eci{i_{L,e}} \Leftrightarrow \eci{i_L}(t_0)=\eci{i_{L,e}}(t_0)$ gilt.
    \begin{align*}
        \intertext{D.h. für verschwindenden flüchtigen Zustand (kein Einschwingvorgang) folgt:}
        &\text{AB:}&\eci{i_L}(t_0) = \eci{i_{L,e}}(t_0) &= \eci{\hat{I}_{L,e}} \cdot \sin(\omega t_0 + \varphi) \overset{!}= 0 &
            &\Rightarrow \eci{i_L}(t')=\eci{i_{L,e}}(t') \Leftrightarrow \eci{i_{L,f}}(t')=0\\
        &&\Leftrightarrow \omega t_0 + \varphi &\overset{!}{=} 0 + n\cdot \pi &
            &\text{mit} \quad n \in \mathbb{N} \\
        &&t_{min} &= \frac{n\cdot \pi - \varphi}{\omega} &
            &\text{entspricht Nulldurchgängen von $\eci{i_{L,e}}$}\\
        \intertext{Und für maximalen flüchtigen Zustand (Schaltzeitpunkt $90\degree$ versetzt) folgt:}
        &&t_{max} &= \frac{n\cdot \pi - \varphi}{\omega} + \frac{T}{4} &
            &\text{entspricht Extremstellen von $\eci{i_{L,e}}$}
    \end{align*}
    \item[d)] Strommaximum während des Einschwingvorgangs gesucht. 
    Schaltzeitpunkt $t_0$ bei negativem Scheitelwert von $i_{L,e}$. 
    Sei $t' = t - t_0$, d.h. $t'=0$ zum Schaltzeitpunkt, so folgt:
    \begin{align*}
        \eci{i_L}(t') &= \eci{i_{L,h}}(t') + \eci{i_{L,e}}(t') \\
        &= K \cdot \mathrm{e}^{-\frac{t'}{\tau}} - \eci{\hat{I}_{L,e}} \cdot \cos(\omega t')\\
        \text{AB:}\quad \eci{i_L}(t'=0) &= K - \eci{\hat{I}_{L,e}}  \overset{!}{=}0 \quad \Rightarrow \quad K = \eci{\hat{I}_{L,e}}\\
        \eci{i_L}(t') &= \eci{\hat{I}_{L,e}} \cdot \left( \mathrm{e}^{-\frac{t'}{\tau}} - \cos(\omega t') \right) \\
        \intertext{Maximum bei positivem Scheitelwert ($180\degree$ nach $t_0$):}
        \eci{i_{L,max}} &= \eci{i_L}(t' = \frac{\pi}{\omega}) \\
        &= \eci{\hat{I}_{L,e}} \cdot \left( \mathrm{e}^{-\frac{T/2}{T}} - \cos(\pi) \right) \\
        &= \eci{\hat{I}_{L,e}} \cdot \left( \mathrm{e}^{-\frac{1}{2}} + 1\right) = 1,606 \cdot \eci{\hat{I}_{L,e}}
    \end{align*}
    \end{itemize}
}