% RLC Schwingkreis Serie
\begin{tikzpicture}
    %  invisible boundbox
    \draw[draw=none](-0.75,-2.5) rectangle (+6.75,+1); % like RLCp: (7.5 x 3.5) % set draw=black (debug) or draw=none (final)
    %\draw[draw=black](-0.75,-1.75) rectangle (+6.75,+1); % RLCs without iq/uq

    % draw RLC-Glied
    \draw (-0.5,0) %U+
    to [short, name=i, i>^=$\underline{I}$, !i, o-] (0,0)
    to [R, l=$R$] (2,0)
    to [L, l=$\mathrm{j}\omega L$] (4,0)
    to [bipole label style={xshift=+17pt,yshift=-5pt},C, l=$\dfrac{1}{\mathrm{j}\omega C}$] (6,0)% alt. yshift=5pt, bbox höher
    to [short, -o] (6.5,0); %U-

    % draw nodes for U arrows
    \draw (0,-0.6)
    to [open, name=R, v=$\underline{U}_R$, !v](2,-0.6)
    to [open, name=L, v=$\underline{U}_L$, !v](4,-0.6)
    to [open, name=C, v=$\underline{U}_C$, !v](6,-0.6);
    
    %draw arrows
    \iarronly{i};
    \varronly{R};
    \varronly{L};
    \varronly{C};

    % Uq Spannungsquelle
    \draw (-0.5,0)
    to [short, *-] (-0.5,-1.75)
    to [vsource] (6.5,-1.75)
    to [short, -*] (6.5,0);
    \draw (-0.75,-2.35) to [open, name=Uq, v, !vi] (6.75,-2.35); % dummy for longer v arrow
    \node at (1.3275,-2.1){\color{voltage}$\underline{U}= \underline{U}_q$};
    \varronly{Uq};
\end{tikzpicture}