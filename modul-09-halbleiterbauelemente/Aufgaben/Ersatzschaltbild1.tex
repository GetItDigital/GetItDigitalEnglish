\subsection{Erstellung eines Kleinsignal-Ersatzschaltbilds\label{Aufg20}}
% Alt: Aufgabe 20
\Aufgabe{
    Zeichnen Sie das Kleinsignalersatzschaltbild der gegebenen Schaltung.
    \begin {figure} [H]
       \centering
       \begin{tikzpicture} 
    \draw (0,0) node[npn] (Q1) {};
    
    \draw (Q1.C)
    to [short] (0,1)
    to[R, l_= $500 \, \Omega$, -*] (0,3)
    to [short, -o] (0,3.5) node[right, color=voltage] {$U_\mathrm{V}$};
    
    \draw (0,1)
    to[short, *-o] (2,1);
    
    \draw (0,3)
    to[short, -]
    (-2,3) to[R=$75 \, k\Omega$] (-2,1)
    to [short, -*] (-2,0)
    to [short, -o] (-3,0);
    
    \draw (-2,0) to (Q1.B);
    
    \draw (Q1.E)
    to [short, -*] (0,-2) node[ground]{};
    
    \draw (-3,-2) to[short, o-o] (2,-2);
    
    \draw (2,1) to[open,v^, name=ua] (2,-2);
    \draw (-3,0) to[open,v>, name=ue] (-3,-2);
    
    \varrmore{ua}{$U_\mathrm{A}$};
    \varrmore{ue}{$U_\mathrm{E}$};
\end{tikzpicture}
       \label{fig:FigErsatzschaltbild1}
    \end {figure}
%     \speech{
% **Aufgabe 20**

%  **Aufgabenstellung:**
% Die Aufgabe fordert, dass das **Kleinsignal-Ersatzschaltbild** der gegebenen Schaltung gezeichnet wird.

%  **Beschreibung der Abbildung:**
% Die dargestellte Schaltung zeigt eine **Transistorverstärkerschaltung in Emitterschaltung** mit Widerständen zur Einstellung des Arbeitspunkts.

% - **Spannungsversorgung**:  
%   - Die Schaltung wird mit einer **Versorgungsspannung \( U_V \)** betrieben.
%   - \( U_V \) liegt am oberen Ende der Schaltung.

% - **Basisbeschaltung**:  
%   - Ein **75 k\(\Omega\) Widerstand** verbindet die **Basis des Transistors** mit der **Versorgungsspannung \( U_V \)**.  
%   - Dieser Widerstand sorgt für eine **Vorspannung der Basis**, um den Transistor in den gewünschten Arbeitspunkt zu bringen.

% - **Kollektorkreis**:  
%   - Der **Kollektor ist über einen 500 \(\Omega\) Widerstand** mit \( U_V \) verbunden.  
%   - Die **Ausgangsspannung \( u_A \)** wird am Kollektor abgegriffen.

% - **Eingangssignal**:  
%   - Das **Eingangssignal \( u_E \)** ist an die **Basis des Transistors** gekoppelt.  
%   - Das Signal wird verstärkt und als **invertiertes Signal am Kollektor** ausgegeben.

% - **Emitterschaltung**:  
%   - Der **Emitter ist direkt mit Masse verbunden**, was typisch für eine **Emitterschaltung** ist.

%  **Zusätzliche Hinweise für eine blinde Person:**
% - Diese Schaltung ist eine **Grundschaltung eines Verstärkers** mit einem **Transistor als aktives Element**.
% - Das **Kleinsignal-Ersatzschaltbild** wird durch Ersetzen des Transistors durch sein **äquivalentes Modell** erstellt.  
%   - Dabei werden Parameter wie **differentieller Widerstand \( r_e \)** und **Stromverstärkung \( \beta \)** berücksichtigt.
% - Das **Eingangssignal \( u_E \)** wird an der **Basis eingespeist**, und die verstärkte Version erscheint an der **Kollektor-Emitter-Strecke**.
% - Der **75 k\(\Omega\) Widerstand** dient zur **Gleichspannungsvorspannung** der Basis, der **500 \(\Omega\) Widerstand** beeinflusst die **Verstärkung und den Arbeitspunkt**.
% }

}



\Loesung{
    \begin {figure} [H]
       \centering
       \begin{tikzpicture}

\draw (0,0) to[short, -o] (1,0)node[above]{$\mathrm{B}$} -- (3,0) to[short, -*,R, l=$r_\mathrm{BE}$] (3,-2) to[short, -*] (2,-2)node[ground]{} to[short, -o] (1,-2) to[short, -*] (0,-2)node[ground]{} to[R,l=$R_\mathrm{2}$](0,0);

\draw (3,-2) to[short, -*] (4,-2) node[ground]{} to[short, -*] (5,-2) to[short, -*] (6.5,-2) to[short, -*] (7,-2)node[ground]{} to[short, -o] (8,-2)node[below]{$\mathrm{E}$} -- (9,-2) ;

%Stromquelle
\draw (5,0) to[short,i,name=i2](5,-0.5) to[I,name=I] (5,-1.5) -- (5,-2);
\draw (5,0) to[short, -*] (6.5,0) to[R,l=$r_\mathrm{CE}$] (6.5,-2);
\draw (6,0) to[short, -o] (8,0)node[above]{$\mathrm{C}$} -- (9,0) to[R,l=$R_\mathrm{C}$] (9,-2);

\draw (1,0)to[open,v>, name=ue](1,-2);
\draw (8,0)to[open,v>, name=ua](8,-2);

\varrmore{ue}{$U_\mathrm{E}$};
\varrmore{ua}{$U_\mathrm{A}$};
\iarrmore{i2}{$\mathrm{B}\cdot I_\mathrm{B}$};

\end{tikzpicture}
       \label{fig:LsgErsatzschaltbild1}
    \end {figure}
    Siehe: Abschnitt \ref{fig:ErsatzschaltbildEinesBipolartransistors} und \ref{fig:ErsatzschaltbilderEmitter-Kollektor-UndBasisschaltung}.
%     \speech{
% **Ersatzschaltbild für Aufgabe 20**  

% Das dargestellte Ersatzschaltbild zeigt die Kleinsignal-Ersatzschaltung eines Transistorverstärkers.  
% Es besteht aus folgenden Bauteilen und Anschlüssen:  

% 1. **Widerstände:**  
%    - Widerstand \( R_2 \) ist mit dem Punkt B verbunden und geht gegen Masse.  
%    - Der Basis-Emitter-Widerstand \( r_{BE} \) ist parallel zur Basis-Emitter-Strecke geschaltet.  
%    - Der Kollektorwiderstand \( R_C \) ist zwischen Punkt C und Masse geschaltet.  

% 2. **Spannungen und Ströme:**  
%    - \( U_E \) ist die Eingangsspannung an Punkt B, die über \( R_2 \) an die Basis des Transistors gelangt.  
%    - \( U_A \) ist die Ausgangsspannung am Punkt C, die über \( R_C \) abgenommen wird.  
%    - Der Basisstrom \( I_B \) fließt in die Basis des Transistors ein.  
%    - Der Kollektorstrom beträgt \( \beta \cdot I_B \), dargestellt durch die gesteuerte Stromquelle im Modell.  

% 3. **Transistormodell:**  
%    - Der Transistor wird durch eine gesteuerte Stromquelle ersetzt, die den Kollektorstrom \( \beta \cdot I_B \) liefert.  
%    - Die Basis-Emitter-Strecke wird durch den differentiellen Widerstand \( r_{BE} \) modelliert.  
%    - Die Kollektor-Emitter-Strecke ist über \( R_C \) mit Masse verbunden, wodurch die Ausgangsspannung \( U_A \) entsteht.  

% **Zusammenfassung:**  
% Das Ersatzschaltbild veranschaulicht das Verhalten eines bipolaren Transistors im Kleinsignalbetrieb.  
% Es zeigt, wie die Basis-Emitter-Spannung \( U_{BE} \) den Stromfluss steuert und wie die Verstärkung durch  
% die Stromquelle \( \beta \cdot I_B \) im Kollektorzweig realisiert wird.  
% }

}



