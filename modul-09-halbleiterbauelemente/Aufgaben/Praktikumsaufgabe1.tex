\subsection{Praktikumsaufgabe 1\label{Aufg29}}
\Aufgabe{
    \begin{figure}[H]
    \centering
    \subsection{Praktikumsaufgabe 1\label{Aufg29}}
\Aufgabe{
    \begin{figure}[H]
    \centering
    \subsection{Praktikumsaufgabe 1\label{Aufg29}}
\Aufgabe{
    \begin{figure}[H]
    \centering
    \subsection{Praktikumsaufgabe 1\label{Aufg29}}
\Aufgabe{
    \begin{figure}[H]
    \centering
    \input{Bilder/Aufgaben/Praktikumsaufgabe1.tex}
    \label{fig:Praktikumsaufgabe1}
    \end{figure}

    Berechnen Sie den Widerstand $\mathrm{R_1}$, sodass an $\mathrm{R_2}$ eine Spannung von 1\,V anliegt. Gegeben ist eine Eingangsspannung von $\mathrm{U_E}$ und ein Widerstand $\mathrm{R_2}$.
    
    \begin{itemize}
        \item[a)] Leiten Sie die allgemeine Formel für den Spannungsteiler her.
        \item[b)] Berechnen Sie $\mathrm{R_1}$ für $\mathrm{U_E} = 5\,\text{V}$ und $\mathrm{R_2} = 2\,\text{k}\Omega$.
    \end{itemize}
}


\Loesung{

}
    \label{fig:Praktikumsaufgabe1}
    \end{figure}

    Berechnen Sie den Widerstand $\mathrm{R_1}$, sodass an $\mathrm{R_2}$ eine Spannung von 1\,V anliegt. Gegeben ist eine Eingangsspannung von $\mathrm{U_E}$ und ein Widerstand $\mathrm{R_2}$.
    
    \begin{itemize}
        \item[a)] Leiten Sie die allgemeine Formel für den Spannungsteiler her.
        \item[b)] Berechnen Sie $\mathrm{R_1}$ für $\mathrm{U_E} = 5\,\text{V}$ und $\mathrm{R_2} = 2\,\text{k}\Omega$.
    \end{itemize}
}


\Loesung{

}
    \label{fig:Praktikumsaufgabe1}
    \end{figure}

    Berechnen Sie den Widerstand $\mathrm{R_1}$, sodass an $\mathrm{R_2}$ eine Spannung von 1\,V anliegt. Gegeben ist eine Eingangsspannung von $\mathrm{U_E}$ und ein Widerstand $\mathrm{R_2}$.
    
    \begin{itemize}
        \item[a)] Leiten Sie die allgemeine Formel für den Spannungsteiler her.
        \item[b)] Berechnen Sie $\mathrm{R_1}$ für $\mathrm{U_E} = 5\,\text{V}$ und $\mathrm{R_2} = 2\,\text{k}\Omega$.
    \end{itemize}
}


\Loesung{

}
    \label{fig:Praktikumsaufgabe1}
    \end{figure}

    Berechnen Sie den Widerstand $\mathrm{R_1}$, sodass an $\mathrm{R_2}$ eine Spannung von 1\,V anliegt. Gegeben ist eine Eingangsspannung von $\mathrm{U_E}$ und ein Widerstand $\mathrm{R_2}$.
    
    \begin{itemize}
        \item[a)] Leiten Sie die allgemeine Formel für den Spannungsteiler her.
        \item[b)] Berechnen Sie $\mathrm{R_1}$ für $\mathrm{U_E} = 5\,\text{V}$ und $\mathrm{R_2} = 2\,\text{k}\Omega$.
    \end{itemize}
}


\Loesung{

}