%Spannungsverlauf eines realen Schalters
\begin{tikzpicture}[x=1cm,y=1cm]
	\begin{axis}[
		name=AxisSpannung,
		%%%%%%% START SAME OPTIONS %%%%%%%
		xmin=-1, xmax=5.5,
		ymin=-0.125, ymax=1.75,
		xlabel={$t$},
		ylabel={$\ecv{u(t)}$},
		axis x line=center,
		axis y line=center,
		ylabel style={anchor=east, at={(ticklabel cs:0.95)}},
		ticks=none,
		mark=none,
		grid=none,
		width=7cm,
		height=3.5cm,
		%%%%%%% END OF SAME OPTIONS %%%%%%%
	]
        %Spannungen
		\addplot[color= voltage, very thick, domain= -1:2.022]{0.05};
        \addplot[color= voltage, very thick] coordinates {(2,0.05)(2.3,0.97)};
        \addplot[color= voltage, very thick, domain= 2.28:4.02]{0.954};
        \addplot[color= voltage, very thick] coordinates {(4,0.97)(4.3,0.05)};
        \addplot[color= voltage, very thick, domain= 4.278:5.4]{0.05};	
        %Asymptoten
        \node at (0,1.016)[black,anchor=east,yshift=-1pt]{$\ecv{U_q}$};
        \addplot[color= black, dashed, domain= 0:5.4]{1.018};
        
    \end{axis}
\end{tikzpicture}