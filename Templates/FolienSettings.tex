% Pakete nur für die Folien
\usepackage{../Templates/beamerthemeGetItDigital}           %äquivalent zu \usetheme{GetItDigital}, nur dass der Ordner Templates benutzt wird.
\usepackage{environ}				                        % Erstellen von eigenen Environments
\usepackage[beamer]{hf-tikz}		                        % Highlighting Text/Math, Ref: https://tex.stackexchange.com/questions/52598/beamer-highlighting-aligned-math-with-overlay
% example1: \tikzmarkin<2->[above offset={\hfoffuppera},below offset={\hfofflowera},blend mode=multiply]{a} R = \frac{U}{I} \tikzmarkend{a}
% example2: \tikzmarkin<2-3>[above left offset={0,\hfoffuppera}, below right offset={0,\hfofflowera}]{markerid} \int^{\pi}_{0}f(x)\d x \tikzmarkend{markerid}
% option: blend mode=multiply (to not overshadow stuff behind the highlight box) and (in beamer mode) <n>, <n->, <n-m> for overlay (ignored in article)

% keine Navigationssymbole:
\setbeamertemplate{navigation symbols}{}



% eigene Befehle
\renewcommand{\d}{\mathrm{d}}	%für das Differential d/dt, überschreibt \d für "Punkt unter"
\newcommand{\s}[1]{} % skript only
\renewcommand{\b}[1]{#1} % beamer only
\newcommand{\ftx}[1]{\frametitle{#1}}
\newcommand{\fta}[1]{\frametitle{#1}\only<1>{\section{#1}}}
\newcommand{\ftb}[1]{\frametitle{#1}\only<1>{\subsection{#1}}}
\newcommand{\ftc}[1]{\frametitle{#1}}

\newcommand{\f}[4]{ \begin{center} \includegraphics[#2]{#3}\end{center}}
\newcommand{\fu}[2]{ \begin{center} #1 \end{center}}	%z.B. für die Circuitikz Umgebung
\newcommand{\fo}[5]{\begin{figure}[H]\centering\begin{overpic}[#2]{#3}{#4}\end{overpic}\end{figure}}
\newcommand{\foo}[5]{\begin{figure}[H]\begin{overpic}[#2]{#3}{#4}\end{overpic}\end{figure}}
\newcommand{\todo}[1]{}
\newcommand{\missingfigure}[1]{}
\newcommand{\resizeboxb}[3]{\resizebox{#1}{#2}{#3}} % resizebox but only apply in beamer (width, height, content)
\newcommand{\resizeboxsb}[3]{\resizebox{#2\textwidth}{!}{#3}} % e.g. \resizeboxx{0.8}{1}{stuff} % #1: width skript, #2: width beamer, #3: content
\newenvironment{eq}{\begin{equation*}}{\end{equation*}}
\NewEnviron{eqa}{%
	\begin{align*}
		\BODY
	\end{align*}
}
\newenvironment{bsp}[2][\ ]{}{}

%Highlighting
\newcommandx*\highlight[5][3=\hfofflower, 4=\hfoffupper,5=2]{\tikzmarkin<#5>[blend mode=multiply]{#1}(0,#3)(0,#4)#2\tikzmarkend{#1}}
% usage: \highlight{markerid}{markercontent}[offset below][offset upper][beamer overlay]
% Beispiele: (mit prädefinierten upper und lower offset values aus Settings.tex)
% $\highlight{id}{f(t)=\sin(\omega t)}$\quad
% $\highlight{ida}{\frac{1}{2}=\int f(t) \d t}[\hfofflowera][\hfoffuppera]$\quad
% $\highlight{idb}{\frac{\frac{\int^1_0 11}{\int^1_0 12}}{\frac{\int^1_0 21}{\int^1_0 22}}}[\hfofflowerb][\hfoffupperb][2-4]$
