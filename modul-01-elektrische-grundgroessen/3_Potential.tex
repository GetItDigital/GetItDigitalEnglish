\newvideofile{Potential}{Das elektrische Potential}

\section{Das elektrische Potential}



\begin{frame}
	\ftx{Lernziele: Das elektrische Potential}

	\begin{Lernziele}{Das elektrische Potential}
        Die Studierenden können
        \begin{itemize}
			\item Verschiebearbeit von Ladungen im elektrischen Feld berechnen
			\item Äquipotentialflächen bestimmen und einzeichnen
			\item elektrische Spannungen aus gegebenen Feldgrößen bestimmen
			
			
	%		\item die elektrische Spannung aus anderen Feldgrößen errechnen 
	%		\item Bestimmung des Zusammenhanges zwischen Ladung, elektrischer Flussdichte, Feldstärke und Permittivität im 
	%		Homogenfeld
        \end{itemize}
    \end{Lernziele}
	\speech{lernziele_potential}{1}{In diesem Video werden wir uns anschauen, was es heißt, Arbeit in einem elektrischen Feld zu verrichten, und werden daraufhin das elektrische Potential einführen.
	Ihr solltet also lernen, wie man die Verschiebearbeit von Ladungen im elektrischen Feld berechnet, wie man Äquipotentialflächen bestimmt, und wie sich letztendlich eine 
	elektrische Spannung aus gegebenen Feldgrößen ermitteln lässt.}
	\silence{3}


\end{frame}	



\subsection{Arbeit im elektrischen Feld}






\begin{frame}


	\s{

	Befindet sich eine Ladung innerhalb eines elektrischen Feldes, so wird eine Kraft 
	auf sie ausgeübt. Wird die Ladung nun innerhalb des Feldes bewegt, muss eine Arbeit verrichtet werden.
	Analog ist hier das Gravitationsfeld zu sehen, bei dem eine Arbeit verrichtet werden muss, wenn eine Masse angehoben wird,
	also die potentielle Energie des Objektes erhöht wird.
	Dieser Abschnitt ist als kurzer Vorgriff auf Modul 2 \textit{Energie und Leistung} zu sehen, in welchem
	 die elektrische Arbeit genau definiert wird.

	 
	Das Vorzeichen einer verrichteten Arbeit ist abhängig davon, ob die Arbeit auf das betrachtete System (hier: Potentielle Energie der Probeladung $Q$ im elektrischen Feld $\vec{E}$)
	oder auf den Erzeuger dieser Arbei bezogen wird. 
	In diesem Skript wird sich auf den Erzeuger bezogen. Muss dieser eine Arbeit verrichten, um die potentielle Energie des Systems zu erhöhen, ist die verrichtete Arbeit als negativ anzusehen.
	Wird eine Arbeit vom System freigesetzt, sprich die potentielle Energie der Prodeladung sinkt, ist die Arbeit positiv.



	Physikalisch gesehen ist die Arbeit das Produkt aus der Kraft, die benötigt wird, um die Ladung 
	zu bewegen, multipliziert mit der zurückgelegten Wegstrecke. Mit der Erkenntnis aus 
	Kapitel 4.1, dass die elektrische Feldstärke der Quotient aus der Kraft auf eine Ladung pro Ladung ist
	 (siehe Gleichung \ref{eq:e_feld}),
	lässt sich die zum Bewegen der Ladung benötigte Arbeit ermitteln.

	Nachfolgend soll berechnet werden, welche Arbeit $W$ bei der positiven Probeladung $Q$ aus Abbildung \ref{fig:arbeitefeld} freigesetzt wird,
	wenn sie sich entlang der Feldlinien in Richtung der negativ geladenen Leiterplatte bewegt.

	\begin{figure}[h!]
		\centering
		\includesvg[width=0.5\textwidth]{homogenfeld_w}
		\caption{Probeladung $Q$ in einem elektrischen Homogenfeld $\vec{E}$}
		\label{fig:arbeitefeld}
	\end{figure}


	
	Die Verlauf der elektrischen Feldstärke sowie Richtung von $\vec{E}$ gegenüber dem Weg $x$ lässt sich wie in Abbildung \ref{fig:evsx} gezeigt darstellen.



	\begin{figure}[h!]
		\centering
		\includesvg[width=0.5\textwidth]{homogenfeld_elektr_arbeit}
		\caption{Feldstärke $E(x)$ im betrachteten Aufbau aus Abbildung \ref{fig:arbeitefeld}}
		\label{fig:evsx}
	\end{figure}



		
\begin{equation*}
	W = \int_{d}^{0}  \vec{F} \, d \vec{x} 
\end{equation*}

Da die Richtung des elektrischen Feldes $\vec{E}$ der Raumrichtung $x$ 
entgegengesetzt ist, führt ein Eliminieren der Vektorpfeile zu einem negativen Vorzeichen vor dem Integral.
Durch Einsetzen der Kraft $F$ in Gleichung \ref{eq:e_feld} ergibt sich:

\begin{equation*}
	W = - Q \int_{d}^{0} E \cdot \mathrm{d}x 
\end{equation*}

Lösen des Integrals und Einsetzen der Integrationsgrenzen ergibt:

\begin{equation*}
	 W = -Q \cdot E \, [x]_d^0 = -Q \cdot E (0-d) 
\end{equation*} 

\begin{equation*}
	 W = Q \cdot E \cdot d
\end{equation*}

Die Differenz der zu- oder abgeführten Arbeit einer Ladung im elektrischen Feld beträgt also:

\begin{equation}
	\Delta W = Q \cdot E \cdot \Delta x 
\end{equation}

\begin{bsp}{}{}

	Für die in Abbildung \ref{fig:arbeitefeld} dargestellte Probeladung soll erreichnet werden, wie viel Energie auf ihrem Weg vom
    Startpunkt zur negativen Platte freigesetzt wird.\\
\phantom{text}\\

	Die Probeladung betrage $Q= 1 \mu \mathrm{C}$, der Abstand zur Platte betrage 0,1 m und das 
	elektrische Feld sei $E= 200$ V/m.

	Die Arbeit $W$ berechnet sich dann zu 

	\begin{equation*}
		W= Q \cdot E \cdot d = 1 \cdot 10^{-6} \mathrm{C} \cdot 200 \mathrm{V/m} \cdot 0,1 \mathrm{m} = 2 \cdot 10^{-5} \mathrm{J} 
	\end{equation*}
\end{bsp}

Analog zu einem Objekt im Gravitationsfeld führt das elektrische Feld einer Ladung Energie zu.
Dadurch wird diese beschleunigt, und die potentielle Energie wird in kinetische Energie
 umgewandelt.


 Es ist unmittelbar einsichtig, dass der Betrag der im elektrischen Feld
gespeicherten und nutzbaren potentiellen Energie der Probeladung
$Q$ vom Ort $x$ der Probeladung linear abhängig ist (siehe Abbildung \ref{fig:zusammenhangEA}). Verfügt die Probeladung
 bereits im Vorfeld über eine kinetische Energie, oder geht das elektrische Feld über den betrachteten Abschnitt hinaus,
 lässt sich die Gleichung um den zusätzlichen Term $W_0$ erweitern:


 \begin{equation}
	W(x) = Q \cdot E \cdot x + W_0
 \end{equation}



 \begin{figure}[h!]
	\centering
	\includesvg[width=0.5\textwidth]{homogenfeld_elektr_arbeit2}
	\caption{Linearer Zusammenhang zwischen dem Ort $x$ und der potentiellen Energie $W(x)$ der Probeladung.}
	\label{fig:zusammenhangEA}
\end{figure}


 

Der Übergang zwischen der im elektrischen Feld gespeicherten potentiellen Energie in die kinetische Energie
der Probeladung kann über das Schaubild in Abbildung \ref{fig:ubergang} dargestellt werden.

\begin{figure}[h!]
	\centering
	\includesvg[width=0.5\textwidth]{energieumwandlung}
	\caption{Übergang von potentieller Energie im elektrischen Feld zu potentieller Energie in der Probeladung}
	\label{fig:ubergang}
\end{figure}




 Zur allgemeinen Beschreibung muss die technisch nutzbare Feldenergie unabhängig von der Probeladung $Q$ sein.
Dazu wird die potentielle Energie auf ein Arbeitspotential normiert. 

 \begin{equation*}
	\frac{W(x)}{Q} =  E \cdot x + \frac{W_0}{Q}
 \end{equation*}

 

	}

	\ftx{Arbeit im elektrischen Feld}
	\b{
	\begin{columns}
       \column[c]{0.5\textwidth}
		%Bekannt: Elektrisches Feld fügt Kraft auf Ladungen aus.

		


		\textbf{Arbeit} zur Bewegung von Ladungen im E-Feld erforderlich \\

		\phantom{.}\\
		
		Arbeit $W$ $\widehat{=}$ Kraft $\cdot$ Weg\\

		[$W$] = J (Joule)

		
		\phantom{.}\\

		Vorzeichen betrachterabhängig, hier aus Erzeugersicht:

		Negative Arbeit $\rightarrow$ Erhöhung potentieller Energie im System




	%	\color{black} Frage: Welche Arbeit $W$ wird der Probeladung $Q$ auf ihrem Weg vom positiven Leiter
	%	 zum negativen Leiter zugeführt?

	\begin{columns}
		\column[c]{0.5\textwidth}



	\begin{figure}[h!]
		\centering
		\includesvg[width=\textwidth]{homogenfeld_w_folien}
	%	\caption{Probeladung $Q$ in einem elektrischen Homogenfeld $\vec{E}$}
	\end{figure}

	\column[c]{0.5\textwidth}



	\begin{figure}[h!]
		\centering
		\includesvg[width=\textwidth]{homogenfeld_w_graph_folien}
	%	\caption{Probeladung $Q$ in einem elektrischen Homogenfeld $\vec{E}$}
	\end{figure}


		
		\end{columns}		

		\pause


		\column[c]{0.5\textwidth}



		\begin{eqa}
			W = \int_{d}^{0}  \vec{F} \, d \vec{x} =  Q \int_{d}^{0} \vec{E} \cdot d\vec{x} \\
			\mathrm{mit} \, \vec{E} = \frac{\vec{F}}{Q}\\
			 W = -Q \cdot E \int_{d}^{0} dx = -Q \cdot E \, [x]_d^0 \\
			 W = -Q \cdot E (0-d) = Q \cdot E \cdot d
		\end{eqa}


		\begin{Merksatz}{}
			$\Delta W = Q \cdot E \cdot \Delta x$
		\end{Merksatz}

		%\begin{equation}
		%\begin{eqa}
		%	W = -Q \cdot E \int_{d}^{0} dx = -Q \cdot E \, [x]_d^0
		%\end{eqa}
			
	
		%W = \int_{d}^{0}\overrightarrow{F} \, d \overrightarrow{x} = - Q \int_{d}^{0} \overrightarrow{E} \cdot d\overrightarrow{x} 
		
		%\end{equation}

		%mit $\overrightarrow{E} = \frac{\overrightarrow{F}}{Q}$


	

	%W = -Q \cdot E \int_{d}^{0} dx = -Q \cdot E \, [x]_d^0

	

	%W = -Q \cdot E (0-d) = Q \cdot \overrightarrow{E} \cdot\overrightarrow{d}



	%\Rightarrow \Delta W = Q \cdot E \cdot \Delta x

		




    \end{columns}
		\speech{Arbeit_E_feld}{1}{Befindet sich eine Ladung innerhalb eines elektrischen Feldes, so wird eine Kraft 
	auf sie ausgeübt. Wenn wir diese Ladung nun entgegen dieser Kraft bewegen möchten, müssen wir dazu eine Arbeit verrichten.
	Analog ist hier das Gravitationsfeld zu sehen, bei dem eine Arbeit verrichtet werden muss, wenn eine Masse angehoben wird,
	also die potentielle Energie des Objektes erhöht wird.
	Das Vorzeichen einer verrichteten Arbeit ist abhängig davon, ob die Arbeit auf das betrachtete System selber, welches hier die potentielle Energie der Probeladung Q im
	elektrischen Feld oder auf den Verrichter dieser Arbeit bezogen wird. 
	Wir wollen uns hier auf den Verrichter beziehen, sprich wenn wir eine Arbeit verrichten, um die potentielle Energie des Systems zu erhöhen, ist die verrichtete 
	Arbeit als negativ anzusehen.
	Wird eine Arbeit vom System freigesetzt, sprich die potentielle Energie der Probeladung sinkt, ist die Arbeit positiv}
	\silence{2}

	\speech{Arbeit_E_feld}{2}{Physikalisch gesehen ist die Arbeit das Produkt aus der Kraft, die benötigt wird, um die Ladung 
	zu bewegen, multipliziert mit der zurückgelegten Wegstrecke. Auf dieser Folie ist dieser Zusammenhang allgemeingültig mit dem Integral der Kraft über die Wegstrecke 
	ausgedrückt. Mit dem Wissen, dass die elektrische Feldstärke der Quotient aus der Kraft auf eine Ladung pro Ladung selbst ist,
	lässt sich die zum Bewegen der Ladung benötigte Arbeit ermitteln.}	
	\silence{3}

	}

\end{frame}




\begin{frame}
	\b{
	\ftx{Arbeit im elektrischen Feld II}
	\begin{columns}
      
		%	\phantom{.}\\
		

%			\begin{eq}	
%				[W] = \rm{J} \, (Joule)
%			\end{eq}

		

		\column[c]{0.5\textwidth}
		\begin{columns}


			\column[c]{0.5\textwidth}
			\begin{figure}[h!]
				\centering
				\includesvg[width=\textwidth]{homogenfeld_w_folien}
				%\caption{Probeladung $Q$ in einem elektrischen Homogenfeld $\vec{E}$}
				%\label{fig:anziehende_ladungen}
			\end{figure}	
			%\f{width=\textwidth}{width=\textwidth}{arbeit_1.png}{Dummytext}
			\column[c]{0.5\textwidth}
			\begin{figure}[h!]
				\centering
				\includesvg[width=\textwidth]{homogenfeld_w_graph_folien}
			%	\caption{Probeladung $Q$ in einem elektrischen Homogenfeld $\vec{E}$}
			%	\label{fig:anziehende_ladungen}
			\end{figure}	
			
			
			
		\end{columns}
 \column[c]{0.5\textwidth}

			\visible<2->{

	   	\begin{figure}[h!]
				\centering
				\includesvg[width=0.75\textwidth]{energieumwandlung}
			%	\caption{Probeladung $Q$ in einem elektrischen Homogenfeld $\vec{E}$}
			%	\label{fig:anziehende_ladungen}
			\end{figure}	
			}

		\end{columns}
    

		\visible<3->{


	 \begin{eq}
		W(x) = Q \cdot E \cdot x + W_0
	 \end{eq}

	 


	}
	\visible<4->{

	\vspace{20pt}

	Problem: Abhängigkeit von Probeladung $Q$\\


	$\rightarrow$ \textbf{Normierung} von $W(x)$ auf $Q$ als Arbeitspotential



	\begin{eq}
		\frac{W(x)}{Q} = E \cdot x + \frac{W_0}{Q}
	 \end{eq}
	}


\speech{Arbeit_E_feld2}{1}{	 
Analog zu einem Objekt im Gravitationsfeld führt das elektrische Feld einer Ladung Energie zu.
Dadurch wird diese beschleunigt, und die potenzielle Energie wird in kinetische Energie
 umgewandelt. Je weiter oben sich die Probeladung also im Homogenfeld befindet, desto mehr potentielle Enerqie ist gespeichert. Das Elektrische Feld wird hier negativ eingezeichnet,
da die Feldlinien der hier gewählten Zählrichtung von X entgegen laufen.}
\silence{1}
\speech{Arbeit_E_feld2}{2}{	
 Den Übergang zwischen der im elektrischen Feld gespeicherten potenziellen Energie in die kinetische
Energie der Probeladung kann man sich wie in diesem Schaubild gezeigt vorstellen. Die potenzielle Energie der Ladung im elektrischen Feld wird 
in kinetische Energie der Probeladung umgewandelt.}
\silence{1}

\speech{Arbeit_E_feld2}{3}{	 
 Verfügt die Probeladung bereits im Vorfeld über eine kinetische Energie, oder geht das elektrische Feld über den betrachteten Abschnitt hinaus,
 lässt sich die Gleichung um den zusätzlichen Term W Null erweitern.}
 \silence{1}
\speech{Arbeit_E_feld2}{4}{	 
 Möchte man die technisch nutzbare Feldenergie jedoch allgemein beschreiben, muss diese Größe unabhängig von unserer Probeladung Q sein.
Dazu wird die potenzielle Energie durch die Ladung Q geteilt, und auf ein allgemeingültiges Arbeitspotenzial normiert.}
\silence{3}

}
\end{frame}

\subsection{Definition des elektrischen Potentials}
\begin{frame}
	\ftx{Definition des elektrischen Potentials}

	\s{

	Die potentielle Energie normiert auf die betrachtete Ladung Q ist das \textbf{elektrische Potential} $\varphi$
	des elektrischen Feldes. Dieses ist für jeden Punkt im Raum unabhängig von weiteren Einflussfaktoren direkt bestimmbar.
	 Das Potentialfeld ist im Gegensatz zum elektrischen Feld ein Skalarfeld. Das bedeutet, dass jedem Punkt im
	Raum unabhängig von weiteren Einflussfaktoren ein eindeutiges elektrisches Potential zugeordnet werden kann, 
	jedoch keine Richtung dieser Größe im Raum vorliegt.

	 \begin{equation}
		\varphi = \frac{W}{Q}
	 \end{equation}
	 \begin{equation*}
		[\varphi] = \text{V} \ (\text{Volt})
	 \end{equation*}

	

	 Abbildung \ref{fig:potihomogen} zeigt ein elektrisches Homogenfeld, in welches die Potentiallinie eingezeichnet ist, auf welcher sich die
	Ladung $Q$ gerade befindet. Mit zunehmender Entfernung von der positiven Platte in $x$-Richtung steigt das Potential 
	 an, und kann wie folgt berechnet werden:

	 \begin{equation}
		\varphi(x) = E \cdot x + \varphi_0
	\label{eq:potential}
	 \end{equation}


	 \begin{figure}[h!]
		\centering
		\includesvg[width=0.5\textwidth]{potential_homogenfeld}
		\caption{Elektrisches Homogenfeld mit eingezeichneter Potentiallinie $\varphi(x)$}
		\label{fig:potihomogen}
	\end{figure}

	
	 Die Steigung des Potentials  ist durch die elektrische Feldstärke $\vec{E}$ bestimmt. 
	 Das elektrische Potential $\varphi$ und die elektrische Feldstärke hängen durch den negativen Gradienten des Potentials 
	 zusammen. Dieser Zusammenhang wird in Abbildung \ref{fig:potigraph} verdeutlicht. 
	



 

	 \begin{figure}[h!]
		\centering
		\includesvg[width=0.5\textwidth]{homogenfeld_phi_graph}
		\caption{Zusammenhang zwischen dem elektrischen Potential $\varphi$ und der elektrischen Feldstärke $E$ im Homogenfeld}
		\label{fig:potigraph}
	\end{figure}
	

	 Die Beziehung kann im Homogenfeld durch die folgende Gleichung ausgedrückt werden:

	 \begin{equation}
		E = - \frac{\Delta \varphi}{\Delta x} 
	 \end{equation}

	 In inhomogenen, dreidimensionalen Feldern gilt die allgemeine Gleichung, bei der der Gradient von $\varphi$ in jede Raumrichtung errechnet wird: 

	 \begin{equation}
		E = - \mathrm{grad} \,\varphi (x,y,z)  = \left( \frac{\partial \varphi}{\partial x}, \frac{\partial \varphi}{\partial y}, \frac{\partial \varphi}{\partial z} \right)
	 \end{equation}


	 Um das elektrische Potential an einem Punkt vollständig zu beschreiben, muss wie in Gleichung \ref{eq:potential} 
	 aufgeführt als notwendige Randbedingung ein Bezugspotential $\varphi_0$ gewählt werden, welches eindeutig definiert ist. 
	 Prinzipiell ist dieses Potential beliebig wählbar, in der Regel wird jedoch eine Bezugselektrode gewählt, für die das Potential
	 gleich null gesetzt wird. Im Fall des in Abbildung \ref{fig:potihomogen} gezeigten Aufbaus wäre dies:

	 

	 \begin{equation*}
		\varphi_0 = \varphi(x=0) = 0
	 \end{equation*}

	 




	}
	\b{
	\begin{columns}
       \column[c]{0.5\textwidth}
	  % \vspace{-20pt}
	 %  Die potentielle Energie normiert auf die betrachtete Ladung $Q$ ist das
	  % Potential $\varphi$ des elektrischen Feldes.
	  \hspace{-2pt}Normierung der potentiellen Energie:
	   \begin{equation*}
		\mathrm{Potential} \, \varphi = \frac{W}{Q}
	   \end{equation*}
	   \vspace{-10pt} \begin{equation*}
		[\varphi] = \mathrm{Volt} = \mathrm{V} 
	   \end{equation*}

	


	   Im elektrischen Homogenfeld: 

	   \begin{figure}[h!]
		\centering
		\includesvg[width=0.8\textwidth]{potential_homogenfeld_folien}
	
	\end{figure}

	

		
		\begin{equation*}
			\varphi(x) = E \cdot x + \varphi_0
		\end{equation*}
	

		\column[c]{0.5\textwidth}

		\pause

		\begin{figure}[h!]
			\centering
			\includesvg[width=\textwidth]{homogenfeld_phi_graph_folien}
	
		\end{figure}



		\begin{equation*}
			E = - \frac{\Delta \varphi}{\Delta x}
		 \end{equation*}


		
	\vspace{5pt}
		Bezugspunkt notwendig:
		\vspace{-5pt}
		\begin{equation*}
			\varphi_0 = \varphi(x=0) =0 
		\end{equation*}
	\end{columns}	

	\speech{Def_Potential}{1}{	 
	Diese auf die betrachtete Ladung Q normierte potentielle Energie wird also als das Potential Vieh
	des elektrischen Feldes bezeichnet. Da eine Ladung an jedem Punkt im Raum eine im Vergleich zum Bezugspunkt positive, negative oder identische potentielle Energie hat, 
	ist auch das Potential an jedem Punkt unabhängig von weiteren Einflussfaktoren direkt bestimmbar.
	Das Potentialfeld ist im Gegensatz zum elektrischen Feld ein Skalarfeld. Das bedeutet, dass jedem Punkt im
	Raum ein eindeutiges elektrisches Potential zugeordnet werden kann, dieses Potential jedoch keine Richtung hat, in die es wirkt.}
	\silence{2}
	\speech{Def_Potential}{2}{Hier zu sehen ist ein elektrisches Homogenfeld, in welches die Potenziallinie eingezeichnet ist, auf welcher sich die
	Ladung Q gerade befindet. Mit zunehmender Entfernung von der positiven Platte in X Richtung steigt das Potential weiter an.
	Die Steigung des Potenzials ist durch die elektrische Feldstärke E bestimmt, wie diese Grafik hier verdeutlicht.
	Wie Eingangs erwähnt, muss zur Bestimmung des elektrischen Potenzials im Raum wie auch des Potentials des Gravitationsfeldes ein Bezugspotential festgelegt werden.
	Während beim Bezugspotential häufig die Erdoberfläche verwendet wird, nutzen wir hier die negativ geladene Platte, welche bei X
	gleich Null liegt, und setzen das Potential dort auf Null Volt.}
	\silence{3}

	}
\end{frame}			
		

		





\begin{frame}
	\ftb{Zusammenhang zwischen Arbeit und Potential}
	
	\s{Im homogenen elektrischen Feld ist das Potential $\varphi$ wie Formel \ref{eq:potential} zeigt
	ausschließlich von der Entfernung $x$ zur Bezugselektrode abhängig.
	Dies bedeutet, dass alle Flächen, die senkrecht zum Feld stehen, ein konstantes Potential haben. 
	Solche Flächen werden als Äquipotentialflächen bezeichnet, wie sie in Abbildung \ref{fig:potentialdifferenz} mit Hilfe der grünen Linien dargestellt sind.
	Es muss keine Arbeit verrichtet werden, um eine Ladung entlang einer solchen Fläche zu bewegen.


	\begin{figure}[h!]
		\centering
		\includesvg[width=0.5\textwidth]{potentialdifferenz}
		\caption{Elektrisches Homogenfeld, in dem sich eine Probeladung $Q$ von der Äquipotentiallinie $\varphi(x_1)$ auf 
	die Äquipotentiallinie $\varphi(x_2)$ bewegt}
		\label{fig:potentialdifferenz}
	\end{figure}


	Die potentielle Energie $W_{12}$ einer Probeladung $Q$, die durch das elektrische Feld von einem Punkt $x_1$ nach $x_2$
	bewegt wird, kann durch die Potentialdifferenz zwischen diesen Punkten beschrieben werden. Die Reihenfolge der Indizes der
	 Arbeit $W_{12}$ beschreibt dabei die Bewegungsrichtung.

	Wird eine Probeladung wie in Abbildung \ref{fig:potentialdifferenz} gezeigt von der Äquipotentialfläche $\varphi(x_1)$ nach $\varphi(x_2)$ bewegt,
	nimmt sie dabei die folgende Energie auf:

	\begin{equation*}
		W_{12} = Q \cdot (\varphi_1 - \varphi_2)
	\end{equation*}




	
	
	\begin{figure}[h!]
		\centering
		\includesvg[width=0.5\textwidth]{phi_diff_graph}
		\caption{Darstellung des Potentialverlaufs $\varphi(x)$ innerhalb des elektrischen Homogenfeldes}
		\label{fig:potentialgraph}
	\end{figure}





	Bei der in Abbildung \ref{fig:potentialgraph} getroffenen Wahl der Bezugsfläche von $\varphi_2 = 0$ ergibt sich für die zugeführte Arbeit 
	das Ergebnis:

	\begin{equation*}
		W_{12} = Q \cdot (\varphi_1 - 0) = Q \cdot \varphi_1
	\end{equation*}


	Mit Hilfe des nachfolgenden Gedankenexperimentes lässt sich eine weitere zentrale Eigenschaft des elektrischen Feldes aufzeigen.

	Wie in Abbildung \ref{fig:wirbelfrei} gezeigt wird eine Probeladung $Q$ in einem Homogenfeld 
	zwischen den Punkten 1-4 in einem geschlossenem Umlauf bewegt. Die bei diesem Vorgang zu verrichtenden Teilenergien werden betrachtet.
	So wird bei einer Bewegung vom Punkt 1 zu Punkt 2 eine Arbeit von $W_{12} = Q \cdot (\varphi_1 - \varphi_2)$ verrichtet, 
	und von Punkt 4 zu Punkt 1 die Energie $W_{41} = Q \cdot (\varphi_4 - \varphi_1)$.


	\begin{figure}[h!]
		\centering
		\includesvg[width=0.5\textwidth]{wirbelfrei}
		\caption{Darstellung des Potentialverlaufs $\varphi(x)$ innerhalb des elektrischen Homogenfeldes}
		\label{fig:wirbelfrei}
	\end{figure}



	Die für einen gesamten Umlauf benötigte Energie ergibt sich durch die Aufsummierung der vier Teilenergien: 

	
	\begin{equation*}
		\begin{split}
			W_\mathrm{ges} = Q \cdot (\varphi_1 - \varphi_2) + Q \cdot (\varphi_2 - \varphi_3)
			+ Q \cdot (\varphi_3 - \varphi_4) + Q \cdot (\varphi_4 - \varphi_1)
		\end{split}
	\end{equation*}

	Da die Ladung $Q$ in allen Termen identisch ist, ergibt sich durch das Ausklammern von $Q$:


	\begin{equation*}
			W_\mathrm{ges} = Q \cdot (\varphi_1 - \varphi_2 + \varphi_2 - \varphi_3
			+ \varphi_3 - \varphi_4 + \varphi_4 - \varphi_1) 
	\end{equation*}
	
	\begin{equation*}
		\rightarrow W_\mathrm{ges} = Q \cdot 0 = 0
	\end{equation*}

	Für einen geschlossenen Umlauf einer Probeladung im Homogenfeld ist also keine Energie erforderlich,
	 da sie ausschließlich vom Anfangs- und Endpunkt der Bewegung, nicht jedoch vom gewählten Weg abhängig ist. 

	Folglich wird beim elektrischen Feld auch von einem \textbf{wirbelfreien Quellenfeld} gesprochen.
	Dies bedeutet, dass die Feldlinien keine geschlossenen Linien bilden, sondern jeweils einen Anfang (auf positiven Ladungen)
	und einen Endpunkt (auf negativen Ladungen) haben.
	
	Diese Betrachtung gilt nicht nur für das Homogenfeld, sondern trifft auf jedes elektrostatische Feld zu.



	
	


	\begin{bsp}{Arbeit im Radialfeld}{}
		In folgendem Beispiel soll die Probeladung $Q_1$ vom Abstand $r_1$ auf den Abstand $r_2$ an die Feste Ladung Q herangeführt werden.

		\begin{center}
		\includesvg[width=0.5\textwidth]{radialfeld}	
		\end{center}

		
		
		

		$Q$ betrage 1 C, $Q_1$ 1 mC. Welche Arbeit ist erforderlich, wenn $r_1 = 1$ m und $r_2 = 50$ cm seien?

		\[
W_{12} = Q_1 \int_{r_1}^{r_2} \vec{E} \cdot d\vec{r}
\]
\[
W_{12} = Q_1 \int_{r_1}^{r_2} \left( \frac{Q}{4 \pi \varepsilon_0 r^2} \right) dr
\]
\[
W_{12} = Q_1 \cdot \frac{Q}{4 \pi \varepsilon_0} \int_{r_1}^{r_2} \frac{1}{r^2} dr
\]
Die Stammfunktion von $\frac{1}{r^2}$ beträgt $-\frac{1}{r}$.
\[
W_{12} = Q_1 \cdot \frac{Q}{4 \pi \varepsilon_0} \left( -\frac{1}{r} \bigg|_{r_1}^{r_2} \right) = Q_1 \cdot \frac{Q}{4 \pi \varepsilon_0} \left( -\frac{1}{r_2} + \frac{1}{r_1} \right)
\]
wobei $\frac{Q}{4 \pi \varepsilon_0} \frac{1}{r}$ dem Potential $\varphi(r)$ entspricht.

\[
W_{12} = Q_1 \cdot (\varphi(r_1) - \varphi(r_2))
\]

Einsetzen der gegebenen Werte:
\[
W_{12} = (1 \cdot 10^{-3} \, \text{C}) \cdot \frac{1 \, \text{C}}{4 \pi \varepsilon_0} \left( \frac{1}{1 \, \text{m}} - \frac{1}{0{,}5 \, \text{m}} \right)
\]
\[
W_{12} = -8,98755 \times 10^{6} \, \text{J}
\]
Folglich müssen etwa 8,99 Megajoule aufgewendet werden, um die Ladung $Q_1$ zu verschieben. 
	\end{bsp}	


	}

	\b{

	\begin{columns}
      \column[c]{0.4\textwidth}
		Potential nur abhängig von Ortskoordinate $x$ in Feldrichtung.

		\phantom{.}\\

		Flächen gleichen Potentials: $\rightarrow$ \textbf{Äquipotentialflächen}

		\phantom{text}\\
		\visible<2->{


		\begin{equation*}
			W_{12} = Q \cdot (\varphi_1 - \varphi_2)
		\end{equation*}
		
		\phantom{text}\\


		Mit Bezugspotential $\varphi_2 = 0$:

		\begin{equation*}
			W_{12} = Q \cdot (\varphi_1 - 0) = Q \cdot \varphi_1
		\end{equation*}
	

		}


		\column[c]{0.6\textwidth}

		\begin{figure}[h!]
			\centering
			\includesvg[width=\textwidth]{potentialdifferenz_folien}
		\end{figure}

		\vspace{10pt}

		\visible<2->{
		\begin{figure}[h!]
			\centering
			\includesvg[width=0.8\textwidth]{phi_diff_graph_folien}
		\end{figure}
		}




		

	\end{columns}

	\speech{zusammenhang_Arbeit_Poti}{1}{Im homogenen elektrischen Feld ist das Potential Vieh ausschließlich von der Entfernung zur Bezugselektrode abhängig.
	Dies bedeutet, dass alle Flächen, die senkrecht zum Feld stehen, ein konstantes Potential haben. 
	Solche Flächen werden als Äquipotentialflächen bezeichnet, und sind hier beispielhaft mit Hilfe der grünen Linien dargestellt.
	Es muss keine Arbeit verrichtet werden, um eine Ladung entlang einer solchen Fläche zu bewegen.}
	\silence{1}

	\speech{zusammenhang_Arbeit_Poti}{2}{Möchte man hingegen die Probeladung von der Äquipotentiallinie Vieh von X eins auf die Äquipotentiallinie Vieh von X Zwei bewegen, 
	muss eine entsprechende Arbeit verrichtet werden,die durch die Potentialdifferenz zwischen diesen Punkten beschrieben werden kann. Die Reihenfolge der Indizes der
	Arbeit beschreibt dabei die Bewegungsrichtung. Ist wie in diesem Beispiel das Potential auf X zwei das Bezugspotential, erhalten wir für die Arbeit von X eins nach X zwei 
	das Ergebnis W gleich Q mal Vieh eins. 
	 Bei einem positiven Potential ist dieser Ausdruck also auch positiv. Das heißt, dass keine Energie zugefügt werden muss, sondern das System potentielle Energie
	freigibt und in Bewegungsenergie umwandelt.}
	\silence{3}
	}
\end{frame}





\begin{frame}

	\b{
	\ftx{Wirbelfreies Feld}


		\begin{columns}
			\column[c]{0.6\textwidth}

			\vspace{-10pt}
			
			Zunächst Betrachtung der Teilenergien bei einem geschlossenen Umlauf
			 im Homogenfeld.

			 \vspace*{5pt}
			 \visible<2->{

			Die gesamte Energie ergibt sich durch
 		die Aufsummierung der Teilenergien:
			 }

		 \visible<3->{

			\begin{equation*}
				\begin{split}
					W_\mathrm{ges} = Q \cdot (\varphi_1 - \varphi_2) + Q \cdot (\varphi_2 - \varphi_3)\\
					+ Q \cdot (\varphi_3 - \varphi_4) + Q \cdot (\varphi_4 - \varphi_1)
				\end{split}
			\end{equation*}
			\begin{equation*}
					W_\mathrm{ges} = Q \cdot (\varphi_1 - \varphi_2 + \varphi_2 - \varphi_3
					+ \varphi_3 - \varphi_4 + \varphi_4 - \varphi_1) 
			\end{equation*}
			\vspace{-10pt}
			\begin{equation*}
				W_\mathrm{ges} = Q \cdot 0 = 0
			\end{equation*}
		}


		 \vspace*{5pt}

		 \visible<4->{
		
		Energie im elektrischen Feld ist nur von Anfangs- 
		und Endpunkt und nicht vom Weg abhängig. Es ist ein „wirbelfreies“ Feld.
		 }
			
		
			\column[c]{0.4\textwidth}
				\includesvg[width=1\textwidth]{wirbelfrei}

				\vspace{30pt}
				\visible<4->{
					\underline{Kennzeichen}:
					\begin{itemize}
						\item keine geschlossenen Feldlinien
						\item Feldlinien beginnen und enden auf Ladungen
					\end{itemize}
				}
			
    \end{columns}
	\speech{wirbelfrei}{1}{	
		Mit Hilfe eines Gedankenexperimentes können wir eine weitere zentrale Eigenschaft des
elektrischen Feldes aufzeigen.
Wir betrachten eine Probeladung Kuh, die sich in einem Homogenfeld zwischen den Punkten
eins bis vier in einem geschlossenen Umlauf bewegt.}

\speech{wirbelfrei}{3}{	
Dabei schauen wir uns die bei diesem Vorgang zu verrichtenden Teilenergien genauer an. So wird bei einer Bewegung vom Punkt 1 zu Punkt 2 eine Arbeit von Weh eins zwei gleich Q mal Vieh eins minus Vieh zwei verrichtet.
 und von Punkt 4 zu Punkt 1 die Arbeit W vier eins gleich Q mal Phi vier minus phi eins.
Klammern wir aus der Energiesumme für den gesamten Umlauf die Ladung Q nun aus und fassen alle Potentiale zusammen, stellen wir fest, dass sich alle Potenziale letztendlich herauskürzen.}
\silence{1}
 \speech{wirbelfrei}{4}{Für einen geschlossenen Umlauf einer Probeladung im Homogenfeld ist also keine Energie erforderlich, 
da sie ausschließlich vom Anfangs- und Endpunkt der Bewegung, nicht jedoch vom gewählten Weg abhängig ist.
Folglich wird beim elektrischen Feld auch von einem wirbelfreien Quellenfeld gesprochen. Dies
bedeutet, dass die Feldlinien keine geschlossenen Linien bilden, sondern jeweils einen Anfang, welcher auf positiven Ladungen ist, und einen Endpunkt auf negativen Ladungen haben.
Diese Betrachtung gilt nicht nur für das Homogenfeld, sondern ist für jedes beliebige elektrostatische Feld gültig.}
\silence{3}
	}
\end{frame}	


\begin{frame}
	\ftb{Definition der elektrischen Spannung}

	\s{

	Die aus dem elektrischen Feld für einen technischen Prozess nutzbare
	Energie ist durch die  \textbf{Potentialdifferenz} zwischen Anfangs- und
	Endpunkt des Weges gegeben.  Durch ihre fundamentale Bedeutung in der Elektrotechnik wird sie mit einem
	Namen und Formelbuchstaben bedacht, und wird als \textbf{elektrische Spannung} $U$ bezeichnet: 

	\begin{equation}
		U_{12} = \varphi_1 - \varphi_2
		\label{eq:spannung}
	\end{equation}

	\begin{equation*}
		[U] = \text{V} \ (\text{Volt})
	 \end{equation*}

	 Durch die Reihenfolge der Indizes wird - analog zur Potentialdifferenz - die Zählrichtung des Spannungspfeiles angegeben.
	 In Abbildung \ref{fig:defispannung} beschreibt die Spannung $U_{12}$ also die Potentialdifferenz zwischen den Potientialen $\varphi_1$ und $\varphi_2$.
	 Durch die Eigenschaften der Äquipotentialflächen ist es hierbei vollkommen unerheblich, von welchem Punkt von $\varphi_1$ gestartet wird
	 und welcher Punkt auf $\varphi_2$ das Ziel darstellt. Auch der Pfad, welcher zwischen den Punkten gewählt wird, ist für die Spannung zwischen
	 den Punkten unerheblich.
	 

	 \begin{figure}[h!]
		\centering
		\includesvg[width=0.3\textwidth]{spannung}
		\caption{Spannung $U_{12}$ zwischen den beiden Äquipotentiallinien $\varphi_1$ und $\varphi_2$}
		\label{fig:defispannung}
	\end{figure}

	 

	 Der Zusammenhang zwischen der elektrischen Feldstärke und dem Potential ist bereits aus Gleichung \ref{eq:potential} 
	 bekannt und wird (bei einem Bezugspotential von $\varphi_0 = 0)$ mit $\varphi (x) = E \cdot x$ angegeben.
	 
	 Eingesetzt in die Definition der Spannung (Gleichung \ref{eq:spannung}) ergibt sich:

	 \begin{equation*}
		 U = \varphi(x= d) - \varphi(x= 0) = E \cdot d - E \cdot 0
	 \end{equation*}

	Die elektrische Spannung $U$ kann in einem wie in Abbildung \ref{fig:spannunghomogen} dargestellten Homogenfeld wie folgt berechnet werden,
	  sofern die Wegstrecke $d$ parallel zu den Feldlinien liegt:

	 
	  \begin{figure}[h!]
		\centering
		\includesvg[width=0.4\textwidth]{spannung_homogenfeld}
		\caption{Elektrische Spannung $U$ im Homogenfeld}
		\label{fig:spannunghomogen}
	\end{figure}

	 



	\begin{equation}
	 U = E \cdot d
	\end{equation}

	
	Alternativ kann aus einer gegebenen Spannung $U$ bei bekannter Wegstrecke $d$ auch die elektrische Feldstärke berechnet werden:
	
	\begin{equation}
		E = \frac{U}{d}
	   \end{equation}

	In Abbildung \ref{fig:spannunggraph} lässt sich erkennen, dass die Spannung nichts anderes als die über die Wegstrecke $d$ aufintegrierte Feldstärke $E$ ist.
	

	\begin{figure}[h!]
		\centering
		\includesvg[width=0.4\textwidth]{homogenfeld_spannung_graph}
		\caption{Zusammenhang zwischen elektrischer Spannung $U$ und elektrischem Homogenfeld $E$}
		\label{fig:spannunggraph}
	\end{figure}



	Während dieses Integral im Homogenfeld durch eine Multiplikation von elektrischem Feld $E$ und Wegstrecke $d$ ermittelt
	werden kann, ist dies bei inhomogenen Feldern nur durch das Lösen des Integrals möglich:
	
	\begin{equation}
		U_{12} =\int_{1}^{2} \vec{E} \cdot d \vec{s}
	\end{equation}

	Zu beachten ist hierbei, dass es sich bei dem Produkt $\vec{E} \cdot \mathrm{d} \vec{s}$ um Vektoren handelt. Zeigen sie in die
	gleiche Richtung, dürfen die Vektorpfeile eliminiert werden, andernfalls muss das Skalarprodukt verwendet werden: 


		\begin{equation*}
			\vec{E} \cdot \, \mathrm{d} \vec{s} =  E \cdot \mathrm{d} s \cdot \mathrm{cos} (\alpha)
		\end{equation*}


		\begin{figure}[h!]
			\centering
			\includesvg[width=0.5\textwidth]{skalarprodukt}
			\caption{Skalarprodukt aus Feldstärke $\vec{E}$ und Wegstück d$\vec{s}$}
			\label{fig:skalarprodukt}
		\end{figure}
	




	










	}

	\newpage

	\b{
	\begin{columns}
      \column[c]{0.65\textwidth}
		\textbf{Potentialdifferenz} bestimmt nutzbare Energie\\
		$\rightarrow$  \textbf{Spannung} $U$: 



		\begin{equation*}
			U_{12} = (\varphi_1 - \varphi_2)
		\end{equation*}

		\begin{equation*}
			[U] = \mathrm{Volt} = \mathrm{V} \, 
		\end{equation*}
		
		\phantom{text}\\

		Indexreihenfolge: Start- und Zielpunkt


		\column[c]{0.35\textwidth}


		\begin{figure}[h!]
			\centering
			\includesvg[width=\textwidth]{spannung}
		\end{figure}


		\begin{figure}[h!]
			\centering
			\includesvg[width=\textwidth]{spannung_homogenfeld}
		\end{figure}
	
	




	\end{columns}
	\speech{def_spannung}{1}{
	Wie wir gesehen haben ist die aus dem elektrischen Feld für einen technischen Prozess nutzbare
	Energie nur durch die Potentialdifferenz zwischen Anfangs und
	Endpunkt des Weges beschrieben. Durch ihre fundamentale Bedeutung in der Elektrotechnik wird sie deshalb mit einem
	eigenen Namen und Formelbuchstaben bedacht, und wird als elektrische Spannung U bezeichnet: 
	Wie bei der Potentialdifferenz auch wird die Zählrichtung des Spannungspfeils durch die Reihenfolge der Indizes angegeben.
	Hier beschreibt die Spannung U eins zwei also die Potentialdifferenz zwischen den Potientialen Vieh eins und Vieh zwei.}
	\silence{1}
	\speech{def_spannung}{1}{
	Durch die Eigenschaften der Äquipotentialflächen ist es hierbei vollkommen unerheblich, von welchem Punkt von Vieh eins gestartet wird
	und welcher Punkt auf Vieh zwei das Ziel darstellt. Auch der Pfad, welcher zwischen den Punkten gewählt wird, ist für die Spannung zwischen
	den Punkten unerheblich.}
	\silence{3}


	 


	}
\end{frame}


\begin{frame}

	\b{
	\ftx{Zusammenhang zwischen Spannung und Feldstärke I}
	

	
	\begin{columns}
      \column[c]{0.6\textwidth}
		\vspace{-10pt}
	  	Im \textbf{Homogenfeld:}

		\begin{equation*}
			\varphi(x) = E \cdot x
		\end{equation*}

		

		\begin{equation*}
			U = \varphi(x= d) - \varphi(x= 0) 
		\end{equation*}

		\begin{equation*}
			U =  E \cdot d - E \cdot 0
		\end{equation*}
   

	%	aus der elektrischen Feldstärke und dem Abstand von Anfangs- und Endpunkt des betrachteten Weges

	\phantom{text}\\


		\begin{Merksatz}{}
			\begin{equation*}
				U = E \cdot d
			\end{equation*}

			\begin{equation*}
				E =  \frac{U}{d}
			\end{equation*}


		\end{Merksatz}
		



		\column[c]{0.4\textwidth}


		\begin{figure}[h!]
			\centering
			\includesvg[width=\textwidth]{spannung_homogenfeld}
		\end{figure}

		\begin{figure}[h!]
			\centering
			\includesvg[width=\textwidth]{homogenfeld_spannung_graph_folie}

		\end{figure}
	\end{columns}
	



	\speech{zusammenhang_spannung_feld1}{1}{Der Zusammenhang zwischen der elektrischen Feldstärke und dem Potential ist bereits von den vorherigen Folien
	 bekannt und wird bei einem Bezugspotential von null für x gleich null mit Phi von x gleich E mal x angegeben.
	 In diesem Beispiel habe der Abstand der beiden Platten, zwischen denen wir die Spannung messen wollen, den Abstand D.
	 Eingesetzt in die Definition der Spannung  ergibt sich folglich U gleich E mal D minus E mal Null. }
	 \speech{zusammenhang_spannung_feld1}{1}{
	 Die Spannung in einem Homogenfeld kann also über E mal D beschrieben werden, umgeformt kann das elektrische Feld also auch über E gleich U durch D ermittelt werden.
	sofern die Wegstrecke D parallel zu den Feldlinien liegt.}
	\silence{3}
	}
\end{frame}


\begin{frame}

	\b{
	\ftx{Zusammenhang zwischen Spannung und Feldstärke II}
	
	\begin{columns}
      \column[c]{0.6\textwidth}

	  Spannung $\widehat{=}$ aufintegrierter Feldstärke: 


	  \begin{figure}[h!]
		\centering
		\includesvg[width=0.7\textwidth]{homogenfeld_spannung_graph_folie}
	\end{figure}
	


	  

 
	 Allgemeines Feld:
		
	  \begin{Merksatz}{}

		\begin{equation*}
			U_{12} = \int_{1}^{2} \vec{E} \, \mathrm{d}\vec{s}
		\end{equation*}

	\end{Merksatz}

		\column[c]{0.4\textwidth}


	Wiederholung Skalarprodukt:

	\begin{figure}[h!]
		\centering
		\includesvg[width=0.9\textwidth]{skalarprodukt}
		
	\end{figure}
	

		\begin{equation*}
			\vec{E} \cdot \, \mathrm{d} \vec{s} =  E \, \mathrm{d} s \cdot \mathrm{cos} (\alpha)
		\end{equation*}






	\end{columns}
	\speech{zusammenhang_spannung_feld2}{1}{
		Die für das Homogenfeld geltende Formel U gleich E mal D legt nahe, dass es sich bei der Spannung um die über die Wegstrecke aufintegrierte elektrische Feldstärke handelt.
		Während dieses Integral im Homogenfeld durch eine Multiplikation ermittelt werden kann, ist dies bei inhomogenen Feldern folglich nur durch das Lösen der Gleichung
	U eins zwei gleich Integral von Punkt eins bis Punkt zwei des elektrischen Feldes D S möglich. 
	Zu beachten ist hierbei, dass es sich bei dem Produkt E D S um Vektoren handelt. Zeigen sie in
	die gleiche Richtung, dürfen die Vektorpfeile einfach eliminiert werden, andernfalls muss das Skalarprodukt
	verwendet werden. Dieses kann aus dem einfachen Produkt der beiden Faktoren sowie dem Kosinus des eingeschlossenen Winkels zwischen ihnen errechnet werden.}
	\silence{3}
	}
\end{frame}



