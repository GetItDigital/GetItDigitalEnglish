\s{
    \section{Klassifizierung elektrischer Maschinen}
    Elektrische Maschinen ist eine andere Bezeichnung für elektromagnetische Energiewandler und beschreibt zum einen den Transformator als ruhende elektrische Maschine und zum anderen Elektromotoren und Generatoren.
    Elektromotoren treiben viele alltägliche Geräte wie Haushaltsgeräte und Elektrofahrzeuge an, während Generatoren und Transformatoren die Erzeugung und Verteilung von elektrischer Energie sicherstellen. 

    Dieses Kapitel bietet eine Einführung in die grundlegenden Konzepte und Funktionsweisen elektrischer Maschinen. Zunächst werden die verschiedenen Typen elektrischer Maschinen vorgestellt, einschließlich ihrer Prinzipien der Energieumwandlung und ihrer Hauptkomponenten. Es folgt eine detaillierte Betrachtung der Funktionsweise von Elektromotoren und Generatoren, der Relevanz von Transformatoren für die Energieübertragung sowie der Einfluss neuer Technologien auf die Effizienz und Leistung elektrischer Maschinen. 

    \begin{Lernziele}{Elektrische Maschinen}
        Die Studierenden
        \begin{itemize}
            \item kennen die verschiedenen elektrischen Maschinen.
            \item können grundlegende Aufgabenstellung im Themenbereich der elektrischen Maschinen lösen.
        \end{itemize}
    \end{Lernziele}
}
\b{
    \begin{frame}{Lernziele}
        \begin{Lernziele}{Elektrische Maschinen}
            Die Studierenden
            \begin{itemize}
                \item kennen die verschiedenen Arten elektrischer Maschinen.
                \item können grundlegende Aufgabenstellung im Themenbereich der elektrischen Maschinen lösen.
            \end{itemize}
        \end{Lernziele}
    \end{frame}
}
\s{
    Elektrische Maschinen sind Energieumwandler. Es wird vornehmlich zwischen Transformatoren und Motoren bzw. Generatoren unterschieden. Transformatoren wandeln elektrische Energie in eine andere Form von elektrischer Energie um, indem sie ein Spannungsniveau auf ein anderes übertragen. 
    Motoren und Generatoren hingegen wandeln elektrische Energie in mechanische Energie um und umgekehrt. Motoren konvertieren elektrische Energie in mechanische Energie, während Generatoren mechanische Energie in elektrische Energie umwandeln. Da der Umwandlungsprozess umkehrbar ist, können Motoren auch als Generatoren und umgekehrt eingesetzt werden. Eine weitere Klassifizierung bei Motoren bzw. Generatoren findet je nach betrieben Strom in Gleichstrommaschinen und Wechselstromaschine statt. Die Wechselstrommaschinen werden des Weiteren in Synchron- und Asychronmaschine gegliedert. 


    Die Funktionsweisen aller elektrischen Maschinen basieren auf den in Modul 6 besprochenen elektromagnetischen Prinzipien der Induktion und der Lorentzkraft. 
    Während der Transformator rein auf dem Prinzip der Induktion beruht, nutzt der Motor bzw. Generator je nach Wandlungsprozess unterschiedliche elektromagnetische Prinzipien. In der Funktion als Generator basiert er auf Induktion, um elektrische Energie durch sich ändernde Magnetfelder zu erzeugen. Das Prinzip der Lorentzkraft macht er sich als Motor zu nutze. 
}


