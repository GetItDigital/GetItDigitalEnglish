\begin{circuitikz}[scale=0.9]

    \draw (0,9) to[V] (0,5)
    to[short,-o] (3.5,5)
    (0.75,9) to[R, l=$R_\mathrm{i}$] (2.75,9)
    to[short, i, name=I, -o] (3.5,9)
    (3.5,8.2) to[R, v, name=R, l=$R$] (3.5,5.8);
    \draw (0,9)  to[short] (0.75,9);
    \draw (3.5,8.94)--(3.5,8.2) (3.5,5.8)--(3.5,5.06);
    \iarrmore{I}{$I$};
    \varrmore{R}{$U$};
    \draw[-latex, thick, color=voltage] (0.6,7.6)  to node[midway,right, color=voltage] {$U_\mathrm{0}= R_\mathrm{i}I_\mathrm{0}$} (0.6,6.4);

\end{circuitikz}
\end{minipage}%
\hfill
\only<2->{\begin{minipage}{0.05\textwidth}
    \centering
    \texttt{\scalebox{2}{$\Leftrightarrow$}}
\end{minipage}%
\hfill
\begin{minipage}{0.45\textwidth}
\centering
\begin{circuitikz}[scale=0.9]
    %   % Zeichne den offenen Kreis
    %   \draw[line width=1pt] (0,-0.04) arc[start angle=10,end angle=170,radius=0.42];
    %   \draw[line width=1pt] (0,-0.19) arc[start angle=-10,end angle=-170,radius=0.42];

    % % Zeichne den Pfeil in der Mitte des offenen Kreises
    % \draw[-latex, thick, line width=1.1pt] (-0.42,-0.45)    to node[right] {} (-0.42,0.25);
    \draw (0,5) to[I,i, name=I, l=$I_\mathrm{0}$] (0,9)
    (0,5) to[short,-o] (3.5,5)
    (0,9)--(2.75,9)
    (2.75,9) to[short, i, name=Ir, -o] (3.5,9)
    (2,9) to[R, l_=$R_\mathrm{i}$] (2,5)
    (2,9) to[short,*-] (2,9)
    (2,5) to[short,-*] (2,5)
    (3.5,8.2) to[R,v,i, name=R, l=$R$] (3.5,5.8);
    \draw (3.5,8.94)--(3.5,8.2) (3.5,5.8)--(3.5,5.06);
    %       % Strompfeil weiter rechts und in der Farbe "current"
    % \draw[-latex, thick, color=current] (2.8,9) -- (3,9) node[midway,above] {$I$};
    \iarrmore{I}{$I_\mathrm{0}=\frac{U_\mathrm{0}}{R_\mathrm{i}}$};
    \varrmore{R}{$U$};
    \iarrmore{Ir}{$I$};


\end{circuitikz}