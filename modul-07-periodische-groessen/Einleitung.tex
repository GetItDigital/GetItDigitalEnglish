%Einleitung

\begin{frame}
    \fta{Einleitung}
    
    \s{
        In den vorangegangenen Modulen wurden die grundsätzlichen physikalischen Gegebenheiten und die Grundbauteile besprochen. 
        Außerdem wurden die grundlegenden technischen Verhaltensweisen in Gleichstromnetzwerken geklärt. Bei den Gleichstromnetzwerken
        werden die elektrischen Netzwerke mit einem DC-Strom (Direct Current) versorgt, Wechselstromnetzwerke mit einem AC-Strom 
        (Alternating Current). Der DC-Strom fließt dauerhaft in eine Richtung. Beim AC-Strom wechselt die Stromrichtung regelmäßig 
        seine Richtung. Die Regelmäßigkeit des Wechsels der Stromrichtung des Wechslestromes erfolgt periodisch nach einer bestimmten
        Frequenz. Beispielsweise ist der ideale elektrische Widerstand als Bauteil nicht frequenzabhängig. Zusätzlich Bauteile, wie der 
        Kondensator oder eine Spule, weisen bei unterschiedlichen Frequenzen verschiedene komplexe Widerstände auf. Um diese periodischen
        Größen zu bestimmen werden die folgenden Themengebiete besprochen:

        \begin{itemize}
            \item Zeigerdiagramm
            \item Komplexe Wechselstromrechnung
            \item Effektivwert
            \item Scheinleistung, Wirkleistung und Blindleistung
            \item Drehstrom
            \item Mitsystem, Gegensystem und Nullsystem
        \end{itemize}

        \fu{
        \resizebox{0.7\textwidth}{!}{
        \begin{tikzpicture}
    \draw[very thin,color=gray] (-0.1,-0.1) grid (6.2,2.1);
    \draw[->] (-0.2,0) -- (6.4,0) node[right] {$\Re$};
    \draw[->] (0,-0.1) -- (0,2.2) node[above] {$\Im$};
    \draw[->, color=voltage] (0,0) coordinate (U) -- (6,0) coordinate (B) node[below] {$\underline{U}_\mathrm{1}$};
    \draw[->, color=red] (0,0.02) -- (3,0.02) node[below]	 {$\underline{I}_\mathrm{1}$};
    \draw[->, color=red] (3,0.02) -- (3,2.02) coordinate (D);
    \draw[color = red] (3,1)node[right]	 {$\underline{I}_\mathrm{c}$};
    \draw[->, color=red] (0,0.02) -- (3,2.02);
    \draw[color=red](2.3,1.9) node	 {$\underline{I}_\mathrm{2}$};
    \draw[->, color=voltage](6,0) -- (6,2) coordinate (C);
    \draw[color=voltage](6.2,1.3) node	 {$\underline{U}_\mathrm{2}$};
    \draw[->, color=voltage](0,0) -- (6,2);
    \draw[color=voltage](4.8,1.2) node	 {$\underline{U}_\mathrm{E}$};
    %Winkel
    \draw pic[draw, voltage, angle radius = 2cm, ->] {angle = B--U--C};
    \draw pic[draw, red, angle radius = 1.2cm, ->] {angle = B--U--D};
    \draw[red] (0.7,0.8) node{$\varphi_\mathrm{2}$};
    \draw[voltage] (2.32,0.2) node{$\varphi_\mathrm{1}$};
\end{tikzpicture}
        }
    }{{\bf Zeigerdiagramm.} Die Größen Spannung und Strom in der Darstellung als Zeiger. \label{ZeigerBeispiel}}
    }

    \b{
        Eine Sinusschwingung weist über der Zeit unterschiedliche Spannungswerte auf. Wiederholen sich die exakten Sinusschwingungen,
        handelt es sich um einen zetlich periodischen Vorgang. 
        Im Modul 7 'Periodische Größen' werden die folgenden Themegebiete behandelt:\\

        \begin{minipage}[t]{0.45\textwidth}
            \begin{itemize}
                \vspace{\baselineskip}
                \item Zeigerdiagramm
                \item Komplexe Wechselstromrechnung
                \item Effektivwert
                \item Scheinleistung, Wirkleistung und Blindleistung
                \item Drehstrom
                \item Mitsystem, Gegensystem und Nullsystem
            \end{itemize}
        \end{minipage}
        \begin{minipage}[t]{0.54\textwidth}
        \fu{
            \resizebox{\textwidth}{!}{
            \begin{tikzpicture}
                \draw[very thin,color=gray] (-0.1,-0.1) grid (6.2,2.1);
                \draw[->] (-0.2,0) -- (6.4,0) node[right] {$\Re$};
                \draw[->] (0,-0.1) -- (0,2.2) node[above] {$\Im$};
                \draw[->, color=voltage] (0,0) coordinate (U) -- (6,0) coordinate (B) node[below] {$\underline{U}_\mathrm{1}$};
                \draw[->, color=red] (0,0.02) -- (3,0.02) node[below]	 {$\underline{I}_\mathrm{1}$};
                \draw[->, color=red] (3,0.02) -- (3,2.02) coordinate (D);
                \draw[color = red] (3,1)node[right]	 {$\underline{I}_\mathrm{c}$};
                \draw[->, color=red] (0,0.02) -- (3,2.02);
                \draw[color=red](2.3,1.9) node	 {$\underline{I}_\mathrm{2}$};
                \draw[->, color=voltage](6,0) -- (6,2) coordinate (C);
                \draw[color=voltage](6.2,1.3) node	 {$\underline{U}_\mathrm{2}$};
                \draw[->, color=voltage](0,0) -- (6,2);
                \draw[color=voltage](4.8,1.2) node	 {$\underline{U}_\mathrm{E}$};
                %Winkel
                \draw pic[draw, voltage, angle radius = 2cm, ->] {angle = B--U--C};
                \draw pic[draw, red, angle radius = 1.2cm, ->] {angle = B--U--D};
                \draw[red] (0.7,0.8) node{$\varphi_\mathrm{2}$};
                \draw[voltage] (2.32,0.2) node{$\varphi_\mathrm{1}$};
            \end{tikzpicture}
            }
        }{}
        \end{minipage}
    }
\end{frame}
\newpage