\subsection{Verhalten eines Transistors bei Wechselspannung\label{Aufg19}}
% Alt: Aufgabe 19
\Aufgabe{
    Was passiert wenn an einen Transistor eine reine Wechselspannung 
    angelegt wird?
    \begin {figure} [H]
    \centering
    \begin{tikzpicture}
    % Spannungspfeile in blau
    \draw(6.3,2.3) to[open,v,name=ua] (6.3,0);
  
    % Batterie links
    \draw
    (0,0) to[sV, v<, name=UE ] (0,1.5);
  
    % Widerstand RV
    \draw
    (0,1.5) -- (2,1.5)
    to[R=$100 \, k \Omega$] (3.5,1.5) to[short,i,name=ib] (4.5,1.5)
    -- (5,1.5) ; % Verbindung mit Basis
    
    % Transistor
    \draw
   (5.5,1.5) node[npn, tr circle] (Q1) {}
   (Q1.base) 
   (Q1.collector)  
   (Q1.emitter)  ;
   
   \draw
   (Q1.C) -- (5.5,3)
   to[R=$500 \,\Omega$] (7,3);

   \draw (7,3) to [V,name=15V] (7,0);
   \draw (Q1.E) -- (5.5,0) -- (7,0);

   \draw (5.5,0) to[short, -o] (6.3,0) -- (7,0);

   \draw (0,0) -- (5.5,0); 
   
   \draw (5.5,2.3) to[short, *-o] (6.3,2.3);
   \varrmore{UE}{$u_\mathrm{E}$};
   \varrmore{15V}{$15\,\mathrm{V}$};
   \iarrmore{ib}{$I_\mathrm{B}$};
   \varrmore{ua}{$u_\mathrm{A}$};
  \end{tikzpicture}
    \label{fig:FigTransistorWechselspannung}
    \end {figure}
}


\Loesung{
    Wenn an einen Transistor eine reine Wechselspannung angelegt wird, tritt kein kontinuierlicher Verstärkungsbetrieb auf.  
    
    Das führt zu folgenden Effekten:
    \begin{itemize}
        \item Verzerrungen im Ausgangssignal, da nur die positiven Halbwellen verstärkt werden.
        \item Die negative Halbwelle wird abgeschnitten, was einer Gleichrichtung des Signals ähnelt.
        \item Es erfolgt keine lineare Verstärkung, da der Transistor zwischen Sperr- und Sättigungszustand wechselt.
    \end{itemize}
    Siehe: Abschnitt \ref{merk:MerksatzWechselspannung}
    
    % **Folgende Effekte treten auf:**  
    
    % - **Verzerrungen im Ausgangssignal:**  
    %   Da nur die positiven Halbwellen verstärkt werden, während die negativen nicht beeinflusst oder unterdrückt werden,  
    %   entsteht eine ungleichmäßige Signalform.  
    
    % - **Die negative Halbwelle wird abgeschnitten:**  
    %   Dies ähnelt einer Gleichrichtung des Signals, da der Transistor während der negativen Halbwelle nicht leitet  
    %   und kein Verstärkungseffekt auftritt.  
    
    % - **Keine lineare Verstärkung:**  
    %   Da der Transistor zwischen Sperr- und Sättigungszustand wechselt, erfolgt keine gleichmäßige Verstärkung über  
    %   den gesamten Signalverlauf.  
    %   Der Verstärkungsprozess ist also nicht linear, was bedeutet, dass das Ausgangssignal nicht proportional  
    %   zum Eingangssignal verstärkt wird.  
    
    % Daraus lässt sich schließen, dass für einen kontinuierlichen Verstärkungsbetrieb eine Gleichspannungskomponente  
    % als Arbeitspunkt erforderlich ist, sodass der Transistor nicht in den Sperrbereich wechselt.
    %
    
}
