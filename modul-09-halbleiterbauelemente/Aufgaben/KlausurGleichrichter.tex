\subsection{Gleichrichterschaltung\label{Aufg28}}
% Alt: Klausuraufgabe 1
\Aufgabe{
    Hier eine Gleichrichterschaltung mit pn-Silizium-Dioden. 
    Die Spannungsquelle $V_1$ liefert sinusförmige $50 \, \mathrm{Hz}$ Wechselspannung mit einer Amplitude von $5\, \mathrm{V}$.
    \begin {figure} [H]
    \centering
    \begin{tikzpicture}
    \draw (-1,0) 
    to[D,l=$D_\mathrm{2}$] (-1,2) -- (-1,4)
    to[D,l=$D_\mathrm{1}$] (-1,6) -- (1,6)
    to[C,l=$C_\mathrm{1}$] (1,4) -- (1,2)
    to[C,l=$C_\mathrm{2}$] (1,0) -- (-1,0);

    \draw (1,0)
    to[short, *-] (3,0)
    to[R, l_=$R_2$] (3,2) -- (3,4)
    to[R, l_=$R_1$] (3,6);  

    \draw (3,0)
    to[short, *-] (5.5,0)
    to[R, l_=$R_3$] (5.5,6)
    to[short, -*] (3,6)
    to[short, -*] (1,6);

    \draw (3,2)
    to[short, *-] (1,2)
    to[short, *-*] (-3,2)    
    node[ground]{}
    to[sV, name=UE, v<] (-3,4)
    to[short, -*] (-1,4); 

    \varrmore{UE}{$u_\mathrm{E}(t)$};
\end{tikzpicture}
    \label{fig:FigKlausurGleichrichter}
 \end {figure}
 \begin{itemize}
 \item[a)]
Welche Gleichspannungen treten über $R_1$,$R_2$ und $R_3$ auf?
 \item[b)]
Welche Funktionen haben die Kondensatoren $C_1$ und $C_2$? 
 \item[c)]
Was ist der besondere Vorteil dieser Gleichrichterschaltung, z.B. für Anwendungen mit Operationsverstärkern?
 \item[d)]
 Skizzieren Sie eine andere Spannungs-Verdopplungs-Schaltung, bei der die maximale positive Gleichsspannung (Ausgangsspannung) gegenüber Massepotential der speisenden Wechselspannung $V_1$ zur Verfügung steht. 
 \end{itemize}
%  \speech{
%   **Aufgabe 28: Gleichrichterschaltung mit pn-Silizium-Dioden**
  
%   Diese Aufgabe behandelt eine Gleichrichterschaltung mit zwei pn-Silizium-Dioden. Die Spannungsquelle \( V_1 \) liefert eine sinusförmige Wechselspannung mit einer Frequenz von 50 Hz und einer Amplitude von 5 V.
  
  
%    **Beschreibung der Schaltung:**
%   Die gegebene Schaltung ist eine **Spannungsverdopplungsschaltung**, die eine Wechselspannung von \( V_1 \) gleichrichtet und eine höhere Gleichspannung erzeugt.
  
%   - **Spannungsquelle \( V_1 \)**:  
%     - Diese Wechselspannungsquelle liefert eine sinusförmige Wechselspannung mit einer Amplitude von 5 V.
%     - Der negative Pol ist mit Masse verbunden.
  
%   - **Dioden \( D_1 \) und \( D_2 \)**:  
%     - Diese Dioden sind in einer Anordnung geschaltet, die eine Gleichrichtung der Spannung bewirkt.
%     - Sie ermöglichen die Umwandlung der Wechselspannung in eine Gleichspannung.
  
%   - **Kondensatoren \( C_1 \) und \( C_2 \)**:  
%     - Diese Kondensatoren speichern Ladung und tragen zur Stabilisierung der Ausgangsspannung bei.
  
%   - **Widerstände \( R_1 \), \( R_2 \) und \( R_3 \)**:  
%     - Diese Widerstände sind Teil des Lastkreises und beeinflussen die Ausgangsspannung.
  
  
%    **Fragen zur Aufgabe:**
%   **(a)** Welche Gleichspannungen treten über den Widerständen \( R_1 \), \( R_2 \) und \( R_3 \) auf?
  
%   **(b)** Welche Funktionen haben die Kondensatoren \( C_1 \) und \( C_2 \)?
  
%   **(c)** Was ist der besondere Vorteil dieser Gleichrichterschaltung, z. B. für Anwendungen mit Operationsverstärkern?
  
%   **(d)** Skizzieren Sie eine andere Spannungs-Verdopplungs-Schaltung, bei der die maximale positive Gleichspannung (Ausgangsspannung) gegenüber dem Massepotential der speisenden Wechselspannung \( V_1 \) zur Verfügung steht.
  
%   }
  

}

\Loesung{
    \begin{itemize}
        \item[a)]
      \end{itemize}
      \begin{itemize}
        \item $U_\mathrm{R1} = 4,3 \, \mathrm{V} \Rightarrow$ da $0,7 \,\mathrm{V}$ über $D_1$ abfallen \\
        \item $U_\mathrm{R2} = -4,3 \, \mathrm{V} \Rightarrow$ da gleiches für die negative Halbwelle \\
        \item $U_\mathrm{R3} = 8,6 \,\mathrm{V} \Rightarrow$ da $R_3$ beide Spannungswerte aufnimmt \\
      \end{itemize}
      \begin{itemize}
        \item[b)]
      \end{itemize}
      \begin{center}
        Die Kondensatoren $C_1$ und $C_2$ haben einerseits die Funktionen die Signale zu glätten, sowohl von der negativen als auch der positiven Halbwelle. 
      \end{center}
      \begin{itemize}
        \item[c)]
      \end{itemize}
      \begin{center}
        Der Vorteil dieser Gleichrichtung ist, das sowohl positive als auch negative Spannungen abgegriffen werden können.
      \end{center}
      \begin{itemize}
        \item[d)]
      \end{itemize}
    \begin {figure} [H]
        \centering
        \begin{tikzpicture}
    \draw (0,0) to[short, o-] (1,0) to[C,l=$C_\mathrm{1}$](2,0) to[short, *-] (2,0) to[D,l=$D_\mathrm{2}$, -*] (4,0) to[short, -o] (6,0);
    \draw (0,-2) to[short, o-] (1,-2) -- (3,-2)node[ground]{};
    \draw (1.5,-2) to[short, -*] (2,-2) to[short, -*] (4,-2) to[short, -o] (6,-2);
    
    
    \draw (6,0) to[open,v>, name=ua] (6,-2);
    
    
    \draw (2,-2) to[D,l=$D_\mathrm{1}$](2,0);
    \draw (4,-2) to[C,l=$C_\mathrm{2}$](4,0);
    
    
    \draw (0,0) to[open, v>, name=ue] (0,-2);
    
    \varrmore{ua}{$U_\mathrm{a}$};
    \varrmore{ue}{$U_\mathrm{e}$};
    
    \end{tikzpicture}
        \label{fig:LsgKlausurGleichrichter}
    \end {figure}
    Siehe: Abschnitt \ref{fig:BrueckengleichrichterSchaltung}
%    \speech{
% **Lösung zu Aufgabe 28**  

% **Teilaufgabe a)**  
% - Die Spannung über R1 beträgt 4,3 Volt, da 0,7 Volt über die Diode D1 abfallen.  
% - Die Spannung über R2 beträgt -4,3 Volt, da der gleiche Effekt für die negative Halbwelle gilt.  
% - Die Spannung über R3 beträgt 8,6 Volt, da dieser Widerstand beide Spannungswerte aufnimmt.  

% **Teilaufgabe b)**  
% Die Kondensatoren **C1** und **C2** haben die Funktion, die Signale zu glätten, sowohl für die negativen als auch für die positiven Halbwellen.  

% **Teilaufgabe c)**  
% Der besondere Vorteil dieser Gleichrichterschaltung ist, dass sowohl **positive als auch negative Spannungen** abgegriffen werden können.  

% **Teilaufgabe d)**  
% Die abgebildete Schaltung zeigt eine Spannungs-Verdopplungs-Schaltung mit zwei Dioden und zwei Kondensatoren:  
% - Die Wechselspannung **U_E** wird an den Kondensator **C1** angelegt.  
% - Die Diode **D1** ist in Durchlassrichtung geschaltet und lädt den Kondensator **C2** auf.  
% - Die zweite Diode **D2** leitet den Strom weiter, sodass die Spannung **U_A** am Ausgang auf das **Doppelte der Eingangsspannung** ansteigen kann.  
% - Durch die Anordnung von **C1, C2, D1 und D2** kann die maximale positive Gleichspannung gegenüber dem Massepotenzial der speisenden Wechselspannung erhöht werden.  

% Diese Schaltung wird oft in **Hochspannungsversorgungen** eingesetzt, um eine höhere Ausgangsspannung aus einer geringen Eingangsspannung zu erzeugen.
% }

      
}