% RC Entladeschaltung mit Wechselschalter erweitertes Labeling
\begin{tikzpicture}

    \ctikzset{bipoles/cuteswitch/thickness=0.3} %Liniendicke für bessere Lesbarkeit reduzieren
    
    %Switch Circuit
    \draw(2,1.69)
        node[cute spdt down arrow, rotate = 180](s1){}% Cute Switch, Position des Schalters mit up/down veränderbar
        (s1.center) node[left, xshift= -5]{t=0};%Label 
    \draw(s1.out 2) -- (0,2.11);% nach Uq+
    \draw([yshift=-0.5mm] s1.cout 1) to[short,-*] (1.70,0);%nach Uq-
    %RC Circuit
    \draw(0,2.11) 
    to[V, name=v, v_>=$U_q$, !v](0,0)
    to[short](5,0) 
    to[C, l=$C$, name= C, v<=$u_C$, !v](5,1.685)
    to[R, l=$R$, name=R, v<=$u_R$, i^<=$i$, !vi](s1.in);
    % Spannungs- und Strompfeile
    \varronly{v};
    \varronly{C};
    \varronly{R};
    \iarronly{R};

\end{tikzpicture} 