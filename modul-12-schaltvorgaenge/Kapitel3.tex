%%%%%%%%%%%%%%%%%%%%%%%%%%%%%%%%%%%%%%%%%%%%%%%%%%%%%%%%%%%%%%%%%%%%%%%%%%%%%%%%%%%
%%%%%%%%%%%%%%%%%%%%%%%%%%%%%%%%%%%%%%%%%%%%%%%%%%%%%%%%%%%%%%%%%%%%%%%%%%%%%%%%%%%
%                               START KAPITEL 3
%%%%%%%%%%%%%%%%%%%%%%%%%%%%%%%%%%%%%%%%%%%%%%%%%%%%%%%%%%%%%%%%%%%%%%%%%%%%%%%%%%%
%%%%%%%%%%%%%%%%%%%%%%%%%%%%%%%%%%%%%%%%%%%%%%%%%%%%%%%%%%%%%%%%%%%%%%%%%%%%%%%%%%%

\section[Berechnung im Zeitbereich]{Schaltvorgänge im Zeitbereich berechnen}
\label{sec:schaltvorgaengezeitbereich}
\begin{frame}\ftx{\secname}
\s{%
    In diesem Kapitel werden Schaltvorgänge in elektrischen Schaltungen im Zeitbereich betrachtet.
    Es wird zwischen Schaltvorgängen bei Gleichspannung und Wechselspannung unterschieden.
    Die Betrachtung erfolgt für RC- und RL-Glieder, sowie RLC-Glieder.
}%
    \begin{Lernziele}{Schaltvorgänge}
        Studierende lernen:
        \begin{itemize}\b{\setlength\itemsep{0.7em}}
            \item Schaltvorgänge im Zeitbereich zu berechnen % Anwendung, Verstehen
            \item das Schaltverhalten von RC-, RL- und RLC-Gliedern kennen und verstehen % Verstehen
            \item die Möglichkeit kennen Schaltvorgänge bei AC-Anregung zu minimieren% Wissen, Anwendung
            \item das Schwingverhalten von Schwingkreisen zu verstehen und zu analysieren % Analyse
        \end{itemize}
    \end{Lernziele}
\end{frame}

\v{% Video only Frame
\begin{frame}\ftx{\secname}
    \begin{Lernziele}{Video - Schaltvorgänge bei Gleichspannung [1/2]}
        In diesen Video lernen Sie:
        \begin{itemize}\b{\setlength\itemsep{0.7em}}
        \item Schaltvorgänge bei DC Anregung im Zeitbereich zu berechnen,
        \item die Zeitkonstante $\tau$ kennen und zu berechnen und
        \item den Verlauf von DC Ladevorgängen bei RC-Gliedern zu verstehen.
        \end{itemize}
    \end{Lernziele}%

\speech{SchaltvorgaengeDC1Lernziele}{1}{%
Willkommen zum vierten Video der Videoreihe für das Modul Schaltvorgänge. 
In diesem Video steigen wir ein in das Kapitel 3 **Schaltvorgänge im Zeitbereich berechnen**
und befassen uns mit den Themen aus Kapitel 3 punkt 1 **Schaltvorgänge bei Gleichspannung**.
Wir lernen die Zeitkonstante Tau kennen und verstehen und berechnen Lade- und Entladevorgänge bei RC-Gliedern.
}% Speech Ende
\end{frame}
}% Video only Frame Ende
%%%%%%%%%%%%%%%%%%%%%%%%%%%%%%%%%%%%%%%%%%%%%%%%%%%%%%%%%%%%%%%%%%%%%%%%%%%%%%%%%%%%%%%%%%%%%%%%%%

\subsection{bei Gleichspannung}
\label{sec:schaltvorgaengezeitbereich:dc}
\begin{frame}\ftx{Schaltvorgänge \subsecname}
\s{%
    In diesem Kapitel werden Lade- und Entladevorgänge von Kapazitäten und Induktivitäten
    jeweils in Reihe mit einem Widerstand bei Gleichspannung berechnet und dargestellt.

    Schaltvorgänge bei sinusförmiger Anregung folgen in Kapitel \ref{sec:schaltvorgaengezeitbereich:ac}.
}%
\b{%
    \centering
    \begin{minipage}{0.48\textwidth}%\centering % linksbündig
        Schaltvorgänge bei Gleichspannung sind:
        \vspace{0.5em}
        \begin{itemize}
            \item Ladevorgänge,
            \item Entladevorgänge.
        \end{itemize}
        \vspace{0.5em}
        \pause
        Berechnung zunächst für:
        \vspace{0.5em}
        \begin{itemize}
            \item eine Kapazität $C$
            \item eine Induktivität $L$
        \end{itemize}
        \vspace{0.5em}
        jeweils in Reihe mit einem Widerstand $R$.
        \pause

        \vspace{1em}
        RC- oder RL-Glied $\Longrightarrow$ \textbf{DGL} 1. Ordnung

        \begin{equation*}
            a_1 \cdot \dt\, y(t) + a_0 \cdot y(t) = b
        \end{equation*}
    \end{minipage}%
    \begin{minipage}{0.48\textwidth}\centering
        \pause
        \resizebox{0.8\textwidth}{!}{\includegraphics{Tikz/pdf/plot_rc_cycle_seperate_u.pdf}}
        Schaltvorgänge bei RC-Glied (DC)
    \end{minipage}
}%

\speech{SchaltvorgaengeDC1}{1}{%
Bei Schaltvorgängen bei Gleichspannung kann es sich prinzipiell um **Ladevorgänge** oder um **Entladevorgänge** handeln.
}% Speech Ende
\speech{SchaltvorgaengeDC1}{2}{%
Diese werden wir exemplarisch jeweils einmal für eine Kapazität C und jeweils einmal für eine Induktivität L, 
in allen Fällen in Reihe mit einem Widerstand R berechnen.
Können wir bereits abschätzen, wie die Lade- und Entladevorgänge bei Gleichspannung aussehen werden?
}% Speech Ende
\speech{SchaltvorgaengeDC1}{3}{%
Wie aus den Grundlagen bekannt ist, ergibt sich für ein einfaches R C oder ein einfaches R L Glied 
jeweils eine **D G L erster Ordnung**, da in beiden Fällen nur ein Energiespeicher vorhanden ist (C oder L).
Die D G L wird daher die hier gezeigte Form: A eins mal erste Ableitung von y von t **plus** A null mal y von t ist gleich b annehmen.
Und wie sehen die Zeitverläufe prinzipiell aus?
}% Speech Ende
\speech{SchaltvorgaengeDC1}{4}{%
Bei einer D G L **erster Ordnung** erhalten wir einen **einzelnen** Eigenwert lambda. 
Dadurch entstehen wie rechts im Diagramm dargestellt ist, charakteristischen Exponentialverläufe beim Laden und Entladen bei Gleichspannung. 
Die Exponentialform resultiert aus dem flüchtigen Zustand, das heißt der homogenen Lösung K mal e hoch lambda t für einfache Nullstellen. 
}% Speech Ende
\end{frame}

%-------------------------------------------------------------------------------------------------%

\subsubsection{bei Kapazitäten}
\label{sec:schaltvorgaengezeitbereich:dc:c}
\begin{frame}[t]\ftx{Schaltvorgänge \subsubsecname, Zeitkonstante}
\s{% skript only, tau
    Beispiel \ref{bsp:rc:chargedc} zeigt den Rechenweg zur Bestimmung des Spannungsverlaufs $u_C(t)$
    einer zu Beginn entladenen Kapazität $C$ beim Laden mit Gleichspannung für $t \geq 0$ über einen Widerstand $R$
    und den Zeitverlauf des Ladestroms $i(t)$.
    Die angewandte Berechnungsmethode entspricht dem Verfahren für DGL im Zeitbereich nach Merksatz \ref{merk:berechnungsverfahren:zeitbereich}.

    Aus den Ergebnissen ablesen lässt sich die exponentielle Annäherung von $u_C(t)$ an die Anregung $U_q$:
    \begin{align}
        u_C(t) &= U_q \cdot \left( 1 - \mathrm{e}^{-\sfrac{t}{\tau}} \right)
    \intertext{und der exponentielle Abfall von $i(t)$ respektive $u_R(t)$ ab deren Maximalwerten $U_q/R$ und $U_q$:}
        i(t) &= \frac{U_q}{R} \cdot \mathrm{e}^{-\sfrac{t}{\tau}}\\[4pt]
        u_R(t) &= U_q \cdot \mathrm{e}^{-\sfrac{t}{\tau}}
    \end{align}
    bis die Kapazität geladen ist. Die \textbf{Zeitkonstante}\index{Größe>Zeitkonstante} $\tau$ bestimmt die Steilheit
    der exponentiellen Verläufe.
    Für das RC-Glied ergibt sich $\tau = RC$. Die Einheit für $\tau$ ist gleich jener der Zeit.

    \begin{minipage}{0.95\textwidth}\centering
    \begin{minipage}[b]{0.48\textwidth}\centering
        \fu{\includegraphics{Tikz/pdf/plot_rc_charge_uc_tau.pdf}}{Exp. Annäherung und $\tau$\label{fig:plot:rc:tau}}%
    \end{minipage}%
    \begin{minipage}[b]{0.48\textwidth}\centering
        \begin{table}[H]\centering
            \caption{Typ. Werte Zeitkonstante $\tau$}
            \label{tab:tau:typischewerte}
            \begin{tabular}{cccc}
                \toprule
                $t$ & $\tau$ & $3\,\tau$ & $5\,\tau$ \\
                \midrule
                &&&\\[-1em]
                $1 - \mathrm{e}^{-t/\tau}$ & $63,2\%$ & $95,0\%$ & $99,3\%$ $\vphantom{\big|}$\\[6pt]
                $ \mathrm{e}^{-t/\tau}$ & $36,8\%$ & $5,0\%$ & $0,7\%$ $\vphantom{\big|}$\\
                \bottomrule
            \end{tabular}\par\bigskip
            Faustregel: \glqq Eingeschwungen\grqq\ nach $5\,\tau$
            \vspace{2pt}
        \end{table}
    \end{minipage}
    \end{minipage}

    Wie in Abbildung \ref{fig:plot:rc:tau} dargestellt, schneidet die Tangente von $u_C$ im Punkt $u_C(t=0)$
    den Endwert $U_q$ zum Zeitpunkt $t=\tau$. Allgemein gilt in jedem Punkt exponentieller Annäherungen/Abfälle, dass die Tangete den Endwert nach $\tau$ erreicht.

    Tabelle \ref{tab:tau:typischewerte} zeigt typische Werte für exponentielle Annäherungen/Abfälle nach Vielfachen von $\tau$.
    Die Werte sind relativ zum Maximalwert (Start- oder angestrebter Grenzwert).
    Als \glqq eingeschwungen\grqq\ gilt ein solcher Verlauf nach $5\,\tau$.

    % Rechenwege
    Beispiel \ref{bsp:rc:discharge} zeigt den Rechenweg zur Bestimmung des Spannungsverlaufs $u_C(t)$
    einer Kapazität $C$ beim Entladen über einen Widerstand $R$ und zugehörige Zeitverläufe.
    Hierbei ergeben sich analog zum Ladevorgang exponentielle Verläufe mit gleicher Zeitkonstante, jedoch mit unterschiedlichen Start- und Endwerten.

    Der Rechenweg in den Beispielen \ref{bsp:rc:chargedc} und \ref{bsp:rc:discharge} ist prinzipiell identisch.
    Beim Lade- und Entladevorgang unterscheiden sich lediglich die Anfangsbedingung $u_C(0)$ (bei Schaltzeitpuntk $t=0$) und die Randbedingung $u_{C,\mathrm{e}}$ (für $t \to \infty$).
    Dadurch ergeben sich unterschiedliche partikuläre Lösungen und andere Konstanten $K$ in der allgemeinen Lösung.

    Verallgemeinert lässt sich der Spannungsverlauf $u_C$ einer Kapazität $C$ beim Laden/Entladen mit Gleichspannung über einen Widerstand $R$
    ab einem Schaltzeitpunkt $t_0$ beschreiben durch:
    \begin{equation}\label{eq:rc:charge:dc:allg:parthom}
        u_C(t-t_0) = \underbrace{ u_{C,\mathrm{e}} \vphantom{\Big|}}_{u_{C,\mathrm{p}}} + \underbrace{ \big(u_C(t_0)-u_{C,\mathrm e}\big) \cdot  \mathrm{e}^{-\frac{t-t_0}{\tau}} \vphantom{\Big|}}_{u_{C,\mathrm{h}}}
    \end{equation}
    Etwas umgeschrieben lässt sich der Spannungsverlauf auch recht anschaulich wie folgt formulieren:
    \begin{equation}\label{eq:rc:charge:dc:allg:startend}
        u_C(t-t_0) = \underbrace{u_C(t_0) \vphantom{\Big|}}_{\text{Startwert}} + \big(\underbrace{u_{C,\mathrm e} - u_C(t_0) \vphantom{\Big|}}_{\text{Differenz}}\big) \cdot \bigg( \underbrace{ 1- \mathrm{e}^{-\frac{t-t_0}{\tau}} \vphantom{\Big|}}_{\text{\clap{Exp. Annäherung}}} \bigg)
    \end{equation}
    Die allgemeine Lösung in Gleichung \ref{eq:rc:charge:dc:allg:parthom} beziehungsweise \ref{eq:rc:charge:dc:allg:startend} gilt nur für Anregungen mit Gleichgröße.
    Die Lösung ist typisch für verlustbehaftete, lineare Systeme erster Ordnung (mit einem Energiespeicher), weshalb sie in ähnlicher Form auch bei RL-Gliedern auftritt.

    \textbf{Exkurs: DGL 1. Ordnung und Zeitkonstante $\tau$}

    % Hinweis DGL 1. Ord., tau ablesen
    Die Exponentialform ist typisch für lineare Systeme erster Ordnung mit Energiespeicher
    und ergibt sich aus der Lösung der linearen, gewöhnlichen DGL 1. Ordnung mit konstanten Koeffizienten.
    Durch Umformung ($a_0=1$) lässt sich die Zeitkonstante $\tau$ direkt aus der DGL ablesen ($a_1=\tau$).:
    \begin{equation}\label{eq:dgl:tau}
        \tau \cdot \dt\,y(t) + y(t) = b(t)
    \end{equation}
    Dadurch lässt sich die homogene Lösung direkt angeben mit.:
    \begin{equation}
        y_{\mathrm{h}}(t) = K\cdot\mathrm{e}^{-\frac{t}{\tau}}
    \end{equation}
    Die Zeitkonstante ist ein Maß für die Schnelligkeit, mit der ein System auf eine Änderung reagiert.
    }% end skript only
\b{% beamer only, $\tau$
\begin{minipage}{\textwidth}\centering%
% Obere Hälfte
% links
\begin{minipage}[t][][t]{0.48\textwidth}\centering\vspace{0cm}% vspace: input empty space before tikzpicture for top alignment%
    \includegraphics{Tikz/pdf/circ_switchable_rc_charge_dc.pdf}\\[4pt]
    \textbf{DGL 1. Ordnung} $\Rightarrow$ Exponentialverlauf.
\end{minipage}%
\pause%
% rechts
\begin{minipage}[t][][t]{0.48\textwidth}\centering\vspace{0cm}% vspace: input empty space before tikzpicture for top alignment%
    \includegraphics{Tikz/pdf/plot_rc_charge_uc_tau.pdf}\\
    $\ecv{u_C}(t) = \ecv{U_q} \left( 1 - \mathrm{e}^{-\sfrac{t}{\tau}} \right)$
\end{minipage}%
\pause%
\vspace{1cm}% Leerzeile

% Untere Hälfte
% links
\begin{minipage}[t][][t]{0.48\textwidth}\vspace{0cm}% vspace: input empty space before tikzpicture for top alignment%
    Die \textbf{Zeitkonstante}\index{Größe>Zeitkonstante} $\tau,\quad [\tau]=1\,\mathrm{s}$:
    \vspace{0.35cm}
    \begin{itemize}
        \item bestimmt die Steilheit bei exponentiellen Verläufen.% Eigentlich "Schnelligkeit", weil Vorfaktor nicht berücksichtigt.
        \item ist die Zeit nach der die Tangente den Endwert erreicht.
        \item ist beim RC-Glied $\tau = RC$.
    \end{itemize}
    \vspace{1.9cm}
\end{minipage}%
\pause%
% rechts
\begin{minipage}[t][][t]{0.48\textwidth}\centering\vspace{0cm}% vspace: input empty space before tikzpicture for top alignment%
    Typ. Werte Zeitkonstante $\tau$
    \begin{tabular}{cccc}
        \toprule
        $t$ & $\tau$ & $3\,\tau$ & $5\,\tau$ \\
        \midrule
        &&&\\[-1em]
        $1 - \mathrm{e}^{-t/\tau}$ & $63,2\%$ & $95,0\%$ & $99,3\%$ $\vphantom{\big|}$\\[6pt]
        $ \mathrm{e}^{-t/\tau}$ & $36,8\%$ & $5,0\%$ & $0,7\%$ $\vphantom{\big|}$\\
        \bottomrule
    \end{tabular}\par\bigskip
    Faustregel: \glqq Eingeschwungen\grqq\ nach $5\,\tau$
\end{minipage}
\end{minipage}
}%

\speech{SchaltvorgaengeDC1Zeitkonstante}{1}{%
Beginnen wir links mit einem exemplarischen Schaltbild.
Das gezeigte R C Serienglied ist über einen Schalter entweder kurzgeschlossen 
oder an eine ideale Gleichspannungsquelle U q angeschlossen. 
Bevor wir im Detail die Zeitverläufe rechnerisch bestimmen, 
führen wir zunächst eine neue Größe ein, um konvergente Exponentialverläufe besser beschreiben zu können.
Wegen des einen Energiespeichers C, hat die D G L erste Ordnung und die Zeitverläufe haben die Form von Exponentialverläufen.
}% Speech Ende
\speech{SchaltvorgaengeDC1Zeitkonstante}{2}{%
Rechts sehen wir im Zeitverlaufsdiagramm die Spannung u C an der Kapazität. 
Die Spannung u C ist zu Beginn null und steigt ab dem Schaltzeitpunkt t gleich null exponentiell bis zum Endwert U q.
Die dazughörige Gleichung ist: u C von t ist gleich U q (der Endwert) mal Klammer auf eins minus e hoch minus t durch tau Klammer zu. 
Der Term: Eins minus e hoch minus t durch tau, entspricht einer exponentiellen Annäherung von null an eins. 
}% Speech Ende
\speech{SchaltvorgaengeDC1Zeitkonstante}{3}{%
Dieses **Tau** wird als **Zeitkonstante** bezeichnet und hat die gleiche Einheit wie die Zeit. 
Mithilfe von Tau lässt sich die Steilheit exponentieller Verläufe darstellen. 
Wie im Zeitverlauf oben rechts zu sehen ist, schneidet die Tangente von exponentiellen Verläufen nach einem Tau den Endwert. Das gilt zu jedem Zeitpunkt. 
In dem hier gezeigten Schaltbild entspricht die Zeitkonstante Tau dem Produkt aus R und C.
}% Speech Ende
\speech{SchaltvorgaengeDC1Zeitkonstante}{4}{%
Die hier gezeigte Tabelle zeigt welche Wert nach einem, nach drei oder nach fünf Tau 
in der mittleren Zeile bei einer exponentiellen Annäherung erreicht werden und
in der unteren Zeile bei einem exponentiellen Abfall erreicht werden.
Beide Verläufe näheren sich einem Endwert (eins oder null), erreichen diesen aber nie.
Deshalb gilt als **Faustregel**, dass ein Exponentialverlauf **nach fünf Tau eingeschwungen** ist.
Dann hat der Wert ca. neunundneunzig Prozent der Differenz zum Endwert überbrückt.
Nach einem Tau sind es ca. dreiundsechzig Prozent, nach drei Tau ca. fünfundneunzig Prozent.
Die Orientierungswerte sind zum Beispiel hilfreich bei der Dimensionierung von Bauteilen oder beim Abschätzen von Latenzen.
}% Speech Ende
\end{frame}

%- - - - - - - - - - - - - - - - - - - - - - - - - - - - - - - - - - - - - - - - - - - - - - - - -%

% Ohne Frames
\s{% RC-Laden, skript only without frames
    \newpage
    \begin{bsp}{RC-Glied DC-Ladevorgang}{rc:chargedc}
        Bestimmung von Spannung $\ecv{u_C}$ und Ladestrom $\eci{i}$ der Kapazität $C$ beim Laden über Widerstand $R$
        mit Gleichspannung $\ecv{U_q}$ für $t \geq 0$. Die Kapazität ist zu Beginn entladen.\\[-6pt]

        \begin{minipage}{\textwidth}\centering
        \begin{minipage}[t][][t]{0.48\textwidth}\centering\vspace{0cm}% vspace: input empty space before tikzpicture for top alignment
            \includegraphics{Tikz/pdf/circ_switchable_rc_charge_dc.pdf}%
            \begin{equation*}\begin{aligned}
            \text{gegeben:}&&       &\eci{i}(0),\ \ecv{U_q},\ R,\ C \\
            \text{gesucht:}&&       &\eci{i}(t), \ecv{u_L}(t)
            \end{aligned}\end{equation*}
        \end{minipage}
        \begin{minipage}[t][][t]{0.48\textwidth}\centering\vspace{0cm}% vspace: input empty space before tikzpicture for top alignment
            \includegraphics{Tikz/pdf/plot_rc_charge_uc_i_seperate.pdf}
        \end{minipage}
        \end{minipage}

        \textbf{1.) DGL aufstellen:} von $\ecv{u_C}(t)$ für $t>0$\vspace{5pt}

        Ausgehend von der Maschengleichung wird $\ecv{u_R}$ in Abhängigkeit von $\ecv{u_C}$ ausgedrückt:%
        \begingroup% Änderungen lokal vornehmen
            \addtolength{\jot}{2pt}% Zeilenabstand in align-Umgebung erhöhen
        \begin{align*}
            &\text{Masche}:&&&
                \ecv{u_R} + \ecv{u_C}  &= \ecv{U_q}&&&
                &\text{mit}\quad \ecv{u_R} = R \cdot \eci{i} \\
            &&&&
                R \cdot {\color{red} i} + \ecv{u_C}    &= \ecv{U_q}&&&
                &\text{mit}\quad \eci{i} = C \cdot \dt \ecv{u_C} \\
            &\text{DGL}:&&&
                R \cdot C \cdot \dt \ecv{u_C}  + \ecv{u_C}   &= \ecv{U_q}
                \\[-15pt]&\hphantom{R1ooooooo}&&&\hphantom{L3oooooooooooooooooooo}&\hphantom{R3ooo}&&&&\hphantom{R5oooooooooooooooooooooo}\nonumber% Dummy-Zeile A für horizontales Alignment zwischen align-Umgebungen [15pt hoch]
            \end{align*}%
        \textbf{2.) Homogene Lösung, allg.:} (flüchtig) (ohne Störterm)%
        \begin{align*}%
            &\text{DGL}_{h}:&&&
                RC \cdot \dt \ecv{u_{C,\mathrm{h}}} + \ecv{u_{C,\mathrm{h}}} &= 0 &&&
                    &\text{Ansatz:}\quad
                    \ecv{u_{C,\mathrm{h}}} = K \cdot \mathrm{e}^{\lambda t}\\
            &&&&
                K \cdot(RC \cdot \quad\lambda \mathrm{e}^{\lambda t}\ +\ \mathrm{e}^{\lambda t}) &= 0 \\
            &\text{char. Pol.:}&&&
                RC \cdot \lambda + 1 &= 0 &&&
                    &\Rightarrow\quad \lambda = -\frac{1}{\tau} = -\frac{1}{RC}\\
            &\text{hom. Lsg.}:&&&
                \ecv{u_{C,\mathrm{h}}} = K \cdot \mathrm{e}^{-\sfrac{t}{\tau}}&&&&
                    &\text{mit}\quad \tau = RC
        \\[-15pt]&\hphantom{R1ooooooo}&&&\hphantom{L3oooooooooooooooooooo}&\hphantom{R3ooo}&&&&\hphantom{R5oooooooooooooooooooooo}\nonumber% Dummy-Zeile A für horizontales Alignment zwischen align-Umgebungen [15pt hoch]
        \end{align*}%
        \textbf{3.) Partikulare Lösung:} (eingeschwungen) (für $t \to \infty$)%
        \begin{align*}%
            &\text{DGL}_{p}:&&&
                \redcancel{RC \cdot \dt \ecv{u_{C,p}}} + \ecv{u_{C,p}} &= \ecv{U_q} &&&
                    &\text{mit}\quad \ecv{u_{C,p}} = konst.  \\
            &\text{part. Lsg.}:&&&
                \ecv{u_{C,p}} &= \ecv{U_q}
                \\[-15pt]&\hphantom{R1ooooooo}&&&\hphantom{L3oooooooooooooooooooo}&\hphantom{R3ooo}&&&&\hphantom{R5oooooooooooooooooooooo}\nonumber% Dummy-Zeile A für horizontales Alignment zwischen align-Umgebungen [15pt hoch]
            \end{align*}%
        \textbf{4.) Inhomogene Lösung, allg.:} (Überlagerung)%
        \begin{align*}%
            &&&&
                \ecv{u_C}(t) &= \ecv{u_{C,\mathrm{h}}}(t) + \ecv{u_{C,p}}(t) &&&
                &\ecv{u_C}(t)= \ecv{u_{C,f}}(t) + \ecv{u_{C,e}}(t)  \\
            &\text{allg. Lsg.:}&&&
                \ecv{u_C}(t) &= K \cdot \mathrm{e}^{-\sfrac{t}{\tau}} + \ecv{U_q}
            \\[-15pt]&\hphantom{R1oooooooooooo}&&&\hphantom{L3ooo}&\hphantom{R3oooooooooooooooo}&&&&\hphantom{R5oooooooooooooooooooooo}\nonumber% Dummy-Zeile B für horizontales Alignment zwischen align-Umgebungen [15pt hoch]
        \end{align*}%
        \textbf{5.) Konstanten bestimmen:} (Anfangsbedingungen einsetzen)%
        \begin{align*}%
            &\text{Anfangsbed.}&&&
                \ecv{u_C}(0) &= K + \ecv{U_q} \overset{!}{=} \frac{0}{1} &&&
                &\Rightarrow\quad K = -\ecv{U_q} \\
            &&&&
                \ecv{u_C}(t) &= -\ecv{U_q} \cdot \mathrm{e}^{-\sfrac{t}{\tau}} + \ecv{U_q} \\
            &\text{spez. allg. Lsg.:}&&&
                \ecv{u_C}(t) &= \ecv{U_q} \left( 1 - \mathrm{e}^{-\sfrac{t}{\tau}} \right)
            \\[-15pt]&\hphantom{R1oooooooooooo}&&&\hphantom{L3ooo}&\hphantom{R3oooooooooooooooo}&&&&\hphantom{R5oooooooooooooooooooooo}\nonumber% Dummy-Zeile B für horizontales Alignment zwischen align-Umgebungen [15pt hoch]
            \end{align*}%
        Strom $\eci{i}$ aus Spannung $\ecv{u_C}$ bestimmen:%
        \begin{align*}%
            &\text{Strom:}&&&
                \eci{i} &= \frac{\ecv{U_q}}{R} \cdot \mathrm{e}^{-\sfrac{t}{\tau}} &&&
                &\text{mit}\quad \eci{i} = C \cdot \dt \ecv{u_C}
            \\[-15pt]&\hphantom{R1oooooooooooo}&&&\hphantom{L3ooo}&\hphantom{R3oooooooooooooooo}&&&&\hphantom{R5oooooooooooooooooooooo}\nonumber% Dummy-Zeile B für horizontales Alignment zwischen align-Umgebungen [15pt hoch]
        \end{align*}
        \endgroup
    \end{bsp}
}% end skript only

%- - - - - - - - - - - - - - - - - - - - - - - - - - - - - - - - - - - - - - - - - - - - - - - - -%

\s{% RC-Entladen, skript only without frames
    \begin{bsp}{RC-Glied Entladevorgang}{rc:discharge}
        Bestimmung von Spannung $\ecv{u_C}$ und Strom $\eci{i_C}$ einer Kapazität $C$ beim Entladen über einen Widerstand $R$
        für $t \geq 0$. Die Kapazität ist zu Beginn geladen auf $\ecv{U_q}$.\\[-6pt]

        \begin{minipage}{\textwidth}\centering
            \begin{minipage}[t][][t]{0.48\textwidth}\centering\vspace{0cm}% vspace: input empty space before tikzpicture for top alignment
                \includegraphics{Tikz/pdf/circ_switchable_rc_discharge.pdf}%
                \begin{equation*}\begin{aligned}
                    &\text{gegeben:}&       &\ecv{u_C}(0),\ \ecv{U_q},\ R,\ C \\
                    &\text{gesucht:}&       &\ecv{u_C}(t)
                \end{aligned}\end{equation*}
            \end{minipage}
            \begin{minipage}[t][][t]{0.48\textwidth}\centering\vspace{0cm}% vspace: input empty space before tikzpicture for top alignment
                \includegraphics{Tikz/pdf/plot_rc_discharge_uc_i_seperate.pdf}
            \end{minipage}
        \end{minipage}\vspace{5pt}

        \begingroup % keep changes local
            \addtolength{\jot}{2pt} % increase line spacing in align environment
        \textbf{1.) DGL aufstellen:} von $\ecv{u_C}(t)$ für $t>0$, analog zu Beispiel \ref{bsp:rc:chargedc}
            \begin{align*}
            &\text{Masche}:&&&
                \ecv{u_R} + \ecv{u_C}  &= 0&&&
                &\text{mit}\quad \ecv{u_R} = R \cdot \eci{i} \\
            &\text{DGL}:&&&
                RC \cdot \dt \ecv{u_C}  + \ecv{u_C}   &= 0 &&&
                &\text{mit}\quad \eci{i} = C \cdot \dt \ecv{u_C}
            \\[-15pt]&\hphantom{R1ooooooo}&&&\hphantom{L3oooooooooooooooooooo}&\hphantom{R3ooo}&&&&\hphantom{R5oooooooooooooooooooooo}\nonumber% Dummy-Zeile A für horizontales Alignment zwischen align-Umgebungen [15pt hoch]
        \end{align*}%
        \textbf{2.) Homogene Lösung, allg.:} (flüchtig) Identisch zu Beispiel \ref{bsp:rc:chargedc}%
        \begin{align*}%
            &\text{hom. Lsg.}:&&&
                \ecv{u_{C,\mathrm{h}}} = K \cdot \mathrm{e}^{-\sfrac{t}{\tau}}&&&&
                &\text{mit}\quad \tau = RC
            \\[-15pt]&\hphantom{R1ooooooo}&&&\hphantom{L3oooooooooooooooooooo}&\hphantom{R3ooo}&&&&\hphantom{R5oooooooooooooooooooooo}\nonumber% Dummy-Zeile A für horizontales Alignment zwischen align-Umgebungen [15pt hoch]
        \end{align*}%
        Die partiklare Lösung $\ecv{u_{C,p}}$ entfällt, da $\ecv{u_C}$ eingeschwungen ($t \to \infty$) null wird.\\[6pt]%
        \textbf{3. Part. Lsg.:} (eingeschwungen) und \textbf{4. Überlagerung:}%
        \begin{align*}%
            &\text{allg. Lsg.:}&&&
                \ecv{u_C} &= \ecv{u_{C,\mathrm{h}}} + \redcancel{\ecv{u_{C,p}}}&&&
                &\text{mit}\quad \ecv{u_{C,p}}=0
            \\[-15pt]&\hphantom{R1oooooooooooo}&&&\hphantom{L3ooo}&\hphantom{R3oooooooooooooooo}&&&&\hphantom{R5oooooooooooooooooooooo}\nonumber% Dummy-Zeile B für horizontales Alignment zwischen align-Umgebungen [15pt hoch]
        \end{align*}%
        \textbf{5.) Konstanten bestimmen:} (Anfangsbedingungen einsetzen)%
        \begin{align*}%
            &\text{Anfangsbed.}&&&
                \ecv{u_C}(0) &= K \cdot \mathrm{e}^{0} \overset{!}{=} U_q &&&&
                    \Rightarrow\quad K = \ecv{U_q}\\
            &\text{spez. allg. Lsg.:}&&&
                \ecv{u_C}(t) &= \ecv{U_q} \cdot  \mathrm{e}^{-\sfrac{t}{\tau}}
            \\[-15pt]&\hphantom{R1oooooooooooo}&&&\hphantom{L3ooo}&\hphantom{R3oooooooooooooooo}&&&&\hphantom{R5oooooooooooooooooooooo}\nonumber% Dummy-Zeile B für horizontales Alignment zwischen align-Umgebungen [15pt hoch]
            \end{align*}
        Strom $\eci{i}$ aus Spannung $\ecv{u_C}$ bestimmen:%
        \begin{align*}%
            &\text{Strom:}&&&
                \eci{i} &= - \frac{\ecv{U_q}}{R} \cdot \mathrm{e}^{-\sfrac{t}{\tau}} &&&
                &\text{mit}\quad \eci{i} = C \cdot \dt \ecv{u_C}
            \\[-15pt]&\hphantom{R1oooooooooooo}&&&\hphantom{L3ooo}&\hphantom{R3oooooooooooooooo}&&&&\hphantom{R5oooooooooooooooooooooo}\nonumber% Dummy-Zeile B für horizontales Alignment zwischen align-Umgebungen [15pt hoch]
        \end{align*}
        \endgroup
    \end{bsp}
}% end skript only


%-------------------------------------------------------------------------------------------------%

\b{% Anfang Folien only Multi Frame
\begin{frame}[t]\ftx{Ladevorgang \subsubsecname}
% Obere Hälfte
\begin{minipage}{\textwidth}\centering
\begin{minipage}[t][][t]{0.48\textwidth}\centering\vspace{0cm}% vspace: input empty space before tikzpicture for top alignment
    \includegraphics{Tikz/pdf/circ_switchable_rc_charge_dc.pdf}%
    \begin{equation*}\begin{aligned}
        &\text{gegeben:}&       &\ecv{u_C}(0),\ \ecv{U_q},\ R,\ C \\
        &\text{gesucht:}&       &\ecv{u_C}(t)
    \end{aligned}\end{equation*}
\end{minipage}
\begin{minipage}[t][][t]{0.48\textwidth}\centering\vspace{0cm}% vspace: input empty space before tikzpicture for top alignment
    \resizebox{!}{4.25cm}{\includegraphics{Tikz/pdf/plot_rc_charge_uc_i_seperate.pdf}}
\end{minipage}
\end{minipage}
% Untere Hälfte
\resizebox{0.9\textwidth}{!}{% to fit on slide
\begin{minipage}{\textwidth}%
\begin{align*}
&\textbf{1. DGL}&\hspace{15pt}\ecv{U_q} &= RC \cdot \dt \ecv{u_C} + \ecv{u_C} \vphantom{\Big|}\\
&\textbf{2. Homogene Lösung}&\hspace{15pt}\ecv{u_{C,h}} &= K \cdot \mathrm{e}^{-\sfrac{t}{\tau}} \hspace{15pt}\text{mit } \tau = RC \vphantom{\Big|}\\
&\textbf{3. Partikulare Lösung}&\hspace{15pt}\ecv{u_{C,p}} &= \ecv{U_q} \vphantom{\Big|}\\
&\textbf{4. Inhomogene Lösung}&\hspace{15pt}\ecv{u_C} &= K \cdot \mathrm{e}^{-\sfrac{t}{\tau}} + \ecv{U_q} \vphantom{\Big|}\\
&\textbf{5. Konstanten bestimmen}&\hspace{15pt}\ecv{u_C} &= \ecv{U_q} \left( 1 - \mathrm{e}^{-\sfrac{t}{\tau}} \right) \vphantom{\Big|}
\end{align*}
\end{minipage}
}%

\speech{SchaltvorgaengeDC1RCLadevorgang1}{1}{%
Kommen wir zur eigentlichen Berechnung des Ladevorganges beim gerade gezeigten R C Glied. 
Oben links sehen wir die gegebenen Größen, das ist u C unmittelbar vor dem Schalten, die Quellenspannug u Q und die Komponenten R und C. 
Gesucht ist u C während des Schaltvorganges für t größer gleich null. 
Oben rechts sehen wir die Zeitverläufe für u C und darunter für den Strom i. 
**Wichtig**: Die Spannung u C ist stetig, der Strom i kann jedoch springen, was er in diesem Fall auch tut!
Unten sehen wir die 5 Schritte des Berechnungsverfahrens in Kurzform und die dazugehörigen Teillösungen die wir im folgenden Einzeln durch gehen.
}% Speech Ende
\end{frame}

\begin{frame}[t]\ftx{Ladevorgang \subsubsecname}
\begin{minipage}{\textwidth}\centering
\begin{minipage}[t][][t]{0.48\textwidth}\centering\vspace{0cm}% vspace: input empty space before tikzpicture for top alignment
\includegraphics{Tikz/pdf/circ_switchable_rc_charge_dc.pdf}
\end{minipage}
\begin{minipage}[t][][t]{0.48\textwidth}\centering\vspace{0cm}% vspace: input empty space before tikzpicture for top alignment
    \vspace{0.8cm}
    \textbf{DGL 1. Ordnung:}\\[0.85em]
    $a_1 \cdot \dt y(t) + a_0 \cdot y(t)=b$\\
\end{minipage}
\end{minipage}
\vspace{0.75cm}
% Hier Leerzeile für Abstand lassen

\textbf{1.) DGL aufstellen:} für $\ecv{u_C}(t)$ ($t\geq 0$)\\[2pt]
Ausgehend von der Maschengleichung wird $\ecv{u_R}$ in Abhängigkeit von $\ecv{u_C}$ ausgedrückt:\\[-1em]
\begin{align*}
\onslide<2->{% <--- Overlay
&\text{Kirchhoff}:&
    \text{Masche:}\quad\ecv{u_R} + \ecv{u_C}  &= \ecv{U_q}&
        &\text{Knoten:}\quad{\color{red} i_R} = {\color{red} i_C} = {\color{red} i}%
}\\[2pt]%
\onslide<3->{% <--- Overlay
&&
    R \cdot {\color{red} i} + \ecv{u_C}    &= \ecv{U_q}&
        &\text{mit}\quad \ecv{u_R} = R \cdot \eci{i} \vphantom{\dt}
}\\[2pt]%
\onslide<4->{% <--- Overlay
&\text{DGL}:&
    R \cdot C \cdot \dt\, \ecv{u_C}  + \ecv{u_C}   &= \ecv{U_q}&
        &\text{mit}\quad \eci{i} = C \cdot \dt\, \ecv{u_C}
}%
\end{align*}%

\speech{SchaltvorgaengeDC1RCLadevorgang2}{1}{%
Schritt eins ist das Aufstellen der D G L für unsere gesuchte Größe u c von t für den Zeitraum t größer gleich null, 
das heißt ab unmittelbar nach dem Schaltzeitpunkt t gleich null.
Wie im Beispiel in den Grundlagen beginnen wir mit der Maschengleichung:
}% Speech Ende
\speech{SchaltvorgaengeDC1RCLadevorgang2}{2}{%
u R plus u C ist gleich U q.
Dann ersetzen wir u R durch R mal i.
}% Speech Ende
\speech{SchaltvorgaengeDC1RCLadevorgang2}{3}{%
Und ersetzen i durch C mal d nach d t von u c. 
Wenn wir unsere D G L mit der allgemeinen Form oben rechts vergleichen, erkennen wir:
R C entspricht unserem konstanten Koeffizienten a eins. 
u c von t entspricht unserer Systemgröße y von t. 
Und die Quellenspannung U q entspricht der Störgröße b.
}% Speech Ende
\end{frame}


\begin{frame}\ftx{Ladevorgang \subsubsecname}
\textbf{2.) Homogene Lösung, allg.:} (flüchtig) (ohne Störterm)\\[-1em]
\begin{align*}
&\text{DGL}_{h}:&
    RC \cdot \dt \ecv{u_{C,h}} + \ecv{u_{C,h}} &= 0 &
        &\text{Ansatz: }
        \ecv{u_{C,h}} = K \cdot \mathrm{e}^{\lambda t}\\
&& K \cdot(
    RC \cdot \quad\lambda \mathrm{e}^{\lambda t}\ +\ \mathrm{e}^{\lambda t}) &= 0 \\
&\text{char. Polynom:}&\Rightarrow\quad
    RC \cdot \lambda + 1 &= 0
        &&\Rightarrow\quad \lambda = -\frac{1}{RC} \\
&\text{hom. Lsg.}:&
    \ecv{u_{C,h}} = K \cdot \mathrm{e}^{-\sfrac{t}{\tau}}&
        &&\text{mit}\quad \tau = RC
\\[-15pt]&\hphantom{R1oooooooooo}&\hphantom{L2oooooooooooooooooooooo}&\hphantom{R2oooo}&&\hphantom{R3oooooooooooooooooo}% Dummy-Zeile RL-DC für horizontales Alignment zwischen align-Umgebungen [15pt hoch]
\end{align*}
\pause% <--- Overlay
\textbf{3. Partikulare Lösung:} (eingeschwungen) (für $t \to \infty$)\\[-1em]
\begin{align*}
&\text{DGL}_{p}:&
    \redcancel{RC \cdot \dt \ecv{u_{C,p}}} + \ecv{u_{C,p}} &= \ecv{U_q}
        &&\text{mit}\quad \left.\dt\right|_{t \to \infty} = 0  \\
&\text{part. Lsg.}:&
    \ecv{u_{C,p}} &= \ecv{U_q} %
\\[-15pt]&\hphantom{R1oooooooooo}&\hphantom{L2oooooooooooooooooooooo}&\hphantom{R2oooo}&&\hphantom{R3oooooooooooooooooo}% Dummy-Zeile RL-DC für horizontales Alignment zwischen align-Umgebungen [15pt hoch]
\end{align*}% new alignments

\speech{SchaltvorgaengeDC1RCLadevorgang3}{1}{%
Um die flüchtige Lösung zu bestimmen, bestimmen wir als nächsten Schritt die Lösung der homogenen D G L, das heißt für die D G L ohne Störterm. 
Aus der D G L können wir direkt das charakteristische Polynom ableiten: R C mal lambda plus eins.
Die Nullstelle erhalten wir durch Umformung. Lambda ist gleich minus eins durch R C. 
Unsere homogene Lösung entspricht allgemein dem Exponentialansatz K mal e hoch lambda t.
Für eine bessere Lesbarkeit der Eigenschaften schreiben wir den Exponentialansatz in Abhängigkeit der Zeitkonstante auf. 
Unsere homogene Lösung ist K mal e hoch minus t durch Tau. 
Mit lambda ist gleich Minus eins durch Tau erhalten wir durch Einsetzen unserer Nullstelle den Wert für Tau. 
Hier folgt: Tau ist gleich R C. 
}% Speech Ende
\speech{SchaltvorgaengeDC1RCLadevorgang3}{2}{%
Die partikuläre Lösung, das heißt den eingeschwungenen Zustand für t gegen unendlich, erhalten wir entweder durch Netzwerkberechnung oder leiten ihn aus der D G L ab.
Für beide Herangehensweisen gilt, dass die eingeschwungenen Größen Gleichgrößen sind, weil die Anregung eine Gleichspannung ist. (D C Fall) 
Aus der Gleichstrom Netzwerkberechnung wissen wir, dass die Kapazität wie eine offene Klemme wirkt. Es fließt kein Strom und die gesamte Spannung U q liegt über C an.
Alternativ können wir wie hier gezeigt ist die D G L für die partikläre Lösung herannehmen.
Da es sich um Gleichgrößen handelt, ist die zeitliche Ableitung von u c p gleich Null und der erste Teil der D G L verschwindet.
Übrig bleibt, dass u C p gleich der Anregung U q ist. 
}% Speech Ende
\end{frame}


\begin{frame}\ftx{Ladevorgang \subsubsecname}
\textbf{4.) Inhomogene Lösung, allg.:} (Überlagerung)\\[-1em]
\begin{align*}
&&
    \ecv{u_C}(t) &= \ecv{u_{C,h}}(t) + \ecv{u_{C,p}}(t) &
    &= \ecv{u_{C,f}}(t) + \ecv{u_{C,e}}(t)  \\
&\text{allg. Lsg.:}&
    \ecv{u_C}(t) &= K \cdot \mathrm{e}^{-\sfrac{t}{\tau}} + \ecv{U_q}%
\\[-15pt]&\hphantom{R1oooooooooo}&\hphantom{L2oooooooo}&\hphantom{R2oooooooooooooooooo}&&\hphantom{R3oooooooooooooooooo}% Dummy-Zeile RL-DC für horizontales Alignment zwischen align-Umgebungen [15pt hoch]
\end{align*}
\pause% <--- Overlay
\textbf{5. Konstanten bestimmen:} (Anfangsbedingung einsetzen)\\[-1em]
\begin{align*}
&\text{Anfangsbed.}&
    \ecv{u_C}(0) &= K + \ecv{U_q} \overset{!}{=} 0 &
        &\Rightarrow\quad K = -\ecv{U_q} && \\[2pt]
&&
    \ecv{u_C}(t) &= -\ecv{U_q} \cdot \mathrm{e}^{-\sfrac{t}{\tau}} + \ecv{U_q} \\[2pt]
&\text{spez. allg. Lsg.:}&
    \ecv{u_C}(t) &= \ecv{U_q} \left( 1 - \mathrm{e}^{-\sfrac{t}{\tau}} \right) \\[2pt]
&\text{Strom: }&
    \eci{i_C}(t) &= \frac{\ecv{U_q}}{R} \cdot \mathrm{e}^{-\sfrac{t}{\tau}} &
    &\text{mit}\quad \eci{i_C} = C \cdot \dt \ecv{u_C}
\\[-15pt]&\hphantom{R1oooooooooo}&\hphantom{L2oooooooo}&\hphantom{R2oooooooooooooooooo}&&\hphantom{R3oooooooooooooooooo}% Dummy-Zeile RL-DC für horizontales Alignment zwischen align-Umgebungen [15pt hoch]
\end{align*}%

\speech{SchaltvorgaengeDC1RCLadevorgang4}{1}{%
Die allgemeine Lösung der inhomogenen D G L erhalten wir durch Überlagerung unserer homogenen und unserer partiklären Lösung.
U c ist gleich u c h plus u c p. Oder alternativ ist gleich u c f plus u c e. 
Eingesetzt ergibt sich daraus die allgemein Lösung: u c von t ist gleich K mal e hoch minus t durch Tau plus U q. 
}% Speech Ende
\speech{SchaltvorgaengeDC1RCLadevorgang4}{2}{%
Die Konstanten bestimmen wir durch einsetzen der Anfangsbedingung. Die Kapazität sei zum Zeitpunkt t gleich Null unmittelbar vor dem Schalten vollständig entladen. 
Der Exponentialterm entfällt für t gleich null.
Durch Umformung erhalten wir K ist gleich Minus U q.
Durch Einsetzen von K und Ausklammern von U q erhalten wir die spezielle allgemeine Lösung:
u c von t ist gleich U q mal Klammer auf eins minus e hoch minus t durch lambda Klammer zu. 
Den Strom können wir aus der Beziehung i gleich C mal d nach d t von u C an der Kapazität herleiten. 
Hier mit tau gleich R C eingesetzt.
Durch kürzen erhalten wir den Stromverlauf i c von t ist gleich U q durch R mal e hoch minus t durch Tau. 
Der maximale Anfangsstrom i C null ist also U q durch R. 
}% Speech Ende
\end{frame}
}% Ende Folien only Multi Frame

\v{% Video only Frame
\begin{frame}\ftx{Entladevorgang \subsubsecname}
    \begin{Lernziele}{Video - Schaltvorgänge bei Gleichspannung [2/2]}
        Sie lernen in diesem Video:
        \begin{itemize}\b{\setlength\itemsep{0.7em}}
            \item den Entladevorgang eines RC-Gliedes zu berechnen,
            \item den DC Lade- und Entladevorgang eines RC-Gliedes im Vergleich kennen,
            \item den DC Lade- und Entladevorgang eines RL-Gliedes zu berechnen und im Vergleich zu RC-Gliedern kennen.
        \end{itemize}
    \end{Lernziele}
\end{frame}%

\speech{SchaltvorgaengeDC2RCEntladevorgangLernziele}{1}{%
In diesem Video setzen wir das Beispiel aus dem letzten Video zu Schaltvorgängen bei Gleichspannung fort.
Ausgehend von der selben Schaltung bestimmen wir nun den Entladevorgang des R C Seriengliedes über einen Kurzschluss.
}% Speech Ende
}% Video only Frame Ende

\b{% Anfang Folien Multi Frame
\begin{frame}[t]\ftx{Entladevorgang \subsubsecname}
% Obere Hälfte
\begin{minipage}{\textwidth}\centering
\begin{minipage}[t][][t]{0.48\textwidth}\centering\vspace{0cm}% vspace: input empty space before tikzpicture for top alignment
    \includegraphics{Tikz/pdf/circ_switchable_rc_discharge.pdf}%
    \begin{equation*}\begin{aligned}
        &\text{gegeben:}&       &\ecv{u_C}(0),\ \ecv{U_q},\ R,\ C \\
        &\text{gesucht:}&       &\ecv{u_C}(t)
    \end{aligned}\end{equation*}
\end{minipage}
\begin{minipage}[t][][t]{0.48\textwidth}\centering\vspace{0cm}% vspace: input empty space before tikzpicture for top alignment
    \resizebox{!}{4.25cm}{\includegraphics{Tikz/pdf/plot_rc_discharge_uc_i_seperate.pdf}}
\end{minipage}
\end{minipage}
% Untere Hälfte
\resizebox{0.9\textwidth}{!}{% to fit on slide
\begin{minipage}{\textwidth}%
\begin{align*}
&\textbf{1. DGL}&\hspace{15pt} 0 &= RC \cdot \dt \ecv{u_C} + \ecv{u_C}\\
&\textbf{2. Homogene Lösung}&\hspace{15pt}\ecv{u_{C,h}} &= K \cdot \mathrm{e}^{-\sfrac{t}{\tau}} \hspace{15pt}\text{mit } \tau = RC\\[0.5em]
&\textbf{3. Partikulare Lösung}&\hspace{15pt}\ecv{u_{C,p}} &= 0\\[0.35em]
&\textbf{4. Inhomogene Lösung}&\hspace{15pt}\ecv{u_C} &= K \cdot \mathrm{e}^{-\sfrac{t}{\tau}}\\[0.3em]
&\textbf{5. Konstanten bestimmen}&\hspace{15pt}\ecv{u_C} &= \ecv{U_q} \cdot\mathrm{e}^{-\sfrac{t}{\tau}}
\end{align*}
\end{minipage}
}% Berechnung analog zu Beispiel \ref{bsp:rc:chargedc}.

\speech{SchaltvorgaengeDC2RCEntladevorgang1}{1}{%
Das Vorgehen ist identisch zum Vorgehen beim Ladevorgang. 
Wir betrachten die gleiche Schaltung, hier oben links zu sehen, mit dem Unterschied, 
dass bei t gleich null von der Spannungsquelle zum Kurzschluss geschalten wird.
Gesucht ist der Zeitverlauf für u C. Alle Konstanten und die Anfangsbedingung sind gegeben.
Oben rechts sehen wir bereits die Zeitverläufe für u C und für i.
U C ist zu Beginn vollständig geladen und fällt dann exponentiell auf null ab. 
Der Strom beginnt bei null, macht dann einen Sprung auf Minus U q durch R und nähert sich von dann exponentiell der Null.
Der Strom ist negativ im Verbraucherzählpfeilsystem, da die Kapazität beim Entladen als Quelle dient.
Hier immer auf die korrekten Vorzeichen achten. 
Unten sind die fünf Schritte des Berechnungsverfahrens mit den jeweiligen Teillösungen die wir nun einzeln durchgehen. 
}% Speech Ende
\end{frame}


\begin{frame}[t]\ftx{Entladevorgang \subsubsecname}
\begin{minipage}{\textwidth}\centering
\begin{minipage}[t][][t]{0.48\textwidth}\centering\vspace{0cm}% vspace: input empty space before tikzpicture for top alignment
    \includegraphics{Tikz/pdf/circ_switchable_rc_discharge.pdf}%
\end{minipage}
\begin{minipage}[t][][t]{0.48\textwidth}\centering\vspace{0cm}% vspace: input empty space before tikzpicture for top alignment
    \vspace{0.8cm}
    \textbf{DGL 1. Ordnung:}\\[0.85em]
    $a_1 \cdot \dt y(t) + a_0 \cdot y(t)=b$\\
\end{minipage}
\end{minipage}%
\vspace{0.75cm}
% Hier Leerzeile für Abstand lassen

\textbf{1.) DGL aufstellen:} für $\ecv{u_C}(t)$ ($t\geq 0$)\\[-1em]
\begin{align*}
% Comment out \intertext rows to enable math-preview in VSCode (or set \textbf to \text)
\intertext{Ausgehend von der Maschengleichung wird $\ecv{u_R}$ in Abhängigkeit von $\ecv{u_C}$ ausgedrückt:}\\[-1em]
&\text{Kirchhoff}:
&\text{Masche:}\quad\ecv{u_R} + \ecv{u_C}  &= 0
&&\text{Knoten:}\quad{\color{red} i_R} = {\color{red} i_C} = {\color{red} i}\\
&&
    R \cdot {\color{red} i} + \ecv{u_C}    &= 0
        &&\text{mit}\quad
        \ecv{u_R} = R \cdot \eci{i_R} = R \cdot \eci{i}\\
&\text{DGL}:&
    RC \cdot \dt \ecv{u_C}  + \ecv{u_C}   &= 0
        &&\text{mit}\quad
        \eci{i} = \eci{i_C} = C \cdot \dt \ecv{u_C}
\end{align*}%

\speech{SchaltvorgaengeDC2RCEntladevorgang2}{1}{%
Die D G L wird im gleichen Verfahren aufgestellt wie beim Ladevorgang. 
Da die Maschie nicht mehr durch die Spannungsquelle, sondern durch den Kurzschluss geschlossen wird, entfällt allerdings der Störterm. 
Dadurch erhalten wir die homogene D G L aus dem Ladevorgang.
}% Speech Ende
\end{frame}


\begin{frame}\ftx{Entladevorgang \subsubsecname}
\textbf{2.) Homogene Lösung, allg.:} (flüchtig)\\[-1.5em]
\begin{align*}
&\text{DGL}_{h}:&
    RC \cdot \dt \ecv{u_{C,h}} + \ecv{u_{C,h}} &= 0 &
        &\text{Ansatz: }\ecv{u_{C,h}} = K \cdot \mathrm{e}^{\lambda t}\\
&\text{char. Polynom:}&\Rightarrow\quad
    RC \cdot \lambda + 1 &= 0 &
        &\Rightarrow\quad \lambda = -\frac{1}{RC}\\
&\text{hom. Lsg.}:&
    \ecv{u_{C,h}} &= K \cdot \mathrm{e}^{-\sfrac{t}{\tau}}&
        &\text{mit}\quad \tau = RC
\intertext{\textbf{3.) Part. Lsg., 4.) Überlagerung:} (eingeschwungen Null)}\\[-1.5em]
&\text{allg. Lsg.:}&
    \ecv{u_C} &= \ecv{u_{C,h}} + \redcancel{\ecv{u_{C,p}}}&
        &\text{mit}\quad \ecv{u_{C,p}=0}
\intertext{\textbf{5.) Konstanten bestimmen:} (AB einsetzen)}\\[-1.5em]
&\text{Anfangsbed.}&
    \ecv{u_C}(0) &= K \cdot \mathrm{e}^0 \overset{!}{=} \ecv{U_q}
        &&\Rightarrow\quad K = \ecv{U_q} &&\\
&\text{spez. allg. Lsg.:}&
    \ecv{u_C}(t) &= \ecv{U_q} \cdot  \mathrm{e}^{-\sfrac{t}{\tau}}\\
&\text{Strom: }&
    \eci{i_C}(t) &= - \frac{\ecv{U_q}}{R} \cdot \mathrm{e}^{-\sfrac{t}{\tau}}&
    &\text{mit}\quad \eci{i_C} = C \cdot \dt \ecv{u_C}
\end{align*}% new alignments

\speech{SchaltvorgaengeDC2RCEntladevorgang3}{1}{%
In Schritt drei können wir die homogene Lösung daher exakt aus der Lösung für den Ladevorgang übernehmen. 
Schritt vier entfällt, da die partikuläre Lösung gleich Null ist.
Unsere allgemeine Lösung entspricht also der homogenen Lösung.
In Schritt fünf bestimmen wir die Konstanten durch die Anfangsbedingung.
Die Kapazität sei zu Beginn vollständig geladen, daher gilt u C von null gleich Uq.
Daraus folgt K ist gleich U q.
Die speziefische allgemeine Lösung lautet damit uc von t ist gleich U q mal e hoch minus t durch Tau. 
Den Strom können wir über die Komponentengeich für C direkt aus u C bestimmen. 
Daraus ergibt sich für den Strom i C von t ist gleich Minus U q durch R mal e hoch minus t durch Tau.
}% Speech Ende
\end{frame}

\begin{frame}[t]\ftx{Vergleich Lade- und Entladevorgang \subsubsecname}
\begin{minipage}{\textwidth}\centering%
% Obere Hälfte
\begin{minipage}[t][][t]{0.48\textwidth}\centering\vspace{0cm}% vspace: input empty space before tikzpicture for top alignment%
    \includegraphics{Tikz/pdf/circ_switchable_rc_charge_dc.pdf}\\[1em]% Leerzeile
    {\flushleft% left alignment
    Rechenwege prinzipiell gleich für (DC) \\ Ladevorgang und Entladevorgang.%
    }%
\end{minipage}
\begin{minipage}[t][][t]{0.48\textwidth}\centering\vspace{0cm}% vspace: input empty space before tikzpicture for top alignment
    \resizebox{0.6\textwidth}{!}{\includegraphics{Tikz/pdf/plot_rc_cycle_seperate_u.pdf}}%
\end{minipage}
\end{minipage}\hfill%
% Untere Hälfte
\begin{minipage}{\textwidth}\centering%
Spannung $\ecv{u_C}$ nach Schaltzeitpunkt $t_0$ [Vgl. Skript]:
\begin{align*}
    %\ecv{u_C}(t-t_0) &= \underbrace{\ecv{u_{C,\mathrm{e}}} \vphantom{\Big|}}_{\ecv{u_{C,\mathrm{p}}}} + \underbrace{ \big(\ecv{u_C}(t_0)-\ecv{u_{C,\mathrm e}}\big) \cdot  \mathrm{e}^{-\frac{t-t_0}{\tau}} \vphantom{\Big|}}_{\ecv{u_{C,\mathrm{h}}}}\\
    \ecv{u_C}(t-t_0) &= \underbrace{\ecv{u_C}(t_0) \vphantom{\Big|}}_{\text{Startwert}} + \big(\underbrace{\ecv{u_{C,\mathrm{e}}} - \ecv{u_C}(t_0) \vphantom{\Big|}}_{\text{Differenz}}\big) \cdot \bigg( \underbrace{ 1- \mathrm{e}^{-\frac{t-t_0}{\tau}} \vphantom{\Big|}}_{\text{\clap{Exp. Annäherung}}} \bigg)
\end{align*}
Unterschiede nur bei Störglied der DGL, Anfangsbedingung und Endwert.
\end{minipage}

\speech{SchaltvorgaengeDC2RCVergleich}{1}{%
Wenn wir den gerade betrachteten Lade- und Entladevorgang vergleichen, ergeben sich in beiden Fällen ähnliche Lösungen.
In beiden Fällen sind die Verläufe exponentialförmig mit der gleichen Zeitkonstante Tau ist gleich R C. 
Wenn wir die Anfangsbedingung allgemein als u c zum Schaltzeitpunkt t null formulieren. 
Und den eingeschwungenen Zustand mit allgemein als u c e bezeichnen,
können wir die Lösung für den Lade und den Entladevorgang der gezeigten Schaltung allgemein für einen Schaltzeitpunkt t null bestimmen.
Durch etwas Umformung können wir die Lösung in die folgende Form bringen:
In Worten: Die Spannung an der Kapazität ab Schaltzeitpunkt t null ist gleich der Startspannung plus einer exponentiellen Annäherung der Differenz von Endwert und Startwert.
Mathematisch ausgedrückt: u c von t minus t0 **ist gleich** u c von t null, der Startwert, **plus** Klammer auf, die Differenz u c e **minus** u c von t null, Klammer zu, **mal** klammer auf Eins minus Exponent minus t minus t null durch Tau Exponent Ende, die exponentielle Annäherung, Klammer zu. 
Die Rechenwege zur Berechnung von Lade- und Entladevorgängen bei D C Fällen sind prinzipiell **identisch**
und unterscheiden sich in diesem Beispiel nur beim Störterm (der Anregung), den Anfangsbedingungen und den eingeschwungenen Zuständen (den Endwerten).
}% Speech Ende
\end{frame}
}% Ende Multi Frame Folien only


%-------------------------------------------------------------------------------------------------%

\subsubsection{bei Induktivitäten}
\label{sec:schaltvorgaengezeitbereich:dc:l}
\begin{frame}[t]\ftx{Vergleich Lade- und Entladevorgang \subsubsecname}

\s{% RL-Laden, skript only without frames
    Der Rechenweg zur Bestimmung des Strom- oder Spannungsverlaufs bei Induktivitäten
    bei Anregung mit Gleichspannung ist identisch zu der Berechnung bei Kapazitäten in Kapitel \ref{sec:schaltvorgaengezeitbereich:dc:c}.
    Das sich einstellende Zeitverhalten mit Exponentialverläufen und Zeitkonstanten ist analog zu dem dort beschriebenen.

    Exemplarisch ist in Beispiel \ref{bsp:rl:chargedc:dgli} und \ref{bsp:rl:chargedc:dglu}
    der Ladestrom $i(t)$ und die Induktivitätsspannung $u_L(t)$ einer Induktivität $L$ beim Laden
    über einen Widerstand $R$ mit Gleichspannung $U_q$ für $t \geq 0$ berechnet und dargestellt.

    Die Beispiele unterscheiden sich lediglich im Rechenweg. In Beispiel \ref{bsp:rl:chargedc:dgli}
    wird die DGL für $i(t)$ und in Beispiel \ref{bsp:rl:chargedc:dglu} für $u_L(t)$ aufgestellt und gelöst.

\newpage
\begin{bsp}{RL-Glied DC-Ladevorgang - Über DGL von $i$}{rl:chargedc:dgli}
    Bestimmung von Ladestrom $\eci{i}$ und Spannung $\ecv{u_L}$ der Induktivität $L$ beim Laden über Widerstand $R$
    mit Gleichspannung $\ecv{U_q}$ ab Schalzeitpunkt $t=0$. Zuvor fließt kein Strom.\\[-6pt]

    \begin{minipage}{\textwidth}\centering
        \begin{minipage}[t][][t]{0.48\textwidth}\centering\vspace{0cm}% vspace: input empty space before tikzpicture for top alignment
            \includegraphics{Tikz/pdf/circ_switchable_rl_charge_dc.pdf}%
            \begin{equation*}\begin{aligned}
            \text{gegeben:}&&       &\eci{i}(0),\ \ecv{U_q},\ R,\ L \\
            \text{gesucht:}&&       &\ecv{u_L}(t),\ \eci{i}(t)
            \end{aligned}\end{equation*}
        \end{minipage}
        \begin{minipage}[t][][t]{0.48\textwidth}\centering\vspace{0cm}% vspace: input empty space before tikzpicture for top alignment
            \includegraphics{Tikz/pdf/plot_rl_charge_ul_i_seperate.pdf}
        \end{minipage}
    \end{minipage}

    \textbf{1.) DGL aufstellen:} von $\eci{i}(t)$ für $t \geq 0$\vspace{5pt}

    Ausgehend von der Maschengleichung wird $\ecv{u_R}$ und $\ecv{u_L}$ nach $\eci{i}$ umgeschrieben:%
    \begingroup% Änderungen lokal vornehmen
        \addtolength{\jot}{2pt}% Zeilenabstand in align-Umgebung erhöhen
    \begin{align*}
        &\text{Masche}:&&&
            \ecv{u_R} + \ecv{u_L}  &= \ecv{U_q}&&&
            &\text{mit}\quad \ecv{u_R} = R \cdot \eci{i} \\
        &&&&
            R \cdot \eci{i} + L \cdot \dt \eci{i} &= \ecv{U_q} &&&
            &\text{mit}\quad \ecv{u_L} = L \cdot \dt \eci{i} \\
        &\text{DGL}:&&&
        \frac{L}{R} \cdot \dt \eci{i}  + \eci{i}  &= \frac{\ecv{U_q}}{R}
        \\[-15pt]&\hphantom{R1oooooooooo}&&&\hphantom{L3ooooooooooo}&\hphantom{R3ooooooooooooooo}&&&&\hphantom{R5oooooooooooooooooo}\nonumber% Dummy-Zeile RL-DC für horizontales Alignment zwischen align-Umgebungen [15pt hoch]
    \end{align*}%
    \textbf{2.) Homogene Lösung, allg.:} (flüchtig) (ohne Störterm)%
    \begin{align*}%
        &\text{DGL}_{h}:&&&
            \frac{L}{R} \cdot \dt \eci{i_{\mathrm h}} + \eci{i_{\mathrm h}} &= 0 &&&
                &\text{Ansatz:}\quad
                \eci{i_{\mathrm h}} = K \cdot \mathrm{e}^{\lambda t}\\
        &\text{char. Pol.:}&&&
            \frac{L}{R} \cdot \lambda + 1 &= 0 &&&
                &\Rightarrow\quad \lambda = -\frac{1}{\tau} = -\frac{R}{L}\\
        &\text{hom. Lsg.}:&&&
            \eci{i_{\mathrm h}} &= K \cdot \mathrm{e}^{-\sfrac{t}{\tau}}&&&
                &\text{mit}\quad \tau = \frac{L}{R}
        \\[-15pt]&\hphantom{R1oooooooooo}&&&\hphantom{L3ooooooooooo}&\hphantom{R3ooooooooooooooo}&&&&\hphantom{R5oooooooooooooooooo}\nonumber% Dummy-Zeile RL-DC für horizontales Alignment zwischen align-Umgebungen [15pt hoch]
    \end{align*}%
    \textbf{3.) Partikulare Lösung:} (eingeschwungen) mit $t \to \infty$%
    \begin{align*}%
        &\text{DGL}_{p}:&&&
            \redcancel{\frac{L}{R} \cdot \dt \eci{i_{\mathrm p}}} + \eci{i_{\mathrm p}} &= \frac{\ecv{U_q}}{R} &&&
                &\text{mit}\quad \eci{i_{\mathrm p}} = konst.  \\
        &\text{part. Lsg.}:&&&
            \eci{i_{\mathrm p}} &= \frac{\ecv{U_q}}{R}
        \\[-15pt]&\hphantom{R1oooooooooo}&&&\hphantom{L3ooooooooooo}&\hphantom{R3ooooooooooooooo}&&&&\hphantom{R5oooooooooooooooooo}\nonumber% Dummy-Zeile RL-DC für horizontales Alignment zwischen align-Umgebungen [15pt hoch]
    \end{align*}%
    \textbf{4.) Inhomogene Lösung, allg.:} (Überlagerung)%
    \begin{align*}%
        &\text{allg. Lsg.:}&&&
            \eci{i}(t) &=
            \underbrace{%
                K \cdot \mathrm{e}^{-\sfrac{t}{\tau}} \vphantom{\frac{\ecv{U_q}}{R}}%
            }_{\eci{i_{\mathrm h}}}
            +
            \underbrace{
                \frac{\ecv{U_q}}{R}
            }_{\eci{i_{\mathrm p}}} &&&
            &\text{mit}\quad \eci{i_{\mathrm h}}=\eci{i_f},\quad \eci{i_{\mathrm p}}=\eci{i_e}
        \\[-15pt]&\hphantom{R1oooooooooo}&&&\hphantom{L3ooooooooooo}&\hphantom{R3ooooooooooooooo}&&&&\hphantom{R5oooooooooooooooooo}\nonumber% Dummy-Zeile RL-DC für horizontales Alignment zwischen align-Umgebungen [15pt hoch]
        \end{align*}%
    \textbf{5.) Konstanten bestimmen:} (Anfangsbedingungen einsetzen)%
    \begin{align*}%
        &\text{Anfangsbed.}&&&
            \eci{i}(0) &= K \cdot \redcancel{\mathrm{e}^0} + \frac{\ecv{U_q}}{R} \overset{!}{=} 0 &&&
            &\Rightarrow\quad K = -\frac{\ecv{U_q}}{R}\\
        &\text{eindeut. Lsg.:}&&&
            \eci{i}(t) &= \frac{\ecv{U_q}}{R} \cdot \left( 1 - \mathrm{e}^{-\sfrac{t}{\tau}} \right)
        \\[-15pt]&\hphantom{R1oooooooooo}&&&\hphantom{L3ooooooooooo}&\hphantom{R3ooooooooooooooo}&&&&\hphantom{R5oooooooooooooooooo}\nonumber% Dummy-Zeile RL-DC für horizontales Alignment zwischen align-Umgebungen [15pt hoch]
    \end{align*}%
    Spannung $\ecv{u_L}$ aus Strom $\eci{i}$ bestimmen:%
    \begin{align*}%
        %&&&&
        %    \ecv{u_L} &= \frac{\ecv{U_q}}{R} \cdot L \cdot \frac{R}{L} \cdot \mathrm{e}^{-\sfrac{t}{\tau}} &&&
        %    &\text{mit}\quad \ecv{u_L} = L \cdot \dt \eci{i} \\
        &\text{Spannung:}&&&
            \ecv{u_L} &= \ecv{U_q} \cdot \mathrm{e}^{-\sfrac{t}{\tau}} &&&
            &\text{mit}\quad \ecv{u_L} = L \cdot \dt \eci{i}
        \\[-15pt]&\hphantom{R1oooooooooo}&&&\hphantom{L3ooooooooooo}&\hphantom{R3ooooooooooooooo}&&&&\hphantom{R5oooooooooooooooooo}\nonumber% Dummy-Zeile RL-DC für horizontales Alignment zwischen align-Umgebungen [15pt hoch]
    \end{align*}
    \endgroup
\end{bsp}
}% end skript only

\b{% Anfang Folien only (bei Induktivitäten)
% Obere Hälfte
\begin{minipage}{\textwidth}\centering%
\begin{minipage}[t][][t]{0.48\textwidth}\centering\vspace{0cm}% vspace: input empty space before tikzpicture for top alignment%
    \includegraphics{Tikz/pdf/circ_switchable_rl_charge_dc.pdf}\\[1em]% Leerzeile
    {\flushleft% left alignment
    Rechenweg bei RL-Glied prinzipiell gleich wie bei RC-Glied. (DGL 1. Ordnung) \\
    Achtung: $\eci{i}$ stetig, $\ecv{u_L}$ sprunghaft.%
    }%
\end{minipage}
\begin{minipage}[t][][t]{0.48\textwidth}\centering\vspace{0cm}% vspace: input empty space before tikzpicture for top alignment
    \resizebox{0.6\textwidth}{!}{\includegraphics{Tikz/pdf/plot_rl_cycle_seperate_i.pdf}}%
\end{minipage}
\end{minipage}\hfill%
% Untere Hälfte
\begin{minipage}{\textwidth}%
\textbf{Vermutung:} Strom u. Spannung \glqq vertauscht\grqq gegenüber RC-Glied.
\begin{align*}
    \eci{i}(t-t_0) &= \underbrace{\eci{i}(t_0) \vphantom{\Big|}}_{\text{Startwert}} + \big(\underbrace{\eci{i_{\mathrm{e}}} - \eci{i}(t_0) \vphantom{\Big|}}_{\text{Differenz}}\big) \cdot \bigg( \underbrace{ 1- \mathrm{e}^{-\frac{t-t_0}{\tau}} \vphantom{\Big|}}_{\text{\clap{Exp. Annäherung}}} \bigg)
\end{align*}
\textbf{Laden:} $\eci{i}$ zunehmend, $\ecv{u_L}$ abnehmend. \quad \textbf{Entladen:} vice versa.
\end{minipage}
}% Ende Folien only

\speech{SchaltvorgaengeDC2RLVergleich}{1}{%
Kommen wir nun zum Lade- und Entladevorgang eines R L Seriengliedes bei Gleichspannung. 
Die Schaltung ist identisch zu den Beispielen mit dem R C Glied aufgebaut, nur dass die Kapazität mit einer Induktivität ersetzt wurde.
Da es sich zum eine Schaltung mit nur einem Energiespeicher handelt, erhalten wir wieder eine D G L erster Ordnung. 
Im Gegensatz zur Kapazität ist hier allerdings i stetig und u L sprunghaft, da die Energie im Magnetfeld der Induktivität gespeicher ist. 
Als Vermutung können wir aufstellen, dass die Zeitverläufe prinzipiell ähnlich aussehen wie bei den D C Schaltvorgängen beim R C Glied.
Sie können das Video kurz pausieren und überlegen, 
ob die hier aufgestellte Gleichung für den Strom plausibel ist und
ob die Zeitverläufe oben rechts für den Strom i und die Spannung u L plausibel ist. 
Überlegen Sie sich, welche eingeschwungenen Zustände jeweils angestrebt werden. 
Wie könnt die Gleichung für die Zeitkonstante lauten? 
Zur Überprüfung bestimmen wir im folgenden detailliert den Ladevorgang und danach kurz den Entladevorgang der gezeigten Induktivität.
}% end speech
\end{frame}

%-------------------------------------------------------------------------------------------------%

\b{% Anfang Folien only Multi Frame -- bei Induktivitäten, DGL über i
\begin{frame}[t]\ftx{Ladevorgang \subsubsecname}
% Obere Hälfte
\begin{minipage}{\textwidth}\centering%
\begin{minipage}[t][][t]{0.48\textwidth}\centering\vspace{0cm}% vspace: input empty space before tikzpicture for top alignment
    \includegraphics{Tikz/pdf/circ_switchable_rl_charge_dc.pdf}%
    \begin{equation*}\begin{aligned}
        &\text{gegeben:}&       &\eci{i}(0),\ \ecv{U_q},\ R,\ L \\
        &\text{gesucht:}&       &\eci{i}(t)
    \end{aligned}\end{equation*}
\end{minipage}
\begin{minipage}[t][][t]{0.48\textwidth}\centering\vspace{0cm}% vspace: input empty space before tikzpicture for top alignment
    \resizebox{!}{4.00cm}{\includegraphics{Tikz/pdf/plot_rl_charge_ul_i_seperate.pdf}}
\end{minipage}
\end{minipage}
% Untere Hälfte
\resizebox{0.9\textwidth}{!}{% to fit on slide
\begin{minipage}{\textwidth}%
\begin{align*}
&\textbf{1. DGL}&\hspace{15pt}\frac{\ecv{U_q}}{R} &= \frac{L}{R} \cdot \dt\, \eci{i} + \eci{i}\\[-2pt]
&\textbf{2. Homogene Lösung}&\hspace{15pt}\eci{i_{\mathrm{h}}} &= K \cdot \mathrm{e}^{-\sfrac{t}{\tau}} \quad\text{mit}\quad \tau = \frac{L}{R}\\[-2pt]
&\textbf{3. Partikulare Lösung}&\hspace{15pt}\eci{i_{\mathrm{p}}} &= \frac{\ecv{U_q}}{R}\\[-2pt]
&\textbf{4. Inhomogene Lösung}&\hspace{15pt}\eci{i} &= K \cdot \mathrm{e}^{-\sfrac{t}{\tau}} + \frac{\ecv{U_q}}{R}\\[-2pt]
&\textbf{5. Konstanten bestimmen}&\hspace{15pt}\eci{i} &= \frac{\ecv{U_q}}{R} \left( 1 - \mathrm{e}^{-\sfrac{t}{\tau}} \right)
\end{align*}
\end{minipage}
}%

\speech{SchaltvorgaengeDC2RLLadevorgang1}{1}{%
Beginnen wir beim Ladevorgang. Oben links sehen Sie das Schaltbild.
Zu Beginn ist das R L Glied kurzgeschlossen und ab t gleich null ist das R L Glied an die Spannungsquelle U q geschlossen.
Zu Beginn fließt kein Strom, U q, R und L sind bekannt, gesucht ist der Strom i für t größer gleich null.
Oben rechts sehen Sie die Zeitverläufe für u L und i. U l springt bei t gleich null von null auf U q.
Weil i stetig ist, fließt zu Beginn kein Strom und die volle Quellenspannung liegt über L an.
U L fällt dann exponentiell auf Null ab.
Der Strom nähert sich ab t gleich null exponentiell seinem Endwert U q durch R, da der Widerstand R den maximalen Strom begrenzt.
Unten sind die fünf Rechenschritte mit Teillösungen zusammen gefasst, welche wir im Folgenden einzeln durch gehen.
}% Speech Ende
\end{frame}

\begin{frame}[t]\ftx{Ladevorgang \subsubsecname}
\begin{minipage}{\textwidth}\centering
\begin{minipage}[t][][t]{0.48\textwidth}\centering\vspace{0cm}% vspace: input empty space before tikzpicture for top alignment
\includegraphics{Tikz/pdf/circ_switchable_rl_charge_dc.pdf}
\end{minipage}
\begin{minipage}[t][][t]{0.48\textwidth}\centering\vspace{0cm}% vspace: input empty space before tikzpicture for top alignment
    \vspace{0.8cm}
    \textbf{DGL 1. Ordnung:}\\[0.85em]
    $a_1 \cdot \dt y(t) + a_0 \cdot y(t)=b$\\
\end{minipage}
\end{minipage}
\vspace{0.75cm}
% Hier Leerzeile für Abstand lassen

\textbf{1. DGL aufstellen:} für $\eci{i}(t)$ ($t\geq 0$)\\[-1em]
\begin{align*}
% Comment out \intertext rows to enable math-preview in VSCode (or set \textbf to \text)
%\intertext{Ausgehend von der Maschengleichung wird $\ecv{u_R}$ und $\ecv{u_L}$ nach $\eci{i}$ umgeschrieben.}\\[-1em]
&\text{Masche}:&
    \ecv{u_R} + \ecv{u_L}  &= \ecv{U_q}&
        &\text{mit}\quad \ecv{u_R} = R \cdot \eci{i} \\
&&
    R \cdot \eci{i} + L \cdot \dt \eci{i} &= \ecv{U_q} &
        &\text{mit}\quad \ecv{u_L} = L \cdot \dt \eci{i} \\
&&
    L \cdot \dt \eci{i} + R \cdot \eci{i} &= \ecv{U_q} &
        &\text{mit}\quad \ecv{u_L} = L \cdot \dt \eci{i} \\
&\text{DGL}:&
    \frac{L}{R} \cdot \dt \eci{i}  + \eci{i}  &= \frac{\ecv{U_q}}{R}
\end{align*}%

\speech{SchaltvorgaengeDC2RLLadevorgang2}{1}{%
Die D G L stellen wir am Besten ausgehend von der Maschengleichung auf.
u R und u L ist gleich U q. 
Durch Einsetzen der Komponentengleichungen für R und für L erhalten wir die Gleichung in Abhängigkeit von unserer gesuchten Größe i:
L mal d nach d t von i plus R mal i ist gleich U q.
Theoretisch könnten wir nun bereits weiter rechnen, allerdings bietet es sich an, 
die D G L weiter umzuformen, dass unsere gesuchte Größe i ohne Koeffizienten einzeln steht.
Also teilen wir durch R und erhaltene unten gezeigte D G L, 
bei der die Einzelterme die gleiche Einheit wie unsere gesuchte Größe haben.
}% Speech Ende
\end{frame}

\begin{frame}\ftx{Ladevorgang \subsubsecname}
\textbf{2.) Homogene Lösung, allg.:} (flüchtig) (ohne Störterm)%
\begin{align*}%
    &\text{DGL}_{h}:&&&
        \frac{L}{R} \cdot \dt \eci{i_{\mathrm h}} + \eci{i_{\mathrm h}} &= 0 &&&
            &\text{Ansatz:}\quad
            \eci{i_{\mathrm h}} = K \cdot \mathrm{e}^{\lambda t}\\
    &\text{char. Pol.}:&&&
        \frac{L}{R} \cdot \lambda + 1 &= 0 &&&
            &\Rightarrow\quad \lambda = -\frac{1}{\tau} = -\frac{R}{L}\\
    &\text{hom. Lsg.}:&&&
        \eci{i_{\mathrm h}} &= K \cdot \mathrm{e}^{-\sfrac{t}{\tau}}&&&
            &\text{mit}\quad \tau = \frac{L}{R}
    \\[-15pt]&\hphantom{R1oooooooooo}&&&\hphantom{L3ooooooooooo}&\hphantom{R3ooooooooooooooo}&&&&\hphantom{R5oooooooooooooooooo}\nonumber% Dummy-Zeile RL-DC für horizontales Alignment zwischen align-Umgebungen [15pt hoch]
\end{align*}%
\pause%
\textbf{3.) Partikulare Lösung:} (eingeschwungen) mit $t \to \infty$%
\begin{align*}%
    &\text{DGL}_{p}:&&&
        \redcancel{\frac{L}{R} \cdot \dt \eci{i_{\mathrm p}}} + \eci{i_{\mathrm p}} &= \frac{\ecv{U_q}}{R} &&&
            &\text{mit}\quad \eci{i_{\mathrm p}} = konst.  \\
    &\text{part. Lsg.}:&&&
        \eci{i_{\mathrm p}} &= \frac{\ecv{U_q}}{R}
    \\[-15pt]&\hphantom{R1oooooooooo}&&&\hphantom{L3ooooooooooo}&\hphantom{R3ooooooooooooooo}&&&&\hphantom{R5oooooooooooooooooo}\nonumber% Dummy-Zeile RL-DC für horizontales Alignment zwischen align-Umgebungen [15pt hoch]
\end{align*}%

\speech{SchaltvorgaengeDC2RLLadevorgang3}{1}{%
Durch Aufstellen des charakteristschen Polnoms können wir den Eigenwert Lambda bestimmen.
Da die Lösung in Exponentialform auftritt, schreiben wir direkt Lambda ist gleich minus eins durch Tau ist gleich minus R durch L.
Dadurch können wir die homogene Lösung direkt in die Form K mal e hoch minus t durch Tau bringen mit Tau ist gleich L durch R.
}% Speech Ende
\speech{SchaltvorgaengeDC2RLLadevorgang3}{2}{%
Für die partikläre Lösung bestimmen wir den eingeschwungenen Zustand. 
Wir können annehmen, dass i p konstant ist, wodurch der Differentialterm entfällt. 
Daraus folgt i p ist gleich U q durch R.
}% Speech Ende
\end{frame}

\begin{frame}\ftx{Ladevorgang \subsubsecname}
\textbf{4.) Inhomogene Lösung, allg.:} (Überlagerung)%
\begin{align*}%
    &\text{allg. Lsg.:}&&&
        \eci{i}(t) &=
        \underbrace{%
            K \cdot \mathrm{e}^{-\sfrac{t}{\tau}} \vphantom{\frac{\ecv{U_q}}{R}}%
        }_{\eci{i_{\mathrm h}}}
        +
        \underbrace{
            \frac{\ecv{U_q}}{R}
        }_{\eci{i_{\mathrm p}}} &&&
        &\text{mit}\quad \eci{i_{\mathrm h}}=\eci{i_f},\quad \eci{i_{\mathrm p}}=\eci{i_e}
    \\[-15pt]&\hphantom{R1oooooooooo}&&&\hphantom{L3ooooooooooo}&\hphantom{R3ooooooooooooooo}&&&&\hphantom{R5oooooooooooooooooo}\nonumber% Dummy-Zeile RL-DC für horizontales Alignment zwischen align-Umgebungen [15pt hoch]
\end{align*}%
\pause%
\textbf{5.) Konstanten bestimmen:} (Anfangsbedingungen einsetzen)%
\begin{align*}%
    &\text{Anfangsbed.}&&&
        \eci{i}(0) &= K \cdot \redcancel{\mathrm{e}^0} + \frac{\ecv{U_q}}{R} \overset{!}{=} 0 &&&
        &\Rightarrow\quad K = -\frac{\ecv{U_q}}{R}\\
    &\text{eindeut. Lsg.:}&&&
        \eci{i}(t) &= \frac{\ecv{U_q}}{R} \cdot \left( 1 - \mathrm{e}^{-\sfrac{t}{\tau}} \right)
    \\[-15pt]&\hphantom{R1oooooooooo}&&&\hphantom{L3ooooooooooo}&\hphantom{R3ooooooooooooooo}&&&&\hphantom{R5oooooooooooooooooo}\nonumber% Dummy-Zeile RL-DC für horizontales Alignment zwischen align-Umgebungen [15pt hoch]
\end{align*}%
Spannung $\ecv{u_L}$ aus Strom $\eci{i}$ bestimmen:%
\begin{align*}%
    &\text{Spannung:}&&&
        \ecv{u_L} &= \ecv{U_q} \cdot \mathrm{e}^{-\sfrac{t}{\tau}} &&&
        &\text{mit}\quad \ecv{u_L} = L \cdot \dt \eci{i}
    \\[-15pt]&\hphantom{R1oooooooooo}&&&\hphantom{L3ooooooooooo}&\hphantom{R3ooooooooooooooo}&&&&\hphantom{R5oooooooooooooooooo}\nonumber% Dummy-Zeile RL-DC für horizontales Alignment zwischen align-Umgebungen [15pt hoch]
\end{align*}%

\speech{SchaltvorgaengeDC2RLLadevorgang4}{1}{%
Die Allgemeine Lösung ergibt sich durch Überlagerung beider Lösungen.
}% Speech Ende
\speech{SchaltvorgaengeDC2RLLadevorgang4}{2}{%
und durch einsetzen unserer Anfangsbedingung bestimmen wir unsere Konstante.
i von null ist gleich K mal e hoch null plus U q durch R ist gleich Null.
Daraus folgt K ist gleich minus U q durch R.
Die eindeutige Lösung ergibt: UU q durch R mal Klammer auf eins minus e hoch minus t durch tau Klammer zu.
Die Spannung u L lässt sich über u L gleich L mal d nach d t von i bestimmen.
Durch Einsetzen und Vereinfachung folgt: u L ist gleich U q mal e hoch minus lambda durch tau.
Als Probe können wir überlegen, ob die Ergebnisse plausibel sind.
Der Strom i von t entspricht einer exponentiellen Annäherung an den maximalen Strom U q durch R.
Die Spannung fällt exponentiell ab vom Maximalwert U q.
Da die Induktivität lädt, passt die positive Spannung gemäß Verbraucherzählpfeilsystem.
Da der eingeschwungene Strom konstant ist, geht u L gegen null.
Beide Ergebnisse sind plausibel.
}% Speech Ende
\end{frame}
}% Ende Folien only Multi Frame

%-------------------------------------------------------------------------------------------------%

\s{% RL-Laden, Skript only
\newpage
\begin{bsp}{RL-Glied DC-Ladevorgang - Über DGL von $u_L$}{rl:chargedc:dglu}
    Bestimmung von Spannung $\ecv{u_L}$ und Ladestrom $\eci{i}$ der Induktivität $L$ beim Laden über Widerstand $R$
    mit Gleichspannung $\ecv{U_q}$ für $t \geq 0$. Der Stromkreis ist vor $t=0$ unterbrochen.\\[-6pt]

    \begin{minipage}{\textwidth}\centering
        \begin{minipage}[t][][t]{0.48\textwidth}\centering\vspace{0cm}% vspace: input empty space before tikzpicture for top alignment
            \includegraphics{Tikz/pdf/circ_switchable_rl_charge_dc.pdf}%
            \begin{equation*}\begin{aligned}
            \text{gegeben:}&&       &\eci{i}(0),\ \ecv{U_q},\ R,\ L \\
            \text{gesucht:}&&       &\ecv{u_L}(t),\ \eci{i}(t)
            \end{aligned}\end{equation*}
        \end{minipage}
        \begin{minipage}[t][][t]{0.48\textwidth}\centering\vspace{0cm}% vspace: input empty space before tikzpicture for top alignment
            \includegraphics{Tikz/pdf/plot_rl_charge_ul_i_seperate.pdf}
        \end{minipage}
    \end{minipage}

    \textbf{1.) DGL aufstellen:} von $\ecv{u_L}(t)$ für $t \geq 0$\vspace{5pt}

    Ausgehend von der Maschengleichung wird $\ecv{u_R}$ in Abhängigkeit von $\ecv{u_L}$ ausgedrückt:%
    \begingroup% Änderungen lokal vornehmen
        \addtolength{\jot}{2pt}% Zeilenabstand in align-Umgebung erhöhen
    \begin{align*}
        &\text{Masche}:&&&
            \ecv{u_R} + \ecv{u_L}  &= \ecv{U_q}&&&
            &\text{mit}\quad \ecv{u_R} = R \cdot \eci{i} \\
        &&&&
            R \cdot \eci{i} + \ecv{u_L} &= \ecv{U_q} &&&
            &\left| \cdot \dt \right. \\
        &&&&
        R \cdot \frac{\ecv{u_L}}{L}  + \dt \ecv{u_L}  &= \dt \ecv{U_q} &&&
            &\text{mit}\quad \ecv{u_L} = L \cdot \dt \eci{i} \\
        &\text{DGL}:&&&
            \frac{L}{R} \cdot \dt \ecv{u_L} + \ecv{u_L}  &= 0
            \\[-15pt]&\hphantom{R1ooooooo}&&&\hphantom{L3oooooooooooooooooooo}&\hphantom{R3ooo}&&&&\hphantom{R5oooooooooooooooooooooo}\nonumber% Dummy-Zeile A für horizontales Alignment zwischen align-Umgebungen [15pt hoch]
        \end{align*}%
    \textbf{2.) Homogene Lösung, allg.:} (flüchtig) (ohne Störterm)%
    \begin{align*}%
        &\text{DGL}_{h}:&&&
            \frac{L}{R} \cdot \dt \ecv{u_{L,h}} + \ecv{u_{L,h}} &= 0 &&&
                &\text{Ansatz:}\quad
                \ecv{u_{L,h}} = K \cdot \mathrm{e}^{\lambda t}\\
        %&&&&
        %    K \cdot(\frac{L}{R} \cdot \quad\lambda \mathrm{e}^{\lambda t}\ +\ \mathrm{e}^{\lambda t}) &= 0 \\
        &\text{char. Pol.:}&&&
            \frac{L}{R} \cdot \lambda + 1 &= 0 &&&
                &\Rightarrow\quad \lambda = -\frac{1}{\tau} = -\frac{R}{L}\\
        &\text{hom. Lsg.}:&&&
            \ecv{u_{L,h}} = K \cdot \mathrm{e}^{-\sfrac{t}{\tau}}&&&&
                &\text{mit}\quad \tau = \frac{L}{R}
    \\[-15pt]&\hphantom{R1ooooooo}&&&\hphantom{L3oooooooooooooooooooo}&\hphantom{R3ooo}&&&&\hphantom{R5oooooooooooooooooooooo}\nonumber% Dummy-Zeile A für horizontales Alignment zwischen align-Umgebungen [15pt hoch]
    \end{align*}%
    \textbf{3.) Partikulare Lösung:} (eingeschwungen) mit $t \to \infty$%
    \begin{align*}%
        &\text{DGL}_{p}:&&&
            \redcancel{\frac{L}{R} \cdot \dt \ecv{u_{L,p}}} + \ecv{u_{L,p}} &= 0 &&&
                &\text{mit}\quad \ecv{u_{L,p}} = konst.  \\
        &\text{part. Lsg.}:&&&
            \ecv{u_{L,p}} &= 0
            \\[-15pt]&\hphantom{R1ooooooo}&&&\hphantom{L3oooooooooooooooooooo}&\hphantom{R3ooo}&&&&\hphantom{R5oooooooooooooooooooooo}\nonumber% Dummy-Zeile A für horizontales Alignment zwischen align-Umgebungen [15pt hoch]
        \end{align*}%
    \textbf{4.) Inhomogene Lösung, allg.:} (Überlagerung)%
    \begin{align*}%
        &&&&
            \ecv{u_L}(t) &= \ecv{u_{L,h}}(t) + \redcancel{\ecv{u_{L,p}}(t)} &&&
            &\ecv{u_L}(t)= \ecv{u_{L,f}}(t) + \redcancel{\ecv{u_{L,e}}(t)}  \\
        &\text{allg. Lsg.:}&&&
            \ecv{u_L}(t) &= K \cdot \mathrm{e}^{-\sfrac{t}{\tau}}
        \\[-15pt]&\hphantom{R1oooooooooooo}&&&\hphantom{L3ooo}&\hphantom{R3oooooooooooooooo}&&&&\hphantom{R5oooooooooooooooooooooo}\nonumber% Dummy-Zeile B für horizontales Alignment zwischen align-Umgebungen [15pt hoch]
    \end{align*}%
    \textbf{5.) Konstanten bestimmen:} (Anfangsbedingungen einsetzen)%
    \begin{align*}%
        &\text{Anfangsbed.}&&&
            \eci{i}(0) &= \frac{\ecv{u_R}(0)}{R} \overset{!}{=} 0 &&&
            &\Rightarrow\quad \ecv{u_L}(0) = \ecv{U_q} - \redcancel{\ecv{u_R}(0)}\\
        &&&&
            \ecv{u_L}(0) &= K \cdot \mathrm{e}^{-0} \overset{!}{=} \ecv{U_q} &&&
            &\Rightarrow\quad K = \ecv{U_q}\\
        &\text{eindeut. Lsg.:}&&&
            \ecv{u_L}(t) &= \ecv{U_q} \cdot \mathrm{e}^{-\sfrac{t}{\tau}}
        \\[-15pt]&\hphantom{R1oooooooooooo}&&&\hphantom{L3ooo}&\hphantom{R3oooooooooooooooo}&&&&\hphantom{R5oooooooooooooooooooooo}\nonumber% Dummy-Zeile B für horizontales Alignment zwischen align-Umgebungen [15pt hoch]
        \end{align*}%
    Strom $\eci{i}$ aus Spannung $\ecv{u_L}$ bestimmen:%
    \begin{align*}%
        &\text{Strom:}&&&
            \eci{i} &= \frac{\ecv{U_q}}{R} \cdot \left( 1 - \mathrm{e}^{-\sfrac{t}{\tau}} \right) &&&
            &\text{mit}\quad \eci{i} = \frac{\ecv{u_R}}{R} = \frac{\ecv{U_q} - \ecv{u_L}}{R}
        \\[-15pt]&\hphantom{R1oooooooooooo}&&&\hphantom{L3ooo}&\hphantom{R3oooooooooooooooo}&&&&\hphantom{R5oooooooooooooooooooooo}\nonumber% Dummy-Zeile B für horizontales Alignment zwischen align-Umgebungen [15pt hoch]
    \end{align*}
    \endgroup
\end{bsp}
}% end skript only

%-------------------------------------------------------------------------------------------------%

\begin{frame}[t]\ftx{Entladevorgang \subsubsecname}%
\s{%
    Der Entladevorgang einer Induktivität $L$ über einen Widerstand $R$ ist ähnlich zum Ladevorgang über einen Widerstand $R$ mit Gleichspannung $U_q$.
    Der Kurzschluss über $R$ entspricht einer Spannungsquelle mit $U_q=0\ \mathrm{V}$.
    Entladevorgang und Ladevorgang unterscheiden sich lediglich in den Anfangsbedingungen und dem Störterm der jeweiligen DGL.
    Die Zeitkonstante $\tau$ ist identisch für den Lade- und Entladevorgang.
    Auf eine detaillierte Berechnung wird daher an dieser Stelle verzichtet.

    \begin{figure}[H]\centering
        \begin{subfigure}{0.48\textwidth}\centering
            \includegraphics{Tikz/pdf/circ_switchable_rl_discharge.pdf}
            \caption{Schaltbild}
            \label{fig:circ:rl:discharge}
        \end{subfigure}
        \begin{subfigure}{0.48\textwidth}\centering
            \includegraphics{Tikz/pdf/plot_rl_discharge_ul_i_seperate.pdf}
            \caption{Spannungs- und Stromverlauf}
            \label{fig:plot:rl:discharge}
        \end{subfigure}
        \caption{RL-Glied, Entladevorgang}
        \label{fig:rl:discharge}
    \end{figure}

    Abbildung \ref{fig:rl:discharge} zeigt das Schaltbild (a) und den Spannungs- und Stromverlauf (b)
    der Induktivität $L$ beim Entladen über einen Widerstand $R$ ab Schaltzeitpunkt $t=0$.
    Die Induktivität ist vor dem Schalten in Reihe mit dem Widerstand $R$ an eine Gleichspannungsquelle $U_q$ angeschlossen.
    Zum Schaltzeitpunkt ist die Induktivität vollständig geladen(stationärer Zustand)
    und führt vor dem Schalten den Konstantstrom $U_q/R$.

    Reale Induktivitäten lassen sich durch eine Reihenschaltung aus einer idealen Induktivität $L$
    und einem ohmschen Widerstand $R_{\text{Cu}}$ modellieren, der den Kupferverlusten entspricht.
    Durch Parallelschaltung eines ohmschen Widerstandes $R_{\text{Fe}}$ unmittelbar zur Induktivität $L$
    lassen sich zudem Eisenverluste modellieren.

    \begin{figure}[H]\centering
        \begin{subfigure}{0.48\textwidth}\centering
            \includegraphics{Tikz/pdf/circ_switchable_rrl_charge_detail.pdf}
            \caption{Schaltbild}
            \label{fig:circ:rl:charge}
        \end{subfigure}
        \begin{subfigure}{0.48\textwidth}\centering
            \includegraphics{Tikz/pdf/plot_rrl_charge_dc.pdf}
            \caption{Spannungs- und Stromverlauf}
            \label{fig:plot:rl:charge}
        \end{subfigure}
        \caption{RL-Glied, realer Induktivität, Ladevorgang an idealer Spannungsquelle}
        \label{fig:rrl:charge}
    \end{figure}

    Abbildung \ref{fig:rrl:charge} zeigt das Schaltbild (a) einer realen Spule mit Induktivität $L$ in Serie zu einem Widerstand $R_{\text{Cu}}$ (hier $R_1$).
    Unmittelbar parallel zur Induktivität befindet sich der ohmsche Widerstand $R_{\text{Fe}}$ (entspricht $R_2$).
    Die Induktivität wird ab $t=0$ über eine ideale Konstantspannungsquelle $U_q$ geladen und ist vorher vollständig entladen.
    Daneben abgebildet sind die zugehörigen den Spannungs- und Stromverläufe (b).

    Im Vergleich zum Ladevorgang eines reine RL-Serienschaltung in Beispiel \ref{bsp:rl:chargedc:dglu} springt die Spannung
    über der Induktivität nicht auf die ganze Quellenspannung sondern auf einen geringeren Wert begrenzt durch den Spannungsteiler $\frac{R_2}{R_1+R_2}$.
    Der Eingangsstrom macht bedingt durch die Eisenverluste einen Sprung zum Schaltzeitpunkt und nähert sich dann exponentiell dem Strom $\frac{U_q}{R_1}$.
    Der Strom durch $R_2$ springt beim Schalten auf den Wert $\frac{U_q}{R_1+R_2}$ und fällt dann exponentiell ab.
    Der Strom durch die Induktivität nähert sich exponentiell dem Strom $\frac{U_q}{R_1}$ an. Die Zeitkonstante beträgt $\tau=\frac{L}{R_1||R_2}$.
}% End Skript only
\b{% Anfang Folien only
\begin{minipage}{\textwidth}\centering
% Obere Hälfte
\begin{minipage}[t][][t]{0.48\textwidth}\centering\vspace{0cm}% vspace: input empty space before tikzpicture for top alignment
    \includegraphics{Tikz/pdf/circ_switchable_rl_discharge.pdf}%
    \begin{equation*}\begin{aligned}
        &\text{gegeben:}&       &\eci{i}(0),\ \ecv{U_q},\ R,\ L \\
        &\text{gesucht:}&       &\eci{i}(t)
    \end{aligned}\end{equation*}
\end{minipage}
\begin{minipage}[t][][t]{0.48\textwidth}\centering\vspace{0cm}% vspace: input empty space before tikzpicture for top alignment
    \resizebox{!}{4.25cm}{\includegraphics{Tikz/pdf/plot_rl_discharge_ul_i_seperate.pdf}}
\end{minipage}
\end{minipage}
% Untere Hälfte
\resizebox{0.9\textwidth}{!}{% to fit on slide
\begin{minipage}{\textwidth}%
\begin{align*}
&\textbf{1. DGL}&\hspace{15pt} 0 &= \frac{L}{R} \cdot \dt\, \eci{i} + \eci{i} \vphantom{\frac{L}{R}}\\[-2pt]
&\textbf{2. Homogene Lösung}&\hspace{15pt}\eci{i_{\mathrm{h}}} &= K \cdot \mathrm{e}^{-\sfrac{t}{\tau}} \quad\text{mit}\quad \tau = \frac{L}{R} \vphantom{\frac{L}{R}}\\[-2pt]
&\textbf{3. Partikulare Lösung}&\hspace{15pt}\eci{i_{\mathrm{p}}} &= 0 \vphantom{\frac{L}{R}}\\[-2pt]
&\textbf{4. Inhomogene Lösung}&\hspace{15pt}\eci{i} &= K \cdot \mathrm{e}^{-\sfrac{t}{\tau}} \vphantom{\frac{U_q}{R}} \vphantom{\frac{L}{R}}\\[-2pt]
&\textbf{5. Konstanten bestimmen}&\hspace{15pt}\eci{i} &= \frac{\ecv{U_q}}{R} \cdot \mathrm{e}^{-\sfrac{t}{\tau}}
\end{align*}
\end{minipage}
}%
}% Ende Folien only

\speech{SchaltvorgaengeDC2RLEntladevorgang}{1}{%
Der Entladevorgang der Induktivität über L lässt sich analog zum Ladevorgang berechnen und ist hier nur kurz als Zusammenfassung dargestellt.
Oben links sehen sie wieder das Schaltbild und rechts die Zeitverläufe für u L und für i beim Entladen.
Die fünf Schritte mit Teillösungen sind unten wieder zusammen gefasst.
**Wichtig**: Die Zeitkonstante Tau ist hier gleich für den Lade- und für den Entladevorgang mit Tau gleich L durch R.
Als Übung können Sie:
* erstens Zeitverlauf für i und für u L bestimmen,
* zweitens die Zeitkonstante Tau allgemein bestimmen.
Im nächsten Video geht es weiter mit Schaltvorgängen bei sinusförmigen Anregungen. 
}% Speech Ende
\end{frame}

%%%%%%%%%%%%%%%%%%%%%%%%%%%%%%%%%%%%%%%%%%%%%%%%%%%%%%%%%%%%%%%%%%%%%%%%%%%%%%%%%%%%%%%%%%%%%%%%%%

\subsection{bei sinusförmiger Anregung}
\label{sec:schaltvorgaengezeitbereich:ac}%
\v{% Video Only Frame
\begin{frame}\ftx{Schaltvorgänge \subsecname}
    \begin{Lernziele}{Video - Schaltvorgänge bei Wechselspannung}
        In diesem Video lernen Sie:
        \begin{itemize}\b{\setlength\itemsep{0.7em}}
            \item Schaltvorgänge bei AC Anregung zu berechnen,
            \item die Abhängigkeit von Frequenz und Schaltzeitpunkt zu verstehen,
            \item mögliche Überströme und -spannungen bei Schaltvorgängen abzuschätzen,
            \item und wie Schaltvorgänge bei AC Anregung vermieden werden können.
        \end{itemize}
    \end{Lernziele}%

\speech{SchaltvorgaengeACLernziele}{1}{%
Willkommen zum Lehrvideo zum Thema Schaltvorgänge bei sinusförmiger Anregung.
In diesem Video werden wir uns die Besonderheiten beim Schalten bei sinusförmiger Anregung herleiten.
Hierfür wandeln wir leicht die Beispiele aus den vorigen Videos ab und betrachten 
Schaltvorgänge bei einfachen R C und R L Serienschaltungen bei Anregung mit Wechselspannung.
Sie lernen:
* Schaltvorgänge bei A C Anregung zu berechnen, 
* welche Rolle Frequenz und Schaltzeitpunkt spielen,
* die Gefahr von Überströmen und -spannungen bei Schaltvorgängen abzuschätzen und
* Möglichkeiten zur Vermeidung von Schaltvorgängen bei A C Anregung kennen.
}% Speech Ende
\end{frame}%
}% Video Only Frame Ende

\begin{frame}\ftx{Schaltvorgänge \subsecname}%
\s{%
    Die Vorgehensweise zur Berechnung von Schaltvorgängen bei sinusförmiger Anregung ist prinzipiell gleich zur Vorgehensweise bei gleichförmiger Anregung.
    Deshalb wird in diesem Kapitel nur auf die Besonderheiten bei sinusförmiger Anregung eingegangen und die Berechnung anhand eines Beispiels erläutert.

    Die Differentialgleichungen bei Schaltungen mit Wechselspannungs- oder stromquellen unterscheiden sich gegenüber den Gleichspannungs- oder -stromfällen lediglich durch die rechte Seite der DGL, dem Störterm.
    Der Störterm ergibt sich aus der Anregung und ist bei sinusförmiger Anregung linearer, zeitinvarianter Systeme ebenfalls sinusförmig.

    Aufgrund der Zeitabhängigkeit sinusförmiger Anregungen, ist der exakte Schaltzeitpunkt entscheiden für das Verhalten des Systems.
    Präziser formuliert ist die Phasenlage der Anregung zum Schaltzeitpunkt entscheidend.

    Am Beispiel eines RC-Gliedes wird in Kapitel \ref{sec:schaltvorgaengezeitbereich:ac:c} die Berechnung des Schaltvorgangs bei sinusförmiger Anregung gezeigt.
    Dabei wird die Abhängigkeit des Schaltvorgangs vom Schaltzeitpunkt und die Möglichkeit transienter Spitzenströme und -spannungen näher erläutert.

    Exemplarisch ist in Beispiel \ref{bsp:rl:chargeac} die Berechnung eines Schaltvorganges für ein RL-Glied bei Anregung mit Wechselspannung zusammenfassend präsentiert.
}%
\b{%
\centering
\begin{minipage}{0.48\textwidth}%\centering % linksbündig
    Schaltvorgänge bei AC-Anregung sind:
    \vspace{0.5em}
    \begin{itemize}
        \item frequenzabhängig ($f,\ \omega$)
        \item schaltzeitpunktabhängig ($t_0$)
    \end{itemize}
    \vspace{0.5em}
    \pause
    Berechnung analog zu DC zunächst für:
    \vspace{0.5em}
    \begin{itemize}
        \item eine Kapazität $C$
        \item eine Induktivität $L$
    \end{itemize}
    \vspace{0.5em}
    jeweils in Reihe mit einem Widerstand $R$.
    \pause

    \vspace{1em}
    RC- oder RL-Glied $\Longrightarrow$ \textbf{DGL} 1. Ordnung

    \begin{equation*}
        a_1 \cdot \dt\, y(t) + a_0 \cdot y(t) = b
    \end{equation*}
\end{minipage}%
\begin{minipage}{0.48\textwidth}\centering
    \pause
    \resizebox{0.8\textwidth}{!}{\includegraphics{Tikz/pdf/plot_rc_charge_uc_ac_different_t0.pdf}}
    Vergleich: Schaltzeitpunkte $t_0$ und $t_1$
\end{minipage}
}%

\speech{SchaltvorgaengeAC}{1}{%
Fangen wir mit den grötßten Unterschieden zu Schaltvorgängen bei D C Anregung an.
Schaltvorgänge bei AC Anregung sind
* frequenzabhängig und
* schaltzeitpunktabhängig.
}% Speech Ende
\speech{SchaltvorgaengeAC}{2}{%
Wir leiten uns beide Merkmale anhand zweier einfacher Beispiele her.
Die Berechnung erfolgt analog zu den D C Beispielen mit dem bekannten Berechnungsverfahren. 
Zunächst für eine Serienschaltung aus C und R und dann für eine Serienschaltung aus L und R. 
}% Speech Ende
\speech{SchaltvorgaengeAC}{3}{%
In beiden Beispielen erhalten wir wieder eine D G L erster Ordnung.
Im Vergleich zu den D C Beispielen unterscheiden sich die D G L en hier nur durch einen sinusförmigen Störterm b. 
Warum hängt unser Schaltvorgang nun speziell von der Frequenz und vom Schaltzeitpunkt ab?
Pausieren Sie am Besten kurz das Video und überlegen Sie selbst eine Lösung, bevor diese eingeblendet wird. 
}% Speech Ende
\speech{SchaltvorgaengeAC}{4}{%
Rechts sehen wir drei Zeitverläufe dargestellt. Zum Vergleich sind zwei Schaltvorgänge für unterschiedliche Schaltzeitpunkte dargestellt.
* Oben ist die Wechselspannung u q zu sehen, welche als AC Anregung dient. 
* In der Mitte ist der Schaltvorgang für die Spannung u C abgebildet, wenn bei t null geschalten wird.
* Unten ist der Schaltvorgang für die Spannung u C abgebildet, wenn bei t eins geschalten wird.
Der erste Schaltvorgang im mittleren Plot ist als Überlagerung von u C in blau aus dem
flüchtigen Zustand u C f in schwarz gestrichelt und dem eingeschwungenen Zustand u C e in schwarz dargestellt.
Der eingeschwungene Zustand ist sinusförmig, der flüchtige Zustand exponentiell abklingend. 
Beim zweiten Schaltvorgang im unteren Plot ist die Spannung u C ab dem Schaltzeitpunkt gleich dem eingeschwungenen Zustand.
Zu erkennen ist, dass der Schaltzeitpunkt direkt im Nulldurchgang von u C e liegt und u C vor dem Schaltzeitpunkt Null beträgt. 
Durch den ideal gewählten Schaltzeitpunkt entfällt der Einschwingvorgang vollständig.
}% Speech Ende
\end{frame}

\subsubsection{bei Kapazitäten}
\label{sec:schaltvorgaengezeitbereich:ac:c}
\begin{frame}[t]\ftx{Schaltvorgänge \subsecname\ \subsubsecname}
\s{%
    Zur Erläuterung der Besonderheiten von Schaltvorgängen bei sinusförmiger Anregung wird das Beispiel eines RC-Gliedes betrachtet.
    Die Berechnung erfolgt wie in Beispiel \ref{bsp:rc:chargedc} nach Merksatz \ref{merk:berechnungsverfahren:zeitbereich}.

    Eine Kapazität $C$ wird über einen Widerstand $R$ ab dem Schaltzeitpunkt $t=t_0$ an eine ideale Wechselspannungsquelle mit Spannung $u_q(t) = \hat{U}_q \cdot \cos(\omega t + \varphi_q)$ angeschlossen.
    Die Kapazität $C$ ist zu Beginn vollständig entladen. Abbildung \ref{fig:circ:rc:charge:ac} zeigt das Schaltbild dieser Anordnung.

    \begin{figure}[H]\centering
        \includegraphics{Tikz/pdf/circ_switchable_rc_charge_ac.pdf}%
        \caption{Schaltbild, RC-Glied (Reihe) nach Schalten an idealer Wechselspannungsquelle%\newline $R$ und $C$ in Reihe, für $t \geq t_0$ an idealer Wechselspannungsquelle $u_q$, vor $t=t_0$ kurzgeschlossen
        }
        \label{fig:circ:rc:charge:ac}
    \end{figure}

    \textbf{1.) Differentialgleichung aufstellen}:

    Die inhomogene DGL von $u_C$ für $t \geq t_0$ ergibt sich analog zu Beispiel \ref{bsp:rc:chargedc} zu:
    \begin{align}
        RC \cdot \dt \ecv{u_C}(t) + \ecv{u_C}(t) ={}& \ecv{u_q}(t) \\
        \text{mit} \quad &\ecv{u_q}(t) = \ecv{\hat{U}_q} \cdot \cos(\omega t + \varphi_q)
    \end{align}
    Gegenüber Beispiel \ref{bsp:rc:chargedc} ist lediglich der Störterm verändert. Der Gleichspannungsterm $U_q$ ist hier durch den Wechselspannungsterm $u_q$ ersetzt.

    \textbf{2.) Homogene Lösung, allgemein} (flüchtiger Zustand):

    Die homogene Lösung $u_{C,h}$ ist unabhängig vom Störterm und damit von der Anregung.
    Die Lösung ist aus Beispiel \ref{bsp:rc:chargedc} übernommen mit Berücksichtigung des variablen Schaltzeitpunktes $t_0$:
    \begin{align}
        \ecv{u_{C,\mathrm{h}}} &= K \cdot \mathrm{e}^{-\frac{t'}{\tau}} \qquad \text{mit} \quad t' = t-t_0 \nonumber\\
        &= K \cdot \mathrm{e}^{-\frac{t-t_0}{\tau}} \qquad \text{mit} \quad \tau = RC  \label{eq:rc:charge:ac:homogeneloesung}
    \end{align}
    Da der Exponentialansatz erst ab Schaltzeitpunkt $t_0$ gilt, wird dieser von der Zeit $t$ subtrahiert.

    \textbf{3.) Partikuläre Lösung} (eingeschwungener Zustand):

    Die partikuläre Lösung $u_{C,p}$ entspricht bei sinusförmiger Anregung ebenfalls einer sinusförmigen Größe
    mit gleicher Frequenz wie die Anregung wie in in Kapitel \ref{sec:grundlagen:zeitbereich:partikulaereloesung} beschrieben ist.
    Der eingeschwungene Zustand lässt sich mittels komplexem Amplitudenzeiger $\hat{\underline{U}}_{C,p}$ wie folgt beschreiben:
    \begin{align}
        \ecv{u_{C,p}}
            &= \ecv{\hat{U}_{C,p}} \cdot \cos(\omega t + \varphi_{C,\mathrm{p}})  \label{eq:rc:charge:ac:partikulaereloesung}\\
            &= \Re{\left\{ \ecv{\hat{\underline{U}}_{C,p}} \cdot \mathrm{e}^{\mathrm{j} \omega t} \right\}} \qquad \text{mit}\quad \ecv{\hat{\underline{U}}_{C,p}} = \ecv{\hat{U}_{C,p}} \cdot \mathrm{e}^{\mathrm{j} \varphi_{C,\mathrm{p}}}
    \end{align}
    Die Amplitude $\hat{U}_{C,p}$ und der (Last-)Phasenwinkel $\varphi_{C,\mathrm{p}}$ lassen sich gemäß komplexen Spannungsteiler bestimmen.
    \begin{align}
        \ecv{\hat{\underline{U}}_{C,p}}
            &= \ecv{\hat{\underline{U}}_q} \cdot \frac{\underline{Z}_C}{R + \underline{Z}_C}
            = \ecv{\hat{\underline{U}}_q} \cdot \frac{\frac{1}{\mathrm{j}\omega C}}{R + \frac{1}{\mathrm{j} \omega C}} \nonumber\\
            &= \frac{\ecv{\hat{\underline{U}}_q}}{1 + \mathrm{j} \omega R C}
            \qquad \text{mit}\quad \ecv{\hat{\underline{U}}_q} = \ecv{\hat{U}_q} \cdot \mathrm{e}^{\mathrm{j} \varphi_q}
            \label{eq:rc:charge:ac:komplex}
    \end{align}
    Für die Amplitude $\hat{U}_{C,p}$ und die Phase $\varphi_{C,\mathrm{p}}$ folgt aus Gl. \ref{eq:rc:charge:ac:komplex}:
    \begin{align}
        \ecv{\hat{U}_{C,p}} = \frac{\ecv{\hat{U}_q}}{\sqrt{1 + \left(\omega RC\right)^2}} \qquad\text{und}\qquad
        \varphi_{C,\mathrm{p}} = \varphi_q - \arctan\left(\omega RC\right) \label{rc:charge:ac:amplitudephase}
    \end{align}
    \textbf{4.) Inhomogene Lösung, allgemein} (Überlagerung):

    Aus der Überlagerung von Gl. \ref{eq:rc:charge:ac:homogeneloesung} und \ref{eq:rc:charge:ac:partikulaereloesung}
    folgt die allgemeine Lösung der inhomogenen DGL:
    \begin{align}
        \ecv{u_C} &= \ecv{u_{C,\mathrm{h}}} + \ecv{u_{C,p}} \nonumber\\
            &= K \cdot \mathrm{e}^{-\frac{t-t_0}{\tau}} + \ecv{\hat{U}_{C,p}} \cdot \cos(\omega t + \varphi_{C,\mathrm{p}}) \label{eq:rc:charge:ac:allg}
    \end{align}
    \textbf{5.) Konstanten bestimmen} (Anfangsbedingung):

    Aus der Anfangsbedingung, dass die Kapazität zu Beginn entladen ist, folgt für die Konstante $K$:
    \begin{align}
        \ecv{u_C}(t_0) &\overset{!}{=} 0 \nonumber\\
        K \cdot \redcancel{\mathrm{e}^{0}} + \ecv{\hat{U}_{C,p}} \cdot \cos(\omega t_0 + \varphi_{C,\mathrm{p}}) &= 0 \nonumber\\
        K &= -\ecv{\hat{U}_{C,p}} \cdot \cos(\omega t_0 + \varphi_{C,\mathrm{p}}) \label{eq:rc:charge:ac:k}
    \end{align}
    Mit Gl. \ref{eq:rc:charge:ac:k} eingesetzt in \ref{eq:rc:charge:ac:allg} ergibt sich die eindeutige Lösung für $u_C(t)$:
    \begin{align}
        \ecv{u_C}(t) &= \underbrace{-\ecv{\hat{U}_{C,p}} \cdot \cos(\omega t_0 + \varphi_{C,\mathrm{p}}) \cdot \mathrm{e}^{-\frac{t-t_0}{\tau}}}_{\ecv{u_{C,\mathrm{h}}}} + \underbrace{\ecv{\hat{U}_{C,p}} \cdot \cos\left(\omega t + \varphi_{C,\mathrm{p}}\right)}_{\ecv{u_{C,p}}} \label{eq:rc:charge:ac:uc}
    \end{align}
    An Gleichung \ref{eq:rc:charge:ac:uc} ist erkennbar, dass der flüchtige Zustand $u_{C,f}=u_{C,h}$ ab $t_0$ exponentiell gegen null strebt.
    Im Gegensatz zum DC-Fall (Beispiel \ref{bsp:rc:chargedc}) ist der Startwert der Exponentialkurve $K$ aus Gleichung \ref{eq:rc:charge:ac:k} abhängig von $t_0$, da der eingeschwungene Zustand $u_{C,e}=u_{C,p}$ zeitabhängig ist.

    \textbf{Verhalten bei Variation des Schaltzeitpunktes/der Zeitkonstante}:

    Abbildung \ref{fig:plot:rc:charge:ac:uc:vergleicht0} zeigt den Spannungsverlauf der Anregung $u_q$ bei einem Nullphasenwinkel $\varphi_q$ von Null
    und den Spannungsverlauf der Kapazität $u_C$.
    Zum Vergleich ist $u_C$ für zwei verschiedenen Schaltzeitpunkten $t_0$ gezeigt.
    Das ist einmal beim negativen Scheitelwert und einmal beim Nulldurchgang von $u_{C,p}$.
    $u_C$ ist als Überlagerung des flüchtigen Zustandes $u_{C,h}$ (schwarz, gestrichelt) und des eingeschwungenen Zustandes $u_{C,p}$ (schwarz) dargestellt.

    \begin{figure}[H]\centering
        \begin{subfigure}{0.48\textwidth}\centering
            \includegraphics{Tikz/pdf/plot_rc_charge_uc_ac_different_t0.pdf} % u_C bei verschiedenen Schaltzeitpunkten
            \caption{Verschiedene Schaltzeitpunkte $t_0$}
            \label{fig:plot:rc:charge:ac:uc:vergleicht0}
        \end{subfigure}%
        \begin{subfigure}{0.48\textwidth}\centering
            \includegraphics{Tikz/pdf/plot_rc_charge_uc_ac_different_tau.pdf} % u_C bei verschiedenen Zeitkonstanten
            \caption{Verschiedene Zeitkonstanten $\tau$}
            \label{fig:plot:rc:charge:ac:uc:vergleichtau}
        \end{subfigure}
        \caption{Spannungsverlauf $u_C$ bei RC-Glied nach Schalten an Wechselspannungsquelle $u_q$}
    \end{figure}

    Wie zu erkennen ist, ist der Ausgleichsvorgang maximal ($|K|=max.$), wenn die Differenz $|u_{C,p}(t_0)- u_C(t_0)|$ maximal ist.
    Erfüllt die $u_{C,p}$ bereits die Anfangsbedingung mit $u_{C,p}(t_0)\overset{!}{=}u_C(t_0)$, entfällt der flüchtige Zustand vollständig $K=0 \Rightarrow u_{C,h}=0$ und es kommt zu keinem Ausgleichsvorgang.

    In Abbildung \ref{fig:plot:rc:charge:ac:uc:vergleichtau} wird der Spannungsverlauf $u_C$ für verschiedenen Zeitkonstanten $\tau$ gezeigt.
    Einmal beträgt $\tau=\frac{2\pi}{\omega}$, einmal $\tau=\frac{20\pi}{\omega}$.
    In beiden Fällen ist $t_0$ so gewählt, dass der Ausgleichsvorgang maximal wird.

    Bei großen Zeitkonstanten klingt der flüchtige Zustand langsamer ab, dadurch kann es bei ungünstigen Schaltzeitpunkten zu Überspannungen/-strömen kommen.
    Ist $u_C$ wie im Beispiel nach Anfangsbedingung zu Beginn entladen ($u_C(t_0)=0$), kann $u_C$ bis zum doppelten Scheitelwert von $u_{C,p}$ ansteigen.
    Ist $u_C$ hingegen zu Beginnn bereits bis zum negativen/positiven Scheitelwert von $u_{C,p}$ geladen, kann $u_C$ sogar bis zum dreifachen Scheitelwert von $u_{C,p}$ ansteigen.

    \begin{Merksatz}{Ausgleichsvorgänge bei AC-Anregung}
        Durch geschicktes Schalten können Ausgleichsvorgänge bei AC-Anregung vermieden werden.\vspace{5pt}

        Bei großen Zeitkonstanten sind Überwerte bis zum doppelten/dreifachen Scheitelwert möglich.
    \end{Merksatz}
}% end skript only
\b{% Anfang Folien only
% Obere Hälfte
\begin{minipage}{\textwidth}\centering%
\begin{minipage}[t][][t]{0.48\textwidth}\centering\vspace{0cm}% vspace: input empty space before tikzpicture for top alignment%
    \includegraphics{Tikz/pdf/circ_switchable_rc_charge_ac.pdf}
    \begin{equation*}\begin{aligned}
        &\text{gegeben:}&       &\ecv{u_C}(t_0),\ \ecv{u_q}(t),\ R,\ C,\ \omega \\
        &\text{gesucht:}&       &\ecv{u_C}(t)\quad\text{für}\quad t \geq t_0
    \end{aligned}\end{equation*}
\end{minipage}\pause%
\begin{minipage}[t][][t]{0.48\textwidth}\centering\vspace{0cm}% vspace: input empty space before tikzpicture for top alignment%
    \resizebox{0.75\textwidth}{!}{\includegraphics{Tikz/pdf/plot_rc_charge_uc_ac_different_t0_short.pdf}}
\end{minipage}%
\end{minipage}\pause\hfill%
% Untere Hälfte
\resizebox{0.95\textwidth}{!}{% to fit on slide
\begin{minipage}{\textwidth}\vspace{-0.75cm}%
\begin{align*}
&\textbf{1. DGL}&                   \ecv{u_q} &= RC \cdot \dt \ecv{u_C}(t) + \ecv{u_C}(t) = \ecv{\hat{U}_q} \cdot \cos(\omega t + \varphi_q)\\
&\textbf{2. Homogene Lösung}&       \ecv{u_{C,\mathrm{h}}} &= K \cdot \mathrm{e}^{-\frac{t-t_0}{\tau}} \quad\text{mit}\quad \tau = RC \vphantom{\mathrm{e}^{-\frac{t-t_0}{\tau}}}\\
&\textbf{3. Partikulare Lösung}&    \ecv{u_{C,\mathrm{p}}} &= \ecv{\hat{U}_{C,\mathrm{p}}} \cdot \cos(\omega t + \varphi_{C,\mathrm{p}}) \vphantom{\mathrm{e}^{-\frac{t-t_0}{\tau}}}\\
&\textbf{4. Inhomogene Lösung}&     \ecv{u_C} &= K \cdot \mathrm{e}^{-\frac{t-t_0}{\tau}} + \ecv{\hat{U}_{C,\mathrm{p}}} \cdot \cos(\omega t + \varphi_{C,\mathrm{p}}) \vphantom{\mathrm{e}^{-\frac{t-t_0}{\tau}}}\\
&\textbf{5. Konstanten bestimmen}&  \ecv{u_C} &= \ecv{\hat{U}_{C,\mathrm{p}}} \cdot \left( - \cos(\omega t_0 + \varphi_{C,\mathrm{p}}) \cdot \mathrm{e}^{-\frac{t-t_0}{\tau}} + \cos(\omega t + \varphi_{C,\mathrm{p}}) \right) \vphantom{\mathrm{e}^{-\frac{t-t_0}{\tau}}}
\end{align*}
\end{minipage}
}
}% end Folien only

\speech{SchaltvorgaengeACbeiRC1}{1}{%
Gehen wir das Beispiel Schritt für Schritt durch. 
Oben links ist unsere bekannte Schaltung aus den DC Beispielen. 
Zum Schaltzeitpunkt t null wird ein zuvor kurzgeschlossenes R C Serienglied an eine ideale Wechselspannungsquelle u q geschlossen. 
Gegeben ist die Spannung u C zum Schaltzeitpunkt t null, die Wechselspannung u q, die Komponenten R, C und die Kreisfrequenz omega. 
Gesucht ist u c für t größer gleich t null.
}% Speech Ende
\speech{SchaltvorgaengeACbeiRC1}{2}{%
Die Lösung bestimmen wir allgemein für einen variablen Schaltzeitpunkt. Je nach t null wird der Zeitverlauf für u C zum Beispiel die rechts gezeigte Form annehmen.
}% Speech Ende
\speech{SchaltvorgaengeACbeiRC1}{3}{%
Hier sehen kurz zur Übersicht die fünf Schritte des Berechnungsverfahrens und die zugehörigen Teillösungen die wir im Folgenden ausführlich berechnen werden.
}% Speech Ende
\end{frame}

\b{% Anfang Folien only Multi Frame -- RC AC - charge
% Schritte 1-5 detaillierter, nur Folien
\begin{frame}[t]\ftx{Schaltvorgänge \subsecname\ \subsubsecname}%Schritt 1&2
    \begin{minipage}{\textwidth}\centering%
        \begin{minipage}[t][][t]{0.48\textwidth}\centering\vspace{0cm}% vspace: input empty space before tikzpicture for top alignment%
            \includegraphics{Tikz/pdf/circ_switchable_rc_charge_ac.pdf}
        \end{minipage}%
        \begin{minipage}[t][][t]{0.48\textwidth}\centering\vspace{0cm}% vspace: input empty space before tikzpicture for top alignment%
            \vspace{0.8cm}
            \textbf{DGL 1. Ordnung:}\\[0.85em]
            $a_1 \cdot \dt y(t) + a_0 \cdot y(t)=b$\\
        \end{minipage}%
    \end{minipage}%
    \vspace{0.85cm}
    \textbf{1.) DGL aufstellen:} für $\ecv{u_c}$ ($t \geq t_0$)\\[-1em]
    \begin{align*}
        RC \cdot \dt \ecv{u_C}(t) + \ecv{u_C}(t) ={}& \ecv{u_q}(t) \\
        \text{mit} \quad &\ecv{u_q}(t) = \ecv{\hat{U}_q} \cdot \cos(\omega t + \varphi_q)\quad\text{(Störterm)}
    \end{align*}\pause

    \textbf{2.) Homogene Lösung, allgemein} (flüchtig) (ohne Störterm):
    \begin{align*}
        \ecv{u_{C,\mathrm{h}}} &= K \cdot \mathrm{e}^{-\frac{t'}{\tau}} \qquad &\text{mit} \quad t' = t-t_0 \\
        &= K \cdot \mathrm{e}^{-\frac{t-t_0}{\tau}} \qquad &\text{mit} \quad \tau = RC
    \end{align*}%

\speech{SchaltvorgaengeACbeiRC2}{1}{%
Der erste Schritt, das Aufstellen der D G L ist im Prinzip identisch zu den D C Beispielen. 
Der Unterschied ist nur, dass unser Störterm keine Gleichgröße ist, sondern eine Wechselspannung.
Hier ist unser Störterm u q von t gleich dem Scheitelwert U q Dach mal Kosinus von omega t plus Phi q.
Der Phasenwinkel Phi q kann hier Null gesetzt werden, hilft uns aber später zu erkennen, an welcher Stelle der Phasenwinkel der Quellenspannung eine Rolle spielt. 
}% Speech Ende
\speech{SchaltvorgaengeACbeiRC2}{2}{%
Der zweite Schritt, die Erstellung der homogenen Lösung, kann aus der Lösung im DC Beispiel übernommen werden. 
Da der Störmterm in der homogenen D G L Null ist, sind beide Lösungen identisch. 
Die einzige Anpassung die vorgenommen wird, ist die zeitliche Verschiebung des Exponentialansatzes. 
Die Exponentialkurve beginnt nicht bei t gleich null sondern beim Schaltzeitpunkt t Null.
Die Zeitkonstante tau gleich R C entspricht hier wieder dem negativen Kehrwert des Eigenwertes, das heißt minus eins durch Lambda. 
}% Speech Ende
\end{frame}

\begin{frame}[t]\ftx{Schaltvorgänge \subsecname\ \subsubsecname}%Schritt 3
    \textbf{3.) Partikuläre Lösung} (eingeschwungen) ($t \to \infty$): sinusförmig!\\[2pt]
    Anwendung komplexer Wechselstromrechnung:
    \begin{align*}
        \ecv{u_{C,p}}
            &= \ecv{\hat{U}_{C,p}} \cdot \cos(\omega t + \varphi_{C,\mathrm{p}})\\
            &= \Re{\left\{ \ecv{\hat{\underline{U}}_{C,p}} \cdot \mathrm{e}^{\mathrm{j} \omega t} \right\}} \qquad \text{mit}\quad \ecv{\hat{\underline{U}}_{C,p}} = \ecv{\hat{U}_{C,p}} \cdot \mathrm{e}^{\mathrm{j} \varphi_{C,\mathrm{p}}}
    \end{align*}\pause
    Komplexer Amplitudenzeiger $\ecv{\hat{\underline{U}}_{C,p}}$ durch komplexen Spannungsteiler:
    \begin{equation*}
        \ecv{\hat{\underline{U}}_{C,p}}
            = \ecv{\hat{\underline{U}}_q} \cdot \frac{\underline{Z}_C}{R + \underline{Z}_C}
            = \ecv{\hat{\underline{U}}_q} \cdot \frac{\frac{1}{\mathrm{j}\omega C}}{R + \frac{1}{\mathrm{j} \omega C}} %\nonumber\\
            = \frac{\ecv{\hat{\underline{U}}_q}}{1 + \mathrm{j} \omega C R}
            \qquad \text{mit}\quad \ecv{\hat{\underline{U}}_q} = \ecv{\hat{U}_q} \cdot \mathrm{e}^{\mathrm{j} \varphi_q}
    \end{equation*}\pause
    Für die Amplitude $\ecv{\hat{U}_{C,p}}$ und den (Last-)Phasenwinkel $\varphi_{C,\mathrm{p}}$ folgt:
    \begin{align*}
        \ecv{\hat{U}_{C,p}} = \frac{\ecv{\hat{U}_q}}{\sqrt{1 + \left(\omega CR\right)^2}} \qquad\text{und}\qquad
        \varphi_{C,\mathrm{p}} = \varphi_q - \arctan\left(\omega CR\right) %\label{rc:charge:ac:amplitudephase}
    \end{align*}%

\speech{SchaltvorgaengeACbeiRC3}{1}{%
Im dritten Schritt bestimmen wir die Partikuläre Lösung, das heißt den eingeschwungenen Zustand. 
Aufgrund der sinusförmigen Eingangsspannung u q erhalten wir in unserer eingeschwungenen L Z I Schaltung 
auch eine sinusförmige Ausgangsspannung u C P mit gleicher Kreisfrequenz omega wie bei der Anregung u q. 
Im Zeitbereich dargestellt ist unsere partikuläre Lösung u C p gleich dem Scheitelwert U C p Dach mal Kosinus von omega t plus Phi C P.
Die Amplitude U C p Dach und den Phasenwinkel Phi C p können wir nun mittels komplexer Wechselstromrechnung bestimmen.
Zur Erinnerung, unser u C p von t im Zeitbereich ist der Realteil des komplexen Amplitudenzeigers U C p Dach komplex mal e hoch j omega t. 
Der komplexe Amplitudenzeiger U C p Dach komplex ist gleich U C p Dach reell mal e hoch j Phi C p. 
}% Speech Ende
\speech{SchaltvorgaengeACbeiRC3}{2}{%
Durch den Komplexen Spannungsteiler können wir U C p komplex in Abhängigkeit von U q komplex und den Impedanzen Z C und R bestimmen. 
Durch etwas Umformung kommen wir auf die Form: 
U c P Dach komplex ist gleich U q Dach komplex durch eins plus j omega C R. 
}% Speech Ende
\speech{SchaltvorgaengeACbeiRC3}{3}{%
Durch Betragsbildung erhalten wir die Amplitude U C P dach mit der Lösungen unten links
und durch Winkelbestimmung erhalten wir den Lastphasenwinkel Phi C P mit der Lösung unten rechts.
Der Winkel Phi C P ist die Phasendifferenz der Quellenspannung u q und der eingeschwungenen Spannung u C P. 
}% Speech Ende
\end{frame}

\begin{frame}[t]\ftx{Schaltvorgänge \subsecname\ \subsubsecname}%Schritt 4&5
    \textbf{4.) Inhomogene Lösung, allgemein} (Überlagerung):
    \begin{align*}
        \ecv{u_C} &= \ecv{u_{C,\mathrm{h}}} + \ecv{u_{C,p}} \\
            &= K \cdot \mathrm{e}^{-\frac{t-t_0}{\tau}} + \ecv{\hat{U}_{C,p}} \cdot \cos(\omega t + \varphi_{C,\mathrm{p}}) %\label{eq:rc:charge:ac:allg}
    \end{align*}

    \textbf{5.) Konstanten bestimmen} (Anfangsbedingung):
    \begin{align*}
        \ecv{u_C}(t_0) &\overset{!}{=} 0 \\
        K \cdot \redcancel{\mathrm{e}^{0}} + \ecv{\hat{U}_{C,p}} \cdot \cos(\omega t_0 + \varphi_{C,\mathrm{p}}) &= 0 \\
        K &= -\ecv{\hat{U}_{C,p}} \cdot \cos(\omega t_0 + \varphi_{C,\mathrm{p}})
    \end{align*}\pause
    Die eindeutige Lösung ergibt sich durch Einsetzen von $K$ in die allgemeine Lösung:
    \begin{align*}
        \ecv{u_C}(t) &= \underbrace{-\ecv{\hat{U}_{C,p}} \cdot \cos(\omega t_0 + \varphi_{C,\mathrm{p}}) \cdot \mathrm{e}^{-\frac{t-t_0}{\tau}}}_{\ecv{u_{C,\mathrm{h}}}} + \underbrace{\ecv{\hat{U}_{C,p}} \cdot \cos\left(\omega t + \varphi_{C,\mathrm{p}}\right)}_{\ecv{u_{C,p}}} %\label{eq:rc:charge:ac:uc}
    \end{align*}%

\speech{SchaltvorgaengeACbeiRC4}{1}{%
Im vierten Schritten erhalten wir die allgemeine Lösung der inhomogenen Lösung durch Überlagerung.
Die Lösung besteht wie wir sehen aus einer abklingenden Exponentialkurve und einem sinusförmigen Schwingung.
Im fünften Schritt bestimmen wir die Konstanten durch Einsetzen unserer Anfangsbedingung.
Die Kapazität sei zum Schaltzeitpunkt vollständig entladen. 
Der Exponentialterm reduziert sich zu e hoch null gleich eins und kann somit gekürzt werden. 
Durch Umstellen erhalten wir unsere Konstante K. Wie wir sehen können, entspricht unser K mit
K ist gleich **Minus** U C P Dach mal Kosinus von omega mal t **null** plus Phi C p, 
genau dem negativen Wert des eingeschwungenen Zustandes zum Schaltzeitpunkt.
Unser K kann damit je nach Schaltzeitpunkt jeden Wert zwischen dem **negativen** und **positiven** Scheitelwert von U C p annehmen.
}% Speech Ende
\speech{SchaltvorgaengeACbeiRC4}{2}{%
Die eindeutige Lösung unten ergibt sich durch Einsetzen von K in die allgemeine Lösung oben.  
Aus der Lösung können wir gut mehrere Eigenschaften des Zeitverhaltens von u C ablesen.
Aufgrund der Anfangsbedingung heben sich u C h und u C p zum Schaltzeitpunkt gegenseitig auf.
Liegt der Schaltzeitpunkt exakt im Nulldurchgang des eingeschwungenen Zustandes u C p, 
ist unser K gleich Null und es kommt zu keinem Einschwingvorgang. 
Umgekehrt wird der Einschwingvorgang maximal, wenn exakt während der Minima oder Maxima des eingeschwungenen Zustandes u C P geschalten wird!
}% Speech Ende
\end{frame}

\begin{frame}\ftx{Schaltvorgänge \subsecname\ \subsubsecname}% Vergleich verschiedene tau und maximale Auslenkung (Überschwinger)
    \centering
    \begin{minipage}{0.48\textwidth}
        \underline{Bei sinusförmiger Anregung:}\\[1em]
        Bei großen Zeitkonstanten sind\\ \textbf{Überspannungen/-ströme möglich}.\\[1em]\pause
        Erreichbar sind Überwerte bis zum\\ doppelten/dreifachen Scheitelwert.\\[1em]\pause
        Durch geschicktes Schalten sind\\ \textbf{Ausgleichsvorgänge vermeidbar}.\\[1em]\pause
        \underline{Beispiel:} Trennschalter im Netz
    \end{minipage}%
    \begin{minipage}{0.48\textwidth}\centering
        \resizebox{0.8\textwidth}{!}{\includegraphics{Tikz/pdf/plot_rc_charge_uc_ac_different_tau.pdf}}
        Vergleich: kleines und großes $\tau$
    \end{minipage}%

\speech{SchaltvorgaengeACUeberspannungenundstroeme}{1}{%
Insbesondere bei großen Zeitkonstanten sind Überspannungen und -Ströme möglich!
}% Speech Ende
\speech{SchaltvorgaengeACUeberspannungenundstroeme}{2}{%
Durch ungünstige Schaltzeitpunkte während der Minima oder Maxima des eingeschwungenen Zustandes, 
können so Werte bis zum doppelten oder dreifachen Scheitelwert erreicht werden. 
Der dreifache Scheitelwert ist möglich, wenn die Anfangsbedingung wäre, dass z.B. u C von t null gleich der einfach Scheitelwert ist. 
}% Speech Ende
\speech{SchaltvorgaengeACUeberspannungenundstroeme}{3}{%
Durch geschicktes Schalten lassen sich Ausgleichsvorgänge aber auch vollständig vermeiden. 
}% Speech Ende
\speech{SchaltvorgaengeACUeberspannungenundstroeme}{4}{%
Ein praktisches Beispiel sind Trennschalter im Netz, die idealerweise ohne Last und deshalb während Nulldurchgängen schalten.
Rechts sehen wir exemplarisch zwei Schaltvorgänge bei ungünstigsten Schaltzeitpunkten mit verschiedenen Zeitkonstanten zum Vergleich. 
Oben ist die Anregung zu sehen, in der Mitte ist der Schaltvorgang mit kleinerem Tau dargestellt 
und unten der bei Schaltvorgang mit Tau deutlich größer als die Periodendauer groß T. 
Beim kleinen Tau kommt es bereits zu einer leichten Überspanung, die schnell abklingt.
Beim großen Tau wird nahezu der doppelte negative Scheitelwert erreicht und über mehrere Periodendauern Überspannungen erzeugt.
}% Speech Ende
\end{frame}
}% Ende Folien only



\subsubsection{bei Induktivitäten}
\label{sec:schaltvorgaengezeitbereich:ac:l}
%\begin{frame}\ftx{\subsubsecname}
\s{%
    Beispiel \ref{bsp:rl:chargeac} zeigt exemplarisch die Berechnung eines Schaltvorgangs für ein RL-Glied bei Anregung mit Wechselspannung nach Merksatz \ref{merk:berechnungsverfahren:zeitbereich}.

    Die Berechnung von Schaltvorgängen bei sinusförmiger Anregung für Induktivitäten erfolgt analog zu der Berechnung für Kapazitäten in Kapitel \ref{sec:schaltvorgaengezeitbereich:ac:c}.
    Es gelten die gleichen Besonderheiten bezüglich Schaltzeitpunkten und möglicher transienter Spitzenströme und -spannungen wie bei Kapazitäten.

    Im Beispiel ist die DGL der partikulären Lösung nach Gleichung \ref{eq:grundlagen:dgl:komplex} transformiert:
    \begin{align*}
        &\text{DGL$_{\mathrm p}$:}&
            \frac{L}{R} \cdot \dt \eci{i_{\mathrm p}} + \eci{i_{\mathrm p}} &= \frac{\ecv{u_q}}{R} &
            &\text{mit}\quad \ecv{u_q} = \Re\left\{\ecv{\underline{\hat{U}}_q}\cdot\mathrm{e}^{\mathrm{j}\omega t}\right\} \vphantom{\Bigg|}\\
        &\text{transformiert:}&
            \Leftrightarrow\quad
            \frac{L}{R} \cdot \mathrm{j}\omega\, \eci{\underline{\hat{I}}_{\mathrm p}} + \eci{\underline{\hat{I}}_{\mathrm p}} &= \frac{\ecv{\underline{\hat{U}}_q}}{R} &
            &\text{mit}\quad \ecv{\underline{\hat{U}}_q} = \ecv{\hat{U}_q}\cdot\mathrm{e}^{\mathrm{j}\varphi_q} \vphantom{\Bigg|}
    \end{align*}
    Der Ansatz bietet eine einfache Umstellung nach dem komplexen Amplitudenzeiger der partikulären Lösung und ist
    prinzipiell gleichwertig mit der Herleitung über komplexe Spannungsteiler wie sie in Gleichung \ref{eq:rc:charge:ac:komplex} am Beispiel des RC-Gliedes gezeigt ist.

    \newpage
    \begin{bsp}{RL-Glied, Schaltvorgang mit Wechselspannungsquelle}{rl:chargeac}
        Bestimmung des Strom $\eci{i}$ durch die Induktivität $L$ in Reihe mit dem ohmschen Widerstand $R$ bei Zuschalten der idealen Wechselspannungsquelle
        $\ecv{u_q} = \ecv{\hat{U}_q} \cdot \cos(\omega t + \varphi_q)$ ab Schaltzeitpunkt $t=t_0$. Zuvor fließt kein Strom. \\[-6pt]
        \begin{minipage}{\textwidth}\centering
            \begin{minipage}[t][][t]{0.48\textwidth}\centering\vspace{0cm}% vspace: input empty space before tikzpicture for top alignment
                \includegraphics{Tikz/pdf/circ_switchable_rl_charge_ac.pdf}%
                \begin{equation*}\begin{aligned}
                \text{gegeben:}&&       &\eci{i}(t_0),\ \ecv{u_q},\ R,\ L \\
                \text{gesucht:}&&       &\eci{i}(t),\ \ecv{u_L}(t)
                \end{aligned}\end{equation*}
            \end{minipage}
            \begin{minipage}[t][][t]{0.48\textwidth}\centering\vspace{0cm}% vspace: input empty space before tikzpicture for top alignment
                \includegraphics{Tikz/pdf/plot_rl_charge_il_ac_different_t0_short.pdf}
            \end{minipage}
        \end{minipage}

        \textbf{1.) DGL aufstellen:} von $\eci{i}$ für $t \geq t_0$, Lösung aus Bsp. \ref{bsp:rl:chargedc:dgli}
        \begin{align*}
            &\text{DGL}:&&&
                \frac{L}{R} \cdot \dt \eci{i}  + \eci{i}   &= \frac{\ecv{u_q}}{R}&&&
                &\text{mit}\quad \ecv{u_q} = \ecv{\hat{U}_q} \cdot \cos(\omega t + \varphi_q)
            \\[-15pt]&\hphantom{R1oooooooooo}&&&\hphantom{L3ooooooooooo}&\hphantom{R3ooooooooooooooo}&&&&\hphantom{R5ooooooooooooooooooooooo}\nonumber% D Dummy-Zeile RL-AC Step 1-3 für horizontales Alignment zwischen align-Umgebungen [15pt hoch]
        \end{align*}%
        \textbf{2.) Homogene Lösung, allg.:} (flüchtig), Lösung aus Bsp. \ref{bsp:rl:chargeac} für Schaltzeitpunkt $t_0$
        \begin{align*}
            &\text{hom. Lsg.:} &&&
                \eci{i_{\mathrm{h}}}&= K \cdot \mathrm{e}^{-\frac{t-t_0}{\tau}} &&&
                &\text{mit}\quad \tau = \frac{L}{R}
            \\[-15pt]&\hphantom{R1oooooooooo}&&&\hphantom{L3ooooooooooo}&\hphantom{R3ooooooooooooooo}&&&&\hphantom{R5ooooooooooooooooooooooo}\nonumber% D Dummy-Zeile RL-AC Step 1-3 für horizontales Alignment zwischen align-Umgebungen [15pt hoch]
        \end{align*}%
        \textbf{3.) Partikulare Lösung:} (eingeschwungen) mit $t \to \infty$, kompl. Wechselstromrechnung%
        \begin{align*}%
                &\text{transf.}:&&&
                \frac{L}{R} \cdot \mathrm{j}\omega\, \eci{\underline{\hat{I}}_{\mathrm p}} + \eci{\underline{\hat{I}}_{\mathrm p}} &= \frac{\ecv{\underline{\hat{U}}_q}}{R} &&&
                    &\text{mit}\quad \ecv{\underline{\hat{U}}_q}=\ecv{\hat{U}_q} \cdot \mathrm{e}^{\mathrm{j}\varphi_q}\\
                &\text{Zeiger:}&&&
                    \eci{\underline{\hat{I}}_{\mathrm p}} &= \frac{\ecv{\underline{\hat{U}}_q}}{R} \cdot \frac{1}{1 + \mathrm{j} \omega \frac{L}{R}} &&&
                    &\text{mit}\quad \eci{\underline{\hat{I}}_{\mathrm p}} = \eci{\hat{I}_{\mathrm p}}\cdot\mathrm{e}^{\mathrm{j}\varphi_I}\\
                &&&&
                    \eci{\hat{I}_{\mathrm p}} &=  \frac{\ecv{\hat{U}_q}/R}{\sqrt{1 + \left(\omega \frac{L}{R}\right)^2}} &&&
                    &\quad \varphi_I = \varphi_q - \arctan\left(\omega \frac{L}{R}\right) \\
                &\text{part. Lsg.}:&&&
                    \eci{i_{\mathrm p}} &= \eci{\hat{I}_{\mathrm p}} \cdot \cos(\omega t + \varphi_I) &&&
                    &\text{mit}\quad \eci{i_{\mathrm p}} = \Re\left\{\eci{\underline{\hat{I}}_{\mathrm p}} \cdot \mathrm{e}^{\mathrm{j}\omega t}\right\} \vphantom{\Bigg|}
            \\[-15pt]&\hphantom{R1oooooooooo}&&&\hphantom{L3ooooooooooo}&\hphantom{R3ooooooooooooooo}&&&&\hphantom{R5ooooooooooooooooooooooo}\nonumber% D Dummy-Zeile RL-AC Step 1-3 für horizontales Alignment zwischen align-Umgebungen [15pt hoch]
        \end{align*}%
        \textbf{4.) Inhomogene Lösung, allg.:} (Überlagerung)%
        \begin{align*}
                &\text{allg. Lsg.:}&&&
                    \eci{i} &=
                    \underbrace{ \vphantom{\big|}
                        K \cdot \mathrm{e}^{-\frac{t-t_0}{\tau}}
                    }_{\eci{i_{\mathrm h}}}
                    +
                    \underbrace{ \vphantom{\big|}
                        \eci{\hat{I}_{\mathrm p}} \cdot \cos(\omega t + \varphi_I)
                    }_{\eci{i_{\mathrm p}}}&&&
            \\[-15pt]&\hphantom{R1oooooooooo}&&&\hphantom{L3oooooo}&\hphantom{R3ooooooooooooooooooooooooooooooooooooooooooo}&&&&\hphantom{R5}\nonumber% Dummy-Zeile RL-AC Step 4-5 für horizontales Alignment zwischen align-Umgebungen [15pt hoch]
        \end{align*}
        \textbf{5.) Konstanten bestimmen:} (AB einsetzen)%
        \begin{align*}
            &\text{Anfangsbed.:}&&&
                \eci{i}(t_0) &= K \cdot \redcancel{\mathrm{e}^0} + \eci{\hat{I}_{\mathrm p}} \cdot \cos(\omega t_0 + \varphi_I) \overset{!}{=} 0 \\
            &&&&
                \Rightarrow\quad K &= -\eci{\hat{I}_{\mathrm p}} \cdot \cos(\omega t_0 + \varphi_I) \\
            &\text{eindeut. Lsg.:}&&&
                \eci{i}(t) &= -\eci{\hat{I}_{\mathrm p}} \cdot \cos(\omega t_0 + \varphi_I) \cdot \mathrm{e}^{-\frac{t-t_0}{\tau}} + \eci{\hat{I}_{\mathrm p}} \cdot \cos(\omega t + \varphi_I)
            \\[-15pt]&\hphantom{R1oooooooooo}&&&\hphantom{L3oooooo}&\hphantom{R3ooooooooooooooooooooooooooooooooooooooooooo}&&&&\hphantom{R5}\nonumber% Dummy-Zeile RL-AC Step 4-5 für horizontales Alignment zwischen align-Umgebungen [15pt hoch]
        \end{align*}
        Spannung $\ecv{u_L}$ über der Induktivität $L$ über $\ecv{u_L} = L \cdot \dt \eci{i}$ berechnen.
        \begin{align*}
            &\text{Spannung:}&&&
                \ecv{u_L}(t) &= R \cdot \eci{\hat{I}_{\mathrm p}} \cdot \cos(\omega t_0 + \varphi_I) \cdot \mathrm{e}^{-\frac{t-t_0}{\tau}} - \omega L \cdot \eci{\hat{I}_{\mathrm p}} \cdot \sin(\omega t + \varphi_I)
        \end{align*}
    \end{bsp}
}% end skript only
%\end{frame}

\b{% Anfang Folien only Multi Frame -- RL AC - charge
\begin{frame}[t]\ftx{Schaltvorgänge \subsecname \, \subsubsecname}
% Obere Hälfte
\begin{minipage}{\textwidth}\centering%
\begin{minipage}[t][][t]{0.48\textwidth}\centering\vspace{0cm}% vspace: input empty space before tikzpicture for top alignment%
    \includegraphics{Tikz/pdf/circ_switchable_rl_charge_ac.pdf}
    \begin{equation*}\begin{aligned}
        &\text{gegeben:}&       &\eci{i}(t_0),\ \ecv{u_q}(t),\ R,\ L,\ \omega \\
        &\text{gesucht:}&       &\eci{i}(t)\quad\text{für}\quad t \geq t_0
    \end{aligned}\end{equation*}
\end{minipage}\pause%
\begin{minipage}[t][][t]{0.48\textwidth}\centering\vspace{0cm}% vspace: input empty space before tikzpicture for top alignment%
    \resizebox{0.75\textwidth}{!}{\includegraphics{Tikz/pdf/plot_rl_charge_il_ac_different_t0_short.pdf}}%
\end{minipage}%
\end{minipage}\pause\hfill%
% Untere Hälfte
\resizebox{0.95\textwidth}{!}{% to fit on slide
\begin{minipage}{\textwidth}\vspace{-0.75cm}%
\begin{align*}
&\textbf{1. DGL}&                   \frac{\ecv{u_q}}{R} &= \frac{L}{R} \cdot \dt\, \eci{i} + \eci{i} =\frac{\ecv{\hat{U}_q}}{R} \cdot \cos(\omega t + \varphi_q)\\
&\textbf{2. Homogene Lösung}&       \eci{i_{\mathrm{h}}}&= K \cdot \mathrm{e}^{-\frac{t-t_0}{\tau}}\quad\text{mit}\quad \tau = \frac{L}{R}\\
&\textbf{3. Partikulare Lösung}&    \eci{i_{\mathrm p}} &= \eci{\hat{I}_{\mathrm p}} \cdot \cos(\omega t + \varphi_I)\\
&\textbf{4. Inhomogene Lösung}&     \eci{i} &= K \cdot \mathrm{e}^{-\frac{t-t_0}{\tau}} + \eci{\hat{I}_{\mathrm p}} \cdot \cos(\omega t + \varphi_I)\\
&\textbf{5. Konstanten bestimmen}&  \eci{i} &= \eci{\hat{I}_{\mathrm{p}}} \cdot \left( - \cos(\omega t_0 + \varphi_I) \cdot \mathrm{e}^{-\frac{t-t_0}{\tau}} + \cos(\omega t + \varphi_I) \right)%Schritt 5 pass
\end{align*}
\end{minipage}
}%

\speech{SchaltvorgaengeACbeiRL1}{1}{%
Sehen wir uns nun zum Vergleich das Schaltverhalten einer R L Serienschaltung bei Anregung mit sinusförmiger Wechselspannung an. 
Analog zu den anderen Beispielen sehen Sie oben links die Schaltung. 
Das R L Glied ist zunächst kurzgeschlossen und wird zum Schaltzeitpunkt t null an eine ideale Wechselspannungsquelle u q angeschlossen. 
Gesucht ist i von t ab dem Schaltzeitpunkt t null. Zum Zeitpunkt t null fließt noch kein Strom, die Wechselspannung, R und L sind gegeben.
Die Berechnung erfolgt analog zum vorigen Beispiel am R C Glied. 
}% Speech Ende
\speech{SchaltvorgaengeACbeiRL1}{2}{%
Oben rechts sind die sich Zeitverläufe von der Anregung u q der gesuchten Größe i von t und für einen Beispiel Schaltzeitpunkt t Null dargestellt. 
}% Speech Ende
\speech{SchaltvorgaengeACbeiRL1}{3}{%
Unten sehen Sie zur Übersicht und Zusammenfassung das Berechnungsverfahren mit Teillösungen.
Diese werden wir nun im Einzelnen ausführlicher berechnen. 
}% Speech Ende
\end{frame}

\begin{frame}[t]\ftx{Schaltvorgänge \subsecname \, \subsubsecname}
    \begin{minipage}{\textwidth}\centering%
        \begin{minipage}[t][][t]{0.48\textwidth}\centering\vspace{0cm}% vspace: input empty space before tikzpicture for top alignment%
            \includegraphics{Tikz/pdf/circ_switchable_rl_charge_ac.pdf}
        \end{minipage}%
        \begin{minipage}[t][][t]{0.48\textwidth}\centering\vspace{0cm}% vspace: input empty space before tikzpicture for top alignment%
            \vspace{0.8cm}
            \textbf{DGL 1. Ordnung:}\\[0.85em]
            $a_1 \cdot \dt y(t) + a_0 \cdot y(t)=b$\\
        \end{minipage}%
    \end{minipage}%
    \vspace{0.85cm}
    \textbf{1.) DGL aufstellen:} für $\eci{i}$ mit $t\geq t_0$
    \begin{align*}
        &\text{DGL}:&&&
            \frac{L}{R} \cdot \dt \eci{i}  + \eci{i}   &= \frac{\ecv{u_q}}{R}&&&
            &\text{mit}\quad \ecv{u_q} = \ecv{\hat{U}_q} \cdot \cos(\omega t + \varphi_q)
    \end{align*}%
    \pause% <-- overlay

    \textbf{2.) Homogene Lösung, allg.:} (flüchtig, für Schaltzeitpunkt $t_0$)
    \begin{align*}
        &\text{hom. Lsg.:} &&&
            \eci{i_{\mathrm{h}}}&= K \cdot \mathrm{e}^{-\frac{t-t_0}{\tau}} &&&
            &\text{mit}\quad \tau = \frac{L}{R}
    \end{align*}%

\speech{SchaltvorgaengeACbeiRL2}{1}{%
Das Aufstellen der D G L erfolgt wieder analog zum DC Beispiel. 
Genau wie beim R C Beispiel ist unsere Störgröße sinusförmig. Hier ist unsere Störgröße der Strom u q durch R
mit u q gleich U q Dach mal Kosinus von omega t plus Phi q.
}% Speech Ende
\speech{SchaltvorgaengeACbeiRL2}{2}{%
Die homogene Lösung können wir wieder aus unserem D C Beispiel übernehmen, hier mit allgemeinem Schaltzeitpunkt t null.
Die Zeitkonstante ergibt sich wieder durch Tau gleich L durch R. 
}% Speech Ende
\end{frame}

\begin{frame}\ftx{Schaltvorgänge \subsecname \, \subsubsecname}
    \textbf{3.) Partikulare Lösung:} (eingeschwungen, mit $t \to \infty$)%
    \begin{align*}%
        &\text{DGL transf.}:&&&
        \frac{L}{R} \cdot \mathrm{j}\omega\, \eci{\underline{\hat{I}}_{\mathrm p}} + \eci{\underline{\hat{I}}_{\mathrm p}} &= \frac{\ecv{\underline{\hat{U}}_q}}{R} &&&
            &\text{mit}\quad \ecv{\underline{\hat{U}}_q}=\ecv{\hat{U}_q} \cdot \mathrm{e}^{\mathrm{j}\varphi_q}\\
        &\text{Zeiger:}&&&
            \eci{\underline{\hat{I}}_{\mathrm p}} &= \frac{\ecv{\underline{\hat{U}}_q}}{R} \cdot \frac{1}{1 + \mathrm{j} \omega \cdot\frac{L}{R}} &&&
            &\text{mit}\quad \eci{\underline{\hat{I}}_{\mathrm p}} = \eci{\hat{I}_{\mathrm p}}\cdot\mathrm{e}^{\mathrm{j}\varphi_I}\\
        &&&&
            \eci{\hat{I}_{\mathrm p}} &=  \frac{\ecv{\hat{U}_q}}{R}\cdot\frac{1}{\sqrt{1 + \left(\omega \cdot\frac{L}{R}\right)^2}}\\
        &&&&
            \varphi_I &= \varphi_q - \arctan\left(\omega \cdot\frac{L}{R}\right) \\
        &\text{part. Lsg.}:&&&
            \eci{i_{\mathrm p}} &= \eci{\hat{I}_{\mathrm p}} \cdot \cos(\omega t + \varphi_I) &&&
            &\text{mit}\quad \eci{i_{\mathrm p}} = \Re\left\{\eci{\underline{\hat{I}}_{\mathrm p}} \cdot \mathrm{e}^{\mathrm{j}\omega t}\right\} \vphantom{\Bigg|}
    \end{align*}%

\speech{SchaltvorgaengeACbeiRL3}{1}{%
Für die partikuläre Lösung wenden wir wieder die komplexe Wechselstromrechnung an. 
Statt eines komplexen Spannungsteilers, transformieren wir die D G L in den Bildbereich und ersetzen dabei die Differentialoperatoren d nach d t durch j omega. 
Durch Ausklammern und Umstellen erhalten wir unseren komplexen Amplitudenzeiger 
I p Dach komplex gleich U q Dach komplex durch R mal eins im Zähler geteilt durch eins plus jot omega mal L durch R im Nenner.
Jetzt wieder den Betrag und die Phasenlage für I p bestimmen. 
Achtung: Phi q ist beliebig wählbar, da t null variabel ist, dadurch kann phi q zur Vereinfachung auch null gesetzt werden.
}% Speech Ende
\end{frame}

\begin{frame}\ftx{Schaltvorgänge \subsecname \, \subsubsecname}
    \textbf{4.) Inhomogene Lösung, allg.:} (Überlagerung)%
    \begin{align*}
        &\text{allg. Lsg.:}&&&
            \eci{i} &=
            \underbrace{ \vphantom{\big|}
                K \cdot \mathrm{e}^{-\frac{t-t_0}{\tau}}
            }_{\eci{i_{\mathrm h}}}
            +
            \underbrace{ \vphantom{\big|}
                \eci{\hat{I}_{\mathrm p}} \cdot \cos(\omega t + \varphi_I)
            }_{\eci{i_{\mathrm p}}}&&&
        \\[-15pt]&\hphantom{R1oooooooooo}&&&\hphantom{L3oooooo}&\hphantom{R3oooooooooooooooooooooooooooooooooooooooooooooo}\nonumber% Dummy-Zeile RL-AC Step 4-5 Folien für horizontales Alignment zwischen align-Umgebungen [15pt hoch]
    \end{align*}\pause

    \textbf{5.) Konstanten bestimmen:} (AB einsetzen)%
    \begin{align*}
        &\text{Anf.bed.:}&&&
            \eci{i}(t_0) &= K \cdot \redcancel{\mathrm{e}^0} + \eci{\hat{I}_{\mathrm p}} \cdot \cos(\omega t_0 + \varphi_I) \overset{!}{=} 0 \\
        &&&&
            \Rightarrow\quad K &= -\eci{\hat{I}_{\mathrm p}} \cdot \cos(\omega t_0 + \varphi_I) \\
        &\text{eind. Lsg.:}&&&
            \eci{i}(t) &= -\eci{\hat{I}_{\mathrm p}} \cdot \cos(\omega t_0 + \varphi_I) \cdot \mathrm{e}^{-\frac{t-t_0}{\tau}} + \eci{\hat{I}_{\mathrm p}} \cdot \cos(\omega t + \varphi_I)
        \\[-15pt]&\hphantom{R1oooooooooo}&&&\hphantom{L3oooooo}&\hphantom{R3oooooooooooooooooooooooooooooooooooooooooooooo}\nonumber% Dummy-Zeile RL-AC Step 4-5 Folien für horizontales Alignment zwischen align-Umgebungen [15pt hoch]
    \end{align*}\pause

    Spannung $\ecv{u_L}$ mit $\ecv{u_L} = L \cdot \dt \eci{i}$ berechnen:
    \begin{align*}
        &\text{Spannung:}&&&
            \ecv{u_L}(t) &= R \cdot \eci{\hat{I}_{\mathrm p}} \cdot \cos(\omega t_0 + \varphi_I) \cdot \mathrm{e}^{-\frac{t-t_0}{\tau}} - \omega L \cdot \eci{\hat{I}_{\mathrm p}} \cdot \sin(\omega t + \varphi_I)
    \end{align*}%

\speech{SchaltvorgaengeACbeiRL4}{1}{%
Die allgemeine Lösung erhalten wir wieder durch Überlagerung der homogenen und der partiklären Lösung. 
}% Speech Ende
\speech{SchaltvorgaengeACbeiRL4}{2}{%
Als Anfangsbedingung sei angenommen, dass i von t Null gleich null ist, das heißt zum Einschaltzeitpunkt kein Strom fließt. 
Aus dieser Anfangsbedingung können wir die Konstante K bestimmen.
Durch Einsetzen in die allgemeine Lösung oben erhalten wir wieder unsere eindeutige Lösung.
Wie auch beim R C Beispiel kann unser K je nach Schaltzeitpunkt alle Werte zwischen 
dem negativen und dem positiven Scheitelwert I p Dach annehmen.
}% Speech Ende
\speech{SchaltvorgaengeACbeiRL4}{3}{%
Die Spannung u L an der Induktivität erhalten wir mit u L ist gleich L mal d nach d t von i. 
**Achtung:** Anders als der Strom, kann die Spannung u L springen! 
Wir sehen das an der Gleichung, dass sich der Exponentialterm und der sinusförmige Term zum Schaltzeitpunkt nicht direkt aufheben. 
Zur Übung können Sie:
* 1. den Zeitverlauf für die Spannung u L in diesem Beispiel skizzieren,
* 2. den Zeitverlauf für den Strom beim R C Beispiel skizzieren und
* 3. Überlegen, welche Zeitverläufe sich beim Zurückschalten von Wechselspannungsquelle zu Kurzschluss ergeben.
}% Speech Ende
\end{frame}

}% end Folien only
%%%%%%%%%%%%%%%%%%%%%%%%%%%%%%%%%%%%%%%%%%%%%%%%%%%%%%%%%%%%%%%%%%%%%%%%%%%%%%%%%%%%%%%%%%%%%%%%%%

\subsection{bei RLC-Serienschwingkreisen}
\label{sec:schaltvorgaengezeitbereich:rlc}
\begin{frame}\ftx{\subsecname}
\s{%
    Systeme mit zwei unterschiedlichen Energiespeichern (Induktivität $L$ und Kapazität $C$) sind
    prinzipiell schwingungsfähig. Daher werden RLC-Glieder auch als Schwingkreise oder Resonanzkreise bezeichnet.
    Die Schwingungsfähigkeit lässt sich anhand der Eigenwerte bestimmen und beschreiben.
    Ein anderer Ansatz ist die Betrachtung der (Schwingkreis-)Güte $Q$. [Vgl. Modul 8]%\todo{link}

    Abbildung \ref{fig:rlc:charge} zeigt das Schaltbild eines RLC-Gliedes beim Laden mit konstanter Gleichspannung $U_q$.
    Daneben sind die Zeitverläufe der Kondensatorspannung für die drei möglichen Fälle dargestellt.  Das sind
    der \textbf{aperiodischen Fall}\index{Fall>aperiodischer Fall},
    der \textbf{aperiodischen Grenzfall}\index{Fall>aperiodischer Grenzfall} und
    der \textbf{periodischen Fall}\index{Fall>periodischer Fall}
    die in Kapitel \ref{sec:schaltvorgaengezeitbereich:rlc:fallunterscheidung} näher beschrieben sind.

    \begin{figure}[H]\centering
        \begin{subfigure}{0.48\textwidth}\centering
            \includegraphics{Tikz/pdf/circ_switchable_rlc_charge_detail.pdf}
            \caption{Schaltbild}
            \label{fig:circ:rlc:charge}
        \end{subfigure}
        \begin{subfigure}{0.48\textwidth}\centering
            \includegraphics{Tikz/pdf/plot_rlcs_charge_dc_uc.pdf}
            \caption{Spannungsverlauf, Fallunterscheidung}
            \label{fig:plot:rlc:charge}
        \end{subfigure}
        \caption{RLC-Glied, Ladevorgang mit konstanter Gleichspannung}
        \label{fig:rlc:charge}
    \end{figure}
    Die Berechnung von Schaltvorgängen bei RLC-Gliedern erfolgt analog zu den Beispielen für RC- und RL-Glieder aus den vorigen Kapiteln.

    \textbf{1.) DGL aufstellen:} für $\ecv{u_C}$ für $t \geq 0$
    \begin{align}
        \ecv{u_L} + \ecv{u_R} +\ecv{u_C} &= \ecv{u} \nonumber\\
        L \cdot \dt \eci{i} + R \cdot \eci{i} + \ecv{u_C}&= \ecv{U_q} \nonumber\\
        LC \cdot \dt[2] \ecv{u_C} + RC \cdot \dt \ecv{u_C} + \ecv{u_C} &= \ecv{U_q} \label{eq:dgl:rlc:dc}
    \end{align}
    Aufgrund der zwei verschiedenen Energiespeichern ergibt sich eine DGL 2. Ordnung.

    \textbf{2.) Homogene Lösung, allg.:} (flüchtig) (ohne Störterm)

    Die Form der homogenen Lösung ist abhängig von der Art der Eigenwerte.
    \begin{align}
        &\text{hom. DGL:}&
            LC \cdot \dt[2] \ecv{u_{C,\mathrm{h}}} + RC \cdot \dt \ecv{u_{C,\mathrm{h}}} + \ecv{u_{C,\mathrm{h}}} &= 0
                &&\text{mit}\quad \ecv{u_{C,\mathrm{h}}} = K \cdot \mathrm{e}^{\lambda t} \label{eq:dgl:rlc:homo}\\[6pt]
        &\text{char. Polynom:}&
            \lambda^2 + \frac{R}{L} \cdot \lambda + \frac{1}{LC} &= 0 \nonumber\\[6pt]
        &\text{Eigenwerte:}&
            \lambda_{1/2} = -\frac{R}{2L} \pm \sqrt{\left(\frac{R}{2L}\right)^2 - \frac{1}{LC}}&
                &&\text{mit}\quad D = \left(\frac{R}{2L}\right)^2 - \frac{1}{LC}
            %\lambda_{1/2} = -\frac{R}{2L} \pm \sqrt{\smash{%
            %    \underbrace{%
            %        \left(\frac{R}{2L}\right)^2 - \frac{1}{LC}%
            %    }_{\mathclap{\text{Diskriminante }D}}%
            %}\vphantom{\left(\frac{R}{L}\right)^2}}
        \label{eq:dgl:rlc:lambda}
        %\\[-10pt]\nonumber % Platzhalterzeile wegen underbrace Text
    \end{align}
    Durch Anwenden der $p$-$q$-Formel (Mitternachtsformel) ergeben sich für die \textbf{Eigenwerte} (EW) $\lambda_{1/2}$
    je nach Wurzelterm entweder zwei unterschiedliche oder zwei identische Werte.
    Die homogenen Lösung nimmt daher, wie in Kapitel \ref{sec:grundlagen:dgl:homogeneloesung} beschrieben ist, eine der folgenden zwei Formen an:
    \begin{equation}
        \text{hom. Lsg.:}\qquad\qquad\qquad
        \ecv{u_{C,\mathrm{h}}} =
        \begin{cases}
            K_1 \cdot \mathrm{e}^{\lambda_1 t} + K_2 \cdot \mathrm{e}^{\lambda_2 t}  & \text{für}\quad\lambda_1 \neq \lambda_2  \\
            (K_1 + K_2 \cdot t) \cdot \mathrm{e}^{\lambda t} & \text{für}\quad\lambda_1 = \lambda_2 = \lambda
        \end{cases}
        \qquad\qquad\qquad \label{eq:dgl:rlc:homolsg:cases}
    \end{equation}
    Die Fallunterscheidung kann anhand der sogenannten \textbf{Diskriminante} $D$ \index{Größe>Diskriminante} vorgenommen werden.
}% Ende Skript only
\b{%
\begin{minipage}{\textwidth}\centering
    \begin{minipage}[t][3.5cm][]{0.48\textwidth}\centering
        \resizebox{\textwidth}{!}{\includegraphics{Tikz/pdf/circ_switchable_rlc_charge_detail.pdf}}%
        \vspace{5pt}

        Wegen $L$ und $C$ $\Rightarrow$ Schwingkreis
    \end{minipage}%
    \begin{minipage}[t][3.5cm][]{0.48\textwidth}\centering
        \resizebox{\textwidth}{!}{\includegraphics{Tikz/pdf/plot_rlcs_charge_dc_uc.pdf}}%
    \end{minipage}
\end{minipage}\vspace{2pt}

    %\textbf{DGL}: inhomogen, linear, gewöhnlich, 2. Ordnung, mit konst. Koeffizienten
    \begin{align*}
    &\text{\textbf{DGL} (für $\ecv{u_C}$ für $t \geq 0$):}&    LC \cdot \dt[2] \ecv{u_C} + RC \cdot \dt \ecv{u_C} + \ecv{u_C} &= \ecv{U_q}\\[-20pt]
    &\hphantom{R1oooooooooooooooooooo}&\hphantom{L2oooooooooooooooooooooooooooooo}&\hphantom{R2ooo} % Dummy-Zeile für horizontales Alignment zwischen Frames
    \end{align*}

    \begin{table}[H]\centering
        \begin{tabular}{lccc} \toprule
            \textbf{Fallunterscheidung} & \textbf{Dämpfung} & \textbf{Diskrimi.} & \textbf{Güte} \\ \midrule
            aperiodischer Fall          & stark             & $D > 0$               & $Q < 0,5$ \\
            aperiodischer Grenzfall     & kritisch          & $D = 0$               & $Q = 0,5$ \\
            periodischer Fall           & schwach           & $D < 0$               & $Q > 0,5$ \\ \bottomrule
        \end{tabular}
    \end{table}
}% Ende Folien only
\end{frame}%

\begin{frame}\ftx{\subsecname}% Fortsetzung für Folien
\b{%
\begin{minipage}{\textwidth}\centering
    \begin{minipage}[t][3.5cm][]{0.48\textwidth}\centering
        \resizebox{\textwidth}{!}{\includegraphics{Tikz/pdf/circ_switchable_rlc_charge_detail.pdf}}%
        \vspace{5pt}

        Ansatz:\quad  $\ecv{u_{C,\mathrm{h}}} = K \cdot \mathrm{e}^{\lambda t}$
    \end{minipage}%
    \begin{minipage}[t][3.5cm][]{0.48\textwidth}\centering
        %        \includegraphics{Tikz/pdf/plot_rlcs_charge_dc_uc.pdf}%
    \begin{align*}
        \textbf{1.) DGL:}\qquad
         \ecv{u_L} + \ecv{u_R} +\ecv{u_C} &= \ecv{u} \nonumber\\
        L \cdot \dt \eci{i} + R \cdot \eci{i} + \ecv{u_C}&= \ecv{U_q} \nonumber\\
        LC \cdot \dt[2] \ecv{u_C} + RC \cdot \dt \ecv{u_C} + \ecv{u_C} &= \ecv{U_q} %\label{eq:dgl:rlc:dc}
    \end{align*}
    \end{minipage}
\end{minipage}\vspace{2pt}

\begin{align*}
    &\textbf{homogene DGL:}&
        LC \cdot \dt[2] \ecv{u_{C,\mathrm{h}}} + RC \cdot \dt \ecv{u_{C,\mathrm{h}}} + \ecv{u_{C,\mathrm{h}}} &= 0  %\label{eq:dgl:rlc:homo}
        \\[-20pt]
        &\hphantom{R1oooooooooooooooooooo}&\hphantom{L2oooooooooooooooooooooooooooooo}&\hphantom{R2ooo} % Dummy-Zeile für horizontales Alignment zwischen Frames
        \\[6pt]
    &\text{char. Polynom:}&
        \lambda^2 + \frac{R}{L} \cdot \lambda + \frac{1}{LC} &= 0 \nonumber\\[6pt]
    &\textbf{Eigenwerte:}&
        \lambda_{1/2} = -\frac{R}{2L} \pm \sqrt{
            \smash{\underbrace{% smash damit Größe der Wurzel nicht angepasst wird
            \left(\frac{R}{2L}\right)^2 - \frac{1}{LC}
            }_{\mathclap{\text{Diskriminante}\ D}}}
            \vphantom{\left(\frac{R}{2L}\right)^2 - \frac{1}{LC}}% vphantom um Größe der Wurzel manuell an Term unter der Wurzel anzupassen
        }&
\end{align*}
}% Ende Folien only
\end{frame}

%-------------------------------------------------------------------------------------------------%

\subsubsection[Fallunterscheidung]{Fallunterscheidung bei RLC-Serienschwingkreis}
\label{sec:schaltvorgaengezeitbereich:rlc:fallunterscheidung}
\begin{frame}[t]\ftx{\subsubsecname}%
\s{%
Die Diskriminante $D$ ist der Term unter der Wurzel in Gleichung \ref{eq:dgl:rlc:lambda} zur Bestimmung von $\lambda_{1/2}$.
\vphantom{$\bigg|$}
\begin{align}
    &&%
        \lambda_{1/2} &= -\frac{R}{2L} \pm \sqrt{\smash{%
            \underbrace{%
                \left(\frac{R}{2L}\right)^2 - \frac{1}{LC}%
            }_{\mathclap{\text{Diskriminante }D}}%
        }\vphantom{\left(\frac{R}{L}\right)^2}
        }\label{eq:dgl:rlc:lambda:allg}%
        \\[-10pt]\nonumber % Platzhalterzeile wegen underbrace Text
        % Dummyline: manual fixed spacing for inter-environment-alignment: zero height after regular row with regular height (15pt), uncomment \nonumber to check height
        \\[-15pt] &\hphantom{1rrrrrrrrrrrrrrrr} &   \hphantom{2llllll}&\hphantom{\mathrel{\phantom{=}}2rrrrrrrrrrrrrrrrrrrrrrrrrrr}     &&\hphantom{3rrrrrrrrrrrrrrrrrrrrrrrr} \nonumber
    \end{align}

%Im periodischen Fall (auch Schwingfall genannt) ist $D$ negativ, es gibt zwei paarweise konjugierte, komplexe EW.
%Im aperiodischen Grenzfall ist $D$ null, es zwei identische negativen, reelle EW.
%Im aperiodischen Fall (auch Kriechfall genannt) ist $D$ positiv, es gibt zwei unterschiedliche, negative, relle EW.

Tabelle \ref{tab:rlc:fallunterscheidung} listet alle drei Fälle mit Art der Eigenwerte und relative Aussagen bezüglich der Dämpfung des Schwingkreises.
Als Unterscheidungsmerkmal sind die Wertebereiche der Diskriminante $D$ respektive der Schwingkreisgüte $Q_S$ (Vgl. Modul 8) mit angegeben.

\begin{table}[H]\centering
    \caption{Fallunterscheidung beim RLC-Schwingkreis}
    \label{tab:rlc:fallunterscheidung}
    \begin{tabular}{lcccc} \toprule
        \textbf{Fallunterscheidung}  & \textbf{Dämpfung} & \textbf{Diskrimi.} & \textbf{Güte} & \textbf{Eigenwerte} $\lambda_{1/2}$ \\ \midrule
        aperiodischer Fall           & stark             & $D > 0$            & $Q_S < 0,5$     & Zwei verschiedene, reelle EW  \\
        aperiodischer Grenzfall      & kritisch          & $D = 0$            & $Q_S = 0,5$     & Eine doppelte, reelle EW \\
        periodischer Fall            & schwach           & $D < 0$            & $Q_S > 0,5$     & konjugierte, komplexe EW \\ \bottomrule
    \end{tabular}
\end{table}

Ein Ortskurvendiagramm der Eigenwerte wird im optionalen Exkurs in Kapitel \ref{sec:schaltvorgaengezeitbereich:exkurse:nst} gezeigt.
Der Zusammenhang zwischen $\lambda$ und $Q_S$ wird im optionalen Exkurs in Kapitel \ref{sec:schaltvorgaengezeitbereich:exkurse:eigenfrequenz} näher erläutert.

}% Ende Skript only
\b{% Beginn Folien only
\begin{minipage}{\textwidth}\centering
    \begin{minipage}[t][3cm][]{0.48\textwidth}\centering
        \resizebox{\textwidth}{!}{\includegraphics{Tikz/pdf/circ_switchable_rlc_charge_detail.pdf}}%
    \end{minipage}%
    \begin{minipage}[t][3cm][]{0.48\textwidth}\centering
        \begin{align*}
            \lambda_{1/2} = -\frac{R}{2L} \pm \sqrt{
                \smash{\underbrace{% smash damit Größe der Wurzel nicht angepasst wird
                \left(\frac{R}{2L}\right)^2 - \frac{1}{LC}
                }_{\mathclap{\text{Diskriminante}\ D}}}
                \vphantom{\left(\frac{R}{2L}\right)^2 - \frac{1}{LC}}% vphantom um Größe der Wurzel manuell an Term unter der Wurzel anzupassen
            }%&
        \end{align*}
    \end{minipage}
\end{minipage}\vspace{2pt}

\only<beamer:1-3| handout:1>{% Beginn overlay only
\textbf{Achtung!} Je nach Diskriminante $D$ andere Eigenwerte $\lambda_{1/2}$.
\onslide<2->{% Beginn overlay onslide
\begin{equation*}
    \textbf{2.) Homogene Lösung:}\qquad\quad
    \ecv{u_{C,\mathrm{h}}} =
    \begin{cases}
        K_1 \cdot \mathrm{e}^{\lambda_1 t} + K_2 \cdot \mathrm{e}^{\lambda_2 t}  & \text{für}\quad\lambda_1 \neq \lambda_2  \\
        (K_1 + K_2 \cdot t) \cdot \mathrm{e}^{\lambda t} & \text{für}\quad\lambda_1 = \lambda_2 = \lambda
    \end{cases}
    \qquad\qquad\qquad
\end{equation*}
}% Ende overlay onslide
\onslide<3->{%
\begin{table}[H]\centering
    \begin{tabular}{lcc} \toprule
        \textbf{Fallunterscheidung}  & \textbf{Diskrimi.} & \textbf{Eigenwerte} $\lambda_{1/2}$ \\ \midrule
        aperiodischer Fall           & $D > 0$               & Zwei verschiedene, reale NST  \\
        aperiodischer Grenzfall      & $D = 0$               & Eine doppelte, reale NST \\
        periodischer Fall            & $D < 0$               & Zwei konjugierte, komplexe NST \\ \bottomrule
    \end{tabular}
\end{table}
}% Ende overlay onslide
}% Ende overlay only
}% Ende Folien only
\end{frame}

%-------------------------------------------------------------------------------------------------%

\subsubsection[Aperiodischer Fall]{Aperiodischer Fall beim RLC-Serienschwingkreis}
\label{sec:schaltvorgaengezeitbereich:rlc:aperiodisch}\index{Fall>aperiodischer Fall}

\begin{frame}[t]\ftx{\subsubsecname}
\s{% Anfang Skript only
    Ist die Diskriminante $D$ positiv, ist der Wurzelterm reellwertig und immer kleiner als der Betrag des Terms $\frac{R}{2L}$ vor der Wurzel.
    Dadurch sind beide Eigenwerte verschieden, reell und negativ.%
}% Ende Skript only
\b{% Anfang Folien only
\begin{minipage}{\textwidth}\centering
    \begin{minipage}[t][3cm][]{0.48\textwidth}\centering
        \resizebox{\textwidth}{!}{\includegraphics{Tikz/pdf/circ_switchable_rlc_charge_detail.pdf}}%
    \end{minipage}%
    \begin{minipage}[t][3cm][]{0.48\textwidth}\centering
        \begin{align*}
            \lambda_{1/2} = -\frac{R}{2L} \pm \sqrt{
                \smash{\underbrace{% smash damit Größe der Wurzel nicht angepasst wird
                \left(\frac{R}{2L}\right)^2 - \frac{1}{LC}
                }_{\mathclap{\text{Diskriminante}\ D}}}
                \vphantom{\left(\frac{R}{2L}\right)^2 - \frac{1}{LC}}% vphantom um Größe der Wurzel manuell an Term unter der Wurzel anzupassen
            }%&
        \end{align*}
    \end{minipage}
\end{minipage}\vspace{2pt}

\textbf{Aperiodischer Fall:} (Kriechfall)%
}% Ende Folien only
% Following line needed, because end of \b{} before \begin{align} inserts an extra empty line
\ifvmode\vspace*{-\baselineskip}\setlength{\parskip}{4pt}\fi% Ref: https://tex.stackexchange.com/questions/543846/how-can-i-ignore-empty-lines-before-after-and-inside-an-equation-environment
\begin{align*}%
    &\text{für}\quad D > 0 \quad\text{gilt}&
        &\phantom{\mathrel{\phantom{=}}-}\frac{R}{2L} \overset{!}{>} \sqrt{\left(\frac{R}{2L}\right)^2 - \frac{1}{LC}}
            &&\Rightarrow\quad \text{2 reelle EW}
    % Dummyline: manual fixed spacing for inter-environment-alignment: zero height after regular row with regular height (15pt), uncomment \nonumber to check height
    \\[-10pt] &\hphantom{1rrrrrrrrrrrrrrrr} &   \hphantom{2llllll}&\hphantom{\mathrel{\phantom{=}}2rrrrrrrrrrrrrrrrrrrrrrrrrrr}     &&\hphantom{3rrrrrrrrrrrrrrrrrrrrrrrr} \nonumber
\end{align*}%
\pause%
\textbf{2.) Homogene Lösung:} (flüchtig)
\begin{align}%
    &&
        \ecv{u_{C,\mathrm{h}}} &= K_1 \cdot \mathrm{e}^{\lambda_1 t} + K_2 \cdot \mathrm{e}^{\lambda_2 t}
            &&\text{mit}\quad \lambda_1 \neq \lambda_2,\ \lambda <  0  \label{eq:dgl:rlc:homolsg:aperiodisch}
    % Dummyline: manual fixed spacing for inter-environment-alignment: zero height after regular row with regular height (15pt), uncomment \nonumber to check height
    \\[-15pt] &\hphantom{1rrrrrrrrrrrrrrrr} &   \hphantom{2llllll}&\hphantom{\mathrel{\phantom{=}}2rrrrrrrrrrrrrrrrrrrrrrrrrrr}     &&\hphantom{3rrrrrrrrrrrrrrrrrrrrrrrr} \nonumber
\end{align}%
\s{%
    Die Konstanten $K_1$ und $K_2$ müssen beide reell sein, damit die Lösung reellwertig bleibt.
    Der flüchtige Zustand entspricht damit der Addition zweier Exponentialfunktionen. 
    Dabei dominiert die langsamer Exponentialfunktion, das heißt die mit der größeren Zeitkonstante
    (Vgl. Exkurs in Kapitel \ref{sec:schaltvorgaengezeitbereich:exkurse:nst}).

}%
%\pause%
\textbf{3.) Partikuläre Lösung:} (eingeschwungen, identisch für alle drei Fälle)%
\begin{align*}
    &&
        \ecv{u_{C,\mathrm{p}}} &= \ecv{U_q}
        \\[-15pt] &\hphantom{1rrrrrrrrrrrrrrrr} &   \hphantom{2llllll}&\hphantom{\mathrel{\phantom{=}}2rrrrrrrrrrrrrrrrrrrrrrrrrrr}     &&\hphantom{3rrrrrrrrrrrrrrrrrrrrrrrr} \nonumber
    \end{align*}%
%\s{%
%    Die partikuläre Lösung $\ecv{u_{C,\mathrm{p}}}$ entspricht der Gleichspannung $\ecv{U_q}$, da sich die Kapazität für $t \to \infty$ vollständig aufgeladen hat.
%    Im Gegensatz zur homogenen Lösung ist $\ecv{u_{C,\mathrm{p}}}$ für alle drei Fälle identisch.
%}%
\end{frame}%
%
\begin{frame}\ftx{\subsubsecname}% Aperiodischer Fall: Zweiter Frame
    \textbf{4.) Allgemeine Lösung:} (Überlagerung)%
    \begin{align}
        &&
            \ecv{u_C}(t) &= K_1 \cdot \mathrm{e}^{\lambda_1 t} + K_2 \cdot \mathrm{e}^{\lambda_2 t} + \ecv{U_q} \nonumber\\[-14pt]
        &\hphantom{1Rooooooo}&\hphantom{2Loooooo}&\hphantom{2Roooooooooooooooooooooooooooooooooo}&\hphantom{3Loooooo}&\hphantom{3Roooooooooo} \nonumber % Dummyzeile alignment zwischen align-Umgebungen[ca. 16pt hoch], RLCs part. Lösung je Fall
    \end{align}
    \pause%
    \textbf{5.) Konstanten bestimmen:} (wie bekannt)

    \s{Sei $\ecv{u_C}(0)=0$ und $\eci{i}(0)=0 \Rightarrow \dt\, \ecv{u_C}(0) = 0 $ als Anfangsbedingungen (AB) gegeben, so folgt:}%
    \b{\vspace{4pt}Anfangsbedingungen (AB): $\ecv{u_C}(0)=0$ und $\eci{i}(0)=0 \Rightarrow \dt\, \ecv{u_C}(0) = 0 $}

    \begin{align*}
        &\hphantom{1Rooooooo}&\hphantom{2Loooooo}&\hphantom{2Roooooooooooooooooooooooooooooooooo}&\hphantom{3Loooooo}&\hphantom{3Roooooooooo}\\[-15pt] % Dummyzeile alignment zwischen align-Umgebungen[ca. 16pt hoch], RLCs part. Lösung je Fall
        &\text{1. AB:}&
            \ecv{u_C}(0) &= K_1 \cdot \redcancel{\mathrm{e}^{\lambda_1 0}} + K_2 \cdot \redcancel{\mathrm{e}^{\lambda_2 0}} + \ecv{U_q} \overset{!}{=} 0 & \Rightarrow\quad K_1 &= -K_2 -\ecv{U_q}\\
        &\text{2. AB:}&
            \dt \ecv{u_C}(0) &= K_1 \cdot \lambda_1 \cdot \redcancel{\mathrm{e}^{\lambda_1 0}} + K_2 \cdot \lambda_2 \cdot \redcancel{\mathrm{e}^{\lambda_2 0}} + \redcancel{0} \overset{!}{=} 0 &  \Rightarrow\quad K_1 &= - K_2 \frac{\lambda_2}{\lambda_1} \nonumber\\
        &\Longrightarrow&
            K_1 &=  \frac{\ecv{U_q}}{\frac{\lambda_1}{\lambda_2}-1} \quad\text{und}\quad
                K_2 = \frac{\ecv{U_q}}{\frac{\lambda_2}{\lambda_1}-1}
    \end{align*}

\end{frame}

%-------------------------------------------------------------------------------------------------%

\subsubsection[Aperiodischer Grenzfall]{Aperiodischer Grenzfall beim RLC-Serienschwingkreis}
\label{sec:schaltvorgaengezeitbereich:rlc:grenzfall}\index{Fall>aperiodischer Grenzfall}

\begin{frame}[t]\ftx{\subsubsecname}
\s{% Anfang Skript only
    Ist die Diskriminante $D$ exakt null, sind beide Eigenwerte identisch, reell und negativ.
    Durch die doppelte Nullstelle ergibt sich eine spezielle Form der homogenen Lösung
    durch Überlagerung der zwei Ansätze %Die homogene Lösung setzt sich aus zwei Ansätzen zusammen:
    ($K_1 \cdot \mathrm{e}^{\lambda_1 t}$ und $K_2 \cdot t \cdot \mathrm{e}^{\lambda_2 t}$).%
}% Ende Skript only
\b{% Anfang Folien only
\begin{minipage}{\textwidth}\centering
    \begin{minipage}[t][3cm][]{0.48\textwidth}\centering
        \resizebox{\textwidth}{!}{\includegraphics{Tikz/pdf/circ_switchable_rlc_charge_detail.pdf}}%
    \end{minipage}%
    \begin{minipage}[t][3cm][]{0.48\textwidth}\centering
        \begin{align*}
            \lambda_{1/2} = -\frac{R}{2L} \pm \sqrt{
                \smash{\underbrace{% smash damit Größe der Wurzel nicht angepasst wird
                \left(\frac{R}{2L}\right)^2 - \frac{1}{LC}
                }_{\mathclap{\text{Diskriminante}\ D}}}
                \vphantom{\left(\frac{R}{2L}\right)^2 - \frac{1}{LC}}% vphantom um Größe der Wurzel manuell an Term unter der Wurzel anzupassen
            }%&
        \end{align*}
    \end{minipage}
\end{minipage}\vspace{2pt}
    \textbf{Aperiodischer Grenzfall:}
}% Ende Folien only
% Following line needed, because end of \b{} before \begin{align} inserts an extra empty line
\ifvmode\vspace*{-\baselineskip}\setlength{\parskip}{4pt}\fi% Ref: https://tex.stackexchange.com/questions/543846/how-can-i-ignore-empty-lines-before-after-and-inside-an-equation-environment
\begin{align*}
    \vphantom{\sqrt{\left(\frac{R}{2L}\right)^2}}
    &\text{für}\quad D = 0 \quad\text{gilt}&
        \lambda_{1/2} &= -\frac{R}{2L} \redcancel{ {} \pm {} \sqrt{0}} = \lambda
            &&\Rightarrow\quad \text{1 reeller EW}
    % Dummyline: manual fixed spacing for inter-environment-alignment: zero height after regular row with regular height (15pt), uncomment \nonumber to check height
    \\[-15pt] &\hphantom{1rrrrrrrrrrrrrrrr} &   \hphantom{2llllll}&\hphantom{\mathrel{\phantom{=}}2rrrrrrrrrrrrrrrrrrrrrrrrrrr}     &&\hphantom{3rrrrrrrrrrrrrrrrrrrrrrrr} \nonumber
\end{align*}%
\pause%
\textbf{2.) Homogene Lösung:} (flüchtig)
\begin{align}
    &&
        \ecv{u_{C,\mathrm{h}}} &= \left( K_1 + K_2 \cdot t \right) \cdot \mathrm{e}^{\lambda t}
    &&\text{mit}\quad \lambda_1 = \lambda_2 < 0  \label{eq:dgl:rlc:homolsg:grenzfall}
    % Dummyline: manual fixed spacing for inter-environment-alignment: zero height after regular row with regular height (15pt), uncomment \nonumber to check height
    \\[-15pt] &\hphantom{1rrrrrrrrrrrrrrrr} &   \hphantom{2llllll}&\hphantom{\mathrel{\phantom{=}}2rrrrrrrrrrrrrrrrrrrrrrrrrrr}     &&\hphantom{3rrrrrrrrrrrrrrrrrrrrrrrr} \nonumber
\end{align}%
%\pause%
\textbf{3.) Partikuläre Lösung:} (eingeschwungen)%
\begin{align*}
    &&
        \ecv{u_{C,\mathrm{p}}} &= \ecv{U_q}
    \\[-15pt] &\hphantom{1rrrrrrrrrrrrrrrr} &   \hphantom{2llllll}&\hphantom{\mathrel{\phantom{=}}2rrrrrrrrrrrrrrrrrrrrrrrrrrr}     &&\hphantom{3rrrrrrrrrrrrrrrrrrrrrrrr} \nonumber
    \end{align*}
\end{frame}


\begin{frame}\ftx{\subsubsecname}% Aperiodischer Grenzfall: Zweiter Frame
    \textbf{4.) Allgemeine Lösung:} (Überlagerung)%
    \begin{align}
        &&
            \ecv{u_C}(t) &= \left( K_1 + K_2 \cdot t \right) \cdot \mathrm{e}^{\lambda t} + \ecv{U_q} \nonumber\\[-14pt]
        &\hphantom{1Rooooooo}&\hphantom{2Loooooo}&\hphantom{2Roooooooooooooooooooooooooooooooooo}&\hphantom{3Loooooo}&\hphantom{3Roooooooooo} \nonumber % Dummyzeile alignment zwischen align-Umgebungen[ca. 16pt hoch], RLCs part. Lösung je Fall
    \end{align}
    %\pause%
    \textbf{5.) Konstanten bestimmen:} (wie bekannt)

    \s{Sei $\ecv{u_C}(0)=0$ und $\eci{i}(0)=0 \Rightarrow \dt\, \ecv{u_C}(0) = 0 $ als Anfangsbedingungen (AB) gegeben, so folgt:}%
    \b{\vspace{4pt}Anfangsbedingungen (AB): $\ecv{u_C}(0)=0$ und $\eci{i}(0)=0 \Rightarrow \dt\, \ecv{u_C}(0) = 0 $}

    \begin{align*}
        &\hphantom{1Rooooooo}&\hphantom{2Loooooo}&\hphantom{2Roooooooooooooooooooooooooooooooooo}&\hphantom{3Loooooo}&\hphantom{3Roooooooooo}\\[-15pt] % Dummyzeile alignment zwischen align-Umgebungen[ca. 16pt hoch], RLCs part. Lösung je Fall
        &\text{1. AB:}&
            \ecv{u_C}(0) &= \left( K_1 + \redcancel{K_2 \cdot 0 }\right) \redcancel{\cdot \mathrm{e}^{\lambda 0}} + \ecv{U_q} \overset{!}{=} 0 & \Rightarrow\quad K_1 &= -\ecv{U_q}\\
        &\text{2. AB:}&
            \dt \ecv{u_C}(0) &= \left( K_1 + \redcancel{K_2 \cdot 0} \right) \cdot \lambda \cdot \redcancel{\mathrm{e}^{\lambda 0}} + K_2 \cdot \redcancel{\mathrm{e}^{\lambda 0}} \overset{!}{=} 0 & \Rightarrow\quad K_2 &= -K_1 \cdot \lambda \nonumber\\
            &\Longrightarrow&
        K_1 &= -\ecv{U_q} \quad\text{und}\quad
                K_2 = +\ecv{U_q} \cdot \lambda
    \end{align*}%
\end{frame}

%-------------------------------------------------------------------------------------------------%

\subsubsection[Periodischer Fall]{Periodischer Fall beim RLC-Serienschwingkreis}
\label{sec:schaltvorgaengezeitbereich:rlc:periodisch}\index{Fall>periodischer Fall}

\begin{frame}\ftx{\subsubsecname}%
\s{% Anfang Skript only
    Ist die Diskriminante negativ, gibt es zwei komplex konjugierte Eigenwerte mit negativem Realteil.
    Umformung mit $\sqrt{-1}=\mathrm{j}\cdot\sqrt{1}$ und Herleitung für $u_{C,h}$ in Kapitel
    \ref{sec:schaltvorgaengezeitbereich:exkurse:herleitung}:%
}% Ende Skript only
\b{% Anfang Folien only
    \begin{minipage}{\textwidth}\centering
        \begin{minipage}[t][3cm][]{0.48\textwidth}\centering
            \resizebox{\textwidth}{!}{\includegraphics{Tikz/pdf/circ_switchable_rlc_charge_detail.pdf}}%
        \end{minipage}%
        \begin{minipage}[t][3cm][]{0.48\textwidth}\centering
            \begin{align*}
                \lambda_{1/2} = -\frac{R}{2L} \pm \sqrt{
                    \smash{\underbrace{% smash damit Größe der Wurzel nicht angepasst wird
                    \left(\frac{R}{2L}\right)^2 - \frac{1}{LC}
                    }_{\mathclap{\text{Diskriminante}\ D}}}
                    \vphantom{\left(\frac{R}{2L}\right)^2 - \frac{1}{LC}}% vphantom um Größe der Wurzel manuell an Term unter der Wurzel anzupassen
                }%&
            \end{align*}
        \end{minipage}
    \end{minipage}\vspace{2pt}
    \textbf{Periodischer Fall:} (Schwingfall)%
}% Ende Folien only
% Following line needed, because end of \b{} before \begin{align} inserts an extra empty line
\ifvmode\vspace*{-\baselineskip}\setlength{\parskip}{4pt}\fi% Ref: https://tex.stackexchange.com/questions/543846/how-can-i-ignore-empty-lines-before-after-and-inside-an-equation-environment
\begin{align}
    &\text{für}\quad D < 0 \quad\text{gilt}&
        \lambda_{1/2} &= -\underbrace{
            \vphantom{\sqrt{\left(\frac{R}{2L}\right)^2}}
            \frac{R}{2L}
        }_{\delta} \pm \, \mathrm{j} \cdot \underbrace{
            \sqrt{\frac{1}{LC} - \left(\frac{R}{2L}\right)^2}
        }_{\omega_d}
        &&\Rightarrow\quad \text{2 kompl. EW} \label{eq:dgl:rlc:lambda:periodisch}
        \\[-15pt] &\hphantom{1rrrrrrrrrrrrrrrr} &   \hphantom{2llllll}&\hphantom{\mathrel{\phantom{=}}2rrrrrrrrrrrrrrrrrrrrrrrrrrr}     &&\hphantom{3rrrrrrrrrrrrrrrrrrrrrrrr} \nonumber
\end{align}%
\pause%
\textbf{2.) Homogene Lösung:} (flüchtig)
\begin{align}
    &&
        \ecv{u_{C,\mathrm{h}}} &= \mathrm{e}^{-\delta t} \cdot C_0 \cdot \sin(\omega_d t + \varphi_0)
        &&\text{mit}\quad \lambda_{1/2} = -\delta \pm \mathrm{j}\,\omega_d \label{eq:dgl:rlc:homolsg:periodisch}
    % Dummyline: manual fixed spacing for inter-environment-alignment: zero height after regular row with regular height (15pt), uncomment \nonumber to check height
    \\[-15pt] &\hphantom{1rrrrrrrrrrrrrrrr} &   \hphantom{2llllll}&\hphantom{\mathrel{\phantom{=}}2rrrrrrrrrrrrrrrrrrrrrrrrrrr}     &&\hphantom{3rrrrrrrrrrrrrrrrrrrrrrrr} \nonumber
\end{align}%
\b{Herleitung für $u_{C,\mathrm{h}}$ in Exkurs. Bestimmung der Konstanten $C_0$ und $\varphi_0$ wie bekannt.}%
\s{% Anfang Skript only
    Die homogene Lösung entspricht einer abklingenden Schwingung mit der
    \textbf{Abklingkonstante} $\delta$\index{Größe>Abklingkonstante} (neg. Realteil der EW)
    und der \textbf{gedämpften Eigenkreisfrequenz} $\omega_d$\index{Größe>Eigenkreisfrequenz, gedämpft} (pos. Imaginärteil der EW).
    Eine Herleitung für Gl. \ref{eq:dgl:rlc:homolsg:periodisch} aus Gl. \ref{eq:dgl:rlc:homolsg:periodisch:herleitung} erfolgt in Kapitel \ref{sec:schaltvorgaengezeitbereich:exkurse:herleitung}.
}% Ende Skript only
\end{frame}


\begin{frame}\ftx{\subsubsecname}% Periodischer Fall: Zweiter Frame
    \b{% Anfang Folien only
    \begin{minipage}{\textwidth}\centering
        \begin{minipage}[t][3cm][]{0.5\textwidth}%\centering
            \textbf{3.) Partikuläre Lösung:} (eingeschw.)%
            \begin{align*}
                &&
                    \ecv{u_{C,\mathrm{p}}} &= \ecv{U_q}
                \end{align*}
            \textbf{4.) Allgemeine Lösung:} (Überlagerung)%
            \begin{align*}
                &&
                    \ecv{u_C}(t) &= \mathrm{e}^{-\delta t} \cdot C_0 \cdot \sin(\omega_d t + \varphi_0) + \ecv{U_q} \nonumber\\[-14pt]
            \end{align*}
        \end{minipage}%
        \begin{minipage}[t][3cm][]{0.5\textwidth}\centering
            \vspace{10pt}
            \includegraphics{Tikz/pdf/plot_rlcs_charge_dc_uc_periodisch.pdf}
        \end{minipage}
    \end{minipage}\vspace{15pt}
    \textbf{5.) Konstanten bestimmen:} (wie bekannt)
    \begin{align*}
        &\hphantom{1Rooooooo}&\hphantom{2Loooooo}&\hphantom{2Roooooooooooooooooooooooooooooooooo}&\hphantom{3Loooooo}&\hphantom{3Roooooooooo}\\[-15pt] % Dummyzeile alignment zwischen align-Umgebungen[ca. 16pt hoch], RLCs part. Lösung je Fall
        &\text{1. AB:}&
            \ecv{u_C}(0) &= \redcancel{\mathrm{e}^{-\delta 0}} \cdot C_0 \cdot \sin(\redcancel{\omega_d t } + \varphi_0) + \ecv{U_q} \overset{!}{=} 0\\
        &\text{2. AB:}&
            \dt \ecv{u_C}(0) &= -\delta \cdot \redcancel{\mathrm{e}^{-\delta 0}} \cdot C_0 \cdot \sin(\varphi_0) + \mathrlap{\redcancel{\mathrm{e}^{-\delta 0}} \cdot C_0 \cdot \omega_d \cdot \cos(\varphi_0) \overset{!}{=} 0 }\nonumber\\
        &\Longrightarrow&
        C_0 &=  -\frac{\ecv{U_q}}{\sin(\varphi_0)} \quad\text{und}\quad
            \varphi_0 = \arctan\left(\frac{\omega_d}{\delta}\right)
    \end{align*}
    }% Ende Folien only
    \s{% Anfang Skript only
    \textbf{3.) Partikuläre Lösung:} (eingeschwungen)%
    \begin{align*}
        &&
            \ecv{u_{C,\mathrm{p}}} &= \ecv{U_q}
        \\[-15pt] &\hphantom{1rrrrrrrrrrrrrrrr} &   \hphantom{2llllll}&\hphantom{\mathrel{\phantom{=}}2rrrrrrrrrrrrrrrrrrrrrrrrrrr}     &&\hphantom{3rrrrrrrrrrrrrrrrrrrrrrrr} \nonumber
        \end{align*}
    %\pause%
    \textbf{4.) Allgemeine Lösung:} (Überlagerung)%
    \begin{align}
        &&
            \ecv{u_C}(t) &= \mathrm{e}^{-\delta t} \cdot C_0 \cdot \sin(\omega_d t + \varphi_0) + \ecv{U_q} \nonumber\\[-14pt]
        &\hphantom{1Rooooooo}&\hphantom{2Loooooo}&\hphantom{2Roooooooooooooooooooooooooooooooooo}&\hphantom{3Loooooo}&\hphantom{3Roooooooooo} \nonumber % Dummyzeile alignment zwischen align-Umgebungen[ca. 16pt hoch], RLCs part. Lösung je Fall
    \end{align}
    %\pause%
    \textbf{5.) Konstanten bestimmen:} (wie bekannt)
    \begin{align*}
        &\hphantom{1Rooooooo}&\hphantom{2Loooooo}&\hphantom{2Roooooooooooooooooooooooooooooooooo}&\hphantom{3Loooooo}&\hphantom{3Roooooooooo}\\[-15pt] % Dummyzeile alignment zwischen align-Umgebungen[ca. 16pt hoch], RLCs part. Lösung je Fall
        &\text{1. AB:}&
            \ecv{u_C}(0) &= \redcancel{\mathrm{e}^{-\delta 0}} \cdot C_0 \cdot \sin(\redcancel{\omega_d t } + \varphi_0) + \ecv{U_q} \overset{!}{=} 0\\
        &\text{2. AB:}&
            \dt \ecv{u_C}(0) &= -\delta \cdot \redcancel{\mathrm{e}^{-\delta 0}} \cdot C_0 \cdot \sin(\varphi_0) + \mathrlap{\redcancel{\mathrm{e}^{-\delta 0}} \cdot C_0 \cdot \omega_d \cdot \cos(\varphi_0) \overset{!}{=} 0 }\nonumber\\
        &\Longrightarrow&
        C_0 &=  -\frac{\ecv{U_q}}{\sin(\varphi_0)} \quad\text{und}\quad
            \varphi_0 = \arctan\left(\frac{\omega_d}{\delta}\right)
    \end{align*}
    In Abbildung \ref{fig:rlc:dc:uc:periodisch} ist der zeitlichen Verlauf der Spannung $u_C(t)$ für $t=0$ bis $t=5\cdot\frac{1}{\delta}$ gezeigt.
    In der gewählten Skalierung beträgt $\omega_0$ das vierfache der Abklingkonstante $\delta$ (Güte $Q_S = 2$).

    \begin{figure}[H]\centering%
        \resizebox{0.7\textwidth}{!}{\includegraphics{Tikz/pdf/plot_rlcs_charge_dc_uc_periodisch.pdf}}%
        \caption{Spannung $u_C(t)$ im periodischen Fall mit zugehörige Konstanten: $C_0$, $\varphi_0$, $\delta$, $\omega_d$}%
        \label{fig:rlc:dc:uc:periodisch}%
    \end{figure}%
    }% Ende Skript only
\end{frame}


%%%%%%%%%%%%%%%%%%%%%%%%%%%%%%%%%%%%%%%%%%%%%%%%%%%%%%%%%%%%%%%%%%%%%%%%%%%%%%%%%%%%%%%%%%%%%%%%%%

\subsection{Exkurse zu Serienschwingkreisen}
\label{sec:schaltvorgaengezeitbereich:exkurse}
\s{% Anfang Skript only
\begin{frame}\ftx{\subsecname: Optionale Inhalte zum Vertiefen}
    In diesem Kapitel sind ergänzende Exkurse zum Verhalten von Schwingkreisen enthalten, die zur Vertiefung des Themas dienen und Brücken zu weiteren Themengebieten schlagen.
    Die Inhalte der Exkurse sind für das Verständnis der restlichen Modulinhalte nicht zwingend erforderlich und können bei Bedarf übersprungen werden.
\end{frame}
}% Ende Skript only
%-------------------------------------------------------------------------------------------------%

\newpage% Achtung: Layout für eine A4-Seite im Skript vorgesehen!
\subsubsection{Herleitung der homogenen Lösung im periodischen Fall}
\label{sec:schaltvorgaengezeitbereich:exkurse:herleitung}
%
\begin{frame}[t]\ftx{Exkurs: \subsubsecname}
\s{% Anfang Skript only
Aus der allgemeinen homog. Lösung im periodischen Fall (Gl. \ref{eq:dgl:rlc:homolsg:cases}) folgt mit $\lambda_{1/2}$ (Gl. \ref{eq:dgl:rlc:lambda:periodisch}):
}% Ende Skript only
\ifvmode\vspace*{-\baselineskip}\setlength{\parskip}{0ex}\fi% Ref: https://tex.stackexchange.com/questions/543846/how-can-i-ignore-empty-lines-before-after-and-inside-an-equation-environment
\begin{align}
    &&
        \ecv{u_{C,\mathrm{h}}} &= \mathrm{e}^{-\delta t} \cdot \left( K_1 \cdot \mathrm{e}^{+\mathrm{j}\omega_d t} + K_2 \cdot \mathrm{e}^{-\mathrm{j}\omega_d t} \right) \nonumber\\
% uncomment intertext for math preview in VSCode
\intertext{$K_1$ und $K_2$ müssen komplex konjugiert sein, damit $\ecv{u_{C,\mathrm{h}}}$ reell ist! \s{\cite[Vgl. S. 379]{hagmann}}}
&\text{mit}&
        K_1 &= a + \mathrm{j} b \mathrlap{\quad\text{und}\quad K_2 = a - \mathrm{j} b\quad\text{mit}\quad a,b \in \mathbb{R}} \nonumber\\[6pt]
    &&
        \ecv{u_{C,\mathrm{h}}} &= \mathrm{e}^{-\delta t} \cdot \left[ (a + \mathrm{j} b) \cdot \mathrm{e}^{+\mathrm{j}\omega_d t} + (a - \mathrm{j} b) \cdot \mathrm{e}^{-\mathrm{j}\omega_d t} \right] \nonumber\\[2pt]
    &&
        \ecv{u_{C,\mathrm{h}}} &= \mathrm{e}^{-\delta t} \cdot
            \left[\vphantom{\Big|}
            \smash{ a \cdot
            \underbrace{\left(
                \mathrm{e}^{+\mathrm{j}\omega_d t} + \mathrm{e}^{-\mathrm{j}\omega_d t}
            \right)}_{2\cos(\omega_d t)}
            + \mathrm{j} b \cdot
            \underbrace{\left(
                \mathrm{e}^{+\mathrm{j}\omega_d t} - \mathrm{e}^{-\mathrm{j}\omega_d t}
            \right)}_{2\mathrm{j}\sin(\omega_d t)}
            }\right] \nonumber\\[20pt]
    &\text{mit}&
        \mathrm{e}^{+\mathrm{j}x} + \mathrm{e}^{-\mathrm{j}x} &= \cos(x) + \redcancel{\mathrm{j}\sin(x)} + \cos(x) - \redcancel{\mathrm{j}\sin(x)} = 2\cos(x) \nonumber\\
    &&
        \mathrm{e}^{+\mathrm{j}x} - \mathrm{e}^{-\mathrm{j}x} &= \redcancel{\cos(x)} + \mathrm{j}\sin(x) - \redcancel{\cos(x)} + \mathrm{j}\sin(x) = 2\mathrm{j}\sin(x) \nonumber\\[6pt]
    &&
        \ecv{u_{C,\mathrm{h}}} &= \mathrm{e}^{-\delta t} \cdot
            \left[\vphantom{\Big|}\smash{
            \underbrace{2a}_{C_1} \cdot \cos(\omega_d t) - \underbrace{2b}_{C_2} \cdot \sin(\omega_d t)
            }\right]
    \label{eq:dgl:rlc:homolsg:periodisch:herleitung}
\end{align}%
\end{frame}%
\begin{frame}[t]\ftx{Exkurs: \subsubsecname}% Exkurs Herleitung: Zweiter Frame
\s{% Anfang Skript only
Die homogene Lösung entspricht im periodischen Fall einer abklingenden ($\mathrm{e}^{-\delta t}$)
Überlagerung der Schwingungen $z_1$ und $-z_2$ mit gleicher Kreisfrequenz $\omega_d$
und Phasenverschiebung von $\sfrac{\pi}{2}$ zueinander. %
\begin{figure}[H]\centering
    \includegraphics{Tikz/pdf/plot_zeigerdiagramm_schwingungen_addieren.pdf}\hfill% C1*cos(wt) - C2*sin(wt) = C0*sin(wt + phi0)
    \begin{subfigure}{0.58\textwidth}\centering
        \caption{Zeitverläufe}%
    \end{subfigure}%
    \begin{subfigure}{0.38\textwidth}\centering
        \caption{Zeigerdiagramm}%
    \end{subfigure}%
    \caption{Überlagerung von Schwingungen gleicher Frequenz, Zeitverlauf u. Zeigerdiagramm}%
    \label{fig:plot:zeiger:ueberlagerung}%
\end{figure}%
}% Ende Skript only
\b{% Anfang Folien only
    Überlagerung zweier Schwingungen gleicher Kreisfrequenz $\omega_d$:
    \begin{align*}
        \ecv{u_{C,\mathrm{h}}} &= \mathrm{e}^{-\delta t} \cdot
        \left[\vphantom{\Big|}\smash{
        \underbrace{\ec[blue]{2a}}_{C_1} \ec[blue]{\cdot \cos(\omega_d t)} \ec[red]{\,-} \underbrace{\ec[red]{2b}}_{C_2} {}\ec[red]{\cdot \sin(\omega_d t)}
        }\right]
    \end{align*}%
    \includegraphics{Tikz/pdf/plot_zeigerdiagramm_schwingungen_addieren.pdf}% C1*cos(wt) - C2*sin(wt) = C0*sin(wt + phi0)%
    \vfill%
    \begin{align*}
        &&
            \ecv{u_{C,\mathrm{h}}} &= \mathrm{e}^{-\delta t} \cdot \ec[GETgreen]{C_0 \cdot \sin(\omega_d t + \varphi_0) }
                &&\text{mit}\quad \lambda_{1/2} = -\delta \pm \mathrm{j} \cdot \omega_d
    \end{align*}%
}% Ende Folien only
\end{frame}%
%
% <- no linebreak (Skript)
%
\begin{frame}[t]\ftx{Exkurs: \subsubsecname}% Exkurs Herleitung: Dritter Frame
\s{% Anfang Skript only
In Abbildung \ref{fig:plot:zeiger:ueberlagerung} ist die Überlagerung in Form addierter Zeitverläufe (a) und
in Form addierter komplexer Amplitudenzeiger im Zeigerdiagramm (b) dargestellt. Mathematisch ausgedrückt:%%
}% Ende Skript only
\b{% Anfang Folien only
\begin{align*}
    \ecv{u_{C,\mathrm{h}}} &= \mathrm{e}^{-\delta t} \cdot
    \left[\vphantom{\Big|}\smash{
    \underbrace{2a}_{C_1} \cdot \cos(\omega_d t) - \underbrace{2b}_{C_2} \cdot \sin(\omega_d t)
    }\right]
\end{align*}%
}% Ende Folien only
\ifvmode\vspace*{-\baselineskip}\setlength{\parskip}{4pt}\fi% Ref: https://tex.stackexchange.com/questions/543846/how-can-i-ignore-empty-lines-before-after-and-inside-an-equation-environment
\begin{align}
    &&&\mathrlap{\textit{Zeitbereich}}&&&&\mathrlap{\textit{Polar.}}&&\mathrlap{\textit{Kart.}} &&\nonumber\\
    &&
        z_1(t) &= +C_1 \cdot \cos(\omega_d t)&
        &\Leftrightarrow&
        \underline{z}_1 &= C_1 \cdot \mathrm{e}^{\mathrm{j}0} &
        \underline{z}_1 &= C_1 \nonumber\\
    &&
        z_2(t) &= -C_2 \cdot \sin(\omega_d t)&
        &\Leftrightarrow&
        \underline{z}_2 &= C_2 \cdot \mathrm{e}^{-\mathrm{j}\frac{\pi}{2}} &
        \underline{z}_2 &= \hphantom{C_1} - \mathrm{j} \cdot C_2 \nonumber\\
    &&
        z_0(t) &= z_1(t) + z_2(t)&
        &\Leftrightarrow&
        \underline{z}_0 &= \underline{z}_1 + \underline{z}_2 &
        \underline{z}_0 &= C_1 - \mathrm{j} \cdot C_2 \nonumber\\
    &&
        &= C_0 \cdot \cos(\omega_d t + \psi)&
        &\Leftrightarrow&
        \hphantom{z_0} &= C_0 \cdot \mathrm{e}^{\mathrm{j}\psi} \nonumber\\[4pt]
    %
    &\text{mit}&
        C_0 &= \sqrt{C_1^2 + C_2^2}&
        &\text{und}&
        \psi  &= \arctan\left(-\frac{C_2}{C_1}\right),
        &\varphi_0 &= \psi + \frac{\pi}{2} \nonumber\\[4pt]
    &&
        z_0(t) &= C_0 \cdot \sin(\omega_d t + \varphi_0) \label{eq:dgl:rlc:homolsg:periodisch:zeigeraddition}
\end{align}%
\b{% Anfang Folien only
\vfill%
\begin{align*}
    &&
        \ecv{u_{C,\mathrm{h}}} &= \mathrm{e}^{-\delta t} \cdot C_0 \cdot \sin(\omega_d t + \varphi_0)
            &&\text{mit}\quad \lambda_{1/2} = -\delta \pm \mathrm{j} \cdot \omega_d
\end{align*}%
}%
\s{% Anfang Skript only
Die homogene Lösung im periodischen Fall aus Gl. \ref{eq:dgl:rlc:homolsg:periodisch:herleitung} lässt sich
nach Gl. \ref{eq:dgl:rlc:homolsg:periodisch:zeigeraddition} wie im Folgenden als abklingende, phasenverschobene Sinuskurve beschreiben (Vgl. Gl. \ref{eq:dgl:rlc:homolsg:periodisch}):
\begin{align*}
    \ecv{u_{C,\mathrm{h}}} &= \mathrm{e}^{-\delta t} \cdot C_0 \cdot \sin(\omega_d t + \varphi_0)
        &&\text{mit}\quad \lambda_{1/2} = -\delta \pm \mathrm{j} \cdot \omega_d
\end{align*}
}% Ende Skript only
\end{frame}

%-------------------------------------------------------------------------------------------------%

\newpage% Achtung: Layout für eine A4-Seite im Skript vorgesehen!
\subsubsection[Ortskurvendiagramm, Nullstellen]{Exkurs: Fallunterscheidung bei RLC-Serienschwingkreis, Nullstellen}
\label{sec:schaltvorgaengezeitbereich:exkurse:nst}
\begin{frame}[t]\ftx{Exkurs: \subsubsecname}
\s{% Anfang Skript only
    \index{Ortskurve}%
    \index{Eigenwert}%\index{Eigenkreisfrequenz, ungedämpft}%
    Die Nullstellen (NST) des charakteristiscen Polynoms, das heißt die Eigenwerte $\lambda_{1/2}$,
    des RLC-Serienschwingkreises hängen von der Diskriminante $D$ ab.

    \begin{figure}[H]\centering
        \begin{subfigure}{0.48\textwidth}\centering
            \includegraphics{Tikz/pdf/plot_rlcs_ortskurve_nst.pdf}
            \caption{Ortskurve der Nullstellen $\lambda$}
            \label{fig:exkurse:nst}
        \end{subfigure}%
        \begin{subfigure}{0.48\textwidth}\centering
            \includegraphics{Tikz/pdf/plot_rlcs_charge_dc_uc.pdf}
            \caption{Spannungsverlauf, Fallunterscheidung}
            \label{fig:rlc:nst:dgl}
        \end{subfigure}
        \caption{Ortskurve der Eigenwerte und Spannungsverläufe im RLC-Serienschwingkreis}
    \end{figure}

    Abbildung \ref{fig:exkurse:nst} zeigt die Lage der Nullstellen im komplexen Zahlenraum als Ortskurve in Abhängigkeit von $D$
    für $\delta = \frac{R}{2L} = konst.$.
    Der Verlauf der NST entspricht einem Anstieg der ungedämpften Eigenkreisfrequenz $\omega_0 = \frac{1}{\sqrt{LC}}$.
    [Vgl. Gl. \ref{eq:dgl:rlc:eigenkreisfrequenz}]

    Im aperiodischen Fall ($D>0$) bewegen sich die beiden negativen reellen NST für $\delta=konst.$ und steigendem $\omega_0$ auf einander zu.
    Die NST näher an der Imaginärachse dominiert das Einschwingverhalten des Systems (langsamerer Abklingvorgang).
    Im aperiodischen Grenzfall ($D=0$) verschmelzen die NST zu einer doppelten Nullstelle bei $\lambda_{1/2}=-\delta$.
    Im periodischen Grenzfall ($D<0$) bewegen sich die konjugiert komplexen NST für $\delta=konst.$ und steigendem $\omega_0$ parallel zur Imaginärachse
    und entfernen sich von der Realachse.

    \textbf{Vergleich: Polstellen der Übertragungsfunktion (Systemtheorie)}

    Aus der Lage der Eigenwerte im Ortskurvendiagramm lassen sich Rückschlüsse auf das Einschwingverhalten des RLC-Serienschwingkreises ziehen.
    Das Vorgehen ist üblich bei der Stabilitätsanalyse von Systemen im Rahmen der Regelungstechnik und stammt als Methode aus der Systemtheorie.

    Für verschwindende Anfangsbedingungen ($\dt[i]\,u_C(0)=0\ \forall i$) lässt sich die DGL aus Gl. \ref{eq:dgl:rlc:dc}
    mithilfe der Laplace-Transformation direkt im Bildbereich beschreiben. Durch umstellen der transformierten Ausgangsgröße $U_C(s)$
    nach der Eingangsgröße $U(s)$ ergibt sich die Übertragungsfunktion $G(s)$ des RLC-Serienschwingkreises:

    \begin{align}
        u(t) &= LC \cdot \dt[2]\,u_{C}(t) + RC \cdot \dt\,u_{C}(t) + u_{C}(t)& &\bigg|\,\mathscr{L}\\
        U(s) &= LC \cdot s^2 \cdot U_C(s) + RC \cdot s \cdot U_C(s) + U_C(s) \vphantom{\bigg|}\\
        G(s) &= \frac{U_C(s)}{U(s)} = \frac{\text{Ausgangsgröße}(s)}{\text{Eingangsgröße}(s)} \vphantom{\bigg|}\\
        %\quad\widehat{=} \underline{F}(\mathrm{j}(\omega)) = \frac{\underline{U}_C}{\underline{U}}
        &= \frac{1}{LC \cdot s^2 + RC \cdot s + 1}
    \end{align}

    Die Polstellen der Übertragungsfunktion $G(s)$ entsprechen den Eigenwerten $\lambda_{1/2}$.

}% Ende Skript only
\b{% Anfang Folien only
\begin{minipage}{\textwidth}\centering
    \begin{minipage}[t][3cm][]{0.48\textwidth}\centering
        \resizebox{\textwidth}{!}{\includegraphics{Tikz/pdf/circ_switchable_rlc_charge_detail.pdf}}%
    \end{minipage}%
    \begin{minipage}[t][3cm][]{0.48\textwidth}\centering
        %\includegraphics{Tikz/pdf/plot_rlcs_charge_dc_uc.pdf}%
    \begin{align*}
        %&\textbf{Eigenwerte:}&
        \lambda_{1/2} = -\frac{R}{2L} \pm \sqrt{
            \smash{\underbrace{% smash damit Größe der Wurzel nicht angepasst wird
            \left(\frac{R}{2L}\right)^2 - \frac{1}{LC}
            }_{\mathclap{\text{Diskriminante}\ D}}}
            \vphantom{\left(\frac{R}{2L}\right)^2 - \frac{1}{LC}}% vphantom um Größe der Wurzel manuell an Term unter der Wurzel anzupassen
        }%&
    \end{align*}
    \end{minipage}
\end{minipage}\vspace{2pt}

\textbf{Nullstellen im Ortskurvendiagramm:} $\lambda(D)$ für $\delta=\frac{R}{2L}=konst.$ \vspace{2pt}

\begin{minipage}{\textwidth}\centering
    \begin{minipage}{0.4\textwidth}\centering
        \includegraphics{Tikz/pdf/plot_rlcs_ortskurve_nst.pdf}%
    \end{minipage}%
    \begin{minipage}{0.56\textwidth}\centering
        \begin{itemize}
            \item[] \makebox[2cm][l]{$\Re\{\lambda\} < 0$:} $\Rightarrow$ stabil
            \item[] $D>0$: NST näher an $\Im$-Achse
            \item[] \hspace{2cm} $\Rightarrow$ langsamer, dominant
            \item[] $D=0$: NST weiter weg von $\Im$-Achse
            \item[] \hspace{2cm} $\Rightarrow$ schnellster aperiod. Fall
            \item[] $D<0$: NST konjugiert, komplex
            \item[] \hspace{2cm} $\Rightarrow$ schwinungsfähig
        \end{itemize}
    \end{minipage}
\end{minipage}
}% Ende Folien only
\end{frame}

%-------------------------------------------------------------------------------------------------%
\newpage% Achtung: Layout für eine A4-Seite im Skript vorgesehen!
\subsubsection{Abklingkonstante, Eigen- und Resonanzkreisfrequenz, Güte}
\label{sec:schaltvorgaengezeitbereich:exkurse:eigenfrequenz}
\begin{frame}[t]\ftx{Exkurs: \subsubsecname}
\s{%
    \index{Größe>Abklingkonstante}%
    \index{Größe>Eigenkreisfrequenz, gedämpft}%
    \index{Größe>Resonanzkreisfrequenz}%
    \index{Größe>Güte}%
    Im periodischen Fall ergibt sich die (gedämpfte) Eigenkreisfrequenz $\omega_d$ und die Abklingkonstante $\delta$ aus den Eigenwerten $\lambda_{1/2}$.
    Die Größen hängen eng zusammen mit der \textbf{Güte} $Q_{\mathrm{S}}$ und der \textbf{Resonanzkreisfrequenz} $\omega_0$ eines Schwingkreises (Resonanzkreises).
    Beide Größen werden als Kenngrößen für Schwingkreise verwendet und ausführlicher in Modul 8 (Frequenzvariable Schaltungen) behandelt.

    Der Zusammenhang zwischen $\omega_d$, $\delta$, $\omega_0$ und $Q_{\mathrm{S}}$ soll anhand des RLC-Serienschwingkreises aus Abbildung \ref{fig:circ:rlc:charge}
    erläutert werden. Die Eigenwerte $\lambda_{1/2}$ sind nach Gleichung \ref{eq:dgl:rlc:lambda:periodisch} gegeben durch:
    \begin{align}
        %\lambda_{1/2} &= -\frac{R}{2L} \pm \, \mathrm{j} \cdot \sqrt{\frac{1}{LC} - \left(\frac{R}{2L}\right)^2}% without underbrace
        \lambda_{1/2} &= -\underbrace{
            \vphantom{\sqrt{\left(\frac{R}{2L}\right)^2}}
            \frac{R}{2L}
        }_{\delta} \pm \, \mathrm{j} \cdot \underbrace{
            \sqrt{\frac{1}{LC} - \left(\frac{R}{2L}\right)^2}
        }_{\omega_d}
        \label{eq:dgl:rlc:lambda:periodisch:exkurs}
    \end{align}
    Der negative Realteil $-\delta$ bestimmt die Abklingrate und der Imaginärteil $\omega_d$ die Kreisfrequenz der linear gedämpften, harmonischen Schwingung.
    Die Eigenkreisfrequenz $\omega_d$ lässt sich mit Gl. \ref{eq:dgl:rlc:resonanzkreisfrequenz} auch in Abhängigkeit von $\omega_0$ und $\delta$ ausdrücken.%
    \begin{align}
        \omega_d &= \sqrt{\frac{1}{LC} - \left(\frac{R}{2L}\right)^2} = \sqrt{\omega_0^2 - \delta^2} \label{eq:dgl:rlc:eigenkreisfrequenz}
    \end{align}
    Das hat den Vorteil, dass die ungedämpfte Eigenkreisfrequenz $\omega_0$ und die Abklingkonstante $\delta$
    direkt aus den Koeffizienten der DGL abgelesen werden können. Hierfür bietet sich die Form:
    \begin{equation}
        \ddot y + 2\delta \cdot \dot y + \omega_0^2 \cdot y = b
    \end{equation}

    für gewöhnliche, homogene, lineare DGLen 2. Ordnung an.
    Die DGL beschreibt das Verhalten linearer gedämpfter Systeme zweiter Ordnung (mit zwei Energiespeichern).%

    Die ungedämpfte Eigenkreisfrequenz entspricht dabei der Resonanzkreisfrequenz $\omega_0$.
    Bei Anregung eines Schwingkreises mit dessen \textbf{Resonanzfrequenz} $f_0$\index{Resonanzfrequenz} tritt eine Resonanz auf.
    Im Resonanzfall verschwindet der Blindanteil der Gesamtimpedanz $\underline{Z}(\omega)$ (Resonanzbedingung). [Vgl. Modul 8]

    Mit der Gesamtimpedanz:
    \begin{align}
            \underline{Z}(\omega) &= R + \mathrm{j} \left( \omega L - \frac{1}{\omega C} \right) \nonumber\\
        \intertext{und der Resonanzbedingung:}
            \Im\{\underline{Z}(\omega_0)\} &= \omega_0 L - \frac{1}{\omega_0 C} \overset{!}{=} 0 \nonumber\\
        \intertext{folgt für die Resonanzkreisfrequenz:}
            \omega_0 &= \sqrt{\frac{1}{LC}} \label{eq:dgl:rlc:resonanzkreisfrequenz}
    \end{align}

    Die Güte $Q_{\mathrm{S}}$ als Maß für die Schwingungsfähigkeit eines Schwingkreises lässt sich ebenfalls aus $\omega_0$ und $\delta$ ableiten.
    Sie ist definiert als Verhältnis der schaltungsintern schwingenden Blindleistung
    zur schaltungsinternen Verlustleistung im Resonanzfall. Dies entspricht beim RLC-Serienschwingkreis dem Verhältnis von
    Blindwiderstand der Induktivität $X_{L,0}$ bei Resonanzfrequenz zur Serienresistenz $R$ (ohne Herleitung).
    Dadurch ergibt sich folgender Zusammenhang:

    \begin{align}
        Q_{\mathrm{S}} &= \frac{X_{L,0}}{R} = \frac{\omega_0 L}{R} = \frac{\omega_0}{2\delta} \label{eq:dgl:rlc:guete}
    \end{align}

    Die Güte ist proportional zum Verhältnis der ungedämpften Eigenkreisfrequenz zur Dämpfungskonstante. %

}%
\b{%
\begin{minipage}{\textwidth}\centering
    \begin{minipage}[t][3cm][]{0.48\textwidth}\centering
        \resizebox{\textwidth}{!}{\includegraphics{Tikz/pdf/circ_switchable_rlc_charge_detail.pdf}}%
    \end{minipage}%
    \begin{minipage}[t][3cm][]{0.48\textwidth}\centering
        %\includegraphics{Tikz/pdf/plot_rlcs_charge_dc_uc.pdf}%
    \begin{align*}
        %\lambda_{1/2} &= -\frac{R}{2L} \pm \, \mathrm{j} \cdot \sqrt{\frac{1}{LC} - \left(\frac{R}{2L}\right)^2}% without underbrace
        \lambda_{1/2} &= -\underbrace{
            \vphantom{\sqrt{\left(\frac{R}{2L}\right)^2}}
            \frac{R}{2L}
        }_{\delta} \pm \, \mathrm{j} \cdot \underbrace{
            \sqrt{\frac{1}{LC} - \left(\frac{R}{2L}\right)^2}
        }_{\omega_d}
    \end{align*}
    \end{minipage}
\end{minipage}\vspace{2pt}

\textbf{Periodischer Fall:} Zusammenhang von $\delta$, $\omega_d$, $\omega_0$ und $Q_{\mathrm{S}}$:
\begin{align*}
&\text{Resonanzkreisfreq.:}&
    \omega_0 &= \sqrt{\frac{1}{LC}}&
    &\text{mit}\quad \Im\{\underline{Z}(\omega_0)\} \overset{!}{=} 0 \vphantom{\bigg|}\\
&\text{ged. Eigenkreisfreq.:}&
    \omega_d &= \sqrt{\omega_0^2 - \delta^2}&
    &\text{ungedämpft gleich $\omega_0$}\vphantom{\bigg|}\\
&\text{Güte:}&
    Q_{\mathrm{S}} &= \frac{X_{L,0}}{R} = \frac{\omega_0 L}{R} = \frac{\omega_0}{2\delta}&
    &\text{Maß für Schwingungsfähigkeit} \vphantom{\bigg|}\\
&\text{Vgl. homo. DGL:}&
    %0 &= \dt[2]\,y + 2\delta \cdot \dt\,y + \omega_0^2 \cdot y&
    0 &= \ddot y + 2\delta \cdot \dot y + \omega_0^2 \cdot y&
    &\text{linear gedämpfte Schwingung} \vphantom{\bigg|}
\end{align*}
%Die Güte $Q_{\mathrm{S}}$ ist Maß für die Schwingungsfähigkeit.

%Die Resonanzfrequenz ist gleich der ungedämpften Eigenfrequenz.
}% Ende Folien only
\end{frame}
