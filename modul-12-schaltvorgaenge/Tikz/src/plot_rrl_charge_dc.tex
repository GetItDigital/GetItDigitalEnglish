% Aufladevorgang RRL Strom und Spannung
% AxisEnv 1: Spannung an der Spule und den Widerständen, wobei U_L = U_R2
% AxisEnv 2: Strom an der Spule und beiden Widerständen
\begin{tikzpicture}[x=1cm,y=1cm]
	\begin{axis}[
		name=AxisSpannung,
		%%%%%%% START SAME OPTIONS %%%%%%%
		xmin=-1, xmax=5.5,
		ymin=-0.125, ymax=1.6,
		xlabel={$t$},
		ylabel={$u(t)$},
		axis x line=center,
		axis y line=center,
		ylabel style={anchor=east, at={(ticklabel cs:0.95)}},
		ticks=none,
		mark=none,
		grid=none,
		width=7cm,
		height=4.5cm,
        clip=false, % Verhindert Abschneiden der Labels
		%%%%%%% END OF SAME OPTIONS %%%%%%%
	]
        %Asymptoten mit Labels
		\addplot[color=black,dashed,	domain=0:5.4] {0.25};			
		\addplot[color=black,dashed,	domain=0:5.4] {1};  	
		\addplot[color=black,dashed,	domain=0:5.4] {1.25};
        \node at (-1,1.1)[black,anchor=south west,yshift=-1pt]{$U_q$};
        \node at (-2.4,0.79)[black,anchor=south west,yshift=-1pt]{$U_q \cdot \frac{R_2}{R_1 + R_2}$};
        \node at (-2.4,0.04)[black,anchor=south west,yshift=-1pt]{$U_q \cdot \frac{R_1}{R_1 + R_2}$};
        %Labels
        \node at (3,1.4)[color=green!50!blue,anchor=west]{$u_{R1}$};	
        \node at (1,0.5)[voltage,anchor=west]{$u_L = u_{R2}$};	
        %Plots	
        \addplot[color=voltage,very thick,	domain=0:5.4,samples=61] {e^(-x)}; 		       		
		\addplot[color=green!50!blue,very thick,	domain=0:5.4,samples=61] {-e^(-x)+1.25};	
		\addplot[color=voltage,very thick,	domain=-0.5:0] {0}; 		       		
		\addplot[color=green!50!blue,very thick, dashed, domain=-0.5:0] {0};		
		%Sprünge
		\addplot[color=voltage, very thick] coordinates {(0,0)(0,1)};
		\addplot[color=green!50!blue, very thick, dashed] coordinates {(0,0)(0,0.25)};					
											
	\end{axis}
	\begin{axis}[
		at={($(AxisSpannung.south west)+(0,-0.25cm)$)}, % Position 2. Plot
		anchor=north west,
		name=AxisStrom,
		%%%%%%% START SAME OPTIONS %%%%%%%
		xmin=-1, xmax=5.5,
		ymin=-0.125, ymax=1.5,
		xlabel={$t$},
		ylabel={$i(t)$},
		axis x line=center,
		axis y line=center,
		ylabel style={anchor=east, at={(ticklabel cs:0.95)}},
		ticks=none,
		mark=none,
		grid=none,
		width=7cm,
		height=4cm,
		%%%%%%% END OF SAME OPTIONS %%%%%%%
		clip=false,
		]
        %Asymptoten mit Labels
        \addplot[color=black,dashed,	domain=0:5.4] {0.35};
		\addplot[color=black,dashed,	domain=0:5.4] {1}; 
        \node at (-1,0.8)[black,anchor=south west,yshift=-1pt]{$\frac{U_q}{R_1}$};			
        \node at (-1.6,0.14)[black,anchor=south west,yshift=-1pt]{$\frac{U_q}{R_1 + R_2}$};	
        %Labels    
        \node at (2,1.15)[red,anchor=west]{$i_{R1}$}; 	
        \node at (2,0.2)[color=blue!25!red,anchor=west]{$i_{R2}$}; 	
        \node at (1,0.5)[color=green!25!red,anchor=west]{$i_L$};
		%Sprünge
		\addplot[color=blue!25!red, very thick] coordinates {(0,0)(0,0.35)};					
		\addplot[color=red, very thick, dashed] coordinates {(0,0)(0,0.35)};	
        %Plots 
        \addplot[color=green!25!red,very thick,	domain=0:5.4,samples=61] {-e^(-x)+1};	
        \addplot[color=red,very thick,	domain=0:5.4,samples=61] {-0.65*e^(-1.25*x)+1};	
        \addplot[color=blue!25!red,very thick,	domain=0:5.4,samples=61] {0.35*e^(-x)}; 	
		\addplot[color=green!25!red, very thick, dash pattern=on 3pt off 6pt, domain=-0.5:0] {0};	
        \addplot[color=blue!25!red, very thick, dash pattern=on 3pt off 6pt, domain=-0.3729:0] {0}; 	
		\addplot[color=red, very thick, dash pattern=on 3pt off 6pt, domain=-0.246:0] {0};	
		\draw[dash pattern=on 3pt off 6pt] (0,0) -- (4,0);
	\end{axis}
\end{tikzpicture} 