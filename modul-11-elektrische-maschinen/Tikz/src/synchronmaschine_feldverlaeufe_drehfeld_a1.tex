\begin{tikzpicture}[domain=0:36, samples=360, scale=0.838]
				\foreach \n in {1,...,4}{
					\draw[very thin, color=gray] (\n*4, 1) -- (\n*4,-1);
				}
				\draw (-0.1,0) -- (12.1,0);
				\draw[->] (0,-1.1) -- (0,1.2) node[above] {$I$};
				\draw (0.2, 0) node[below] {0};
				\foreach \n in {1,...,6}{
					\draw (\n*2, 0.1) -- (\n*2,-0.1);
					\fill[fill=white] (\n*2 - 0.1,-0.6) rectangle ++(0.2,0.5); % weißes Overlay über Linie hinter Zahlen
					\draw (\n*2, 0) node[below] {\pgfmathparse{60*\n}\pgfmathprintnumber[]{\pgfmathresult}\degree};
				}
				\draw[color=statorw1, scale=1/3, smooth, very thick] plot[id=drehfeldverlauf1] function{3 * cos(0.0556 * pi * x)};
				\draw[color=statorw1] (-0.3,0.55) node[above] {$I_1$};
				\draw[color=statorw2, scale=1/3, smooth, very thick] plot[id=drehfeldverlauf3] function{3 * cos(0.0556 * pi * x - pi * 2/3)};
				\draw[color=statorw2] (-0.3,-0.1) node[above] {$I_2$};
				\draw[color=statorw3, scale=1/3, smooth, very thick] plot[id=drehfeldverlauf2] function{3 * cos(0.0556 * pi * x + pi * 2/3)};
				\draw[color=statorw3] (-0.3,-0.75) node[above] {$I_3$};
				\end{tikzpicture}