\subsection{Berechnung der Ausgangsspannung einer OP-Schaltung mit zwei Eingängen\label{Aufg4}}
% Alt: Aufgabe 4
\Aufgabe{
    Gegeben ist die abgebildete Schaltung mit einem idealen 
    Operationsverstärker. Dabei ist der Widerstand $R_\mathrm{1} = \mathrm{100\,k\Omega}$ und $R_2=\mathrm{10\,k\Omega}$. Die Betriebsspannung beträgt $\mathrm{\pm 15\,V}$.
    \begin{itemize}
        \item [ a)] Ermitteln Sie die Ausgangsspannung $U_\mathrm{A}$ bei $U_\mathrm{E}=2\,\mathrm{mV}$ und $U_\mathrm{bat} = \mathrm{0\,V}$
        \item [ b)] Ermitteln Sie die Ausgangsspannung $U_\mathrm{A}$ bei $U_\mathrm{E}=20\,\mathrm{mV}$ und $U_\mathrm{bat}=\mathrm{0\,V}$
    \end{itemize}
    \begin {figure} [H]
   \centering
   \begin{tikzpicture}
    %OPV
\ctikzset{tripoles/en amp/input height=0.45}
\draw (0,0)node[en amp](E){}
    (E.out)
    (E.-)
    (E.+);
    \draw (E.out) to[short, -*] (2,0) to[short, -o] (2.5,0);
    \draw (E.-) to[short, -*] (-2,0.5) to[short, -o] (-3,0.5);
    \draw (-2,0.5) to[short, -*] (-2,2) to[R, name=R, l=$R_\mathrm{1}$] (2,2) -- (2,0);
    \draw (-5,2) to[R,l=$R_\mathrm{2}$, o-] (-2,2);

    
    \draw (-5,-2) to[short, o-] (-5,-2.1)node[ground]{};
    \draw (-3,-2) to[short, o-] (-3,-2.1)node[ground]{};
    \draw (E.+) -- (-2,-0.5)--(-2,-1.5) to[battery2, name=Ub] (-2,-2.1)node[ground]{};
    \draw (2.5,-2) to[short, o-] (2.5,-2.1)node[ground]{};
    
    \draw (2.5,0) to[open, v, name=ue] (2.5,-2);
    \draw (-3,0.5)to[open, v, name=ux] (-3,-2);
    \draw (-5,2) to[open, v, name=ua] (-5,-2);

    \varrmore{ue}{$U_\mathrm{A}$};
    \varrmore{ux}{$U_\mathrm{X}$};
    \varrmore{ua}{$U_\mathrm{E}$};
    \varrmore{Ub}{$U_\mathrm{bat}$};
\end{tikzpicture}

   \label{fig:FigOPV2Eingaenge}
    \end {figure}
   % \speech{
      %  Aufgabe 4: Operationsverstärkerschaltung  
      %  
      %  Gegeben ist eine Schaltung mit einem idealen Operationsverstärker.  
      %  Der Widerstand R1 beträgt 100 Kiloohm, der Widerstand R2 beträgt 10 Kiloohm.  
      %  Die Betriebsspannung beträgt plus minus 15 Volt.  
      %  
      %  Die Schaltung besteht aus einem Operationsverstärker mit einer Rückkopplung über die Widerstände R1 und R2.  
      %  Die Eingangsspannung U E wird über R2 an den invertierenden Eingang des Operationsverstärkers angelegt.  
      %  Der nichtinvertierende Eingang ist mit Masse verbunden.  
      %  Der Ausgang U A ist über den Widerstand R1 mit dem invertierenden Eingang verbunden.  
      %  Zusätzlich gibt es eine Batteriespannung U Bat, die auf den invertierenden Eingang wirkt.  
     %   
    %    Es sind zwei Teilaufgaben zu lösen:  
   %     
  %      a) Berechnung der Ausgangsspannung U A für eine Eingangsspannung U E von 2 Millivolt und eine Batteriespannung U Bat von 0 Volt.  
 %       b) Berechnung der Ausgangsspannung U A für eine Eingangsspannung U E von 20 Millivolt und eine Batteriespannung U Bat von 0 Volt.  
%        }
        
}

\Loesung{
    \begin {figure} [H]
   \centering
   \begin{tikzpicture}
\ctikzset{tripoles/en amp/input height=0.45}
\draw (0,0)node[en amp](E){}
    (E.out)
    (E.-)
    (E.+);
    \draw (E.out) to[short, -*] (2,0) to[short, -o] (2.5,0);
    \draw (E.-) to[short, -*] (-2,0.5) to[R,name=R2, l_=$R_\mathrm{2}$, -o] (-4,0.5);
    \draw (-2,0.5) to[short] (-2,2) to[R, name=R, l=$R_\mathrm{1}$] (2,2) -- (2,0);

    
    \draw (-4,-2) to[short, o-] (-4,-2.1)node[ground]{};
    \draw (E.+) -- (-2,-0.5)--(-2,-1.5) to[short] (-2,-2.1)node[ground]{};
    \draw (2.5,-2) to[short, o-] (2.5,-2.1)node[ground]{};
    
    \draw (2.5,0) to[open, v, name=ua] (2.5,-2);
    \draw (-4,0.5) to[open, v, name=ue] (-4,-2);

    \varrmore{ua}{$U_\mathrm{A}$};
    \varrmore{ue}{$U_\mathrm{E}$};
\end{tikzpicture}
   \label{fig:LsgOPV2Eingaenge}
    \end {figure}
    Verstärkung berechnen:
    \begin{align*}
        V =\mathrm{- \frac{100\,k\Omega}{10\,k\Omega} = -10}
    \end{align*}
    Formel für Verstärkung:
    \begin{eqa}
        V = \frac{U_\mathrm{A}}{U_\mathrm{E}}
    \end{eqa}
	Formel umstellen nach $U_\mathrm{A}$:
    \begin{align*}
        U_\mathrm{A} = V \cdot U_\mathrm{E}
    \end{align*}
    \begin{itemize}
        \item [ \bf a)] 
        \begin{align*}
        U_\mathrm{A} = \mathrm{-10 \cdot 2\,mV = -20\,mV}
        \end{align*}
        \item [ \bf b)]
        \begin{align*}
        U_\mathrm{A} = \mathrm{-10 \cdot 20\,mV = -200\,mV}
        \end{align*}
    \end{itemize}
    Siehe: Abschnitt \ref{fig: Verschaltung Verstaerker} und Tabelle \ref{tab:Grundschaltungen}
%    \speech{
%        Berechnung der Ausgangsspannung eines invertierenden Verstärkers  
%        
%        Gegeben ist eine Operationsverstärkerschaltung mit einer Rückkopplung.  
%        Die Widerstandswerte betragen:  
%        R1 = 100 Kiloohm, R2 = 10 Kiloohm.  
%        
%        Die Verstärkung wird mit folgender Formel berechnet:  
%        V = - R1 durch R2  
%        Einsetzen der Werte ergibt:  
%        V = - 100 Kiloohm durch 10 Kiloohm = -10  
%        
%        Die Ausgangsspannung U A wird mit der Formel berechnet:  
%        U A = V mal U E  
%        
%        Es sind zwei Fälle gegeben:  
%        
%        a) Für eine Eingangsspannung U E = 2 Millivolt:  
%        U A = -10 mal 2 Millivolt = -20 Millivolt  
%        
%        b) Für eine Eingangsspannung U E = 20 Millivolt:  
%        U A = -10 mal 20 Millivolt = -200 Millivolt  
%        
%        Die Berechnung zeigt, dass die Ausgangsspannung U A die Eingangsspannung um den Faktor -10 verstärkt.  
%        Das negative Vorzeichen zeigt die Phasenumkehr, die typisch für einen invertierenden Verstärker ist.  
%        }
        
}