\subsection{Spannungsteiler mit Operationsverstärker als Spannungsfolger\label{Aufg27}}
% Alt: Praktikumsaufgabe 13
\Aufgabe{
        Zeichnen Sie einen neuen Schaltplan bestehend aus einem belasteten Spannungsteiler, der durch einen Operationsverstärker als Impedanzwandler ergänzt wird.  
        Die Bauteilwerte sind wie folgt gegeben:
        
        \begin{itemize}
            \item Widerstände: $R_1 = R_2 = 10\,\mathrm{k\Omega}$
            \item Lastwiderstand: $R_\mathrm{L} = 2{,}2\,\mathrm{k\Omega}$
            \item Versorgungsspannung: $\mathrm{U_{CC} = +15\,V}$
        \end{itemize}
        
        Der Operationsverstärker wird als Impedanzwandler (Spannungsfolger) geschaltet.  
        Sein Ausgang speist den Lastwiderstand $R_\mathrm{L}$.
        
        \begin{itemize}
            \item[a)] Berechnen Sie die Ausgangsspannung $U_\mathrm{C}$ mit und ohne Lastwiderstand $\mathrm{R_L}$.
            \item[b)] Vergleichen Sie die Ergebnisse mit denen aus Aufgabe 13.  
            \item[c)] Erklären Sie, warum sich die Spannung $U_\mathrm{C}$ durch den Lastwiderstand in dieser Schaltung nicht ändert.
        \end{itemize}   
}

\Loesung{
    \begin{figure}[H]
        \centering
            \begin{tikzpicture}
    \draw (4,2) node[op amp, noinv input up, anchor=+] (opamp) {}
    (opamp.+) 
    (opamp.-) 
    (opamp.out) ;
    
    \draw (2,4) to[R,l=$R_\mathrm{1}$, -*] (2,2) to[R,l=$R_\mathrm{2}$,-*] (2,-1);
    \draw (2,2) -- (opamp.+);
    \draw (opamp.out) to[short, -*] (7,1.5) -- (7,0) -- (3.5,0) -- (3.5, 1) -- (opamp.-);
    \draw (7,1.5) -- (8,1.5) to[R,l=$R_\mathrm{L}$] (8,-1) -- (2,-1);
    \draw (2,-1) -- (0,-1);
    \draw (2,4) -- (0,4) to[V,name=V] (0,-1);
    
    \varrmore{V}{$U_\mathrm{0}$};
\end{tikzpicture}
   
            \label{fig:LsgPrakSpannungsfolger}
    \end{figure}
    \begin{itemize}
        \item[a)] Berechnung der Ausgangsspannung $U_\mathrm{C}$ mit und ohne Lastwiderstand \\
    Da der Operationsverstärker als Spannungsfolger (Impedanzwandler) geschaltet ist, entspricht die Ausgangsspannung $U_\mathrm{A}$ exakt der Spannung am Spannungsteiler:
    \[
    U_\mathrm{A} = U_\mathrm{C} = 7{,}5\,\mathrm{V}
    \]
    Da der Operationsverstärker einen sehr hohen Eingangswiderstand hat, wird die Spannung nicht durch den Lastwiderstand beeinflusst.  
    Somit bleibt die Ausgangsspannung konstant bei 7,5 V, unabhängig davon, ob $R_\mathrm{L}$ angeschlossen ist oder nicht.

    \item[b)] Vergleich mit Aufgabe 13 \\
    Im Gegensatz zur ersten Schaltung sinkt die Spannung hier nicht ab, wenn der Lastwiderstand $R_\mathrm{L}$ angeschlossen wird.  
    Der Operationsverstärker entkoppelt den Spannungsteiler von der Last, sodass der Spannungsteiler immer seine ursprüngliche Spannung hält.

    \item[c)] Erklärung der Spannungsstabilität durch den Operationsverstärker \\
    Der Operationsverstärker als Impedanzwandler besitzt einen hohen Eingangswiderstand und einen geringen Ausgangswiderstand.  
    Dadurch wird der Lastwiderstand nicht mehr direkt mit dem Spannungsteiler verbunden, sondern erhält seinen Strom aus dem OPV.  
    Dies führt dazu, dass die Ausgangsspannung stabil bleibt, unabhängig von der Belastung.
    \end{itemize}
    Siehe: Abschnitt \ref{fig: Verschaltung Verstaerker} und Tabelle \ref{tab:Grundschaltungen}
}