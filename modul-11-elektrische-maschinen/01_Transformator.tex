\newvideofile{oeffStromver}{Der Transformator in der öffentlichen Stromversorgung}
\begin{frame}\fta{Der Transformator}
	\s{Der Transformator ist eine ruhende elektrische Maschine und dient der Energiewandlung von elektrischer Energie über den magnetischen Kreis zurück in elektrische Energie, wobei insbesondere die Spannungshöhe gewandelt wird. Transformatoren gibt es in verschiedenen Ausführungen und Leistungsklassen von wenigen Watt (Abbildung \ref{bildtrafo2}) bis in den hohen Megawattbereich (Abbildung \ref{bildtrafo1}).\index{Transformator}}
	\begin{columns}
		\column[c]{0.55\textwidth}
		\f{width=\textwidth}{width=\textwidth}{20120329Umspannwerk_Altlussheim06.jpg}{{\bf Ein 40 MVA Leistungstransformator.} Aufgenommen im Umspannwerk Altlußheim. \label{bildtrafo1}{\tiny(Quelle: Wikimedia, Lizenz CC0)}}
		\column[c]{0.4\textwidth}
		\f{width=0.4\textwidth}{width=0.7\textwidth}{Transformator2b.png}{{\bf Fotografie eines Kleintransformators.} \label{bildtrafo2}}
	\end{columns}
	\b{
		\vspace{0.5cm}
		\begin{itemize}
			\item Wandlung elektrische Energie $\longrightarrow$ elektrische Energie
			\item Spannungswandler
		\end{itemize}
	}
	\speech{Transformator}{1}{Der Transformator ist ein Energiewandler, der elektrische Energie wieder in elektrische Energie umwandelt. Er dient hauptsächlich zur Änderung des Spannungsniveaus einer Wechsel- oder Drehspannung. Die elektrische Energie wird hierbei zuerst in einen magnetischen Kreis und dann zurück in elektrische Energie umgewandelt. Transformatoren gelten als ruhende elektrische Maschinen, da sie im Gegensatz zu den klassischen elektrischen Maschinen keine beweglichen Teile wie Rotoren, die Drehbewegungen ausführen, besitzen.
		Anwendung finden sie immer dort, wo die Spannungsebenen einer Wechselspannung geändert werden sollen. Transformatoren gibt es in unterschiedlichsten Leistungsklassen.
		Im Haushalt findet man sie zum Beispiel in Elektrogeräten selbst oder in ihren Netzteilen. Ihre Aufgaben sind zum einen das Umwandeln der anliegenden Netzspannung von typischerweise Zweihunderdreißig Volt auf die Betriebsspannung des Elektrogerätes. Sie haben aber auch einen wichtigen Sicherheitsaspekt. Der Transformator ist ein oft verwendetes galvanisch trennendes Bauelement und dient mit seinen zwei elektrisch nicht verbundenen Stromkreisen der Isolation zwischen dem Netzstrom und dem Gerät selbst. Dies verhindert, dass bei Fehlern im Gerät die hohe Netzspannung am Gerät anliegt und den Bediener gefährden kann. Durch die technologische Entwicklung in der Leistungselektronik haben kleine Schaltnetzteile den konventionellen Trafonetzteilen mit ihren großen und schweren Transformatoren im Alltag größtenteils ersetzt. Die deutlich höhere Frequenz der Energieübertragung, bis zu mehreren 100 Kilohertz im Vergleich zu den 50 Hertz Netzfrequenz, ermöglicht erheblich kleinere Netzteile bei gleicher Leistung.}
\end{frame}

\begin{frame}\ftx{Öffentliche Stromversorgung}
	\s{In der öffentlichen Energieversorgung wird die von Kraftwerken erzeugte Energie auf eine höhere Spannung transformiert, um sie über weite Strecken transportieren zu können. Die Über\-tra\-gungs\-ver\-lus\-te sinken mit höherer Spannung und damit bei gleicher Leistung sinkendem Strom quadratisch (Gleichung \ref{GlVerlustleistung} im Dreiphasenwechselstrom).

		\begin{eq}
			P_\mathrm{v} = 3\cdot R_\mathrm{L}\cdot I_\mathrm{L}^2\label{GlVerlustleistung}
		\end{eq}\pause

		Die übertragene Leistung im Dreiphasenwechselstrom beträgt:

		\begin{eq}
			\underline{S}_\mathrm{N} = \sqrt{3}\cdot \underline{U}_\mathrm{N} \cdot \underline{I}_\mathrm{L}^\ast
		\end{eq}

		Die Verluste sind damit proportional zu $\frac{1}{\underline{U}_\mathrm{N}^2}$

		Ein Transformator macht dadurch erst eine flächendeckende Energieversorgung möglich, da Kraftwerke auch verbraucherfern positioniert werden können.

		In Energieversorgungsnetzen (siehe Abbildung \ref{Abbversorgungsnetz}) gibt es mehrere Spannungsebenen. Die Höchst\-spannungs\-ebene, die in Europa überwiegend $380\,$kV beträgt, dient als Übertragungsnetz zum Transport elektrischer Leistung über weitere Strecken, also mehrere hundert Kilometer auch über Länder\-grenzen hinweg. Große Kraftwerke sind dort direkt angekoppelt. Die Hochspannungsebene\index{Hochspannung} ($110\,$kV oder $220\,$kV) nennt sich Verteilnetz und dient der Ankopplung von Städten und großen Industrien. Die Mittelspannungsebene\index{Mittelspannung} ($5-35\,$kV) sorgt für die Ankopplung der Ortznetztransformatoren zur Versorgung von einzelnen Straßenzügen. Während die höheren Spannungsebenen normalerweise mit Freileitungen ausgeführt sind, ist das Mittelspannungsnetz überwiegend mit Erdkabeln realisiert. Im ländlichen Bereich werden aber auch hier Freileitungen eingesetzt.}

	\s{\f{width=\textwidth}{width=\textwidth}{Stromversorgung_gesamt.pdf}{\bf{Struktur der öffentlichen Stromversorgung. \label{Abbversorgungsnetz}}} Das Stromnetz ist in mehrere Spannungsebenen unterteilt, die den Transport und die Verteilung der Energie ermöglichen.}
	\b{\f{}{height=0.85\textheight}{Stromversorgung_p1}{}}

	\speech{OeffentlicheVersorgung}{1}{Ein wichtiges Anwendungsfeld von Transformatoren ist die öffentliche Energieversorgung. Durch Hochtransformieren der Spannung sinken die Verluste bei der Stromübertragung. Erst mit Transformatoren wird eine flächendeckende Energieversorgung möglich, da Kraftwerke die Verbraucher über weite Distanzen hinweg effizient mit Energie versorgen können.
		Schauen wir uns hierfür das Energieversorgungsnetz an. Wir haben Energieproduzenten und Energieabnehmer, die mittels des Stromnetzes verbunden sind. Das Stromnetz ist in vier Spannungsebenen nach den zu transportierenden Nennspannungen unterteilt. Zwischen diesen einzelnen Spannungsebenen, als auch zwischen den Spannungsnetzen und deren Energieproduzenten sowie vereinzelten größeren Energieabnehmern, sind Umspannwerke installiert, um die elektrische Energie auf die erforderliche Spannungsebene zu transformieren.
		Die Höchstspannungsebene, die in Europa überwiegend 380 Kilovolt beträgt, dient zum Transport elektrischer Leistung über weite Strecken, also mehrere hundert Kilometer, auch über Ländergrenzen hinweg. Sie wird deshalb Übertragungsnetz genannt. An diese Höchstspannungsebenen sind große Kraftwerke direkt angekoppelt, deren Energie mittels Umspannwerken auf die Höchstspannung transformiert wird. Die Hochspannungsebene mit der Übertragungsspannung von 110 Kilovolt oder 220 Kilovolt dient der breiten regionalen Ankopplung von Städten. Mit den Mittelspannungs- und Niederspannungsnetzen bildet sie die höchste Ebene des Verteilnetzes. Energieintensive Industrien werden auch direkt an das Hochspannungsnetz angeschlossen. Ein Beispiel dafür sind Aluminiumwerke oder Glas-hütten, die sehr viel Strom benötigen.}
	\speech{OeffentlicheVersorgung}{2}{Das Mittelspannungsnetz dient typischerweise der elektrischen Energieversorgung von kleineren Ortschaften im ländlichen Bereich, oder von Stadtteilen innerhalb einer Stadt. Das Spannungsniveau kann zwischen 1 und 50 Kilovolt betragen, wobei in Deutschland 10 oder 20 Kilovolt üblich sind. Im Gegensatz zu den höheren Spannungsebenen, die normalerweise mit Freileitungen ausgeführt sind, wird das Mittelspannungsnetz überwiegend mit Erdkabeln umgesetzt. An dieses Netz sind Stromproduzenten wie städtische Kraftwerke sowie Windkraftanlagen und Solarkraftwerke integriert, als auch größere Energieabnehmer aus Industrie- und Dienstleistungsbereich angeschlossen.
		Die unterste Ebene des Verteilnetzes sind die Niederspannungsnetze. Diese werden aus übergeordneten Mittelspannungsnetzen gespeist und verteilen mittels örtlichen Transformatorstationen die Energie an Haushalte und Kleinunternehmen. In Deutschland wird das Niederspannungsnetz mit einer Nennspannung von 230 Volt einphasig beziehungsweise 400 Volt dreiphasig betrieben.}
	\speech{OeffentlicheVersorgung}{3}{Die Notwendigkeit der verschiedenen Spannungsebenen liegt an der Verlustleistung, die proportional zum quadratischen Strom ist. Ein Teil der eingesetzten elektrischen Energie wird dadurch in Wärmeenergie umgewandelt und verpufft. In einem Dreiphasenwechselstrom errechnet sich die Verlustleistung, P V, aus dem Produkt des Widerstands der Leitung, R L, und dem Quadrat des durch ihn fließenden Stromes multipliziert mit Drei.}
	\speech{OeffentlicheVersorgung}{4}{Da ein Transformator das Verhältnis der Spannung, U N, und des Stroms, i L, verändert, ohne dabei die Gesamtleistung, S N, zu ändern, werden die Wärmeverluste reduziert, wenn die Spannung erhöht wird.}
	\speech{OeffentlicheVersorgung}{5}{Folglich sind die Verluste nicht nur quadratisch zum fließenden Strom, sondern auch Eins durch die quadratische Spannung.}
\end{frame}


\b{
	\begin{frame}\ftx{Öffentliche Stromversorgung}
		\f{}{height=0.85\textheight}{Stromversorgung_p2}{}
	\end{frame}

	\begin{frame}\ftx{Öffentliche Stromversorgung}
		\begin{columns}
			\begin{column}{0.4\textwidth}
				\f{}{height=0.8\textheight}{Stromversorgung_gesamt}{}
			\end{column}
			\begin{column}{0.58\textwidth}
				Verluste proportional zum quadratischen Strom.
				\begin{eq}
					P_\mathrm{v} = 3\cdot R_\mathrm{L}\cdot I_\mathrm{L}^2
				\end{eq}\pause
				Übertragene Leistung:
				\begin{eq}
					\underline{S}_\mathrm{N} = \sqrt{3}\cdot \underline{U}_\mathrm{N} \cdot \underline{I}_\mathrm{L}^\ast
				\end{eq}\pause
				Verluste proportional zu $\frac{1}{\underline{U}_\mathrm{N}^2}$
			\end{column}
		\end{columns}
	\end{frame}
}

\newvideofile{prinzipTrafo}{Prinzip des Transformators}
\begin{frame}
	\ftb{Prinzip des Transformators} \label{PrinzipTransformator}
	\s{
		Um das Prinzip des Transformators zu erörtern, starten wir mit einem vollständig verlustfreien, idealen Transformator. An der Primärspule wird eine sinusförmige (eigentlich Kosinusförmig, was aber die gleiche Form ist und nur einer Phasenverschiebung von $90\degree$ entspricht) Spannung $\underline{U}_1$ angelegt. Der zeitliche Verlauf von $\underline{U}_1$ entspricht $u$ in Gleichung \ref{GlSpannung1}.

		\begin{eq}
			u = \hat{U}\cdot \cos(\omega t)\label{GlSpannung1}
		\end{eq}

	}

	\fu{
		\b{\vspace{-0.8cm}}
		\resizebox{\textwidth}{!}{
			\input{Tikz/src/magnetische_kopplung}
		}
		\b{\vspace{-1.5cm}}
	}{{\bf Prinzip der magnetische Kopplung eines Transformators.} Die Spannung $\underline{U}_1$ der linken Leiterschleife erzeugt einen magnetischen Fluss $\varPhi$, welcher in der rechten Leiterschleife die Spannung $\underline{U}_2$ induziert. \label{AbbMagnetischeKopplung}} %von JK


	\s{
		Diese Spannung bewirkt entsprechend der Umkehrung des Induktionsgesetzs (Gleichung \ref{GlInduktionsgesetz}) einen wechselnden magnetischen Fluss.
		\begin{eq}
			u = -N\cdot \frac{\d \varPhi}{\d t}\label{GlInduktionsgesetz}
		\end{eq}

		Die Form des magnetischen Flusses ist bei sinusförmiger Spannungsform ebenfalls sinusförmig, da das Integral eines Kosinus der Sinus ist. Der Effektivwert des magnetischen Flusses kann daher durch Gleichung \ref{GlMagnetischerFluss} dargestellt werden, indem das Induktionsgesetz nach $\varPhi$ aufgelöst wird und der Effektivwert gebildet wird. Der Sinus und damit die zeitliche Abhängigkeit fällt dadurch weg.
		\begin{eq}
			\varPhi = \frac{\sqrt{2}\cdot \underline{U}_1}{2\pi\cdot N_1 \cdot f}\label{GlMagnetischerFluss}
		\end{eq}
	}

	\b{
		\begin{columns}
			\column[c]{0.5\textwidth}
			\onslide<2->{
				\begin{eqa}
					u_\mathrm{i} &= -N\cdot \frac{\d \varPhi}{\d t}\\
					u &= \hat{U}\cdot \cos(\omega t)\\
					\varPhi &= \frac{\sqrt{2}\cdot \underline{U}_1}{2\pi\cdot N_1 \cdot f}
				\end{eqa}}

			\column[c]{0.5\textwidth}
			\onslide<3->{
				Für die Sekundärspule gilt:
				\begin{eqa}
					u_2 &= -N_2\cdot \frac{\d \varPhi}{\d t}\\
					\frac{\underline{U}_1}{\underline{U}_2} &= \frac{-N_1\cdot \frac{\d \varPhi}{\d t}}{-N_2\cdot \frac{\d \varPhi}{\d t}} = \frac{N_1}{N_2} = \textit{ü}
				\end{eqa}
			}
			\nomenclature[F]{\textit{ü}}{Übersetzungsverhältnis eines Transformators}
		\end{columns}
	}
	\s{
		Die Flussrichtung des Magnetfelds bewegt sich stets gemäß der \glqq rechten Handregel\grqq\ (siehe Modul 4), in diesem Beispiel gegen den Uhrzeigersinn. Der so erzeugte magnetische Fluss induziert wiederum an der Sekundärspule des Transformators, die optimalerweise mit dem gleichen Fluss $\varPhi$ durchdrungen wird, eine Spannung $\underline{U}_2$, die sich auch nach dem Induktionsgesetz (Gleichung \ref{GlInduktionsgesetz}) berechnen lässt.
		\begin{eq}
			u_2 = -N_2\cdot \frac{\d \varPhi}{\d t}\tag{\ref{GlInduktionsgesetz}}
		\end{eq}

		Die angelegte Primärspannung $\underline{U}_1$ und die Sekundärspannung $\underline{U}_2$ können in ein Verhältnis gesetzt werden (Gleichung \ref{Gluebertragung}). Die Änderungsrate des magnetischen Flusses $\varPhi$ kürzt sich dadurch heraus, es bleiben nur noch die Windungszahlen der Primärseite $N_1$ und der Sekundärseite $N_2$ übrig. Dieses Verhältnis wird Übersetzungsverhältnis \textit{ü} (Gleichung \ref{Gluebertragung}) genannt. Ist das Übersetzungsverhältnis \textit{ü} größer als Eins wird die Spannung herabgesetzt (Spannungsabwärtstransformator). Ist das Übersetz"-ungs"-verhältnis \textit{ü} kleiner als Eins, wird die Spannung erhöht (Spannungsaufwärtstransformator). Da der Transformator in beide Richtungen eingesetzt werden kann, wird in den technischen Daten immer ein Übersetzungsverhältnis größer oder gleich Eins angegeben.
		\begin{eq}
			\frac{\underline{U}_1}{\underline{U}_2} = \frac{-N_1\cdot \frac{\d \varPhi}{\d t}}{-N_2\cdot \frac{\d \varPhi}{\d t}} = \frac{N_1}{N_2} = \textit{ü} \label{Gluebertragung} \\
		\end{eq}

		\begin{eqa}
			\textit{ü} < 1 &\rightarrow \text{Spannung hoch}\nonumber\\
			\textit{ü} > 1 &\rightarrow \text{Spannung runter}\nonumber
		\end{eqa}
		\begin{center}
			Angegeben wird immer $\textit{ü} > 1$
		\end{center}

		Bei einem realen Transformator durchdringt nicht der komplette magnetische Fluss, der durch die Primärwicklung erzeugt wird, auch die Sekundärseite (in Abbildung \ref{AbbMagnetischeKopplung} rot gestrichelt dargestellt). Diese Verluste werden Streufluss genannt und werden im Ersatzschaltbild durch die Streuinduktivität \label{Streuinduktivität} dargestellt. Daher ist die reale Sekundärspannung kleiner als die hier berechnete ideale Sekundärspannung.
	}
	\speech{PrinzipTransformator}{1}{Um das Prinzip des Transformators zu erklären, greifen wir auf die Grundlagen des Elektromagnetismus zurück. An einer Spule wird eine cosinusförmige Wechselspannung, U Eins komplex, angelegt. Diese Spule nennen wir Primärspule.}
	\speech{PrinzipTransformator}{2}{Wie im Kapitel Vier kennengelernt, erzeugt ein stromdurchflossener Leiter ein kreisförmiges elektromagnetisches Feld, welches sich gemäß der Rechten-Hand-Regel verhält. In unserem Beispiel bewegt es sich gegen den Uhrzeigersinn. Die Spannung bewirkt entsprechend der Umkehrung des Induktionsgesetzes einen wechselnden magnetischen Fluss. Mathematisch ausgedrückt verhält sich die zeitlich abhängige induzierte Spannung, klein U i, gleich der Änderungsrate des magnetischen Flusses mit der Zeit, Delta F i durch Delta T, multipliziert mit der negativen Anzahl der Windungen, hier als, minus N, angezeigt. Das negative Vorzeichen ergibt sich aus der Lenz'schen Regel. Zur Berechnung der zeitlich abhängigen Variable klein U, wird der Scheitelwert, U Dach, mal dem Cosinus Term, der die zeitliche Änderung der Spannung angibt, multipliziert. Das Omega bezeichnet die Winkelgeschwindigkeit. Die Form des magnetischen Flusses ist bei sinusförmigen Spannungen ebenfalls sinusförmig, da das Integral eines Kosinus der Sinus ist. Indem das Induktionsgesetz nach, F i, aufgelöst wird, kann der Effektivwert des magnetischen Flusses ermittelt werden.
		Er bildet sich aus der Wurzel von 2, multipliziert mit der angelegten Spannung, U Eins komplex, durch die Winkelgeschwindigkeit in Form 2 Pi, multipliziert mit der Frequenz, klein F, multipliziert mit der Anzahl der Windungen der Spule, N Eins. Die zeitlich abhängige Komponente in Form des Sinus fällt bei der Effektivwertberechnung weg.}
	\speech{PrinzipTransformator}{3}{Durchdringt der magnetische Strom nun eine zweite Spule, Sekundärspule genannt, induziert er bei dieser die elektrische Spannung, U2 komplex, die sich auch nach dem Induktionsgesetz berechnen lässt. Klein U Zwei, errechnet sich also analog zu klein U i über die negative Windungszahl der Sekundärspule, N2, multipliziert mit der Änderungsrate des magnetischen Flusses. Die Primärspannung, U Eins komplex, und die Sekundärspannung, U Zwei komplex, können nun ins Verhältnis gesetzt werden. Stellt man die jeweiligen Induktionsberechnungen gegenüber, lassen sich die Änderungsraten des magnetischen Flusses herauskürzen und es bleiben nur die Windungszahlen übrig. Das Verhältnis N1 durch N2 ergibt das Übersetzungsverhältnis, klein Ü, welches im Modell des idealen Transformators Aufschluss über die Spannungsänderung gibt. Ist das Übersetzungsverhältnis, Ü größer als Eins, wird die Spannung herabgesetzt, auch Abwärtstransformator genannt. Ist das Übersetzungsverhältnis, Ü kleiner als Eins, wird die Spannung erhöht. Diese werden Aufwärtstransformatoren genannt. Da der Transformator in beide Richtungen eingesetzt werden kann, wird als Angabe in den technischen Daten das Übersetzungsverhältnis immer größer als Eins angegeben.}
	\speech{PrinzipTransformator}{4}{Die gestrichelten Linien in der Abbildung stellen den in realen Transformator vorkommenden Streufluss dar. Der Streufluss beschreibt den Anteil des von der Primärspule erzeugten magnetischen Flusses, der die Sekundärspule nicht durchdringt. Die Gründe für den Streufluss sind vielseitig. Ein höherer Abstand zwischen den beiden Spulen, höhere Frequenzen, fehlende Abschirmung des Transformators vor magnetischen Fremdquellen und die geometrische Anordnung der Spulenwicklungen und des Kerns sind einige Beispiele für das Auftreten des Streuflusses. Diese Verluste werden in der Streuinduktivität gemessen, worauf wir gleich in der Betrachtung des Ersatzschaltbilds des realen Transformators näher eingehen werden. Festzuhalten ist, dass im realen Transformator die Sekundärspannung stets kleiner ist als die errechnete Spannung im idealen Trafo.}
\end{frame}

\newvideofile{aufbauTrafo}{Aufbau des Transformators}
\begin{frame}
	\ftb{Aufbau des Transformators}
	\s{Vereinfacht dargestellt besteht der Transformator aus zwei Spulen, die mittels eines magnetisch leitenden Kerns miteinander gekoppelt werden (siehe Abbildung \ref{AbbTrafoAufbau}). Der Kern besteht aus einem ferromagnetischen Material. Durch seine hohe Permeabilität hat das ferromagnetische Material die Fähigkeit, den magnetischen Fluss möglichst effizient zwischen zwei Spulen zu leiten (siehe Modul 6). Effekte wie die Hysterese und Wirbelstromverluste stören jedoch den widerstandslosen Fluss. Diese Effekte werden als Eisenverluste bezeichnet. % In der Speech auf den Aufbau in der Praxis eingehen? Blech/Luftspalte etc

		Die Spulen sind um den magnetischen Leiter gewickelt. Wie oben erwähnt, bestimmen die Windungszahlen das Übersetzungsverhältnis der Spannung (Gleichung \ref{Gluebertragung}). Sie bestehen aus elektrisch gut leitenden Drähten wie Kupfer. Auch diese weisen in der Realität Widerstandsverluste auf, die auch Kupferverluste genannt werden.
	}
	\begin{columns}
		\column[c]{0.62\textwidth}
		\fu{
			\def\bi{3.0}%Breite innen
			\def\hi{4.0}%Höhe innen
			\def\ba{5.0}%breite außen
			\def\ha{6.0}%Höhe außen
			\def\dx{0.8}%x-Verschiebung
			\def\dy{0.5}%y-Verschiebung
			\def\dr{0.02}% round corner correction
			\def\countPrim{8}%Anzahl Primärwindungen
			\def\countSek{12}%Anzahl Sekundärwindungen
			
			%Berechnungen
			\def\lx{(\ba/2 - \bi/2)}%Breite Schenkel
			\def\ly{(\ha/2 - \hi/2)}%Höhe Schenkel
			\def\dyPrim{((\hi-\dy) / (\countPrim+1)/4)} %y-Verschiebung Wicklung Primärseite
			\def\dySek{((\hi-\dy) / (\countSek+1)/4)} %y-Verschiebung Wicklung Sekundärseite
			
			
			\begin{circuitikz}[thick, every node/.style={transform shape, scale=1}, decoration={markings, mark=at position 0.5 with {\arrow{latex}}}]%global scale/.style={scale=1.0}, rotate=-5, xslant=-0.1
				\begin{scope}[even odd rule]
					\filldraw[rounded corners=2pt, fill=gray, rotate=-0, opacity=1.0] (\dx,\dy) rectangle ++(\ba,\ha) ({\lx+\dx},{\ly+\dy}) rectangle ++(\bi, \hi);%hinten
					\fill [rounded corners=2pt, fill=gray] (\ba, 0) --++ (0, \dy+\dr+\dr) --++(\dx, 0) --cycle;%rechte Seite
					\fill [rounded corners=2pt, fill=gray] (0, \ha) --++ (\dx+\dr+\dr, 0) --++(0, \dy)--cycle;%obere Seite
					\filldraw[rounded corners=2pt, fill=gray!50, rotate=-0] (0,0) rectangle ++(\ba, \ha) ({\lx},{\ly}) rectangle ++(\bi, \hi);%vorne
					\draw (\ba-\dr,\dr) --++(\dx, \dy);%Eckverbindungen
					\draw (\ba-\dr,\ha-\dr) --++(\dx, \dy);%Eckverbindungen
					\draw (\dr,\ha-\dr) --++(\dx, \dy);%Eckverbindungen
				\end{scope}
				
				%Primärwicklung
				%Zuleitung oben
				\draw[rounded corners=2pt, blue, thick, postaction={decorate}] (-1,{\ly+\dr+\hi-\dy}) coordinate (U1oben)%Startpunkt oben
				-- ++(1,0) node[black, above, pos=0.4] {$\underline{I}_1$}; %Zuleitung
				%oberste Wicklung Primärwicklung
				\draw[rounded corners=2pt, blue, thick]
				(0,{\ly+\dr+\hi-\dy}) %Startpunkt oben
				-- ++({\lx}, {-\dyPrim})% vordere Linie
				-- ++({\dx + 0.07},{-\dyPrim + \dy})%Linie nach hinten
				-- ++(-0.06,0.06); %Kreis rechts nach hinten
				%restliche Primärwicklungen
				\foreach \n in {2,...,\countPrim}
					{
						\draw[rounded corners=2pt, blue, thick]
						(0,{\ly+\n*(\dyPrim * 4) + 0.08}) %Startpunkt oben
						-- ++(-0.07, -0.08)%Kreis links
						-- ++({\lx + 0.07}, {-\dyPrim})% vordere Linie
						-- ++({\dx + 0.07},{-\dyPrim + \dy})%Linie nach hinten
						-- ++(-0.06,0.06); %Kreis rechts nach hinten
					}
				%untere Zuleitung Primärwicklung
				\draw [blue, thick, postaction={decorate}] (0, {\ly+((\hi-\dy) / (\countPrim+1))}) -- +(-1,0) coordinate (U1unten);
				%Primärspannung
				\draw (U1oben) to[open, v^=$\underline{U}_1$, o-o] (U1unten);
				
				%Sekundärwicklung
				%obere Zuleitung
				\draw [red, thick, postaction={decorate}] ({\ba+\dx + 1},{\ly+\hi-\dySek}) coordinate (U2oben)
				-- ++(-0.83,0) node[black, above, pos=0.4] {$\underline{I}_2$};
				%oberste Windung Sekundärwicklung
				\draw[rounded corners=2pt, red, thick]
				({\bi+\lx},{\ly+\hi-\dy-\dySek + 0.08}) %Startpunkt unten
				-- ++(-0.07, -0.08)%Kreis links
				-- ++({\lx + 0.07}, {\dySek})% vordere Linie
				-- ++({\dx + 0.07}, {-\dySek + \dy}) coordinate(TO)
				-- ++(0.1,0);%Linie nach hinten
				%restliche Sekundärwicklungen
				\foreach \n in {2,...,\countSek}
					{
						\draw[rounded corners=2pt, red, thick]
						({\bi+\lx},{\ly+\n*(\dySek * 4) - \dySek + 0.08}) %Startpunkt unten
						-- ++(-0.07, -0.08)%Kreis links
						-- ++({\lx + 0.07}, {\dySek})% vordere Linie
						-- ++({\dx + 0.07}, {-\dySek + \dy})%Linie nach hinten
						-- ++(-0.06,0.06); %Kreis rechts nach hinten
					}
				%untere Zuleitung Sekundärwicklung
				\draw [red, thick, postaction={decorate}] (\ba+\dx-\dr,{\ly+\dy+((\hi-\dy) / (\countSek+1))})-- +(1,0) coordinate (U2unten);
				%Primärspannung
				\draw (U2oben) to[open, v=$\underline{U}_2$, o-o] (U2unten);
				
				%Richtungspfeile Fluss
				\draw[dashed, rounded corners=10pt, <-] ({\ba*3/5},{\ly/2}) -- ({\ba-\lx/2},{\ly/2}) -- ++(0,{\ha-\ly}) -- ++({-\ba+\lx},0) -- ++(0,{-\ha+\ly}) -- ++({\ba*2/5},0) node[right] {$\varPhi$};
				
			\end{circuitikz}
		}{{\bf Schematischer Aufbau des Transformators.} Zwei Spulen sind mittels eines ferromagnetischen Kerns miteinander gekopppelt. Die Windungszahlen bestimmen dabei das Übersetzungsverhältnis der Spannung. \label{AbbTrafoAufbau}}

		\s{In dem Schaltbild des idealen Transformators werden die magnetisch gekoppelten Spulen als nebeneinanderstehende, schwarz ausgefüllte Rechtecke dargestellt (siehe Abbildung \ref{AbbTrafoESBIdeal}). Die zwei Striche zwischen den Spulen weisen darauf hin, dass diese mittels eines durchgehenden magnetisch leitenden Kerns verbunden sind. Die Punkte an den jeweiligen Spulen geben Aufschluss über die Führung der Wicklung. Sie können sich unten oder oben befinden. Liegen sie dabei auf der gleichen Höhe, ist die Phasenlage der beiden Spulen gleich. Liegen diese auf unterschiedlichen Höhen, bedeutet dies einen umgekehrten Wicklungssinn und dadurch eine Phasenverschiebung von $180\degree$.}

		\column[c]{0.38\textwidth}
		\pause
		\fu{
			\begin{circuitikz}
	\draw (0,0) node[transformer core] (T) {};
	\draw (T.A2) to[short, -o] ++(-0.4,0) coordinate (U1unten);
	\draw (T.A1) to[short, -o, i_<=$\underline{I}_1$] ++(-0.4,0) to[open, v^=$\underline{U}_1$] (U1unten);
	\draw (T.B2) to[short, -o] ++(0.4,0) coordinate (U2unten);
	\draw (T.B1) to[short, -o, i<=$\underline{I}_2$] ++(0.4,0) to[open, v=$\underline{U}_2$] (U2unten);
	\draw (T.inner dot A1) node[circ]{};
	\draw (T.inner dot B1) node[circ]{};
\end{circuitikz}
		}{{\bf Ideales Schaltbild eines Transformators.} \label{AbbTrafoESBIdeal}}
		\b{\begin{eqa}
				\textit{ü} < 1 &\rightarrow \text{Spannung hoch}\nonumber\\
				\textit{ü} > 1 &\rightarrow \text{Spannung runter}\nonumber
			\end{eqa}
			Angegeben wird immer $\textit{ü} > 1$
		}
	\end{columns}
	\speech{AufbauTransformator}{1}{Vereinfacht dargestellt besteht der Transformator aus zwei Spulen, die mittels eines magnetisch leitenden Kerns miteinander gekoppelt werden. Der Kern besteht für gewöhnlich aus einem ferromagnetischen Material. Wie in Kapitel Vier Punkt Zwei erwähnt, weisen ferromagnetische Materialien eine hohe Permeabilität auf, um den magnetischen Fluss zwischen den Spulen möglichst effizient zu gestalten. Eine widerstandslose Leitung ist leider nicht möglich. Die Hysterese und Wirbelstromverluste stören den magnetischen Fluss. Diese beiden Effekte werden als Eisenverluste bezeichnet. Um die Wirbelstromverluste zu reduzieren, besteht in der Praxis der ferromagnetische Leiter nicht aus einem zusammenhängenden Kern, sondern aus vielen kleinen laminierten Blechen. Dies schränkt die Bewegung der Wirbelströme ein und reduziert somit die Übertragungsverluste. Hystereseverluste werden durch die Wahl eines Materials mit hoher Permeabilität eingegrenzt. Die Spulen sind um den magnetischen Leiter gewickelt. Wie zuvor erwähnt, bestimmt das Verhältnis der Windungszahlen das Übersetzungsverhältnis der Spannungen. Der Wicklungssinn an der Primärspule bestimmt mittels Rechter-Hand-Regel die Richtung des magnetischen Flusses. Die Ausrichtung der Sekundärspule gibt Auskunft über die Phasenlagen der beiden Spulen. Ist der Wicklungssinn der Spulen gleichgerichtet, also beide sind im Uhrzeigersinn oder beide gegen den Uhrzeigersinn gewickelt, dann ist die Phasenlage der Spulen gleich. Ist der Wicklungssinn entgegengerichtet, sind die Phasenlagen um Einhundertachzig Grad versetzt. Bei Transformatoren mit u-förmigem und geschlossenem Eisenkern sollten die Drehungen des magnetischen Kerns berücksichtigt werden. Die Wicklungen wirken aufgrund der Drehungen des Kerns entgegengesetzt gewickelt, sind jedoch im gleichen Wicklungssinn. Die Spulen bestehen aus isolierten Drähten aus Kupfer oder Aluminium. In der Realität weisen auch diese Widerstandsverluste auf. Die Verluste werden im Kontext des Transformators Kupferverluste genannt.}
	\speech{AufbauTransformator}{2}{Im Ersatzschaltbild des idealen Transformators werden die magnetisch gekoppelten Spulen als nebeneinanderstehende, schwarz ausgefüllte Rechtecke dargestellt. Die zwei Striche zwischen den Spulen weisen darauf hin, dass diese mittels eines durchgehenden magnetisch leitenden Kerns verbunden sind. Die Punkte an den jeweiligen Spulen geben Aufschluss über die Führung der Wicklung. Sie können sich unten oder oben befinden. Liegen sie dabei auf der gleichen Höhe, haben sie den gleichen Wicklungssinn. Die Phasenlage der beiden Spulen ist in diesem Fall gleich. Liegen diese auf unterschiedlichen Höhen, bedeutet dies einen umgekehrten Wicklungssinn und dadurch eine Phasenverschiebung von Einhundertachzig Grad. Generell werden Transformatoren so konzipiert, dass die elektrische Energie möglichst verlustfrei von der Primärseite zur Sekundärseite transportiert wird. Jedoch werden auch Transformatoren bewusst so konstruiert, dass sie einen höheren Streufluss haben, um eine Kurzschlussfestigkeit oder eine Strombegrenzung am Ausgang zu erreichen. Diese werden umgangssprachlich als weiche Transformatoren bezeichnet. Transformatoren, deren Ziel die effiziente Leistungsübertragung ist, werden als harte Transformatoren bezeichnet. Abschließend kommen wir auf das charakteristische Summen des Transformators. Dieses lässt sich auf den Wechselstrom und den Aufbau des Trafos zurückführen. Am besten lässt sich das Summen am Beispiel eines Transformators mit lamellierten Metallkern beschreiben. Durch den Wechselstrom ziehen sich die einzelnen Bleche zusammen und dehnen sich wieder aus, was sie zum Schwingen bringt. Diese Schwingungen erzeugen das bekannte Summen des Transformators. An sich schwingt jeder Transformator. Die Lautstärke des Summens ist abhängig von der Größe und des Aufbaus des Transformators sowie dessen Belastung.}
\end{frame}


\newvideofile{uebungTrafoIdeal}{Beispiel: Idealer dreiphasiger Netztransformator}
\begin{frame}
	\ftx{Beispiel: Idealer dreiphasiger Netztransformator}
	\begin{bsp}{Idealer dreiphasiger Netztransformator}{}
		Ein idealer dreiphasiger Netztrafo\index{Netztransformator} hat eine Leistung von $\underline{S}_\mathrm{N} = 100\,\rm{kVA}$, \\ eine Oberspannung von $\underline{U}_1 = 20\,\rm{kV}$ (Dreieck) und eine Unterspannung von $\underline{U}_2 = 400\,\rm{V}$ (Dreieck).
		\b{\vspace{1cm}}
		\begin{enumerate}
			\only<1-2>{
			\item[a)] Wie groß ist das Übersetzungsverhältnis?\pause
			      \begin{eqa}
				      \textit{ü} &= \frac{\underline{U}_1}{\underline{U}_2} = \frac{N_1}{N_2} \nonumber\\
				      &= \frac{20\,\rm{kV}}{400\,\rm{V}} = 50\nonumber
			      \end{eqa}
			      }
			      \only<3->{
			\item[b)] Wie groß sind der Primär- und Sekundärstrom im Nennbetrieb bei einem Leistungsfaktor $\cos(\varphi)=1$?\pause
			      \begin{eqa}
				      \cos(\varphi)=1 &\rightarrow |\underline{S}| = P\nonumber\\
				      P &= U \cdot I\nonumber\\
				      I_1 &= \frac{P}{U} = \frac{100\,\rm{kW}}{20\,\rm{kV}} = 5\,\rm{A}\nonumber\\
				      I_2 &= \frac{P}{U} = \frac{100\,\rm{kW}}{400\,\rm{V}} = 250\,\rm{A}\nonumber
			      \end{eqa}
			      }
		\end{enumerate}
	\end{bsp}
	\speech{UebungTrafoIdeal}{1}{Wollen wir nun einige Kenngrößen des idealen Transformators berechnen. Im folgenden Beispiel ist eine Scheinleistung, S N komplex, von 100 Kilovoltampere gegeben. Die angelegte Spannung an der Primärseite der Spule, die generell als Oberspannung bezeichnet wird, beträgt 20 Kilovolt. Diese Spannung wandelt der Transformator in eine Sekundärspannung, auch Unterspannung genannt, in Höhe von 400 Volt um. Mittels dieser Kenngrößen ermitteln wir zuerst das Übersetzungsverhältnis.}
	\speech{UebungTrafoIdeal}{2}{Wie zuvor erwähnt, lässt sich das Übersetzungsverhältnis, klein Ü, aus dem Spannungsverhältnis der Oberspannung, U Eins komplex, zu der Unterspannung, U Zwei komplex, bzw. dem Windungsverhältnis, N1 zu N2, berechnen. Leider sind uns in diesem Beispiel die Windungen der Primär- und Sekundärspule nicht bekannt. Da in einem idealen Transformator keine Spannungsverluste entstehen, können wir anstatt der Windungszahlen der Spulen, die Spannungsverhältnisse zur Berechnung nutzen. Die nun eingesetzte Oberspannung von 20 Kilovolt geteilt durch die Unterspannung von 400 Volt ergibt ein Übersetzungsverhältnis von Fünfzig.}
	\speech{UebungTrafoIdeal}{3}{Im nächsten Schritt soll herausgefunden werden, wie hoch der Primär- und der Sekundärstrom im Nennbetrieb bei einem Leistungsfaktor, Cosinus F i, von Eins ist. Betrachten wir nun den Leistungsfaktor. Dieser beschreibt das Verhältnis von Wirkleistung, groß P, zur Scheinleistung, S komplex. Ist der Leistungsfaktor gleich Eins, bedeutet das, dass keine Verluste zwischen Scheinleistung und  Wirkleistung entstehen. Die angegebene Scheinleistung von 100 Kilovoltampere wird also komplett in die Wirkleistung von 100 Kilowatt gewandelt. Zur Berechnung der Primär- und Sekundärströme nehmen wir die Formel zur elektrischen Leistung zuhilfe. Die Leistung, P, setzt sich zusammen aus dem Produkt von Spannung, U, sowie dem Strom, i . Da uns die Spannungen und nun auch die elektrische Leistung bekannt sind, lassen sich durch Umformung der Formel die jeweiligen Ströme berechnen. Hierzu wird die Spannung, U, durch Division auf die Seite der Leistung, P, übertragen. Die einzelnen Ströme, I Eins und I Zwei, lassen sich nun durch das Verhältnis von Leistung und jeweiliger Spannung berechnen. Der Primärstrom, I Eins, entspricht also dem Quotienten der Leistung von 100 Kilowatt und der Oberspannung von 20 Kilovolt. Damit ergibt sich ein Primärstrom von 5 Ampere. In der selben Weise ergibt sich für, I Zwei, mit einer Unterspannung von 400 Volt ein Sekundärstrom von Zweihundertfünfzig Ampere.}
\end{frame}

\begin{frame}
	\ftb{Ersatzschaltbild}
	\s{Zum besseren Verständnis wird das Ersatzschaltbild des realen Transformators (Abbildung \ref{AbbTrafoESBReal}) zunächst funktionsgetreu als galvanisch getrenntes System mit einem Primär- und einem Sekundär\-stromkreis dargestellt. Im Vergleich zum idealen Transformator ist dieses Ersatzschaltbild um die zuvor beschriebenen Verluste erweitert. Der magnetische Kreis wird mit der Hauptinduktivität $L_h$ und den Streuinduktivitäten $L_{\sigma1}$ und $L_{\sigma2}$ in einen elektrischen Kreis transformiert. Die zuvor beschriebenen Kupferverluste in der Primär- und Sekundärspule werden durch die Widerstände $R_1$ und $R_2$ repräsentiert. Die Größe $\underline{I}\mu$ steht für den Magnetisierungsstrom und beschreibt jene Komponente des Leerlaufstroms, die erforderlich ist, um den magnetischen Kern des Transformators zu erregen. Die Eisenverluste werden aus Gründen der Vereinfachung in diesem Beispiel außer acht gelassen. Wenn sie berücksichtigt werden sollen, werden die Eisenverluste durch eine Parallelschaltung eines Widerstandes zur Hauptinduktivität dargestellt.}
	\fu{
		\b{\vspace{-1cm}}
		\resizebox{\textwidth}{!}{
			\b{
				\begin{circuitikz}
					\onslide<1-5>{
						\draw (0,0) node[transformer core] (T) {} (T.base) node{T};
						\draw (T.A2) to[short] ++(-1,0) coordinate (T1U);
						\draw (T.B1) to[short] ++(0.2,0) coordinate (T2O);
						\draw (T.B2) to[short] ++(0.2,0) coordinate (T2U);
						\draw (T.inner dot A1) node[circ]{};
						\draw (T.inner dot B1) node[circ]{};
						%Strom transformiert
						\onslide<2->{\draw (T.A1) to[short, i_<=$\underline{I}^\prime_2$] ++(-1,0) coordinate (T1O);}
						\onslide<1> {\draw (T.A1) to[short] ++(-1,0) coordinate (T1O);}
						%Spannung transformiert
						\onslide<2>{\draw (T1O) to[open, v=$\underline{U}^\prime_2$] (T1U);}
						%Sekundärseite
						\onslide<4-5>{\draw (T2O) to[L=$L_{\sigma 2}$] ++(2,0) coordinate (y1);} %Streuinduktivität rechts
						\onslide<1-3>{\draw (T2O) to[short] ++(2,0) coordinate (y1);}
						\onslide<5>  {\draw (y1) to[R=$R_2$] ++(2,0) coordinate (y2);} %Widerstand rechts
						\onslide<1-4>{\draw (y1) to[short] ++(2,0) coordinate (y2);}
						\draw (y2) to[short, i=$\underline{I}_2$, -o] ++(1,0) coordinate (U2O);%Anschluß rechts oben
						\draw (T2U) to[short, -o] ++(5,0) coordinate (U2U); %Anschluß rechts unten
					}
					\onslide<6>{
						\draw (4,0) node[transformer core] (T) {} (T.base) node{T};
						\draw (T.A1) to[short, i_<=$\underline{I}^\prime_2$] ++(-1,0) coordinate (u2transO) to[L=$L_{\sigma 2}^\prime$] ++(-2,0) to[R=$R_2^\prime$] ++(-2,0) coordinate (T1O);
						\draw (T.A2) to[short] ++(-1,0) coordinate (u2transU) -- ++(-4,0) coordinate (T1U);
						\draw (T.B1) to[short] ++(0.2,0) coordinate (T2O);
						\draw (T.B2) to[short] ++(0.2,0) coordinate (T2U);
						\draw (T.inner dot A1) node[circ]{};
						\draw (T.inner dot B1) node[circ]{};
						\draw (T2O) to[short, i=$\underline{I}_2$, -o] ++(1,0) coordinate (U2O);%Anschluß rechts oben
						\draw (T2U) to[short, -o] ++(1,0) coordinate (U2U); %Anschluß rechts unten
						%Spannung transformiert
						\draw (u2transO) to[open, v=$\underline{U}^\prime_2$] (u2transU);
					}

					%Primärseite
					\onslide<3-> {\draw (T1O) to[L=$L_\mathrm{h}$, *-*, i=$\underline{I}_\mu$](T1U);} %Hauptinduktivität
					\onslide<4-> {\draw (T1O) to[L=$L_{\sigma 1}$] ++(-2,0) coordinate (x1);} %Streuinduktivität links
					\onslide<1-3>{\draw (T1O) to[short] ++(-2,0) coordinate (x1);}
					\onslide<5-> {\draw (x1) to[R=$R_1$] ++(-2,0) coordinate (x2);} %Widerstand links
					\onslide<1-4>{\draw (x1) to[short] ++(-2,0) coordinate (x2);}
					\draw (x2) to[short, i_<=$\underline{I}_1$, -o] ++(-1,0) coordinate (U1O); %Anschluß links oben
					\draw (T1U) to[short, -o] ++(-5,0) coordinate (U1U); %Anschluß links unten
					\draw (U1O) to[open, v=$\underline{U}_1$] (U1U); %Spannung U1 links

					%Sekundärseite
					\draw (U2O) to[open, v=$\underline{U}_2$] (U2U); %Spannung U2 rechts
				\end{circuitikz}
			}
			\s{
				\begin{circuitikz}
	\draw (0,0) node[transformer core] (T) {} (T.base) node{T};
	\draw (T.A2) to[short] ++(-1,0) coordinate (T1U);
	\draw (T.B1) to[short] ++(0.2,0) coordinate (T2O);
	\draw (T.B2) to[short] ++(0.2,0) coordinate (T2U);
	\draw (T.inner dot A1) node[circ]{};
	\draw (T.inner dot B1) node[circ]{};
	%Strom transformiert
	\draw (T.A1) to[short, i_<=$\underline{I}^\prime_2$] ++(-1,0) coordinate (T1O);
	%Sekundärseite
	\draw (T2O) to[L=$L_{\sigma 2}$] ++(2,0) coordinate (y1); %Streuinduktivität rechts
	\draw (y1) to[R=$R_2$] ++(2,0) coordinate (y2); %Widerstand rechts
	\draw (y2) to[short, i=$\underline{I}_2$, -o] ++(1,0) coordinate (U2O);%Anschluß rechts oben
	\draw (T2U) to[short, -o] ++(5,0) coordinate (U2U); %Anschluß rechts unten
	%Primärseite
	\draw (T1O) to[L=$L_\mathrm{h}$, *-*, i=$\underline{I}_\mu$](T1U); %Hauptinduktivität
	\draw (T1O) to[L=$L_{\sigma 1}$] ++(-2,0) coordinate (x1); %Streuinduktivität links
	\draw (x1) to[R=$R_1$] ++(-2,0) coordinate (x2); %Widerstand links
	\draw (x2) to[short, i_<=$\underline{I}_1$, -o] ++(-1,0) coordinate (U1O); %Anschluß links oben
	\draw (T1U) to[short, -o] ++(-5,0) coordinate (U1U); %Anschluß links unten
	\draw (U1O) to[open, v=$\underline{U}_1$] (U1U); %Spannung U1 links

	%Sekundärseite
	\draw (U2O) to[open, v=$\underline{U}_2$] (U2U); %Spannung U2 rechts
\end{circuitikz}
			}
		}
	}{{\bf Ersatzschaltbild eines galvanisch getrennten Transformators.} \label{AbbTrafoESBReal}}

	\s{Anschließend wird in dem T-Ersatzschaltbild des Transformators (Abbildung \ref{AbbTrafoESBReal1}) unterstellt, dass keine galvanische Trennung zwischen Primär- und Sekundärseite besteht, obwohl dies in der Realität natürlich der Fall ist. Dieser Schritt ermöglicht die Umrechnung der Größen der Sekundärseite auf die Primärseite und die Anwendung der aus Modul 4 bekannten Rechenregeln. Die Umrechnungsgrößen werden als gestrichne Größen, in diesem Beispiel als $\underline{U}^\prime_2$, $\underline{I}^\prime_2$, ${R}^\prime_2$ und ${L}^\prime_{\sigma2}$, gekennzeichnet.
		\fu{
			%\resizebox{\textwidth}{!}{
			\begin{circuitikz}
    \draw (0,0) to[open, v<=$\underline{U}_1$] ++(0,3)
    to[short, i=$\underline{I}_1$, o-] ++(1,0)
    to[R=$R_1$] ++(2,0)
    to[L=$L_{\sigma 1}$] ++(2,0) coordinate (x1)
    to[L=$L_{\sigma 2}^\prime$] ++(2,0)
    to[R=$R_2^\prime$] ++(2,0)
    to[short, i=$\underline{I}_2^\prime$, -o] ++(1,0)
    to[open, v=$\underline{U}_2^\prime$] ++(0,-3)
    to[short, o-o] (0,0)
    (x1) to[L=$L_\mathrm{h}$, *-*, i=$\underline{I}_\mu$] ++(0,-3);
\end{circuitikz}
			%}
		}{{\bf T-Ersatzschaltbild eines Transformators.} \label{AbbTrafoESBReal1}}
	}
	\begin{columns}
		\column[c]{0.5\textwidth}
		\onslide<2->{
			\b{
				\begin{eqa}
					\underline{U}_1 &= \underline{U}^\prime_2 = \frac{N_1}{N_2}\cdot \underline{U}_2\\
					\underline{I}_1 &= \underline{I}^\prime_2 \only<3>{\red{ + \underline{I}_\mu}} \only<4->{ + \underline{I}_\mu} = \frac{N_2}{N_1}\cdot \underline{I}_2 \only<3>{\red{ + \underline{I}_\mu}} \only<4->{ + \underline{I}_\mu}
				\end{eqa}
			}
			\s{Für die Umrechung der Spannung auf die Primärseite wird die Spannung $\underline{U}^\prime_2$ mit dem Übersetzungsverhältnis multipliziert (Gleichung \ref{Glspannungsumrechnung}). Zur Berechnung des eingehenden Stroms $\underline{I}_1$ wird der Strom $\underline{I}_2$ mit dem umgekehrten Übersetzungsverhältnis multipliziert und anschließend mit dem Magnetisierungsstrom $\underline{I}_\mu$ addiert (Gleichung \ref{Glstromumrechnung}).
				%negatives Vorzeichen wird ausgeblendet: Da zur Erhaltung des Spannungsgleichgewichts im primärseitigen Kreis der durch den Strom im Sekundärkreis hervorgerufene magnetische Fluss durch den primärseitigen Fluss kompensiert werden muss, gilt bei gleichgerichtetem Wicklungssinn das negative Vorzeichen und bei entgegengerichtetem Wicklungssinn das positive Vorzeichen.\fi
				\begin{eqa}
					\underline{U}_1 = \underline{U}^\prime_2 &= \frac{N_1}{N_2}\cdot \underline{U}_2 \label{Glspannungsumrechnung} \\
					\underline{I}_1 = \underline{I}^\prime_2 + \underline{I}_\mu &= \frac{N_2}{N_1}\cdot \underline{I}_2 + \underline{I}_\mu \label{Glstromumrechnung}
				\end{eqa}
			}
		}
		\onslide<3->{
			\b{
				\begin{description}
					\item[$L_\mathrm{h}$] Hauptinduktivität\index{Hauptinduktivität}
					      \onslide<4->{\item[$L_\sigma$] Streuinduktivität\index{Streuinduktivität}}
					      \onslide<5->{\item[$R$] Widerstandsverluste}
				\end{description}
			}
		}
		\column[c]{0.5\textwidth}
		\onslide<6->{
			\s{Die auftretenden Verluste auf der Sekundärseite $L_{\sigma 2}^\prime$ und $R_2^\prime$ werden mit dem quadrierten Übersetzungsverhältnis auf die Primärseite umgerechnet (Gleichungen \ref{Glstreuinduktivitätumrechnung} und \ref{Glwiderstandumrechnung}).} %Hintergrund für die Quadrierung sind die Joulesche Verluste. joulesche Verluste
			\b{
				\vspace{-1cm}
				\begin{eqa}
					L_{\sigma 2}^\prime &= \left(\frac{N_1}{N_2}\right)^2\cdot L_{\sigma 2}\\
					R_2^\prime &= \left(\frac{N_1}{N_2}\right)^2\cdot R_2
				\end{eqa}
			}
			\s{
				\begin{eqa}
					L_{\sigma 2}^\prime &= \left(\frac{N_1}{N_2}\right)^2\cdot L_{\sigma 2} \label{Glstreuinduktivitätumrechnung} \\
					R_2^\prime &= \left(\frac{N_1}{N_2}\right)^2\cdot R_2\label{Glwiderstandumrechnung}
				\end{eqa}
			}
			\b{Der ideale Trafo rechts wird meistens weggelassen.}
		}
	\end{columns}
	\nomenclature[F]{$L_\mathrm{h}$}{Hauptinduktivität eines Transformators}
	\nomenclature[F]{$L_\sigma$}{Streuinduktivität eines Transformators}
	\speech{realerTrafo}{1}{Zur besseren Übersicht und als Berechnungsgrundlage des realen Transformators bedienen wir uns zunächst des bekannten Ersatzschaltbilds des idealen Transformators des galvanisch getrennten Systems. Der Primär- und der Sekundärstromkreis mit ihren jeweiligen Spannungsgrößen, U1 und U2, sowie Stromgrößen, I1 und I2, sind visuell separiert. Im Folgenden werden alle für die Berechnung des realen Transformators notwendigen Größen schrittweise auf die Primärseite gebracht und umgerechnet. Auch die zuvor erwähnten Verluste werden nun mit berücksichtigt.}
	\speech{realerTrafo}{2}{Die Sekundärspannung und der Sekundärstrom werden zunächst auf die Primärseite transformiert. Hierzu werden im Ersatzschaltbild zunächst die Größen, U2 komplex, und, i2 komplex, als Transformationsgrößen, U Strich 2 komplex, und, i Strich Zwei komplex, auf der Primärseite dargestellt. Rechnerisch verhält sich die Sekundärspannung, U2 komplex, gleich der Primärspannung, U Eins komplex, da die Sekundärspannung, U2 komplex, wie in der Berechnung des idealen Transformators mit dem Übersetzungsverhältnis, N1 durch N2, multipliziert wird. Analog hierzu verhält sich der Sekundärstrom, i Strich 2 komplex, gleich dem Primärstrom, i Strich 1 komplex, jedoch mit dem Unterschied, dass der Sekundärstrom mit dem umgekehrten Übersetzungsverhältnis, N2 durch N1, multipliziert wird.}
	\speech{realerTrafo}{3}{Als nächstes wird das Ersatzschaltbild um die Hauptinduktivität, L H, und dem Magnetisierungsstrom, i Mü komplex, erweitert. Die Hauptinduktivität erzeugt das magnetische Hauptfeld, welches die Primärseite mit der Sekundärseite verbindet. Es spiegelt somit die energetische Übertragung von der Primärseite zu Sekundärseite wider. Zur Erzeugung des magnetischen Hauptfeldes wird ein Strom benötigt. Dieser wird als Magnetisierungsstrom bezeichnet und wird beim Aufbau des magnetischen Feldes verbraucht. Folglich stellt der Magnetisierungsstrom einen Verlust dar, welcher nicht im Sekundärstrom umgesetzt wird. Dementsprechend wird der Magnetisierungsstrom, i Mü komplex, in der Berechnung zum transformierten Sekundärstrom, i Strich Zwei komplex, hinzu addiert.}
	\speech{realerTrafo}{4}{Kommen wir zu den zuvor erwähnten Verlusten. In unserem Ersatzschaltbild sind dies zum einen die Streuinduktivität, L Sigma Eins und L Sigma Zwei, und die Widerstandsverluste, R Eins und R Zwei ... Wie beim Prinzip des Transformators erwähnt, beziffert die Streuinduktivität den entstandenen Streufluss der jeweiligen Spule. Streuinduktivitäten fallen auf beiden Seiten an. In unserem Ersatzschaltbild finden wir die Streuinduktivität, L Sigma Eins, auf der Primärseite und, L Sigma Zwei, auf der Sekundärseite. Im Aufbau des Transformators wurde auf die Kupferverluste bereits eingegangen. Dies sind die Widerstandsverluste, welche in den Drähten der beiden Spulen auftreten. Folglich werden auch jeweils ein Widerstand auf der Primärseite, R Eins, und ein Widerstand auf der Sekundärseite, R Zwei, im Ersatzschaltbild dargestellt. Generell wird unterstellt, dass die Streuinduktivität und die Widerstandsverluste auf der Primär- und auf der Sekundärseite gleich sind. Zu beachten ist, dass in unserem Beispiel bewusst auf Eisenverluste verzichtet wird. Sie stehen für die Hystereseverluste und Wirbelströme, die im Eisenkern entstehen. Im Ersatzschaltbild werden die Eisenverluste, groß R f e, parallel zur Hauptinduktivität illustriert. Da die Einbeziehung der Eisenverluste zu einer komplexen Rechnung führt, verzichten wir für ein besseres Verständnis des realen Transformators in diesem Beispiel auf die Eisenverluste.}
	\speech{realerTrafo}{5}{Die Streuinduktivität, L Sigma 2, und die Kupferverluste, R2, müssen nun auf die Primärseite gebracht werden. Im Ersatzschaltbild erscheinen dabei die beiden Verluste als die Größen, L Strich Sigma 2 und R Strich 2. Durch die Transformierung der beiden Größen verschiebt sich der Transformator in unseren Ersatzschaltbild nach rechts. Die Hauptinduktivität bleibt in der Mitte. Da nun alle relevanten Größen auf die Primärseite transformiert sind, verzichtet man in der Regel auf die Darstellung des idealen Trafos rechts im Bild. Um die beiden Verlustgrößen rechnerisch auf die Primärseite zu transformieren werden diese jeweils mit dem Übersetzungsverhältnis, N1 durch N2, zum Quadrat multipliziert. Die transformierte Streuinduktivität, L Strich Sigma 2, errechnet sich aus dem Produkt des quadrierten Übersetzungsverhältnisses multipliziert mit der Streuinduktivität, L Sigma 2. Entsprechend werden die Kupferverluste auf der Sekundärseite, R2, auch mit dem quadrierten Übersetzungsverhältnis multipliziert, um sie als, R Strich 2, auf die Primärseite zu bringen.}
	\speech{realerTrafo}{6}{Wort}
\end{frame}

\newvideofile{uebungTrafoReal}{Beispiel: Realer dreiphasiger Netztransformator}
\begin{frame}
	\ftx{Beispiel: Realer dreiphasiger Netztransformator}
	\begin{bsp}{Realer dreiphasiger Netztransformator}{}
		Ein realer dreiphasiger Netztrafo ($f=\,50\,\rm{Hz}$) hat folgende Daten:
		\begin{columns}
			\column[c]{0.4\textwidth}
			\begin{itemize}
				\item Leistung: $\underline{S}_\mathrm{N} = 100\,\rm{kVA}$
				\item Oberspannung $\underline{U}_1 = 20\,\rm{kV}$
				\item Übersetzungsverhältnis: $50$
			\end{itemize}
			\column[c]{0.6\textwidth}
			\begin{itemize}
				\item $L_\mathrm{h} = 500\,\rm{H}$ \onslide<2->{$\Rightarrow \underline{Z}_\mathrm{Lh} = j 157,08\,\rm{k}\Omega$}
				\item $L_{\sigma 1} = L_{\sigma 2}^\prime = 190\,\rm{mH}$ \onslide<2->{$\Rightarrow \underline{Z}_{\rm{L}\sigma} = j 59,69\,\Omega$}
				\item $R_1 = R_2^\prime = 30\,\Omega$
			\end{itemize}
		\end{columns}
		\b{\vspace{1cm}}
		\begin{enumerate}
			\only<1-3>{
			\item[a)] Wie groß ist die Leerlaufspannung $\underline{U}_2$ bei diesem Übersetzungsverhältnis?
			      \onslide<3>{
				      \begin{eqa}
					      \underline{U}_2^\prime &= \underline{U}_1\cdot\frac{\underline{Z}_\mathrm{h}}{\underline{Z}_\mathrm{h} + \underline{Z}_{\sigma 1} + R_1}\nonumber\\
					      &= 19,992\,\text{kV}\cdot e^{j0,011\degree}\nonumber\\
					      \underline{U}_2 &= 399,85\,\text{V}\cdot e^{j0,011\degree}\nonumber
				      \end{eqa}
			      }
			      }
			      \only<4-5>{
			\item[b)] Wie groß ist der Leerlaufstrom (auf der Primärseite)?
			      \onslide<5>{
				      \begin{eqa}
					      \underline{I}_1 &= \frac{\underline{U}_1}{\underline{Z}_\mathrm{h} + \underline{Z}_{\sigma 1} + R_1}\nonumber\\
					      &= \frac{20\,\text{kV}}{157,139\,\text{k}\Omega \cdot e^{j 89,989\degree}}\nonumber\\
					      &= 127,28\,\text{mA}\cdot e^{-j 89,989\degree}\nonumber
				      \end{eqa}
			      }
			      }
			      \only<6-7>{
			\item[c)] Wie groß sind die Leerlaufverluste?
			      \onslide<7>{
				      \begin{eqa}
					      \underline{S}_\mathrm{v} &= \underline{U}\cdot \underline{I}^\ast\nonumber\\
					      &= 20\,\text{kV} \cdot (127,28\,\text{mA}\cdot e^{-j 89,989\degree})^\ast\nonumber\\
					      &= 2,546\,\text{kVA}\cdot e^{j 89,989\degree}\nonumber
				      \end{eqa}
			      }
			      }
		\end{enumerate}
	\end{bsp}
	\speech{UebungRealerTrafo}{1}{Kommen wir zu einem Rechenbeispiel eines realen Transformators. Hierzu bedienen wir uns einiger Kenngrößen aus dem vorherigen Beispiel. Gegeben sei eine Scheinleistung, S N komplex, von Einhundert Kilovoltampere, eine Oberspannung, U Eins komplex, in Höhe von Zwanzig Kilovolt und ein Übersetzungsverhältnis von fünfzig. Diese Kenngrößen werden nun um die Hauptinduktivität, L H, in Höhe von Fünfhundert Henry, den beiden Streuinduktionen, L Sigma Eins, und, L Strich Sigma Zwei, von jeweils Einhundertneunzig Milli-Henry und den jeweiligen Kupferverlusten, R Eins, und, R Strich Zwei, in Höhe von Dreißig Ohm, erweitert. Als erste Aufgabe soll die Leerlaufspannung, U Zwei komplex, ermittelt werden.}
	\speech{UebungRealerTrafo}{2}{Für die Berechnung müssen zunächst die Induktivitätsgrößen in der Aufgabenstellung in elektrische Größen umgewandelt werden. Die Induktivitäten werden zur Vereinfachung als komplexe Zahlen in kartesischer Form vorgegeben. Für die Hauptinduktivität ist die Impedanz, Z L H komplex, von, J, Einhundersiebenundfünfzig Komma Null Acht Kiloohm gegeben. Die Streuinduktivität hat umgerechnet eine Impedanz, Z L Sigma komplex, von Neunundfünfzig Komma Sechs Neun Ohm. Nehmen wir uns die Zeit für einen kurzen Rückblick in das fünfte Kapitel. Die Impedanz einer Induktivität ergibt sich aus dem Produkt der imaginären Einheit, J, multipliziert mit der Kreisfrequenz, Omega, multipliziert mit dem Wert der Induktivität in der Einheit Henry.}
	\speech{UebungRealerTrafo}{3}{Für die Berechnung der Leerlaufspannung betrachten wir zunächst nur die transformierte Sekundärspannung, also, U Strich Zwei komplex. Da im Leerlauf auf der Sekundärseite der Strom Null ist, fallen die Streuinduktivität und der Widerstand der Sekundärseite weg. Die gesamte sekundäre Impedanz wird hauptsächlich durch die Hauptinduktivität repräsentiert. Relevant für die Berechnung sind nur die Impedanzen auf der Primärseite. Hierzu wird die Eingangsspannung mit dem Quotienten aus Hauptinduktivität, Z H komplex, zu der Summe aller relevanten Impedanzen, Z H komplex, Z Sigma Eins komplex und R Eins, multipliziert. Diese Formel ist aus Kapitel Drei Punkt Drei zur Berechnung des Spannungsteilers bekannt. Wichtig zu erwähnen ist, dass die Formel des Spannungsteilers, wie in unserem Beispiel, nur bei unbelasteten Systemen anwendbar ist. Da sich die polare Form besser zur Multiplikation und Division von komplexen Zahlen eignet und zusätzlich eine direkte Interpretation der Impedanz hinsichtlich ihres Betrags ermöglicht, wandeln wir mit dem Taschenrechner die komplexen Zahlen kartesischer Darstellung in die polare Darstellung um. Nach der Umstellung der Zahlen lässt sich die transformierte Sekundärspannung errechnen. Die auf die Primärseite transformierte Sekundärspannung beträgt Neunzehn Komma Neun Neun Zwei Kilovolt multipliziert mit einer Phasenverschiebung von Null Komma Null Eins Eins Grad. Die Phasenverschiebung wird als eulersche Zahl ausgedrückt, exponiert mit der imaginären Einheit, j ... Wird nun diese Zahl durch das vorgegebene Übersetzungsverhältnis von Fünfzig geteilt, erhalten wir die an der Sekundärseite anliegende Leerlaufspannung, von Dreihundertneunundneunzig Komma Acht Fünf Volt multipliziert mit der Phasenverschiebung von Null Komma Null Eins Eins Grad.}
	\speech{UebungRealerTrafo}{4}{Als Zweite Teilaufgabe wird nach dem Leerlaufstrom auf der Primärseite gefragt.}
	\speech{UebungRealerTrafo}{5}{Hierfür nutzen wir das Ohmsche Gesetz und stellen die Formel nach dem zu ermittelnden Strom um. In der Formel wird der Leerlaufstrom auf der Primärseite, i Eins komplex, mit dem Quotienten von, U Eins komplex, durch die Summe aller primärseitigen Impedanzen, Z h komplex, Z Sigma Eins komplex, und, R Eins, geteilt. Setzen wir unsere bekannten Größen nun ein, erhalten wir den Quotienten. Mit dem Taschenrechner lässt sich das Ergebnis von Einhundertsiebenundzwanzig Komma Zwei Acht Milliampere multipliziert mit einer negativen Phasenverschiebung von, E hoch Minus J, Neunundachzig Komma Neun Acht Neun Grad ermitteln. Wir sehen, dass sich der Primärstrom nahe an der perfekten Phasenverschiebung einer Induktion von Neunzig Grad befindet. Das negative Vorzeichen versinnbildlicht den in Kapitel Fünf gelernten Merksatz: 'Induktivität, Strom zu spät.'}
	\speech{UebungRealerTrafo}{6}{Als abschließende Aufgabe sollen nun noch die entstandenen Leerlaufverluste ermittelt werden.}
	\speech{UebungRealerTrafo}{7}{Hier schwenken wir wieder ins Kapitel Fünf, wo die Scheinleistung betrachtet wurde. Die Scheinleistung ist als das Produkt von Spannung, U komplex, und konjugiert komplexem Strom, I, definiert. Setzen wir nun die uns bekannten Größen in die Berechnung der Scheinleistungsverluste, S v komplex, ein, erhalten wir sodann das Produkt aus unserer Eingangsspannung, S v komplex, gleich Zwei Komma Fünf-Vier-Sechs Kilovoltampere mal, E hoch J, Neunundachzig Komma Neun Acht Neun Grad.}
\end{frame}
