\subsection{Maschenstromverfahren 2}
\Aufgabe{
    
    Im folgenden Netzwerk wird ein Netzwerk bestehend aus drei Widerständen und zwei Spannungsquellen dargestellt.
    Führen Sie die Analyse des Netzwerkes mit dem Maschenstromverfahren durch. 

    \textbf{Werte:}
    \[
    R_1 = 4{,}7\,\text{k}\Omega \quad R_2 = 3{,}3\,\text{k}\Omega \quad R_3 = 2{,}2 \text{k} \Omega \quad U_1 = 24\,\text{V} \quad U_2 = 12\,\text{V}
    \]

    \begin{center}
        \begin{circuitikz} 
            \draw (0,0) to [V^<, name=U1] (0,4)
            to [R, l=$R_1$, v, i, name=R1,-*] (4,4)
            to [R, l=$R_3$, v<, i<, name=R3] (8,4)
            to [V, name=U2] (8,0)
            to [short, -*] (4,0) 
            to [short] (0,0)
            (4,4) to [R, l=$R_\mathrm{2}$, v, i, name=R2] (4,0);
            
            \varrmore{U1}{$U_\mathrm{1}$};
            \varrmore{U2}{$U_\mathrm{2}$};
            \varrmore{R1}{$U_\mathrm{R1}$};
            \iarrmore{R1}{$I_\mathrm{R1}$};
            \varrmore{R2}{$U_\mathrm{R2}$};
            \iarrmore{R2}{$I_\mathrm{R2}$};
            \varrmore{R3}{$U_\mathrm{R3}$};
            \iarrmore{R3}{$I_\mathrm{R3}$};
            
            \draw[->,shift={(2,2)},voltage] (150:0.8) arc (150:-150:0.8) node at(0,0){$M_1$};
            \draw[->,shift={(6,2)},voltage] (150:0.8) arc (150:-150:0.8) node at(0,0){$M_2$};
        \end{circuitikz}
    \end{center}
        
    \begin{itemize}
        \item[\bf a)] Vorbereitung des Netzwerkes
        \item[\bf b)] Maschen und Maschenströme definieren
        \item[\bf c)] Widerstandsmatrix bestimmen
        \item[\bf d)] Quellspannungen zuordnen
        \item[\bf e)] Gleichungssystem aufstellen
    \end{itemize}
    }

\Loesung{
	\begin{itemize}
        \item[\bf a)]
        Vorbereitung des Netzwerkes:\\
        Das Netzwerk besteht aus:
        \begin{itemize}
            \item Zwei Spannungsquellen: \(U_1\) und \(U_2\),
            \item Drei Widerständen: \(R_1\), \(R_2\), \(R_3\),
            \item Drei mögliche Maschen.
        \end{itemize}
        Es sind Spannungsquellen und Widerstandswerte vorhanden.

		\item[\bf b)]
		Maschen und Maschenströme definieren:\\
        \begin{itemize}
            \item \(M_1\): Maschenstrom in der linken Masche,
            \item \(M_2\): Maschenstrom in der rechten Masche,
            \item Der Strom \(I_\mathrm{R2}\) durch \(R_2\) ergibt sich zu: \[ I_\mathrm{R2} =  \begin{bmatrix} I_{M1} \\ I_{M2} \end{bmatrix} \]
        \end{itemize}

        \item[\bf c)]
        Widerstandsmatrix bestimmen:
        \[
        \mathbf{R_M} = 
        \begin{bmatrix}
        R_1 + R_2 & -R_2 \\
        -R_2 & R_2 + R_3
        \end{bmatrix}
        \]
        Einsetzen der Werte:
        \[
        \mathbf{R_M} = 
        \begin{bmatrix}
        4{,}7\ \Omega + 3{,}3\ \Omega & -3{,}3\ \Omega \\
        -3{,}3\ \Omega & 3{,}3\ \Omega + 2{,}2\ \Omega
        \end{bmatrix}
        =
        \begin{bmatrix}
        8{,}0\ \Omega & -3{,}3\ \Omega \\
        -3{,}3\ \Omega & 5{,}5\ \Omega
        \end{bmatrix}
        \]

        \item[\bf d)]
        Quellspannungen zuordnen:\\
        \begin{eq}
            \mathbf{U_M} = \begin{bmatrix} U_1 \\ -U_2 \end{bmatrix} = \begin{bmatrix} 24\ V \\ -12\ V \end{bmatrix}  \nonumber
        \end{eq}

        \item[\bf e)]
        Gleichungssystem aufstellen:\\

        \[
        \mathbf{R_M} \cdot \mathbf{I_M} = \mathbf{U_M},
        \]
        wobei:
        \[
        \mathbf{R_M} = 
        \begin{bmatrix}
        8{,}0\ \Omega & -3{,}3\ \Omega \\
        -3{,}3\ \Omega & 5{,}5\ \Omega
        \end{bmatrix}, \quad
        \mathbf{I_M} = 
        \begin{bmatrix}
        I_{M1} \\
        I_{M2}
        \end{bmatrix}, \quad
        \mathbf{U_M} = 
        \begin{bmatrix}
        24\ V \\
        -12\ V
        \end{bmatrix}.
        \]

        Das Lösen des Gleichungssystems liefert:
        \begin{eq}
            I_{M1} = 2{,}79\,\text{mA} \quad \text{und} \quad I_{M2} = -0{,}507\,\text{mA}     \nonumber
        \end{eq}
        
        Daraus ergeben sich:
        \begin{eqa}
            I_\mathrm{R1} &= I_\mathrm{M1} = 2,79\ mA    \nonumber \\
            I_\mathrm{R2} &= I_\mathrm{M1} + (-I_\mathrm{M2}) = 2,79\ mA + 0,507\ mA = 3,297\ mA     \nonumber  \\
            I_\mathrm{R3} &= -I_\mathrm{M2} = 0,507\ mA            \nonumber
        \end{eqa}

        und

        \begin{eqa}
            U_\mathrm{R1} &= R_1 \cdot I_\mathrm{R1} = 4,7\ k\Omega \cdot 2,79\ mA = 13,12\ V    \nonumber \\
            U_\mathrm{R2} &= R_2 \cdot I_\mathrm{R1} = 3,3\ k\Omega \cdot 3,297\ mA = 10,88\ V    \nonumber  \\
            U_\mathrm{R3} &= R_3 \cdot I_\mathrm{R1} = 2,2\ k\Omega \cdot 0,507\ mA = 1,12\ V           \nonumber
        \end{eqa}

        Überprüfung der Maschen:

        \begin{eqa}
            U_\mathrm{1} &= U_\mathrm{R1} + U_\mathrm{R2} = 24\ V    \nonumber \\
            U_\mathrm{2} &= U_\mathrm{R2} + U_\mathrm{R3} = 12\ V           \nonumber
        \end{eqa}
        

	\end{itemize}
}