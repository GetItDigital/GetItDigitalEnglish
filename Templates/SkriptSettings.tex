% Pakete nur für das Skript
\usepackage{fancyhdr}      		% Unterstrichene Kopfzeilen: selbstdefinierbar
\usepackage{sanitize-umlaut}	   % Umlaute im Index
\usepackage[pdftex,bookmarks,bookmarksopen=false,colorlinks,bookmarksnumbered]{hyperref} % Links im PDF anklickbar
\hypersetup{% \ref Links im Skript umfärben
    ,urlcolor=blue
    ,citecolor=blue
    ,linkcolor=blue
    }
\usepackage{hf-tikz}		                        % Highlighting Text/Math, Ref: https://tex.stackexchange.com/questions/52598/beamer-highlighting-aligned-math-with-overlay
% example1: \tikzmarkin<2->[above offset={\hfoffuppera},below offset={\hfofflowera},blend mode=multiply]{a} R = \frac{U}{I} \tikzmarkend{a}
% example2: \tikzmarkin<2-3>[above left offset={0,\hfoffuppera}, below right offset={0,\hfofflowera}]{markerid} \int^{\pi}_{0}f(x)\d x \tikzmarkend{markerid}
% option: blend mode=multiply (to not overshadow stuff behind the highlight box) and (in beamer mode) <n>, <n->, <n-m> for overlay (ignored in article)

\usepackage{enumitem} % Aufzählungen nummerisch oder alphabetisch
\usepackage{todonotes}	%Anmerkungen und to-dos erstellen

% Symbolverzeichnis
\makenomenclature
\makeindex                      % Indizierung aktiv

% Seitengrößen
\topmargin-.7cm                 % oberen Rand verkürzen
\textheight23.3cm               % Kerntextbereich vergrößern
\textwidth15.35cm               % Bereichsverbreiterung
\oddsidemargin0.5cm             % Den linken Rand verschmälern
\evensidemargin0.5cm            % Den linken Rand verschmälern (gerade S.)
\parindent0em                   % erste Zeile eines Absatzes nicht einrücken
\parskip1.5ex                   % Absatzabstand

% Seitentitel
\fancyhead{}
\fancyhead[EL,OR]{\thepage}% E - Even, O - Odd, R - Right, L - Left, C - Center
\fancyfoot{}
\newcommand{\sectionversion}{\tiny{Stand: \today}}
\fancyfoot[EL,OR]{}

\newcommand{\setheadempty}{\fancyhead[ER,OL]{}} % Seitennummerierung im Kopf
% Deklaration des Dokumentenkopfes:
\newcommand{\sectiontitle}{}
\newcommand{\newsection}[1]{\section{#1}\renewcommand{\sectiontitle}{#1}}
\newcommand{\newchapter}[1]{\chapter{#1}\renewcommand{\GetItDigitalModulname}{#1}\renewcommand{\GetItDigitalModulnumber}{\thechapter}}

\newcommand{\setheadleftmark}{\fancyhead[ER,OL]{\leftmark}}
% Deklaration des Anhangs:
\newcommand{\setheadanhang}{
	\fancyhead[OL]{Anhang \thesection : \sectiontitle}
}
\newcommand{\setheadkapitel}{
	\fancyhead[OL]{\textsc{Kapitel \thesection : \sectiontitle}}
   \fancyhead[ER]{GET it digital Modul \GetItDigitalModulnumber : \GetItDigitalModulname}
	\fancyfoot[EL,OR]{}
}



% eigene Befehle
\renewcommand{\d}{\mathrm{d}}	%für das Differential d/dt, überschreibt \d für "Punkt unter"
\newcommand{\s}[1]{#1} % skript only
\renewcommand{\b}[1]{} % beamer only
\renewcommand{\v}[1]{} % Video only
\newcommand{\ftx}[1]{}
\newcommand{\fta}[1]{\FloatBarrier\section{#1}}
\newcommand{\ftb}[1]{\subsection{#1}}
\newcommand{\ftc}[1]{\subsubsection{#1}}

\newcommand{\f}[4]{\begin{figure}[H]\centering\includegraphics[#1]{#3}\caption{#4}\end{figure}}
\newcommand{\fu}[2]{\begin{figure}[H]\centering{#1}\caption{#2}\end{figure}}
\newcommand{\fo}[5]{\begin{figure}[H]\centering\begin{overpic}[#1]{#3}{#4}\end{overpic}\caption{#5}\end{figure}}
\newcommand{\foo}[5]{\begin{figure}[H]\begin{overpic}[#1]{#3}{#4}\end{overpic}\end{figure}}
\newcommand{\resizeboxb}[3]{#3} % resizebox but only apply in beamer (width, height, content)
\newcommand{\resizeboxsb}[3]{\resizebox{#1\textwidth}{!}{#3}} % #1: width skript, #2: width beamer, #3: content e.g. \resizeboxx{0.8}{1}{stuff}
\newenvironment{eq}{\begin{equation}}{\end{equation}}
\newenvironment{eqa}{\align}{\endalign}

%Übungsaufgaben
\newcommand{\Aufgabe}[1]{}
\renewcommand{\Loesung}[1]{}

%Beispiel
\newtcbtheorem[number within=section]{bsp}{Beispiel}{enhanced,
	before skip=1cm,
	after skip=1cm,
	colframe=GETgreen,
	colbacktitle=GETgreen,
	colback=white,
	boxrule=0.5mm,
	attach boxed title to top left={xshift=1.14cm,yshift*=-\tcboxedtitleheight/2}, varwidth boxed title*=-3cm,
	boxed title style={frame code={
					\path[left color=tcbcolback,right color=tcbcolback,
					middle color=tcbcolback]
					([xshift=1.6mm]frame.north west)[rounded corners=0.5mm]
					-- ([xshift=1.6mm]frame.north east)
					-- ([xshift=-1mm]frame.south east)
					-- ([xshift=-1mm]frame.south west)
					-- cycle;
					\path[left color=tcbcolback,right color=tcbcolback,
					middle color=tcbcolback]
					([xshift=-3mm]frame.north west)[rounded corners=0.5mm]
					-- ([xshift=0.3mm]frame.north west)
					-- ([xshift=-2.2mm]frame.south west)
					-- ([xshift=-5.5mm]frame.south west)
					-- cycle;
					\path[left color=tcbcolback,right color=tcbcolback,
					middle color=tcbcolback]
					([xshift=-7.6mm]frame.north west)[rounded corners=0.5mm]
					-- ([xshift=-4.3mm]frame.north west)
					-- ([xshift=-6.8mm]frame.south west)
					-- ([xshift=-10.1mm]frame.south west)
					-- cycle;
				},interior engine=empty,
		},
	fonttitle=\bfseries,
	title={#2},#1,breakable}{bsp}

%Highlighting
\newcommandx*\highlight[5][3=\hfofflower, 4=\hfoffupper,5=2]{\tikzmarkin[blend mode=multiply]{#1}(0,#3)(0,#4)#2\tikzmarkend{#1}}
% usage: \highlight{markerid}{markercontent}[offset below][offset upper][beamer overlay]
% Beispiele: (mit prädefinierten upper und lower offset values aus Settings.tex)
% $\highlight{id}{f(t)=\sin(\omega t)}$\quad
% $\highlight{ida}{\frac{1}{2}=\int f(t) \d t}[\hfofflowera][\hfoffuppera]$\quad
% $\highlight{idb}{\frac{\frac{\int^1_0 11}{\int^1_0 12}}{\frac{\int^1_0 21}{\int^1_0 22}}}[\hfofflowerb][\hfoffupperb][2-4]$



\pagenumbering{roman}
\setcounter{secnumdepth}{3}

% Fett-Text bei Aufzählungen
\setlist[enumerate]{font=\bfseries}
\setlist[itemize]{font=\bfseries}


% Das Ende der Seiten muss nicht immer auf gleicher Höhe sein
\raggedbottom

% Nummerierung von Formeln und Bilder
\numberwithin{figure}{section}
\numberwithin{equation}{section}
\numberwithin{table}{section}

% Symbolverzeichnis
\renewcommand\nomgroup[1]{%
	\item[\bfseries
	\ifstrequal{#1}{F}{Formelzeichen}{%
		\ifstrequal{#1}{C}{Konstanten}{%
			\ifstrequal{#1}{E}{Einheiten}{}}}%
	]}
\newcommand{\nomunit}[1]{%
	\renewcommand{\nomentryend}{\hspace*{\fill}#1}}
\renewcommand{\nomname}{Verwendete Größen und Formelzeichen}

% Bibliographie
\usepackage[
	backend=biber,
	sorting=none,% Sortierung nach Reihenfolge der Zitate
]{biblatex} %biblatex mit biber laden
\addbibresource{../Templates/Literatur.bib}
\usepackage[nottoc,notlot,notlof]{tocbibind}	%% Literaturverzeichnis im TOC
