\subsection{Zeichnung und Berechnung einer Kollektorschaltung\label{Aufg36}}
% Alt: Praktikumsaufgabe 9
\Aufgabe{
    \begin{itemize}
        \item[a)] Zeichnen Sie die folgende Schaltung:  
        Beschalten Sie einen npn-Transistor mit einem Basisvorwiderstand von  
        $$ R_\mathrm{B} = 4,7\,\mathrm{k\Omega}. $$  
        Der Emitterwiderstand beträgt 
        $$ R_\mathrm{E} = 1\,\mathrm{k\Omega}. $$
        Die Spannungsversorgung beträgt 
        $$ V_\mathrm{0} = 15\,\mathrm{V}. $$
        Schalten Sie eine $10 \,\mathrm{V}$-Z-Diode in Sperrrichtung von der Basis auf Masse.  
        
        Verwenden Sie nur eine Spannungsversorgung, welche sowohl die Basis als auch den Kollektor versorgt.
        \item[b)] Berechnen Sie die Emitterspannung $V_\mathrm{E}$, und den Emitterstrom $I_\mathrm{E}$.
        \item[c)] Berechnen Sie die Leistung am Emitterwiderstand $P_\mathrm{RE}$.
    \end{itemize}
}


\Loesung{
\begin{itemize}
    \item[a)] Schaltung:
    \begin{figure}[H]
        \centering
        \begin{tikzpicture}
    \draw
    (0,0) node[npn] (Q1) {}
    (Q1.base) 
    (Q1.collector)  
    (Q1.emitter);

\draw (Q1.C) -- (0,2);
\draw (Q1.E) to[R,l=$R_\mathrm{E}$] (0,-2.5) -- (-2,-2.5) to[zDo,*-*] (-2,0) to[R,l=$R_\mathrm{B}$] (-2,2);
\draw (Q1.B) -- (-2,0);

\draw (-2,-2.5) -- (-4,-2.5);
\draw (-4,2) to[V,name=V0] (-4,-2.5);
\draw (-4,2)to[short,-*](-2,2)--(0,2);

\varrmore{V0}{$V_\mathrm{0}$};
\end{tikzpicture}
   
        \label{fig:LsgPraktikumZeichnung}
        \end{figure}


    \item[b)] Berechnung der Emitterspannung $V_\mathrm{E}$ \\
    Die Zenerdiode fixiert die Basisspannung des Transistors auf $V_\mathrm{Z} = 10\,\mathrm{V}$. 
    Die Basis-Emitter-Spannung beträgt $V_{BE} = 0,7\,\mathrm{V}$.
    $$ V_\mathrm{E} = V_\mathrm{Z} - V_\mathrm{BE} = 10\,\mathrm{V} - 0,7\,\mathrm{V} = 9,3\,\mathrm{V} $$
    $$ I_\mathrm{E} = \frac{V_\mathrm{E}}{R_\mathrm{E}} = \frac{9,3\,\mathrm{V}}{1\,\mathrm{k\Omega}} = 9,3\,\mathrm{mA} $$

\item[c)] Berechnung der Leistung am Emitterwiderstand $P_{RE}$ \\
    $$ P_\mathrm{RE} = I_\mathrm{E}^2 \cdot R_\mathrm{E} = (9,3\,\mathrm{mA})^2 \cdot 1\,\mathrm{k\Omega} = 86,49\,\mathrm{mW} $$

\end{itemize}
Siehe: Abschnitt \ref{fig:GrundschaltungenBipolartransistor}
}