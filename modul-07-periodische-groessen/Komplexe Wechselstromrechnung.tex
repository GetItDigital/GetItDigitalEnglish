%Komplexe Wechselstromrechnung

\begin{frame}
    \fta{Komplexe Wechselstromrechnung}

    \s{
        In Gleichstromnetzwerken wird mit konstanten Spannungsquellen und Stromquellen gearbeitet. Im Gegensatz dazu kommen bei der 
        komplexen Wechselstromrechnung Quellen mit sinusförmigen Wechselgrößen zum Einsatz. Die elektrische Analyse von Komponenten 
        in einem Wechselspannungsnetzwerk erfolgt komplex. 
        Die komplexen Spannungen \underline{U} und Ströme \underline{I} erzeugen zeitabhängige Spannungswerte u(t) und Stromwerte 
        i(t). Die Gleichung \ref{GleichungURIKomplex} beschreibt den angegebenen Zusammenhang zwischen komplexen und zeitabhängigen
        Strom- und Spannungswerten für einen ohmschen Widerstand.
    }

    \b{
        Komplexe und zeitabhängige Strom- und Spannungsangaben der Wechselstromrechnung:
    }

    \begin{eq}
        \underline{U} = \underline{Z}_\mathrm{R} \cdot \underline{I} \quad \rightarrow \quad u(t) = R \cdot i(t)  \label{GleichungURIKomplex}
    \end{eq}

    \s{
        Abgesehen von der komplexität der Wechselstromrechnungen, gelten weiterhin die Regeln zur Berechnung elektrischer Netzwerke,
        wie Maschen- und Knotenregeln, das Verhalten von Reihen- und Paralellschaltungen und die Knotenpotentialanalyse sowie die 
        Maschenstromanalyse. Bei all diesen Beispielen ist darauf zu achten, mit den komplexen Wechselgrößen zu rechnen. Für die 
        Erläuterung von elektrische Netzwerken mit komplexen Wechselgrößen werden im Folgenden diese Tehmengebiete näher betrachtet:
    }

    \b{
        Für die Berechnung von elektrische Netzwerken mit Wechselgrößen werden im Folgenden diese Tehmengebiete näher betrachtet:
    }

    
    \begin{Lernziele}{Wechselstromrechnung}
        Die Studierenden
        \begin{itemize}
            \item kennen die komplexe Impedanz und die komplexe Attmitanz.
            \item verstehen das Verhalten eines Widerstandes, eines Kondensators und einer Spule an einer Wechselspannung.
            \item kennen Wechselquellen und die dazugehörigen Umwandlungsvorschriften. 
        \end{itemize}
    \end{Lernziele}

\end{frame}

\begin{frame}
    \ftb{Impedanz und Attmitanz in der komplexen Ebene}

    \s{
        Durch eine Phasenverschiebung zwischen der Spannung und dem Strom kommt es zu einer Erzeugung eines Blindanteils der Impedanz. Der Blindanteil lässt sich jedoch 
        auf der reellen Achse nicht darstellen, weshalb die reelle Achse um eine imaginäre Achse erweitert wird, welche gemeinsam die 
        komplexe Ebene bilden. Die Abbildung \ref{BildKomplexeImpedanz} zeigt den Zeiger der komplexen Impedanz in der komplexen Ebene 
        mit Imaginäranteil und Realanteil. 

        \fu{
            \begin{tikzpicture}
                \draw (0,0) coordinate (U);
                \draw[very thin,color=gray] (-0.1,-0.1) grid (3.2,3.1);
                \draw[->] (-0.2,0) -- (3.4,0) node[right] {$\Re(\underline{Z})$};
                \draw[->] (0,-0.1) -- (0,3.2) node[above] {$\Im(\underline{Z})$};
                \draw (3,0) coordinate (R);
                \draw (3,3) coordinate (Z);
                \draw[->] (0,0) -- (2.5,2.5);
                \draw(1.25,1.75) node {$\hat{Z}$};
                \draw pic[draw, angle radius = 1cm, ->] {angle = R--U--Z};
                \draw(1.2,0.5) node {$\varphi_Z$};
                \draw(3,2.5) node [right] {$\underline{Z}=\hat{Z} \cdot e^{j \varphi_\mathrm{Z}}$};
                \pause
    
                \draw[dashed](2.5,2.5) -- (2.5,-0.2) node[below] {$R$};
                \pause
    
                \draw[dashed](2.5,2.5) -- (-0.2,2.5) node[left] {$j \cdot X$};
            \end{tikzpicture}
        }{{\bf Zeiger der komplexen Impedanz.} Die Impedanz in der komplexen Ebene mit der Aufteilung in Realanteil R und Imaginäranteil X. \label{BildKomplexeImpedanz}}
         
    }

    \b{
        \begin{minipage}[t]{0.4\textwidth}
            \begin{itemize}
                \item<1-> Komplexer Zeiger der Impedanz
                \item<2-> Wirkanteil der Impedanz
                \item<3-> Blindanteil der Impedanz
                \item<4-> Wirkwiderstand und Blindwiderstand der Impedanz
            \end{itemize}
        \end{minipage}
        \begin{minipage}[t]{0.4\textwidth}
            \fu{
                \begin{tikzpicture}
                    \draw (0,0) coordinate (U);
                    \draw[very thin,color=gray] (-0.1,-0.1) grid (3.2,3.1);
                    \draw[->] (-0.2,0) -- (3.4,0) node[right] {$\Re(\underline{Z})$};
                    \draw[->] (0,-0.1) -- (0,3.2) node[above] {$\Im(\underline{Z})$};
                    \draw (3,0) coordinate (R);
                    \draw (3,3) coordinate (Z);
                    \draw[->] (0,0) -- (2.5,2.5);
                    \draw(1.25,1.75) node {$\hat{Z}$};
                    \draw pic[draw, angle radius = 1cm, ->] {angle = R--U--Z};
                    \draw(1.2,0.5) node {$\varphi_Z$};
                    \draw(3,2.5) node [right] {$\underline{Z}=\hat{Z} \cdot e^{j \varphi_\mathrm{Z}}$};
                    \pause
        
                    \draw[dashed](2.5,2.5) -- (2.5,-0.2) node[below] {$R$};
                    \pause
        
                    \draw[dashed](2.5,2.5) -- (-0.2,2.5) node[left] {$j \cdot X$};
                \end{tikzpicture}
            }{}
        \end{minipage}
    }

    \s{
        So besteht auch die komplexe Impedanz \underline{Z} nach Gleichung \ref{GleichungImpedanz} aus einem Realteil und aus einem Imaginärteil. Der
        Realteil besteht aus dem ohmschen Anteil, welcher den Wirkwiderstand einer Impedanz darstellt. Der Wirkwiderstand wird auch als 
        Resistanz R bezeichnet. Zweiter Bestandteil der Impedanz ist der Imaginärteil, welcher auch als Reaktanz X bezeichnet wird.
        Die Reaktanz eines elektrischen Netzwerkes setzt sich beisplielsweise aus den kapazitiven und induktiven Auswirkungen zusammen
        und wird mit der imaginären Einheit j versehen. Die Einheit der komplexen Impedanz ist wie in der Gleichstromtechnik Ohm $\Omega$. 
    }

    \only<4>{
    \begin{eq}
        Impedanz = Resistanz + j \cdot Reaktanz \quad \rightarrow \quad \underline{Z} = R + j \cdot X   \label{GleichungImpedanz}
    \end{eq}
    \begin{eq}
        [\underline{Z}] = 1\ Ohm = 1\ \Omega \nonumber
    \end{eq}
    }
    
\end{frame}




\begin{frame}
    \ftx{Impedanz und Attmitanz in der komplexen Ebene}

    \s{
        Wie in der Gleichstromtechnik existiert ein Leitwert des elektrischen Widerstandes in der komplexen Wechselstromtechnik.
        Dieser Kehrwert der komplexen Impedanz wird als Admittanz \underline{Y} bezeichnet (vlg. Gleichung \ref{GleichungImpedanzAttmitanz}). Die Einheit des komplexen Leitwertes bleibt 
        wie in der Gleichstromtechnik Siemens S. 
    }

    \b{
        Komplexer Leitwert als Kehrwert der komplexen Impedanz:
    }

    \begin{eq}
        R = \frac{1}{G} \quad \rightarrow \quad \underline{Z} = \frac{1}{\underline{Y}}   \label{GleichungImpedanzAttmitanz}
    \end{eq}

    \s{
        Auch die Admittanz lässt sich in einen Wirkleitwert und einen Blindleitwert differenzieren. 
        Der Wirkleitwert G der Admittanz wird als Konduktanz und der Blindleitwert B der Admittanz wird als Suszeptanz bezeichnet (vgl. Gleichung \ref{GleichungAdmittanz} ). 
    }

    \b{
        Wirkleitwert und Blindleitwert der Admittanz:
    }
    
    \begin{eq}
        Admittanz = Konduktanz + j \cdot Suszeptanz \quad \rightarrow \quad \underline{Y} = G + j \cdot B           \label{GleichungAdmittanz}
    \end{eq}

    \begin{eq}
        [\underline{Y}] = 1\ Siemens = 1\ S \nonumber
    \end{eq}

\end{frame}


\begin{frame}
    \ftx{Komplexe Impedanz und komplexe Admittanz}

    \begin{Merksatz}{Komplexe Impedanz und komplexe Admittanz}
        Der Gleichstromwiderstand wird bei einer {\bf Wechselspannung} zur {\bf komplexen Impedanz} \underline{Z}. Der 
        Kehrwert des Widerstandes, der Leitwert, wird bei einer Wechselspannung zur {\bf komplexen Admittanz} \underline{Y}.
    \end{Merksatz}
\end{frame}


\begin{frame}
    \ftb{Komplexer Widerstand}

    \s{
        Der komplexe Widerstand eines ohmschen Verbrauchers besteht aus dem Wirkannteil R und dem Blindanteil des des ohmschen 
        Widerstandes.
        Der ohmsche Widerstand R verkörpert die Eigenschaft leitfähiger Materialien elektrische Leistung in thermische Leistung 
        umzuwandeln. Der Blindanteil des Widerstands wandelt die Leistung der Wechselströme nicht in Wärme um. 
        Der Blindanteil des elektrischen Widerstands wird als $X_\mathrm{R}$ beziechnet. Da der ideale elektrische Widerstand allerdings rein
        ohmsch und damit rein Real wirkt, wird in der Regel oft einfach das Formelzeichen R verwendet. Die Gleichung \ref{GleichungWid1}
        erklärt diese Zusammenhänge. 
    }

    \b{
        \begin{itemize}
            \item Komplexer Widerstand an einem ohmschen Verbraucher
            \item Kein Blindanteil bei einem idealen ohmschen Widerstand 
            \item Idealer komplexer Widerstand als reiner Wirkanteil 
        \end{itemize}
    }

    \begin{eq}
       \underline{Z}_\mathrm{R} = R + j \cdot X_\mathrm{R} \quad \rightarrow \quad X_\mathrm{R-ideal} = j \cdot 0\ \Omega \quad \rightarrow \quad \underline{Z}_\mathrm{R} = R       \label{GleichungWid1}
    \end{eq}

    \s{
        An den zeitabhängigen Werten für die Spannung und den Strom ergeben sich die Zusammenhänge des idealen ohmschen Widerstandes 
        nach Gleichung \ref{GleichungWid2}. 
    }

    \begin{eq}
        \underline{U} = \underline{Z}_\mathrm{R} \cdot \underline{I} \quad \rightarrow \quad u(t) = R \cdot i(t)        \label{GleichungWid2}
    \end{eq}

\end{frame}

\begin{frame}
    \ftx{Komplexer Widerstand}

    \s{
        Ein idealer elektrischer Widerstand ist ein reiner Wirkwiderstand und weist keinerlei Blindwiderstandanteile auf. 
        Der reale elektrische Widerstand ist mit parasitären Effekten behaftet. Die Schaltbilder eines idealen und eines 
        realen elektrischen Widerstandes werden in der Abbildung \ref{BildRealerWiderstand} gegenübergestellt. Diese parasitären 
        Effekte wirken sich induktiv, verdeutlicht durch die Spule, und kapazitiv, verdeutlicht durch die Kapazität, auf die 
        Schaltung aus. Neben der Frequenzabhängikeit der parasitären Effekte, werden auch die Spannung und der Strom so aufgeteilt,
        dass nicht mehr die gesamte Leistung am Widerstand abfällt. die Impedanz des idealen elektrischenWiderstands ist nicht 
        frequenzabhängig.
    }
 
    \b{
        Elektrischer Widerstand in der Wechselstromrechnung
        \begin{itemize}
            \item<1-> Idealer Widerstand
            \item<2-> Realer Widerstand
            \item<3-> Keine Phasenverschiebung beim idealen Widerstand
        \end{itemize}
    }

    \fu{
            \begin{circuitikz}
                \draw (0,2) to [short, o-] (2,2) 
                to [R=$R_\mathrm{ideal}$] (2,0) 
                to [short,-o] (0,0);
                \pause

                \draw [->](4,1) -- (5,1); 
                \draw (6,2) to [L=$L_\mathrm{paras.}$, o-*] (8,2) 
                to [R=$R_\mathrm{real}$] (8,0) 
                to [short,*-o] (6,0);
                \draw (8,2) to [short] (10,2)
                to [C=$C_\mathrm{paras.}$] (10,0)
                to [short] (8,0);

                %[R=$R_2$, i<, name=R2] (6,3);
                %\draw(0,0) to[V, V<, name=Uq1] (0,3);	
                %\varrmore{Uq1}{$U_{q1}$};
                %\iarrmore{R2}{$I_2$};}
            \end{circuitikz}
    }{{\bf Gegenüberstellung eines idealen und eines realen elektrischen Widerstandes.} Der reale Widerstand weist im 
    Gegensatz zum idealen Widerstand seriell induktive und parallel kapazitive Effekte auf. \label{BildRealerWiderstand}}

\end{frame}


\begin{frame}
    \ftx{Komplexer Widerstand}
    
    \s{
        Zwischen der Spannung und dem Strom existiert beim idealen elektrischen Widerstand keine Phasenverschiebung. In der 
        Abbildung \ref{BildPhasenWiderstand} werden die Sinuswellen für die Spannung und den Strom an einem idealen Widerstand 
        aufgetragen. Sie liegen ohne Phasenverschiebung übereinander, somit befinden sie sich in Phase. 
    }

    \b{
        Phasenverschiebung am idealen ohmschen Widerstand:
        \begin{itemize}
            \item<1-> Spannung $u_\mathrm{R}(t)$
            \item<2-> Strom $i_\mathrm{R}(t)$
            \item<3-> Spannung und Strom am idealen ohmschen Wiederstand ohne Phasenverschiebung
        \end{itemize}
    }

    \fu{
        \begin{tikzpicture}[scale=1.5]
            % Achsen
            \draw[->] (-0.5, 0) -- (4.2, 0) node[right] {Zeit $t$};
            \draw[->] (0, -1) -- (0, 1.2) node[right] {Amplitude};
        
            % Spannung (u)
            \draw[color=blue, very thick, smooth, domain=0:4] plot[id=Sinus31] function{sin(pi * x)} node[right] {};
            \node[blue] at (-0.5, 0.8) {$u_\mathrm{R}(t)$};
            \pause
            % Strom (i) - in Phase mit Spannung
            \draw[color=red, very thick, smooth, dashed, domain=0:4] plot[id=Sinus32] function{0.8 * sin(pi * x)} node[right] {};
            \node[red] at (-0.5, 0.5) {$i_\mathrm{R}(t)$};
            \pause
            \node at (0.5, -1.2) {$\Delta\varphi_\mathrm{R} = \varphi_\mathrm{U} - \varphi_\mathrm{I} = 0$};

        \end{tikzpicture}
    }{{\bf Sinuschwingungen an einem Widerstand.} Wechselspannung (blau) und Wechselstrom (rot) an einem idealen Widerstand. Es resultiert an dem 
    Widerstand keine Phasenverschiebung zwischen der Wechselspannung und dem Wechselstrom. \label{BildPhasenWiderstand}}

\end{frame}


\begin{frame}
    \ftx{Sinusschwingung an einem Widerstand}

    \begin{Merksatz}{Sinusschwingung an einem Widerstand}
        Der {\bf elektrische Widerstand} weist zwischen der Wechselspannung und dem zugehörigen Wechselstrom {\bf keine 
        Phasenverschiebung} auf.     
    \end{Merksatz}
\end{frame}


\begin{frame}
    \ftb{Kapazität}

    \s{
        Der Kondesnator als Bauteil in der Elektrotechnik weist die Fähigkeit auf Energie zwischen den zwei Platten in 
        dem elektrischen Feld zu speichern. Die Größe eines Kondensators wird in der Kapazität C angegeben. Die Einheit 
        der Kapazität ist das Farad F. Die komplexe Impedanz eines Kondensators lässt sich nach Gleichung \ref{GleichungKap1}
        berechnen. Im Gegensatz zum idealen Widerstand wirkt der ideale Kondensator im Wechselstromkreis nicht lediglich als
        Wirkwiderstand. Der reale Kondensator wirkt sowohl durch einen Wirkwanteil, als auch durch einen Blindanteil. Die 
        Kreisfrequenz $\omega$ ist frequenzabhängig. So ergeben sich bei hohen Frequenzen niedrige Impedanzwerte für den 
        Kondensator. Andersherum ergeben sich hohe Impedanzwerte bei niedrigen Frequenzen. Der Blindanteil des Kondensators
        wirkt sich als kapazitive Reaktanz aus. 
        Es kann beim Kondensator auch mit der komplexen Admittanz gerechnet werden. Die komplexe Admittanz ergibt sich aus
        dem kehrwert der komplexen Impedanz des Kondensators. 
    }

    \b{
        Negativer Blindwiderstand $\leftrightarrow$ Darstellung im negativen Imaginärbereich
    }

    \begin{eq}
        \underline{Z}_\mathrm{C} = \frac{1}{j \omega C} \quad \rightarrow \quad \underline{Y}_\mathrm{C} = j \omega C          \label{GleichungKap1}
    \end{eq}

    \s{
        Der ideale Kondensator wirkt rein im Blindanteil, so wird nach Gleichung \ref{GleichungKap2} aus der komplexen Notation $Z_C$ der 
        Blindanteil des Kondensators $X_C$ mit zeitabhängiger Spannung und Strom. 
    }

    \b{
        Idealer Kondensator an Wechselspannung und Wechselstrom:  
    }
    
    \begin{eq}
        \underline{U} = \underline{Z}_\mathrm{C} \cdot \underline{I} \qquad \qquad u(t) = \underline{X}_\mathrm{C} \cdot i(t)     \label{GleichungKap2}
    \end{eq}

    \s{
        Es Wird davon ausgegangen, dass die Wechselspannung an einem Kondensator mit der Phasenverschiebung von 0° anliegt. 
        Mit dieser Feststellung kann die verursachte Phasenverschiebung des Stromes am Kondensator nach Gleichung \ref{GleichungKap3} festgelegt
        werden. Durch die Phasenverschiebung des Stromes gleicht der Verlauf einer Cosinusfunktion im Gegensatz zur Sinusfunktion
        der Spannung. Sollen sowohl die Spannung als auch der Strom durch die Sinusfunktion beschrieben werden, wird die 
        Phasenverschiebung $\Delta\varphi$ von $\pi/2$ ins Argument mit aufgenommen. 
    }

    \b{
        Pasenverschiebung des Stromes am idealen Kondensator: 
    }

    \begin{eq}
        u(t) = \hat{U} \cdot \sin(\omega t)      \qquad      i(t) = \hat{I} \cdot \cos(\omega t) = \hat{I} \cdot \sin(\omega t + \pi/2)        \label{GleichungKap3}
    \end{eq}

\end{frame}

\begin{frame}
    \ftx{Kapazität}

    \s{
        Bei Induktivitäten und Kapazitäten kommt es zu einer Phasenverschiebung zwischen der Wechselspannung und dem
        Wechselstrom. Beim Kondensator bedeutet dies, dass der Strom der Spannung um 90° (Gradmaß) voreilt. Der beschriebene 
        Sachverhalt der Phasenverschiebung wird in der Abbildung \ref{BildPhasenKondensator} verdeutlicht. Der in rot 
        dargestellte Strom an einem Kondensator erreicht den Maximalwert um $2\pi$ (Bogenmaß) früher als der blau gefärbte 
        Maximalwert der Spannung.  
    }

    \b{
        Phasenverschiebung am Kondensator:
        \begin{itemize}
            \item Spannung $u_\mathrm{C}(t)$
            \item<2-> Strom $i_\mathrm{C}(t)$
            \item<3-> Phasenverschiebung zwischen Strom und Spannung
            \item<4-> Phasenverschiebung $\Delta\varphi$
        \end{itemize}
    }

    \fu{
        \begin{tikzpicture}[scale=1.5]
            % Achsen
            \draw[->] (-0.5, 0) -- (4.2, 0) node[right] {Zeit $t$};
            \draw[->] (0, -1) -- (0, 1.2) node[right] {Amplitude};
            \draw (0.5,0.1) -- (0.5,-0.1);
            \draw (1,0.1) -- (1,-0.1) node[below] {$\pi$};
            \draw (1.5,0.1) -- (1.5,-0.1);
            \draw (2,0.1) -- (2,-0.1) node[below] {$2\pi$};
        
            % Spannung (u)
            \draw[color=blue, very thick, smooth, domain=0:4] plot[id=Sinus33] function{sin(pi * x)} node[right] {};
            \node[blue] at (-0.5, 0.8) {$u_\mathrm{C}(t)$};
            \node[blue] at (5, -0.5) {$\underline{U}=\hat{U} \cdot e^{j \varphi_\mathrm{U}}$};
            \pause
            % Strom (i) - in Phase mit Spannung
            \draw[color=red, very thick, smooth, dashed, domain=0:4] plot[id=Sinus34] function{0.8 * sin((pi * x)+(pi/2))} node[right] {};
            \node[red] at (-0.5, 0.5) {$i_\mathrm{C}(t)$};
            \node[red] at (5, 0.5) {$\underline{I}=\hat{I} \cdot e^{j \varphi_\mathrm{I}}$};
            \pause
            \draw[<->] (0.5, -0.2) -- (0, -0.2) node[below right] {$\pi/2$};
            \pause
            \node at (0.5, -1.2) {$\Delta\varphi_\mathrm{C} = \varphi_\mathrm{U} - \varphi_\mathrm{I} = -\pi/2$};
        \end{tikzpicture}
    }{{\bf Phasenverschiebung zwischen Strom und Spannung an einem Kondensator.} Der Strom eilt der Spannung am Kondensator um 90° vor. 
    \label{BildPhasenKondensator}}

\end{frame}


\begin{frame}
    \ftx{Sinusschwingung an einem Kondensator}

    \begin{Merksatz}{Sinusschwingung an einem Kondensator}
        An der idealen Kapazität eilt der Wechselstrom der Wechselspannung um 90° vor. 
    \end{Merksatz}
\end{frame}


\begin{frame}
    \ftx{Idealer und realer Kondensator}

    \s{
        In der Regel wird mit idealen Bauteilen und deren Eigenschaften gerechnet. Jedoch weist der reale Kondensator, wie auch 
        der zuvor beschriebene reale Widerstand, induktive und ohmsche parasitäre Effekte auf. Der ideale Kondensator wird in der 
        Abbildung \ref{BildRealerKondensator} dem realen Kondensator gegenübergestellt. Die Zu- und Ableitungen weisen induktive 
        ($L_\mathrm{paras.}$) und ohmsche ($R_\mathrm{paras.2}$) Anteile auf. Außerdem ist der Kondensator einer ständigen 
        Selbstentladung ausgesetzt. Diese Selbstentladung erfolgt symbolisch über den Paralellwiderstand $R_\mathrm{paras.1}$.
    }

    \b{
        Idealer und realer Kondensator:
        \begin{itemize}
            \item Idealer Kondensator
            \item<2-> Realer Kondensator mit parasitären Effekten
        \end{itemize}
    }

    \fu{
        \begin{circuitikz}
            \draw (0,2) to [short, o-] (2,2) 
                to [C=$C_\mathrm{ideal}$] (2,0) 
                to [short,-o] (0,0);
                \pause

                \draw [->](4,1) -- (5,1); 
                \draw (6,2) to [L=$L_\mathrm{paras.}$, o-*] (8,2) 
                to [C=$C_\mathrm{real}$] (8,0) 
                to [R=$R_\mathrm{paras.2}$,*-o] (6,0);
                \draw (8,2) to [short] (10,2)
                to [R=$R_\mathrm{paras.1}$] (10,0)
                to [short] (8,0);
            \end{circuitikz}
    }{{\bf Gegenüberstellung eines idealen und eines realen Kondensators.} Der reale Kondensator weist im 
    Gegensatz zum idealen Widerstand seriell induktive und parallel kapazitive Effekte auf. \label{BildRealerKondensator}}

\end{frame}


\begin{frame}
    \ftb{Induktivität}

    \s{
        Die Spule stellt ebenso einen Energiespeicher dar, wie der Kondensator. Nur wird bei der Spule die Energie nicht in einem
        elektrischen Feld gespiechert, sondern in einem magnetischen Feld nach den Gegebenheiten der Induktionsgesetze. Die Größe
        einer Spule wird in der Induktivität L angegeben. Die Einheit der Induktivität ist Henry H. Die komplexe Impedanz einer Spule
        ergibt sich aus der Imaginären Einheit j, der Kreisfrequenz $\omega$ und der Induktivität L (vgl. Gleichung \ref{GleichungInd1}).
        die komplexe Admittanz wird durch den Kehrwert der Spulenimpedanz berechnet. Das führt zu Zusammenhängen, welche sich entgegen
        der Impedanzberechnungen bei Kapazitäten entwickeln. Bei hohen frequenzen werden die Impedanzen bei Spulen entsprechend ihrer 
        Induktivität ebenfalls groß und bei niedrigen Frequenzen entsprechend gering. 
    }

    \b{
        Positiver Blindwiderstand $\leftrightarrow$ Darstellung im positiven Imaginärbereich
    }

    \begin{eq}
        \underline{Z}_\mathrm{L} = j \omega L \quad \rightarrow \quad \underline{Y}_\mathrm{L} = \frac{1}{j \omega L}     \label{GleichungInd1}
    \end{eq}

    \s{
        Ebenso wie beim Kondensator wirkt die ideale Spule rein als Blindanteil der Impedanz. In der Gleichung \ref{GleichungInd2}
        wird auf der linken Seite die komplexe Notation der Spulenimpedanz $Z_L$ verwendet. Auf der rechten Seite wird die Zeitfunktion
        lediglich mit der Schreibweise des Blindanteils der Spule $X_L$ dargestellt.     
    }

    \b{
        Ideale Spule an Wechselspannung und Wechselstrom:  
    }
    
    \begin{eq}
        \underline{U} = \underline{Z}_\mathrm{L} \cdot \underline{I} \qquad  \qquad u(t) = \underline{X}_\mathrm{L} \cdot i(t)        \label{GleichungInd2}
    \end{eq}

    \s{
        Prinzipiell verhalten sich die Zeitfunktionen von Induktivitäten für die Spannung und den Strom vergleichbar, mit denen Zeitfunktionen
        von Kapazitäten. Lediglich die Phasenverschiebung des Stromes an Spulen ist der Phasenverschiebung der Kapazitäten entgegengesetzt. 
        Die Phasenverschiebung $\Delta\varphi$ zwischen der Spannung und dem Strom beträgt an einer Induktivität $\pi/2$. 
    }

    \b{
        Pasenverschiebung des Stromes an der idealen Spule: 
    }

    \begin{eq}
        u(t) = \hat{U} \cdot \cos(\omega t) = \hat{U} \cdot \sin(\omega t + \pi/2)    \qquad     i(t) = \hat{I} \cdot \sin(\omega t)      \label{GleichungInd3}
    \end{eq}

\end{frame}

\begin{frame}
    \ftx{Induktivität}

    \s{
        Äquivalent zu der Phasenverschiebung an einem Kondensator existiert an der Induktivität ebenfalls eine Phasenverschiebung.
        Jedoch ist diese Phasenverschiebung des Stromes an der Induktivität der Phasenverschiebung der Kapazität entgegengerichtet.
        Im konkreten Fall in der Abbildung \ref{BildPhasenSpule} wird gezeigt, dass der der Strom (rot) an einer Induktivität der 
        Spannung (blau) nacheilt. 
    }

    \b{
        Phasenverschiebung an der Spule:
        \begin{itemize}
            \item Spannung $u_\mathrm{L}(t)$
            \item<2-> Strom $i_\mathrm{L}(t)$
            \item<3-> Phasenverschiebung zwischen Strom und Spannung
            \item<4-> Phasenverschiebung $\Delta\varphi$
        \end{itemize}
    }

    \fu{
        \begin{tikzpicture}[scale=1.5]
            % Achsen
            \draw[->] (-0.5, 0) -- (4.2, 0) node[right] {Zeit $t$};
            \draw[->] (0, -1) -- (0, 1.2) node[right] {Amplitude};
            \draw (0.5,0.1) -- (0.5,-0.1);
            \draw (1,0.1) -- (1,-0.1) node[below] {$\pi$};
            \draw (1.5,0.1) -- (1.5,-0.1);
            \draw (2,0.1) -- (2,-0.1) node[below] {$2\pi$};
        
            % Spannung (u)
            \draw[color=blue, very thick, smooth, domain=0:4] plot[id=Sinus35] function{sin(pi * x)} node[right] {};
            \node[blue] at (-0.5, 0.8) {$u_\mathrm{L}(t)$};
            \node[blue] at (5, -0.5) {$\underline{U}=\hat{U} \cdot e^{j \varphi_\mathrm{U}}$};
            \pause
            % Strom (i) - in Phase mit Spannung
            \draw[color=red, very thick, smooth, dashed, domain=0:4] plot[id=Sinus36] function{0.8 * sin((pi * x)-(pi/2))} node[right] {};
            \node[red] at (-0.5, 0.5) {$i_\mathrm{L}(t)$};
            \node[red] at (5, 0.5) {$\underline{I}=\hat{I} \cdot e^{j \varphi_\mathrm{I}}$};
            \pause
            \draw[<->] (0.5, -0.2) -- (0, -0.2) node[below right] {$\pi/2$};
            \pause
            \node at (0.5, -1.2) {$\Delta\varphi_\mathrm{L} = \varphi_\mathrm{U} - \varphi_\mathrm{I} = +\pi/2$};
        \end{tikzpicture}
    }{{\bf Phasenverschiebung zwischen Strom und Spannung an einer Spule.} Der Strom folgt der Spannung an der Spule um 90° nach. 
    \label{BildPhasenSpule}}

\end{frame}



\begin{frame}
    \ftx{Sinusschwingung an einer Spule}

    \begin{Merksatz}{Sinusschwingung an einer Spule}
        Bei einer idealen Induktivität folgt der Wechselstrom der Wechselspannung um 90° nach.     
    \end{Merksatz}
\end{frame}


\begin{frame}
    \ftx{Ideale und reale Spule}

    \s{
        Wie auch der ohmsche Widerstand und der Kondensator weist die reale Spule parasitäre Effekte auf. Über den Spulendraht 
        wirkt ein ohmscher Anteil ($R_\mathrm{paras.}$) auf die komplexe Impedanz der Spule ein. Zusätzlich erzeugt die Wicklung 
        der Spule kapazitive Effekte ($C_\mathrm{paras.}$). In der Abbildung \ref{BildRealeSpule} wird das Ersatzschaltbild einer 
        idealen Spule und einer realen Spule dargestellt. 
    }

    \b{
        Ideale und reale Spule:
        \begin{itemize}
            \item Ideale Spule
            \item<2-> Reale Spule mit parasitären Effekten
        \end{itemize}
    }

    \fu{
        \begin{circuitikz}
            \draw (0,2) to [short, o-] (2,2) 
                to [L=$L_\mathrm{ideal}$] (2,0) 
                to [short,-o] (0,0);
                \pause

                \draw [->](4,1) -- (5,1); 
                \draw (6,2) to [R=$R_\mathrm{paras.}$, o-*] (8,2) 
                to [L=$L_\mathrm{real}$] (8,0) 
                to [short,*-o] (6,0);
                \draw (8,2) to [short] (10,2)
                to [C=$C_\mathrm{paras.}$] (10,0)
                to [short] (8,0);
            \end{circuitikz}
    }{{\bf Gegenüberstellung einer idealen und einer realen Spule.} Die reale Spule weist serielle ohmsche und parallele 
    kapazitive parasitäre Effekte auf. \label{BildRealeSpule}}

\end{frame}


\begin{frame}
    \ftb{Quellen von Wechselgrößen}

    \s{
        Wie bei Quellen von Gleichströmen und Gleichspannungen existieren Wechselquellen mit vergleichbaren Merkmalen. Bei der
        idealen Wechselspannungsquelle ist die Spannung eingeprägt. Die Wechselspannung ändert sich nicht mit der Abhängigkeit 
        des Stromes, welcher durch sie hindurchfließt. Die ideale Wechselstromquelle stellt einen eingesprägten Strom zur 
        Verfügung. Dieser wird nicht durch die angeschlossene Last beeinflusst. Die Schaltungssymbole einer idealen 
        Wechselspannungsquelle und einer idealen Wechselstromquelle werden in der Abbildung \ref{BildIdealeQuellen} dargestellt. 
    }

    \b{
        Quellen von Wechselgrößen:
        \begin{itemize}
            \item Ideale Wechselspannungsquelle
            \item<2-> Ideale Wechselstromquelle
        \end{itemize}
    }

    \fu{
        \begin{circuitikz}
            \draw (0,0) to[vsourcesin] (0,2)
                to[short,-o] (2,2)
                (0,0) to[short,-o] (2,0);
            \draw[-latex, thick, draw=voltage] (2,1.8) to node[right, color=voltage] {$\underline{U}$} (2,0.2);
            \pause

            %\draw [->](4,1) -- (5,1); 

            \draw (6,0) to[vsourcesin, i^>, name=I] (6,2)
                to[short,-o] (8,2)
                (6,0) to[short,-o] (8,0); 
            \iarrmore{I}{$\underline{I}$};
        \end{circuitikz}
    }{{\bf Quellen von Wechselgrößen.} Ideale Wechselspannungsquelle und ideale Wechselstromquelle. \label{BildIdealeQuellen}}
\end{frame}


\begin{frame}
    \ftx{Reale Wechselspannungsquelle}

    \s{
        Bei den beiden vorgestellten Wechselquellen werden ideale Quellen abgebildet. Reale Wechselquellen weisen wie auch die 
        Quellen von Gleichspannungen und Gleichströmen Innenwiderstände auf. Diese Innenwiderstände prägen sich bei Quellen von 
        Wechselgrößen als komplexe Impedanzen aus. Die reale Wechselspannungsquelle hat also eine komplexe Innenimpedanz 
        $\underline{Z}_\mathrm{i}$, welche mit einer idealen Wechselspannungsquelle in Serie geschaltet ist. Bei realen 
        Wechselspannungsquellen ist die Ausgangsspannung $\underline{U}$ von der angeschlossenen Last abhängig. Die Gegenüberstellung
        einer idealen und einer realen Wechselspannungsquelle ist in der Abbildung \ref{BildIdealSpannung} dargestellt. 
    }

    \b{
        Wechselspannungsquelle:
        \begin{itemize}
            \item Ideale Wechselspannungsquelle
            \item<2-> Reale Wechselspannungsquelle mit Innenimpedanz $\underline{Z}_\mathrm{i}$
        \end{itemize}
    }

    \fu{
        \begin{circuitikz}
            \draw (0,0) to[vsourcesin] (0,2)
                to[short,-o] (2,2)
                (0,0) to[short,-o] (2,0);
            \draw[-latex, thick, draw=voltage] (2,1.8) to node[right, color=voltage] {$\underline{U}$} (2,0.2);
            \pause

            \draw [->](3,1) -- (4,1); 

            \draw (6,0) to[vsourcesin, v^<, name=U] (6,2)
                to[R=$\underline{Z}_\mathrm{i}$, i^>, v, name=R, -o] (8,2)
                (6,0) to[short,-o] (8,0);
            \draw[-latex, thick, draw=voltage] (8,1.8) to node[right, color=voltage] {$\underline{U}$} (8,0.2);
            \varrmore{U}{$\underline{U}_\mathrm{0}$};
            \varrmore{R}{$\underline{U}_\mathrm{i}$};
            \iarrmore{R}{$\underline{I}$};
        \end{circuitikz}
    }{{\bf Gegenüberstellung einer idealen und einer realen Wechselspannungsquelle.} Die reale Wechselspannungsquelle weist eine
    Serienimpedanz $\underline{Z}_\mathrm{i}$ auf. \label{BildIdealSpannung}}

    \s{
        Die Ausgangsspannung einer realen Wechselspannungsquelle lässt sich nach Gleichung \ref{GleichungQuel1} bestimmen. 
        Hier liegt die Wechselspannung der Wechselspannungsquelle nicht direkt an den Klemmen an. Sie wird durch die Spannung
        über die Innenimpedanz $Z_\mathrm{i}$ reduziert. 
    }

    \only<3>{
    \b{
        Ausgangsspannung einer realen Wechselspannungsquelle:
    }

    \begin{eq}
        \underline{U} = \underline{U}_\mathrm{0} - \underline{Z}_\mathrm{i} \cdot \underline{I}     \label{GleichungQuel1}
    \end{eq}
    }
\end{frame}

\begin{frame}
    \ftx{Charakteristik von realen Wechselspannungsquellen}

    \s{
        Reale Wechselspannungsquellen weisen gewisse Charakteristika ohne Last beim Leerlauf und bei einem Kurzschluss auf.
        Beim Leerlauf liegt zwischen den Klemmen eine unendlich große Impedanz. Es wird kein geschlossener Stromkreis gebildet 
        und es stellt sich kein Stromfluss ein (Gleichung \ref{GleichungQuel2}). Die gesamte Spannung der Wechselspannungsquelle 
        liegt an den offenen Klemmen an, siehe Gleichung \ref{GleichungQuel3}.
    }

    \b{
        Leerlauf:
    }

    \begin{eq}
        \underline{I} = 0 \qquad \rightarrow \qquad \underline{U}_\mathrm{i} = \underline{I} \cdot \underline{Z}_\mathrm{i} = 0 \label{GleichungQuel2}
    \end{eq}
    \begin{eq}
        \underline{U}_\mathrm{0} - \underline{U}_\mathrm{i} - \underline{U} = 0 \qquad \rightarrow \qquad \underline{U} = \underline{U}_\mathrm{0} \label{GleichungQuel3}
    \end{eq}

    \s{
        Beim Kurzschluss geht die Impedanz zwischen den Klemmen gegen null und es fällt dort keine Spannung ab. So stellt sich ein 
        größter möglicher Kurzschlussstrom ein, welcher lediglich durch die Innenimpedanz begrenzt wird (Gleichung \ref{GleichungQuel4}).
        Die Spannungs der Wechselspannungsquelle wirkt so in Gänze über die Innenimpedanz (Gleichung \ref{GleichungQuel5}). Der 
        Kurzschlussstrom lässt sich nach Gleichung \ref{GleichungQuel6} aus dem Verhältnis aus der Spannung über die Innenimpedanz 
        $\underline{U}_\mathrm{i}$ und der Innenimpedanz $\underline{Z}_\mathrm{i}$ bestimmen.
    }

    \b{
        Kurzschluss:
    }

    \begin{eq}
        \underline{U} = 0 \qquad \rightarrow \qquad \underline{I} = \underline{I}_\mathrm{k} \label{GleichungQuel4}
    \end{eq}
    \begin{eq}
        \underline{U}_\mathrm{0} - \underline{U}_\mathrm{i} - \underline{U} = 0 \qquad \rightarrow \qquad \underline{U}_\mathrm{i} = \underline{U}_\mathrm{0} \label{GleichungQuel5}
    \end{eq}
    \begin{eq}
        \underline{I}_\mathrm{k} = \frac{\underline{U}_\mathrm{i}}{\underline{Z}_\mathrm{i}} \label{GleichungQuel6}
    \end{eq}

\end{frame}


\begin{frame}
    \ftx{Beispielaufgabe}
	\begin{bsp}{Reale Spannungsquelle}{Reale Spannungsquelle:}

    \begin{enumerate}
        \only<1>{\item[]	
                
        Eine Reale Spannungsquelle (230\ V, 50\ Hz) verfügt nach der unten stehenden Abbildung über eine Innenimpedanz $\underline{Z}_\mathrm{i}$, welche sich aus 
        einer Spule $L=20\ mH$ und einem ohmschen Widerstand $R=10\ m\Omega$ zusammensetzt. 
        
        \fu{
            \resizebox{0.5\textwidth}{!}{
                \begin{circuitikz}
                    \draw (0,0) to[vsourcesin, v^<, name=U] (0,2)
                    to[L=$L$] (2,2)
                    to[R=$R$, i^>, name=R, -o] (4,2)
                    (0,0) to[short,-o] (4,0);
                    \draw[-latex, thick, draw=voltage] (4,1.8) to node[right, color=voltage] {$\underline{U}$} (4,0.2);
                    \varrmore{U}{$\underline{U}_\mathrm{0}$};
                    \iarrmore{R}{$\underline{I}$};
                \end{circuitikz}
                }
            }{{\bf Beispiel.} Wechselspannungsquelle mit einer Innenimpedanz aus einer Spule und einem Widerstand. \label{BildBeispielQuelle}}
        
        Bestimmen Sie die {\bf Leerlaufspannung} und den {\bf Kurzschlussstrom} der Wechselspannungsquelle.
        }

        \only<2>{
            \item[] 
            \onslide<2>{
        \begin{eqa}
            \underline{U}_\mathrm{0} &= 230\ V \cdot e^{\mathrm{j}0^o}          \nonumber    \\
            \underline{Z}_\mathrm{i} &= R + j X_L       \nonumber    \\
            \underline{Z}_\mathrm{i} &= 0,01\ \Omega + j \cdot 2\pi \cdot 50\ Hz \cdot 20\ mH      \nonumber    \\
            \underline{Z}_\mathrm{i} &= 0,01\ \Omega + j 6,28\ \Omega       \nonumber
        \end{eqa}

        {\bf Leerlaufspannung}:
        \begin{eqa}
            \underline{U} &= \underline{U}_\mathrm{0}        \nonumber    \\
            \underline{U} &= 230\ V \cdot e^{\mathrm{j}0^o}         \nonumber
        \end{eqa}

        {\bf Kurzschlussstrom}:
        \begin{eqa}
            \underline{I}_\mathrm{k} &= \frac{\underline{U}_\mathrm{0}}{\underline{Z}_\mathrm{i}}           \nonumber        \\
            \underline{I}_\mathrm{k} &= \frac{230\ V \cdot e^{\mathrm{j}0^o}}{0,01\ \Omega + j 6,28\ \Omega}           \nonumber        \\
            \underline{I}_\mathrm{k} &= (0,058 - j36,6)\ A = 36,6\ A \cdot e^{-\mathrm{j}89,9^o}    \nonumber
        \end{eqa}
            }
        }    
    \end{enumerate}    
    \end{bsp}
\end{frame}


\begin{frame}
    \ftx{Reale Wechselstromquelle}

    \s{
        Die reale Wechselstromquelle in der Abbildung \ref{BildIdealStrom} verfügt wie die Wechselspannungsquelle über 
        eine Innenimpedanz $\underline{Z}_\mathrm{i}$, wobei diese Innenimpedanz nicht wie bei der Wechselspannungsquelle 
        in serie geschlatet ist, sondern parallel zu einer idealen Wechselstromquelle liegt. Der sich ergebende Ausgangsstrom 
        $\underline{I}$ der Quelle ist von der angeschlossenen Last, bzw. von der Ausgangsspannung $\underline{U}$ abhängig. 
    }

    \b{
        Wechselstromquelle:
        \begin{itemize}
            \item<1-> Ideale Wechselstromquelle
            \item<2-> Reale Wechselstromquelle mit Innenimpedanz $\underline{Z}_\mathrm{i}$
        \end{itemize}
    }

    \fu{
        \begin{circuitikz}
            \draw (0,0) to[isourcesin, i, name=I1] (0,2)
                to[short,-o] (2,2)
                (0,0) to[short,-o] (2,0);
            \draw[-latex, thick, draw=voltage] (2,1.8) to node[right, color=voltage] {$\underline{U}$} (2,0.2);
            \pause

            \draw [->](3,1) -- (4,1); 

            \draw (6,0) to[isourcesin, i, name=I2] (6,2)
                to [short] (8,2) 
                to [short, i, name=S, -o] (10,2)
                (8,2) to [R=$\underline{Z}_\mathrm{i}$, i>^, name=R, *-*] (8,0)
                (6,0) to [short,-o] (10,0);
            \draw[-latex, thick, draw=voltage] (10,1.8) to node[right, color=voltage] {$\underline{U}$} (10,0.2);
            \iarrmore{I1}{$\underline{I}_\mathrm{0}$};
            \iarrmore{I2}{$\underline{I}_\mathrm{0}$};
            \iarrmore{R}{$\underline{I}_\mathrm{i}$};
            \iarrmore{S}{$\underline{I}$};
        \end{circuitikz}
    }{{\bf Quellen von Wechselgrößen.} Ideale Wechselstromquelle und ideale Wechselstromquelle. \label{BildIdealStrom}}

    \s{
        Der Wechselstrom der idealen Wechselstromquelle wird durch einen Anteil durch die Innenimpedanz reduziert. Der 
        Ausgangsstrom der realen Wechselstromquelle lässt sich über die Gleichung \ref{GleichungQuel7} bestimmen. Hier 
        wird ausgehend vom Quellstrom $\underline{I}_\mathrm{0}$ der ausgangsspannungsabhängige ($\underline{U}$) 
        Wechselstromanteil  $\underline{I}_\mathrm{i}$ durch die Innenimpedanz $\underline{Z}_\mathrm{i}$ abgezogen.
    }

    \only<3>{
    \b{
        Ausgangswechselstrom einer realen Wechselstromquelle:
    }

    \begin{eq}
        \underline{I} = \underline{I}_\mathrm{0} - \underline{I}_\mathrm{i} = \underline{I}_\mathrm{0} - \frac{\underline{U}}{\underline{Z}_\mathrm{i}}     \label{GleichungQuel7}
    \end{eq}
    }
\end{frame}


\begin{frame}
    \ftx{Äquivalenz von Wechselquellen}

    \s{
        {\bf Wechselquellenumwandlung} \\


        Wechselstrom- und Wechselspannungsquellen können ineinander umgewandelt werden, wenn beide Wechselquellen den identischen
        Kurzschlussstrom und die gleiche Leerlaufspannung aufweisen. 
    } 
    
    \b{
        Umwandlung von Wechselstromquellen und Wechselspannungsquellen:
    }
    
    \fu{
        \begin{circuitikz}
            \draw (0,0) to[vsourcesin, v^<, name=U] (0,2)
                to[R=$\underline{Z}_\mathrm{i}$, i^>, v, name=R1, -o] (2,2)
                (0,0) to[short,-o] (2,0);
            \draw[-latex, thick, draw=voltage] (2,1.8) to node[right, color=voltage] {$\underline{U}$} (2,0.2);

            \draw [<->](3.5,1) -- (4.5,1); 

            \draw (6,0) to[isourcesin, i, name=I2] (6,2)
                to [short] (8,2) 
                to [short, i, name=S, -o] (10,2)
                (8,2) to [R=$\underline{Z}_\mathrm{i}$, i>^, name=R2, *-*] (8,0)
                (6,0) to [short,-o] (10,0);
            \draw[-latex, thick, draw=voltage] (10,1.8) to node[right, color=voltage] {$\underline{U}$} (10,0.2);
            \iarrmore{I1}{$\underline{I}_\mathrm{0}$};
            \iarrmore{I2}{$\underline{I}_\mathrm{0}$};
            \iarrmore{R2}{$\underline{I}_\mathrm{i}$};
            \iarrmore{S}{$\underline{I}$};        

            \varrmore{U}{$\underline{U}_\mathrm{0}$};
            \varrmore{R1}{$\underline{U}_\mathrm{i}$};
            \iarrmore{R1}{$\underline{I}$};
        \end{circuitikz}
    }{{\bf Gegenüberstellung einer realen Wechselspannungsquelle und einer realen Wechselstromquelle.} Die reale Wechselspannungsquelle 
    und die reale Wechselstromquelle sind ineinander umwandelbar. \label{BildQuelleUmwandlung}}

    \s{
        Beide Wechselquellen sind äquivalent, wenn nach Gleichung \ref{GleichungQuel8} und Gleichung \ref{GleichungQuel9} das ohmsche 
        Gesetz mit identischen Innenimpedanzen für die Wechselspannungsquelle und die Wechselstromquelle gilt. 
    }
    
    \b{
        Äquivalente Wechselquellen, bei identischen Innenimpedanzen:
    }

    \begin{eq}
        \underline{U}_\mathrm{0} = \underline{Z}_\mathrm{0} \cdot \underline{I}_\mathrm{0}      \label{GleichungQuel8}
    \end{eq}
    \begin{eq}
        \underline{Z}_\mathrm{i} = \underline{Z}_\mathrm{0}      \label{GleichungQuel9}
    \end{eq}

    \s{
        Hierbei soll der Kurzschlussstrom $\underline{I}_\mathrm{K}$ gleich dem Quellstrom $\underline{I}_\mathrm{0}$ der 
        Wechselstromquelle sein, welcher identisch zu dem Ausgangsstrom $\underline{I}$ ist. Die Ausgangswechselspannung
        $\underline{U}$ der Wechselquellen enspricht der Leerlaufspannung $\underline{U}_\mathrm{L}$.
    }

    \b{
        Quellstrom der Stromquelle gleich dem Kurzschlussstrom. \\
        Quellspannung der Spannungsquelle gleich der Leerlaufspannung.
    }

    \begin{eq}
        \underline{I} = \underline{I}_\mathrm{k} = \underline{I}_\mathrm{0} \qquad \text{und} \qquad \underline{U} = \underline{U}_\mathrm{L} = \underline{U}_\mathrm{0}     \label{GleichungQuel10}
    \end{eq}

\end{frame}



\begin{frame}
    \ftx{Beispielaufgabe}
	\begin{bsp}{Wechselquellenumwandlung}{Wechselquellenumwandlung:}

    \begin{enumerate}
        \only<1>{\item[]
        Gegeben ist eine reale Wechselspannungsquelle mit $\underline{U}_\mathrm{0} = 20\ kV \cdot \mathrm{e}^{\mathrm{j}30^o}$ und 
        $\underline{Z}_\mathrm{0} = (0,01 + j \cdot 2)\ \Omega$. Wandeln Sie die Wechselspannungsquelle in eine äquivalente 
        Wechselstromquelle um, berechnen Sie hierzu den Kurzschlussstrom.\\
        }
        \only<2>{
            \item[] 
            \onslide<2>{
            {\bf Kurzschlussstrom:}


            \begin{eq}
                \underline{I}_\mathrm{0} = \frac{\underline{U}_\mathrm{0}}{\underline{Z}_\mathrm{0}}
                = \frac{20\ kV \cdot \mathrm{e}^{\mathrm{j}30^o}}{(0,01 + j \cdot 2)\ \Omega}
                = (5,04 + j \cdot 8,64)\ kA     \nonumber
            \end{eq}
            }
        }
    \end{enumerate}

    \end{bsp}
\end{frame}


\begin{frame}
    \ftb{Komplexer Spannungsteiler und komplexer Stromteiler}

    \s{
        Wie bei der Betrachtung von Gleichstromnetzwerken gelten die Grundgesetze zur Analyse von Wechselstromnetzen.
        Die Knotenregel und die Maschenregel sind hier äquivalent anzuwenden. Der bereits vorgestellte Beispielknoten 
        und die Beispielmasche werden in der Abbildung \ref{BildKnotenMaschenKomplex} mit komplexen Wechselgrößen noch 
        einmal dargestellt. Hieraus resultiert die Ausformulierung der folgenden Gleichung \ref{KnotenregelKomplex} als 
        Beispiel der komplexen Knotenregel:

        \begin{eq}
            \sum_{k = 1}^{N}{\underline{I}_\mathrm{k}} = 0 = \underline{I}_\mathrm{1} + \underline{I}_\mathrm{2} - 
            \underline{I}_\mathrm{3} - \underline{I}_\mathrm{4} - \underline{I}_\mathrm{5}    \label{KnotenregelKomplex}
        \end{eq}

        Sowie die Gleichung \ref{MaschengleichungKomplex} als Beispiel der komplexen Maschenregel:

        \begin{eq}
            \sum_\mathrm{k = 1}^{N}{\underline{U}_\mathrm{k}} = \underline{U}_\mathrm{Z2} - \underline{U}_\mathrm{Z1} 
            = 0  \label{MaschengleichungKomplex}
        \end{eq}

        Hier ist im Gegensatz zur Gleichstrombetrachtung, wo ausschließlich mit reellen Werten gerechnet wird, lediglich 
        darauf zu achten, dass mit komplexen Wechselgrößen gearbeitet wird. 
    }

    \b{
        Wie bei der Betrachtung von Gleichstromnetzwerken gelten die Grundgesetze zur Analyse von Wechselstromnetzen.\\
        \begin{itemize}
            \item<2-> Knotenregel:
                \begin{eq}
                    \sum_{k = 1}^{N}{\underline{I}_\mathrm{k}} = 0 = \underline{I}_\mathrm{1} + \underline{I}_\mathrm{2} - 
                    \underline{I}_\mathrm{3} - \underline{I}_\mathrm{4} - \underline{I}_\mathrm{5}   \nonumber
                \end{eq}
            \item<3-> Maschenregel:
                \begin{eq}
                    \sum_\mathrm{k = 1}^{N}{\underline{U}_\mathrm{k}} = \underline{U}_\mathrm{Z2} - \underline{U}_\mathrm{Z1} 
                    = 0  \nonumber
                \end{eq}
        \end{itemize}
    }

    \fu{
        \resizebox{0.7\textwidth}{!}{
        \begin{circuitikz}
            \pause

            \draw (0,3) to [short, i=\red{$\underline{I}_\mathrm{1}$},-*] (2,2)
            (0,1) to [short, i=\red{$\underline{I}_\mathrm{2}$}] (2,2)
            (2,2) to [short, i=\red{$\underline{I}_\mathrm{3}$}] (3.5,3.5)
            (2,2) to [short, i=\red{$\underline{I}_\mathrm{4}$}] (4,2)
            (2,2) to [short, i=\red{$\underline{I}_\mathrm{5}$}] (3.5,0.5);
            \pause

            \draw (6,0) to [vsourcesin, v^<, name=U0] (6,4)
            to [short, i=\red{$\underline{I}_\mathrm{0}$}, -*] (10,4)
			to [short] (14,4)
			(10,4) to [R=$\underline{Z}_\mathrm{1}$, v, i, name=Z1] (10,0)
			(14,4) to [R=$\underline{Z}_\mathrm{2}$, v, i, name=Z2] (14,0)
            to[short,-*] (10,0)
            to[short] (6,0);
            \draw[->,shift={(12,2)},voltage] (150:0.8) arc (150:-150:0.8) node at(0,0){$M_\mathrm{1}$};
			\varrmore{Z1}{$\underline{U}_\mathrm{{\underline{Z}1}}$};
			\iarrmore{Z1}{$\underline{I}_\mathrm{{\underline{Z}1}}$};
			\varrmore{Z2}{$\underline{U}_\mathrm{{\underline{Z}2}}$};
			\iarrmore{Z2}{$\underline{I}_\mathrm{{\underline{Z}2}}$};
			\varrmore{U0}{$\underline{U}_\mathrm{0}$};
        \end{circuitikz}
        }
    }{{\bf Knotenregel und Maschenregel mit komplexen Wechselgrößen.} Die Knotenregel und die Maschenregel verhalten sich 
    mit komplexen Wechselgrößen identisch wie bei Gleichstromnetzwerken. \label{BildKnotenMaschenKomplex}}



\end{frame}

\begin{frame}
    \ftx{Komplexer Spannungsteiler}

    \s{
        Der komplexe Spannungsteiler dient der Reduzierung einer Gesamtspannung z.B. der Anpassung eines Signalpegels an dem 
        Eingangsspannungsbereich einer Messschaltung. In der Abbildung \ref{BildKomplxSpannungsteiler1} wird ein komplexer 
        Spannungsteiler aus zwei komplexen Impedanzen dargestellt. Die komplexe Gesamtspannung $\underline{U}_\mathrm{0}$
        wird über die beiden Impedanzen $\underline{Z}_\mathrm{1}$ und $\underline{Z}_\mathrm{2}$ in die Teilspannungen 
        $\underline{U}_\mathrm{1}$ und $\underline{U}_\mathrm{2}$ aufgeteilt.  

        \fu{
            \resizebox{0.3\textwidth}{!}{
            \begin{circuitikz}
                \draw (0,0) to [short, o-o] (3,0)
                (0,4) to [short, o-] (1.5,4)
                to [short,-o] (3,4)
                (1.5,2) to [short, *-o] (3,2)
                (1.5,4) to [R=$\underline{Z}_\mathrm{1}$, *-] (1.5,2)
                to [R=$\underline{Z}_\mathrm{2}$, -*] (1.5,0);
                \draw[-latex, thick, draw=voltage] (0,3.8) to node[left, color=voltage] {$\underline{U}_\mathrm{0}$} (0,0.2);
                \draw[-latex, thick, draw=voltage] (3,3.8) to node[right, color=voltage] {$\underline{U}_\mathrm{1}$} (3,2.2);
                \draw[-latex, thick, draw=voltage] (3,1.8) to node[right, color=voltage] {$\underline{U}_\mathrm{2}$} (3,0.2);

                \draw (0,4) to [short, i, name=I, o-] (1.5,4);
                \iarrmore{I}{$\underline{I}$};     
            \end{circuitikz}
            }
        }{{\bf Ein komplexer Spannungsteiler bestehend aus zwei Impedanzen in Serie.} Die komplexe Gesamtspannung
        $\underline{U}_\mathrm{0}$ wird in die Teilspannungen $\underline{U}_\mathrm{1}$ und $\underline{U}_\mathrm{2}$
        aufgeteilt. \label{BildKomplxSpannungsteiler1}}

        Die Gesamtspannung teilt sich im Verhältnis der komplexen Impedanzen auf (vgl. Gleichung \ref{GleichungTeiler1}). Basierend
        auf dem komplexen Strom $\underline{I}$, welcher die gesamte Serienschaltung durchströmt, lassen sich die komplexen Teilspannungen
        $\underline{U}_\mathrm{1}$ und $\underline{U}_\mathrm{2}$ und ihre komplexen Impedanzen $\underline{Z}_\mathrm{1}$ und 
        $\underline{Z}_\mathrm{2}$ im Verhältnis zu der komplexen Gesamtspannung $\underline{U}_\mathrm{0}$ und der gesamten komplexen 
        Impedanz $\underline{Z}_\mathrm{1} + \underline{Z}_\mathrm{2}$ bestimmen. 

        \begin{eq}
            \underline{I} = \frac{\underline{U}_\mathrm{0}}{\underline{Z}_\mathrm{1} + \underline{Z}_\mathrm{2}} 
            = \frac{\underline{U}_\mathrm{1}}{\underline{Z}_\mathrm{1}}
            = \frac{\underline{U}_\mathrm{2}}{\underline{Z}_\mathrm{2}}     \label{GleichungTeiler1}
        \end{eq}
                
        Durch das Umstellen der Gleichung \ref{GleichungTeiler1} kann das komplexe Teilverhältnis $\underline{T}$ für die Gleichung 
        \ref{GleichungTeiler2} aufgestellt werden: 

        \begin{eq}
            \underline{T} = \frac{\underline{U}_\mathrm{2}}{\underline{U}_\mathrm{0}}
            = \frac{\underline{Z}_\mathrm{2}}{\underline{Z}_\mathrm{1} + \underline{Z}_\mathrm{2}}    \label{GleichungTeiler2}  
        \end{eq}

        Hier entspricht das Verhältnis aus der komplexen Teilspannung $\underline{U}_\mathrm{2}$ und der komplexen Gesamtspannung 
        $\underline{U}_\mathrm{0}$ dem Verhältnis aus der komplexen Impedanz $\underline{Z}_\mathrm{2}$ und der gesamten komplexen 
        Impedanz der Serienschaltung $\underline{Z}_\mathrm{1} + \underline{Z}_\mathrm{2}$. 
        Das Aufstellen des komplexen Teilverhältnises funktioniert in gleicher Weise für die komplexe Teilspannung $\underline{U}_\mathrm{1}$.
    }

    \b{
        \begin{minipage}[t]{0.4\textwidth}
                \fu{
                    \resizebox{\textwidth}{!}{
                    \begin{circuitikz}
                        \draw (0,0) to [short, o-o] (3,0)
                        (0,4) to [short, o-] (1.5,4)
                        to [short,-o] (3,4)
                        (1.5,2) to [short, *-o] (3,2)
                        (1.5,4) to [R=$\underline{Z}_\mathrm{1}$, *-] (1.5,2)
                        to [R=$\underline{Z}_\mathrm{2}$, -*] (1.5,0);
                        \draw[-latex, thick, draw=voltage] (0,3.8) to node[left, color=voltage] {$\underline{U}_\mathrm{0}$} (0,0.2);
                        \draw[-latex, thick, draw=voltage] (3,3.8) to node[right, color=voltage] {$\underline{U}_\mathrm{1}$} (3,2.2);
                        \draw[-latex, thick, draw=voltage] (3,1.8) to node[right, color=voltage] {$\underline{U}_\mathrm{2}$} (3,0.2);
                        \pause

                        \draw (0,4) to [short, i, name=I, o-] (1.5,4);
                        \iarrmore{I}{$\underline{I}$};     
                    \end{circuitikz}
                    }
                }{}
        \end{minipage}
        \begin{minipage}[t]{0.5\textwidth}
            \begin{itemize}
                \item<1-> Reduzierung einer Gesamtspannung z.B. zur Anpassung eines Signalpegels an dem 
                    Eingangsspannungsbereich einer Messschaltung
                \item<2-> Die Gesamtspannung teilt sich im Verhältnis der komplexen Impedanzen auf:
                    \begin{eq}
                        \underline{I} = \frac{\underline{U}_\mathrm{0}}{\underline{Z}_\mathrm{1}+\underline{Z}_\mathrm{2}} 
                        = \frac{\underline{U}_\mathrm{1}}{\underline{Z}_\mathrm{1}}
                        = \frac{\underline{U}_\mathrm{2}}{\underline{Z}_\mathrm{2}}     \nonumber
                    \end{eq}
                \item<3-> Komplexes Teilerverhältnis
                    \begin{eq}
                        \underline{T} = \frac{\underline{U}_\mathrm{2}}{\underline{U}_\mathrm{0}}
                        = \frac{\underline{Z}_\mathrm{2}}{\underline{Z}_\mathrm{1}+\underline{Z}_\mathrm{2}}    \nonumber
                    \end{eq}
            \end{itemize}
        \end{minipage}
    }
\end{frame}


\begin{frame}
    \ftx{Komplexer Spannungsteiler: Anwendungsbeispiel Hochspannungstechnik}

    \s{
        Der komplexe Spannungsteiler findet beispielsweise im ohmsch-kapazitiven Teiler in der Hochspannungstechnik Anwendung. 
        In der Abbildung \ref{BildKomplxSpannungsteiler2} werden vier verschiedene Teiler dargestellt, welche beispielsweise in 
        der Hochspannungstechnik verwendet werden. 

        \fu{
            \resizebox{0.5\textwidth}{!}{
            \begin{circuitikz}
                \draw (0,10) to [short, o-] (1,10) 
                to [R] (1,8) to [R] (1,6) to [R] (1,4) to [R] (1,2) to [R] (1,0)
                (1,2) to [short, *-o] (2,2)
                (0,0) to [short, o-*] (1,0) to [short, -o] (2,0);
                \draw[-latex, thick, draw=voltage] (0,9.8) to node[left, color=voltage] {$\underline{U}_\mathrm{E}$} (0,0.2);
                \draw[-latex, thick, draw=voltage] (2,1.8) to node[right, color=voltage] {$\underline{U}_\mathrm{A}$} (2,0.2);
                \draw (1,0) node [label=below:Ohmscher Teiler]{};

                \draw (4,10) to [short, o-*] (5,10) 
                to [R] (5,8) to [R] (5,6) to [R] (5,4) to [R] (5,2) to [R] (5,0)
                (5,10) to [short] (6,10)
                to [C] (6,8) to [C] (6,6) to [C] (6,4) to [C] (6,2) to [C] (6,0)
                (5,8) to [short, *-*] (6,8)
                (5,6) to [short, *-*] (6,6)
                (5,4) to [short, *-*] (6,4)
                (5,2) to [short, *-*] (6,2) to [short, -o] (7,2)
                (4,0) to [short, o-o] (7,0) (5,0) to [short, *-*] (6,0);
                \draw[-latex, thick, draw=voltage] (4,9.8) to node[left, color=voltage] {$\underline{U}_\mathrm{E}$} (4,0.2);
                \draw[-latex, thick, draw=voltage] (7,1.8) to node[right, color=voltage] {$\underline{U}_\mathrm{A}$} (7,0.2);
                \draw (5.5,0) node [label=below:Ohmsch-kapazitiver Teiler]{};

                \draw (9,10) to [short, o-] (10,10) 
                to [C] (10,8) to [C] (10,6) to [C] (10,4) to [C] (10,2) to [C] (10,0)
                (10,2) to [short, *-o] (11,2)
                (9,0) to [short, o-*] (10,0) to [short, -o] (11,0);
                \draw[-latex, thick, draw=voltage] (9,9.8) to node[left, color=voltage] {$\underline{U}_\mathrm{E}$} (9,0.2);
                \draw[-latex, thick, draw=voltage] (11,1.8) to node[right, color=voltage] {$\underline{U}_\mathrm{A}$} (11,0.2);
                \draw (10,0) node [label=below:Kapazitiver Teiler]{};

                \draw (13,10) to [short, o-] (13.5,10) to [L] (14.5,10) to [short] (15,10) 
                to [R] (15,8.6) to [C] (15,8) to [R] (15,6.6) to [C] (15,6) to [R] (15,4.6) to [C] (15,4) to [R] (15,2.6) to [C] (15,2) to [R] (15,0.6) to [C] (15,0)
                (15,2) to [short, *-o] (16,2) 
                (13,0) to [short, o-*] (15,0) to [short, -o] (16,0);
                \draw[-latex, thick, draw=voltage] (13,9.8) to node[left, color=voltage] {$\underline{U}_\mathrm{E}$} (13,0.2);
                \draw[-latex, thick, draw=voltage] (16,1.8) to node[right, color=voltage] {$\underline{U}_\mathrm{A}$} (16,0.2);
                \draw (14.5,0) node [label=below:Gedämpft kapazitiver Teiler]{};
            \end{circuitikz}
            }
        }{{\bf Gegenüberstellung von verschiedenen Teilern der Hochspannungstechnik.} Ohmscher Teiler zur Verwendung mit Gleichspannung und 
        der kapazitive Teiler und der gedämpft kapazitive Teiler zur Verwendung mit Wechselspannung. Der ohmsch-kapazitive Teiler ist für 
        beide Spannungsarten verwendbar. \label{BildKomplxSpannungsteiler2}}

        Neben dem ohmsch-kapazitiven Teiler werden noch der ohmsche Teiler, der 
        kapazitive Teiler und der gedämpft kapazitive Teiler vorgestellt. Der ohmsche Teiler besteht rein aus ohmschen Widerständen, 
        welche seriell verschaltet sind und der kapazitive Teiler besteht rein aus Kondensatoren. Der gedämpft kapazitive Teiler 
        besteht aus mehreren Modulen mit einer vorgeschalteten Spule. Die Module setzten sich aus Reihenschaltungen aus einem 
        ohmschen Widerstand und einem Kondensator zusammensetzten.  
    }
        
    \b{
        \begin{minipage}[t]{0.55\textwidth}
        \fu{
            \resizebox{\textwidth}{!}{
            \begin{circuitikz}
                \draw (0,10) to [short, o-] (1,10) 
                to [R] (1,8) to [R] (1,6) to [R] (1,4) to [R] (1,2) to [R] (1,0)
                (1,2) to [short, *-o] (2,2)
                (0,0) to [short, o-*] (1,0) to [short, -o] (2,0);
                \draw[-latex, thick, draw=voltage] (0,9.8) to node[left, color=voltage] {$\underline{U}_\mathrm{E}$} (0,0.2);
                \draw[-latex, thick, draw=voltage] (2,1.8) to node[right, color=voltage] {$\underline{U}_\mathrm{A}$} (2,0.2);
                \draw (1,0) node [label=below:Ohmscher Teiler]{};
                \pause

                \draw (4,10) to [short, o-*] (5,10) 
                to [R] (5,8) to [R] (5,6) to [R] (5,4) to [R] (5,2) to [R] (5,0)
                (5,10) to [short] (6,10)
                to [C] (6,8) to [C] (6,6) to [C] (6,4) to [C] (6,2) to [C] (6,0)
                (5,8) to [short, *-*] (6,8)
                (5,6) to [short, *-*] (6,6)
                (5,4) to [short, *-*] (6,4)
                (5,2) to [short, *-*] (6,2) to [short, -o] (7,2)
                (4,0) to [short, o-o] (7,0) (5,0) to [short, *-*] (6,0);
                \draw[-latex, thick, draw=voltage] (4,9.8) to node[left, color=voltage] {$\underline{U}_\mathrm{E}$} (4,0.2);
                \draw[-latex, thick, draw=voltage] (7,1.8) to node[right, color=voltage] {$\underline{U}_\mathrm{A}$} (7,0.2);
                \draw (5.5,0) node [label=below:Ohmsch-kapazitiver Teiler]{};
                \pause

                \draw (9,10) to [short, o-] (10,10) 
                to [C] (10,8) to [C] (10,6) to [C] (10,4) to [C] (10,2) to [C] (10,0)
                (10,2) to [short, *-o] (11,2)
                (9,0) to [short, o-*] (10,0) to [short, -o] (11,0);
                \draw[-latex, thick, draw=voltage] (9,9.8) to node[left, color=voltage] {$\underline{U}_\mathrm{E}$} (9,0.2);
                \draw[-latex, thick, draw=voltage] (11,1.8) to node[right, color=voltage] {$\underline{U}_\mathrm{A}$} (11,0.2);
                \draw (10,0) node [label=below:Kapazitiver Teiler]{};
                \pause

                \draw (13,10) to [short, o-] (13.5,10) to [L] (14.5,10) to [short] (15,10) 
                to [R] (15,8.6) to [C] (15,8) to [R] (15,6.6) to [C] (15,6) to [R] (15,4.6) to [C] (15,4) to [R] (15,2.6) to [C] (15,2) to [R] (15,0.6) to [C] (15,0)
                (15,2) to [short, *-o] (16,2) 
                (13,0) to [short, o-*] (15,0) to [short, -o] (16,0);
                \draw[-latex, thick, draw=voltage] (13,9.8) to node[left, color=voltage] {$\underline{U}_\mathrm{E}$} (13,0.2);
                \draw[-latex, thick, draw=voltage] (16,1.8) to node[right, color=voltage] {$\underline{U}_\mathrm{A}$} (16,0.2);
                \draw (14.5,0) node [label=below:Gedämpft kapazitiver Teiler]{};
            \end{circuitikz}
            }
        }{}
        \end{minipage}
        \begin{minipage}[t]{0.35\textwidth}
            \begin{itemize}
                \item<1-> Ohmscher Teiler (Gleichspannung)
                \item<2-> Ohmsch-kapazitiver Teiler (Gleichspannung und Wechselspannung)
                \item<3-> Kapazitiver Teiler (Wechselspannung)
                \item<4-> Gedämpft kapazitiver Teiler (Wechselspannung)
            \end{itemize}
        \end{minipage}
    }
\end{frame}

\begin{frame}
    \ftx{Komplexer Spannungsteiler}
        
    \s{
        Das Einfache Beispiel eines solchen ohmsch-kapazitiven Teilers
        wird in der Abbildung \ref{BildKomplxSpannungsteiler3} gezeigt. Der ohmsch-kapazitive Teiler setzt sich aus mindestens zwei 
        in serie geschlateten Parallelschaltungen aus einem ohmschen Widerstand und einem Kondensator, einem sogennanten RC-Glied 
        zusammen. In der Abbildung \ref{BildKomplxSpannungsteiler3} besteht das erste RC-Glied aus dem Widerstand $R_1$ und dem 
        Kondensator $C_1$. Das zweite RC-Glied besteht aus dem Widerstand $R_2$ und Kondensator $C_2$. Die komplexe Gesamtspannung 
        $\underline{U}_\mathrm{0}$ liegt über beide RC-Glieder an. Die komplexe Teilspannung $\underline{U}_\mathrm{2}$ liegt über 
        dem zweiten RC-Glied an. Das Teilverhältnis hat somit ohmsche und kapazitive Anteile. So liegt durch die Kapazität des 
        Kondensators ein frequenzabhängiges Teilverhältnis vor. 

        \fu{
            \resizebox{0.3\textwidth}{!}{
            \begin{circuitikz}
                \draw (0,0) to [short, o-*] (2.25,0)
                to [short, -o] (4.5,0)
                (0,6) to [short, o-] (2.25,6) to [short] (2.25,5.5)
                (1.25,5.5) to [short] (2.25,5.5) to [short, *-] (3.25,5.5)
                (1.25,3.5) to [short] (2.25,3.5) to [short, *-] (3.25,3.5)
                (1.25,2.5) to [short] (2.25,2.5) to [short, *-] (3.25,2.5)
                (1.25,0.5) to [short] (2.25,0.5) to [short, *-] (3.25,0.5)
                (2.25,3.5) to [short] (2.25,2.5)
                (2.25,0) to [short] (2.25,0.5)
                (2.25,3) to [short, *-o] (4.5,3)
                (1.25,5.5) to [R=$R_\mathrm{1}$] (1.25,3.5)
                (3.25,5.5) to [C=$C_\mathrm{1}$] (3.25,3.5)
                (1.25,2.5) to [R=$R_\mathrm{2}$] (1.25,0.5)
                (3.25,2.5) to [C=$C_\mathrm{2}$] (3.25,0.5);
                \draw[-latex, thick, draw=voltage] (0,5.8) to node[left, color=voltage] {$\underline{U}_\mathrm{0}$} (0,0.2);
                \draw[-latex, thick, draw=voltage] (4.5,2.8) to node[right, color=voltage] {$\underline{U}_\mathrm{2}$} (4.5,0.2);
            \end{circuitikz}
            }
        }{{\bf Ein ohmsch-kapazitiver Spannungsteiler aus zwei RC-Gliedern.} Die RC-Glieder bestehen jeweils aus einer Parallelschaltung 
        eines Widerstandes und eines Kondensators. \label{BildKomplxSpannungsteiler3}}

        Ein besonderer Zustand liegt vor, wenn der komplexe Spannungsteiler abgeglichen ist. Dieser Zustand beschreibt das identische 
        Verhältnis der Bauteile zueinander. Dieser Zustand wird durch die Gleichung \ref{GleichungTeiler3} beschrieben. Hier 
        muss das Produkt aus dem Widerstand $R_1$ und dem Kondensator $C_1$ identisch mit dem Produkt aus dem Widerstand $R_2$ und dem 
        Kondensator $C_2$ sein. Diese Abgleichbedingung lässt sich in der Praxis durch veränderliche Bauteile, wie Trimmkondensatoren, 
        realisieren. 

        \begin{eq}
            R_1 \cdot C_1 = R_2 \cdot C_2   \label{GleichungTeiler3} 
        \end{eq} 
        
    }

    \b{
        \begin{minipage}[t]{0.4\textwidth}
                \fu{
                    \resizebox{\textwidth}{!}{
                    \begin{circuitikz}
                        \draw (0,0) to [short, o-*] (2.25,0)
                        to [short, -o] (4.5,0)
                        (0,6) to [short, o-] (2.25,6) to [short] (2.25,5.5)
                        (1.25,5.5) to [short] (2.25,5.5) to [short, *-] (3.25,5.5)
                        (1.25,3.5) to [short] (2.25,3.5) to [short, *-] (3.25,3.5)
                        (1.25,2.5) to [short] (2.25,2.5) to [short, *-] (3.25,2.5)
                        (1.25,0.5) to [short] (2.25,0.5) to [short, *-] (3.25,0.5)
                        (2.25,3.5) to [short] (2.25,2.5)
                        (2.25,0) to [short] (2.25,0.5)
                        (2.25,3) to [short, *-o] (4.5,3)
                        (1.25,5.5) to [R=$R_\mathrm{1}$] (1.25,3.5)
                        (3.25,5.5) to [C=$C_\mathrm{1}$] (3.25,3.5)
                        (1.25,2.5) to [R=$R_\mathrm{2}$] (1.25,0.5)
                        (3.25,2.5) to [C=$C_\mathrm{2}$] (3.25,0.5);
                        \draw[-latex, thick, draw=voltage] (0,5.8) to node[left, color=voltage] {$\underline{U}_\mathrm{0}$} (0,0.2);
                        \draw[-latex, thick, draw=voltage] (4.5,2.8) to node[right, color=voltage] {$\underline{U}_\mathrm{2}$} (4.5,0.2);
                    \end{circuitikz}
                    }
                }{}
        \end{minipage}
        \begin{minipage}[t]{0.5\textwidth}
            \begin{itemize}
                \item<1-> Ohmsches und kapazitives Teilungsverhältnis
                \item<2-> Impedanzen anhängig von der Frequenz
                \item<3-> Komplexes Teilerverhältnis ist frequenzabhängig
                \item<4-> Das Teilerverhältnis wird nur im Ausnahmefall
                unabhängig von der Frequenz!
                    \begin{eq}
                        R_1 \cdot C_1 = R_2 \cdot C_2   \nonumber    
                    \end{eq}
            \end{itemize}
        \end{minipage}
    }
\end{frame}

\begin{frame}
    \ftx{Komplexer Spannungsteiler: Anwendungsbeispiel Tastkopf}

    \s{
        Ein weiteres Anwedungsbeispiel für einen ohmsch-kapazitiven Teiler ist der Tastkopf eines Oszilloskops. Solch ein
        Teiler wird in der Abbildung \ref{BildKomplxSpannungsteiler4} vorgestellt. Im Prinzip unterscheid sich das Bild 
        nicht erheblich vom ohmsch-kapazitiven Teiler. Jedoch sind hier das erste und das zweite RC-Glied von einander 
        räumlich getrennt und mit einem Gehäuse umgeben. So steht das erste RC-Glied für den Tastkopf und das zweite 
        RC-Glied für die beschaltung im Oszilloskop. Mit dem ohmsches Teilverhältnis:
        \begin{eq}
            \frac{\underline{U}_\mathrm{A}}{\underline{U}_\mathrm{E}} = \frac{R_\mathrm{2}}{R_\mathrm{1}+R_\mathrm{2}}   
        \end{eq}

        und dem kapazitiven Teilverhältnis:
        \begin{eq}
            \frac{\underline{U}_\mathrm{A}}{\underline{U}_\mathrm{E}} = \frac{C_\mathrm{2}}{C_\mathrm{1}+C_\mathrm{2}}  
        \end{eq}

        kann zum Abgleichen des Tastkopfes das ohmsche und das kapazitive Teilerverhältnis gleichgesetzt werden, da die 
        Teilerverhältniss identisch sein müssen:
        \begin{eq}
            \frac{R_\mathrm{2}}{R_\mathrm{1}+R_\mathrm{2}} = \frac{C_\mathrm{2}}{C_\mathrm{1}+C_\mathrm{2}}      
        \end{eq}

        Die Abgleichbedingung nach Gleichung \ref{GleichungTeiler3} gilt weiterhin. Über die verstellbare Kapazität $C_1$
        wird der Abgleich des Teilers vorgenommen. 

        \fu{
            \resizebox{0.3\textwidth}{!}{
            \begin{circuitikz}
                \draw (0,0) to [short, o-*] (2.25,0) 
                to [short, -o] (4.5,0)
                (0,6) to [short, o-] (2.25,6) to [short] (2.25,5.5)
                (1.25,5.5) to [short] (2.25,5.5) to [short, *-] (3.25,5.5)
                (1.25,3.5) to [short] (2.25,3.5) to [short, *-] (3.25,3.5)
                (1.25,2.5) to [short] (2.25,2.5) to [short, *-] (3.25,2.5)
                (1.25,0.5) to [short] (2.25,0.5) to [short, *-] (3.25,0.5)
                (2.25,3.5) to [short] (2.25,2.5)
                (2.25,0) to [short] (2.25,0.5)
                (2.25,3) to [short, *-o] (4.5,3)
                (1.25,5.5) to [R=$R_\mathrm{1}$] (1.25,3.5)
                (3.25,5.5) to [C=$C_\mathrm{1}$] (3.25,3.5)
                (1.25,2.5) to [R=$R_\mathrm{2}$] (1.25,0.5)
                (3.25,2.5) to [C=$C_\mathrm{2}$] (3.25,0.5);
                \draw[-latex, thick, draw=voltage] (0,5.8) to node[left, color=voltage] {$\underline{U}_\mathrm{E}$} (0,0.2);
                \draw[-latex, thick, draw=voltage] (4.5,2.8) to node[right, color=voltage] {$\underline{U}_\mathrm{A}$} (4.5,0.2);

                \draw [dashed] (0.5,5.75) rectangle (4.25,3.25);
                \draw (3.25,5.5) node [label=above:Tastkopf]{};
                
                \draw [dashed] (0.5,2.75) rectangle (4.25,0.25);
                \draw (1.25,2.5) node [label=above:Oszilloskop]{};
            \end{circuitikz}
            }
        }{{\bf Ein ohmsch-kapazitiver Spannungsteiler als Tastkopf eines Oszilloskops.} Über den Kondensator $C_1$ wird der komplexe 
        Spannungsteiler abgeglichen. \label{BildKomplxSpannungsteiler4}}

    }
        
    \b{
        \begin{minipage}[t]{0.5\textwidth}
        \fu{
            \resizebox{\textwidth}{!}{
            \begin{circuitikz}
                \draw (0,0) to [short, o-o] (4.5,0)
                (0,6) to [short, o-] (2.25,6) to [short] (2.25,5.5)
                (1.25,5.5) to [short] (2.25,5.5) to [short, *-] (3.25,5.5)
                (1.25,3.5) to [short] (2.25,3.5) to [short, *-] (3.25,3.5)
                (1.25,2.5) to [short] (2.25,2.5) to [short, *-] (3.25,2.5)
                (1.25,0.5) to [short] (2.25,0.5) to [short, *-] (3.25,0.5)
                (2.25,3.5) to [short] (2.25,2.5)
                (2.25,0) to [short, *-] (2.25,0.5)
                (2.25,3) to [short, *-o] (4.5,3)
                (1.25,5.5) to [R=$R_\mathrm{1}$] (1.25,3.5)
                (3.25,5.5) to [C=$C_\mathrm{1}$] (3.25,3.5)
                (1.25,2.5) to [R=$R_\mathrm{2}$] (1.25,0.5)
                (3.25,2.5) to [C=$C_\mathrm{2}$] (3.25,0.5);
                \draw[-latex, thick, draw=voltage] (0,5.8) to node[left, color=voltage] {$\underline{U}_\mathrm{E}$} (0,0.2);
                \draw[-latex, thick, draw=voltage] (4.5,2.8) to node[right, color=voltage] {$\underline{U}_\mathrm{A}$} (4.5,0.2);

                \draw [dashed] (0.5,5.75) rectangle (4.25,3.25);
                \draw (3.25,5.5) node [label=above:Tastkopf]{};
                
                \draw [dashed] (0.5,2.75) rectangle (4.25,0.25);
                \draw (1.25,2.5) node [label=above:Oszilloskop]{};
            \end{circuitikz}
            }
        }{}
        \end{minipage}
        \begin{minipage}[t]{0.45\textwidth}
            \begin{itemize}
                \item<1-> Ohmsches Teilverhältnis:
                    \begin{eq}
                        \frac{\underline{U}_\mathrm{A}}{\underline{U}_\mathrm{E}} = \frac{R_\mathrm{2}}{R_\mathrm{1}+R_\mathrm{2}}   \nonumber    
                    \end{eq}
                \item<2-> Kapazitives Teilverhältnis:
                    \begin{eq}
                        \frac{\underline{U}_\mathrm{A}}{\underline{U}_\mathrm{E}} = \frac{C_\mathrm{2}}{C_\mathrm{1}+C_\mathrm{2}}   \nonumber  
                    \end{eq}
                \item<3-> Gleichsetzung von ohmschen und kapazitiven Teilverhältnis:
                    \begin{eq}
                        \frac{R_\mathrm{2}}{R_\mathrm{1}+R_\mathrm{2}} = \frac{C_\mathrm{2}}{C_\mathrm{1}+C_\mathrm{2}}   \nonumber    
                    \end{eq}
                \item<4-> Abgleichbedingung:
                    \begin{eq}
                        R_2 \cdot C_2 = R_1 \cdot C_1   \nonumber    
                    \end{eq}
            \end{itemize}
        \end{minipage}
    }
\end{frame}


\begin{frame}
    \ftx{Komplexer Stromteiler}

    \s{
        Der komplexe Stromteiler, wie er in der Abbildung \ref{BildKomplxStromteiler} vorgestellt wird, teilt den komplexen 
        Gesamtstrom $\underline{I}$ auf in die Teilströme $\underline{I}_\mathrm{1}$ und $\underline{I}_\mathrm{2}$ auf. Die beiden
        Teilströme $\underline{I}_\mathrm{1}$ und $\underline{I}_\mathrm{2}$ durchfließen die Admittanzen $\underline{Y}_\mathrm{1}$ 
        und $\underline{Y}_\mathrm{2}$.

        \fu{
            \resizebox{0.3\textwidth}{!}{
                \begin{circuitikz}
                        \draw (0,0) to [short, o-] (2.25,0)
                        to [short, -*] (2.25,0.5)
                        (1.25,0.5) to [short] (3.25,0.5) 
                        (1.25,0.5) to [R=$\underline{Y}_\mathrm{1}$, i_<, name=I1] (1.25,3.5)
                        (3.25,0.5) to [R=$\underline{Y}_\mathrm{2}$, i_<, name=I2] (3.25,3.5)
                        (1.25,3.5) to [short] (3.25,3.5)
                        (2.25,3.5) to [short, *-] (2.25,4)
                        (0,4) to [short, i, name=I, o-] (2.25,4);
                        \draw[-latex, thick, draw=voltage] (4,3.3) to node[right, color=voltage] {$\underline{U}$} (4,0.7);

                        \iarrmore{I}{$\underline{I}_\mathrm{0}$}; 
                        \iarrmore{I1}{$\underline{I}_\mathrm{1}$};
                        \iarrmore{I2}{$\underline{I}_\mathrm{2}$};  
                    \end{circuitikz}
            }
        }{{\bf Ein komplexer Stromteiler bestehend aus zwei Admittanzen.} Der komplexe Strom $\underline{I}$ teilt sich 
        auf in die Ströme $\underline{I}_\mathrm{1}$ und $\underline{I}_\mathrm{2}$. \label{BildKomplxStromteiler}}

        Die komplexe Spannung $\underline{U}$ lässt sich nach Gleichung \ref{GleichungStromteiler1} über das Verhältnis aus 
        dem Gesamtstrom $\underline{I}$ und der Gesamtadmittanz $\underline{Y}_\mathrm{1} + \underline{Y}_\mathrm{2}$ oder 
        aus den Verhältnissen der einzelnen Stromzweige:

        \begin{eq}
            \underline{U} = \frac{\underline{I}_\mathrm{0}}{\underline{Y}_\mathrm{1}+\underline{Y}_\mathrm{2}} = 
            \frac{\underline{I}_\mathrm{1}}{\underline{Y}_\mathrm{1}} = 
            \frac{\underline{I}_\mathrm{2}}{\underline{Y}_\mathrm{2}}   \label{GleichungStromteiler1}    
        \end{eq}  

        Das Teilerverhältnis lässt sich nach Gleichung \ref{GleichungStromteiler2} für die beiden Stromzweige definieren: 

        \begin{eqa}
            \underline{T}_\mathrm{i1} = \frac{\underline{I}_\mathrm{1}}{\underline{I}_\mathrm{0}} = 
            \frac{\underline{Y}_\mathrm{1}}{\underline{Y}_\mathrm{1}+\underline{Y}_\mathrm{2}}  \label{GleichungStromteiler2}  \\ 
            \underline{T}_\mathrm{i2} = \frac{\underline{I}_\mathrm{2}}{\underline{I}_\mathrm{0}} = 
            \frac{\underline{Y}_\mathrm{2}}{\underline{Y}_\mathrm{1}+\underline{Y}_\mathrm{2}} \label{GleichungStromteiler3}    
        \end{eqa} 
        
    }

    \b{  
        \begin{minipage}[t]{0.4\textwidth}
                \fu{
                    \resizebox{\textwidth}{!}{
                    \begin{circuitikz}
                        \draw (0,0) to [short, o-] (2.25,0)
                        to [short, -*] (2.25,0.5)
                        (1.25,0.5) to [short] (3.25,0.5) 
                        (1.25,0.5) to [R=$\underline{Y}_\mathrm{1}$, i_<, name=I1] (1.25,3.5)
                        (3.25,0.5) to [R=$\underline{Y}_\mathrm{2}$, i_<, name=I2] (3.25,3.5)
                        (1.25,3.5) to [short] (3.25,3.5)
                        (2.25,3.5) to [short, *-] (2.25,4)
                        (0,4) to [short, i, name=I, o-] (2.25,4);
                        \draw[-latex, thick, draw=voltage] (4,3.3) to node[right, color=voltage] {$\underline{U}$} (4,0.7);

                        \iarrmore{I}{$\underline{I}_\mathrm{0}$}; 
                        \iarrmore{I1}{$\underline{I}_\mathrm{1}$};
                        \iarrmore{I2}{$\underline{I}_\mathrm{2}$};  
                    \end{circuitikz}
                    }
                }{}
        \end{minipage}
        \begin{minipage}[t]{0.5\textwidth}
            \begin{itemize}
                \item<1-> Der Gesamtstrom teilt sich im Verhältnis der komplexen Admittanten auf: 
                    \begin{eq}
                        \underline{U} = \frac{\underline{I}_\mathrm{0}}{\underline{Y}_\mathrm{1}+\underline{Y}_\mathrm{2}} = 
                        \frac{\underline{I}_\mathrm{1}}{\underline{Y}_\mathrm{1}} = 
                        \frac{\underline{I}_\mathrm{2}}{\underline{Y}_\mathrm{2}}   \nonumber    
                    \end{eq}    
                \item<2-> Impedanzen anhängig von der Frequenz
                    \begin{eqa}
                        \underline{T}_\mathrm{i1} = \frac{\underline{I}_\mathrm{1}}{\underline{I}_\mathrm{0}} = 
                        \frac{\underline{Y}_\mathrm{1}}{\underline{Y}_\mathrm{1}+\underline{Y}_\mathrm{2}}  \\ 
                        \underline{T}_\mathrm{i2} = \frac{\underline{I}_\mathrm{2}}{\underline{I}_\mathrm{0}} = 
                        \frac{\underline{Y}_\mathrm{2}}{\underline{Y}_\mathrm{1}+\underline{Y}_\mathrm{2}} \nonumber    
                    \end{eqa}
            \end{itemize}
        \end{minipage}
    }
    
\end{frame}

\newpage