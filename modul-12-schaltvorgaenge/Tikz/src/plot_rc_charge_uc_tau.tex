% Aufladevorgang RC Spannung mit Tau
% AxisEnv 1: Spannung am Kondensator
\begin{tikzpicture}[x=1cm,y=1cm]
	\begin{axis}[
		name=AxisSpannung,
		%%%%%%% START SAME OPTIONS %%%%%%%
		xmin=-1, xmax=5.5,
		ymin=-0.125, ymax=1.35,
		xlabel={$t$},
		ylabel={$u(t)$},
		axis x line=center,
		axis y line=center,
		ylabel style={anchor=east, at={(ticklabel cs:0.95)}},
		xtick={1,3,5},
		ytick=\empty,
		xticklabels={$\tau$,3$\tau$,5$\tau$},
		%xticklabel={$\pgfmathprintnumber{\tick}$$\tau$},	%Alternative
		mark=none,
		grid=none,
		width=7cm,
		height=3.5cm,
		%%%%%%% END OF SAME OPTIONS %%%%%%%
		clip=false,
	]
		\addplot[color=black,dashed,	domain=-1:5.4] {1};          			% Asymptote für t gegen unendlich (UQ)
		\addplot[color=voltage,very thick,	domain=-1:0] {0};           		% t<0
		\addplot[color=voltage,very thick,	domain=0:5.4,samples=61] {1-e^(-x)};% t>0 
		\addplot[color=lightgray, very thick] coordinates {(0,0)(1,1)}; 		% Tangente 
		\addplot[color=lightgray, thick, dotted] coordinates {(1,1)(1,0)}; 		% Dotted line
		\node at (1,0.5)[voltage,anchor=west]{$u_C$};
		\node at (1,1)[black,anchor=south west,yshift=-1pt]{$U_q$};
		%\addplot[color=black,dashed, domain=0:1]{x}; % Gradient in t=0, schneidet Asymptote bei t=tau
	\end{axis}
\end{tikzpicture}