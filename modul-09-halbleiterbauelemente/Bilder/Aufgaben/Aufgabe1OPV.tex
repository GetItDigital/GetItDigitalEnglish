\documentclass{standalone}
\input{circuitSettings}
\usepackage{circuitikz}

\begin{document}

\begin{circuitikz}
    
    \draw (0,0) rectangle (2,-3); 
    

    \node[rotate=270] at (0.5,-0.5) {$\triangle$};
    \draw (1.5,-0.5) node {$\infty$};   
    
   
    \node[anchor=west] at (0.1,-1) {$-$};  % Invertierender Eingang
    \node[anchor=west] at (0.1,-1.5) {$+$}; % Nicht-invertierender Eingang
    \node[anchor=west] at (0.1,-2) {$U+$}; % U+ Eingang
    \node[anchor=west] at (0.1,-2.5) {$U-$}; % U- Eingang

    % Ausgang rechts für N1
    \draw (2,-1.5) -- (2.5,-1.5) node[right] {};  % Ausgang

    % Beschriftung oben links für N1
    \draw (0,0) node[above right] {N1};

    % Zeichne ein Rechteck für den OPV N2 (tiefer)
    \draw (0,-6) rectangle (2,-9); % Rechteck für N2
    
    % Zeichne das Symbol für den Operationsverstärker N2 und das Unendlichkeitszeichen
    \node[rotate=270] at (0.5,-6.5) {$\triangle$};
    \draw (1.5,-6.5) node {$\infty$};   % Unendlichkeitszeichen
    
    % Platziere die Eingänge und Beschriftungen am linken Rand innerhalb des OPV N2
    \node[anchor=west] at (0.1,-7) {$-$};  % Invertierender Eingang
    \node[anchor=west] at (0.1,-7.5) {$+$}; % Nicht-invertierender Eingang
    \node[anchor=west] at (0.1,-8) {$U+$}; % U+ Eingang
    \node[anchor=west] at (0.1,-8.5) {$U-$}; % U- Eingang

    % Ausgang rechts für N2
    \draw (2,-7.5) -- (2.5,-7.5) node[right] {};  % Ausgang

    % Beschriftung oben links für N2
    \draw (0,-6) node[above right] {N2};


    \draw (-1,2) node[above]{$15V$} to[short, o-*] (-1,-2) to (-1,-8) to (0,-8);
    \draw (-1,-2) to[*-] (0,-2);

    \draw (-3,2) node[above]{$-15V$} to[short, o-*] (-3, -2.5) to (-3,-8.5) to (0,-8.5);
    \draw (-3,-2.5) to(0,-2.5);

    \draw (-5, -7) to[short, o-*] (-2,-7) to (0,-7)
    (-2,-7) to (-2,-1.5) to (0,-1.5);
    \draw (0,-1) to (-0.3,-1);
    \draw (0,-7.5) to (-0.3,-7.5);

    \draw (-5,-9) to[short, o-] (-5,-9.1)node[ground]{};

    \draw (2.5,-1.5) to (3,-1.5) to[photodiode, -o] node[left] {V1} (3,-3.5) node[right] {$-15V$};  % Photodiode
    \draw (2.5,-7.5) to (3,-7.5) to[photodiode, -o] node[left] {V2} (3,-9.5) node[right] {$-15V$};

    \draw (-5,-7) to[open,v>, name=ua] (-5,-9);

\varrmore{ua}{$U_\mathrm{a}$};

    

\end{circuitikz}

\end{document}
