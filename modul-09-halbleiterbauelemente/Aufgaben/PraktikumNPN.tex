\subsection{Grundverhalten eines npn-Bipolartransistors\label{Aufg35}}
% Alt: Praktikumsaufgabe 8
\Aufgabe{
    Analysieren Sie das Grundverhalten eines Bipolartransistors in einer Basisschaltung.
    Der npn-Transistor weist einen Stromverstärkungsfaktor von $\beta = 100$ auf, 
    es wird ein Basisstrom von $I_\mathrm{B} = 20\,\mathrm{\mu A}$ angenommen. 
    
    \begin{figure}[H]
        \centering
        \begin{tikzpicture}
    \draw
    (0,0) node[npn] (Q1) {}
    (Q1.base) 
    (Q1.collector)  
    (Q1.emitter);

\draw (-3,0) to[R,l=$R_\mathrm{3}$,*-](Q1.B);
\draw (-3,2) to[R,l=$R_1$](-3,0);
\draw (-3,2) -- (-5,2) to[V,name=V1,-*](-5,-2)node[ground]{} to[short,-*](-3,-2) -- (0,-2);
\draw (-3,0) to[R,l=$R_\mathrm{2}$](-3,-2);

\draw (Q1.C) to[R,l=$R_\mathrm{4}$] (0,2.5) -- (2,2.5) to[V,name=V2] (2,-2) -- (0,-2);

\draw (Q1.E) to[short,-*] (0,-2);

\varrmore{V1}{$V_\mathrm{1}$};
\varrmore{V2}{$V_\mathrm{2}$};
\end{tikzpicture}
   
        \label{fig:FigPraktikumNPN}
    \end{figure}
    
    
    \begin{itemize}
        \item[a)] Berechnen Sie den Kollektorstrom $I_\mathrm{C}$.
        \item[b)] Bestimmen Sie die Kollektor-Emitter-Spannung $U_\mathrm{CE}$, wenn der Kollektorwiderstand $R_\mathrm{4} = 1\,\mathrm{k\Omega}$ und die Versorgungsspannung $U_\mathrm{2} = 12\,\mathrm{V}$ beträgt.
    \end{itemize}
}


\Loesung{
    \begin{itemize}
        \item[a)] Berechnung des Kollektorstroms $I_\mathrm{C}$
              Der Kollektorstrom eines Bipolartransistors ergibt sich aus der Beziehung:
              $$ I_\mathrm{C} = \beta \cdot I_\mathrm{B} $$
              mit den gegebenen Werten:
              \begin{equation*}
                  I_\mathrm{C} = 100 \cdot 20\,\mathrm{\mu A} = 2\,\mathrm{mA}
              \end{equation*}
              
              
        \item[b)] Bestimmung der Kollektor-Emitter-Spannung $U_\mathrm{CE}$
              Die Kollektor-Emitter-Spannung ergibt sich aus der Gleichung:
              $$ U_\mathrm{CE} = U_\mathrm{CC} - I_\mathrm{C} \cdot R_\mathrm{4} $$
              Einsetzen der Werte:
              \begin{equation*}
                  U_\mathrm{CE} = 12\,\mathrm{V} - (2\,\mathrm{mA} \cdot 1\,\mathrm{k\Omega}) = 12\,\mathrm{V} - 2\,\mathrm{V}  = 10\,\mathrm{V}
              \end{equation*}
              
    \end{itemize}
    Siehe: Abschnitt \ref{fig:UebersichtBipolartransistoren}
}