\subsection{Superposition 1}
\Aufgabe{
    Gegeben ist unten stehendes Netzwerk.

    \begin{figure}[H]
        \resizebox{0.6\textwidth}{!}{%  
        \begin{circuitikz}
            \draw (0,0) to (0,1) to [I, name=I1] (0,3) to (0,4) to [short, -*] (4,4)
            (0,0) to [short, -*] (4,0);
            \draw (4,2) to [V, name=U1] (4,0) 
            (4,2) to [R=$R_\mathrm{i}$, i, v, name=Ri] (4,4);
            \draw (4,4) to (8,4) to (8,3) 
            to [R=$R_\mathrm{L}$, i, v, name=RL] (8,1) to (8,0) to (4,0);
            
            \varrmore{U1}{$U_\mathrm{g}$};
            \iarrmore{I1}{$I_\mathrm{g}$};
            \varrmore{Ri}{$U_\mathrm{Ri}$};
            \iarrmore{Ri}{$I_\mathrm{i}$};
            \varrmore{RL}{$U_\mathrm{RL}$};
            \iarrmore{RL}{$I_\mathrm{L}$};
        \end{circuitikz}
        }\centering                                                           
    \end{figure}

    Bestimmen Sie mit Hilfe des Superpositionsverfahrens die Spannung, die über dem Widerstand $R_L$ abfällt.
    Befolgen Sie hierzu die folgenden Schritte:
        
    \begin{itemize}
            \item[\bf a)]
                Bestimmen Sie den Anteil von $U_g$.
            \item[\bf b)]
                Bestimmen Sie den Anteil von $I_g$.
            \item[\bf c)]
                Führen Sie die Ergebnisse durch den Überlagerungssatz zusammen.
    \end{itemize}
    }

\Loesung{
	\begin{itemize}
        \item[\bf a)]
        Anteil von $U_g$:\\
        \begin{figure}[H]
            \resizebox{0.6\textwidth}{!}{%  
            \begin{circuitikz}
                \draw (0,0) to [short, -o] (0,1); 
                \draw (0,3) to [short, o-] (0,4) to [short, -*] (4,4)
                (0,0) to [short, -*] (4,0);
                \draw (4,2) to [V, name=U1] (4,0) 
                (4,2) to [R=$R_\mathrm{i}$, i, v, name=Ri] (4,4);
                \draw (4,4) to (8,4) to (8,3) 
                to [R=$R_\mathrm{L}$, i, v, name=RL] (8,1) to (8,0) to (4,0);
                
                \varrmore{U1}{$U_\mathrm{g}$};
                \varrmore{Ri}{$U_\mathrm{Ri}$};
                \iarrmore{Ri}{$I_\mathrm{i}$};
                \varrmore{RL}{$U_\mathrm{RL}$};
                \iarrmore{RL}{$I_\mathrm{L}$};
            \end{circuitikz}
            }\centering                                                           
        \end{figure}

        \begin{eq}
            U_\mathrm{RL}(U_\mathrm{g}) = \frac{R_\mathrm{L}}{R_\mathrm{i} + R_\mathrm{L}} \cdot U_\mathrm{g}       \nonumber
        \end{eq}

		\item[\bf b)]
		Anteil von $I_g$:\\
        \begin{figure}[H]
            \resizebox{0.65\textwidth}{!}{%  
            \begin{circuitikz}
                \draw (0,0) to (0,1) to [I, name=I1] (0,3) to (0,4) to [short, -*] (4,4)
                (0,0) to [short, -*] (4,0);
                \draw (4,2) to [short] (4,0) 
                (4,2) to [R=$R_\mathrm{i}$, i, v, name=Ri] (4,4);
                \draw (4,4) to (8,4) to (8,3) 
                to [R=$R_\mathrm{L}$, i, v, name=RL] (8,1) to (8,0) to (4,0);
                
                \iarrmore{I1}{$I_\mathrm{g}$};
                \varrmore{Ri}{$U_\mathrm{Ri}$};
                \iarrmore{Ri}{$I_\mathrm{i}$};
                \varrmore{RL}{$U_\mathrm{RL}$};
                \iarrmore{RL}{$I_\mathrm{L}$};
            \end{circuitikz}
            }\centering                                                           
        \end{figure}

        \begin{eq}
            U_\mathrm{RL}(I_\mathrm{g}) = \frac{R_\mathrm{i} \cdot R_\mathrm{i}}{R_\mathrm{i} + R_\mathrm{L}} \cdot I_\mathrm{g}       \nonumber
        \end{eq}

        \item[\bf c)]
        Überlagerungssatz:\\

        \begin{eq}
            U_\mathrm{RL} = U_\mathrm{RL}(U_\mathrm{g}) + U_\mathrm{RL}(I_\mathrm{g}) = 
            \frac{R_\mathrm{L}}{R_\mathrm{i} + R_\mathrm{L}} \cdot U_\mathrm{g} + 
            \frac{R_\mathrm{i} \cdot R_\mathrm{i}}{R_\mathrm{i} + R_\mathrm{L}} \cdot I_\mathrm{g}       \nonumber
        \end{eq}
	\end{itemize}
}