% Kapitel 4 - Resonanzkreise, Schwingungen, Ortskurven
%       - Resonanzkreis, Resonanzfrequenz
%       - Schwingungen, Dämpfung
%       - Ortskurven, Pol-Nullstellen-Diagramm


\section{Resonanzkreise}
\label{sec:resonanzkreise}
\begin{frame}\ftx{\secname}%
\index{Resonanzkreis}%
\index{Schwingkreis}
% Einleitung
\s{
    Resonanzkreise, auch Schwingkreise genannt, sind schwingungsfähige Stromkreise, 
    die auf Anregung mit einer bestimmten Frequenz stark reagieren (Resonanzfall).
    Unterschieden werden die zwei Typen des Parallel-Resonanzkreises und Serien-Resonanzkreises.

    Schwingungsfähig ist ein System, wenn es Energie in zwei Formen speichern und zwischen beiden Formen hin und her wandeln kann.
    Resonanzkreise verfügen daher immer über eine Kapazität (elektrischer Speicher) und eine Induktivität (magnetischer Speicher).

    Anwendung finden Resonanzkreise unter anderem im Bereich der Energie-, Mess- und Kommunikationstechnik.
    Dort werden sie zum Beispiel zur Filterung von Netzrückwirkungen oder Signalen eingesetzt.
    
    Resonanzkreise können unter Umständen zu unerwünschten Effekten wie extreme Über- oder Unterspannungen führen. 
}
\begin{Lernziele}{Resonanzkreise}
    Studierende lernen:
    \begin{itemize}
        \item Resonanzkreise zu erkennen und zu berechnen
        \item Resonanzfrequenz, Kennwiderstand und Güte zu bestimmen
        \item Resonanzkurven zu berechnen, zu beschreiben und zu zeichnen
    \end{itemize}
\end{Lernziele}
\end{frame}

%-------------------------------------------------------------------------------------------------%

\subsection{Resonanzerscheinung}
\begin{frame}\ftx{\subsecname}
\index{Resonanzerscheinung}%
\index{Schwingung>freie Schwingung}%
\index{Schwingung>erzwungene Schwingung}%
%Folien: Schaltbild Schwingkreis RLCs frei schwingend und angeregt + Zeitplots (abklingend mit w0 bzw. konst. Amplitude mit wq)
%Skript: Zusätzlich Beschreibung: Wenn Systeme schwingungsfähig sind (Energieumwandlungsfähig mit zwei Speichern), 
%   dann mögliche Resonanz bei Anregung mit Frequenz nahe der Resonanzfrequenz (oder vielfachem davon). 
%   Wenn Resonanzfrequenz gleich Frequenz externer Anregung -> Aufschwingen (Resonanzerscheinung, instabiler Zustand).
\s{% skript only
    Eine Resonanz tritt auf, wenn schwingungsfähige Systeme mit einer Frequenz nahe ihrer Resonanzfrequenz angeregt werden.

    In diesem Abschnitt werden zunächst der Zustand einer freien Schwingung und der einer erzwungener Schwingung beschrieben. 
    In Unterkapitel \ref{sec:resonanzkreise:definition} wird die Bedingung für Resonanz mathematisch festgelegt 
    und im Folgekapitel in Beispiel \ref{bsp:resonanz:lc} für einen Serien- und Parallel-Schwingkreis berechnet.

    % Ein Beispiel aus der Mechanik ist ein Pendel, bei welchem Energie zwischen kinetischer Energie und Lageenergie und und her pendelt. 
    % Schlechtes Beispiel? Frequenz abhängig von Winkel der Auslenkung außer bei Zykloider (?) Bahn ... Vgl. Zykloid Pendel oder so
    % In elektrischen Systemen kann Energie beispielsweise in Induktivitäten (magnetisch) und in Kapazitäten (elektrisch) Energie.

    % Freie Schwingung
    Abbildung \ref{fig:resonanzkreise:rlcs:frei} zeigt exemplarisch das Schaltbild eines RLC-Serienschwingkreises.
    Die Reihenschaltung aus R, L und C ist über den gezeigten Schalter kurzschließbar. 
    Zum Zeitpunkt $t=0$ schließt der Schalter, dabei 
    beträgt die Spannung der Kapazität den Wert $U_0>0$ (Kapazität geladen)
    und es fließt kein Strom $I=0$ (Induktivität entladen).

    \begin{figure}[H]\centering
        \begin{subfigure}{0.4\textwidth}\centering
            \includegraphics{Tikz/pdf/circ_schwingkreis_rlcs_frei.pdf}% Schaltbild
            \caption{Schaltbild}
            \label{fig:resonanzkreise:rlcs:frei:circ}
        \end{subfigure}%
        \begin{subfigure}{0.58\textwidth}\centering
            \resizebox{\textwidth}{!}{\includegraphics{Tikz/pdf/plot_zeitverlauf_RLCs_gedaempft.pdf}}% Plot
            \caption{Freie Schwingung}
            \label{fig:resonanzkreise:rlcs:frei:plot}
        \end{subfigure}
    \caption{RLC-Serienschwingkreis, frei schwingend}
    \label{fig:resonanzkreise:rlcs:frei}
    \end{figure}

    Abbildung \ref{fig:resonanzkreise:rlcs:frei:plot} zeigt den Zeitverlauf der Kapazitätsspannung $u$
    und des Stromes $i$ durch den RLC-Serienschwingkreis ab $t=0$. %(Serie: $I = I_C = I_L = I_R$)
    Der Resonanzkreis befindet sich im Zustand freier Schwingung. 
    Das heißt Ströme und Spannungen oszillieren ohne äußere Anregung mit der \textbf{Eigenfrequenz}\index{Eigenfrequenz} $f_d$ 
    (gedämpft) des Resonanzkreises.

    Die Amplituden klingen aufgrund der Dämpfung ($R>0$) ab. 
    Sie lassen sich jeweils durch eine Exponentialkurve beschreiben, 
    welche den jeweiligen Zeitverlauf einhüllt (gestrichelte Kurve).
    Für genauere Berechnungen von Ausgleichsvorgängen, siehe Modul 12: Schaltvorgänge.% \ref Modul

    Die Energie $E$ pendelt zwischen 
    der elektrischen Form im $\vec{\underline{E}}$-Feld der Kapazität ($E_{el} \sim u^2$) und 
    der magnetischen Form im $\vec{\underline{H}}$-Feld der Induktivität ($E_{mag} \sim i^2$). 
    Anhand der Minima beziehungsweise Maxima und der Nullstellen beider Zeitverläufe ist erkennbar, 
    wie die Energie in jeder Periode zwei mal hin- und zurück gewandelt wird. 

    Aufgrund der reellen (Verlust-)Leistung im ohmschen Widerstand ($R > 0$), 
    wird bei jeder Energieumwandlung ein Teil der Energie in Form von Wärme im Widerstand umgesetzt.
    Der Widerstand dämpft den Schwingkreis und führt zum Abklingen von Strom und Spannung in Zustand freier Schwingung.

    Die Eigenkreisfrequenz $\omega_d = 2\pi f_d$ ist unabhängig von der Amplitude.
    Bei ungedämpften Systemen entspricht sie exakt der \textbf{Resonanzkreisfrequenz} $\omega_0$.
    Bei leicht gedämpften Systemen ist $\omega_d$ geringfügig kleiner als $\omega_0$.
    Stark gedämpfte Systeme schwingen nicht, sondern klingen aperiodisch ab.
    %\index{Eigenfrequenz>ungedämpft}%
    %\index{Eigenfrequenz>gedämpft}%

    % Erzwungene Schwingung
    Abbildung \ref{fig:resonanzkreise:rlcs:erzwungen:circ} zeigt zum Vergleich einen RLC-Serienschwingkreis
    mit Anregung durch eine ideale Spannungsquelle $u_e$ an einer Kapazität (erzwungene Schwingung). 
    Abbildung \ref{fig:resonanzkreise:rlcs:erzwungen:plot} zeigt den eingeschwungenen Zeitverlauf 
    von Strom und Spannung der Kapazität bei sinusförmiger Anregung. 
%    (Achtung! Ideale Spanungsquelle an Kapazität ohne Innenwiderstand mit $i_C \rightarrow \infty$ bei Spannungssprüngen.) 

    \begin{figure}[H]\centering
        \begin{subfigure}{0.4\textwidth}\centering
            \includegraphics{Tikz/pdf/circ_schwingkreis_rlcs_erzwungen.pdf}% Schaltbild
            \caption{Schaltbild}
            \label{fig:resonanzkreise:rlcs:erzwungen:circ}
        \end{subfigure}\hfill%
        \begin{subfigure}{0.58\textwidth}\centering
            \resizebox{\textwidth}{!}{\includegraphics{Tikz/pdf/plot_zeitverlauf_RLCs_angeregt.pdf}}% Plot
            \caption{Erzwungene Schwingung}
            \label{fig:resonanzkreise:rlcs:erzwungen:plot}
        \end{subfigure}
    \caption{RLC-Serienschwingkreis, erzwungene Schwingung}
    \label{fig:resonanzkreise:rlcs:erzwungen}
    \end{figure}

    Strom und Spannung schwingen im eingeschwungenen Zustand mit der Frequenz der Spannungsquelle, 
    welche in diesem Fall auch Erregerfrequenz genannt wird. 

    Entspricht die Erregerfrequenz der Resonanzfrequenz, tritt der \textbf{Resonanzfall} ein.
    Dies kann durch Variation der Erregerfrequenz oder der Bauteilgrößen $L$ oder $C$ geschehen.
}%Ende Skript only
\b{%Start beamer only
\noindent\begin{minipage}[c][4cm][c]{\textwidth}\centering% upper half
    \begin{minipage}[c][4cm][c]{0.25\textwidth}\centering% left
        \resizebox*{!}{0.35\textheight}{\includegraphics{Tikz/pdf/circ_schwingkreis_rlcs_frei.pdf}}%ignores minipage's height
    \end{minipage}\hfill%
    \begin{minipage}[c][4cm][c]{0.38\textwidth}\centering% middle
        Freie Schwingung:\newline
        \resizebox*{\textwidth}{!}{\includegraphics{Tikz/pdf/plot_zeitverlauf_RLCs_gedaempft.pdf}}
    \end{minipage}\hfill%
    \begin{minipage}[c][4cm][c]{0.35\textwidth}% right
        Randbedingung:
        \begin{equation*}\begin{aligned}
            u(t=0) &= U_0\\
            i(t=0) &= 0%\ \mathrm{A}
        \end{aligned}\end{equation*}
        \quad mit Eigenkreisfrequenz $\omega_d$,

        \quad abklingend (gedämpft)
    \end{minipage}
\end{minipage}

\noindent\begin{minipage}[c][4cm][c]{\textwidth}\centering% lower half
    \begin{minipage}[c][4cm][c]{0.25\textwidth}\centering% left
        \resizebox{!}{0.35\textheight}{\includegraphics{Tikz/pdf/circ_schwingkreis_rlcs_erzwungen.pdf}}
    \end{minipage}\hfill
    \begin{minipage}[c][4cm][c]{0.38\textwidth}\centering% middle
        Erzwungene Schwingung:\newline
        \resizebox*{\textwidth}{!}{\includegraphics{Tikz/pdf/plot_zeitverlauf_RLCs_angeregt.pdf}}
    \end{minipage}\hfill%
    \begin{minipage}[c][4cm][c]{0.35\textwidth}% right
        Erregerspannung:
        \begin{equation*}u_e(t) = \hat{U}\sin(\omega_e t)\end{equation*}
        \quad mit Erregerkreisfrequenz $\omega_e$
    \end{minipage}
\end{minipage}
}% Ende Beamer only
\end{frame}
%-------------------------------------------------------------------------------------------------%

\subsection[Definitionen]{Definition Resonanzbedingung, -frequenz, Güte, Kennwiderstand}
\label{sec:resonanzkreise:definition}
\begin{frame}\ftx{\subsecname}
\s{%Start Skript only
    Für Resonanzkreise gilt im Resonanzfall, dass der Imaginärteil der Impedanz gleich null wird.
    Respektive gilt, dass der Imaginärteil der Admittanz gleich null wird 
    sowie Strom und Spannung an den Ausgangsklemmen in Phase liegen.
    % -> Imaginärteil der Leistung null
    % Leistung maximal (Vgl. Impedanzkurven)

    Als Resonanzbedingung gilt an den Ausgangsklemmen des Resonanzkreises:

    \index{Resonanzbedingung}%
    \begin{equation}\begin{aligned}\label{eq:resonanzbedingung}
        \varphi_{u0} - \varphi_{i0} &\overset{!}{=} 0 &&\text{Phasenverschiebung gleich null}\\\leftrightarrow
        \Im\{ \underline{Z}_0 \}    &\overset{!}{=} 0 &&\text{Reaktanz gleich null}
    \end{aligned}\end{equation}

    Frequenzvariable Größen sind im Resonanzfall im Folgenden mit Index $0$ gekennzeichnet:

    \begin{equation*}
        \underline{Z},\ \underline{U},\ \underline{I},\ \varphi_{u},\ \varphi_{i}
        = \underline{Z}_0,\ \underline{U}_0,\ \underline{I}_0,\ \varphi_{u0},\ \varphi_{i0}
        \quad \text{für}\quad \omega=\omega_0
    \end{equation*}

    \index{Resonanzfrequenz}%
    Die \textbf{Resonanzfrequenz} $f_0$ ist die Frequenz, bei der eine Anregung den Resonanzfall erzeugt 
    beziehungsweise die Resonanzbedingung erfüllt. 
    Teilweise wird die \textbf{Resonanzkreisfrequenz} $\omega_0$ auch unpräziserweise als Resonanzfrequenz bezeichnet.
    Allgemein gilt $\omega_0 = 2\pi f_0$.

    \index{Kennwiderstand}%
    \index{Kennleitwert}%
    Bei einfachen RLC-Schwingkreisen hebt sich im Resonanzfall die Reaktanz $X_L$ der Induktivität und $X_C$ der Kapazität gegenseitig auf.
    Betragsmäßig sind diese gleich groß, weshalb deren Betrag im Resonanzfall auch als \textbf{Kennwiderstand} $X_k$ bezeichnet wird.
    Analog dazu ist der \textbf{Kennleitwert} $B_k$ definiert als der Betrag des Leitwertes der Induktivität und der Kapazität im Resonanzfall:

    \begin{align}
        X_k = |X_{L,0}| &= |X_{C,0}|\label{eq:kennwiderstand}\\
        B_k = |B_{L,0}| &= |B_{C,0}| = \frac{1}{X_k}\label{eq:kennleitwert}
    \end{align}

    Kennwiderstand und -leitwert haben die gleichen Einheiten wie Impedanz [$\Omega$] und Admittanz [$\mathrm{S}$]. 

    \index{Güte}%
    Die \textbf{Güte} $Q$, auch Gütefaktor oder Resonanzschärfe genannt, ist eine einheitenlose Größe.
    Sie ist allgemein definiert als das Verhältnis des Blindleistungsbetrages der Induktivität oder der Kapazität 
    zur (Wirk-)Leistung des Schwingkreises im Resonanzfall:

    \begin{align}\label{eq:q:leistung}
        Q &= \frac{|Q_{L,0}|}{P_{R,0}} = \frac{|Q_{C,0}|}{P_{R,0}}&&\text{allgemein}
    \end{align}
    %\intertext{
        Die Güte ist manchmal auch im Speziellen für Reihen- oder für Parallelschwingkreise definiert.

        Bei RLC-Reihenschwingkreisen entspricht die Güte dem Verhältnis von Kennwiderstand $X_k$ zu ohmschem Widerstand $R$,
        bei RLC-Parallelschwingkreisen entspricht die Güte dem Verhältnis von Kennleitwert $B_k$ zu ohmschem Leitwert $G$:
    %}
    \begin{align}
        Q &= \frac{X_k}{R}&&\text{im Reihenschwingkreis}\label{eq:q:kennwiderstand}\\
        Q &= \frac{Y_k}{G}&&\text{im Parallelschwingkreis}\label{eq:q:kennleitwert}\\
    %\end{align} 
    \intertext{
        \index{Überspannung}%
        \index{Überstrom}%
        \index{Resonanz, >Spannungs- (spannungsbezogen)}%
        \index{Resonanz, >Strom- (strombezogen)}%
        Damit beschreibt die Güte auch das Verhältnis möglicher 
        Überspannungen zu Quellenspannungen für Reihenschwingkreise bei Spannungsresonanz beziehungsweise das Verhältnis von 
        Überströmen zu Quellenströmen für Parallelschwingkreise bei Stromresonanz:
    }
    %\begin{align}
        Q &= \frac{U_{L,0}}{U_{q}} = \frac{U_{C,0}}{U_{q}} && \text{bei Spannungsresonanz}\label{eq:q:spannung}\\
        Q &= \frac{I_{L,0}}{I_{q}} = \frac{I_{C,0}}{I_{q}} && \text{bei Stromresonanz}\label{eq:q:strom}
    \end{align}
    
    %Die Güte ist wie in \ref{} gezeigt ist Indikator für die Steilheit des Phasenganges und gibt in Zusammenhang mit dem Kennwiderstand
    %auch Auskufnt für die Dämpfung eines Resonanzkreises.

}%Ende Skript only
\b{%
\begin{columns}
    \column[c]{0.02\textwidth}\rotatebox{90}{\textbf{Grundtypen}}
    \column[c]{0.3\textwidth}

    Reihen-Resonanzkreis
    \resizebox{!}{0.8\textwidth}{\includegraphics{Tikz/pdf/circ_schwingkreis_lcs.pdf}}

    Parallel-Resonanzkreis
    \resizebox{!}{0.8\textwidth}{\includegraphics{Tikz/pdf/circ_schwingkreis_lcp.pdf}}

    \column[c]{0.6\textwidth}
    \begin{align*}
        \textbf{Resonanzbedingung} && \Im\{\underline{Z}\} &\overset{!}{=} 0
            \vphantom{\bigg|}\\
        \textbf{Resonanzfrequenz} && f_0% &\quad\text{mit }\omega_0=2\pi f_0
            \vphantom{\bigg|}\\
        \textbf{Kennwiderstand} && X_k &= |X_{L,0}| = |X_{C,0}|
            \vphantom{\bigg|}\\
        \textbf{Kennleitwert} && B_k &= \frac{1}{X_k}% = |B_{L,0}| = |B_{C,0}|
            \vphantom{\bigg|}\\
        \textbf{Güte} && Q &= \frac{|Q_{L,0}|}{P_{0}} = \frac{|Q_{C,0}|}{P_{0}}
            \vphantom{\bigg|}\\
        %\textbf{seriell, Güte} && Q_s &= \frac{X_k}{R} = \frac{U_{L,0}}{U}
        %    \vphantom{\bigg|}\\
        %\textbf{parallel, Güte} && Q_p &= \frac{Y_k}{G} = \frac{I_{L,0}}{I}
        %    \vphantom{\bigg|}\\
        \textbf{Eigenfrequenz}&& f_d &\quad\text{(gedämpft)}
            \vphantom{\bigg|}\\
        \textbf{Eigenfrequenz}&& f_0 &\quad\text{(ungedämpft)}
            \vphantom{\bigg|}
    \end{align*}

\end{columns}
}
\end{frame}

%-------------------------------------------------------------------------------------------------%

\subsection{Resonanzfrequenz am Beispiel idealer LC-Schwingkreise}
\begin{frame}\ftx{\subsecname}
\index{Schwingkreis>LC-Reihenschwingkreis}%
\index{Schwingkreis>LC-Parallelschwingkreis}%
\s{%Start Skript only
    In Beispiel \ref{bsp:resonanz:lc} wird exemplarisch die Resonanzkreisfrequenz 
    für ideale LC-Serien- und LC-Parallel-Schwingkreise bestimmt.

    Anhand der Beispiele wird rechnerisch verdeutlicht, weshalb in Reihenschwingkreisen
    eine Resonanz durch Spannungen und in Parallelschwingkreisen durch Ströme hervorgerufen wird.
    Zur Unterscheidung wird deshalb auch von einer \textbf{spannungsbezogenen} und 
    einer \textbf{strombezogenen} Resonanz gesprochen. % 

\begin{bsp}{Resonanz in LC-Schwingkreisen}{resonanz:lc}
    Berechnung der Resonanzfrequenz $\omega_0$ für LC-Serien- und LC-Parallelschwingkreise.
    \begin{figure}[H]\centering
        \begin{subfigure}{0.48\textwidth}\centering
            \resizebox{6cm}{!}{\includegraphics{Tikz/pdf/circ_schwingkreis_lcs_komplex.pdf}}% textwidth different as below
            \captionsetup{labelformat=empty}
            \caption{LC-Serienschwingkreis}
            \label{fig:circ:lcs}
        \end{subfigure}\hfill%
        \begin{subfigure}{0.48\textwidth}\centering
            \resizebox{6cm}{!}{\includegraphics{Tikz/pdf/circ_schwingkreis_lcp_komplex.pdf}}% textwidth different as above
            \captionsetup{labelformat=empty}
            \caption{LC-Parallelschwingkreis}
            \label{fig:circ:lcp}
        \end{subfigure}
    \end{figure}

    \begin{minipage}{\textwidth}\centering%Beide Spalten
        \begin{minipage}{0.48\textwidth}\centering%Linke Spalte
            Bei Anregung mit Spannung $\underline{U}$:
            \begin{align*}
                \underline{U}
                &= \underline{I} \cdot \left(\mathrm{j}\omega L + \frac{1}{\mathrm{j}\omega C}\right) \\
                &= \underline{I} \cdot \mathrm{j}\bigg(\underbrace{\omega L - \frac{1}{\omega C}}_{=0\ \text{Resonanz}}\bigg) \\
                &\omega_0 L - \frac{1}{\omega_0 C} = 0 \\
                \Rightarrow \underline{I} &\rightarrow \infty \quad \text{für} \quad \underline{U} = konst.
            \end{align*}
            \textbf{Spannungsbezogene Resonanz}\newline
            im Serienschwingkreis
        \end{minipage}\hfill%
        \begin{minipage}{0.48\textwidth}\centering%Rechte Spalte
            Bei Anregung mit Strom $\underline{I}$:
            \begin{align*}
                \underline{I}
                &= \underline{U} \cdot \left(\mathrm{j}\omega C + \frac{1}{\mathrm{j}\omega L}\right) \\
                &= \underline{U} \cdot \mathrm{j} \bigg(\underbrace{\omega C - \frac{1}{\omega L}}_{=0\ \text{Resonanz}}\bigg)\\
                &\omega_0 C - \frac{1}{\omega_0 L} = 0 \\
                \Rightarrow \underline{U} &\rightarrow \infty\quad \text{für} \quad \underline{I} = konst.
            \end{align*}
            \textbf{Strombezogene Resonanz}\newline
            im Parallelschwingkreis
        \end{minipage}

        \begin{equation*}
            \omega_0 = \frac{1}{\sqrt{LC}}
        \end{equation*}
        Die \textbf{Resonanzfrequenz} ist für beide Schwingkreise gleich.
    \end{minipage}
\end{bsp}

%Dies liegt daran, dass sich bei Reihen-Anordnung die Spannungen und bei Parallel-Anordnung die Ströme jeweils über/durch $L$ und $C$ gegenseitig aufheben.
Im Resonanzfall heben sich im idealen Reihenschwingkreis die Spannungen über $L$ und $C$ gegenseitig auf.
Daher können theoretisch beliebig hohe Spannungen über $L$ und $C$ entstehen.
In realen Schhaltungen wird die Spannung jedoch durch den ohmschen Widerstand $R$ begrenzt. 
[Vgl. RLC-Schwingkreis in Kapitel \ref{sec:resonanzkreise:rlcs}]

Analog gilt das gleiche im Resonanzfall für die Ströme durch $L$ und $C$ im Parallelschwingkreis.

Die Resonanzfrequenz einfacher RLC-Schwingkreise ist identisch mit der idealer LC-Schwingkreise,
da der Widerstand $R$ sich nicht auf den Blindanteil der Impedanz auswirkt,
mit $R$, $L$ und $C$ alle in Reihe oder alle parallel.

}%Ende Skript only
\b{%Start Beamer only
\noindent\begin{minipage}{0.5\textwidth}\centering%Linke Spalte
    \begin{tabular}{l|c}
        \rotatebox[origin=l]{90}{\quad in Reihe}
        &\resizebox{!}{0.4\textheight}{\includegraphics{Tikz/pdf/circ_schwingkreis_lcs_komplex.pdf}}
        \\\rotatebox[origin=c]{90}{spannungsbezogen}
        &$\begin{aligned}
            \underline{U}
            &= \underline{I} \cdot \left(\mathrm{j}\omega L + \frac{1}{\mathrm{j}\omega C}\right) \\
            &= \underline{I} \cdot \mathrm{j}\bigg(\underbrace{\omega L - \frac{1}{\omega C}}_{=\mathrlap{\,0\ \mathrm{Resonanz}}}\bigg) \\
            &\omega_0 L - \frac{1}{\omega_0 C} = 0 \\
            \Rightarrow \underline{I} &\rightarrow \infty \quad \text{für} \quad \underline{U} = konst.
        \end{aligned}$
    \end{tabular}
\end{minipage}\hfill%
\noindent\begin{minipage}{0.5\textwidth}\centering%Rechte Spalte
    \begin{tabular}{c|r}
        \resizebox{!}{0.4\textheight}{\includegraphics{Tikz/pdf/circ_schwingkreis_lcp_komplex.pdf}}
        &\rotatebox[origin=l]{90}{\quad parallel}
        \\$\begin{aligned}
            \underline{I}
            &= \underline{U} \cdot \left(\mathrm{j}\omega C + \frac{1}{\mathrm{j}\omega L}\right) \\
            &= \underline{U} \cdot \mathrm{j} \bigg(\underbrace{\omega C - \frac{1}{\omega L}}_{=\mathrlap{\,0\ \mathrm{Resonanz}}}\bigg)\\
            &\omega_0 C - \frac{1}{\omega_0 L} = 0 \\
            \Rightarrow \underline{U} &\rightarrow \infty\quad \text{für} \quad \underline{I} = konst.
        \end{aligned}$
        &\rotatebox[origin=c]{90}{strombezogen}
    \end{tabular}
\end{minipage}\vspace{-2.5mm}
\begin{equation*}
    \omega_0 = \frac{1}{\sqrt{LC}}
\end{equation*}
}%Ende Beamer only
\end{frame}

%-------------------------------------------------------------------------------------------------%

\subsection{RLC-Reihenschwingkreis - Resonanzverhalten}\label{sec:resonanzkreise:rlcs}
\index{Schwingkreis>RLC-Reihenschwingkreis}%
\begin{frame}\ftx{\subsecname}
\b{%
\begin{minipage}{0.4\textwidth}%
\resizebox{\textwidth}{!}{\includegraphics{Tikz/pdf/circ_schwingkreis_rlcs_komplex_2.pdf}}\newline

\begin{tcolorbox}[colback=white,colframe=GETgreen!25!black,toprule=0.5pt,rightrule=0.5pt,bottomrule=0.5pt,leftrule=0.5pt,
    title=\textbf{Resonanzbedingung (allg.)},center title,colbacktitle=GETgreen!5!white,coltitle=black,titlerule=0.5pt,left=2pt,right=2pt]
    \centering
    Phasenverschiebung zwischen\newline
    $\underline{U}$ und $\underline{I}$ entspricht $\varphi = 0$.\newline
    Oder:
    \begin{equation*}
        \Im\left\{\frac{\underline{U}}{\underline{I}}\right\} =
        \Im\left\{\underline{Z}\right\} \overset{!}{=} 0
    \end{equation*}
\end{tcolorbox}
\end{minipage}\hfill%
\begin{minipage}{0.55\textwidth}
    \begin{equation*}
        \underline{Z} = R + \mathrm{j}\Big(\underbrace{\omega L - \frac{1}{\omega C}}_{=\mathrlap{\ 0\quad\text{für}\ \omega=\omega_0}}\Big)
    \end{equation*}
    \begin{tcolorbox}[colback=white,colframe=GETgreen!25!black,
        toprule=0.5pt,bottomrule=0.5pt,rightrule=0.5pt,leftrule=0.5pt,
        top=2pt,bottom=2pt,left=2pt,right=2pt,middle=2pt]
        Fall 1: $\mathbf{\displaystyle{\omega < \omega_0}}$ 
        \tcblower
        $\rightarrow\left(\omega L - \frac{1}{\omega C}\right) < 0$ \hfill ohmsch-kapazitiv
    \end{tcolorbox}

    \begin{tcolorbox}[colback=white,colframe=GETgreen!25!black,
        toprule=0.5pt,bottomrule=0.5pt,rightrule=0.5pt,leftrule=0.5pt,
        top=2pt,bottom=2pt,left=2pt,right=2pt,middle=2pt]
        Fall 2: $\mathbf{\displaystyle{\omega = \omega_0}}$ \hfill (Resonanzfall)
        \tcblower
        $\rightarrow\left(\omega L - \frac{1}{\omega C}\right) = 0$ \hfill rein ohmsch
    \end{tcolorbox}

    \begin{tcolorbox}[colback=white,colframe=GETgreen!25!black,
        toprule=0.5pt,bottomrule=0.5pt,rightrule=0.5pt,leftrule=0.5pt,
        top=2pt,bottom=2pt,left=2pt,right=2pt,middle=2pt]
        Fall 3: $\mathbf{\displaystyle{\omega > \omega_0}}$ 
        \tcblower
        $\rightarrow\left(\omega L - \frac{1}{\omega C}\right) > 0$ \hfill ohmsch-induktiv
    \end{tcolorbox}
\end{minipage}
}%
\s{%
    In diesem Unterkapitel wird exemplarisch das Resonanzverhalten eines RLC-Serienschwingkreis untersucht.
    Der Schwingkreis ist in Abbildung \ref{fig:circ:rlcs} gezeigt und entspricht dem idealen LC-Serienschwingkreis
    aus Beispiel \ref{bsp:resonanz:lc} mit einem zusätzlichen ohmschen Widerstand $R$ in Reihe.

    \fu{\includegraphics{Tikz/pdf/circ_schwingkreis_rlcs_komplex_2.pdf}}{Schaltbild RLC-Serienschwingkreis\label{fig:circ:rlcs}}

    Die Impedanz $\underline{Z}$ des RLC-Serienschwingkreis aus Abbildung \ref{fig:circ:rlcs} ist gegeben durch:

    \begin{align}
        \underline{U} &= \underline{I} \cdot \left(R + \mathrm{j}\omega L - \mathrm{j}\frac{1}{\omega C}\right)\\
        \underline{Z} &= R + \mathrm{j}\left(\omega L - \frac{1}{\omega C}\right)\label{eq:rlcs:z}
    \end{align}
    
    Im Resonanzfall verschwindet nach Gleichung \ref{eq:resonanzbedingung} der Imaginärteil der Impedanz.
    Foglich heben sich bei dieser Anordnung die Reaktanzen von Induktivität und Kapazität auf.
    Durch einsetzen von $\underline{Z}$ in die Resonanzbedingung lässt sich die Resonanzfrequenz $\omega_0$ bestimmen:

    \begin{equation}\label{eq:rlcs:w0}
        \begin{aligned}
            \Im\{\underline{Z}\} = \left(\omega L - \frac{1}{\omega C}\right) &\overset{!}{=} 0 \\
            \omega_0 L - \frac{1}{\omega_0 C} &= 0 &&\Rightarrow& \omega_0 &= \frac{1}{\sqrt{LC}}
        \end{aligned}
    \end{equation}

    Da die Reaktanz der Induktivität positiv und proportional zur Frequenz ist ($X_L \sim \omega$),
    und da die Reaktanz der Kapazität negativ und proportional zum Kehrwert der Frequenz ist ($-X_C \sim \frac{1}{\omega}$),
    überwiegt für Frequenzen oberhalb der Resonanzfrequenz der induktive Anteil und für Frequenzen unterhalb der kapazitive Anteil.

    Daraus ergibt sich die folgende Fallunterscheidung für den RLC-Serienschwingkreis:

    \begin{equation}
        \Im\{\underline{Z}\} =
    \begin{cases}
        \omega L - \frac{1}{\omega C} < 0 & \qquad \omega < \omega_0 \qquad\text{ohmsch-kapazitiv}\vphantom{\Big|}\\
        \omega L - \frac{1}{\omega C} = 0 & \qquad \omega = \omega_0 \qquad\text{rein ohmsch}\vphantom{\Big|}\\
        \omega L - \frac{1}{\omega C} > 0 & \qquad \omega > \omega_0 \qquad\text{ohmsch-induktiv}\vphantom{\Big|}
    \end{cases}
    \label{eq:rlcs:fallunterscheidung}
    \end{equation}

    Die Fallunterscheidung wird in Abbildung \ref{fig:plot:rlcs:zeigerdiagramm} anhand eines Zeigerdiagramms 
    und in Abbildung \ref{fig:plot:rlcs:impedanzkurve} anhand einer Impedanzkurve verdeutlicht.

    \textbf{Exkurs: Freie Schwingung als Grenzfall der Resonanz} % Sonanz (Hall), Solsonanz (Alleinhall?), Suisonanz (Selbsthall)
    
    Die Bezeichnung Resonanz, von lateinisch \textit{resonare} (widerhallen), 
    bezieht sich, der Wortbedeutung entsprechend, immer auf Systemzustände bei externer Anregung.

    Wie in Abbildung \ref{fig:resonanzkreise:rlcs:frei} gezeigt 
    schwingt ein geladener RLC-Serienschwingkreis bei Kurzschluss der äußeren Anschlussklemmen im Zustand freier Schwingung mit dessen Eigenfrequenz.
    
    Der Kurzschluss kann als ideale Spannungsquelle (Innenwiderstand $R_i=0$) mit Spannung $U=0$ betrachtet werden.
    Die Eigenfrequenz (mit Dämpfung) ergibt sich aus der Lösung der homogenen Differentialgleichung des Schwingkreises. 
    Mehr dazu in Modul 12: Schaltvorgänge.%TODO VERWEIS MODUL
        
    Knoten- und Maschengleichungen:

    \begin{align*}
        u(t) &= u_R + u_L + u_C \overset{!}{=} 0 \\
        i(t) &= i_R = i_L = i_C
    \end{align*}
  
    DGL:

    \begin{align*}u(t)=
        i R + L \frac{\d i}{\d t} + \frac{1}{C} \int i \d t &\overset{!}{=} 0 \\
        L \frac{\d^2i}{\d t^2} + R \frac{\mathrm{d}i}{\d t} + \frac{1}{C} i &\overset{!}{=} 0 
    \end{align*}

    Ansatz:

    \begin{align*}
        i(t) 
            &= k \mathrm{e}^{\lambda t}&%\rightarrow
        \frac{\d i}{\d t}
            &= k\lambda \mathrm{e}^{\lambda t}&%\rightarrow
        \frac{\d^2 i}{\d t^2}
            &= k\lambda^2 \mathrm{e}^{\lambda t}
    \end{align*}
    
    Einsetzen:

    \begin{align*}
        L \frac{\d^2i}{\d t^2} + R \frac{\mathrm{d}i}{\d t} + \frac{1}{C} i &= 0 \\
        k \cdot \bigg( \underbrace{L \lambda^2 + R \lambda + \frac{1}{C}}_{=\mathclap{\ 0}} \bigg) \mathrm{e}^{\mathrm{j} \lambda t} &= 0
    \end{align*}

    Auflösen:
    \begin{equation*}
        \lambda_{1/2} = -\frac{R}{2L} \pm \mathrm{j} \sqrt{\frac{1}{LC} - \left(\frac{R}{2L}\right)^2}
    \end{equation*}

    Wie in Modul 12 näher erläutert wird, ergibt sich für den Fall, dass $\frac{1}{LC} > \left(\frac{R}{2L}\right)^2$ (geringe Dämpfung) 
    eine komplexe Lösung. Nur in diesem Fall kann der Resonanzkreis frei schwingen mit, wobei
    der Realteil der Lösung die Abklingkonstante und der Imaginärteil (Wurzelterm) die Eigenkreisfrequenz $\omega_d$ definieren:

    \begin{equation*}
        \omega_{d} = \sqrt{\frac{1}{LC} - \left(\frac{R}{2L}\right)^2}
    \end{equation*}
}
\end{frame}

%.................................................................................................%

\subsubsection{Fallunterscheidung im Zeigerdiagramm}
\begin{frame}\ftx{\subsubsecname}
\b{%
\begin{minipage}{0.4\textwidth}%
\resizebox{0.825\textwidth}{!}{\includegraphics{Tikz/pdf/circ_schwingkreis_rlcs_komplex_2_const_i.pdf}}
\end{minipage}\hfill%
\begin{minipage}{0.6\textwidth}
\begin{align*}\underline{U}
    &= \underline{U}_R\ +\ \underline{U}_L\ +\ \underline{U}_C&
        \underline{I} &= konst.\\
    &= \underline{I} \cdot \big[R + \mathrm{j}\big(\underbrace{\omega L - \frac{1}{\omega C}}_{=\mathrlap{\ \Im\{\underline{Z}\}}}\big)\big]&
        \omega_0 &= \frac{1}{\sqrt{LC}}
\end{align*}
\end{minipage}\vfill
\begin{minipage}{\textwidth}\centering
    \begin{minipage}{0.3\textwidth}\centering \rotatebox{90}{Fall 1 \quad ohmsch-kapazitiv}
    \resizebox{!}{0.5\textheight}{\includegraphics{Tikz/pdf/plot_zeigerdiagramm_rlcs_2.pdf}}\newline $\qquad\omega < \omega_0$
    \end{minipage}\hfill%
    \begin{minipage}{0.3\textwidth}\centering \rotatebox{90}{Fall 2 \quad rein ohmsch}
    \resizebox{!}{0.5\textheight}{\includegraphics{Tikz/pdf/plot_zeigerdiagramm_rlcs_3.pdf}}\newline $\qquad\omega = \omega_0$
    \end{minipage}\hfill%
    \begin{minipage}{0.3\textwidth}\centering \rotatebox{90}{Fall 3 \quad ohmsch-induktiv}
    \resizebox{!}{0.5\textheight}{\includegraphics{Tikz/pdf/plot_zeigerdiagramm_rlcs_1.pdf}}\newline $\qquad\omega > \omega_0$
    \end{minipage}
\end{minipage}
}%
\s{%
Abbildung \ref{fig:plot:rlcs:zeigerdiagramm} zeigt für alle drei Fälle exemplarisch ein Zeigerdiagramm.
Gezeigt sind die entstehenden Spannungen über $R$, $L$ und $C$, sowie die Gesamtspannung $\underline{U}$ bei einer Anregung mit konstantem Strom $\underline{I}$.
Die maßstabsgetreuen Längenverhältnisse $\frac{U_L}{U_C}$ entsprechen: $\frac{1}{2}$, $\frac{\sqrt{2}}{\sqrt{2}}$, $\frac{2}{1}$.
%für die Frequenzen $\omega = \frac{\omega_0}{2}$, $\omega = \omega_0$ und $\omega = 2\omega_0$.

\begin{figure}[H]\centering
    \captionsetup[subfigure]{justification=centering,}%
    \begin{subfigure}{0.3\textwidth}\centering
        \includegraphics{Tikz/pdf/plot_zeigerdiagramm_rlcs_2.pdf}
        \caption{Fall 1: $\omega < \omega_0$\newline(ohmsch-kapazitiv)}
    \end{subfigure}\hfill%
    \begin{subfigure}{0.3\textwidth}\centering
        \includegraphics{Tikz/pdf/plot_zeigerdiagramm_rlcs_3.pdf}
        \caption{Fall 2: $\omega = \omega_0$\newline(rein ohmsch)}
    \end{subfigure}\hfill%
    \begin{subfigure}{0.3\textwidth}\centering
        \includegraphics{Tikz/pdf/plot_zeigerdiagramm_rlcs_1.pdf}
        \caption{Fall 3: $\omega > \omega_0$\newline(ohmsch-induktiv)}
    \end{subfigure}
    \s{\caption{Zeigerdiagramm RLC-Serienschwingkreis}}
    \label{fig:plot:rlcs:zeigerdiagramm}
\end{figure}

Da es sich um einen passiven Schwingkreis handelt (Verbraucherpfeilsystem), 
ist die Wirkleistung $P$ des Schwingkreis immer positiv mit $P=\Re\{\underline{U}\cdot\underline{I}\}$.

Bei reell eingeprägtem Strom $\underline{I}$ ist der Realanteil der Gesamtspannung $\underline{U}$ daher immer positiv.
Der Zeiger der Gesamtspannung $\underline{U}$ deutet also entweder in den ersten Quadranten (ohmsch-induktiv), 
in den vierten Quadranten (ohmsch-kapazitiv) oder auf die reale Achse (rein ohmsch).
}
\end{frame}

%.................................................................................................%

% TODO: extra frame Impedanzkurve (basic, Werte ablesen, ohne Güte)

\subsubsection{Fallunterscheidung in Impedanzkurve}
\begin{frame}\ftx{\subsubsecname}
\s{%
Für den Reihenschwingkreis aus Abbildung \ref{fig:circ:rlcs} ergibt sich die Impedanz zu:
\begin{equation}
    \begin{aligned}\underline{Z}
        &= R + \mathrm{j}\left(X_L + X_C\right) \\
        &= R + \mathrm{j}\left(\omega L - \frac{1}{\omega C}\right)
    \end{aligned}
\end{equation}

Die kapazitive Reaktanz (hyperbolisch) überwiegt für Frequenzen unterhalb der Resonanzfrequenz, 
die induktive Reaktanz (linear) überwiegt für Frequenzen oberhalb der Resonanzfrequenz
wie in Gleichung \ref{eq:rlcs:fallunterscheidung} gezeigt ist. 

Abbildung \ref{fig:plot:rlcs:impedanzkurve} zeigt die Impedanzkurve eines RLC-Serienschwingkreises
in Abhängigkeit der Frequenz bezogen auf die Resonanzfrequenz $\sfrac{f}{f_0}$ äquivivalent $\sfrac{\omega}{\omega_0}$.
Gezeigt sind die Beträge der Impedanz $|\underline{Z}|$, der Reaktanz $|X|$,
der Reaktanz der Induktivität $|X_L|$, der Reaktanz der Kapazität $|X_C|$ und des ohmschen Widerstandes $R$.

\begin{figure}[H]\centering
    \includegraphics{Tikz/pdf/plot_rlcs_impedanzkurve.pdf}
    \caption[Impedanzkurve RLC-Serienschwingkreis]{Impedanzkurve RLC-Serienschwingkreis\newline mit $R=10\ \Omega$, $L=10\ \mathrm{mH}$, $C=10\ \mu\mathrm{F}$}
    \label{fig:plot:rlcs:impedanzkurve}
\end{figure}

Wie in der Impedanzkurve zu erkennen ist, ist die Impedanz im Resonanzfall rein ohmsch und entspricht dem ohmschen Widerstand $R$:
\begin{equation}\label{eq:rlcs:z0}
    \underline{Z}_0 = R
\end{equation}

Die Reaktanz der Induktivität $X_L$ (positiv) und die Reaktanz der Kapazität $X_C$ (negativ) heben sich im Resonanzfall auf (Schnittpunkt der Kurven).
Die Beträge beider Reaktanzen sind im Resonanzfall gleich und entsprechen dem Kennwiderstand $X_k$:

\begin{equation}\label{eq:rlcs:xk}
    \begin{aligned}
        X_k &= |X_{L,0}| = |X_{C,0}| = \sqrt{\frac{L}{C}} \\
        &= \omega_0 L \ \ = \frac{1}{\omega_0 C} 
    \end{aligned}
\end{equation}

Zu erkennen ist noch, dass der Betrag der Impedanz sich für sehr niedrige Frequenzen $\omega << \omega_0$ 
dem kapazitiven Verlauf annähert mit $\lim_{\omega\rightarrow 0}|Z|=|X_C|$ und für sehr hohe Frequenzen $\omega >> \omega_0$
dem induktiven Verlauf annähert mit $\lim_{\omega\rightarrow \infty}|Z|=|X_L|$.
}
\b{%
\begin{minipage}{0.3\textwidth}%
\resizebox{1.1\textwidth}{!}{\includegraphics{Tikz/pdf/circ_schwingkreis_rlcs_komplex_2.pdf}}
\begin{align*}
    \intertext{\small Resonanzbedingung}
    \onslide<1->{\Im\{\underline{Z}\} &\overset{!}= 0}\\
    \intertext{\small Resonanzfrequenz}
    \onslide<1->{\omega_0 &= \frac{1}{\sqrt{LC}}}\\
    \intertext{\small Resonanzimpedanz}
    \onslide<1->{\underline{Z}_0 &= R}
\end{align*}
\end{minipage}\hfill%
\begin{minipage}{0.65\textwidth}\centering
    \begin{equation*}
        \smash{\underline{Z} = R + \mathrm{j}\left(\omega L - \frac{1}{\omega C}\right)}
    \end{equation*}
    \resizebox{\textwidth}{!}{\includegraphics{Tikz/pdf/plot_rlcs_impedanzkurve.pdf}}
    \begin{equation*}
        \smash{X_k = |X_{L,0}| = |X_{C,0}| = \sqrt{\frac{L}{C}}}
    \end{equation*}
\end{minipage}
}
\end{frame}

%.................................................................................................%

\subsubsection[Resonanzkurve und Gütefaktor]{Resonanzkurve und Gütefaktor eines RLC-Reihenschwingkreises}
\begin{frame}[t]\ftx{\subsubsecname}
\s{%
Die Impedanz eines RLC-Serienschwingkreises lässt wie in Gleichung \ref{eq:rlcs:z} über die 
Bauteilgrößen $R$, $L$ und $C$ in Abhängigkeit der Frequenz $\omega$ ausdrücken:

\begin{equation}
    \underline{Z} = R + \mathrm{j}\left(\omega L - \frac{1}{\omega C}\right)
\end{equation}

Mithilfe der Resonanzkreisfrequenz aus Gleichung \ref{eq:rlcs:w0} kann die Kreisfrequenz auf diese normiert werden.
Hierfür werden die $\omega$ der Reaktanz-Terme mit $\omega_0$ erweitert ($\omega=\omega\frac{\omega_0}{\omega_0}$).
Dadurch lassen sich die Reaktanzen $X_L$ und $X_C$ mithilfe des Kennwiderstandes $X_k$ aus Gl. \ref{eq:rlcs:xk} ausdrücken:

\begin{equation}\label{eq:rlcs:zxk}
    \begin{aligned}
    \underline{Z}
    &= R + \mathrm{j}\left( \omega L \frac{\omega_0}{\omega_0} - \frac{1}{\omega C}\frac{\omega_0}{\omega_0}\right)\\
    &= R + \mathrm{j}\left( X_{L,0} \frac{\omega}{\omega_0} + X_{C,0} \frac{\omega}{\omega_0}\right)\\
    &= R + \mathrm{j}\left( \frac{\omega}{\omega_0} - \frac{\omega_0}{\omega}\right) X_k
    \end{aligned}
\end{equation}

Dabei gilt:

\begin{align}
    X_L &= +X_k \frac{\omega}{\omega_0} &= +\sqrt{\frac{L}{C}}\cdot\frac{\omega \sqrt{LC}}{1} &= \omega L &&\text{(linear)}\label{eq:rlcs:xlw0}\\
    X_C &= -X_k \frac{\omega_0}{\omega} &= -\sqrt{\frac{L}{C}}\cdot\frac{1}{\omega \sqrt{LC}} &= -\frac{1}{\omega C} &&\text{(hyperbolisch)}\vphantom{\Big|}\label{eq:rlcs:xcw0}\\
    X_k &= \sqrt{\frac{L}{C}}&\text{mit}\quad\omega_0 &= \frac{1}{\sqrt{LC}} &&\text{(konstant)}\label{eq:rlcs:xkw0}
\end{align}

\index{Güte}%
Wie in Gleichung \ref{eq:q:kennwiderstand} konstatiert, ist die Güte $Q$ eines Reihenschwingkreises 
das Verhältnis dessen Kennwiderstandes $X_k$ zum dessen ohmschen Widerstand $R$. 
Dadurch lässt sich die Impedanz des Schwingkreises normiert auf dessen Kennwiderstand $X_k$ mithilfe der Güte $Q$ ausdrücken:

\begin{equation}\label{eq:rlcs:znorm}
    \frac{\underline{Z}}{X_k} = \frac{1}{Q} + \mathrm{j}\left(\frac{\omega}{\omega_0} - \frac{\omega_0}{\omega}\right) \vphantom{\Big|}
    \qquad\text{mit}\quad Q = \frac{X_k}{R}
\end{equation}

\index{relative Verstimmung}%
Der Term $\frac{\omega}{\omega_0} - \frac{\omega_0}{\omega}$ wird auch als relative Verstimmung $\nu_r$ bezeichnet.\cite{hagmann}

Abbildung \ref{fig:plot:rlcs:resonanzkurve:impedanz} zeigt exemplarisch mehrere Resonanzkurven eines RLC-Serienschwingkreises für unterschiedliche Gütefaktoren $Q$ zum Vergleich.
Die y-Achse zeigt den Impedanzbetrag normiert auf den Kennwiderstand $X_k$ und die x-Achse zeigt die Frequenz normiert auf die Resonanzfrequenz $\omega_0$. 

\fu{\includegraphics{Tikz/pdf/plot_rlcs_impedanzkurve_guete.pdf}}{Resonanzkurve der Impedanz, RLC-Serienschwingkreis, Vergleich Güten\label{fig:plot:rlcs:resonanzkurve:impedanz}}
%\begin{figure}[H]\centering
%    \begin{subfigure}{\textwidth}\centering
%        \includegraphics{Tikz/pdf/plot_rlcs_impedanzkurve_guete_noxaxis.pdf}
%    \end{subfigure}
%    \begin{subfigure}{\textwidth}\centering
%        \includegraphics{Tikz/pdf/plot_rlcs_phikurve_guete.pdf}
%    \end{subfigure}
%    \caption{Resonanzkurve der Impedanz und Phase eines RLC-Serienschwingkreis\newline Vergleich von Güten}
%    \label{fig:plot:rlcs:resonanzkurve}
%\end{figure}
Im Bereich der Resonanzfrequenz zeigt sich der deutlichste Unterschied der einzelnen Impedanzkurven für unterschiedliche Gütefaktoren.
Wie in Gleichung \ref{eq:rlcs:z0} gezeigt ist, gilt im Resonanzfall: $Z_0 = R$.

Eine hohe Güte bedeutet, dass der Schwingkreis bei Resonanzfrequenz eine sehr geringe Impedanz bezogen auf den Kennwiderstand $X_k$ aufweist.
Der Gütefaktor ist daher auch Maß für die Schwingungsfähigkeit des Schwingkreises beziehungsweise umgekehrt für die Dämpfung.

Dabei gilt für den Serienschwingkreis:
\begin{equation}
    Q = \frac{X_k}{R} = \frac{\sqrt{\frac{L}{C}}}{R}
\end{equation}

Für Frequenzen weit unterhalb der Grenzfrequenz $\omega << \omega_0$ und weit oberhalb $\omega >> \omega_0$
dominieren die Blindanteile der Impedanz wie in Kapitel \ref{sec:resonanzkreise:rlcs} gezeigt ist. 
Daher ist der Einfluss der Güte auf die Impedanz bei diesen Frequenzen gering (Annäherung der Kurven).

\fu{\includegraphics{Tikz/pdf/plot_rlcs_phikurve_guete.pdf}}{Resonanzkurve der Phase, RLC-Serienschwingkreis, Vergleich Güten\label{fig:plot:rlcs:resonanzkurve:phi}}

Abbildung \ref{fig:plot:rlcs:resonanzkurve:phi} zeigt die Phasenkurve des Schwingkreises für unterschiedliche Güten mit $\varphi = \varphi_U - \varphi_I$.
Für Frequenzen deutlich unterhalb der Resonanzfrequenz $\omega << \omega_0$ ist die Phasenverschiebung $\varphi$ nahezu $-90^\circ$ (rein kapazitiv).
Für Frequenzen deutlich oberhalb der Resonanzfrequenz $\omega >> \omega_0$ ist die Phasenverschiebung $\varphi$ nahezu $+90^\circ$ (rein induktiv).

Je höher die Güte (Resonanzschärfe), desto steiler ist der Übergang im Bereich der Resonanzfrequenz von kapazitiv zu induktiv.
}%
\b{%
\begin{minipage}{0.3\textwidth}%
\resizebox{1.1\textwidth}{!}{\includegraphics{Tikz/pdf/circ_schwingkreis_rlcs_komplex_2.pdf}}
\begin{align*}
    \underline{Z} &= R + \mathrm{j}\left(\omega L - \frac{1}{\omega C}\right)\\
        \intertext{normiert}
    \frac{\underline{Z}}{X_k} &= \frac{1}{Q} + \mathrm{j}\left(\frac{\omega}{\omega_0} - \frac{\omega_0}{\omega}\right)\\
        \intertext{Güte und Kennwiderstand}
    Q &= \frac{X_k}{R} \qquad X_k = \sqrt{\frac{L}{C}}
\end{align*}
\end{minipage}\hfill%
\begin{minipage}{0.65\textwidth}\centering
    \resizebox{\textwidth}{!}{\includegraphics{Tikz/pdf/plot_rlcs_impedanzkurve_guete_noxaxis.pdf}}
    \resizebox{\textwidth}{!}{\includegraphics{Tikz/pdf/plot_rlcs_phikurve_guete.pdf}}
\end{minipage}
}%
\end{frame}

%-------------------------------------------------------------------------------------------------%
\subsubsection[Spannungsresonanz]{Spannungsresonanz am Beispiel eines RLC-Reihenschwingkreises}
\begin{frame}\ftx{\subsubsecname\ I}
\index{Resonanz, >Spannungs- (spannungsbezogen)}%
\s{%
    Bei Anregung eines Serienschwingkreises mit einer konstanten Spannungsquelle $\underline{U}_q$ 
    kann es zu einer \textbf{Spannungsresonanz} kommen, die sich durch Überspannungen an $L$ und $C$ auszeichnet.

    Untersucht sei der RLC-Serienschwingkreis aus Abbildung \ref{fig:circ:rlcs} mit der Impedanz $\underline{Z}$ aus Gleichung \ref{eq:rlcs:z}.
    Die Spannung $\underline{U}$ über den Schwingkreis entspricht der konstanten Quellenspannung $\underline{U}_q$ an dessen Außenklemmen
    wie in Abbildung \ref{fig:circ:rlcs:uq} gezeigt.

    \fu{\includegraphics{Tikz/pdf/circ_schwingkreis_rlcs_komplex_2_const_u.pdf}}{Schaltbild RLC-Serienschwingkreis mit konstanter Spannungsquelle\label{fig:circ:rlcs:uq}}

    Mit der Knoten- und Maschengleichung:

    \begin{equation*}\begin{aligned}    % U_L = Uq * (Zk * w/w0)/(sqrt(R^2 + Zk^2 * (w/w0 - w0/w)^2)) 
        \underline{U} &= \underline{U}_R + \underline{U}_L + \underline{U}_C = \underline{U}_q\\
        \underline{I} &= \underline{I}_R = \underline{I}_L = \underline{I}_C
    \end{aligned}\end{equation*}

    und der Bauteil-Impedanzen folgen die komplexen Spannungsteiler:

    \begin{align*}
        \frac{\underline{U}_R}{\underline{U}} &= \frac{\underline{I}}{\underline{I}}
            \cdot \frac{R}{R+\mathrm{j}\left(\omega L-\frac{1}{\omega C}\right)}\\
        \frac{\underline{U}_L}{\underline{U}} &= \frac{\underline{I}}{\underline{I}}
            \cdot \frac{\mathrm{j}\omega L}{R+\mathrm{j}\left(\omega L-\frac{1}{\omega C}\right)}\\
        \frac{\underline{U}_C}{\underline{U}} &= \frac{\underline{I}}{\underline{I}}
            \cdot \frac{\frac{1}{\mathrm{j}\omega C}}{R+\mathrm{j}\left(\omega L-\frac{1}{\omega C}\right)}
    \end{align*}

    Nach Kürzen der Ströme bleiben die frequenzvariablen Impedanzterme. 
    Die einzelnen Spannungen ergeben sich durch deren Multiplikation mit der Spannung $\underline{U}$.

    Um die frequenzabhängigen Spannungen zu normieren, lassen sich die Reaktanzen wie in Gleichung \ref{eq:rlcs:xk} 
    mithilfe des Kennwiderstandes $X_k$ und der Resonanzkreisfrequenz $\omega_0$ aus Gleichung \ref{eq:rlcs:w0} ausdrücken. 
    Mit der so normierten Impedanz aus Gleichung \ref{eq:rlcs:zxk} ergibt sich für die Spannungsteiler die Terme:

    \begin{align*}
        \frac{\underline{U}_R}{\underline{U}} &= 
            \frac{R}{R+\mathrm{j}\left(\frac{\omega}{\omega_0}-\frac{\omega_0}{\omega}\right)X_k}\\
        \frac{\underline{U}_L}{\underline{U}} &=
            \frac{\mathrm{j}X_k\frac{\omega}{\omega_0}}{R+\mathrm{j}\left(\frac{\omega}{\omega_0}-\frac{\omega_0}{\omega}\right)X_k}\\
        \frac{\underline{U}_C}{\underline{U}} &= 
            \frac{-\mathrm{j}X_k\frac{\omega_0}{\omega}}{R+\mathrm{j}\left(\frac{\omega}{\omega_0}-\frac{\omega_0}{\omega}\right)X_k}
    \end{align*}

    Abbildung \ref{fig:plot:rlcs:spannungsresonanz} zeigt die Beträge der Spannungen $U_R$, $U_L$, $U_C$ des RLC-Serienschwingkreises 
    in Abhängigkeit der Kreisfrequenz normiert auf die Resonanzkreisfrequenz.
    Die Normierung auf $\omega_0$ ermöglicht zum Beispiel den Vergleich der Güte von Reihenschwingkreisen mit unterschiedlichen Resonanzfrequenzen.

    \fu{\includegraphics{Tikz/pdf/plot_rlcs_spannung_resonanz.pdf}}%
    {Resonanzkurve der Spannung am RLC-Serienschwingkreis\newline
    $U_q=10\ \mathrm{V},\ Q=3,162$\label{fig:plot:rlcs:spannungsresonanz}}

    Wie in der Abbildung zu erkennen ist, ergibt sich im Bereich der Resonanzfrequenz $\omega = \omega_0$ 
    eine deutlich höhere Spannung über $L$ und $C$ im Vergleich zum ohmschen Widerstand $R$. 
    
    Das Maß der Spannungsüberhöhung bei der Resonanzkreisfrequenz wird durch die Güte ausgedrückt. 
    Wie in Gleichung \ref{eq:q:spannung} definiert ist, entspricht sie dem Verhältnis von Überspannung der Scheinwiderstände ($X_L$, $X_C$) $U_{L,0}$, $U_{C,0}$
    zu Spannung des Wirkwiderstandes ($R$) $U_{R,0}$ im Resonanzfall ($omega = \omega_0$). 

    \begin{equation}
        Q = \frac{U_{L,0}}{U} = \frac{U_{C,0}}{U} = \frac{X_k}{R}
    \end{equation}

    Wie in der Graphik zu erkennen ist, sind die Spannungsbeträge $|U_L|$ und $|U_C|$ nicht exakt bei der Resonanzfrequenz maximal!

    \textbf{Exkurs: Herleitung der Güte über die Blindleistung}

    Allgemein gilt für die Scheinleistung eines Bauteiles:
    \begin{align}\label{eq:ql}\underline{S}
        &= S \cdot \mathrm{e}^{\mathrm{j}\varphi} \\
        &= U \cdot I \cos{\varphi} + \mathrm{j} U \cdot I \sin{\varphi} \\
        &= P \cdot \mathrm{j} Q
    \end{align}

    Induktivität und Kapazität können als Scheinwiderstände lediglich Blindleistung aufnehmen und abgeben. 
    Mit $\varphi_L = +90\degree$ und $\varphi_C = -90\degree$ folgt für den Resonanzfall für die Blindleistung beider Bauteile:

    \begin{equation}\label{eq:blindleistung:rlcs}
        \begin{aligned}
            Q_L &= U_L I_L \cdot \sin{(+\frac{\pi}{2})} = + U_L \cdot I_L \vphantom{\big|}\\
            Q_C &= U_C I_C \cdot \sin{(-\frac{\pi}{2})} = - U_L \cdot I_L \vphantom{\big|}
        \end{aligned}
    \end{equation}

    Bei Anregung mit einer konstanten Spannungsquelle $\underline{U}_q$ fließt im Reihenschwingkreis
    der gleiche Strom $\underline{I}$ durch alle Bauteile. 

    Mit der allgemeinen Definition der Güte über die Blindleistung in Gleichung \ref{eq:q:leistung}
    der Induktivität $Q_L$ respektive der Kapazität $Q_C$ und der Wirkleistung $P$ des Schwingkreises 
    im Resonanzfall folgt für den Reihenschwingkreis:

    \begin{equation}\label{eq:rlcs:q:guete}
        \begin{aligned}
            Q &= \frac{|Q_{L,0}|}{P_0} = \frac{U_{L,0}I_L}{U_{R,0}I_R} = \frac{U_{L,0}I}{U_{R,0}I} = \frac{U_{L,0}}{U}\\
            &= \frac{|Q_{C,0}|}{P_0} = \frac{U_{C,0}I_C}{U_{R,0}I_R} = \frac{U_{C,0}I}{U_{R,0}I} = \frac{U_{C,0}}{U}
        \end{aligned}
    \end{equation}

    Die Güte entspricht deshalb wie in Gleichung \ref{eq:rlcs:q:guete} gezeigt und in Gleichung \ref{eq:q:spannung} konstatiert, 
    dem Verhältnis der Überspannung der Scheinwiderstände ($L$, $C$) gegenüber der Spannung des Wirkwiderstandes ($R$) im Resonanzfall.

    Über die Spannungsteiler-Regel lässt sich die Güte durch einsetzen der Reaktanzen für den Resonanzfall auch als Verhältnis von 
    Kennwiderstand zu Wirkwiderstand ausdrücken wie in Gleichung \ref{eq:q:kennwiderstand} konstatiert. 
    Mit $X_k$ aus Gl. \ref{eq:rlcs:xk} und $\underline{Z}$ aus Gl. \ref{eq:rlcs:z} gilt:

    \begin{equation}\begin{aligned}
        \frac{U_{L,0}}{U}
        &= \frac{\redcancel{I}}{\redcancel{I}} \cdot \frac{|\mathrm{j}X_{L,0}|}{\big|R + \mathrm{j}(X_{L,0}+X_{C,0})\big|}\vphantom{\Bigg|}\\
        &= \frac{|X_{L,0}|}{ \big|R + \mathrm{j}( \underbrace{X_{L,0} + X_{C,0}}_{=\mathrlap{\ 0}})\big|}\vphantom{\Bigg|}\\
        &= \frac{|X_{L,0}|}{R} = \frac{X_k}{R} = Q
    \end{aligned}\end{equation}
}
\b{%
\begin{minipage}{0.3\textwidth}%
\resizebox{1.1\textwidth}{!}{\includegraphics{Tikz/pdf/circ_schwingkreis_rlcs_komplex_2_const_u.pdf}}
\begin{align*}
        \intertext{Spannungsteiler (Bsp.)}
    \frac{\underline{U}_L}{\underline{U}} &= \frac{\mathrm{j}\left(\frac{\omega}{\omega_0}\right)}{\frac{1}{Q}+\mathrm{j}\left(\frac{\omega}{\omega_0}-\frac{\omega_0}{\omega}\right)}\\
        \intertext{Güte und Überspannung}
    Q &= \frac{X_k}{R} = \frac{U_{L,0}}{U} = \frac{U_{C,0}}{U}
\end{align*}
\end{minipage}\hfill%
\begin{minipage}{0.65\textwidth}\centering
    \resizebox{\textwidth}{!}{\includegraphics{Tikz/pdf/plot_rlcs_spannung_resonanz.pdf}}\newline% TODO RENAME
    $U_q(\omega_0) \Rightarrow$ Überspannung!
\end{minipage}
}
\end{frame}

\b{%
\begin{frame}{Nebenrechnung}
\begin{minipage}{0.3\textwidth}%
    \resizebox{1.1\textwidth}{!}{\includegraphics{Tikz/pdf/circ_schwingkreis_rlcs_komplex_2_const_u.pdf}}
    \begin{align*}
        \underline{U} &= \underline{U}_R + \underline{U}_L + \underline{U}_C = \underline{U}_q\\
        \underline{I} &= \underline{I}_R = \underline{I}_L = \underline{I}_C \\
        \underline{Z} &= R + \mathrm{j}(\omega L - \frac{1}{\omega C}) \\
        \mathllap{\Im\{\underline{Z}\}} &\overset{!}{=} 0 \Rightarrow \omega_0 =\frac{1}{\sqrt{LC}}\\
        X_k &= |X_{L,0}| = |X_{C,0}| =  \mathrlap{\sqrt{\frac{L}{C}}}\\
        Q &= \frac{U_{L,0}I}{U_{R,0} I} = \frac{U_{L,0}}{U_q} = \frac{X_k}{R}
    \end{align*}
\end{minipage}\hfill%
\begin{minipage}{0.65\textwidth}\centering
    \begin{align*}
        \frac{\underline{U}_R}{\underline{U}} &= \frac{\underline{I}}{\underline{I}}
            \cdot \frac{R}{R+\mathrm{j}\left(\omega L-\frac{1}{\omega C}\right)}\\
        \frac{\underline{U}_L}{\underline{U}} &= \frac{\underline{I}}{\underline{I}}
            \cdot \frac{\mathrm{j}\omega L}{R+\mathrm{j}\left(\omega L-\frac{1}{\omega C}\right)}\\
        \frac{\underline{U}_C}{\underline{U}} &= \frac{\underline{I}}{\underline{I}}
            \cdot \frac{\frac{1}{\mathrm{j}\omega C}}{R+\mathrm{j}\left(\omega L-\frac{1}{\omega C}\right)}\\
        \intertext{normiert}
        \frac{\underline{U}_R}{\underline{U}} &= 
            \frac{\sfrac{1}{Q}}{\frac{1}{Q}+\mathrm{j}\left(\frac{\omega}{\omega_0}-\frac{\omega_0}{\omega}\right)}\\
        \frac{\underline{U}_L}{\underline{U}} &=
            \frac{+\mathrm{j}\frac{\omega}{\omega_0}}{\frac{1}{Q}+\mathrm{j}\left(\frac{\omega}{\omega_0}-\frac{\omega_0}{\omega}\right)}\\
        \frac{\underline{U}_C}{\underline{U}} &= 
            \frac{-\mathrm{j}\frac{\omega_0}{\omega}}{\frac{1}{Q}+\mathrm{j}\left(\frac{\omega}{\omega_0}-\frac{\omega_0}{\omega}\right)}
    \end{align*}
\end{minipage}
\end{frame}
}

%-------------------------------------------------------------------------------------------------%
\subsubsection{Stromkurve des RLC-Reihenschwingkreises}\label{sec:resonanzkreise:rlcs:stromkurve}
\begin{frame}\ftx{\subsubsecname\ II}
\s{
    Bei Anregung mit konstanter Spannung $\underline{U}_q$ ergibt sich für den RLC-Serienschwingkreis aus Abbildung \ref{fig:circ:rlcs:uq}
    ein frequenzabhängiger Stromverlauf. Dieser ist betragsmäßig in Abbildung \ref{fig:plot:rlcs:uq:i} für verschiedene Gütefaktoren $Q$ dargestellt.
   
    \fu{\includegraphics{Tikz/pdf/plot_rlcs_stromkurve_uq_guete.pdf}}{%
        Stromkurve des RLC-Serienschwingkreises bei konstanter Spannung\newline$X_k=31,6\ \Omega,\ U_q=10\ \mathrm{V},\ Q=var.$%
        \label{fig:plot:rlcs:uq:i}%
    }%
    
    Der Strombetrag wird maximal bei Resonanzfrequenz wie der Graphik zu entnehmen ist.
    Wie in Abbildung \ref{fig:plot:rlcs:impedanzkurve} gezeigt ist, 
    ist die Impedanz im Resonanzfall minimal und entspricht dem ohmschen Widerstand $R$.

    Allgemein gilt für den Strom $\underline{I}$ der folgende Zusammenhang mit der Impedanz $\underline{Z}$ 
    aus Gleichung \ref{eq:rlcs:zxk} und Resonanzfrequenz $\omega_0$ aus Gleichung \ref{eq:rlcs:w0}:

    \begin{equation}
        \begin{aligned}
            \underline{I} 
            &= \frac{\underline{U}}{\underline{Z}} \\
            &= \frac{\underline{U}_q}{R + \mathrm{j}\left(\omega L - \frac{1}{\omega C}\right)} \\
            &= \frac{\underline{U}_q}{R + \mathrm{j}\left(\frac{\omega}{\omega_0} - \frac{\omega_0}{\omega}\right)X_k}
        \end{aligned}
    \end{equation}

    Der Maximalstrom $I_0$ ergibt sich für eine reelle Quellenspannung $\underline{U}_q = U_q$ 
    aus dem Teiler der Quellenspannung durch den Widerstand $R$: 

    \begin{equation}
        I_0 = \frac{U_q}{R} \qquad \text{mit} \quad \underline{Z}_0 = R
    \end{equation}

    Das heißt sowohl der Gesamtstrom $\underline{I}$ als auch dessen Wirkanteil $\Re\{\underline{I}\}$ werden durch den Widerstand $R$ begrenzt.

    Für den Strom $\underline{I}$ bezogen auf den Maximalstrom $I_0$ gilt mit der Güte aus Gleichung \ref{eq:rlcs:q:guete}:

    \begin{equation}\label{eq:rlcs:inorm}
        \begin{aligned}
            \frac{\underline{I}}{I_0} 
            &= \frac{U_q}{\underline{Z}} \cdot \frac{R}{U_q} 
            =  \frac{R}{\underline{Z}}\\
            &= \frac{1}{1 + \mathrm{j} \left(\frac{\omega}{\omega_0} - \frac{\omega_0}{\omega}\right) Q}
        \end{aligned}
    \end{equation}
}
\b{%
    \begin{minipage}{0.3\textwidth}%
    \resizebox{1.1\textwidth}{!}{\includegraphics{Tikz/pdf/circ_schwingkreis_rlcs_komplex_2_const_u.pdf}}
    \begin{align*}
        \intertext{Strom}
        \underline{I} &= \frac{I_0}{1 + \mathrm{j}\left(\frac{\omega}{\omega_0} - \frac{\omega_0}{\omega}\right)Q}\\
        \intertext{Maximalstrom, Güte}
        \underline{I}_0 &= \frac{\underline{U}_q}{R}
        \qquad Q = \frac{X_k}{R}        
    \end{align*}
    \end{minipage}\hfill%
    \begin{minipage}{0.65\textwidth}\centering
        \resizebox{\textwidth}{!}{\includegraphics{Tikz/pdf/plot_rlcs_stromkurve_uq_guete.pdf}}
        \par Strom maximal bei $\omega_0$

        Begrenzung durch $R$
    \end{minipage}
}
\end{frame}

\b{%
\begin{frame}\ftx{Nebenrechnung}%
\begin{minipage}{0.3\textwidth}%
    \resizebox{1.1\textwidth}{!}{\includegraphics{Tikz/pdf/circ_schwingkreis_rlcs_komplex_2_const_u.pdf}}
    \begin{align*}
        \underline{Z} &= R + \mathrm{j}(X_L + X_C) \\
        X_L &= +X_k \frac{\omega}{\omega_0} = \omega L \\
        X_C &= -X_k \frac{\omega_0}{\omega} = -\frac{1}{\omega C}\\
        X_k &= |X_{L,0}| = |X_{C,0}| = \sqrt{\frac{L}{C}}\\
        Q &= \frac{X_k}{R} \quad \omega_0 = \frac{1}{\sqrt{LC}}
    \end{align*}
\end{minipage}\hfill%
\begin{minipage}{0.65\textwidth}\centering
    \begin{align*}
        \underline{I} &= \frac{\underline{U}_q}{\underline{Z}} \\
        &= \frac{\underline{U}_q}{R + \mathrm{j}\left(\omega L - \frac{1}{\omega C}\right)} \\
        &= \frac{\underline{U}_q}{R + \mathrm{j}\left(\frac{\omega}{\omega_0} - \frac{\omega_0}{\omega}\right)X_k} \\
        &= \frac{\underline{U}_q}{R} \cdot \frac{1}{1 + \mathrm{j}\left(\frac{\omega}{\omega_0} - \frac{\omega_0}{\omega}\right)\frac{X_k}{R}} \\
        &= \underline{I}_0 \cdot \frac{1}{1 + \mathrm{j}\left(\frac{\omega}{\omega_0} - \frac{\omega_0}{\omega}\right)Q}
    \end{align*}
\end{minipage}
\end{frame}
}

%-------------------------------------------------------------------------------------------------%
\subsubsection{Spannungskurve des RLC-Reihenschwingkreis}
\begin{frame}\ftx{\subsubsecname}
% RLCs an Iq
\b{%
    \begin{minipage}{0.3\textwidth}%
    \resizebox{1.1\textwidth}{!}{\includegraphics{Tikz/pdf/circ_schwingkreis_rlcs_komplex_2_const_i.pdf}}
    \begin{align*}
        \intertext{Stromquelle}
        \underline{I} &= I_q\\
        \intertext{Spannung}
        \underline{U} &= I_q \cdot \underline{Z}\\
        \intertext{Impedanz}
        \underline{Z} &= R + \mathrm{j}X
    \end{align*}
    \end{minipage}\hfill%
    \begin{minipage}{0.65\textwidth}\centering
        \resizebox{\textwidth}{!}{\includegraphics{Tikz/pdf/plot_rlcs_impedanzkurve_guete.pdf}}\newline
        Spannungskurve $\sim$ Impedanzkurve
    \end{minipage}
}%
\s{%
    Wird ein RLC-Serienschwingkreis mit einem konstanten Strom $\underline{I}_q$ wie in Abbildung \ref{fig:circ:rlcs:iq}
    angeregt, ergeben sich Bauteilspannungen.

    \fu{\includegraphics{Tikz/pdf/circ_schwingkreis_rlcs_komplex_2_const_i.pdf}}{Schaltbild RLC-Serienschwingkreis mit konstanter Stromquelle\label{fig:circ:rlcs:iq}}

    Die Bauteilspannungen sind dabei direkt proportional zu den jeweiligen Impedanzen. Für die Gesamtspannung gilt:

    \begin{equation*}
        \underline{U} = \underline{I}_q \cdot \underline{Z}
    \end{equation*}

    Die Spannungskurve entspricht daher exakt dem Verlauf der Impedanzkurve in Abbildung \ref{fig:plot:rlcs:resonanzkurve:impedanz}.
}%
\end{frame}

%%%%%%%%%%%%%%%%%%%%%%%%%%%%%%%%%%%%%%%%%%%%%%%%%%%%%%%%%%%%%%%%%%%%%%%%%%%%%%%%%%%%%%%%%%%%%%%%%%%
% RLC-Parallelschwingkreis - Resonanzverhalten
%%%%%%%%%%%%%%%%%%%%%%%%%%%%%%%%%%%%%%%%%%%%%%%%%%%%%%%%%%%%%%%%%%%%%%%%%%%%%%%%%%%%%%%%%%%%%%%%%%%

% Viel Copy-Paste von Kapitel "RLC-Serienschwingkreis - Resonanzverhalten"
\subsection{RLC-Parallelschwingkreis - Resonanzverhalten}\label{sec:resonanzkreise:rlcp}
\index{Schwingkreis>RLC-Parallelschwingkreis}%
\begin{frame}\ftx{\subsecname}%
    \b{%
        \begin{minipage}{0.4\textwidth}%
        \resizebox{\textwidth}{!}{\includegraphics{Tikz/pdf/circ_schwingkreis_rlcp_komplex_2.pdf}}\newline
        
        \begin{tcolorbox}[colback=white,colframe=GETgreen!25!black,toprule=0.5pt,rightrule=0.5pt,bottomrule=0.5pt,leftrule=0.5pt,
            title=\textbf{Resonanzbedingung (allg.)},center title,colbacktitle=GETgreen!5!white,coltitle=black,titlerule=0.5pt,left=2pt,right=2pt]
            \centering
            Phasenverschiebung zwischen\newline
            $\underline{U}$ und $\underline{I}$ entspricht $\varphi = 0$.\newline
            Oder:
            \begin{equation*}
                \Im\left\{\frac{\underline{I}}{\underline{U}}\right\} =
                \Im\left\{\underline{Y}\right\} \overset{!}{=} 0
            \end{equation*}
        \end{tcolorbox}
        \end{minipage}\hfill%
        \begin{minipage}{0.55\textwidth}
            \begin{equation*}
                \underline{Y} = G + \mathrm{j}\Big(\underbrace{\omega C - \frac{1}{\omega L}}_{=\mathrlap{\ 0\quad\text{für}\ \omega=\omega_0}}\Big)
            \end{equation*}
            \begin{tcolorbox}[colback=white,colframe=GETgreen!25!black,
                toprule=0.5pt,bottomrule=0.5pt,rightrule=0.5pt,leftrule=0.5pt,
                top=2pt,bottom=2pt,left=2pt,right=2pt,middle=2pt]
                Fall 1: $\mathbf{\displaystyle{\omega < \omega_0}}$ 
                \tcblower
                $\rightarrow\left(\omega C - \frac{1}{\omega L}\right) < 0$ \hfill ohmsch-induktiv
            \end{tcolorbox}
        
            \begin{tcolorbox}[colback=white,colframe=GETgreen!25!black,
                toprule=0.5pt,bottomrule=0.5pt,rightrule=0.5pt,leftrule=0.5pt,
                top=2pt,bottom=2pt,left=2pt,right=2pt,middle=2pt]
                Fall 2: $\mathbf{\displaystyle{\omega = \omega_0}}$ \hfill (Resonanzfall)
                \tcblower
                $\rightarrow\left(\omega C - \frac{1}{\omega L}\right) = 0$ \hfill rein ohmsch
            \end{tcolorbox}
        
            \begin{tcolorbox}[colback=white,colframe=GETgreen!25!black,
                toprule=0.5pt,bottomrule=0.5pt,rightrule=0.5pt,leftrule=0.5pt,
                top=2pt,bottom=2pt,left=2pt,right=2pt,middle=2pt]
                Fall 3: $\mathbf{\displaystyle{\omega > \omega_0}}$ 
                \tcblower
                $\rightarrow\left(\omega C - \frac{1}{\omega L}\right) > 0$ \hfill ohmsch-kapazitiv
            \end{tcolorbox}
        \end{minipage}
        }%
        \s{%
            In diesem Unterkapitel wird das Resonanzverhalten eines RLC-Parallelschwingkreis untersucht.
            Der Schwingkreis ist in Abbildung \ref{fig:circ:rlcp} gezeigt und entspricht dem idealen LC-Parallelschwingkreis
            aus Beispiel \ref{bsp:resonanz:lc} mit einem zusätzlichen parallelen ohmschen Widerstand $R$.
            
            Die Untersuchung erfolgt analog zu jener für den RLC-Serienschwingkreis in Kapitel \ref{sec:resonanzkreise:rlcs}.
            Im Parallelschwingkreis wird dabei vornehmlich die Admittanz $\underline{Y}$ betrachtet.
            Dadurch ergeben sich ähnliche Fallunterscheidungen und Rechenwege wie im Serienschwingkreis,
            wobei die Rollen von Strom und Spannung, von Induktivität und Kapazität, sowie von Widerstand und Leitwert vertauscht sind:

            \begin{equation}
                \underline{Y} = \underline{Z}^{-1} = G + \mathrm{j}B = \frac{\underline{I}}{\underline{U}}
            \end{equation}
        
            \fu{\includegraphics{Tikz/pdf/circ_schwingkreis_rlcp_komplex_2.pdf}}{Schaltbild RLC-Parallelschwingkreis\label{fig:circ:rlcp}}
        
            Die Admittanz $\underline{Y}$ des RLC-Parallelschwingkreis aus Abbildung \ref{fig:circ:rlcp} ist gegeben durch:
        
            \begin{align}
                \underline{I} &= \underline{U} \cdot \left(G + \mathrm{j}B_C + \mathrm{j}B_L\right)\\
                \underline{Y} &= G + \mathrm{j}\left(\omega C - \frac{1}{\omega L}\right)\label{eq:rlcp:y}
            \end{align}
            
            Im Resonanzfall verschwindet nach Gleichung \ref{eq:resonanzbedingung} der Imaginärteil der Impedanz respektive Admittanz.
            Folglich heben sich bei dieser Anordnung die Suszeptanzen von Induktivität $B_L$ und Kapazität $B_C$ auf.
            Durch einsetzen von $\underline{Y}$ in die Resonanzbedingung lässt sich die Resonanzfrequenz $\omega_0$ bestimmen:
        
            \begin{equation}\label{eq:rlcp:w0}
                \begin{aligned}
                    \Im\{\underline{Y}\} = \left(\omega C - \frac{1}{\omega L}\right) &\overset{!}{=} 0 \\
                    \omega_0 C - \frac{1}{\omega_0 L} &= 0 &&\Rightarrow& \omega_0 &= \frac{1}{\sqrt{LC}}
                \end{aligned}
            \end{equation}
        
            Da die Suszeptanz der Kapazität positiv und proportional zur Frequenz ist ($B_C = \omega C \sim \omega$),
            und da die Suszeptanz der Induktivität negativ und proportional zum Kehrwert der Frequenz ist ($-B_L \sim \frac{1}{\omega}$),
            überwiegt für Frequenzen unterhalb der Resonanzfrequenz der induktive Anteil und für Frequenzen oberhalb der kapazitive Anteil.
        
            Daraus ergibt sich die folgende Fallunterscheidung für den RLC-Serienschwingkreis:
        
            \begin{equation}
                \Im\{\underline{Y}\} =
            \begin{cases}
                \omega C - \frac{1}{\omega L} < 0 & \qquad \omega < \omega_0 \qquad\text{ohmsch-induktiv}\vphantom{\Big|}\\
                \omega C - \frac{1}{\omega L} = 0 & \qquad \omega = \omega_0 \qquad\text{rein ohmsch}\vphantom{\Big|}\\
                \omega C - \frac{1}{\omega L} > 0 & \qquad \omega > \omega_0 \qquad\text{ohmsch-kapazitiv}\vphantom{\Big|}
            \end{cases}
            \label{eq:rlcp:fallunterscheidung}
            \end{equation}
        
            Die Fallunterscheidung wird in Abbildung \ref{fig:plot:rlcp:zeigerdiagramm} anhand eines Zeigerdiagramms 
            und in Abbildung \ref{fig:plot:rlcp:admittanzkurve} anhand einer Admittanzkurve verdeutlicht.
        }%
\end{frame}

%-------------------------------------------------------------------------------------------------%

\subsubsection{Fallunterscheidung im Zeigerdiagramm}
\begin{frame}[t]\ftx{\subsubsecname}
    \b{%
    \begin{minipage}{0.4\textwidth}%
    \resizebox{\textwidth}{!}{\includegraphics{Tikz/pdf/circ_schwingkreis_rlcp_komplex_2_const_u.pdf}}
    \end{minipage}\hfill%
    \begin{minipage}{0.6\textwidth}%
    \begin{align*}\underline{I}
        &= \underline{I}_R\ +\ \underline{I}_L\ +\ \underline{I}_C&
            \underline{U} &= konst.\\
        &= \underline{U} \cdot \big[G + \mathrm{j}\big(\underbrace{\omega C - \frac{1}{\omega L}}_{=\mathrlap{\ \Im\{\underline{Y}\}}}\big)\big]&
            \omega_0 &= \frac{1}{\sqrt{LC}}
    \end{align*}
    \end{minipage}\vfill
    \begin{minipage}{\textwidth}\centering
        \begin{minipage}{0.3\textwidth}\centering \rotatebox{90}{Fall 1 \quad ohmsch-induktiv}
        \resizebox{!}{0.5\textheight}{\includegraphics{Tikz/pdf/plot_zeigerdiagramm_rlcp_2.pdf}}\newline $\qquad\omega < \omega_0$
        \end{minipage}\hfill%
        \begin{minipage}{0.3\textwidth}\centering \rotatebox{90}{Fall 2 \quad rein ohmsch}
        \resizebox{!}{0.5\textheight}{\includegraphics{Tikz/pdf/plot_zeigerdiagramm_rlcp_3.pdf}}\newline $\qquad\omega = \omega_0$
        \end{minipage}\hfill%
        \begin{minipage}{0.3\textwidth}\centering \rotatebox{90}{Fall 3 \quad ohmsch-kapazitiv}
        \resizebox{!}{0.5\textheight}{\includegraphics{Tikz/pdf/plot_zeigerdiagramm_rlcp_1.pdf}}\newline $\qquad\omega > \omega_0$
        \end{minipage}
    \end{minipage}
    }%
    \s{%
    Abbildung \ref{fig:plot:rlcp:zeigerdiagramm} zeigt für alle drei Fälle exemplarisch ein Zeigerdiagramm.

    Gezeigt sind die entstehenden Ströme durch $R$, $L$ und $C$, sowie der Gesamtstrom $\underline{I}$ bei Anregung mit konstanter Spannung $\underline{U}$.
    Die maßstabsgetreuen Längenverhältnisse $\frac{I_L}{I_C}$ entsprechen: $\frac{1}{2}$, $\frac{\sqrt{2}}{\sqrt{2}}$, $\frac{2}{1}$.
    %für die Frequenzen $\omega = \frac{\omega_0}{2}$, $\omega = \omega_0$ und $\omega = 2\omega_0$.

    \begin{figure}[H]\centering
        \captionsetup[subfigure]{justification=centering,}
        \begin{subfigure}{0.3\textwidth}\centering
            \includegraphics{Tikz/pdf/plot_zeigerdiagramm_rlcp_2.pdf}
            \caption{Fall 1: $\omega < \omega_0$\newline(ohmsch-induktiv)}
        \end{subfigure}\hfill%
        \begin{subfigure}{0.3\textwidth}\centering
            \includegraphics{Tikz/pdf/plot_zeigerdiagramm_rlcp_3.pdf}
            \caption{Fall 2: $\omega = \omega_0$\newline(rein ohmsch)}
        \end{subfigure}\hfill%
        \begin{subfigure}{0.3\textwidth}\centering
            \includegraphics{Tikz/pdf/plot_zeigerdiagramm_rlcp_1.pdf}
            \caption{Fall 3: $\omega > \omega_0$\newline(ohmsch-kapazitiv)}
        \end{subfigure}
        \s{\caption{Zeigerdiagramm RLC-Parallelschwingkreis}}
        \label{fig:plot:rlcp:zeigerdiagramm}
    \end{figure}

    Da es sich um einen passiven Schwingkreis handelt (Verbraucherzählpfeilsystem), 
    ist die Wirkleistung $P$ des Schwingkreises immer positiv mit $P=\Re\{\underline{U}\cdot\underline{I}\}$.
    
    Bei angelegter Spannung $\underline{U}$ (mit Phasenwinkel $0$ eingezeichnet) 
    ist der Realanteil des Gesamtstrom $\underline{I}$ immer positiv.
    Der Zeiger des Gesamtstrom $\underline{I}$ deutet also entweder in den ersten Quadranten (ohmsch-induktiv), 
    auf die reale Achse (rein ohmsch) oder in den vierten Quadranten (ohmsch-kapazitiv).
    }
\end{frame}

%-------------------------------------------------------------------------------------------------%

\subsubsection{Fallunterscheidung in Admittanzkurve}
\begin{frame}[t]\ftx{\subsubsecname}
\s{%
    Für den Parallelschwingkreis aus Abbildung \ref{fig:circ:rlcp} ergibt sich die Admittanz wie in Gleichung \ref{eq:rlcp:y} zu:
    \begin{equation}
        \begin{aligned}\underline{Y}
            &= G + \mathrm{j}\left(B_C + B_L\right) \\
            &= G + \mathrm{j}\left(\omega C - \frac{1}{\omega L}\right)
        \end{aligned}
    \end{equation}
    
    Die induktive Suszeptanz (hyperbolisch) überwiegt für Frequenzen unterhalb der Resonanzfrequenz, 
    die kapazitive Suszeptanz (linear) überwiegt für Frequenzen oberhalb der Resonanzfrequenz
    wie in Gleichung \ref{eq:rlcp:fallunterscheidung} gezeigt ist. 
    
    Abbildung \ref{fig:plot:rlcp:admittanzkurve} zeigt die Admittanzkurve eines RLC-Parallelschwingkreises
    in Abhängigkeit der Frequenz bezogen auf die Resonanzfrequenz $\sfrac{f}{f_0}$ äquivalent $\sfrac{\omega}{\omega_0}$.
    Gezeigt sind die Beträge der Admittanz $|\underline{Y}|$, der Suszeptanz $|B|$,
    der Suszeptanz der Kapazität $|B_C|$, der Suszeptanz der Induktivität $|B_L|$ und der ohmsche Leitwert $G$.
    
    \begin{figure}[H]\centering
        \includegraphics{Tikz/pdf/plot_rlcp_admittanzkurve.pdf} % TODO WIP
        \caption[Admittanzkurve RLC-Parallelschwingkreis]%
        {Admittanzkurve RLC-Parallelschwingkreis\newline mit $R=10\ \Omega$, $L=10\ \mathrm{mH}$, $C=10\ \mu\mathrm{F}$}
        \label{fig:plot:rlcp:admittanzkurve}
    \end{figure}
    
    Wie in der Admittanzkurve zu erkennen ist, ist die Admittanz im Resonanzfall rein ohmsch und entspricht exakt dem ohmschen Leitwert $G$:
    \begin{equation}
        \underline{Y}_0 = G
    \end{equation}
    
    Die Suszeptanzen der Kapazität $B_C$ (positiv) und der Induktivität $B_L$ (negativ) heben sich im Resonanzfall auf (Schnittpunkt der Kurven).
    Die Beträge beider Suszeptanzen sind im Resonanzfall gleich und entsprechen dem Kennleitwert $B_k$:
    
    \begin{equation}\label{eq:rlcp:bk}
        \begin{aligned}
            B_k &= |B_{C,0}| = |B_{L,0}| = \sqrt{\frac{C}{L}} \\
            &= \omega_0 C \ \ = \frac{1}{\omega_0 L} 
        \end{aligned}
    \end{equation}
    
    Zu erkennen ist noch, dass der Betrag der Admittanz sich für sehr niedrige Frequenzen $\omega << \omega_0$ 
    dem induktiven Verlauf annähert mit $\lim_{\omega\rightarrow 0}|Y|=|B_L|$ und für sehr hohe Frequenzen $\omega >> \omega_0$
    dem kapazitiven Verlauf annähert mit $\lim_{\omega\rightarrow \infty}|Y|=|B_C|$.
}
\b{%
    \begin{minipage}{0.3\textwidth}%
    \resizebox{1.1\textwidth}{!}{\includegraphics{Tikz/pdf/circ_schwingkreis_rlcp_komplex_2.pdf}}
    \begin{align*}
        \intertext{\small Resonanzbedingung}
        \onslide<1->{\Im\{\underline{Y}\} &\overset{!}= 0}\\
        \intertext{\small Resonanzfrequenz}
        \onslide<1->{\omega_0 &= \frac{1}{\sqrt{LC}}}\\
        \intertext{\small Resonanzadmittanz}
        \onslide<1->{\underline{Y}_0 &= G}
    \end{align*}
    
    \end{minipage}\hfill%
    \begin{minipage}{0.65\textwidth}\centering
        \begin{equation*}
            \smash{\underline{Y} = G + \mathrm{j}\left(\omega C - \frac{1}{\omega L}\right)}
        \end{equation*}
        \resizebox{\textwidth}{!}{\includegraphics{Tikz/pdf/plot_rlcp_admittanzkurve.pdf}}
        \begin{equation*}
            \smash{B_k = |B_{C,0}| = |B_{L,0}| = \sqrt{\frac{C}{L}}}
        \end{equation*}
    \end{minipage}
}
\end{frame}


%-------------------------------------------------------------------------------------------------%

\subsubsection[Resonanzkurve und Gütefaktor]{Resonanzkurve und Gütefaktor eines RLC-Parallelschwingkreis}
\begin{frame}\ftx{\subsubsecname}
    \s{%
    Die Admittanz eines RLC-Parallelschwingkreis lässt wie in Gleichung \ref{eq:rlcp:y} über die 
    Bauteilgrößen $R$, $L$ und $C$ in Abhängigkeit der Frequenz $\omega$ ausdrücken mit:
    
    \begin{equation}
        \underline{Y} = G + \mathrm{j}B = \frac{1}{R} + \mathrm{j}\left(\omega C - \frac{1}{\omega L}\right)
    \end{equation}
    
    Mithilfe der Resonanzkreisfrequenz aus Gleichsung \ref{eq:rlcp:w0} kann die Kreisfrequenz auf diese normiert werden.
    Die Berechnung erfolgt analog zur Frequenznormierung bei der Impedanz des RLC-Serienschwingkreises in Gleichung \ref{eq:rlcs:zxk}.
    Die Suszeptanzen $B_C$ und $B_L$ werden dafür mithilfe des Kennleitwertes $B_k$ aus Gl. \ref{eq:rlcp:bk} 
    und dem Frequenzverhältnis von $\omega$ zu $\omega_0$ beschrieben. Für die Admittanz folgt:
    
    \begin{equation}\label{eq:rlcp:ybk}
        \begin{aligned}
        \underline{Y}
        &= G + \mathrm{j}\left( \omega C \frac{\omega_0}{\omega_0} - \frac{1}{\omega L}\frac{\omega_0}{\omega_0}\right)\\
        &= G + \mathrm{j}\left( B_{C,0} \frac{\omega}{\omega_0} - B_{L,0} \frac{\omega}{\omega_0}\right)\\
        &= G + \mathrm{j}\left( \frac{\omega}{\omega_0} - \frac{\omega_0}{\omega}\right) B_k
        \end{aligned}
    \end{equation}
    
    Dabei gilt:
    
    \begin{align}
        B_C &= +B_k \frac{\omega}{\omega_0} &%=+\sqrt{\frac{C}{L}}\cdot\frac{\omega \sqrt{LC}}{1}
        &= \omega C &&\text{(linear)}\label{eq:rlcp:bc}\\
        B_L &= -B_k \frac{\omega_0}{\omega} &%=-\sqrt{\frac{C}{L}}\cdot\frac{1}{\omega \sqrt{LC}}
        &= -\frac{1}{\omega L} &&\text{(hyperbolisch)}\vphantom{\Big|}\label{eq:rlcp:bl}\\
        B_k &= \sqrt{\frac{C}{L}}&\text{mit}\quad\omega_0 &= \frac{1}{\sqrt{LC}} &&\text{(konstant)}\label{eq:rlcp:bkw0}
    \end{align}
    
    Wie in Gleichung \ref{eq:q:kennleitwert} konstatiert, lässt sich die Güte $Q$ eines Parallelschwingkreises 
    durch das Verhältnis von Kennleitwert $B_k$ zum Wirkleitwert $G$ beschreiben. 
    Die Definition unterscheidet sich in dieser Hinsicht von jener im Serienschwingkreis. 

    Über die allgemeine Definition der Güte in Gleichung \ref{eq:q:leistung} anhand des Verhältnisses von Blindleistung zu Wirkleistung
    lässt sich die spezielle Güte-Definition für den Parallelschwingkreis ableiten. 
    Gleiches gilt für den Serienschwingkreis wie in Gleichung \ref{eq:rlcs:q:guete} gezeigt ist.:

    \begin{equation}\label{eq:rlcp:q}
        \begin{aligned}
        Q &= \frac{|Q_{L,0}|}{P_0} = \frac{U \cdot I_{L,0}}{U\cdot I_R} = \frac{I_{L,0}}{I_R} = \frac{|B_{L,0}|}{G}\\
         & = \frac{|Q_{C,0}|}{P_0} = \frac{U \cdot I_{C,0}}{U\cdot I_R} = \frac{I_{C,0}}{I_R} = \frac{|B_{C,0}|}{G} = \frac{B_k}{G}
        \end{aligned}
    \end{equation}

    Dadurch lässt sich die Impedanz des Schwingkreises normiert auf dessen Kennleitwert $B_k$ mithilfe der Güte $Q$ ausdrücken:
    
    \begin{equation}\label{eq:rlcp:ynorm}
        \frac{\underline{Y}}{B_k} = \frac{1}{Q} + \mathrm{j}\left(\frac{\omega}{\omega_0} - \frac{\omega_0}{\omega}\right) \vphantom{\Big|}
        \qquad\text{mit}\quad Q = \frac{B_k}{G}
    \end{equation}
    
    Der Term $\frac{\omega}{\omega_0} - \frac{\omega_0}{\omega}$ wird auch als relative Verstimmung $\nu_r$ bezeichnet.\cite{hagmann}

    Abbildung \ref{fig:plot:rlcp:resonanzkurve} zeigt exemplarisch mehrere Resonanzkurven eines RLC-Parallelschwingkreises für unterschiedliche Gütefaktoren $Q$ zum Vergleich.
    Die y-Achse zeigt den Admittanzbetrag normiert auf den Kennleitwert $B_k$ und die x-Achse zeigt die Kreisfrequenz normiert auf die Resonanzkreisfrequenz $\omega_0$.
    
    \fu{\includegraphics{Tikz/pdf/plot_rlcp_admittanzkurve_guete.pdf}}{Resonanzkurve RLC-Parallelschwingkreis, Vergleich Gütefaktoren\label{fig:plot:rlcp:resonanzkurve}}
    
    Die Bauteilgrößen sind gleich gewählt wie für die Impedanzkurve des RLC-Serienschwingkreises in Abbildung \ref{fig:plot:rlcs:impedanzkurve}.
    Anhand der Gleichungen \ref{eq:rlcp:ybk} und \ref{eq:rlcp:ynorm} zeigt sich, dass die normierte Darstellung bei beiden Kurven den direkten Vergleich
    der Gütefaktoren ermöglicht. 

    Im Bereich der Resonanzfrequenz zeigt sich der deutlichste Unterschied der einzelnen Admittanzkurven für unterschiedliche Gütefaktoren.
    Mit $\underline{Y}_0 = G = \frac{B_k}{Q}$ entspricht die Admittanz im Resonanzfall direkt dem ohmschen Leitwert $G$ 
    und ist umgekehrt proportional zur Güte.
        
    Die Güte ist Maß für die Schwingungsfähigkeit des Schwingkreises und Maß für die umgekehrte Dämpfung des Schwingkreises.
    Im Gegensatz zu einem Reihenschwingkreis bedeutet eine hohe Güte im Parallelschwingkreis einen 
    geringen Wirkleitwert $G$ (hoher Wirkwiderstand $R$) im Verhältnis zu einem
    hohen Blindleitwert $B$ (geringer Blindwiderstand $X$).
    }%
    \b{%
    \begin{minipage}{0.3\textwidth}%
    \resizebox{1.1\textwidth}{!}{\includegraphics{Tikz/pdf/circ_schwingkreis_rlcp_komplex_2.pdf}}
    \begin{align*}
        \underline{Y} &= G + \mathrm{j}\left(\omega C - \frac{1}{\omega L}\right)\\
            \intertext{normiert}
        \frac{\underline{Y}}{B_k} &= \frac{1}{Q} + \mathrm{j}\left(\frac{\omega}{\omega_0} - \frac{\omega_0}{\omega}\right)\\
            \intertext{Güte und Kennleitwert}
        Q &= \frac{B_k}{G} \qquad B_k = \sqrt{\frac{C}{L}}
    \end{align*}
    \end{minipage}\hfill%
    \begin{minipage}{0.65\textwidth}\centering
        \resizebox{\textwidth}{!}{\includegraphics{Tikz/pdf/plot_rlcp_admittanzkurve_guete_noxaxis.pdf}}
        \resizebox{\textwidth}{!}{\includegraphics{Tikz/pdf/plot_rlcp_phikurve_guete.pdf}}
    \end{minipage}
    }
    \end{frame}    

%-------------------------------------------------------------------------------------------------%

\subsubsection{Stromresonanz am Beispiel eines RLC-Parallelschwingkreises}
\begin{frame}\ftx{\subsubsecname}
\index{Resonanz, >Strom- (strombezogen)}%
    \s{%
        Bei Anregung eines Parallelschwingkreises mit einer konstanten Stromquelle $\underline{I}_q$ 
        kann es zu einer \textbf{Stromresonanz} kommen, die sich durch Überströmen zwischen $L$ und $C$ auszeichnet.
    
        Untersucht sei der RLC-Parallelschwingkreis aus Abbildung \ref{fig:circ:rlcp} mit der Admittanz $\underline{Y}$ aus Gleichung \ref{eq:rlcp:y}.
        Der Strom $\underline{I}$ durch den Schwingkreis entspricht dem konstanten Quellenstrom $\underline{I}_q$ an dessen Außenklemmen
        wie in Abbildung \ref{fig:circ:rlcp:iq} gezeigt ist.

        \fu{\includegraphics{Tikz/pdf/circ_schwingkreis_rlcp_komplex_2_const_i.pdf}}{Schaltbild RLC-Parallelschwingkreis mit konstanter Stromquelle\label{fig:circ:rlcp:iq}}
    
        Mit der Knoten- und Maschengleichung:
        \begin{equation*}\begin{aligned}    % U_L = Uq * (Zk * w/w0)/(sqrt(R^2 + Zk^2 * (w/w0 - w0/w)^2)) 
            \underline{U} &= \underline{U}_R = \underline{U}_L = \underline{U}_C\\
            \underline{I} &= \underline{I}_R + \underline{I}_L + \underline{I}_C = \underline{I}_q \\
        \end{aligned}\end{equation*}
        und der Bauteil-Admittanzen folgen die komplexen Stromteiler:
        \begin{align*}
            \frac{\underline{I}_R}{\underline{I}} &= \frac{\underline{U}}{\underline{U}}\cdot\frac{\underline{Y}_R}{\underline{Y}}
            =\frac{G}{G+\mathrm{j}\left(\omega C-\frac{1}{\omega L}\right)}
            \vphantom{\bigg|}\\
            \frac{\underline{I}_L}{\underline{I}} &= \frac{\underline{U}}{\underline{U}}\cdot\frac{\underline{Y}_L}{\underline{Y}} 
            =\frac{\frac{1}{\mathrm{j}\omega L}}{G+\mathrm{j}\left(\omega C-\frac{1}{\omega L}\right)}
            \vphantom{\bigg|}\\
            \frac{\underline{I}_C}{\underline{I}} &= \frac{\underline{U}}{\underline{U}}\cdot\frac{\underline{Y}_C}{\underline{Y}} 
            =\frac{\mathrm{j}\omega C}{G+\mathrm{j}\left(\omega C-\frac{1}{\omega L}\right)}
            \vphantom{\bigg|}
        \end{align*}
    
        Die Stromverhältnisse von Bauteilströmen zu Gesamtstrom entsprechen dem frequenzvariablen Verhältnis von Bauteiladmittanz zu Gesamtadmittanz.
        Die Berechnung des Stromteilers über die Admittanzen bietet sich in Parallelschaltungen aufgrund der einfachen Addition der Einzeladmittanzen
        an ähnlich wie die Berechnung über die Impedanz in Serienschaltungen.
  
        Um die frequenzabhängigen Ströme zu normieren, lassen sich die Suszeptanzen wie in Gleichung \ref{eq:rlcp:bk} 
        mithilfe des Kennleitwerts $B_k$ und der Resonanzkreisfrequenz $\omega_0$ aus Gleichung \ref{eq:rlcp:w0} ausdrücken. 
        Mit der so normierten Admittanz aus Gleichung \ref{eq:rlcp:ybk} ergeben sich für die Spannungsteiler die Terme:
    
        \begin{equation}\label{eq:rlcp:ibk:iq}
            \begin{aligned}
            \frac{\underline{I}_R}{\underline{I}} &= 
                \frac{G}{G+\mathrm{j}\left(\frac{\omega}{\omega_0}-\frac{\omega_0}{\omega}\right)B_k}\\
            \frac{\underline{I}_L}{\underline{I}} &= 
                \frac{-\mathrm{j}\frac{\omega_0}{\omega}B_k}{G+\mathrm{j}\left(\frac{\omega}{\omega_0}-\frac{\omega_0}{\omega}\right)B_k}\\
            \frac{\underline{I}_C}{\underline{I}} &=
                \frac{\mathrm{j}\frac{\omega}{\omega_0}B_k}{G+\mathrm{j}\left(\frac{\omega}{\omega_0}-\frac{\omega_0}{\omega}\right)B_k}
            \end{aligned}
        \end{equation}
    
        Abbildung \ref{fig:plot:rlcp:stromresonanz} zeigt die Beträge der Ströme $I_R$, $I_L$, $I_C$  
        in Abhängigkeit der Kreisfrequenz normiert auf die Resonanzkreisfrequenz.
    
        \fu{\includegraphics{Tikz/pdf/plot_rlcp_strom_resonanz.pdf}}%
        {Resonanzkurve des Stromes am RLC-Parallelschwingkreis\newline
        $U_q=10\ \mathrm{V},\ Q=3,162$\label{fig:plot:rlcp:stromresonanz}}
    
        Wie in der Abbildung zu erkennen ist, ergibt sich im Bereich der Resonanzkreisfrequenz $\omega = \omega_0$ 
        eine Stromüberhöhung bei $L$ und $C$ im Vergleich zum ohmschen Widerstand $R$.
        Dabei liegt das jeweilige Maximum des Stromes nicht exakt bei $\omega_0$, sondern 
        für $L$ leicht unterhalb und für $C$ leicht oberhalb der Resonanzfrequenz.
   
        \index{Güte}%
        Die Güte entspricht wie in Gleichung \ref{eq:rlcp:q} gezeigt und in Gleichung \ref{eq:q:strom} konstatiert,
        dem Verhältnis des Überstromes zum Gesamtstrom im Resonanzfall. Im Resonanzfall gilt $I_0 = I_{R,0} = I_{q,0}$.
        Deshalb lässt sich die Güte der Stromteiler-Regel entsprechend als Verhältnis von Kennleitwert $B_k$ zu Wirkleitwert $G$ ausdrücken:

        \begin{equation}
            Q = \frac{I_{L,0}}{I} = \frac{I_{C,0}}{I} = \frac{B_k}{G}
        \end{equation}

        Die frequenzabhängigen Stromverhältnisse ausgedrückt durch die Güte $Q$ mit $\underline{Y}$ aus Gleichung \ref{eq:rlcp:ynorm}:

        \begin{equation}\label{eq:rlcp:ibk:iq:q}
            \begin{aligned}
                \frac{\underline{I}_R}{\underline{I}} &= 
                \frac{\frac{1}{Q}}{\frac{1}{Q}+\mathrm{j}\left(\frac{\omega}{\omega_0}-\frac{\omega_0}{\omega}\right)}\\
            \frac{\underline{I}_L}{\underline{I}} &= 
                \frac{-\mathrm{j}\frac{\omega_0}{\omega}}{\frac{1}{Q}+\mathrm{j}\left(\frac{\omega}{\omega_0}-\frac{\omega_0}{\omega}\right)}\\
            \frac{\underline{I}_C}{\underline{I}} &=
                \frac{\mathrm{j}\frac{\omega}{\omega_0}}{\frac{1}{Q}+\mathrm{j}\left(\frac{\omega}{\omega_0}-\frac{\omega_0}{\omega}\right)}
            \end{aligned}
        \end{equation}
    }%
    \b{%
    \begin{minipage}{0.3\textwidth}%
    \resizebox{1.1\textwidth}{!}{\includegraphics{Tikz/pdf/circ_schwingkreis_rlcp_komplex_2_const_i.pdf}}
    \begin{align*}    % I_C = Iq * (Bk * w/w0)/(sqrt(G^2 + Bk^2 * (w/w0 - w0/w)^2)) 
        \underline{U} &= \underline{U}_R = \underline{U}_L = \underline{U}_C \\
        \underline{I} &= \underline{I}_R + \underline{I}_L + \underline{I}_C = \underline{I}_q\\
            \intertext{Stromteiler (Bsp.)}
        \frac{\underline{I}_C}{\underline{I}} &= \frac{\mathrm{j}\left(\frac{\omega}{\omega_0}\right)}{\frac{1}{Q}+\mathrm{j}\left(\frac{\omega}{\omega_0}-\frac{\omega_0}{\omega}\right)}\\
            \intertext{Güte und Überstrom}
        Q &= \frac{B_k}{G} = \frac{I_{C,0}}{I} = \frac{I_{L,0}}{I}
    \end{align*}
    \end{minipage}\hfill%
    \begin{minipage}{0.65\textwidth}\centering
        \resizebox{\textwidth}{!}{\includegraphics{Tikz/pdf/plot_rlcp_strom_resonanz.pdf}}% 
        \par Stromqelle $I_q(\omega_0) \Rightarrow$ Überstrom!
    \end{minipage}
    }
    \end{frame}

%-------------------------------------------------------------------------------------------------%

\subsubsection{Spannungskurve des RLC-Parallelschwingkreis}
\begin{frame}\ftx{\subsubsecname\ II}
\s{
    Bei Anregung mit konstantem Strom $\underline{I}_q$ wie in Abbildung \ref{fig:circ:rlcp:iq} 
    ergibt sich für den RLC-Parallelschwingkreis aus Abbildung \ref{fig:circ:rlcp:iq}
    ein frequenzabhängiger Spannungsverlauf. Dieser ist betragsmäßig in Abbildung \ref{fig:plot:rlcp:iq:u} für verschiedene Gütefaktoren $Q$ dargestellt.
   
    \fu{\includegraphics{Tikz/pdf/plot_rlcp_spannungskurve_iq_guete.pdf}}{%
        Spannungskurve des RLC-Parallelschwingkreises bei konstantem Strom\newline$B_k=31,6\ \mathrm{mS},\ I_q=10\ \mathrm{mA},\ Q=var.$%
        \label{fig:plot:rlcp:iq:u}%
    }%
    
    Das Spannungsverhalten des RLC-Parallelschwingkreis ist analog zum Stromverhalten des RLC-Serienschwingkreises, 
    wie es in Kapitel \ref{sec:resonanzkreise:rlcs:stromkurve} beschrieben wurde. 

    Mit der Admittanz normiert auf den Kennleitwert aus Gleichung \ref{eq:rlcp:ynorm},
    den Kirchhoffschen Regeln und dem ohmschen Gesetz folgt für die Spannung $\underline{U}$:

    \begin{equation}\label{eq:rlcp:u:iq:bknorm}
        \begin{aligned}
            \underline{U} 
            &= \frac{\underline{I}}{\underline{Y}} \\
            &= \frac{\underline{I}_q}{G + \mathrm{j}\left(\omega L - \frac{1}{\omega C}\right)} \\
            &= \frac{\underline{I}_q}{G + \mathrm{j}\left(\frac{\omega}{\omega_0} - \frac{\omega_0}{\omega}\right)B_k}
        \end{aligned}
    \end{equation}

    Die Maximalspannung $U_0$ ergibt sich im Resonanzfall. 
    Sie entspricht dem Teiler des reellen Quellenstrom $\underline{I}_q = I_q$ 
    durch den Wirkleitwert $G$: 

    \begin{equation}
        U_0 = \frac{I_q}{G} \qquad \text{mit} \quad \underline{Y}_0 = G
    \end{equation}

    Dadurch werden die komplexe Gesamtspannung $\underline{U}$ als auch deren Wirkanteil $\Re\{\underline{U}\}$ 
    durch den Leitwert $G$ und den Quellenstrom $I_q$ begrenzt.

    Für die Spannung $\underline{U}$ normiert auf die Maximalspannung $U_0$ folgt aus Gl. \ref{eq:rlcp:u:iq:bknorm}:

    \begin{equation}\label{eq:rlcp:unorm}
        \begin{aligned}
            \frac{\underline{U}}{U_0} 
            &= \dfrac{\dfrac{I_q}{\underline{Y}}}{\dfrac{I_q}{G}} 
            =  \frac{G}{\underline{Y}}\\
            &= \frac{1}{1 + \mathrm{j} \left(\frac{\omega}{\omega_0} - \frac{\omega_0}{\omega}\right) Q}
        \end{aligned}
    \end{equation}

    Der normierte Spannungsverlauf des RLC-Parallelschwingkreises entspricht also 
    exakt dem normierten Stromverlauf des RLC-Serienschwingkreises mit Abbildung \ref{fig:plot:rlcs:spannungsresonanz}
    und Gleichung \ref{eq:rlcs:inorm}.
}%
\b{%
    \begin{minipage}{0.3\textwidth}%
    \resizebox{1.1\textwidth}{!}{\includegraphics{Tikz/pdf/circ_schwingkreis_rlcp_komplex_2_const_i.pdf}}
    \begin{align*}
        \intertext{Spannung}
        %\underline{I} &= \frac{\underline{U}_q}{\underline{Z}}\\
        \underline{U} &= \frac{U_0}{1 + \mathrm{j}\left(\frac{\omega}{\omega_0} - \frac{\omega_0}{\omega}\right)Q}\\
        \intertext{Maximalspannung, Güte}
        \underline{U}_0 &= \frac{\underline{I}_q}{G}
        \qquad Q = \frac{B_k}{G}        
    \end{align*}
    \end{minipage}\hfill%
    \begin{minipage}{0.65\textwidth}\centering
        \resizebox{\textwidth}{!}{\includegraphics{Tikz/pdf/plot_rlcp_spannungskurve_iq_guete.pdf}}
        \par Spannung maximal bei $\omega_0$
        \par Begrenzung durch $G$
    \end{minipage}
}
\end{frame}

\b{
\begin{frame}{Nebenrechnung}
\begin{minipage}{0.3\textwidth}
    \resizebox{1.1\textwidth}{!}{\includegraphics{Tikz/pdf/circ_schwingkreis_rlcp_komplex_2_const_i.pdf}}
    \begin{align*}
        \underline{Y} &= G + \mathrm{j}(B_L + B_C) \\
        B_L &= -B_k \frac{\omega_0}{\omega} = -\frac{1}{\omega L}\\
        B_C &= +B_k \frac{\omega}{\omega_0} = \omega C \\
        B_k &= |B_{L,0}| = |X_{C,0}| = \sqrt{\frac{C}{L}}\\
        Q &= \frac{B_k}{G} \quad \omega_0 = \frac{1}{\sqrt{LC}}
    \end{align*}
\end{minipage}\hfill%
\begin{minipage}{0.65\textwidth}\centering
    \begin{align*}
        \underline{U} &= \frac{\underline{I}_q}{\underline{Y}} \\
        &= \frac{\underline{I}_q}{G + \mathrm{j}\left(\omega C - \frac{1}{\omega L}\right)} \\
        &= \frac{\underline{I}_q}{G + \mathrm{j}\left(\frac{\omega}{\omega_0} - \frac{\omega_0}{\omega}\right)B_k} \\
        &= \frac{\underline{I}_q}{G} \cdot \frac{1}{1 + \mathrm{j}\left(\frac{\omega}{\omega_0} - \frac{\omega_0}{\omega}\right)\frac{B_k}{G}} \\
        &= \underline{U}_0 \cdot \frac{1}{1 + \mathrm{j}\left(\frac{\omega}{\omega_0} - \frac{\omega_0}{\omega}\right)Q}
    \end{align*}
\end{minipage}
\end{frame}
}

%-------------------------------------------------------------------------------------------------%

\subsubsection{Stromkurve des RLC-Parallelschwingkreises}
\begin{frame}\ftx{\subsubsecname}
% RLCp an Uq
\b{
    \begin{minipage}{0.3\textwidth}
    \resizebox{1.1\textwidth}{!}{\includegraphics{Tikz/pdf/circ_schwingkreis_rlcp_komplex_2_const_u.pdf}}
    \begin{align*}
        \intertext{Spannungsquelle}
        \underline{U} &= U_q\\
        \intertext{Strom}
        \underline{I} &= U_q \cdot \underline{Y}\\
        \intertext{Admittanz}
        \underline{Y} &= G + \mathrm{j}B
    \end{align*}
    \end{minipage}\hfill%
    \begin{minipage}{0.65\textwidth}\centering
        \resizebox{\textwidth}{!}{\includegraphics{Tikz/pdf/plot_rlcp_admittanzkurve_guete.pdf}}\newline
        Stromkurve $\sim$ Admittanzkurve
    \end{minipage}
}
\s{
    Die Stromkurve des RLC-Parallelschwingkreises bei Anregung mit einer konstanten Spannungsquelle $\underline{U}_q$
    ergibt sich aus der Admittanz $\underline{Y}$ und der Spannungsquelle $\underline{U}_q$:

    \begin{equation*}
        \underline{I} = \underline{U}_q \cdot \underline{Y}
    \end{equation*}

    Damit ist der Strom $\underline{I}$ direkt proportional zur Admittanz $\underline{Y}$ und damit zur Kreisfrequenz $\omega$.
    Die Stromkurve des RLC-Parallelschwingkreises entspricht also der Admittanzkurve des Schwingkreises
    wie sie in Abbildung \ref{fig:plot:rlcp:resonanzkurve} dargestellt ist.
}
\end{frame}