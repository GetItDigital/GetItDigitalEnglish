\begin{circuitikz}
	\draw (0,0) node[transformer core] (T) {} (T.base) node{T};
	\draw (T.A2) to[short] ++(-1,0) coordinate (T1U);
	\draw (T.B1) to[short] ++(0.2,0) coordinate (T2O);
	\draw (T.B2) to[short] ++(0.2,0) coordinate (T2U);
	\draw (T.inner dot A1) node[circ]{};
	\draw (T.inner dot B1) node[circ]{};
	%Strom transformiert
	\draw (T.A1) to[short, i_<=$\underline{I}^\prime_2$] ++(-1,0) coordinate (T1O);
	%Sekundärseite
	\draw (T2O) to[L=$L_{\sigma 2}$] ++(2,0) coordinate (y1); %Streuinduktivität rechts
	\draw (y1) to[R=$R_2$] ++(2,0) coordinate (y2); %Widerstand rechts
	\draw (y2) to[short, i=$\underline{I}_2$, -o] ++(1,0) coordinate (U2O);%Anschluß rechts oben
	\draw (T2U) to[short, -o] ++(5,0) coordinate (U2U); %Anschluß rechts unten
	%Primärseite
	\draw (T1O) to[L=$L_\mathrm{h}$, *-*, i=$\underline{I}_\mu$](T1U); %Hauptinduktivität
	\draw (T1O) to[L=$L_{\sigma 1}$] ++(-2,0) coordinate (x1); %Streuinduktivität links
	\draw (x1) to[R=$R_1$] ++(-2,0) coordinate (x2); %Widerstand links
	\draw (x2) to[short, i_<=$\underline{I}_1$, -o] ++(-1,0) coordinate (U1O); %Anschluß links oben
	\draw (T1U) to[short, -o] ++(-5,0) coordinate (U1U); %Anschluß links unten
	\draw (U1O) to[open, v=$\underline{U}_1$] (U1U); %Spannung U1 links

	%Sekundärseite
	\draw (U2O) to[open, v=$\underline{U}_2$] (U2U); %Spannung U2 rechts
\end{circuitikz}