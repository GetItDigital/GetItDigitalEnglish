%\newsection{Kapitel1}

%Hier den Inhalt mit \input einfügen und die folgenden Zeilen entfernen




	\section{Die Leistung}

\s{
	Die Leistung gibt an, wie schnell Arbeit verrichtet oder Energie übertragen wird. 
	Sie wird definiert als eine Menge an Energie, die pro Zeiteinheit umgesetzt wird. 
	Die SI-Einheit der Leistung ist das Watt (W), wobei ein Watt einem Joule pro Sekunde entspricht. 
	Anders gesagt ist die zuvor beschriebene Arbeit das Integral der Leistung über die Zeit.
	

		\begin{equation}
			W = \int P \cdot \mathrm{d}t
			\end{equation}
	
	Durch Umstellung der Formel ergibt sich, dass die Leistung $P$ die Änderung der Arbeit über die Zeit ist.
	
		\begin{equation}
			P = \frac{\mathrm{d}W}{\mathrm{d}t} \label{eq:PdWdt}
			\end{equation}
}
\begin{frame}
\ftx{Die Leistung}
\b{
\begin{equation*}
	W = \int P \cdot \mathrm{d}t
\end{equation*}

Durch Umstellung der Formel ergibt sich, dass die Leistung $P$ die Änderung der Arbeit über die Zeit ist.

\begin{equation*}
	P = \frac{\mathrm{d}W}{\mathrm{d}t}
\end{equation*}
}
\end{frame}

	\subsection{Die elektrische Leistung}

\s{

	Im Fall der elektrischen Leistung ergibt sich bei gleichbleibender Arbeit über die Zeit nach Einsetzen der elektrischen Arbeit $W_\mathrm{{el}}$ aus Formel \ref{eq:WUIt} der folgende Ausdruck:

	\begin{equation}
		P = \frac{W_\mathrm{{el}}}{t} = \frac{U \cdot I \cdot \colorcancel{\phantom{t}t\phantom{t}}{red}}
		{\colorcancel{\phantom{t}t\phantom{t}}{red}} = U \cdot I \quad \quad [P] = \text{1 Watt = 1 W}
	\end{equation}
}
\begin{frame}
	\ftx{Die Leistung}
	\b{
	Bei gleichbleibender Arbeit über die Zeit ergibt sich der folgende Ausdruck:
	\begin{equation*}
		P = \frac{W_\mathrm{{el}}}{t} = \frac{U \cdot I \cdot \colorcancel{\phantom{t}t\phantom{t}}{red}}
		{\colorcancel{\phantom{t}t\phantom{t}}{red}} = U \cdot I \quad \quad [P] = \text{1 Watt = 1 W}
	\end{equation*}
	}
\end{frame}

		\subsection{Die verschiedenen Leistungsarten}
\s{
	Neben der elektrischen Leistung gibt es noch weitere Leistungsgrößen, welche sich aus den verschiedenen Energieformen ergeben.
	Ergänzend zu den in Abschnitt \ref{sec:energieformen} erörterten Energieformen, sind in der folgenden Tabelle die entsprechenden Leistungen aufgelistet. 
	Dabei wird jeweils die Potentialgröße mit der Flussgröße multipliziert. Die Tabelle gilt nur für Gleichspannung. \\
}

	%%%%% Leistungs Tabelle einfach
	% \begin{table}[H]

	% 	\centering
	% 	\begin{tabular}{llll} % {|l|l|l|l|} Vier Spalten, alle links ausgerichtet
	% 	\toprule[1.5pt] % Oberste horizontale Linie, etwas dicker (1.5pt)
	% 	\textbf{Arten} &\textbf{Potentialgröße} & \textbf{Flussgröße} & \textbf{Formel} \\ % Überschriften für jede Spalte
	% 	\midrule[1.5pt] % Horizontale Linie unter der Überschrift
	% 	elektrisch & Spannung (U) & Strom ($I$) & $P_{\text{el}} = U \cdot I$ \\ % Daten für die erste Zeile
	% 	\midrule % Horizontale Linie
	% 	translat.& Kraft (F)  & Geschwindigkeit ($v$) &  $P_{\text{tr}} = F \cdot v$ \\ % Daten für die zweite Zeile
	% 	\midrule % Horizontale Linie
	% 	rotatorisch & Moment (M)  & Winkelgeschw. ($\omega$) & $P_{\text{rot}} = M \cdot \omega$ \\% Daten für die dritte Zeile
	% 	\midrule % Horizontale Linie
	% 	thermisch &Temperaturdiff. ($\Delta T$) & Wärmedurchgang ($k\cdot A $) & $P_{\text{th}} = \Delta T \cdot k \cdot A$ \\ % Daten für die vierte Zeile
	% 	\midrule % Horizontale Linie
	% 	fluidisch & Druck (p) & Volumenstrom ($\dot{V}$) & $P_\mathrm{{f_{l}}} = p \cdot \dot{V}$ \\ % Daten für die fünfte Zeile
	% 	\bottomrule % Untere horizontale Linie
	% 	\end{tabular}
	% 	\caption{Arten von Leistung} 
	% 	\label{tab:groessen} % Label für Verweise
	% 	\end{table}
\begin{frame}
	\ftx{Verschiedene Leistungsarten}
	%%%%% Leistungs Tabelle Kästchen
		\begin{table}[H]
			\centering
			\begin{tabular}{|l|l|l|l|} % Vier Spalten, alle links ausgerichtet
				\hline
				\textbf{Arten} & \textbf{Potentialgröße} & \textbf{Flussgröße} & \textbf{Formel} \\ % Überschriften für jede Spalte
				\hline
				elektrisch & Spannung ($U$) & Strom ($I$) & $P_{\text{el}} = U \cdot I$ \\ % Daten für die erste Zeile
				\hline
				translatorisch & Kraft ($F$)  & Geschwindigkeit ($v$) &  $P_{\text{tr}} = F \cdot v$ \\ % Daten für die zweite Zeile
				\hline
				rotatorisch & Moment ($M$)  & Winkelgeschw. ($\omega$) & $P_{\text{rot}} = M \cdot \omega$ \\ % Daten für die dritte Zeile
				\hline
				thermisch & Temperaturdiff. ($\Delta T$) & Wärmedurchgang ($k\cdot A $) & $P_{\text{th}} = \Delta T \cdot k \cdot A$ \\ % Daten für die vierte Zeile
				\hline
				fluidisch & Druck ($p$) & Volumenstrom ($\dot{V}$) & $P_\mathrm{f_{l}} = p \cdot \dot{V}$ \\ % Daten für die fünfte Zeile
				\hline
			\end{tabular}
			\s{\caption{\textbf{Arten von Leistung.} Zeigt die verschiedenen Arten von Leistung in verschiedenen physikalischen Systemen.}}
			\label{tab:groessen} % Label für Verweise
		\end{table}
\end{frame}
\begin{frame}
	\ftx{Merksatz: Leistung}
	\begin{Merksatz}
		
	Elektrische Arbeit misst die übertragene Energie, während elektrische Leistung die Übertragungsgeschwindigkeit angibt.
	\end{Merksatz}
\end{frame}
\s{
	Die verschiedenen Leistungsarten sind relevant für die Effizienz eines Systems. 
	Im Regelfall treten neben der erwünschte auch unerwünschte Leistungsgröße auf, welche nach Möglichkeit vermieden werden sollten.
	Dies wird im folgenden Abschnitt über den Wirkungsgrad behandelt.

}
	


	






