% Aufladevorgang RC Strom und Spannung
% AxisEnv 1: Spannung am Kondensator
% AxisEnv 2: Strom am Kondensator
\begin{tikzpicture}[x=1cm,y=1cm]
	\begin{axis}[
		name=AxisSpannung,
		%%%%%%% START SAME OPTIONS %%%%%%%
		xmin=-1, xmax=5.5,
		ymin=-0.125, ymax=1.35,
		xlabel={$t$},
		ylabel={$u(t)$},
		axis x line=center,
		axis y line=center,
		ylabel style={anchor=east, at={(ticklabel cs:0.95)}},
		ticks=none,
		mark=none,
		grid=none,
		width=7cm,
		height=3.5cm,
		%%%%%%% END OF SAME OPTIONS %%%%%%%
		clip=false,
	]
		\node at (2,1)[black,anchor=south west,yshift=-1pt]{$U_q$};				% u_max label
		\addplot[color=black,dashed,	domain=-1:5.4] {1};          			% u_max Grenze
		\addplot[color=voltage,very thick,	domain=-1:0] {0};           		% t<0
		\addplot[color=voltage,very thick,	domain=0:5.4,samples=61] {1-e^(-x)};% t>0 
		\node at (1,0.5)[voltage,anchor=west]{$u_C$};							% u_C label
		%\addplot[color=black,dashed, domain=0:1]{x}; % Gradient in t=0, schneidet Asymptote bei t=tau
	\end{axis}
	\begin{axis}[
		at={($(AxisSpannung.south west)+(0,-0.25cm)$)}, % Position 2. Plot
		anchor=north west,
		name=AxisStrom,
		%%%%%%% START SAME OPTIONS %%%%%%%
		xmin=-1, xmax=5.5,
		ymin=-0.125, ymax=1.35,
		xlabel={$t$},
		ylabel={$i(t)$},
		axis x line=center,
		axis y line=center,
		ylabel style={anchor=east, at={(ticklabel cs:0.95)}},
		ticks=none,
		mark=none,
		grid=none,
		width=7cm,
		height=3.5cm,
		%%%%%%% END OF SAME OPTIONS %%%%%%%
		clip=false,
	]
		\node at (2,1)[black,anchor=south west,yshift=-1pt]{$U_q/R$};		% i_max label
		\addplot[color=black,dashed,	domain=-1:5.4] {1};          		% i_max Grenze
        \addplot[color=red,very thick,	domain=-1:0] {0};  					% t<0
        \addplot[color=red, very thick] coordinates {(0,1)(0,0)}; 			% Stromsprung
        \addplot[color=red,very thick,	domain=0:5.4,samples=61] {e^(-x)}; 	% Stromverlauf
		\node at (1,0.5)[red,anchor=west]{$i$}; 							% i label
	\end{axis}
\end{tikzpicture}
