\subsection{Berechnungen in einer Transistorschaltungen\label{Aufg17}}
% Alt: Aufgabe 17
\Aufgabe{
    Gegeben ist die folgende Schaltung mit dem Ausgangskennlinienfeld eines Transistors.
    
    \begin{figure}[H]
        \centering
        \begin{subfigure}[b]{0.45\textwidth}
            \centering
            \begin{tikzpicture} 
    \draw (0,0) node[npn] (Q1) {};
    
    \draw (Q1.C)
    to [short] (0,1)
    to[R, l_= $R_\mathrm{C}$] (0,3)
    to [short, -o] (0,3.5) node[right, color=voltage] {$U_\mathrm{V}$};
    
    \draw (0,1)
    to[short, *-o] (2,1);
    
    \draw (Q1.E)
    to [short, -*] (0,-2) node[ground]{};
    
    \draw (-2,-2)
    to[short, o-o] (2,-2);
    
    \draw (2,1)
    to[open,v^, name=ua] (2,-2);
    
    \draw (Q1.B) to[short, -o] (-2,0)
    to[open,v>, name=ue] (-2,-2);
    
    \varrmore{ua}{$U_\mathrm{CE}$};
    \varrmore{ue}{$U_\mathrm{BE}$};
\end{tikzpicture}
        \end{subfigure}
        \begin{subfigure}[b]{0.45\textwidth}
            \centering
            \includesvg[width=0.9\textwidth]{Bilder/Aufgaben/FigTransistorKennlinien}
        \end{subfigure}
        \label{fig:FigTransistorKennlinien}
    \end{figure}
    
    \begin{itemize}
        \item[a)] Wie groß ist der maximale Kollektorstrom des idealen Transistors (bei $R_\mathrm{CE} = 0\,\Omega$), wenn der Kollektorwiderstand $R_\mathrm{C} = 7,5\,\Omega$ beträgt und die Versorgungsspannung $U_\mathrm{V} = 3\,\mathrm{V}$ ist?
        \item[b)] Bestimmen Sie  $U_\mathrm{CE}$, $U_\mathrm{RC}$ und $I_\mathrm{C}$ des realen Transistors bei einem Basisstrom von \( I_\mathrm{B} = \mathrm{2\,mA} \). Tragen Sie dazu die Widerstandsgerade in das Kennlinienfeld ein.
        \item[c)] Für einen neuen Arbeitspunkt fallen nur noch $1,7\,\mathrm{V}$ über dem Kollektorwiderstand $R_\mathrm{C}$ ab, der Basisstrom beträgt weiterhin $2\,\mathrm{mA}$. Bestimmen Sie den neuen Widerstand $R_\mathrm{C}$.
    \end{itemize}
}


\Loesung{
    \begin{itemize}
        \item[a)] Berechnung des maximalen Kollektorstroms bei idealem Transistor (\( R_\mathrm{CE} = 0 \)):
              \begin{equation*}
                  I_\mathrm{C}  = \frac{U_\mathrm{V}}{R_\mathrm{C}} = \frac{3\,\mathrm{V}}{7,5\,\Omega} = 400\,\mathrm{mA}
              \end{equation*}
              
              
        \item[b)] Bestimmung der Spannungen und des Kollektorstroms laut Kennlinienfeld:
              \begin{figure}[H]
                  \centering
                  \includesvg[width=0.7\textwidth]{Bilder/Aufgaben/LsgTransistorKennlinien}
                  \caption{Kennlinienfeld mit eingezeichneter Widerstandsgerade}
              \end{figure}
              \begin{gather*}
                  U_\mathrm{CE} = 1,2\,\mathrm{V} \\
                  U_\mathrm{RC} = 1,8\,\mathrm{V} \\
                  I_\mathrm{C} = \frac{U_\mathrm{RC}}{R_\mathrm{C}} = \frac{1,8\,\mathrm{V}}{7,5\,\Omega} = 240\,\mathrm{mA}
              \end{gather*}
              
        \item[c)] Berechnung des neuen Widerstandswerts bei gleichem $I_\mathrm{C}$ und neuer Spannung $U_\mathrm{RC} = 1,7\,\mathrm{V}$:
              \begin{equation*}
                  R_\mathrm{C} = \frac{U_\mathrm{RC}}{I_\mathrm{C}} = \frac{1,7\,\mathrm{V}}{240\,\mathrm{mA}} = 7,08\,\Omega
              \end{equation*}
    \end{itemize}
    
    Siehe: Abschnitt \ref{sec:ElektrischesVerhaltenBipolartransistor}
}