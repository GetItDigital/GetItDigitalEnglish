% RL-Hochpass 1. Ordnung
\begin{tikzpicture}
    % invisible boundbox
    \draw[draw=none](-0.75,-0.25) rectangle (+4.75,+3); % gleiche Maße wie bei 
    % circ_tiefpass_rc_annotated.tex
    % circ_hochpass_rc_annotated.tex
    % circ_tiefpass_rl_annotated.tex
    % circ_hochpass_rl_annotated.tex

    %\draw(0,3) node[anchor=north] {\text{Hochpass}}; % Beschriftung

    % draw nodes
    \draw 
    (0,2) node[anchor=east] (U1+) {}
    (0,0) node[anchor=east] (U1-) {}
    (4,2) node[anchor=west] (U2+) {}
    (4,0) node[anchor=west] (U2-) {};

    % draw RL-Glied
    \draw (0,2) %U1+
    to [R, l=$R$, o-*] (3,2)					% R
    to [L, l=$L$, *-*] (3,0)					% L
    to [short, *-o] (0,0); %U1-

    % draw U2 conenctions
    \draw 
    (3,2) to [short, *-o] (4,2) %U2+
    (3,0) to [short, *-o] (4,0); %U2-

    % draw U1 and U2
    \draw 
    (U1+) to [open, name=u1, v=$\underline{U}_1$, !v] (U1-)	
    (U2+) to [open, name=u2, v^=$\underline{U}_2$, !v] (U2-);	

    %draw arrows
    \varronly{u1};
    \varronly{u2};

\end{tikzpicture}