\subsection{Praktikumsaufgabe 2\label{Aufg30}}
\Aufgabe{
    \begin{figure}[H]
    \centering
    \begin{tikzpicture}
    \draw (0,0) to[short, -*] (1,0)node[ground]{} to[short, -*] (2.5,0) -- (5,0);
    \draw (5,5) to[C,name=C1, l=$\mathrm{C_1}$](5,3) to[V,name=V2] (5,1) -- (5,0);

    \draw (0,5) to[V, name=V1] (0,0); 
    \draw (0,5) to[R, name=R2, l=$\mathrm{R_2}$, -*] (2.5,5) to[R,name=R3, l=$\mathrm{R_3}$] (5,5);
    
    \draw (2.5,0) to[R,name=R1, l=$\mathrm{R_1}$] (2.5,5);
    
    \varrmore{V1}{$\mathrm{V_1}$};
    \varrmore{V2}{$\mathrm{V_2}$};
\end{tikzpicture}
   
    \label{fig:Praktikumsaufgabe2}
    \end{figure}
    Berechnen Sie den Gleich- und Wechselspannungsanteil an einem Widerstand $\mathrm{R_1}$.
    
    Gegeben:
    \begin{itemize}
        \item Eine gemischte Spannung mit Gleichspannungsanteil $\mathrm{U_{DC}} = 7{,}5\,\text{V}$
        \item Wechselspannungsanteil $\mathrm{U_{AC}} = 2{,}3\,\text{V}$
        \item Ein Kondensator mit $\mathrm{C} = 10\,\mu\text{F}$ wird vorgeschaltet.
    \end{itemize}
    
    \begin{itemize}
        \item[a)] Berechnen Sie die Gesamtspannung ohne den Kondensator.
        \item[b)] Beschreiben Sie den Effekt des Kondensators auf den Wechsel- und Gleichspannungsanteil.
    \end{itemize}
}



\Loesung{

}