\subsection{Analyse des Arbeitsbereichs eines Transistors\label{Aufg16}}
% Alt: Aufgabe 16
\Aufgabe{
    Der Stromverstärkungsfaktor des verwendeten Siliziumtransistors beträgt $ \beta = 150$ und die Versorgungsspannung liegt bei $U_\mathrm{V} = 15\,\mathrm{V}$. Die Schaltung soll so dimensioniert werden, dass der Kollektorstrom $I_\mathrm{C} = 2\,\mathrm{mA}$ beträgt.
    
    \begin{figure}[H]
        \centering
        \begin{tikzpicture}
    \draw (0,0) node[npn, tr circle] (Q1) {};
    
    \draw (Q1.C) to[short, -, i<, name=ic] (0,2)
    to[short, -] ++ (-3, 0) 
    to[R=$R_1$]  (-3, 0) 
    to[R=$R_2$]  (-3, -3) 
    node[ground]{};
    
    \draw (-3,2) to[short, *-o] ++(-1,0)
    to [open, v, name=uv] (-4,-3.5); 
    
    \draw (Q1.B) to[short, -*] (-3, 0);
    
    \draw (Q1.E) to[short, -*] ++(0, -0.5) coordinate(E)
    to[R=$R_3$] (0, -3) node[ground]{};
    
    \draw (E) to[short, -] ++(2, 0)
    to[R=$R_\mathrm{L}$] (2, -3) node[ground]{};
    
    \draw (E)++(2, 0) to[short, *-o] ++(1, 0)
    to[open, v^, name=ua] (3,-3.5)    ;
    
    \varrmore{ua}{$U_\mathrm{A}$};
    \varrmore{uv}{$U_\mathrm{B}$};
    \iarrmore{ic}{$I_\mathrm{C}$};
\end{tikzpicture}
        \label{fig:FigTransistorArbeitsbereich}
    \end{figure}
    
    \begin{itemize}
        \item[a)] Dimensionieren Sie den Widerstand $R_3$ so, dass bei einer Spannung am Ausgang von $U_\mathrm{A}=3\,\mathrm{V}$ der Strom durch $R_3$ genau $75\%$ des Emitterstroms beträgt.
        \item[b)] Bestimmen Sie die Widerstände $R_1$ und $R_2$, unter der Annahme, dass die Kollektor-Basis-Spannung $U_\mathrm{BE} = 0,6\,\mathrm{V}$ beträgt und durch $R_2$ etwa das 9-fache des Basisstroms $I_\mathrm{B}$ fließt.
    \end{itemize}
}


\Loesung{
    \begin{itemize}
        \item[a)] Berechnung des Basisstroms über den Stromverstärkungsfaktor:
              \begin{gather*}
                  I_\mathrm{B} = \frac{I_\mathrm{C}}{\beta} = \frac{\mathrm{2\,mA}}{150} = \mathrm{13,3\,\mu A} \\
                  I_\mathrm{E} = I_\mathrm{C} + I_\mathrm{B} = \mathrm{2\,mA} + \mathrm{13,3\,\mu A} = \mathrm{2,013\,mA} \\
                  I_\mathrm{R_3} = 0,75 \cdot I_\mathrm{E} = 0,75 \cdot \mathrm{2{,}013\,mA} = \mathrm{1,51\,mA}          \\
                  R_3  = \frac{\mathrm{3\,V}}{\mathrm{1,51\,mA}} = \mathrm{1,99\,k\Omega}
              \end{gather*}
        \item[b)] Berechnung des Stroms durch \( R_2 \):
              \begin{gather*}
                  I_{R_2}  = 9 \cdot I_\mathrm{B} = 9 \cdot \mathrm{13,3\,\mu A} = \mathrm{0,12\,mA}                                                              \\
                  R_2 = \frac{U_\mathrm{a} + U_\mathrm{BE}}{I_\mathrm{R_2}} = \frac{\mathrm{3\,V} + \mathrm{0,6\,V}}{\mathrm{0,12\,mA}} = \mathrm{30\,k\Omega}
              \end{gather*}
              
              Berechnung des Stroms durch $R_1$, welcher gleich $I_\mathrm{R_2}$ ist:
              \begin{gather*}
                  R_1  = \frac{U_\mathrm{BE}}{I_\mathrm{R_1}} = \frac{0,6\,\mathrm{V}}{0,12\,\mathrm{mA}} = 5,6\,\mathrm{k\Omega}
              \end{gather*}
    \end{itemize}
    
    Siehe: Abschnitt \ref{sec:ElektrischesVerhaltenBipolartransistor}
}